\documentclass[a4paper]{report}
\input{preamble_ask.tex}
\newcommand{\vect}[2]{(#1_1,\ldots, #1_#2)}
%%%%%%% nesting newcommands $$$$$$$$$$$$$$$$$$$
\newcommand{\function}[1]{\newcommand{\nvec}[2]{#1(##1_1,\ldots, ##1_##2)}}

\newcommand{\linode}[2]{#1_n(x)#2^{(n)}+#1_{n-1}(x)#2^{(n-1)}+\cdots +#1_0(x)#2=g(x)}

\newcommand{\vecoffun}[3]{#1_0(#2),\ldots ,#1_#3(#2)}


% \input{tikz.tex}


\begin{document}

\begin{center}
  \minibox{\textcolor{Col1}{\large\textbf{Σειρές Taylor και Maclaurin}}}
\end{center}

Αν για τη συνάρτηση $ f(x) $, υπάρχουν οι παράγωγοι όλων των τάξεων, σε μια περιοχή 
$ I $ του σημείου $ x=a $, τότε για κάθε $ n \in \mathbb{N} $ και για κάθε $ x \in I $, 
ισχύει: 
\[
  \textcolor{Col1}{\text{\textbf{Τύπος Taylor:}}} \quad f(x) = f(a) + \frac{f'(a)}{1!} (x-a) + \frac{f''(a)}{2!} (x-a)^{2} + 
  \cdots + \frac{f^{(n)}(a)}{n!} (x-a)^{n} + \cdots 
\] 
Η παραπάνω έκφραση για την $ f(x) $ ονομάζεται (δυναμο) σειρά Taylor (ή απλώς ανάπτυγμα 
Taylor) με κέντρο το σημείο $x=a$, και ισοδύναμα γράφεται:
\[
  f(x) = f(a) + \frac{f'(a)}{1!} (x-a) + \frac{f''(a)}{2!} (x-a)^{2} + \cdots +
  \frac{f^{(n)}(a)}{n!} (x-a)^{n} + R_{n}(x)
\]
όπου 
\[
  \textcolor{Col1}{\text{\textbf{Υπόλοιπο Lagrange:}}} \quad  R_{n}(x) = \frac{f^{(n+1)}(\xi)}{(n+1)!} (x-a)^{n+1}, 
  \quad \text{όπου $\xi \in (a,x)$ ή $\xi \in (x,a)$.}
\]

\[
  f(x) \approx f(a) + \frac{f'(a)}{1!} (x-a) + \frac{f''(a)}{2!} (x-a)^{2} + \cdots +
  \frac{f^{(n)}(a)}{n!} (x-a)^{n}. 
\]

%todo να τελειωσω σημειωσεις για σειρες Taylor

\end{document}
