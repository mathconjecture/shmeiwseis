\documentclass[a4paper,12pt]{article}
\usepackage{etex}
%%%%%%%%%%%%%%%%%%%%%%%%%%%%%%%%%%%%%%
% Babel language package
\usepackage[english,greek]{babel}
% Inputenc font encoding
\usepackage[utf8]{inputenc}
%%%%%%%%%%%%%%%%%%%%%%%%%%%%%%%%%%%%%%

%%%%% math packages %%%%%%%%%%%%%%%%%%
\usepackage{amsmath}
\usepackage{amssymb}
\usepackage{amsfonts}
\usepackage{amsthm}
\usepackage{proof}

\usepackage{physics}

%%%%%%% symbols packages %%%%%%%%%%%%%%
\usepackage{dsfont}
\usepackage{stmaryrd}
%%%%%%%%%%%%%%%%%%%%%%%%%%%%%%%%%%%%%%%


%%%%%% graphicx %%%%%%%%%%%%%%%%%%%%%%%
\usepackage{graphicx}
\usepackage{color}
%\usepackage{xypic}
\usepackage[all]{xy}
\usepackage{calc}
%%%%%%%%%%%%%%%%%%%%%%%%%%%%%%%%%%%%%%%

\usepackage{enumerate}

\usepackage{fancyhdr}
%%%%% header and footer rule %%%%%%%%%
\setlength{\headheight}{14pt}
\renewcommand{\headrulewidth}{0pt}
\renewcommand{\footrulewidth}{0pt}
\fancypagestyle{plain}{\fancyhf{}
\fancyhead{}
\lfoot{}
\rfoot{\small \thepage}}
\fancypagestyle{vangelis}{\fancyhf{}
\rhead{\small \leftmark}
\lhead{\small }
\lfoot{}
\rfoot{\small \thepage}}
%%%%%%%%%%%%%%%%%%%%%%%%%%%%%%%%%%%%%%%

\usepackage{hyperref}
\usepackage{url}
%%%%%%% hyperref settings %%%%%%%%%%%%
\hypersetup{pdfpagemode=UseOutlines,hidelinks,
bookmarksopen=true,
pdfdisplaydoctitle=true,
pdfstartview=Fit,
unicode=true,
pdfpagelayout=OneColumn,
}
%%%%%%%%%%%%%%%%%%%%%%%%%%%%%%%%%%%%%%



\usepackage{geometry}
\geometry{left=25.63mm,right=25.63mm,top=36.25mm,bottom=36.25mm,footskip=24.16mm,headsep=24.16mm}

%\usepackage[explicit]{titlesec}
%%%%%% titlesec settings %%%%%%%%%%%%%
%\titleformat{\chapter}[block]{\LARGE\sc\bfseries}{\thechapter.}{1ex}{#1}
%\titlespacing*{\chapter}{0cm}{0cm}{36pt}[0ex]
%\titleformat{\section}[block]{\Large\bfseries}{\thesection.}{1ex}{#1}
%\titlespacing*{\section}{0cm}{34.56pt}{17.28pt}[0ex]
%\titleformat{\subsection}[block]{\large\bfseries{\thesubsection.}{1ex}{#1}
%\titlespacing*{\subsection}{0pt}{28.80pt}{14.40pt}[0ex]
%%%%%%%%%%%%%%%%%%%%%%%%%%%%%%%%%%%%%%

%%%%%%%%% My Theorems %%%%%%%%%%%%%%%%%%
\newtheorem{thm}{Θεώρημα}[section]
\newtheorem{cor}[thm]{Πόρισμα}
\newtheorem{lem}[thm]{λήμμα}
\theoremstyle{definition}
\newtheorem{dfn}{Ορισμός}[section]
\newtheorem{dfns}[dfn]{Ορισμοί}
\theoremstyle{remark}
\newtheorem{remark}{Παρατήρηση}[section]
\newtheorem{remarks}[remark]{Παρατηρήσεις}
%%%%%%%%%%%%%%%%%%%%%%%%%%%%%%%%%%%%%%%




\newcommand{\vect}[2]{(#1_1,\ldots, #1_#2)}
%%%%%%% nesting newcommands $$$$$$$$$$$$$$$$$$$
\newcommand{\function}[1]{\newcommand{\nvec}[2]{#1(##1_1,\ldots, ##1_##2)}}

\newcommand{\linode}[2]{#1_n(x)#2^{(n)}+#1_{n-1}(x)#2^{(n-1)}+\cdots +#1_0(x)#2=g(x)}

\newcommand{\vecoffun}[3]{#1_0(#2),\ldots ,#1_#3(#2)}





\begin{document}

\begin{center}
	\fbox{\large\bfseries Σύνοψη της μελέτης μιας σειράς ως προς τη σύγκλιση}
\end{center}

\vspace{2\baselineskip}

Έστω ότι θέλουμε να μελετήσουμε ως προς τη σύγκλιση τη σειρά $ \sum_{n=1}^{+\infty} a_{n} = a_{1} +
a_{2} + \dots + a_{n} $.

Θεωρούμε την ακολουθία  $ a_{n} $ και αξιοποιώντας το κριτήριο του $n$-οστού όρου
έχουμε:

\begin{enumerate}
	\item Υπολογίζουμε το $ \lim_{n\to +\infty} a_{n} $ και:
		\begin{enumerate}[i)]
			\item Αν $ \lim_{n\to +\infty} a_{n} \neq 0 $ τότε η σειρά $ \sum_{n=1}^{+\infty}a_{n}
				$	αποκλίνει.
			\item Αν $ \lim_{n\to +\infty} a_{n} = 0 $ τότε δεν ξέρουμε, και η σειρά
				$ \sum_{n=1}^{+\infty}a_{n} $ μπορεί να συγκλίνει ή να αποκλίνει.
		\end{enumerate}

	\item Αν η ακολουθία $ a_{n} $ είναι ρητή συνάρτηση  εφαρμόζουμε το κριτήριο του ορίου,
		θεωρώντας κατάλληλη ακολουθία $ b_{n} = 1 / n^{p} $, ώστε $ \lim_{n\to +\infty} a_{n}/b_{n} = l
		$ με $ 0<l<+\infty $.

	\item Αν στην ακολουθία $ a_{n} $ υπάρχει γινόμενο με δυνάμεις της μορφής $ 2^{n} $, $ n^{n}
		$ κτλ. εφαρμόζουμε το κριτήριο της $n$-οστής ρίζας (\textlatin{Cauchy}).

	\item Αν στην ακολουθία $ a_{n} $ υπάρχει παραγοντικό ή άθροισμα δυνάμεων της μορφής $ 2^{n} +
		3^{n}$ κτλ. εφαρμόζουμε το κριτήριο του λόγου (\textlatin{D' Alembert}). Αν κατά την
		εφαρμογή του κριτηρίου προκύψει $ l=1 $ τότε εφαρμόζουμε το γενικευμένο κριτήριο του λόγου $
		a_{n+1}/a_{n} = 1 - \lambda/n $.

	\item Αν από τα παραπάνω δεν προκύψει απάντηση εξετάζουμε αν πληρούνται οι προυποθέσεις του
		κριτηρίου του ολοκληρώματος και το εφαρμόζουμε αν είναι εύκολος ο υπολογισμός του
		ολοκληρώματος.

	\item Αν η σειρά είναι εναλλάσσουσα και δεν συγκλίνει απολύτως τότε εφαρμόζουμε το κριτήριο του
		\textlatin{Leibniz}.


\end{enumerate}

\end{document}
