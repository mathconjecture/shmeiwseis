\input{preamble.tex}
\newcommand{\vect}[2]{(#1_1,\ldots, #1_#2)}
%%%%%%% nesting newcommands $$$$$$$$$$$$$$$$$$$
\newcommand{\function}[1]{\newcommand{\nvec}[2]{#1(##1_1,\ldots, ##1_##2)}}

\newcommand{\linode}[2]{#1_n(x)#2^{(n)}+#1_{n-1}(x)#2^{(n-1)}+\cdots +#1_0(x)#2=g(x)}

\newcommand{\vecoffun}[3]{#1_0(#2),\ldots ,#1_#3(#2)}





\begin{document}

\begin{center}
  \minibox{\large\bfseries \textcolor{Col1}{Σύνοψη της μελέτης μιας σειράς ως προς τη
  σύγκλιση}}
\end{center}

\vspace{2\baselineskip}

Έστω ότι θέλουμε να μελετήσουμε ως προς τη σύγκλιση τη σειρά 
$ \sum_{n=1}^{+\infty} a_{n} = a_{1} + a_{2} + \dots + a_{n} + \dotsb $.

Θεωρούμε την ακολουθία  $ a_{n} $ και έχουμε:
\begin{enumerate}
  \item Αξιοποιώντας το κριτήριο του $n$-οστού όρου, υπολογίζουμε το 
    $ \lim_{n\to +\infty} a_{n} $ και:
    \begin{enumerate}[i)]
      \item Αν $ \lim_{n\to +\infty} a_{n} \neq 0 $ τότε η σειρά 
        $ \sum_{n=1}^{+\infty}a_{n} $	αποκλίνει.
      \item Αν $ \lim_{n\to +\infty} a_{n} = 0 $ τότε δεν ξέρουμε, και η σειρά
        $ \sum_{n=1}^{+\infty}a_{n} $ μπορεί να συγκλίνει ή να αποκλίνει.
    \end{enumerate}

  \item Αν η ακολουθία $ a_{n} $ είναι ρητή συνάρτηση  εφαρμόζουμε το κριτήριο του ορίου,
    θεωρώντας κατάλληλη ακολουθία $ b_{n} = 1 / n^{p} $, ώστε 
    $ \lim_{n\to +\infty} a_{n}/b_{n} = l $ με $ 0<l<+\infty $.

  \item Αν στην ακολουθία $ a_{n} $ υπάρχει γινόμενο με δυνάμεις της μορφής 
    $ 2^{n} $, $ n^{n} $ κτλ. εφαρμόζουμε το κριτήριο της $n$-οστής ρίζας (Cauchy).

  \item Αν στην ακολουθία $ a_{n} $ υπάρχει παραγοντικό ή άθροισμα δυνάμεων της μορφής 
    $ 2^{n} + 3^{n}$ κτλ. εφαρμόζουμε το κριτήριο του λόγου (D' Alembert). Αν κατά την
    εφαρμογή του κριτηρίου προκύψει $ l=1 $ τότε εφαρμόζουμε το γενικευμένο κριτήριο 
    του λόγου $ a_{n+1}/a_{n} = 1 - \lambda/n $.

  \item Αν από τα παραπάνω δεν προκύψει απάντηση εξετάζουμε αν πληρούνται οι 
    προυποθέσεις του κριτηρίου του ολοκληρώματος και το εφαρμόζουμε αν είναι εύκολος 
    ο υπολογισμός του ολοκληρώματος.

  \item Αν η σειρά είναι εναλλάσσουσα και δεν συγκλίνει απολύτως τότε εφαρμόζουμε το 
    κριτήριο του Leibniz.
\end{enumerate}


\end{document}
