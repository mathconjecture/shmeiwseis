\documentclass[a4paper,12pt]{article}
\usepackage{etex}
%%%%%%%%%%%%%%%%%%%%%%%%%%%%%%%%%%%%%%
% Babel language package
\usepackage[english,greek]{babel}
% Inputenc font encoding
\usepackage[utf8]{inputenc}
%%%%%%%%%%%%%%%%%%%%%%%%%%%%%%%%%%%%%%

%%%%% math packages %%%%%%%%%%%%%%%%%%
\usepackage{amsmath}
\usepackage{amssymb}
\usepackage{amsfonts}
\usepackage{amsthm}
\usepackage{proof}

\usepackage{physics}

%%%%%%% symbols packages %%%%%%%%%%%%%%
\usepackage{dsfont}
\usepackage{stmaryrd}
%%%%%%%%%%%%%%%%%%%%%%%%%%%%%%%%%%%%%%%


%%%%%% graphicx %%%%%%%%%%%%%%%%%%%%%%%
\usepackage{graphicx}
\usepackage{color}
%\usepackage{xypic}
\usepackage[all]{xy}
\usepackage{calc}
%%%%%%%%%%%%%%%%%%%%%%%%%%%%%%%%%%%%%%%

\usepackage{enumerate}

\usepackage{fancyhdr}
%%%%% header and footer rule %%%%%%%%%
\setlength{\headheight}{14pt}
\renewcommand{\headrulewidth}{0pt}
\renewcommand{\footrulewidth}{0pt}
\fancypagestyle{plain}{\fancyhf{}
\fancyhead{}
\lfoot{}
\rfoot{\small \thepage}}
\fancypagestyle{vangelis}{\fancyhf{}
\rhead{\small \leftmark}
\lhead{\small }
\lfoot{}
\rfoot{\small \thepage}}
%%%%%%%%%%%%%%%%%%%%%%%%%%%%%%%%%%%%%%%

\usepackage{hyperref}
\usepackage{url}
%%%%%%% hyperref settings %%%%%%%%%%%%
\hypersetup{pdfpagemode=UseOutlines,hidelinks,
bookmarksopen=true,
pdfdisplaydoctitle=true,
pdfstartview=Fit,
unicode=true,
pdfpagelayout=OneColumn,
}
%%%%%%%%%%%%%%%%%%%%%%%%%%%%%%%%%%%%%%



\usepackage{geometry}
\geometry{left=25.63mm,right=25.63mm,top=36.25mm,bottom=36.25mm,footskip=24.16mm,headsep=24.16mm}

%\usepackage[explicit]{titlesec}
%%%%%% titlesec settings %%%%%%%%%%%%%
%\titleformat{\chapter}[block]{\LARGE\sc\bfseries}{\thechapter.}{1ex}{#1}
%\titlespacing*{\chapter}{0cm}{0cm}{36pt}[0ex]
%\titleformat{\section}[block]{\Large\bfseries}{\thesection.}{1ex}{#1}
%\titlespacing*{\section}{0cm}{34.56pt}{17.28pt}[0ex]
%\titleformat{\subsection}[block]{\large\bfseries{\thesubsection.}{1ex}{#1}
%\titlespacing*{\subsection}{0pt}{28.80pt}{14.40pt}[0ex]
%%%%%%%%%%%%%%%%%%%%%%%%%%%%%%%%%%%%%%

%%%%%%%%% My Theorems %%%%%%%%%%%%%%%%%%
\newtheorem{thm}{Θεώρημα}[section]
\newtheorem{cor}[thm]{Πόρισμα}
\newtheorem{lem}[thm]{λήμμα}
\theoremstyle{definition}
\newtheorem{dfn}{Ορισμός}[section]
\newtheorem{dfns}[dfn]{Ορισμοί}
\theoremstyle{remark}
\newtheorem{remark}{Παρατήρηση}[section]
\newtheorem{remarks}[remark]{Παρατηρήσεις}
%%%%%%%%%%%%%%%%%%%%%%%%%%%%%%%%%%%%%%%




\newcommand{\vect}[2]{(#1_1,\ldots, #1_#2)}
%%%%%%% nesting newcommands $$$$$$$$$$$$$$$$$$$
\newcommand{\function}[1]{\newcommand{\nvec}[2]{#1(##1_1,\ldots, ##1_##2)}}

\newcommand{\linode}[2]{#1_n(x)#2^{(n)}+#1_{n-1}(x)#2^{(n-1)}+\cdots +#1_0(x)#2=g(x)}

\newcommand{\vecoffun}[3]{#1_0(#2),\ldots ,#1_#3(#2)}



\let\vec\mathbf



\begin{document}


\chapter{Διανυσματικοί Χώροι}


\begin{dfn}
\item {}
    Έστω $V$, μή κενό σύνολο και $\mathbb{K}$ ένα σώμα αντιμεταθετικό 
    (συνήθως θεωρούμε ότι $ \mathbb{K} = \mathbb{R} $ ή $\mathbb{C}$ ). 
    Το σύνολο $V$ μαζί με τις πράξεις:
    \begin{alignat*}{2}
        + \colon V \times V &\to V & \qquad \text{και} \qquad \cdot \colon \mathbb{K} 
        \times V &\to V \\ ( \vec{u}, \vec{v} ) &\mapsto \vec{u} + \vec{v} 
                 & ( \lambda, \vec{u} ) &\mapsto \lambda \vec{u} 
    \end{alignat*}
    \vspace{\baselineskip}
    που ικανοποιούν τις ιδιότητες 

    \begin{minipage}{0.5\textwidth}
        \begin{enumerate}
            \item $ \vec{u} + \vec{v} = \vec{v} + \vec{u}, \quad \forall \mathbf{u}, 
                \mathbf{v} \in V $ 
                \hfill\tikzmark{a}
            \item $ ( \vec{u} + \vec{v} ) + \vec{w} = \vec{u} + ( \vec{v} + \vec{w}),
                \quad \forall \mathbf{u}, \mathbf{v}, \mathbf{w} \in V $ 
            \item $ \exists \vec{0} \in V, \; \forall \mathbf{u} \in V \quad 
                \vec{u} + \vec{0} = \vec{0} + \vec{u} = \vec{u} $ 
            \item $ \forall \mathbf{u} \in V, \; \exists (- \mathbf{u}) \in V \quad  
                \vec{u} + ( - \vec{u} ) = ( - \vec{u} ) + \vec{u} = \vec{0} $ 
                \hfill\tikzmark{b}
            \item $ ( \lambda + \mu ) \vec{u} = \lambda \vec{u} + \mu \vec{u}, \quad 
                \forall \mathbf{u} \in V \; \text{και} \; \forall \lambda, \mu \in 
                \mathbb{K} $ 
            \item $ ( \lambda \mu ) \vec{u} = \lambda ( \mu) \vec{u}, \quad  
                \forall \mathbf{u} \in V \; \text{και} \; \forall \lambda, \mu \in 
                \mathbb{K} $ 
            \item $ \lambda ( \vec{u} + \vec{v} ) = \lambda \vec{u} + \lambda \vec{v}, 
                \quad \forall \mathbf{u}, \mathbf{v} \in V \; \text{και} \; \forall  
                \lambda \in \mathbb{K} $ 
            \item $ 1 \vec{u} = \vec{u}, \quad \forall \mathbf{u} \in V $ 
        \end{enumerate}
        \mybrace{a}{b}[ομάδα αντιμεταθετική]
    \end{minipage}

    ονομάζεται \textcolor{Col2}{διανυσματικός χώρος} επί του $\mathbb{K}$, 
    (ή απλά $ \mathbb{K} $ - χώρος).
\end{dfn}

\begin{dfn}
Αν $ (V,+,\cdot) $ είναι ένας $ \mathbb{K} $ - χώρος, τότε τα στοιχεία του καλούνται 
\textcolor{Col2}{διανύσματα}.
\end{dfn}


\begin{examples}
\item {}
    \begin{enumerate}
        \item \label{ex:Rn}\textcolor{Col2}{Ο διανυσματικός χώρος $ \mathbb{K}^{n} $ 
                επί του $ \mathbb{K} \quad (n \geq 1) $}

            Αν $ \mathbb{K} $ ένα σώμα, τότε το σύνολο $ \mathbb{K}^{n} = 
            \{ \mathbf{u} = (x_{1},\ldots,x_{n}) \; : \; x_{i} \in \mathbb{K}\} $ 
            μαζί με τις πράξεις 
            \[
                \mathbf{u}+ \mathbf{v} = (x_{1}+ y_{1}, \ldots , x_{n}+y_{n}) 
                \quad \text{και} \quad \lambda \mathbf{u} = 
                ( \lambda x_{1}, \ldots, \lambda x_{n})
            \]
            είναι ένας διανυσματικός χωϱος επί του $ \mathbb{K} $. 

            Το μηδέν του χώρου 
            είναι $ \vec{0} = (0,\ldots,0) $, όπου 0 είναι το μηδέν του σώματος 
            $ \mathbb{K} $ και το αντίθετο του $ \mathbf{u} $ είναι το $ - \mathbf{u} =
            (- x_{1}, \ldots, - x_{n}) $.

        \item\label{ex:funs} \textcolor{Col2}{Ο χώρος των συναρτήσεων $\mathbb{R} ^{A}$ 
            (ή $\mathbf{F}(A, \mathbb{R}$)}.

            Αν $ A \subseteq \mathbb{R} $, τότε το σύνολο $ \mathbb{R}^{A} = 
            \{ f \colon A \to \mathbb{R} \; : \; f \; \text{συνάρτηση} \} $, μαζί με 
            τις πράξεις
            \[
                (f+g)(x) = f(x) +g(x), \; \forall x \in A \quad \text{και} 
                \quad (\lambda f)(x)= \lambda f(x), \; \forall x \in A
            \] 
            είναι ένας διανυσματικός χώρος επί του $ \mathbb{R} $.

            Το μηδέν του χώρου είναι η μηδενική συνάρτηση $ \vec{0} $ και το 
            αντίθετο της $f$ είναι η συνάρτηση $ -f $ με τιμή $ (-f)(x) = 
            - f(x), \; \forall x \in A $.

        \item \textcolor{Col2}{Ο χώρος των συνεχών συναρτήσεων $ \mathbf{C}[a,b] $}

            Έστω $ V = \mathbf{C}{[a,b]} = \{ f \colon [a,b] \to \mathbb{R} \; 
            : \; f \; \text{συνεχής συνάρτηση} \}  $. 

            Τότε ως προς τις πράξεις του 
            παραδείγματος~\ref{ex:funs}, ο $V$ είναι ένας διανυσματικός χωϱος 
            επί του $ \mathbb{R} $, 
            γιατί το άθροισμα $ f+g $ δύο συνεχών συναρτήσεων $ f,g $ είναι 
            συνεχής συνάρτηση καθώς και το γινόμενο 
            $ \lambda f, \; \lambda \in \mathbb{R} $, μιας συνεχής συνάρτησης 
            $f$ είναι επίσης συνεχής συνάρτηση. 

        \item \textcolor{Col2}{Ο χώρος των πολυωνύμων $ P_{n}[x] $}.

            Έστω $ V = P_{n}[x] $ το σύνολο των πολυωνύμων βαθμού 
            $ \leq n, \; n \in \mathbb{N}  $ με συντελεστές πραγματικούς. Τότε αν 
            \begin{align*}
                \mathbf{a} \in P_{n}[x] \Rightarrow \mathbf{a}  
                     &= a_{n}x^{n}+a_{n-1}x^{n-1}+\cdots + a_{0} \\
                     \mathbf{b} \in P_{n}[x] \Rightarrow \mathbf{b}  
                     &= b_{n}x^{n}+b_{n-1}x^{n-1}+\cdots + b_{0} 
            \end{align*} 
            έχουμε, ότι το σύνολο $ V $ μαζί με τις πράξεις 
            \begin{align*}
                \mathbf{a}+ \mathbf{b}&= (a_{n}+ b_{n})x^{n} + 
                (a_{n-1}+b_{n-1})x^{n-1}+ \cdots + (a_{0}+ b_{0}) \\
                \lambda \mathbf{a} &= (\lambda a_{n})x^{n}+
                ( \lambda a_{n-1})x^{n-1}+ \cdots + ( \lambda a_{0})
            \end{align*} 
            είναι ένας διανυσματικός χώρος επί του $ \mathbb{R} $.

        \item \textcolor{Col2}{Ο χώρος των λύσεων μιας γραμμικής, ομογενούς 
            διαφορικής εξίσωσης}

            Έστω μια γραμμική, ομογενής διαφορική εξίσωση
            \begin{equation}\label{eq:diff}
                f''(x)+5f'(x)+6f(x)=0 
            \end{equation} 
            Τότε, ως προς τις πράξεις του παραδείγματος~\ref{ex:funs} το 
            σύνολο $ V $ των λύσεων είναι ένας διανυσματικός χώρος επί του
            $ \mathbb{R} $. Πράγματι, αν $ f,g $ λύσεις της~\eqref{eq:diff},
            δηλαδή αν $ f,g \in V $, έχουμε:
            \begin{align*}
                (f+g)''(x)+5(f+g)'(x)+6(f+g)(x) &= [f''(x)+5f'(x)+6f(x)] 
                + [g''(x)+5g'(x)+6g(x)] = 0+0=0
                \intertext{\centering και}
                (\lambda f)''(x) + 5(\lambda f)'(x)+6(\lambda f)(x)&= \lambda
                [f''(x)+5f'(x)+6f(x)] = \lambda \cdot 0=0
            \end{align*}
            Συνεπώς $ f+g \in V $ και $ \lambda f \in V $. 

        \item \textcolor{Col2}{Ο χώρος των λύσεων ενός γραμμικού, ομογενούς 
            συστήματος}

            Έστω το σύστημα των ομογενών εξισώσεων

            \begin{equation}
                % no need to use "align*" env.
                \setlength\arraycolsep{1.5pt} % default value: 5pt
                \left.
                    \begin{array}{ccc ccc c @{\extracolsep{2.5pt}}c
                        @{\extracolsep{2.5pt}}c}
                        a_{11}x_{1} & + & a_{12}x_{2} & + & \cdots & + & a_{1n}x_{n} & =
                                    & 0 \\
                        a_{21}x_{1} & + & a_{22}x_{2} & + & \cdots & + & a_{2n}x_{n} & =
                                    & 0 \\
                        \vdots & & \vdots & & \ddots & &  \vdots & &  \vdots \\
                        a_{m1}x_{1} & + & a_{m2}x_{2} & + & \cdots & + & a_{mn}x_{n} & =
                                    & 0 \\
                    \end{array}
                \right\}
            \end{equation}
            Λύση του είναι ένα στοιχείο $ \mathbf{u} = (x_{1}, x_{2}, \ldots x_{n}) 
            \in \mathbb{R}^{n} $.  Τότε, ως προς τις πράξεις τους χώρουν 
            $ \mathbb{R}^{n} $, του παραδείγματος~\ref{ex:Rn}, το σύνολο των 
            λύσεων είναι ένας διανυσματικός χώρος (που περιέχεται στο χώρο 
            $ \mathbb{R}^{n} $).
    \end{enumerate}
\end{examples}

\end{document}
