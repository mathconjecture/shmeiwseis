\input{preamble/preamble.tex}
\newcommand{\vect}[2]{(#1_1,\ldots, #1_#2)}
%%%%%%% nesting newcommands $$$$$$$$$$$$$$$$$$$
\newcommand{\function}[1]{\newcommand{\nvec}[2]{#1(##1_1,\ldots, ##1_##2)}}

\newcommand{\linode}[2]{#1_n(x)#2^{(n)}+#1_{n-1}(x)#2^{(n-1)}+\cdots +#1_0(x)#2=g(x)}

\newcommand{\vecoffun}[3]{#1_0(#2),\ldots ,#1_#3(#2)}



\let\vec\mathbf

\pagestyle{vangelis}

\begin{document}


\chapter{Διανυσματικοί Χώροι}

\section{Ορισμός}

\begin{dfn}
\item {}
    Έστω $V$, μη κενό σύνολο και $\mathbb{K}$ ένα σώμα αντιμεταθετικό 
    (συνήθως θεωρούμε ότι $ \mathbb{K} = \mathbb{R} $ ή $\mathbb{C}$ ). 
    Το σύνολο $V$ μαζί με τις πράξεις:
    \begin{alignat*}{2}
        + \colon V \times V &\to V & \qquad \text{και} \qquad \cdot \colon \mathbb{K} 
        \times V &\to V \\ ( \vec{u}, \vec{v} ) &\mapsto \vec{u} + \vec{v} 
                 & ( \lambda, \vec{u} ) &\mapsto \lambda \vec{u} 
    \end{alignat*}
    \vspace{\baselineskip}
    που ικανοποιούν τα παρακάτω αξιώματα:

    \twocolumnside{%
        \textbf{Το $V$ είναι ομάδα αντιμεταθετική}
        \begin{enumerate}
            \item $ \vec{u} + \vec{v} = \vec{v} + \vec{u}, \quad \forall \mathbf{u}, 
                \mathbf{v} \in V $ 
            \item $ ( \vec{u} + \vec{v} ) + \vec{w} = \vec{u} + ( \vec{v} + \vec{w}),
                \quad \forall \mathbf{u}, \mathbf{v}, \mathbf{w} \in V $ 
            \item $ \exists \vec{0} \in V, \; \forall \mathbf{u} \in V \quad 
                \vec{u} + \vec{0} = \vec{0} + \vec{u} = \vec{u} $ 
            \item $ \forall \mathbf{u} \in V, \; \exists (- \mathbf{u}) \in V \quad  
                \vec{u} + ( - \vec{u} ) = ( - \vec{u} ) + \vec{u} = \vec{0} $ 
        \end{enumerate}
        }{%
        \textbf{και επίσης ισχύει}
        \begin{enumerate}
            \setcounter{enumi}{4}
        \item $ ( \lambda + \mu ) \vec{u} = \lambda \vec{u} + \mu \vec{u}, \quad 
            \forall \mathbf{u} \in V \; \text{και} \; \forall \lambda, \mu \in 
            \mathbb{K} $ 
        \item $ ( \lambda \mu ) \vec{u} = \lambda ( \mu \vec{u}), \quad  
            \forall \mathbf{u} \in V \; \text{και} \; \forall \lambda, \mu \in 
            \mathbb{K} $ 
        \item $ \lambda ( \vec{u} + \vec{v} ) = \lambda \vec{u} + \lambda \vec{v}, 
            \quad \forall \mathbf{u}, \mathbf{v} \in V \; \text{και} \; \forall  
            \lambda \in \mathbb{K} $ 
        \item $ 1 \vec{u} = \vec{u}, \quad \forall \mathbf{u} \in V $ 
    \end{enumerate}
}

\vspace{\baselineskip}

ονομάζεται \textcolor{Col2}{διανυσματικός χώρος} επί του $\mathbb{K}$, 
(ή απλά $ \mathbb{K} $ - χώρος). Τα στοιχεία του συνόλου $V$ καλούνται 
\textcolor{Col2}{διανύσματα}.
\end{dfn}

\section{Παραδείγματα}

\begin{examples}
\item {}
    \begin{enumerate}
        \item \label{ex:Rn}\textcolor{Col2}{Ο διανυσματικός χώρος $ \mathbb{K}^{n} $ 
            επί του $ \mathbb{K} \quad (n \geq 1) $}

            Αν $ \mathbb{K} $ ένα σώμα, τότε το σύνολο $ \mathbb{K}^{n} = 
            \{ \mathbf{u} = (x_{1},\ldots,x_{n}) \; : \; x_{i} \in \mathbb{K}\} $ 
            μαζί με τις πράξεις 
            \[
                \mathbf{u}+ \mathbf{v} = (x_{1}+ y_{1}, \ldots , x_{n}+y_{n}) 
                \quad \text{και} \quad \lambda \mathbf{u} = 
                ( \lambda x_{1}, \ldots, \lambda x_{n})
            \]
            είναι ένας διανυσματικός χώϱος επί του $ \mathbb{K} $. 

            Το μηδέν του χώρου 
            είναι $ \vec{0} = (0,\ldots,0) $, όπου 0 είναι το μηδέν του σώματος 
            $ \mathbb{K} $ και το αντίθετο του $ \mathbf{u} $ είναι το $ - \mathbf{u} =
            (- x_{1}, \ldots, - x_{n}) $.

        \item\label{ex:funs} \textcolor{Col2}{Ο χώρος των συναρτήσεων $\mathbb{R} ^{A}$ 
            (ή $\mathbf{F}(A, \mathbb{R}$)}.

            Αν $ A \subseteq \mathbb{R} $, τότε το σύνολο $ \mathbb{R}^{A} = 
            \{ f \colon A \to \mathbb{R} \; : \; f \; \text{συνάρτηση} \} $, μαζί με 
            τις πράξεις
            \[
                (f+g)(x) = f(x) +g(x), \; \forall x \in A \quad \text{και} 
                \quad (\lambda f)(x)= \lambda f(x), \; \forall x \in A
            \] 
            είναι ένας διανυσματικός χώρος επί του $ \mathbb{R} $.

            Το μηδέν του χώρου είναι η μηδενική συνάρτηση $ \vec{0} $ και το 
            αντίθετο της $f$ είναι η συνάρτηση $ -f $ με τιμή $ (-f)(x) = 
            - f(x), \; \forall x \in A $.

        \item \textcolor{Col2}{Ο χώρος των συνεχών συναρτήσεων $ \mathbf{C}[a,b] $}

            Έστω $ V = \mathbf{C}{[a,b]} = \{ f \colon [a,b] \to \mathbb{R} \; 
            : \; f \; \text{συνεχής συνάρτηση} \}  $. 

            Τότε ως προς τις πράξεις του 
            παραδείγματος~\ref{ex:funs}, ο $V$ είναι ένας διανυσματικός χώϱος 
            επί του $ \mathbb{R} $, 
            γιατί το άθροισμα $ f+g $ δύο συνεχών συναρτήσεων $ f,g $ είναι 
            συνεχής συνάρτηση καθώς και το γινόμενο 
            $ \lambda f, \; \lambda \in \mathbb{R} $, μιας συνεχής συνάρτησης 
            $f$ είναι επίσης συνεχής συνάρτηση. 

        \item \textcolor{Col2}{Ο χώρος των πολυωνύμων $ P_{n}[x] $}.

            Έστω $ V = P_{n}[x] $ το σύνολο των πολυωνύμων βαθμού 
            $ \leq n, \; n \in \mathbb{N}  $ με συντελεστές πραγματικούς. Τότε αν 
            \begin{gather*}
                \mathbf{a} \in P_{n}[x] \Rightarrow \mathbf{a}  
                = a_{n}x^{n}+a_{n-1}x^{n-1}+\cdots + a_{0} \quad \text{και} 
                \quad \mathbf{b} \in P_{n}[x] \Rightarrow \mathbf{b}  
                = b_{n}x^{n}+b_{n-1}x^{n-1}+\cdots + b_{0} 
            \end{gather*} 
            έχουμε, ότι το σύνολο $ V $ μαζί με τις πράξεις 
            \begin{gather*}
                \mathbf{a}+ \mathbf{b}= (a_{n}+ b_{n})x^{n} + 
                (a_{n-1}+b_{n-1})x^{n-1}+ \cdots + (a_{0}+ b_{0}) \\
                \lambda \mathbf{a} = (\lambda a_{n})x^{n}+
                ( \lambda a_{n-1})x^{n-1}+ \cdots + ( \lambda a_{0})
            \end{gather*} 
            είναι ένας διανυσματικός χώρος επί του $ \mathbb{R} $.

        \item \textcolor{Col2}{Ο χώρος των λύσεων μιας γραμμικής, ομογενούς 
            διαφορικής εξίσωσης}

            Έστω μια γραμμική, ομογενής διαφορική εξίσωση
            \begin{equation}\label{eq:diff}
                f''(x)+5f'(x)+6f(x)=0 
            \end{equation} 
            Τότε, ως προς τις πράξεις του παραδείγματος~\ref{ex:funs} το 
            σύνολο $ V $ των λύσεων είναι ένας διανυσματικός χώρος επί του
            $ \mathbb{R} $. Πράγματι, αν $ f,g $ λύσεις της~\eqref{eq:diff},
            δηλαδή αν $ f,g \in V $, έχουμε:
            \begin{gather*}
                (f+g)''(x)+5(f+g)'(x)+6(f+g)(x) = [f''(x)+5f'(x)+6f(x)] 
                + [g''(x)+5g'(x)+6g(x)] = 0+0=0 \\
                (\lambda f)''(x) + 5(\lambda f)'(x)+6(\lambda f)(x)= \lambda
                [f''(x)+5f'(x)+6f(x)] = \lambda \cdot 0=0
            \end{gather*}
            Συνεπώς $ f+g \in V $ και $ \lambda f \in V $. 

        \item \label{ex:mat} \textcolor{Col2}{Το σύνολο $ M_{m \times n}(\mathbb{K}) $ 
            των $ m \times n $ πινάκων με στοιχεία από το σώμα $ \mathbb{K} $}

            Το σύνολο $ M_{m \times n}(\mathbb{K}) $ των $ m \times n $ πινάκων 
            με στοιχεία από το σώμα $ \mathbb{K} $ είναι διανυσματικός χώρος 
            με πράξεις τη γνωστή πρόσθεση πινάκων και τον πολλαπλασιασμό 
            αριθμού επί πίνακα.

            Το μηδέν του χώρου είναι ο μηδενικός πίνακας, $ \mathbf{0} $ όπου 
            όλα του τα στοιχεία είναι μηδέν και αντίθετο του πίνακα $A$ είναι 
            ο πίνακας $ -A $.

        \item \textcolor{Col2}{Ο χώρος των λύσεων ενός γραμμικού, ομογενούς 
            συστήματος}

            Έστω το ομογενές σύστημα $ A\cdot X = \mathbf{0} $, όπου 
            $ A \in M_{m \times n}( \mathbb{R}) $ πίνακας

            \begin{equation*}
                % no need to use "align*" env.
                \setlength\arraycolsep{1.5pt} % default value: 5pt
                \left.
                    \begin{array}{ccc ccc c @{\extracolsep{2.5pt}}c
                        @{\extracolsep{2.5pt}}c}
                        a_{11}x_{1} & + & a_{12}x_{2} & + & \cdots & + & a_{1n}x_{n} & =
                                    & 0 \\
                        a_{21}x_{1} & + & a_{22}x_{2} & + & \cdots & + & a_{2n}x_{n} & =
                                    & 0 \\
                        \vdots & & \vdots & & \ddots & &  \vdots & &  \vdots \\
                        a_{m1}x_{1} & + & a_{m2}x_{2} & + & \cdots & + & a_{mn}x_{n} & =
                                    & 0 \\
                    \end{array}
                \right\} \Leftrightarrow 
                \underbrace{\begin{pmatrix}
                        a_{11} & a_{12} & \cdots & a_{1n} \\
                        a_{21} & a_{22} & \cdots & a_{2n} \\
                        \vdots & \vdots & \ddots & \vdots \\
                        a_{m1} & a_{m2} & \cdots & a_{mn} 
                \end{pmatrix}}_{A_{m \times n}}
                \cdot 
                \underbrace{\begin{pmatrix*}
                        x_{1} \\
                        x_{2} \\
                        \vdots \\
                        x_{n}
                \end{pmatrix*}}_{X} = 
                \begin{pmatrix*}[r]
                    0 \\
                    0 \\
                    \vdots \\
                    0
                \end{pmatrix*}
            \end{equation*} 
            Λύση του είναι ένα στοιχείο $ \mathbf{u} = 
            \begin{pmatrix*}[r]
                x_{1} & x_{2} & \cdots & x_{n} 
            \end{pmatrix*}^{T} \in M_{n \times 1}(\mathbb{R}) $.  Τότε, ως 
            προς τις πράξεις τους χώρου 
            $ M_{n \times 1}(\mathbb{R}) $, του παραδείγματος~\ref{ex:mat}, 
            το σύνολο των λύσεων είναι ένας διανυσματικός χώρος 
            (που περιέχεται στο χώρο $ M_{n \times 1}(\mathbb{R}) $). Πράγματι, 
            αν $ X =\begin{pmatrix*}[r]
                x_{1} & x_{2} & \cdots & x_{n} 
                \end{pmatrix*}^{T} $ και $ Y = \begin{pmatrix*}[r]
                y_{1} & y_{2} & \cdots & y_{n} 
            \end{pmatrix*}^{T} $ είναι λύσεις του συστήματος, τότε
            \begin{gather*}
                A \cdot (X+Y) = A \cdot X + A \cdot Y = \mathbf{0}+ \mathbf{0} = 
                \mathbf{0} \\
                A \cdot (\lambda X) = \lambda (A \cdot X) = \lambda \mathbf{0} = 
                \mathbf{0}
            \end{gather*} 
    \end{enumerate}
\end{examples}

\begin{rem}
\item {}
    Αν $ \mathbf{u}, \mathbf{v} \in V $ τότε θα γράφουμε ότι 
    $ \mathbf{u} + (- \mathbf{v}) = \mathbf{u} - \mathbf{v} $, 
    όπου $ - \mathbf{v} $ είναι το αντίθετο του $ \mathbf{v} $.
\end{rem}


\begin{prop}
\item {}
    \begin{enumerate}[i)]
        \item Το ουδέτερο στοιχείο ενός διανυσματικού χώρου είναι μοναδικό.
            \begin{proof}
            \item {}
            Έστω ότι υπάρχουν δύο ουδέτερα $ \mathbf{0} $ και $ \mathbf{0}' $. 
             Τότε από το αξίωμα 3 έχουμε:
                 \begin{align*}
                     \mathbf{0}+ \mathbf{0}' = \mathbf{0} \quad 
                     \text{αφού $ \mathbf{0}'$ 
                     είναι ουδέτερο} \hfill\tikzmark{a} \\
                     \mathbf{0}'+ \mathbf{0} = \mathbf{0}' \quad 
                     \text{αφού $ \mathbf{0} $ είναι ουδέτερο}\hfill\tikzmark{b} 
                 \end{align*} 
             \mybrace{a}{b}[$ \mathbf{0} = \mathbf{0}' $]
        \end{proof}
        \item Το αντίθετο στοιχείο ενός διανυσματικού χώρου είναι μοναδικό.
            \begin{proof}
            \item {}
                Έστω ότι υπάρχουν δύο αντίθετα $ \mathbf{u}' $ και $ \mathbf{u}''$. 
                Τότε από το αξίωμα 4 έχουμε:
                \begin{align*}
                    \mathbf{u} + \mathbf{u}' = \mathbf{0} \quad 
                    \text{αφού $ \mathbf{u}'$ είναι αντίθετο} \\
                    \mathbf{u} +  \mathbf{u}''= \mathbf{0} \quad 
                    \text{αφού $ \mathbf{u}'' $ είναι αντίθετο}
                \end{align*} 
                Άρα
                \[
                    \mathbf{u}' = \mathbf{u}' + \mathbf{0}= \mathbf{u}' + (\mathbf{u} 
                    + \mathbf{u}'') = (\mathbf{u}' + \mathbf{u}) + \mathbf{u}'' = 
                    \mathbf{0} + \mathbf{u}'' = \mathbf{u}'' 
                \] 
            \end{proof}
    \end{enumerate}
\end{prop}

\begin{thm}
\item {}
    \begin{enumerate}[i)]
        \item $ \mathbf{u} + \mathbf{v} = \mathbf{w} + \mathbf{v} 
            \Rightarrow \mathbf{u} = \mathbf{w} $ \quad (νόμος διαγραφής)
        \item $ 0 \cdot \mathbf{u} = \mathbf{0}, \quad \forall \mathbf{u} \in V $
        \item $ \lambda \cdot \mathbf{0} = \mathbf{0}, \quad \forall \lambda \in 
            \mathbb{K} $
        \item $ (-1)\cdot \mathbf{u} = - \mathbf{u}, \quad \forall \mathbf{u} \in V $ 
        \item $ \lambda \cdot (\mathbf{u} - \mathbf{v}) = 
            \lambda \mathbf{u} - \lambda \mathbf{v}, \quad \forall \mathbf{u}, 
            \mathbf{v} \in V $ και $ \lambda \in \mathbb{K} $
        \item $ (\lambda - \mu ) \cdot \mathbf{u} = \lambda \mathbf{u} - 
            \mu \mathbf{u}, \quad \forall \mathbf{u} \in V $ και $ \lambda, 
            \mu \in \mathbb{K} $
        \item $ -(- \lambda ) \cdot \mathbf{u} = \lambda \mathbf{u}, 
            \quad \forall \mathbf{u} \in V $ και $ \lambda \in \mathbb{K} $
    \end{enumerate}
\end{thm}

\begin{proof}
\item {}
    \begin{enumerate}[i)]
        \item $ \mathbf{u} = \mathbf{u} + \mathbf{0} = \mathbf{u} + 
            (\mathbf{v} + (- \mathbf{v})) = (\mathbf{u} + \mathbf{v}) + (- \mathbf{v}) =
            (\mathbf{w} + \mathbf{v}) + (- \mathbf{v}) = \mathbf{w} + 
            (\mathbf{v} + (- \mathbf{v})) = \mathbf{w} + \mathbf{0} = \mathbf{w}$
        \item Έχουμε $ 0\cdot \mathbf{u} = (0 + 0) 
            \cdot \mathbf{u}= 0\cdot \mathbf{u}+ 0\cdot \mathbf{u} $, άρα
            \begin{align*}
                0\cdot \mathbf{u} = 0\cdot \mathbf{u} + 0\cdot
                \mathbf{u}\tikzmark{a} \\
                0\cdot \mathbf{u} = \mathbf{0} + 0\cdot \mathbf{u} 
                \tikzmark{b}
            \end{align*} 
            \mybrace{a}{b}[$ 0\cdot \mathbf{u} + 0\cdot \mathbf{u} = 
            \mathbf{0}+ 0\cdot \mathbf{u} \overset{\text{i)}}{\Rightarrow}  
            0\cdot \mathbf{u} = \mathbf{0}$]
        \item Έχουμε $ \lambda \cdot \mathbf{0} = \lambda \cdot 
            (\mathbf{0} + \mathbf{0}) = \lambda \cdot \mathbf{0} + \lambda 
            \cdot \mathbf{0}$, άρα 
            \begin{align*}
                \lambda \cdot \mathbf{0} = \lambda \cdot \mathbf{0} + \lambda \cdot 
                \mathbf{0} \tikzmark{a} \\
                \lambda \cdot \mathbf{0} = \mathbf{0} + \lambda \cdot \mathbf{0}
                \tikzmark{b}
            \end{align*}
            \mybrace{a}{b}[$ \lambda \cdot \mathbf{0}+ \lambda \cdot \mathbf{0} = 
            \mathbf{0} + \lambda \cdot \mathbf{0} \Rightarrow \lambda 
            \cdot \mathbf{0} = \mathbf{0}$]
        \item Έχουμε $ \mathbf{u} + (-1)\cdot \mathbf{u} = 1\cdot 
            \mathbf{u} + (-1)\cdot \mathbf{u} = [1+(-1)] \cdot \mathbf{u}= 0 
            \cdot \mathbf{u} = \mathbf{0} $. 

            Όμως το αντίθετο του $ \mathbf{u} $ είναι μοναδικό. Επομένως 
            $ (-1)\cdot \mathbf{u} = - \mathbf{u}$.

        \item Έχουμε $ \lambda\cdot (\mathbf{u} - \mathbf{v}) + \lambda \cdot 
            \mathbf{v} = \lambda \cdot [(\mathbf{u} - \mathbf{v}) + \mathbf{v}] = 
            \lambda \cdot [\mathbf{u} + (- \mathbf{v}+ \mathbf{v})] = \lambda \cdot 
            ( \mathbf{u} + \mathbf{0}) = \lambda \cdot \mathbf{u} $, αρα 
            \begin{gather*}
                \lambda \cdot (\mathbf{u} - \mathbf{v}) + \lambda \cdot \mathbf{v} = 
                \lambda \cdot \mathbf{u} \Leftrightarrow \\ 
                [\lambda \cdot (\mathbf{u} - \mathbf{v}) + \lambda \cdot \mathbf{v}] -
                (\lambda \cdot \mathbf{v})= 
                \lambda \cdot \mathbf{u} - \lambda \cdot \mathbf{v} \Leftrightarrow \\
                \lambda \cdot (\mathbf{u} - \mathbf{v}) + 
                [\lambda \cdot \mathbf{v} - \lambda \cdot \mathbf{v}] =
                \lambda \cdot \mathbf{u} - \lambda \cdot \mathbf{v} \Leftrightarrow \\
                \lambda \cdot (\mathbf{u}- \mathbf{v}) + \mathbf{0} = \lambda \cdot 
                \mathbf{u} - \lambda \cdot \mathbf{v} \Leftrightarrow \\
                \lambda \cdot (\mathbf{u} - \mathbf{v}) = \lambda \cdot \mathbf{u} - 
                \lambda \cdot \mathbf{v}
            \end{gather*}

        \item Έχουμε $ (\lambda - \mu ) \cdot \mathbf{u} + \mu \cdot \mathbf{u} = 
            (\lambda - \mu + \mu )\cdot \mathbf{u} = \lambda \cdot \mathbf{u}$, άρα 
            \begin{gather*}
                [(\lambda - \mu )\cdot \mathbf{u} + \mu \cdot \mathbf{u}] - 
                \mu \cdot \mathbf{u} = \lambda \cdot \mathbf{u}- \mu \cdot 
                \mathbf{u} \Leftrightarrow \\
                (\lambda - \mu ) \cdot \mathbf{u} + (\mu \cdot \mathbf{u} - 
                \mu \cdot \mathbf{u}) = \lambda \cdot \mathbf{u}- \mu \cdot 
                \mathbf{u} \Leftrightarrow \\
                (\lambda - \mu )\cdot \mathbf{u} + \mathbf{0} = 
                \lambda \cdot \mathbf{u}- \mu \cdot \mathbf{u} \Leftrightarrow \\
                (\lambda - \mu )\cdot \mathbf{u} = 
                \lambda \cdot \mathbf{u}- \mu \cdot \mathbf{u} \Leftrightarrow \\
            \end{gather*}

        \item $ -(- \mathbf{u}) = -1\cdot [(-1)\cdot \mathbf{u}] = [(-1)(-1)] \cdot 
            \mathbf{u}= 1 \cdot \mathbf{u} = \mathbf{u} $
    \end{enumerate}
\end{proof}

\begin{prop}
\item {}
    Αν $ \lambda \in \mathbb{K}, \; \mathbf{u} \in V $ και $ \lambda \cdot \mathbf{u} =
    \mathbf{0} $ τότε $ \lambda = 0 $ ή $ \mathbf{u} = \mathbf{0} $. 
\end{prop}

\begin{proof}
\item {}
    Έστω $ \lambda \neq 0 $. Θα δείξουμε ότι αναγκαστικά $ \mathbf{u} = \mathbf{0} $. 

    Πράγματι, αφού $ \lambda \neq 0 \Rightarrow \exists \lambda ^{-1} \in \mathbb{K} $, 
    όπου ισχύει ότι $ \lambda \cdot \lambda ^{-1} = 1 $. Οπότε 
    \begin{gather*}
        \lambda \cdot \mathbf{u} = \mathbf{0} \Leftrightarrow 
        \lambda ^{-1}\cdot (\lambda \cdot \mathbf{u}) = \lambda ^{-1} \cdot \mathbf{0}
        \Leftrightarrow 
        (\lambda \cdot \lambda ^{-1}) \cdot \mathbf{u} = \mathbf{0} \Leftrightarrow 
        1 \cdot \mathbf{u} = \mathbf{0} \Leftrightarrow 
        \mathbf{u} = \mathbf{0} 
     \end{gather*} 
\end{proof}

\begin{exercises}
\item {}
    \begin{enumerate}
        \item Να δείξετε ότι $ - ( \mathbf{u}+ \mathbf{v} ) = - \mathbf{u} - 
            \mathbf{v} $.
        \item Αν $ \lambda \in \mathbb{K}, \; \lambda \neq 0 $ και 
            $ \lambda \cdot \mathbf{u}= \lambda \cdot \mathbf{v}$, τότε να 
            δείξετε ότι $ \mathbf{u} = \mathbf{v} $.
        \item Αν $ \lambda_{1}, \; \lambda _{2} \in \mathbb{K} $ και 
            $ \mathbf{u} \in V, \; \mathbf{u} \neq \mathbf{0} $ με 
            $ \lambda _{1}\cdot \mathbf{u} = \lambda _{2} \cdot \mathbf{u}$, 
            τότε να δείξετε ότι $ \lambda _{1} = \lambda _{2} $.
    \end{enumerate}
\end{exercises}


\end{document}
