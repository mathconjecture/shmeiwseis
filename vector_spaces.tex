\documentclass[a4paper,12pt]{article}
\usepackage{etex}
%%%%%%%%%%%%%%%%%%%%%%%%%%%%%%%%%%%%%%
% Babel language package
\usepackage[english,greek]{babel}
% Inputenc font encoding
\usepackage[utf8]{inputenc}
%%%%%%%%%%%%%%%%%%%%%%%%%%%%%%%%%%%%%%

%%%%% math packages %%%%%%%%%%%%%%%%%%
\usepackage{amsmath}
\usepackage{amssymb}
\usepackage{amsfonts}
\usepackage{amsthm}
\usepackage{proof}

\usepackage{physics}

%%%%%%% symbols packages %%%%%%%%%%%%%%
\usepackage{dsfont}
\usepackage{stmaryrd}
%%%%%%%%%%%%%%%%%%%%%%%%%%%%%%%%%%%%%%%


%%%%%% graphicx %%%%%%%%%%%%%%%%%%%%%%%
\usepackage{graphicx}
\usepackage{color}
%\usepackage{xypic}
\usepackage[all]{xy}
\usepackage{calc}
%%%%%%%%%%%%%%%%%%%%%%%%%%%%%%%%%%%%%%%

\usepackage{enumerate}

\usepackage{fancyhdr}
%%%%% header and footer rule %%%%%%%%%
\setlength{\headheight}{14pt}
\renewcommand{\headrulewidth}{0pt}
\renewcommand{\footrulewidth}{0pt}
\fancypagestyle{plain}{\fancyhf{}
\fancyhead{}
\lfoot{}
\rfoot{\small \thepage}}
\fancypagestyle{vangelis}{\fancyhf{}
\rhead{\small \leftmark}
\lhead{\small }
\lfoot{}
\rfoot{\small \thepage}}
%%%%%%%%%%%%%%%%%%%%%%%%%%%%%%%%%%%%%%%

\usepackage{hyperref}
\usepackage{url}
%%%%%%% hyperref settings %%%%%%%%%%%%
\hypersetup{pdfpagemode=UseOutlines,hidelinks,
bookmarksopen=true,
pdfdisplaydoctitle=true,
pdfstartview=Fit,
unicode=true,
pdfpagelayout=OneColumn,
}
%%%%%%%%%%%%%%%%%%%%%%%%%%%%%%%%%%%%%%



\usepackage{geometry}
\geometry{left=25.63mm,right=25.63mm,top=36.25mm,bottom=36.25mm,footskip=24.16mm,headsep=24.16mm}

%\usepackage[explicit]{titlesec}
%%%%%% titlesec settings %%%%%%%%%%%%%
%\titleformat{\chapter}[block]{\LARGE\sc\bfseries}{\thechapter.}{1ex}{#1}
%\titlespacing*{\chapter}{0cm}{0cm}{36pt}[0ex]
%\titleformat{\section}[block]{\Large\bfseries}{\thesection.}{1ex}{#1}
%\titlespacing*{\section}{0cm}{34.56pt}{17.28pt}[0ex]
%\titleformat{\subsection}[block]{\large\bfseries{\thesubsection.}{1ex}{#1}
%\titlespacing*{\subsection}{0pt}{28.80pt}{14.40pt}[0ex]
%%%%%%%%%%%%%%%%%%%%%%%%%%%%%%%%%%%%%%

%%%%%%%%% My Theorems %%%%%%%%%%%%%%%%%%
\newtheorem{thm}{Θεώρημα}[section]
\newtheorem{cor}[thm]{Πόρισμα}
\newtheorem{lem}[thm]{λήμμα}
\theoremstyle{definition}
\newtheorem{dfn}{Ορισμός}[section]
\newtheorem{dfns}[dfn]{Ορισμοί}
\theoremstyle{remark}
\newtheorem{remark}{Παρατήρηση}[section]
\newtheorem{remarks}[remark]{Παρατηρήσεις}
%%%%%%%%%%%%%%%%%%%%%%%%%%%%%%%%%%%%%%%




\newcommand{\vect}[2]{(#1_1,\ldots, #1_#2)}
%%%%%%% nesting newcommands $$$$$$$$$$$$$$$$$$$
\newcommand{\function}[1]{\newcommand{\nvec}[2]{#1(##1_1,\ldots, ##1_##2)}}

\newcommand{\linode}[2]{#1_n(x)#2^{(n)}+#1_{n-1}(x)#2^{(n-1)}+\cdots +#1_0(x)#2=g(x)}

\newcommand{\vecoffun}[3]{#1_0(#2),\ldots ,#1_#3(#2)}




\let\oldhat\hat
\renewcommand{\vec}[1]{\mathbf{#1}}
\renewcommand{\hat}[1]{\oldhat{\mathbf{#1}}} 

\begin{document}


\chapter{Διανυσματικοί Χώροι}


\begin{dfn}
\item {}
    Έστω $V$, μή κενό σύνολο και $\mathbb{K}$ ένα σώμα αντιμεταθετικό 
    (συνήθως θεωρούμε ότι $ \mathbb{K} = \mathbb{R} $ ή $\mathbb{C}$ ). 
    Το σύνολο $V$ μαζί με τις πράξεις:
    \begin{alignat*}{2}
        + \colon V \times V &\to V & \qquad \text{και} \qquad \cdot \colon \mathbb{K} 
        \times V &\to V \\ ( \vec{u}, \vec{v} ) &\mapsto \vec{u} + \vec{v} 
                 & ( \lambda, \vec{u} ) &\mapsto \lambda \vec{u} 
    \end{alignat*}
    \vspace{\baselineskip}
    που ικανοποιούν τις ιδιότητες 

    \begin{minipage}{0.5\textwidth}
        \begin{enumerate}
            \item $ \vec{u} + \vec{v} = \vec{v} + \vec{u}, \quad \forall \mathbf{u}, 
                \mathbf{v} \in V $ 
                \hfill\tikzmark{a}
            \item $ ( \vec{u} + \vec{v} ) + \vec{w} = \vec{u} + ( \vec{v} + \vec{w}),
                \quad \forall \mathbf{u}, \mathbf{v}, \mathbf{w} \in V $ 
            \item $ \exists \vec{0} \in V, \; \forall \mathbf{u} \in V \quad 
                \vec{u} + \vec{0} = \vec{0} + \vec{u} = \vec{u} $ 
            \item $ \forall \mathbf{u} \in V, \; \exists (- \mathbf{u}) \in V \quad  
                \vec{u} + ( - \vec{u} ) = ( - \vec{u} ) + \vec{u} = \vec{0} $ 
                \hfill\tikzmark{b}
            \item $ ( \lambda + \mu ) \vec{u} = \lambda \vec{u} + \mu \vec{u}, \quad 
                \forall \mathbf{u} \in V \; \text{και} \; \forall \lambda, \mu \in 
                \mathbb{K} $ 
            \item $ ( \lambda \mu ) \vec{u} = \lambda ( \mu) \vec{u}, \quad  
                \forall \mathbf{u} \in V \; \text{και} \; \forall \lambda, \mu \in 
                \mathbb{K} $ 
            \item $ \lambda ( \vec{u} + \vec{v} ) = \lambda \vec{u} + \lambda \vec{v}, 
                \quad \forall \mathbf{u}, \mathbf{v} \in V \; \text{και} \; \forall  
                \lambda \in \mathbb{K} $ 
            \item $ 1 \vec{u} = \vec{u}, \quad \forall \mathbf{u} \in V $ 
        \end{enumerate}
        \mybrace{a}{b}[ομάδα αντιμεταθετική]
    \end{minipage}

    ονομάζεται διανυσματικός χώρος επί του $\mathbb{K}$, 
    (ή απλά $ \mathbb{K} $ - χώρος).
\end{dfn}

\begin{dfn}
Αν $ (V,+,\cdot) $ είναι ένας $ \mathbb{K} $ - χώρος, τότε τα στοιχεία του καλούνται 
διανύσματα.
\end{dfn}


\begin{examples}
\item {}
    \begin{enumerate}
        \item Ο διανυσματικός χώρος $ \mathbb{K}^{n} $ επί του $ \mathbb{K} $.
            Αν $ \mathbb{K} $ ένα σώμα, τότε το σύνολο $ \mathbb{K}^{n} $
    \end{enumerate}
\end{examples}

\end{document}
