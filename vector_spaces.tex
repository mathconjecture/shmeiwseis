\documentclass[a4paper,12pt]{article}
\usepackage{etex}
%%%%%%%%%%%%%%%%%%%%%%%%%%%%%%%%%%%%%%
% Babel language package
\usepackage[english,greek]{babel}
% Inputenc font encoding
\usepackage[utf8]{inputenc}
%%%%%%%%%%%%%%%%%%%%%%%%%%%%%%%%%%%%%%

%%%%% math packages %%%%%%%%%%%%%%%%%%
\usepackage{amsmath}
\usepackage{amssymb}
\usepackage{amsfonts}
\usepackage{amsthm}
\usepackage{proof}

\usepackage{physics}

%%%%%%% symbols packages %%%%%%%%%%%%%%
\usepackage{dsfont}
\usepackage{stmaryrd}
%%%%%%%%%%%%%%%%%%%%%%%%%%%%%%%%%%%%%%%


%%%%%% graphicx %%%%%%%%%%%%%%%%%%%%%%%
\usepackage{graphicx}
\usepackage{color}
%\usepackage{xypic}
\usepackage[all]{xy}
\usepackage{calc}
%%%%%%%%%%%%%%%%%%%%%%%%%%%%%%%%%%%%%%%

\usepackage{enumerate}

\usepackage{fancyhdr}
%%%%% header and footer rule %%%%%%%%%
\setlength{\headheight}{14pt}
\renewcommand{\headrulewidth}{0pt}
\renewcommand{\footrulewidth}{0pt}
\fancypagestyle{plain}{\fancyhf{}
\fancyhead{}
\lfoot{}
\rfoot{\small \thepage}}
\fancypagestyle{vangelis}{\fancyhf{}
\rhead{\small \leftmark}
\lhead{\small }
\lfoot{}
\rfoot{\small \thepage}}
%%%%%%%%%%%%%%%%%%%%%%%%%%%%%%%%%%%%%%%

\usepackage{hyperref}
\usepackage{url}
%%%%%%% hyperref settings %%%%%%%%%%%%
\hypersetup{pdfpagemode=UseOutlines,hidelinks,
bookmarksopen=true,
pdfdisplaydoctitle=true,
pdfstartview=Fit,
unicode=true,
pdfpagelayout=OneColumn,
}
%%%%%%%%%%%%%%%%%%%%%%%%%%%%%%%%%%%%%%



\usepackage{geometry}
\geometry{left=25.63mm,right=25.63mm,top=36.25mm,bottom=36.25mm,footskip=24.16mm,headsep=24.16mm}

%\usepackage[explicit]{titlesec}
%%%%%% titlesec settings %%%%%%%%%%%%%
%\titleformat{\chapter}[block]{\LARGE\sc\bfseries}{\thechapter.}{1ex}{#1}
%\titlespacing*{\chapter}{0cm}{0cm}{36pt}[0ex]
%\titleformat{\section}[block]{\Large\bfseries}{\thesection.}{1ex}{#1}
%\titlespacing*{\section}{0cm}{34.56pt}{17.28pt}[0ex]
%\titleformat{\subsection}[block]{\large\bfseries{\thesubsection.}{1ex}{#1}
%\titlespacing*{\subsection}{0pt}{28.80pt}{14.40pt}[0ex]
%%%%%%%%%%%%%%%%%%%%%%%%%%%%%%%%%%%%%%

%%%%%%%%% My Theorems %%%%%%%%%%%%%%%%%%
\newtheorem{thm}{Θεώρημα}[section]
\newtheorem{cor}[thm]{Πόρισμα}
\newtheorem{lem}[thm]{λήμμα}
\theoremstyle{definition}
\newtheorem{dfn}{Ορισμός}[section]
\newtheorem{dfns}[dfn]{Ορισμοί}
\theoremstyle{remark}
\newtheorem{remark}{Παρατήρηση}[section]
\newtheorem{remarks}[remark]{Παρατηρήσεις}
%%%%%%%%%%%%%%%%%%%%%%%%%%%%%%%%%%%%%%%




\newcommand{\vect}[2]{(#1_1,\ldots, #1_#2)}
%%%%%%% nesting newcommands $$$$$$$$$$$$$$$$$$$
\newcommand{\function}[1]{\newcommand{\nvec}[2]{#1(##1_1,\ldots, ##1_##2)}}

\newcommand{\linode}[2]{#1_n(x)#2^{(n)}+#1_{n-1}(x)#2^{(n-1)}+\cdots +#1_0(x)#2=g(x)}

\newcommand{\vecoffun}[3]{#1_0(#2),\ldots ,#1_#3(#2)}



\usepackage{xspace}
\pagestyle{vangelis}

\begin{document}



\chapter{Διανυσματικοί Χώροι}


\section{Ορισμός}

\begin{dfn}
\item {}
    Έστω $V \neq \emptyset $, σύνολο και $\mathbb{K}$ ένα σώμα αντιμεταθετικό 
    (συνήθως θεωρούμε ότι $ \mathbb{K} = \mathbb{R} $ ή $\mathbb{C}$). 
    Το σύνολο $V$ μαζί με τις πράξεις:
    \begin{alignat*}{2}
        + \colon V \times V &\to V & \qquad \text{και} \qquad \cdot \colon \mathbb{K} 
        \times V &\to V \\ ( \mathbf{u}, \mathbf{v} ) &\mapsto \mathbf{u} + \mathbf{v} 
                 & ( \lambda, \mathbf{u} ) &\mapsto \lambda \mathbf{u} 
    \end{alignat*}
    \vspace{\baselineskip}
    που ικανοποιούν τα παρακάτω αξιώματα:

    \twocolumnside{%
        \textbf{Το $V$ είναι ομάδα αντιμεταθετική}
        \begin{enumerate}
            \item $ \mathbf{u} + \mathbf{v} = \mathbf{v} + \mathbf{u}, \quad \forall \mathbf{u}, 
                \mathbf{v} \in V $ 
            \item $ ( \mathbf{u} + \mathbf{v} ) + \mathbf{w} = \mathbf{u} + ( \mathbf{v} + \mathbf{w}),
                \quad \forall \mathbf{u}, \mathbf{v}, \mathbf{w} \in V $ 
            \item $ \exists \mathbf{0} \in V, \; \forall \mathbf{u} \in V \quad 
                \mathbf{u} + \mathbf{0} = \mathbf{0} + \mathbf{u} = \mathbf{u} $ 
            \item $ \forall \mathbf{u} \in V, \; \exists \mathbf{v} \in V \quad  
                \mathbf{u} + \mathbf{v} = \mathbf{v} + \mathbf{u} = \mathbf{0} $ 
        \end{enumerate}
        }{%
        \textbf{και επίσης ισχύει}
        \begin{enumerate}[resume]
            \setcounter{enumi}{4}
        \item $ ( \lambda + \mu ) \mathbf{u} = \lambda \mathbf{u} + \mu \mathbf{u}, \quad 
            \forall \mathbf{u} \in V \; \text{και} \; \forall \lambda, \mu \in 
            \mathbb{K} $ 
        \item $ ( \lambda \mu ) \mathbf{u} = \lambda ( \mu \mathbf{u}), \quad  
            \forall \mathbf{u} \in V \; \text{και} \; \forall \lambda, \mu \in 
            \mathbb{K} $ 
        \item $ \lambda ( \mathbf{u} + \mathbf{v} ) = \lambda \mathbf{u} + \lambda \mathbf{v}, 
            \quad \forall \mathbf{u}, \mathbf{v} \in V \; \text{και} \; \forall  
            \lambda \in \mathbb{K} $ 
        \item $ 1 \mathbf{u} = \mathbf{u}, \quad \forall \mathbf{u} \in V $ 
    \end{enumerate}
}

\vspace{\baselineskip}

ονομάζεται \textcolor{Col2}{διανυσματικός χώρος} επί του $\mathbb{K}$, 
(ή απλά $ \mathbb{K}-$ χώρος).  Τα 
στοιχεία του συνόλου $V$ καλούνται \textcolor{Col2}{διανύσματα}.
\end{dfn}

\begin{rem}
\item {}
    \begin{enumerate}
        \item Συνήθως για έναν διανυσματικό χώρο $V$ γράφουμε 
            $V = (V(\mathbb{K}), +, \cdot) $.
        \item Το διάνυσμα $ \mathbf{v} $ στο αξίωμα 4, λέγεται 
            \textcolor{Col2}{αντίθετο} του $ \mathbf{u} $ και συμβολίζεται με 
            $ \mathbf{v} = - \mathbf{u} $.
        \item Είναι σημαντικό να προσέξουμε ότι στον ορισμό του διανυσματικού χώρου 
            ορίσαμε τις πράξεις έτσι ώστε να ικανοποιούνται οι συνθήκες κλειστότητας:
            \begin{myitemize}
            \item $ \forall \mathbf{u}, \mathbf{v} \in V \quad \mathbf{u} + 
                \mathbf{v} \in V \quad \text{(κλειστότητα πρόσθεσης)} $ 
            \item $ \forall \mathbf{u} \in V \; \text{και} \; \lambda \in \mathbb{K} 
                \quad \lambda \mathbf{u} \in V \quad 
                \text{(κλειστότητα βαθμωτού πολ/σμού)} $
            \end{myitemize}
    \end{enumerate}
\end{rem}


\section{Παραδείγματα}


\begin{example}\label{ex:Rn}
    \textcolor{Col2}{Ο διανυσματικός χώρος $ \mathbb{K}^{n} $ 
    επί του $ \mathbb{K} \quad (n \geq 1) $}

    Αν $ \mathbb{K} $ ένα σώμα, τότε το σύνολο $ \mathbb{K}^{n} = 
    \{ \mathbf{u} = (x_{1},\ldots,x_{n}) \; : \; x_{i} \in \mathbb{K}\} $ 
    μαζί με τις πράξεις 
    \[
        \mathbf{u}+ \mathbf{v} = (x_{1}+ y_{1}, \ldots , x_{n}+y_{n}) 
        \quad \text{και} \quad \lambda \mathbf{u} = 
        ( \lambda x_{1}, \ldots, \lambda x_{n})
    \]
    είναι ένας διανυσματικός χώρος επί του $ \mathbb{K} $. 

    Το μηδέν του χώρου 
    είναι $ \mathbf{0} = (0,\ldots,0) $, όπου $0$ είναι το μηδέν του σώματος 
    $ \mathbb{K} $ και το αντίθετο του $ \mathbf{u} $ είναι το $ - \mathbf{u} =
    (- x_{1}, \ldots, - x_{n}) $.

    Άρα ως ειδικές περιπτώσεις παίρνουμε για $ n=1 $ ότι το $ \mathbb{R} $ 
    είναι διανυσματικός χώρος επί του $ \mathbb{R} $, όπως επίσης και ότι το 
    $ \mathbb{C} $ είναι διανυσματικός χώρος επί του $ \mathbb{C} $.
\end{example} 

\begin{example}\label{ex:funs} 
    \textcolor{Col2}{Ο χώρος των πραγματικών συναρτήσεων 
    $\mathbf{F}(A, \mathbb{R})$ ή $\mathbb{R} ^{A}$}

    Αν $ A \subseteq \mathbb{R} $, τότε το σύνολο 
    $ \mathbf{F}{(A, \mathbb{R})} = 
    \{ f \colon A \to \mathbb{R} \; : \; f \; \text{συνάρτηση} \} $, μαζί με 
    τις πράξεις
    \[
        (f+g)(x) = f(x) +g(x), \; \forall x \in A \quad \text{και} 
        \quad (\lambda f)(x)= \lambda f(x), \; \forall x \in A
    \] 
    είναι ένας διανυσματικός χώρος επί του $ \mathbb{R} $.

    Το μηδέν του χώρου είναι η μηδενική συνάρτηση $ \mathbf{0} $, με τιμή 
    $ \mathbf{0}(x)=0, \; \forall x \in A $ και το αντίθετο της $f$ είναι 
    η συνάρτηση $ -f $ με τιμή $ (-f)(x) = - f(x), \; \forall x \in A $.
\end{example}

\begin{example}
    \textcolor{Col2}{Ο χώρος $\mathbf{P_{n}(\mathbb{K})}$ των 
    πολυωνύμων βαθμού $ \leq n $}.

    Έστω $ V = \mathbf{P_{n}}(\mathbb{K}) $ το σύνολο των πολυωνύμων βαθμού 
    $ \leq n, \; n \in \mathbb{N}  $ με συντελεστές από ένα σώμα 
    $ \mathbb{K} $. Δηλαδή 
    \[
        \mathbf{P_{n}}(\mathbb{K}) = \{ a_{n}x^{n}+a_{n-1}x^{n-1}+\cdots +
        a_{0} \; : \; a_{i} \in \mathbb{K} \}  
    \]
    Τότε αν 
    \begin{gather*}
        p \in \mathbf{P_{n}}(\mathbb{K}) \Rightarrow p(x)
        = a_{n}x^{n}+a_{n-1}x^{n-1}+\cdots + a_{0} \quad \text{και} 
        \quad q \in \mathbf{P_{n}}(\mathbb{K}) \Rightarrow 
        q(x) = b_{n}x^{n}+b_{n-1}x^{n-1}+\cdots + b_{0} 
    \end{gather*} 
    έχουμε, ότι το σύνολο $ V $ μαζί με τις πράξεις 
    \begin{gather*}
        (p+q)(x) = (a_{n}+ b_{n})x^{n} + 
        (a_{n-1}+b_{n-1})x^{n-1}+ \cdots + (a_{0}+ b_{0}) \\
        (\lambda p)(x) = (\lambda a_{n})x^{n}+
        ( \lambda a_{n-1})x^{n-1}+ \cdots + ( \lambda a_{0})
    \end{gather*} 
    είναι ένας διανυσματικός χώρος επί του $ \mathbb{K} $.

    Το μηδέν του χώρου είναι το μηδενικό πολυώνυμο $ \mathbf{0} $, 
    με τιμή $ \mathbf{0}(x)=0+0x+\cdots +0x^{n} $ και το αντίθετο του 
    $p$ είναι το $ - p $ με τιμή $ (- p)(x) = 
    (-a_{n})x^{n}+(-a_{n-1})x^{n-1}+\cdots + (-a_{0}), \; \forall x \in A $.  
\end{example}

\begin{example}
    \textcolor{Col2}{Ο χώρος $ \mathbf{P}(\mathbb{K}) $ των πολυωνύμων}.

    Έστω $ V = \mathbf{P}(\mathbb{K}) $ το σύνολο των πολυωνύμων  
    με συντελεστές από ένα σώμα $ \mathbb{K} $. Δηλαδή 
    \[
        \mathbf{P}(\mathbb{K}) = \{ a_{n}x^{n}+a_{n-1}x^{n-1}+\cdots + a_{0} \; 
        : \; \text{για κάποιο} \; n \in \mathbb{N}, \; a_{i} \in \mathbb{K}\}  
    \] 
    Τότε αν 
    \begin{gather*}
        p \in \mathbf{P}(\mathbb{K}) \Rightarrow p(x)
        = a_{n}x^{n}+a_{n-1}x^{n-1}+\cdots + a_{0} \quad \text{και} 
        \quad q \in \mathbf{P}(\mathbb{K}) \Rightarrow 
        q(x) = b_{m}x^{m}+b_{m-1}x^{m-1}+\cdots + b_{0} 
    \end{gather*} 
    και αν θέτωντας $ k = \max \{ n,m \} $ ξαναγράψουμε 
    \begin{gather*}
        p(x) = a_{k}x^{k}+a_{k-1}x^{k-1}+\cdots + a_{0} \quad \text{και} \quad
        q(x) = b_{k}x^{k}+b_{k-1}x^{k-1}+\cdots + b_{0} 
    \end{gather*}
    όπου οι επιπλέον όροι που προστέθηκαν έχουν συντελεστή 0, τότε έχουμε, 
    ότι το σύνολο $ V $ μαζί με τις πράξεις 
    \begin{gather*}
        (p+q)(x) = (a_{k}+ b_{k})x^{k} + 
        (a_{k-1}+b_{k-1})x^{k-1}+ \cdots + (a_{0}+ b_{0}) \\
        (\lambda p)(x) = (\lambda a_{k})x^{k}+
        ( \lambda a_{k-1})x^{k-1}+ \cdots + ( \lambda a_{0})
    \end{gather*} 
    είναι ένας διανυσματικός χώρος επί του $ \mathbb{K} $.
\end{example}

\begin{example}\label{ex:mat} 
  \textcolor{Col2}{Το σύνολο $ M_{m \times n}(\mathbb{K}) $ 
    των $ m \times n $ πινάκων με στοιχεία από το σώμα $ \mathbb{K} $}

    Το σύνολο $ M_{m \times n}(\mathbb{K}) $ των $ m \times n $ πινάκων 
    με στοιχεία από το σώμα $ \mathbb{K} $ είναι διανυσματικός χώρος 
    με πράξεις τη γνωστή πρόσθεση πινάκων και τον πολλαπλασιασμό 
    αριθμού επί πίνακα.

    Το μηδέν του χώρου είναι ο μηδενικός πίνακας, $ \mathbf{0} $ όπου 
    όλα του τα στοιχεία είναι μηδέν και αντίθετο του πίνακα $A$ είναι 
    ο πίνακας $ -A $.
\end{example}

\begin{example}\label{ex:omsy} 
    \textcolor{Col2}{Ο χώρος των λύσεων ενός γραμμικού, ομογενούς 
    συστήματος}

    Έστω το ομογενές σύστημα $ A\cdot X = \mathbf{0} $, όπου 
    $ A \in M_{m \times n}( \mathbb{R}) $ πίνακας

    \begin{equation*}
        \setlength\arraycolsep{1.5pt} 
        \left.
            \begin{array}{ccc ccc c @{\extracolsep{2.5pt}}c
                @{\extracolsep{2.5pt}}c}
                a_{11}x_{1} & + & a_{12}x_{2} & + & \cdots & + & a_{1n}x_{n} & =
                            & 0 \\
                a_{21}x_{1} & + & a_{22}x_{2} & + & \cdots & + & a_{2n}x_{n} & =
                            & 0 \\
                \vdots & & \vdots & & \ddots & &  \vdots & &  \vdots \\
                a_{m1}x_{1} & + & a_{m2}x_{2} & + & \cdots & + & a_{mn}x_{n} & =
                            & 0 \\
            \end{array}
        \right\} \Leftrightarrow
        \underbrace{\begin{pmatrix}
                a_{11} & a_{12} & \cdots & a_{1n} \\
                a_{21} & a_{22} & \cdots & a_{2n} \\
                \vdots & \vdots & \ddots & \vdots \\
                a_{m1} & a_{m2} & \cdots & a_{mn} 
        \end{pmatrix}}_{A_{m \times n}}
        \cdot 
        \underbrace{\begin{pmatrix*}
                x_{1} \\
                x_{2} \\
                \vdots \\
                x_{n}
        \end{pmatrix*}}_{X} = 
        \begin{pmatrix*}[r]
            0 \\
            0 \\
            \vdots \\
            0
        \end{pmatrix*}
    \end{equation*} 
    με $ \begin{pmatrix*}[r]
        x_{1} & x_{2} & \cdots & x_{n} 
    \end{pmatrix*}^{T} \in M_{n \times 1}(\mathbb{R}) \cong \mathbb{R}^{n} $.
    Τότε, ως προς τις πράξεις τους χώρου 
    $ M_{n \times 1}(\mathbb{R}) $, του παραδείγματος~\ref{ex:mat}, 
    το σύνολο των λύσεων είναι ένας διανυσματικός χώρος 
    (που περιέχεται στο χώρο $ M_{n \times 1}(\mathbb{R}) \cong 
    \mathbb{R}^{n} $), ονομάζεται \textcolor{Col2}{Μηδενόχωρος} του πίνακα 
    $A$ και συμβολίζεται με $ N(A) $. Δηλαδή
    \[
        N(A) = \{ \mathbf{X} \in \mathbb{R}^{n} \; : \; A \cdot \mathbf{X} = 
        \mathbf{0} \}  
    \] 
    Πράγματι, αν $ \mathbf{X} =\begin{pmatrix*}[r]
        x_{1} & x_{2} & \cdots & x_{n} 
        \end{pmatrix*}^{T} $ και $ \mathbf{Y} = \begin{pmatrix*}[r]
        y_{1} & y_{2} & \cdots & y_{n} 
    \end{pmatrix*}^{T} $ είναι λύσεις του συστήματος, τότε
    \begin{gather*}
        A \cdot (\mathbf{X}+\mathbf{Y}) = A \cdot \mathbf{X} + A 
        \cdot \mathbf{Y} = \mathbf{0}+ \mathbf{0} = \mathbf{0} \\
        A \cdot (\lambda \mathbf{X}) = \lambda (A \cdot \mathbf{X}) = 
        \lambda \mathbf{0} = \mathbf{0}
    \end{gather*} 
\end{example}

\begin{rem}\label{rem:lines}
    Οι εξισώσεις $ ax + by = 0 $ και $ ax + by + cz = 0 $ είναι ειδικές 
    περιπτώσεις ομογενούς συστήματος. Οπότε σύμφωνα με το παράδειγμα~\ref{ex:omsy} 
    οι λύσεις τους, δηλαδή τα σύνολα των σημείων όπου επαληθεύουν αυτές τις εξισώσεις, 
    είναι διανυσματικοί χώροι. Επομένως ευθείες του επιπέδου $ \mathbb{R}^{2} $ 
    όπου περνούν από την αρχή των αξόνων, αλλά και ευθείες και επίπεδα του χώρου 
    $ \mathbb{R}^{3} $ που επίσης περνούν από την αρχή των αξόνων είναι 
    διανυσματικοί χώροι.
\end{rem}

\begin{rem}
    Αν $ \mathbf{u}, \mathbf{v} \in V $ τότε θα γράφουμε ότι 
    $ \mathbf{u} - \mathbf{v} = \mathbf{u} + (- \mathbf{v}) $, 
    όπου $ - \mathbf{v} $ είναι το \textcolor{Col2}{αντίθετο} του $ \mathbf{v} $.
\end{rem}

\begin{exercise}
    Έστω $ V = \mathbb{R}_{+} $, το σύνολο των θετικών πραγματικών αριθμών, με 
    πράξεις
    \begin{gather*}
         x \oplus y = xy \\
         \lambda \odot x = x^{\lambda}
     \end{gather*} 
     Να εξετάσετε αν ο $V$ με τις πράξεις αυτές είναι διανυσματικός χώρος επί του 
     $ \mathbb{R} $.
\end{exercise}

\begin{solution}
\item {}
  \begin{enumerate}[i)]
        \item $ x \oplus y = xy = yx = y \oplus x $
        \item $ (x \oplus y) \oplus z = (xy) \oplus z = (xy)z = x(yz) = x \oplus (yz) 
            = x \oplus (y \oplus z) $
        \item Έχουμε $x \oplus 1 = x\cdot 1= 1 \cdot x = 1 \oplus x = x, \; 
            \forall x \in V $, άρα το μηδενικό στοιχείο του $V$ είναι το 1.
        \item Έχουμε $ x \in V $ τότε $ x \oplus \frac{1}{x} = x \cdot \frac{1}{x} = 
            \frac{1}{x} \cdot x = \frac{1}{x} \oplus x = 1, \; \forall x \in V $, άρα 
            το αντίθετο του $x$ είναι το $ \frac{1}{x} $.
        \item Αν $ x,y \in V $ και $ \lambda \in \mathbb{R} $ τότε $ \lambda \odot 
            (x \oplus y) = \lambda \odot (xy) = {(xy)}^{\lambda} = 
            x^{\lambda } y^{\lambda} = x^{\lambda } \oplus y^{\lambda } = 
            (\lambda \odot x) \oplus (\lambda \odot y ) $
        \item Αν $ \lambda , \mu \in \mathbb{R} $ και $ x \in V $, τότε 
            $ (\lambda + \mu ) \odot x = x^{\lambda + \mu } = 
            x^{\lambda } x^{\mu} = x^{\lambda} \oplus x^{\mu } = 
            (\lambda \odot x) \oplus (\mu \odot x)  $ 
        \item Αν $ \lambda , \mu \in \mathbb{R} $ και $ x \in V $, τότε 
            $ (\lambda \mu ) \odot x = x^{\lambda \mu } = {(x^{\mu })}^{\lambda } = 
            (\mu \odot x)^{\lambda } = \lambda \odot (\mu \odot x)  $
        \item Αν $ x \in V $, τότε $ 1 \odot x = x^{1} = x $
    \end{enumerate}
\end{solution}

\begin{prop}
\item {}
    \begin{enumerate}[i)]
        \item Το ουδέτερο στοιχείο ενός διανυσματικού χώρου είναι μοναδικό.
            \begin{proof}
            \item {}
                Έστω ότι υπάρχουν δύο ουδέτερα $ \mathbf{0} $ και $ \mathbf{0}' $. 
                Τότε από το αξίωμα 3 έχουμε:
                \begin{align*}
                    \mathbf{0}+ \mathbf{0}' = \mathbf{0} \quad 
                    \text{αφού $ \mathbf{0}'$ 
                    είναι ουδέτερο} \hfill\tikzmark{a} \\
                    \mathbf{0}'+ \mathbf{0} = \mathbf{0}' \quad 
                    \text{αφού $ \mathbf{0} $ είναι ουδέτερο}\hfill\tikzmark{b} 
                \end{align*} 
                \mybrace{a}{b}[$ \mathbf{0} = \mathbf{0}' $]
            \end{proof}
        \item Το αντίθετο στοιχείο ενός διανυσματικού χώρου είναι μοναδικό.
            \begin{proof}
            \item {}
                Έστω ότι υπάρχουν δύο αντίθετα $ \mathbf{u}' $ και $ \mathbf{u}''$. 
                Τότε από το αξίωμα 4 έχουμε:
                \begin{align*}
                    \mathbf{u} + \mathbf{u}' = \mathbf{0} \quad 
                    \text{αφού $ \mathbf{u}'$ είναι αντίθετο} \\
                    \mathbf{u} +  \mathbf{u}''= \mathbf{0} \quad 
                    \text{αφού $ \mathbf{u}'' $ είναι αντίθετο}
                \end{align*} 
                Άρα
                \[
                    \mathbf{u}' = \mathbf{u}' + \mathbf{0}= \mathbf{u}' + (\mathbf{u} 
                    + \mathbf{u}'') = (\mathbf{u}' + \mathbf{u}) + \mathbf{u}'' = 
                    \mathbf{0} + \mathbf{u}'' = \mathbf{u}'' 
                \] 
            \end{proof}
    \end{enumerate}
\end{prop}

\begin{thm}
\item {}
    \begin{enumerate}[i)]
        \item $ \mathbf{u} + \mathbf{v} = \mathbf{w} + \mathbf{v} 
            \Rightarrow \mathbf{u} = \mathbf{w}, \quad \forall \mathbf{u}, 
            \mathbf{v}, \mathbf{w} \in V $ \quad (νόμος διαγραφής)
        \item $ 0 \cdot \mathbf{u} = \mathbf{0}, \quad \forall \mathbf{u} \in V $
        \item $ \lambda \cdot \mathbf{0} = \mathbf{0}, \quad \forall \lambda \in 
            \mathbb{K} $
        \item $ (-1)\cdot \mathbf{u} = - \mathbf{u}, \quad \forall \mathbf{u} \in V $ 
        \item $ \lambda \cdot (\mathbf{u} - \mathbf{v}) = 
            \lambda \mathbf{u} - \lambda \mathbf{v}, \quad \forall \mathbf{u}, 
            \mathbf{v} \in V $ και $ \lambda \in \mathbb{K} $
        \item $ (\lambda - \mu ) \cdot \mathbf{u} = \lambda \mathbf{u} - 
            \mu \mathbf{u}, \quad \forall \mathbf{u} \in V $ και $ \lambda, 
            \mu \in \mathbb{K} $
        \item $ -(- \lambda ) \cdot \mathbf{u} = \lambda \mathbf{u}, 
            \quad \forall \mathbf{u} \in V $ και $ \lambda \in \mathbb{K} $
    \end{enumerate}
\end{thm}

\begin{proof}
\item {}
    \begin{enumerate}[i)]
        \item $ \mathbf{u} = \mathbf{u} + \mathbf{0} = \mathbf{u} + 
            (\mathbf{v} + (- \mathbf{v})) = (\mathbf{u} + \mathbf{v}) + 
            (- \mathbf{v}) = (\mathbf{w} + \mathbf{v}) + (- \mathbf{v}) = \mathbf{w} + 
            (\mathbf{v} + (- \mathbf{v})) = \mathbf{w} + \mathbf{0} = \mathbf{w}$
        \item Έχουμε $ 0\cdot \mathbf{u} = (0 + 0) 
            \cdot \mathbf{u}= 0\cdot \mathbf{u}+ 0\cdot \mathbf{u} $, άρα
            \begin{align*}
                0\cdot \mathbf{u} = 0\cdot \mathbf{u} + 0\cdot
                \mathbf{u}\tikzmark{a} \\
                0\cdot \mathbf{u} = \mathbf{0} + 0\cdot \mathbf{u} 
                \tikzmark{b}
            \end{align*} 
            \mybrace{a}{b}[$ 0\cdot \mathbf{u} + 0\cdot \mathbf{u} = 
            \mathbf{0}+ 0\cdot \mathbf{u} \overset{\text{i)}}{\Rightarrow}  
            0\cdot \mathbf{u} = \mathbf{0}$]
        \item Έχουμε $ \lambda \cdot \mathbf{0} = \lambda \cdot 
            (\mathbf{0} + \mathbf{0}) = \lambda \cdot \mathbf{0} + \lambda 
            \cdot \mathbf{0}$, άρα 
            \begin{align*}
                \lambda \cdot \mathbf{0} = \lambda \cdot \mathbf{0} + \lambda \cdot 
                \mathbf{0} \tikzmark{a} \\
                \lambda \cdot \mathbf{0} = \mathbf{0} + \lambda \cdot \mathbf{0}
                \tikzmark{b}
            \end{align*}
            \mybrace{a}{b}[$ \lambda \cdot \mathbf{0}+ \lambda \cdot \mathbf{0} = 
            \mathbf{0} + \lambda \cdot \mathbf{0} \overset{\text{i)}}{\Rightarrow}  
            \lambda \cdot \mathbf{0} = \mathbf{0}$]
        \item Έχουμε $ \mathbf{u} + (-1)\cdot \mathbf{u} = 1\cdot 
            \mathbf{u} + (-1)\cdot \mathbf{u} = [1+(-1)] \cdot \mathbf{u}= 0 
            \cdot \mathbf{u} = \mathbf{0} $. 

            Όμως το αντίθετο του $ \mathbf{u} $ είναι μοναδικό. Επομένως 
            $ (-1)\cdot \mathbf{u} = - \mathbf{u}$.

        \item Έχουμε $ \lambda\cdot (\mathbf{u} - \mathbf{v}) + \lambda \cdot 
            \mathbf{v} = \lambda \cdot [(\mathbf{u} - \mathbf{v}) + \mathbf{v}] = 
            \lambda \cdot [\mathbf{u} + (- \mathbf{v}+ \mathbf{v})] = \lambda \cdot 
            ( \mathbf{u} + \mathbf{0}) = \lambda \cdot \mathbf{u} $, άρα 
            \begin{gather*}
                \lambda \cdot (\mathbf{u} - \mathbf{v}) + \lambda \cdot \mathbf{v} = 
                \lambda \cdot \mathbf{u} \Leftrightarrow \\ 
                [\lambda \cdot (\mathbf{u} - \mathbf{v}) + \lambda \cdot \mathbf{v}] -
                (\lambda \cdot \mathbf{v})= 
                \lambda \cdot \mathbf{u} - \lambda \cdot \mathbf{v} \Leftrightarrow \\
                \lambda \cdot (\mathbf{u} - \mathbf{v}) + 
                [\lambda \cdot \mathbf{v} - \lambda \cdot \mathbf{v}] =
                \lambda \cdot \mathbf{u} - \lambda \cdot \mathbf{v} \Leftrightarrow \\
                \lambda \cdot (\mathbf{u}- \mathbf{v}) + \mathbf{0} = \lambda \cdot 
                \mathbf{u} - \lambda \cdot \mathbf{v} \Leftrightarrow \\
                \lambda \cdot (\mathbf{u} - \mathbf{v}) = \lambda \cdot \mathbf{u} - 
                \lambda \cdot \mathbf{v}
            \end{gather*}

        \item Έχουμε $ (\lambda - \mu ) \cdot \mathbf{u} + \mu \cdot \mathbf{u} = 
            (\lambda - \mu + \mu )\cdot \mathbf{u} = \lambda \cdot \mathbf{u}$, άρα 
            \begin{gather*}
                (\lambda - \mu ) \cdot \mathbf{u} + \mu \cdot \mathbf{u} = 
                \lambda \cdot \mathbf{u} \Leftrightarrow \\
                [(\lambda - \mu )\cdot \mathbf{u} + \mu \cdot \mathbf{u}] - 
                \mu \cdot \mathbf{u} = \lambda \cdot \mathbf{u}- \mu \cdot 
                \mathbf{u} \Leftrightarrow \\
                (\lambda - \mu ) \cdot \mathbf{u} + [\mu \cdot \mathbf{u} - 
                \mu \cdot \mathbf{u}] = \lambda \cdot \mathbf{u}- \mu \cdot 
                \mathbf{u} \Leftrightarrow \\
                (\lambda - \mu )\cdot \mathbf{u} + \mathbf{0} = 
                \lambda \cdot \mathbf{u}- \mu \cdot \mathbf{u} \Leftrightarrow \\
                (\lambda - \mu )\cdot \mathbf{u} = 
                \lambda \cdot \mathbf{u}- \mu \cdot \mathbf{u} \Leftrightarrow \\
            \end{gather*}

        \item $ -(- \mathbf{u}) = -1\cdot [(-1)\cdot \mathbf{u}] = [(-1)(-1)] \cdot 
            \mathbf{u}= 1 \cdot \mathbf{u} = \mathbf{u} $
    \end{enumerate}
\end{proof}

\begin{prop} \label{prop:prod}
\item {}
    Αν $ \lambda \in \mathbb{K}, \; \mathbf{u} \in V $ και $ \lambda \cdot \mathbf{u} =
    \mathbf{0} $ τότε $ \lambda = 0 $ ή $ \mathbf{u} = \mathbf{0} $. 
\end{prop}

\begin{proof}
\item {}
    Έστω $ \lambda \neq 0 $. Θα δείξουμε ότι αναγκαστικά $ \mathbf{u} = \mathbf{0} $. 

    Πράγματι, αφού $ \lambda \neq 0 \Rightarrow \exists \lambda ^{-1} \in \mathbb{K} $, 
    όπου ισχύει ότι $ \lambda \cdot \lambda ^{-1} = 1 $. Οπότε 
    \begin{gather*}
        \lambda \cdot \mathbf{u} = \mathbf{0} \Leftrightarrow 
        \lambda ^{-1}\cdot (\lambda \cdot \mathbf{u}) = \lambda ^{-1} \cdot \mathbf{0}
        \Leftrightarrow 
        (\lambda \cdot \lambda ^{-1}) \cdot \mathbf{u} = \mathbf{0} \Leftrightarrow 
        1 \cdot \mathbf{u} = \mathbf{0} \Leftrightarrow 
        \mathbf{u} = \mathbf{0} 
    \end{gather*} 
\end{proof}



\section{Υπόχωροι}


\begin{dfn}
\item {}
    Έστω $V(\mathbb{K}) $ διανυσματικός χώρος. Ένα υποσύνολο $W$ του $V$ λέγεται 
    \textcolor{Col2}{υπόχωρος} του $ V $, αν είναι διανυσματικός χώρος επί του 
    $ \mathbb{K} $, ως προς τις πράξεις του $V$ και το συμβολίζουμε με $ W \leq V $.
\end{dfn}



\begin{prop}
    \label{prop:subsp}
\item {}
    Έστω $ (V(\mathbb{K}),+,\cdot) $ ένας διανυσματικός χώρος και $ W \subseteq V $.  
    Τότε το $ W $ είναι υπόχωρος του $V$ αν και μόνον αν ικανοποιούνται οι 
    παρακάτω συνθήκες:
    \begin{enumerate}[i)]
        \item $ \mathbf{0}_{V} \in W $
        \item $ \mathbf{u}+ \mathbf{v} \in W, \quad \forall \mathbf{u}, 
            \mathbf{v} \in W $ \quad ( $W$ κλειστό ως προς $ + $)
        \item $ \lambda \mathbf{u} \in W, \quad \forall \mathbf{u} \in W \; 
            \text{και} \; \lambda \in \mathbb{K} $ \quad ( $W$ κλειστό ως προς 
            $ \cdot $)
    \end{enumerate}
\end{prop}

\begin{proof}
\item {}
    \begin{description}
        \item [($ \Rightarrow $)] Αν $W \leq V$ τότε $W$ είναι $ \mathbb{K} $ - χώρος
            οπότε ικανοποιούνται αυτόματα οι συνθήκες $ \mathrm{ii)} $ και 
            $ \mathrm{iii)} $ και επίσης, ο $ W $ έχει μηδενικό στοιχείο, 
            άρα είναι μη-κενός. 
            Για το $ \mathrm{i)} $ θα δείξουμε ότι $ \mathbf{0}_{W} = \mathbf{0}_{V} $.
            Πράγματι
            \begin{align*}
                \mathbf{0}_{W} &= \mathbf{0}_{W} + \mathbf{0}_{W} 
                \quad (W \; \text{διαν. χώρος}) \\
                \mathbf{0}_{W} &= \mathbf{0}_{W} + \mathbf{0}_{V} 
                \; \quad (V \; \text{διαν. χώρος και $ \mathbf{0}_{W} \in V $})
            \end{align*}
            Άρα $ \mathbf{0}_{W} + \mathbf{0}_{W} = \mathbf{0}_{W} + \mathbf{0}_{V} 
            \Rightarrow
            \mathbf{0}_{W} = \mathbf{0}_{V} $. Άρα $ \mathbf{0}_{V} \in W $.
        \item [($ \Leftarrow $)] Αντίστροφα, αν ισχύουν οι συνθήκες 
            $ \mathrm{i)}, \mathrm{ii)} $ και $ \mathrm{iii)} $, και από το 
            γεγονός ότι $ W \subseteq V $, όπου $V$ είναι διανυσματικός χώρος, 
            έπεται ότι προφανώς ικανοποιούνται τα αξιώματα του ορισμού, για το $W$, 
            εκτός ίσως από το $ 4 $, για το οποίο έχουμε:
            Έστω $ \mathbf{u} \in W  $. Τότε το $ - \mathbf{u} \in V $, 
            γιατί ο $V$ είναι διανυσματικός χώρος, θα ανήκει και στον $ W $, 
            γιατί:
            \[
                - \mathbf{u} = (-1)\cdot \mathbf{u} \in W \quad 
                \text{λόγω του $\mathrm{iii)}$} \; \text{γιατί $-1 \in \mathbb{K}$ 
                \; και \; $ \mathbf{u} \in W$ } 
            \]
    \end{description}
\end{proof}


\section{Παραδείγματα}

\begin{example}
    Το σύνολο $ \{ \mathbf{0}_{V} \} $ είναι υπόχωρος του $V$, 
    για κάθε $V$ διανυσματικό χώρο.
\end{example}

\begin{example}\label{ex:linesplanes} 
    \textcolor{Col2}{Ευθείες και επίπεδα}

    Όπως είδαμε στην παρατήρηση~\ref{rem:lines} δοθέντων 
    $ a,b \in \mathbb{R} $ το  σύνολο 
    $
    W = \{(x,y)\in \mathbb{R}^{2} \; : \; ax+by=0 \} 
    $
    των ευθειών του $ \mathbb{R}^{2} $ που διέρχονται από την αρχή των 
    αξόνων είναι διανυσματικός χώρος, υπόχωρος τους $ \mathbb{R}^{2} $.

    Επίσης, δοθέντων $ a,b,c \in \mathbb{R}^{3} $, τα σύνολα  
    \begin{align*}
        W &= 
        \{
            (x,y,z) \in \mathbb{R}^{3} \; : \; ax+by+cz=0 
        \} 
        \quad (\text{επίπεδα του $ \mathbb{R}^{3}$ που διέρχονται από 
        $(0,0,0)$}) \\
        W &= 
        \{
            (x,y,z) \in \mathbb{R}^{3} \; : \; ax+by+cz=0, 
            \; \text{και} \; a'x+b'y+c'z=0 
        \} 
        \quad (\text{ευθείες του $ \mathbb{R}^{3}$ που διέρχονται από 
        $(0,0,0)$}) 
    \end{align*}
    είναι διανυσματικοί χώροι, υπόχωροι του $ \mathbb{R}^{3} $.
\end{example}

\begin{example}
    \textcolor{Col2}{Ο χώρος των συνεχών πραγματικών συναρτήσεων 
    $ \mathbf{C}[a,b] $} 

    Έστω $ W = \mathbf{C}{[a,b]} = \{ f \colon [a,b] \to \mathbb{R} \; 
    : \; f \; \text{συνεχής συνάρτηση} \} \subseteq 
    \mathbf{F}(A, \mathbb{R}) $. 

    Τότε ως προς τις πράξεις του 
    παραδείγματος~\ref{ex:funs}, των διανυσματικών χώρων ο $W$ 
    είναι ένας διανυσματικός χώρος επί του $ \mathbb{R} $, υπόχωρος του 
    $\mathbf{F}(A, \mathbb{R})$,
    γιατί το άθροισμα $ f+g $ δύο συνεχών συναρτήσεων $ f,g $ είναι 
    συνεχής συνάρτηση καθώς και το βαθμωτό γινόμενο 
    $ \lambda f, \; \lambda \in \mathbb{R} $, μιας συνεχούς συνάρτησης 
    $f$ είναι επίσης συνεχής συνάρτηση. 
\end{example}

\begin{example}\label{ex:c1} 
    \textcolor{Col2}{Ο χώρος των διαφορίσιμων με 
    συνεχή παράγωγο πραγματικών συναρτήσεων $ \mathbf{C^{1}}[a,b] $} 

    Έστω $ W = \mathbf{C^{1}}{[a,b]} = \{ f \colon [a,b] \to \mathbb{R} \; 
    : \; f \; \text{διαφορίσιμη με συνεχή παράγωγο} \} \subseteq 
    \mathbf{C}[a,b] $. 

    Τότε ως προς τις πράξεις του 
    παραδείγματος~\ref{ex:funs}, των διανυσματικών χώρων ο $W$ 
    είναι ένας διανυσματικός χώρος επί του $ \mathbb{R} $, υπόχωρος του 
    $ \mathbf{C}[a,b] $.
\end{example}

\begin{example}
    \textcolor{Col2}{Ο χώρος των λύσεων μιας γραμμικής, ομογενούς 
    διαφορικής εξίσωσης}

    Έστω μια γραμμική, ομογενής διαφορική εξίσωση $ f''(x)+5f'(x)+6f(x)=0 $ 
    και έστω 
    \[
        W = \{ f \in \mathbf{C^{2}}(\mathbb{R}) \; : \; 
        \text{$f$ λύση της διαφ. εξίσωσης} \} \subseteq \mathbf{C}[a,b]
    \] 
    Τότε, ως προς τις πράξεις του παραδείγματος~\ref{ex:funs}, των 
    διανυσματικών χώρων, το 
    σύνολο $ W $ των λύσεων είναι ένας διανυσματικός χώρος επί του
    $ \mathbb{R} $, υπόχωρος του $ \mathbf{C}[a,b] $. 
    Πράγματι, αν $ f,g $ λύσεις της διαφ. εξίσωσης δηλαδή αν 
    $ f,g \in V $, έχουμε:
    \begin{gather*}
        (f+g)''(x)+5(f+g)'(x)+6(f+g)(x) = [f''(x)+5f'(x)+6f(x)] 
        + [g''(x)+5g'(x)+6g(x)] = 0+0=0 \\
        (\lambda f)''(x) + 5(\lambda f)'(x)+6(\lambda f)(x)= \lambda
        [f''(x)+5f'(x)+6f(x)] = \lambda \cdot 0=0
    \end{gather*}
    Συνεπώς $ f+g \in V $ και $ \lambda f \in V $. 
\end{example}

\begin{example}
    \textcolor{Col2}{Το σύνολο των συμμετρικών πινάκων}

    Έστω $ V = M_{n}(\mathbb{K}) $ και $ W = \{ A \in M_{n}(\mathbb{K}) \;
    : \; A^{T}=A \}  $. Τότε $ W \leq V $.
    \begin{proof}
    \item {}
        \begin{enumerate}[i)]
            \item  $ \mathbf{0} \in W $, αφού $ \mathbf{0}^{T}= 
                \mathbf{0} $
            \item Αν $ A, B \in W \Rightarrow A^{T}=A $ και $ B^{T}=B $, 
                οπότε $ (A+B)^{T}= A^{T}+B^{T}=A+B $, άρα $A+B \in W$
            \item Αν $ A \in W $ και $ \lambda \in \mathbb{K} $, τότε 
                $A^{T}=A$, οπότε $(\lambda A)^{T} = \lambda A^{T} = 
                \lambda A  $ , άρα $ \lambda A \in W $
        \end{enumerate}
    \end{proof}
\end{example}

\begin{example}
    \textcolor{Col2}{Το σύνολο των αντι-συμμετρικών πινάκων}

    Έστω $ V = M_{n}(\mathbb{K}) $ και $ W = \{ A \in M_{n}(\mathbb{K}) \;
    : \; A^{T}=-A \}  $. Τότε $ W \leq V $.
    \begin{proof}
    \item {}
        \begin{enumerate}[i)]
            \item  $ \mathbf{0} \in W $, αφού $ \mathbf{0}^{T}= 
                \mathbf{0} = - \mathbf{0} $
            \item 
                Αν $ A, B \in W \Rightarrow A^{T}=-A $ και $ B^{T}=-B $, οπότε
                $ (A+B)^{T}= A^{T}+B^{T}=-A-B = - (A+B) $, άρα $A+B \in W$
            \item Αν $ A \in W $ και $ \lambda \in \mathbb{K} $, τότε 
                $A^{T}=-A$, οπότε $(\lambda A)^{T} = \lambda A^{T} =
                -\lambda A  $, άρα $ \lambda A \in W $
        \end{enumerate}
    \end{proof}
\end{example}

\begin{example}
    \textcolor{Col2}{Το σύνολο των άνω τριγωνικών πινάκων}

    Έστω $ V = M_{n}(\mathbb{K}) $ και $ W = \{ A \in M_{n}(\mathbb{K}) \;
    : \; \; a_{ij} = 0, \; i>j \}  $. Τότε $ W \leq V $.

    \begin{proof}
    \item {}
        \begin{enumerate}[i)]
            \item  $ \mathbf{0} \in W $, αφού $ \mathbf{0} $ 
                προφανώς είναι άνω τριγωνικός.
            \item  Αν $ A, B \in W \Rightarrow a_{ij} = b_{ij} = 0, 
                \; i>j$, οπότε $a_{ij} + b_{ij} = 0, \; i>j$, άρα $A+B \in W$
            \item Αν $ A \in W $ και $ \lambda \in \mathbb{K} $, τότε 
                $a_{ij} = 0, \; i>j$, οπότε $ \lambda a_{ij} = 0, \; i>j$, 
                άρα $ \lambda A \in W $
        \end{enumerate}
    \end{proof}
\end{example}

\begin{example}
    \textcolor{Col2}{Το σύνολο των κάτω τριγωνικών πινάκων}
    \begin{proof}
        Ομοίως
    \end{proof}
\end{example}

\begin{example}
    \textcolor{Col2}{Το σύνολο των διαγώνιων πινάκων}

    \begin{proof}
        Ομοίως
    \end{proof}
\end{example}

\begin{example}
\item Έστω $ V = \mathbf{P}(\mathbb{K}) $ και 
    $ W = \mathbf{P_{n}}(\mathbb{K})$.  Τότε $ W \leq V $.  
    \begin{proof}
    \item {}
        \begin{enumerate}[i)]
            \item   $ \mathbf{0} \in W $, γιατί προφανώς 
                $ \mathbf{0} = 0 x_{n} + 0 x^{n-1} + \cdots + 0 $
            \item 
                Αν $ p, q \in W $, δηλαδή $ p,q $ πολυώνυμα βαθμού $ \leq n $, 
                τότε και $ p+q $ είναι πολυώνυμο βαθμού $ \leq n $, άρα 
                $p+q \in W$
            \item Αν $ p \in W $ και $ \lambda \in \mathbb{K} $, τότε 
                $p$, πολυώνυμο βαθμού $ \leq n $ οπότε $ \lambda p$ πολυώνυμο 
                βαθμού $ \leq n $, άρα $ \lambda p \in W $
        \end{enumerate}
    \end{proof}
\end{example}

\begin{prop}
    Ισχύουν οι ιδιότητες:
    \begin{enumerate}[i)]
        \item $ V \leq V $
        \item Αν $ W \leq V $ και $ V \leq W $, τότε $ V = W $.
        \item Αν $ U \leq W $ και $ W \leq V $, τότε $ U \leq V $.
    \end{enumerate}
\end{prop}

\begin{rem}
    Οι υπόχωροι $ \{ \mathbf{0}_{V} \} $ και $V$, ενός διανυσματικού χώρου $V$, 
    λέγονται \textcolor{Col2}{τετριμμένοι υπόχωροι}.
\end{rem}


\begin{prop}[Τομή 2 υποχώρων] \item {}
    Αν $ W_{1}, W_{2} $ υπόχωροι του $V$, τότε και $ W_{1} \cap W_{2} $ είναι 
    επίσης υπόχωρος του $V$.
\end{prop}
\begin{proof}
\item {}
    Έχουμε ότι $ W_{1} \cap W_{2} \neq \emptyset $, γιατί 
    $ \mathbf{0} \in W_{1} \cap W_{2} $.  Πράγματι:
    \begin{align*}
        W_{1} \leq V \Rightarrow \mathbf{0} \in W_{1} \tikzmark{a} \\
        W_{2} \leq V \Rightarrow \mathbf{0} \in W_{2} \tikzmark{b}
    \end{align*} 
    \mybrace{a}{b}[$ \mathbf{0} \in W_{1} \cap W_{2}$] 
    \begin{description}
        \item [i)] Έχουμε
            \begin{align*}
                \mathbf{w}_{1} \in W_{1} \cap W_{2} \Rightarrow 
                \mathbf{w}_{1} \in W_{1} \; 
                \text{και} \; \mathbf{w}_{1} \in W_{2} \tikzmark{a} \\
                \mathbf{w}_{2} \in W_{1} \cap W_{2} \Rightarrow 
                \mathbf{w}_{2} \in W_{1} \; 
                \text{και} \; \mathbf{w}_{2} \in W_{2} \tikzmark{b} 
                \mybrace{a}{b}[$ \mathbf{w}_{1}+ \mathbf{w}_{2} \in W_{1} 
                \; \text{και} \; \mathbf{w}_{1}+ \mathbf{w}_{2} \in W_{2} $] 
            \end{align*}
            άρα $ \mathbf{w}_{1}+ \mathbf{w}_{2} \in W_{1} \cap W_{2}$ 
        \item [ii)]
            Αν $ \mathbf{w} \in W_{1} \cap W_{2} $ και $ \lambda \in \mathbb{K} $, 
            τότε 
            $ \mathbf{w} \in W_{1} $ και $ \mathbf{w} \in W_{2} $, 
            οπότε $ \lambda \mathbf{w} \in W_{1} $ και 
            $ \lambda \mathbf{w} \in W_{2} $, άρα $ \lambda \mathbf{w} \in W_{1} 
            \cap W_{2} $
    \end{description}
\end{proof}

\begin{prop}[Τομή οποιουδήποτε πλήθους υποχώρων]
\item {}
    Έστω $ I $ σύνολο και $ \{ W_{i} \; : \; i \in I \}$ οικογένεια υποχώρων του $V$. 
    Τότε το σύνολο $ W = \smash{\bigcap\limits_{i \in I}} W_{i} = 
    \{ \mathbf{v} \in V \; : \; \mathbf{v} \in W_{i}, \; \forall i \in I \} $ 
    είναι υπόχωρος του $V$.
\end{prop}

\begin{rem}
    Η ένωση 2 υποχώρων $ W_{1} $ και $ W_{2} $ ενός διανυσματικού χώρου $V$ δεν είναι 
    απαραίτητα υπόχωρος του $V$. Πράγματι, αν $ W_{1} $ είναι ο άξονας $x$ και 
    $ W_{2} $ ο άξονας $y$, υπόχωροι του $ \mathbb{R}^{2} $ 
    (ως ευθείες που διέρχονται από το (0,0)), τότε έχουμε
    \begin{align*}
        \mathbf{w}_{1} = (2,0) \in W_{1} \tikzmark{a} \\
        \mathbf{w}_{2} = (0,3) \in W_{2} \tikzmark{b} 
    \end{align*} 
    \mybrace{a}{b}[$ \mathbf{w}_{1}+ \mathbf{w}_{2} = (2,3) \not \in W_{1} 
    \cup W_{2} $]
\end{rem}

\begin{prop}
\item {}
    Η ένωση δύο υποχώρων $ W_{1} $ και $ W_{2} $ ενός διανυσματικού χώρου $V$ 
    είναι υπόχωρος του $V$ αν και μόνον αν $ W_{1} \leq W_{2} $ ή $ W_{2} \leq W_{1} $.
\end{prop}
\begin{proof}
\item {}
    \begin{description}
        \item [$(\Leftarrow)$] Αν $ W_{1} \leq W_{2} $ τότε 
            $ W_{1} \cup W_{2} = W_{2} $ και αν $ W_{2} \leq W_{1} $ τότε 
            $ W_{1} \cup W_{2} = W_{1} $. 
            Σε κάθε περίπτωση $ W_{1} \cup W_{2} $ είναι υπόχωρος του $V$.
        \item [$(\Rightarrow)$ (Με άτοπο)]  Έστω ότι $ W_{1} \cup W_{2} $ είναι 
            υπόχωρος του $V$ και $ W_{1} \not \leq W_{2} $ και $ W_{2} \not \leq 
            W_{1}$. Τότε υπάρχει $ \mathbf{v}_{2} \in W_{2} \setminus W_{1} $ και 
            $ \mathbf{v}_{1} \in W_{1} \setminus W_{2}$. 
            Όμως $ \mathbf{v} = \mathbf{v}_{1}+ \mathbf{v}_{2} \in W_{1} \cup W_{2} $, 
            αφού $ W_{1} \cup W_{2} $ υπόχωρος. Τότε έχουμε 
            \begin{align*}
                \text{Αν $ \mathbf{v} \in W_{1}$ τότε} \; \mathbf{v}_{2}= 
                \mathbf{v}- \mathbf{v}_{1} \in W_{1} \quad \text{άτοπο} \\
                \text{Αν $ \mathbf{v} \in W_{2}$ τότε} \; \mathbf{v}_{1}= 
                \mathbf{v}- \mathbf{v}_{2} \in W_{2} \quad \text{άτοπο}  
            \end{align*}
    \end{description}
\end{proof}

\begin{dfn}[Άθροισμα Υποχώρων]
    Έστω $V$ διανυσματικός χώρος και $ W_{1}, W_{2} \leq V $. Το 
    \textcolor{Col2}{άθροισμα} των $ W_{1} $ και $ W_{2} $ ορίζουμε να είναι το σύνολο 
    \[
        W_{1}+W_{2} = \{ \mathbf{w}_{1}+ \mathbf{w}_{2} \; : \; \mathbf{w}_{1} 
        \in W_{1} \; \text{και} \; \mathbf{w}_{2} \in W_{2} \} 
    \]
    Πιο γενικά, αν $ n \in \mathbb{N} $ και $ W_{1}, \ldots, W_{n} \leq V $, ορίζουμε
    \[
        W_{1}+\cdots +W_{n} = \{ \mathbf{w_{1}}+\cdots + \mathbf{w}_{n} \; : \; 
        \mathbf{w}_{i} \in W_{i}, \; \forall i, \; 1 \leq i \leq n \} 
    \] 
\end{dfn}

\begin{prop}
    Το άθροισμα $ W_{1}+W_{2} $ είναι υπόχωρος του $V$ και ισχύει ότι είναι ο 
    \textbf{ελάχιστος} υπόχωρος του $V$ που περιέχει την ένωση 
    $ W_{1} \cup W_{2} $, των υποχώρων $ W_{1} $ και $ W_{2} $. 
\end{prop}

\begin{proof}
\item {}
    Καταρχάς δείχνουμε ότι $ W_{1} \cup W_{2} \subseteq W_{1}+W_{2} $. Πράγματι
    \begin{gather*}
        \mathbf{w} \in W_{1} \cup W_{2} \Rightarrow \mathbf{w} \in W_{1} \; \text{ή} \; 
        \mathbf{w} \in W_{2} 
    \end{gather*}
    Τότε αν
    \begin{gather*}
        \mathbf{w} \in W_{1} \Rightarrow \mathbf{w} + \mathbf{0} \in W_{1}+W_{2} 
        \tikzmark{a} \\
        \mathbf{w} \in W_{2} \Rightarrow \mathbf{0} + \mathbf{w} \in W_{1}+W_{2} 
        \tikzmark{b}
    \end{gather*}
    \mybrace{a}{b}[$ \mathbf{w} \in W_{1}+W_{2} $]
    Άρα $ W_{1} \cup W_{2} \leq W_{1}+W_{2} $.
    \begin{myitemize}
    \item 
        Έχουμε ότι $ W_{1}+W_{2} \leq V $, γιατί
        \begin{enumerate}[i)]
            \item $ \mathbf{0}+ \mathbf{0} = \mathbf{0} \in W_{1}+W_{2} $, άρα 
                $ W_{1}+W_{2} \neq \emptyset $
            \item Αν $ \mathbf{v_{1}}, \mathbf{v_{2}} \in W_{1}+W_{2} $, τότε 
                $ \mathbf{v_{1}} = \mathbf{w}_{1}+ \mathbf{w}_{2} $, με 
                $ \mathbf{w}_{1} \in W_{1} $ και $ \mathbf{w}_{2} \in W_{2} $ και 
                $ \mathbf{v_{2}} = \mathbf{w}_{1}' + \mathbf{w_{2}}' $, με 
                $ \mathbf{w_{1}}' \in W_{1} $ και $ \mathbf{w_{2}}' \in W_{2} $. Τότε
                \[
                    \mathbf{v_{1}}+ \mathbf{v_{2}} = \mathbf{w_{1}}+ \mathbf{w_{2}} + 
                    \mathbf{w_{1}}' + \mathbf{w_{2}}' = \underbrace{(\mathbf{w_{1}}+
                        \mathbf{w_{1}}')}_{\in W_{1}} + \underbrace{(\mathbf{w_{2}}+ 
                    \mathbf{w_{2}}')}_{\in W_{2}} \in W_{1}+W_{2}
                \] 
            \item Αν $ \mathbf{v} \in W_{1}+W_{2} $ και $ \lambda \in \mathbb{K} $, 
                τότε $ \mathbf{v} = \mathbf{w_{1}}+ \mathbf{w_{2}} $, με 
                $ \mathbf{w_{1}} \in W_{1} $ και $ \mathbf{w_{2}} \in W_{2} $.  Τότε
                \[
                    \lambda \mathbf{v} = \lambda (\mathbf{w_{1}}+ \mathbf{w_{2}}) = 
                    \underbrace{\lambda \mathbf{w_{1}}}_{\in W_{1}} + 
                    \underbrace{\lambda \mathbf{w_{2}}}_{\in W_{2}} \in W_{1}+W_{2}
                \] 
        \end{enumerate}

    \item Θα δείξουμε ότι $ W_{1}+W_{2} $ είναι ο μικρότερος υπόχωρος που περιέχει 
        την ένωση $ W_{1} \cup W_{2} $. Πράγματι
        έστω $ W \leq V $ με $ W_{1} \cup W_{2} \subseteq W $. Θα δείξουμε ότι 
        $ W_{1}+W_{2} \leq W $.

        Έχουμε $ \forall \mathbf{v} \in W_{1} + W_{2}, \; \exists \mathbf{w_{1}} 
        \in W_{1}  $ και $ \mathbf{w_{2}} \in W_{2} $ με $ \mathbf{v} = 
        \mathbf{w_{1}}+ \mathbf{w_{2}} $.  Τότε $ \mathbf{w_{1}} \in W_{1} \cup W_{2} 
        \subseteq W $
        και όμοια $ \mathbf{w_{2}} \in W_{1} \cup W_{2} \subseteq W $, άρα 
        $ \mathbf{w_{1}}+ \mathbf{w_{2}} \in W \Rightarrow \mathbf{v} \in W $. 
        Άρα $ W_{1}+W_{2} \leq W $ 
        και αυτό ισχύει για όλα τα $W$. 
    \end{myitemize}
\end{proof} 

\begin{examples}\label{ex:sums}
\item {}
    \begin{enumerate}
        \item \label{ex:r2} Έστω $ V = \mathbb{R}^{2} $ και $ W_{1} = 
            \{(x,y)\in \mathbb{R}^{2} \; : \; y =3x \} $ και 
            $ W_{2} = \{(x,y)\in \mathbb{R}^{2} \mid y=2x \} $, ευθείες που διέρχονται 
            από την αρχή των αξόνων, επομένως υπόχωροι του $ \mathbb{R}^{2}$. 

            Θα δείξουμε ότι $ W_{1}+W_{2} = V $. Πράγματι, αρκεί να δείξουμε ότι 
            κάθε $ (x,y) \in \mathbb{R}^{2} $ γράφεται ως $ w_{1}+w_{2} $ με 
            $ w_{1} \in W_{1} $ και $ w_{2} \in W_{2} $, δηλαδή ότι 
            \begin{align*}
                (x,y) &= (x_{1}, y_{1}) + (x_{2}, y_{2}), \quad (x_{1}, y_{1}) 
                \in W_{1} \; \text{και} \; (x_{2}, y_{2}) \in W_{2}
                \Leftrightarrow \\
                (x,y) &= (x_{1}, 3x_{1}) + (x_{2}, 2x_{2}), \quad x_{1}, x_{2} 
                \in \mathbb{R} \Leftrightarrow \\
                (x,y) &= (x_{1}+ x_{2}, 3x_{1}+ 2x_{2}), \quad x_{1}, x_{2} 
                \in \mathbb{R} 
            \end{align*}
            δηλαδή ότι το σύστημα 
            \[
            \sysdelim.\}\systeme{x_{1} + x_{2}=x,3x_{1} + 2x_{2}= y} 
        \] 
        έχει λύση, το οποίο πράγματι ισχύει, αφού η ορίζουσα του συστήματος είναι 
        \[
            \begin{vmatrix*}[r]
                1 & 1 \\
                3 & 2
            \end{vmatrix*} = -1 \neq 0.
        \]
        Άρα το σύστημα έχει μοναδική λύση, ως προς $ x_{1}, x_{2} \in \mathbb{R} $ 
        για κάθε $ x, y \in \mathbb{R} $.

    \item\label{ex:sym} Έστω $ V = M_{m \times n}(\mathbb{K}) $ και $ W_{1} = 
        \{ \text{συμμετρικοί πίνακες} \}, W_{2} = 
        \{ \text{αντι-συμμετρικοί πίνακες} \} $ υπόχωροι του $V$.

        Θα δείξουμε ότι $ W_{1}+W_{2}=V $. Αρκεί να δείξουμε ότι κάθε πίνακας 
        $ A \in V $ γράφεται ως $ B+C $, όπου $ B $ συμμετρικός και $ C $ 
        αντι-συμμετρικός. Πράγματι:
        \[
            A = \underbrace{\frac{1}{2} (A+A^{T})}_{B} + 
            \underbrace{\frac{1}{2} (A-A^{T})}_{C} 
        \] 
        όπου
        \begin{align*}
            B^{T} &= \left[\frac{1}{2} (A+A^{T})\right]^{T} = 
            \frac{1}{2} (A+A^{T})^{T} = \frac{1}{2} (A^{T}+A) = 
            \frac{1}{2} (A+A^{T}) = B \quad \text{άρα $B$ συμμετρικός} \\
            C^{T} &= \left[\frac{1}{2} (A-A^{T})\right]^{T} = 
            \frac{1}{2} (A-A^{T})^{T} = \frac{1}{2} (A^{T}-A) = - 
            \frac{1}{2} (A-A^{T}) = -C \quad \text{άρα $C$ άντι-συμμετρικός}
        \end{align*}

    \item\label{ex:art} Έστω $ V = \{ f \colon \mathbb{R} \to \mathbb{R} \} $ και 
        $ W_{1} = \{ \text{άρτιες συναρτήσεις} \}$, $ W_{2} = 
        \{ \text{περιττές συναρτήσεις} \} $, υπόχωροι του $V$.  

        Θα δείξουμε ότι $ W_{1}+W_{2}=V $. Αρκεί να δείξουμε ότι κάθε συνάρτηση
        $ f \in V $ γράφεται ως $ g+h $, όπου $ g $ άρτια και $ h $ περιττή. Πράγματι:
        \[
            f = \underbrace{\frac{1}{2} [f(x)+f(-x)]}_{g} + 
            \underbrace{\frac{1}{2} [f(x)-f(-x)]}_{h} 
        \] 
        όπου
        \begin{align*}
            g(-x) &= \frac{1}{2} [f(-x)+f(-(-x))] = 
            \frac{1}{2} [f(-x)+f(x)] = \frac{1}{2} [f(x)+f(-x)] = 
            g(x) \quad \text{άρα $g$ άρτια} \\
            h(-x) &= \frac{1}{2} [f(-x)-f(-(-x))] = 
            \frac{1}{2} [f(-x)-f(x)] = -\frac{1}{2} [f(x)-f(-x)] = 
            -h(x) \quad \text{άρα $h$ περιττή} \\
        \end{align*}
\end{enumerate}
\end{examples}

\begin{dfn}[Ευθύ Άθροισμα Υποχώρων]
\item {}
    Έστω $ W_{1}, W_{2} \leq V $. Λέμε ότι ο $V$ είναι \textcolor{Col2}{ευθύ άθροισμα}
    των $W_{1}$ και $ W_{2} $ και γράφουμε $ V = W_{1} \oplus W_{2} $ αν:
    \begin{enumerate}[i)]
        \item $ W_{1}+W_{2} = V $ και 
        \item $ W_{1} \cap W_{2} = \{ \mathbf{0} \} $
    \end{enumerate}
\end{dfn}

\begin{examples}\label{ex:directsums}
\item {}
  \begin{enumerate}
    \item Έστω $ V = \mathbb{R}^{2} $ και $ W_{1} = 
      \{(x,y)\in \mathbb{R}^{2} \; : \; y =0 \} $ και 
      $ W_{2} = \{(x,y)\in \mathbb{R}^{2} \mid x=0 \}$. 
      Τότε $ \mathbb{R}^{2} = W_{1} \oplus W_{2} $. Πράγματι
      \begin{gather*} 
        \text{Έστω} \; (x,y) \in \mathbb{R}^{2} 
        \Rightarrow (x,y) = \underbrace{(x,0)}_{\in W_{1}}+
        \underbrace{(0,y)}_{\in W_{2}} \in W_{1} + W_{2} 
      \end{gather*} 
      Θα δείξουεμ ότι η $ W_{1} \cap W_{2} = \{ \mathbf{0} \} $. Πράγματι, έστω 
      \[ (x,y) \in W_{1} \cap W_{2} \Rightarrow (x,y) 
        \in W_{1} \; \text{και} \; (x,y) \in W_{2} \Rightarrow y=0 \; 
        \text{και} \; x=0 \Rightarrow (x,y) = (0,0) 
      \Rightarrow W_{1} \cap W_{2} = \{ \mathbf{0} \} \]
    \item Από το παράδειγμα~\ref{ex:sums} -~\ref{ex:r2} έχουμε ότι 
      $ \mathbb{R}^{2} = W_{1}+W_{2} $ και επίσης, προφανώς ισχύει ότι 
      $ W_{1} \cap W_{2} = \{ \mathbf{0} \} $. 
      Επομένως $ \mathbb{R}^{2} = W_{1} \oplus W_{2} $.
    \item Έστω $ V = \mathbb{R}^{3} $ και $ W_{1} $ είναι η ευθεία $ \frac{x}{2} = 
      \frac{y}{3} = \frac{z}{5} = \lambda$ και $ W_{2} $ η ευθεία $ x = -y = 
      \frac{z}{2} = \mu $, υπόχωροι του $ V $. Έχουμε
      \begin{align*}
        W_{1}+W_{2} 
                &= \{ 
                  \mathbf{w_{1}}+ \mathbf{w_{2}} \; : \; 
                  \mathbf{w_{1}} \in W_{1}, \mathbf{w_{2}} \in W_{2} 
                \} \Leftrightarrow \\
                &= \{ 
                  (2 \lambda, 3 \lambda, 5 \lambda ) + 
                  (\mu, - \mu, 2 \mu) \; : \; \lambda, \mu \in 
                  \mathbb{R} 
                \} \Leftrightarrow   \\ 
                &= \{ 
                  (2 \lambda + \mu, 3 \lambda - \mu, 5 \lambda + 2 
                  \mu) \; : \; \lambda, \mu \in \mathbb{R} ) 
                \} 
      \end{align*}
      Γεωμετρικά, αυτό είναι επίπεδο, το επίπεδο που περιέχει τα σημεία 
      $ (0,0,0), (2,3,5) $ και $ (1,-1,2) $. Εναλλακτικά για να βρούμε την 
      εξίσωση του επιπέδου που έχει ως λύση το σύνολο 
      $ \{ (2 \lambda + \mu , 3 \lambda - \mu, 5 \lambda + 2 \mu) \; : \; 
      \lambda, \mu \in \mathbb{R} \} $. Ψάχνουμε σταθερές 
      $ a,b,c \in \mathbb{R} $ έτσι ώστε 
      \begin{align*}
        a(2 \lambda + \mu ) + b(3 \lambda - \mu )+ c (5 \lambda + 2 \mu ) =
        0, \quad \forall \lambda, \mu \in \mathbb{R} \\
        \lambda (2a +3b+5c) + \mu (a-b+2c) = 0, \quad \forall \lambda, \mu 
        \in \mathbb{R}
      \end{align*} 
      Θέτουμε $ \lambda =1, \mu = 0 $ και παίρνουμε $ 2a+3b+5c =0 $ και 
      $ \lambda = 0, \mu =1 $ και παίρνουμε $ a-b+2c=0 $. Οπότε έχουμε
      \[
        \left.
          \begin{matrix*}[r]
            2a+3b+5c=0 \\
            a-b+2c=0
          \end{matrix*}
        \right\} \Leftrightarrow 
        \left.
          \begin{matrix*}[r]
            2a+3b+5c=0 \\
            a=b-2c
          \end{matrix*}
        \right\} \Leftrightarrow
        \left.
          \begin{matrix*}[r]
            5b+c=0 \\
            a=b-2c
          \end{matrix*}
        \right\} \Leftrightarrow
        \left.
          \begin{matrix*}[r]
            c=-5b \\
            a=b-2c
          \end{matrix*}
        \right\} \Leftrightarrow
        \left.
          \begin{matrix*}[r]
            c=-5b \\
            a=11b
          \end{matrix*}
        \right\}
      \]

      Άρα για $ b=1, c=-5 $ και $ a=11 $ έχουμε το επίπεδο $ 11x+y-5z=0 $.

      Επομένως έχουμε ότι το επίπεδο $ V: 11x+y-5z=0 $ είναι το 
      ευθύ άθροισμα των ευθειών $ W_{1}: \frac{x}{2} = \frac{y}{3} = 
      \frac{z}{5}$ και $ W_{2}: x=-y= \frac{z}{2} $, καθώς 
      $ V = W_{1}+W_{2} $ και $ W_{1}\cap W_{2} = \{ \mathbf{0} \} $.

    \item Για το παράδειγμα\ref{ex:sums} -~\ref{ex:sym} έχουμε ότι 
      $ V = W_{1} \oplus W_{2} $, 
      γιατί δείξαμε ότι $ V= W_{1}+W_{2} $ και επίσης έχουμε 
      $ W_{1} \cap W_{2} = \{ \mathbf{0} \} $, 
      αφού αν $A \in W_{1} \cap W_{2}$, δηλαδή $ A $ συμμετρικός και 
      αντι-συμμετρικός ταυτόχρονα, μας δίνει ότι $ A^{T} = A $ και 
      $ A^{T}=-A $ και άρα $ 2A = \mathbf{0} \Rightarrow A= \mathbf{0} $.

    \item Για το παράδειγμα\ref{ex:sums} -~\ref{ex:art} έχουμε ότι 
      $ V = W_{1} \oplus W_{2} $, 
      γιατί δείξαμε ότι $ V= W_{1}+W_{2} $ και επίσης έχουμε 
      $ W_{1} \cap W_{2} = \{ \mathbf{0} \} $, 
      αφού αν $f \in W_{1} \cap W_{2}$, δηλαδή $ f $ άρτια και περιττή 
      ταυτόχρονα, μας δίνει ότι $ f(-x)=f(x)=-f(x), \; \forall x \in 
      \mathbb{R} $ και άρα $ 2f(x) = 0, \; \forall x \in 
      \mathbb{R} \Rightarrow f(x)=0, \; \forall x \in \mathbb{R} 
      \Rightarrow f = \mathbf{0} $.
  \end{enumerate} 
\end{examples}


\end{document}
