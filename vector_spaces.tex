\input{$HOME/Desktop/preamble/preamble.tex}
\newcommand{\vect}[2]{(#1_1,\ldots, #1_#2)}
%%%%%%% nesting newcommands $$$$$$$$$$$$$$$$$$$
\newcommand{\function}[1]{\newcommand{\nvec}[2]{#1(##1_1,\ldots, ##1_##2)}}

\newcommand{\linode}[2]{#1_n(x)#2^{(n)}+#1_{n-1}(x)#2^{(n-1)}+\cdots +#1_0(x)#2=g(x)}

\newcommand{\vecoffun}[3]{#1_0(#2),\ldots ,#1_#3(#2)}





\begin{document}


\chapter{Διανυσματικοί Χώροι}

\begin{dfn}
	Έστω $V \neq \emptyset$ σύνολο και $\mathbb{K}$ ένα σώμα (συνήθως θεωρούμε ότι $ \mathbb{K} =
	\mathbb{R} $ ή $\mathbb{C}$ ). Το σύνολο $V$ μαζί με τις πράξεις:
	\begin{alignat*}{2}
		+ \colon V \times V &\to V & \qquad \text{και} \qquad \cdot \colon V \times V &\to V \\
		( \vec{u}, \vec{v} ) &\mapsto \vec{u} + \vec{v} & ( \lambda, \vec{u} ) &\mapsto \lambda \vec{u} 
	\end{alignat*}
που ικανοποιούν τις ιδιότητες 

\begin{enumerate}
	\item $ \vec{u} + \vec{v} = \vec{v} + \vec{u} $ 
	\item $ ( \vec{u} + \vec{v} ) + \vec{w} = \vec{u} + ( \vec{v} + \vec{w}) $ 
	\item $ \vec{u} + \vec{0} = \vec{0} + \vec{u} = \vec{u} $ 
	\item $ \vec{u} + ( - \vec{u} ) = ( - \vec{u} ) + \vec{u} = \vec{0} $ 
	\item $ ( \lambda + \mu ) \vec{u} = \lambda \vec{u} + \mu \vec{u} $ 
	\item $ ( \lambda \mu ) \vec{u} = \lambda ( \mu) \vec{u} $ 
	\item $ \lambda ( \vec{u} + \vec{v} ) = \lambda \vec{u} + \lambda \vec{v} $ 
	\item $ 1 \vec{u} = \vec{u} $ 
\end{enumerate}
ονομάζεται \textcolor{blue}{διανυσματικός χώρος} πάνω στο σώμα $\mathbb{K}$, τα στοιχεία του οποίου
καλούμε διανύσματα.

\end{dfn}


\end{document}
