\input{preamble_ask.tex}
\input{definitions_ask.tex}

\everymath{\displaystyle}
\pagestyle{vangelis}

\begin{document}

% \begin{center}
%     \fbox{\large\bfseries  Τοπικά και Απόλυτα ακρότατα}
% \end{center}

\chapter*{Τοπικά και Ολικά Ακρότατα}

Έστω συνάρτηση $ f \colon A \subseteq \mathbb{R} \to \mathbb{R} $.

\begin{dfn}
  Έστω $ x_{0} \in A $.  Λέμε ότι η τιμή $ f(x_{0}) $ είναι:
  \begin{itemize}
    \item \textcolor{Col1}{τοπικό μέγιστο}\phantom{a} της συνάρτησης $f$, αν 
      $ f(x_{0}) \geq f(x) $, κοντά στο $ x_{0} $.
    \item \textcolor{Col1}{τοπικό ελάχιστο} της συνάρτησης $f$, αν 
      $ f(x_{0}) \leq f(x) $, κοντά στο $ x_{0} $.  
  \end{itemize}
\end{dfn}

Με την έννοια <<κοντά>> στο σημείο $ x_{0} $ εννοούμε σε κάποιο ανοιχτό διάστημα που 
περιέχει το $ x_{0} $.

\begin{dfn}
  Έστω $ x_{0} \in A $.  Λέμε ότι η τιμή $ f(x_{0}) $ είναι:
  \begin{itemize}
    \item \textcolor{Col1}{ολικό μέγιστο}\phantom{a} της συνάρτησης $f$ στο $ A $, αν 
      $ f(x_{0}) \geq f(x),\quad \forall x \in A $
    \item \textcolor{Col1}{ολικό ελάχιστο} της συνάρτησης $f$ στο $ A $, an 
      $ f(x_{0}) \leq f(x),\quad \forall x \in A $
  \end{itemize}
\end{dfn}

Ένα ολικό μέγιστο ή ελάχιστο πολλές φορές λέγεται και \textbf{απόλυτο} μέγιστο ή 
ελάχιστο, ενώ ένα τοπικό μέγιστο ή ελάχιστο λέγεται και \textbf{σχετικό} μέγιστο ή 
ελάχιστο.  Οι μέγιστες και ελάχιστες τιμές μιας συνάρτησης λέγονται \textbf{ακρότατες} 
τιμές.

\begin{rem}
  Ένα ολικό ακρότατο είναι προφανώς και τοπικό ακρότατο. Το αντίστροφο δεν ισχύει. 
  Επομένως ο υπολογισμός \textbf{όλων} των τοπικών ακροτάτων μιας συνάρτησης, 
  εξασφαλίζει και τον υπολογισμό των ολικών ακροτάτων της, εφόσον αυτά υπάρχουν.
\end{rem}

\begin{thm}[Fermat]
  Αν μια συνάρτηση $ f \colon A \to \mathbb{R}$ έχει τοπικό μέγιστο ή ελάχιστο στο 
  $ x_{0} $ το οποίο είναι \textbf{εσωτερικό} σημείου του $A$ και είναι παραγωγίσιμη στο 
  $ x_{0} $, τότε $ f'(x_{0}) = 0 $.
\end{thm}

\begin{rem}
\item {}
  \begin{myitemize}
    \item Το αντίστροφο του θεωρήματος Fermat δεν ισχύει. Για παράδειγμα για τη 
      συνάρτηση $ f(x)=x^{3} $, ισχύει ότι $ f'(0)=0 $, όμως στο σημείο $ x_{0} = 0 $ η 
      συνάρτηση δεν έχει κάποιο ακρότατο, αλλά αντίθετα παρουσιάζει σημείο καμπής. 
    \item Μια συνάρτηση είναι δυνατόν να παρουσιάζει τοπικό ακρότατο σ᾽ ένα σημείο 
      $ x_{0} $ χωρίς να είναι παραγωγίσιμη σε αυτό, όπως για παράδειγμα η συνάρτηση 
      $ f(x) = \abs{x} $ παρόλο που δεν ορίζεται η παράγωγός της 
      στο σημείο $ x_{0} = 0 $, έχει ελάχιστο στο σημείο αυτό. 
    \item Επίσης, αν το σημείο τοπικού ακροτάτου είναι 
      άκρο του διαστήματος ορισμού της συνάρτησης, τότε η παράγωγος μπορεί να μη 
      μηδενίζεται σε αυτό, όπως για παράδειγμα στη συνάρτηση $ f(x)=x^{2}+1 $ με 
      $ x \geq -1 $ η οποία παρουσιάζει τοπικό μέγιστο στο σημείο $ x_{0}= -1 $, ενώ 
      $ f'(-1)=-2 \neq 0 $.
  \end{myitemize}
\end{rem}

\begin{rem}
  Επομένως αυτό που μας λέει το θεώρημα Fermat είναι ότι τα μόνα σημεία τα οποία
  μπορεί να είναι ακρότατα (τοπικά ή ολικά) για μια συνάρτηση $f$ είναι 

  \begin{enumerate}
    \item Τα εσωτερικά σημεία του πεδίου ορισμού της στα οποία $ f'=0 $
    \item Τα εσωτερικά σημεία του πεδίου ορισμού της στα οποία η $ f' $ δεν ορίζεται
    \item Τα ακραία σημεία του πεδίου ορισμού της
  \end{enumerate}
\end{rem}

\begin{dfn}
  Έστω $ x_{0} $ εσωτερικό σημείο του $A$. Αν $ f'(x_{0}) = 0 $ ή $ f'(x_{0}) $ 
  δεν ορίζεται τότε το σημείο $ x_{0} $ λέγεται \textcolor{Col1}{κρίσιμο} σημείο της 
  συνάρτησης $f$. 
\end{dfn}

\begin{prop}
  Αν μια συνάρτηση $f$ έχει τοπικό ακρότατο στο σημείο $ x_{0} $, τότε αυτό είναι κρίσιμο
  σημείο της.
\end{prop}

\begin{thm}
  Αν μια συνάρτηση $f$ είναι συνεχής σε ένα κλειστό διάστημα $ [a,b] $, τότε η $f$
  παίρνει μέγιστη και ελάχιστη τιμή στο $ [a,b] $.
\end{thm}

\section*{Κριτήριο 1ης παραγώγου}
Έστω $ x_{0} $ κρίσιμο σημείο μιας συνεχούς συνάρτησης $f$, η οποία είναι 
παραγωγίσιμη σε κάθε σημείο οποιουδήποτε ανοιχτού διαστηματος περιέχει το $ x_{0} $, 
εκτός ίσως από το ίδιο το $ x_{0} $. Τότε
\begin{enumerate}
  \item Αν η $f'$ αλλάζει πρόσημο, από αρνητική σε θετική, εκατέρωθεν του $ x_{0} $, 
    τότε η $f$ έχει τοπικό \textbf{ελάχιστο} στο $ x_{0} $.
  \item Αν η $f'$ αλλάζει πρόσημο, από θετική σε αρνητική, εκατέρωθεν του $ x_{0} $, 
    τότε η $f$ έχει τοπικό \textbf{μέγιστο} στο $ x_{0} $.
\end{enumerate}

\section*{Κριτήριο 2ης παραγώγου}
Έστω ότι η συνάρτηση $ f'' $ είναι συνεχής σε κάποιο ανοιχτό διάστημα που περιέχει 
το σημείο $ x_{0} $.

\begin{enumerate}
  \item Αν $ f'(x_{0}) = 0 $ και $ f''(x_{0}) < 0 $, τότε η συνάρτηση $f$ έχει 
    τοπικό \textbf{μέγιστο} στο σημείο $ x_{0} $.
  \item Αν $ f'(x_{0}) = 0 $ και $ f''(x_{0}) > 0 $, τότε η συνάρτηση $f$ έχει 
    τοπικό \textbf{ελάχιστο} στο σημείο $ x_{0} $.
\end{enumerate}

\section*{Υπολογισμός τοπικών ακροτάτων}
\begin{enumerate}
  \item Υπολογίζουμε την 1η και 2η παράγωγο της συνάρτησης $ f(x) $.
  \item Υπολογίζουμε τα κρίσιμα σημεία της συνάρτησης.
  \item Για \textbf{κάθε} κρίσιμο σημείο, έστω $ x_{0} $ υπολογίζουμε την 
    $ f''(x_{0}) $ και εφαρμόζουμε το κριτήριο 2ης παραγώγου για το χαρακτηρισμό των 
    σημείων.
\end{enumerate}

\begin{rem}
  Αν το κριτήριο 2ης παραγώγου αποτύχει να μας δώσει απάντηση, είτε γιατί 
  $ f''(x_{0}) = 0 $, είτε γιατί η $ f''(x_{0}) $ δεν ορίζεται, τότε χρησιμοποιούμε 
  το κριτήριο 1ης παραγώγου.
\end{rem}

\section*{Ακρότατα σε κλειστό διάστημα}

Για να υπολογίσουμε τα ολικά μέγιστα και ελάχιστα μιας συνεχούς συνάρτησης $f$ σε ένα 
κλειστό διάστημα $ [a,b] $
\begin{enumerate}
  \item Υπολογίζουμε τις τιμές της συνάρτησης στα κρίσιμα σημεία της στο εσωτερικό του 
    $ [a,b] $
  \item Υπολογίζουμε τις τιμές της συνάρτησης στα άκρα του διαστήματος $ [a,b] $
  \item Η μεγαλύτερη από όλες αυτές τις τιμές είναι το ολικό μέγιστο και η μικρότερη 
    είναι το ολικό ελάχιστο της συνάρτησης $f$.
\end{enumerate}


\end{document}
