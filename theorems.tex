\documentclass[a4paper,12pt]{article}
\usepackage{etex}
%%%%%%%%%%%%%%%%%%%%%%%%%%%%%%%%%%%%%%
% Babel language package
\usepackage[english,greek]{babel}
% Inputenc font encoding
\usepackage[utf8]{inputenc}
%%%%%%%%%%%%%%%%%%%%%%%%%%%%%%%%%%%%%%

%%%%% math packages %%%%%%%%%%%%%%%%%%
\usepackage{amsmath}
\usepackage{amssymb}
\usepackage{amsfonts}
\usepackage{amsthm}
\usepackage{proof}

\usepackage{physics}

%%%%%%% symbols packages %%%%%%%%%%%%%%
\usepackage{dsfont}
\usepackage{stmaryrd}
%%%%%%%%%%%%%%%%%%%%%%%%%%%%%%%%%%%%%%%


%%%%%% graphicx %%%%%%%%%%%%%%%%%%%%%%%
\usepackage{graphicx}
\usepackage{color}
%\usepackage{xypic}
\usepackage[all]{xy}
\usepackage{calc}
%%%%%%%%%%%%%%%%%%%%%%%%%%%%%%%%%%%%%%%

\usepackage{enumerate}

\usepackage{fancyhdr}
%%%%% header and footer rule %%%%%%%%%
\setlength{\headheight}{14pt}
\renewcommand{\headrulewidth}{0pt}
\renewcommand{\footrulewidth}{0pt}
\fancypagestyle{plain}{\fancyhf{}
\fancyhead{}
\lfoot{}
\rfoot{\small \thepage}}
\fancypagestyle{vangelis}{\fancyhf{}
\rhead{\small \leftmark}
\lhead{\small }
\lfoot{}
\rfoot{\small \thepage}}
%%%%%%%%%%%%%%%%%%%%%%%%%%%%%%%%%%%%%%%

\usepackage{hyperref}
\usepackage{url}
%%%%%%% hyperref settings %%%%%%%%%%%%
\hypersetup{pdfpagemode=UseOutlines,hidelinks,
bookmarksopen=true,
pdfdisplaydoctitle=true,
pdfstartview=Fit,
unicode=true,
pdfpagelayout=OneColumn,
}
%%%%%%%%%%%%%%%%%%%%%%%%%%%%%%%%%%%%%%



\usepackage{geometry}
\geometry{left=25.63mm,right=25.63mm,top=36.25mm,bottom=36.25mm,footskip=24.16mm,headsep=24.16mm}

%\usepackage[explicit]{titlesec}
%%%%%% titlesec settings %%%%%%%%%%%%%
%\titleformat{\chapter}[block]{\LARGE\sc\bfseries}{\thechapter.}{1ex}{#1}
%\titlespacing*{\chapter}{0cm}{0cm}{36pt}[0ex]
%\titleformat{\section}[block]{\Large\bfseries}{\thesection.}{1ex}{#1}
%\titlespacing*{\section}{0cm}{34.56pt}{17.28pt}[0ex]
%\titleformat{\subsection}[block]{\large\bfseries{\thesubsection.}{1ex}{#1}
%\titlespacing*{\subsection}{0pt}{28.80pt}{14.40pt}[0ex]
%%%%%%%%%%%%%%%%%%%%%%%%%%%%%%%%%%%%%%

%%%%%%%%% My Theorems %%%%%%%%%%%%%%%%%%
\newtheorem{thm}{Θεώρημα}[section]
\newtheorem{cor}[thm]{Πόρισμα}
\newtheorem{lem}[thm]{λήμμα}
\theoremstyle{definition}
\newtheorem{dfn}{Ορισμός}[section]
\newtheorem{dfns}[dfn]{Ορισμοί}
\theoremstyle{remark}
\newtheorem{remark}{Παρατήρηση}[section]
\newtheorem{remarks}[remark]{Παρατηρήσεις}
%%%%%%%%%%%%%%%%%%%%%%%%%%%%%%%%%%%%%%%




\newcommand{\vect}[2]{(#1_1,\ldots, #1_#2)}
%%%%%%% nesting newcommands $$$$$$$$$$$$$$$$$$$
\newcommand{\function}[1]{\newcommand{\nvec}[2]{#1(##1_1,\ldots, ##1_##2)}}

\newcommand{\linode}[2]{#1_n(x)#2^{(n)}+#1_{n-1}(x)#2^{(n-1)}+\cdots +#1_0(x)#2=g(x)}

\newcommand{\vecoffun}[3]{#1_0(#2),\ldots ,#1_#3(#2)}


\input{tikz.tex}
% \usetikzlibrary{decorations.pathreplacing,}
% \newcommand{\tikzmark}[1]{\tikz[baseline={(#1.base)},overlay,remember picture]
% \node[outer sep=0pt, inner sep=0pt] (#1) {\phantom{A}};}

\everymath{\displaystyle}
\thispagestyle{empty}

\begin{document}

\chapter*{Θεωρήματα Διαφορικού Λογισμού}


\vspace{\baselineskip}

\begin{thmbreak}[\bfseries Bolzano]
\begin{minipage}[t]{3.6cm}
	\begin{enumerate}[i)]
	\item $f$ συνεχής στο $ [a,b] $ \hfill \tikzmark{a} 
	\item $ f(a)\cdot f(b) < 0 $ \hfill \tikzmark{b}
\end{enumerate}
\end{minipage}
\end{thmbreak}

\begin{tikzpicture}[overlay, remember picture,decoration={brace,amplitude=4pt}]
	\draw[decorate,thick] (a.north east) -- (b.south east) node[midway,
	right=0.1cm] {$\Rightarrow$}node[midway, right=0.5cm,text=black,text width = 2in,]
	{$\quad \exists x_{0} \in (a,b) \; : \; f(x_{0}) = 0$};
\end{tikzpicture}


\begin{thmbreak}[\bfseries Rolle ]
	\begin{minipage}[t]{4.6cm}
		\begin{enumerate}[i)]
			\item $f$ συνεχής στο $ [a,b] $ \hfill \tikzmark{a}
			\item $f$ παραγωγίσιμη στο $ (a,b) $ 
			\item $ f(a) = f(b) $ \hfill  \tikzmark{b}
		\end{enumerate}
	\end{minipage}
\end{thmbreak}

\begin{tikzpicture}[overlay, remember picture,decoration={brace,amplitude=4pt}]
	\draw[decorate,thick] (a.north east) -- (b.south east) node[midway,
	right=0.1cm] {$\Rightarrow$}node[midway, right=0.5cm,text=black,text width = 2in,]
	{$\quad \exists x_{0} \in (a,b) \; : \; f'(x_{0}) = 0$};
\end{tikzpicture}

\begin{rem}
  Η Γεωμετρική ερμηνεία του θεωρήματος Rolle είναι, ότι εφόσον μια συνάρτηση 
  $ y=f(x) $ ικανοποιεί τις προυποθέσεις του θεωρήματος, τότε υπάρχει τουλάχιστον ένα 
  σημείο $ x_{0} $ στο διάστημα $ (a,b) $ στο οποίο η εφαπτομένη της γραφικής παράστασης 
  της συνάρτησης, να είναι \textbf{παράλληλη} προς τον άξονα $x$.
\end{rem}


\begin{thmbreak}[{\bfseries  Μέσης Τιμής}] 
	\begin{minipage}[t]{4.6cm}
		\begin{enumerate}[i)]
			\item $f$ συνεχής στο $ [a,b] $ \hfill \tikzmark{a}
			\item $f$ παραγωγίσιμη στο $ (a,b) $ \hfill \tikzmark{b}
		\end{enumerate}
	\end{minipage}
\end{thmbreak}

\begin{tikzpicture}[overlay, remember picture,decoration={brace,amplitude=4pt}]
	\draw[decorate,thick] (a.north east) -- (b.south east) node[midway,
	right=0.1cm] {$\Rightarrow$}node[midway, right=0.5cm,text=black,text width = 3in,]
	{$\quad \exists \xi \in (a,b) \; : \; f'(\xi) = \frac{f(b) - f(a)}{b - a}$};
\end{tikzpicture}

\begin{rem}
  Η Γεωμετρική ερμηνεία του θεωρήματος Μέσης Τιμής είναι, ότι εφόσον μια συνάρτηση 
  $ y=f(x) $ ικανοποιεί τις προυποθέσεις του θεωρήματος, τότε υπάρχει τουλάχιστον ένα 
  σημείο $\xi$ στο διάστημα $ (a,b) $ στο οποίο η εφαπτομένη της γραφικής παράστασης 
  της συνάρτησης, να είναι \textbf{παράλληλη} προς τη χορδή $AB$ όπου $ A(a,f(a)) $ 
  και $ B(b,f(b)) $.
\end{rem}


\begin{thmbreak}[{\bfseries Θεώρημα Μέσης Τιμής Cauchy}]
	\begin{minipage}[t]{5cm}
	\begin{enumerate}[i)]
		\item $f, g$ συνεχείς στο $ [a,b] $ \hfill \tikzmark{a}
		\item $f, g$ παραγωγίσιμες στο $ (a,b) $
		\item $ g'(x) \neq 0 $, $ \forall x \in (a,b) $ \hfill \tikzmark{b}
	\end{enumerate}	
	\end{minipage}
\end{thmbreak}

\begin{tikzpicture}[overlay, remember picture,decoration={brace,amplitude=4pt}]
	\draw[decorate,thick] (a.north east) -- (b.south east) node[midway,
	right=0.1cm] {$\Rightarrow$}node[midway, right=0.5cm,text=black,text width = 3in,]
	{$\quad \exists \xi \in (a,b) \; : \; \frac{f'(\xi)}{g'(\xi)} =
	\frac{f(b) - f(a)}{b - a}$};
	\end{tikzpicture}


	\begin{thmbreak}[\bfseries Fermat]
      Έστω $ f \colon \Delta \to \mathbb{R} $ και $ x_{0} $ \textbf{εσωτερικό} σημείο 
      του $\Delta$. 

		\begin{minipage}[t]{0.30\textwidth}
			\begin{enumerate}[i)]
				\item $f(x_{0})$ τοπικό ακότατο \hfill \tikzmark{a}
        \item η $ f $ παραγωγίσιμη στο $ x_{0} \in \Delta $  \hfill \tikzmark{b}
			\end{enumerate}
		\end{minipage}
	\end{thmbreak}
	
\begin{tikzpicture}[overlay, remember picture,decoration={brace,amplitude=4pt}]
	\draw[decorate,thick] (a.north east) -- (b.south east) node[midway,
	right=0.1cm] {$\Rightarrow$}node[midway, right=0.5cm,text=black,text width = 3in,]
	{$\quad  f'(x_{0}) = 0 $};
\end{tikzpicture}

\end{document}
