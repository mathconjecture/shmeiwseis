\input{preamble_ask.tex}
\input{definitions_ask.tex}

\geometry{left=15.63mm,right=15.63mm,top=30.25mm, bottom=36.25mm,footskip=24.16mm,
  headsep=24.16mm}

\pagestyle{vangelis}
\everymath{\displaystyle}


\begin{document}

\chapter*{Διανυσματικές Συναρτήσεις}

\begin{dfn} Η διανυσματική συνάρτηση
  $ \mathbf{r} \colon \mathbb{R} \to \mathbb{R}^{3} $ με 
  $ \mathbf{r}(t) = x(t) \, \mathbf{i}+y(t) \, \mathbf{j}+z(t) \, \mathbf{k} $, 
  παριστάνει μια \textbf{καμπύλη} στον τρισδιάστατο χώρο $ \mathbb{R}^{3} $. 
  Οι πραγματικές συναρτήσεις $ x(t) $, $ y(t) $ και $ z(t) $ ονομάζονται
  \textcolor{Col1}{συναρτήσεις συντεταγμένων} της διανυσματικής συνάρτησης και αποτελούν 
  τις \textbf{παραμετρικές εξισώσεις} της καμπύλης.
  %todo σχημα καμπύλης στο χωρο
\end{dfn}

\begin{examples}
\item {}
  \begin{enumerate}
    %todo σχήμα για έλικα 
    \item Η έλικα $ \mathbf{r}(t)= a \cos{\omega t}\, \mathbf{i} + a \sin{\omega t}\,
      \mathbf{j} + bt \, \mathbf{k} $ 
    \item Η τυχαία καμπύλη στο χώρο 
      $ \mathbf{r}(t)=t^{2}\, \mathbf{i} + (2t+5)\, \mathbf{j} + 2t^{3} \, \mathbf{k}$ 
    \item Η ευθεία στο χώρο $ \mathbf{r}(t)=(x_{0}+at)\, \mathbf{i} + (y_{0}+bt)\,
      \mathbf{j} + (z_{0}+ct) \, \mathbf{k}$, που διέρχεται από το σημείο 
      $ P(x_{0}, y_{0}, z_{0}) $ και είναι παράλληλη στο διάνυσμα 
      $ \mathbf{u} = (a,b,c) $.
  \end{enumerate}
\end{examples}




\end{document}
