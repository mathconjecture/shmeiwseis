\input{preamble_ask.tex}
\input{definitions_ask.tex}
\input{tikz.tex}

\geometry{left=15.63mm,right=15.63mm,top=30.25mm, bottom=36.25mm,footskip=24.16mm,
  headsep=24.16mm}

\pagestyle{vangelis}
\everymath{\displaystyle}

\newcommand{\twocolumnsidescc}[2]{\begin{minipage}[c]{0.40\linewidth}\raggedright
        #1
        \end{minipage}\hfill\begin{minipage}[c]{0.50\linewidth}\raggedright
        #2
    \end{minipage}
}

\begin{document}

\chapter*{Διανυσματικές Συναρτήσεις}

\begin{dfn} Η διανυσματική συνάρτηση
  $ \mathbf{r} \colon \mathbb{R} \to \mathbb{R}^{3} $ με 
  $ \mathbf{r}(t) = x(t) \, \mathbf{i}+y(t) \, \mathbf{j}+z(t) \, \mathbf{k} $, 
  παριστάνει μια \textbf{καμπύλη} στον τρισδιάστατο χώρο $ \mathbb{R}^{3} $. 
  Οι πραγματικές συναρτήσεις $ x(t) $, $ y(t) $ και $ z(t) $ ονομάζονται
  \textcolor{Col1}{συναρτήσεις συντεταγμένων} της διανυσματικής συνάρτησης και αποτελούν 
  τις \textbf{παραμετρικές εξισώσεις} της καμπύλης.
  %todo σχημα καμπύλης στο χωρο
\end{dfn}

\begin{example}
  Η \textbf{κυκλική έλικα} $ \mathbf{r}(t)= a \cos{(\omega t)}\, \mathbf{i} + a 
  \sin{(\omega t)}\, \mathbf{j} + bt \, \mathbf{k} $, με $ a,b,\omega \in \mathbb{R} $. 
      \begin{center}
      \begin{tikzpicture}
        \draw[axis,<->,>=stealth] (0,2.6) node[left] {$z$} -- (0,0) -- (-1.5,-1.5) 
          node[left] {$x$} ;
        \draw[axis,-stealth] (0,0) -- (2.6,0) node[below] {$y$} ;
        \draw [graph,domain=0:720,smooth,variable=\t,samples=200] 
          plot ({sin(\t)},\t/360,{cos(\t)}) ;
      \end{tikzpicture}
      \end{center}
  \end{example}
  \begin{rem*}
    Ο αριθμός $a$ είναι η ακτίνα της έλικας, ενώ ο $ 2 \pi b $ παριστανεί την σταθερή, 
    κάθετη απόσταση ανάμεσα σε 2 σπείρες.
  \end{rem*}
  \begin{example}
    Η ευθεία στο χώρο, που διέρχεται από το σημείο $ P_0(x_{0}, y_{0}, z_{0}) $ 
    και είναι παράλληλη στο διάνυσμα $ \mathbf{u} = (a,b,c) $, δίνεται από την 
    εξίσωση. 
    $ \mathbf{r}(t)=(x_{0}+at)\, \mathbf{i} + (y_{0}+bt)\, \mathbf{j} + (z_{0}+ct) 
    \, \mathbf{k}$

    \vspace{\baselineskip}
    \twocolumnsidescc{
      \begin{tikzpicture}[scale=0.7]
        \node  (0) at (0, 3) {};
        \node  (1) at (0, 0) {};
        \node  (2) at (-2, -1) {};
        \node  (3) at (3, 0) {};
        \node  (5) at (-1, 2) {};
        \node  (6) at (4, 3) {};
        \node[above] (e) at (6) {$\varepsilon$};
        \node  (7) at ($(5)!0.40!(6)$) {};
        \node  (8) at ($(5)!0.8!(6)$) {};
        \node  (u) at (2.375, 0.625) {};
        \node[above,yshift=-2pt,xshift=-5pt,Col2] (a) at (u) {$\mathbf{u}$};
        \draw[-latex,very thin] (1.center) node[left,yshift=3pt]{$O$} to (2.center) 
          node[left]{$x$};
        \draw[-latex,ultra thick,Col2] (1.center) to (u);
        \draw[-latex,very thin] (1.center) to (3.center) node[below]{$y$};
        \draw[-latex,very thin] (1.center) to (0.center) node[left]{$z$};
        \draw[very thick,Col1] (5.center) to (6.center);
        \draw[-latex] (1.center) to node[above left=-3pt]{$\mathbf{r_{0}}$} (7.center) 
          node[pin={[pin edge={black,latex-},Col2]93:\smaller$t=0$}]{}
          node[above,xshift=5pt]{\smaller$P_{0}$};
        \draw[-latex] (1.center) to node[right=5pt]{$\mathbf{r}$} (8.center) ;
        \draw[-latex,ultra thick,Col2] (7.center) to (8.center) node[above left] 
          {$t\mathbf{u}$};
        \node[above right] at (8) {\smaller$P$};
        \draw[fill] (7.center) circle (1.5pt); 
        \draw[fill] (8.center) circle (1.5pt); 
      \end{tikzpicture}
      }{
      Πράγματι, παρατηρούμε ότι $ \mathbf{r(t)} = \mathbf{r_0} + t \mathbf{u} $, δηλαδή:
      \[
        \mathbf{r(t)} = (x_{0}, y_{0}, z_{0}) + t(a,b,c) = (x_{0}+ta, y_{0}+tb, z_{0}+tc)
      \]  
    }

  \end{example}




  \end{document}
