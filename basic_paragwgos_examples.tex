\documentclass[a4paper,table]{report}
\input{preamble2.tex}
\input{definitions2.tex}
\input{tikz.tex}
\input{myboxes.tex}

\usepackage[font={color=Col1},labelfont={bf},hypcap=false]{caption}

\everymath{\displaystyle}
\pagestyle{vangelis}

\setcounter{chapter}{1}

%adds a rectangle background box behind every picture
 % \tikzset{every picture/.append style={background rectangle/.style={fill=blue!05,rounded
 %   corners},show background rectangle}}

%paragwgos
\newcommand{\twocolumnsidesr}[2]{\begin{minipage}[t]{0.36\linewidth}
        #1
        \end{minipage}\hfill\begin{minipage}[t]{0.60\linewidth}
        #2
    \end{minipage}
}

\newcommand{\twocolumnsiderrp}[2]{\begin{minipage}[c]{0.30\linewidth}
    #1
    \end{minipage}\hfill\begin{minipage}[c]{0.68\linewidth}
    #2
  \end{minipage}
}

\newcommand{\twocolumnsidell}[2]{\begin{minipage}[c]{0.68\linewidth}
    #1
    \end{minipage}\hfill\begin{minipage}[c]{0.30\linewidth}
    #2
  \end{minipage}
}

\newcommand{\twocolumnsiderrr}[2]{\begin{minipage}[c]{0.80\linewidth}
    #1
    \end{minipage}\hfill\begin{minipage}[c]{0.15\linewidth}
    #2
  \end{minipage}
}

\newcommand{\threecolumnside}[3]{\begin{minipage}[t]{0.30\linewidth}\raggedright
        #1
        \end{minipage}\hfill\begin{minipage}[t]{0.30\linewidth}\raggedright
        #2
        \end{minipage}\hfill\begin{minipage}[t]{0.30\linewidth}\raggedright
        #3
    \end{minipage}
}

\newcommand{\twocolumnsidesrr}[2]{\begin{minipage}[t]{0.35\linewidth}
    #1
    \end{minipage}\hfill\begin{minipage}[t]{0.60\linewidth}
    #2
  \end{minipage}
}


\newcommand{\twocolumnsiderr}[2]{\begin{minipage}[c]{0.30\linewidth}
    #1
    \end{minipage}\hfill\begin{minipage}[c]{0.68\linewidth}
    #2
  \end{minipage}
}

\newcommand{\twocolumnsidesss}[2]{\begin{minipage}[c]{0.48\linewidth}\raggedright
    #1
    \end{minipage}\hfill\begin{minipage}[c]{0.48\linewidth}\raggedright
    #2
  \end{minipage}
}




\geometry{top=2.0cm,left=1.5cm,right=1.5cm}

\input{insbox}

\begin{document}


\section{Παράγωγοι Βασικών Συναρτήσεων}

Με τη βοήθεια του ορισμού της παραγώγου, αποδεικνύονται οι τύποι για τις παραγώγους των βασικών συναρτήσεων, στον πίνακα που ακολουθεί.

\begin{center}
\begin{Mytable}
  \renewcommand{\arraystretch}{2.0}
  \begin{tabular}{|c||c|}
    \TabCellHead Βασικές Συναρτήσεις & \TabCellHead Τριγωνομετρικές Συναρτήσεις \\[4pt] \hline
    $ (c)' = 0, \quad c \in \mathbb{R} $ & $ (\sin{x})' = \cos{x} $ \\[4pt] \hline
    $ (x)' = 1 $ & $ (\cos{x})' = - \sin{x} $ \\[4pt] \hline 
    $ (x^{a})' = a x^{a-1} $ & $ (\tan{x})' = \frac{1}{\cos^{2}{x}}$ \\[4pt] \hline
    $ \Bigl(\frac{1}{x}\Bigr)' = - \frac{1}{x^{2}} $ & $ (\cot{x})' = - \frac{1}{\sin^{2}{x}}  $ \\[4pt] \hline
    $ \Bigl(\frac{1}{x^{2}}\Bigr)' = - \frac{2}{x^{3}} $ & $ (\arctan{x})' = \frac{1}{1 + x^{2}} $ \\[4pt] \hline
    $ (a^{x})' = a^{x}\cdot \ln{a} $ & $ (\arccot{x})' = \frac{-1}{1 + x^{2}} $ \\[4pt] \hline
    $ (e^{x})' = e^{x} $ &  $ (\arcsin{x})' = \frac{1}{\sqrt{1 - x^{2}}} $ \\[4pt] \hline
    $ (\ln{x})' = \frac{1}{x} $ & $ (\arccos{x})' = \frac{-1}{\sqrt{1 - x^{2}}} $ \\[4pt] \hline
    $ (\sqrt[n]{x^{m}})' = (x^{\frac{m}{n}})'= \frac{m}{n} x^{\frac{m}{n} -1} $ & $ (\sinh{x})' = \cosh{x} $ \\[4pt] \hline
    $ (\sqrt{x})' = \frac{1}{2 \sqrt{x}} $ & $ (\cosh{x})' = \sinh{x} $ \\[4pt] \hline
  \end{tabular}
\end{Mytable}
\end{center}

\subsection{Παραδείγματα}

\twocolumnsidesr{
  \begin{example}
    $ (3)'= 0 $
  \end{example}
  \begin{example}
    $ (x^{3})' = 3x^{2} $
  \end{example}
  \begin{example}
    $ (3^{x})' = 3^{x} \ln{3} $
  \end{example}
  \begin{example}
    $ \Bigl(\frac{1}{x^{4}}\Bigr)'\!\! =\! (x^{-4})'\!\! =\! -4x^{-5} \!\!\! $
  \end{example}
}{
  \begin{example}
    $ (x^{2/3})' = \frac{2}{3}x^{\frac{2}{3} - 1} = \frac{2}{3} x^{- \frac{1}{3}} = 
    \frac{2}{3} \frac{1}{x^{\frac{1}{3}}} = \frac{2}{3 \sqrt[3]{x}}$
  \end{example}
  \begin{example}
    $ ( \sqrt[3]{x} )' = (x^{\frac{1}{3} })' = \frac{1}{3} x^{\frac{1}{3} - 1} =
    \frac{1}{3} x^{-\frac{2}{3} } = \frac{1}{3 x^{\frac{2}{3}}} = \frac{1}{3
    \sqrt[3]{x^{2}}} $
  \end{example}
  \begin{example}
    $ \bigl(\sqrt[4]{x^{3}}\bigr)' \!= \bigl(x^{\frac{3}{4}}\bigr)' 
    \!= \frac{3}{4} x^{\frac{3}{4} -1} = \frac{3}{4} x^{-\frac{1}{4}} = \frac{3}{4}
    \frac{1}{x^{\frac{1}{4}}} = \frac{3}{4 \sqrt[4]{x}} $
  \end{example}
}



\section{Κανόνες Παραγώγισης}

Οι προτάσεις του, επόμενου πίνακα, γνωστές ως \textcolor{Col1}{κανόνες παραγώγισης}, 
σε συνδυασμό με τους τύπους του προηγούμενου πίνακα, μας δίνουν τη δυνατότητα να
υπολογίζουμε τις παραγώγους, πιο πολύπλοκων συναρτήσεων.

\begin{center}
  \begin{Mytable}
    \renewcommand{\arraystretch}{2.0}
    \begin{tabular}{|c|c|}
      \TabCellHead Αθροίσματος & $ (f(x)+g(x))' = f'(x)+ g'(x) $ \\[4pt] \hline 
      \TabCellHead Σταθεράς & $ (a f(x))' = a f'(x)$ \\[4pt] \hline
      \TabCellHead Γινομένου & $ (f(x)\cdot g(x))' = f'(x)\cdot g(x) 
      + f(x) g'(x) $ \\[4pt] \hline
      \TabCellHead Πηλίκου & $ \Bigl(\frac{f(x)}{g(x)}\Bigr)' = \frac{f'(x)\cdot
      g(x) - f(x) g'(x)}{g^{2}(x)} $ \\[4pt] \hline
      \end{tabular}
    \end{Mytable}
  \end{center}


  \section{Παραδείγματα}

  \subsection{Κανόνας Αθροίσματος}
  \begin{example}
    $( \cos{x} + \sqrt{x})' = (\cos{x} )' + (\sqrt{x} )' = - \sin{x} + \frac{1}{2
    \sqrt{x}} $
  \end{example}
  \begin{example}
    $ \Bigl(\frac{1}{x} - \tan{x} \Bigr)' = \Bigl(\frac{1}{x}\Bigr)' - (\tan{x} )' 
    = - \frac{1}{x^{2}} - \frac{1}{\cos^{2}{x}} $
  \end{example}


  \subsection{Κανόνας Σταθεράς}
  \begin{example}
    $( 3 \ln{x})' = 3 \cdot (\ln{x} )' = 3 \cdot \frac{1}{x} = \frac{3}{x} $
  \end{example}
  \begin{example}
    $\Bigl(\frac{2}{x}\Bigr)' = 2 \cdot \Bigl(\frac{1}{x}\Bigr)' = 2 \cdot
    \Bigl(-\frac{1}{x^{2}}\Bigr) = -\frac{2}{x^{2}} $
  \end{example}

  \subsection{Κανόνας Γινομένου}
  \begin{example}
    $ (x^{2}\cdot \ln{x})' = (x^{2})' \cdot \ln{x} + x^{2}\cdot (\ln{x})' = 2x \cdot
    \ln{x} + x^{2} \cdot \frac{1}{x} = 2x \ln{x} + x = x(2 \ln{x} +1) $
  \end{example}
  \begin{example}
    $ (\mathrm{e}^{x} \cdot \sin{x})' = (\mathrm{e}^{x} )'\cdot \sin{x} +
    \mathrm{e}^{x} \cdot (\sin{x} )' = \mathrm{e}^{x} \cdot \sin{x} + \mathrm{e}^{x}
    \cos{x} = \mathrm{e}^{x} \cdot (\sin{x} + \cos{x} ) $
  \end{example}

  \subsection{Κανόνας Πηλίκου}
  \begin{example}
    $ \Bigl(\frac{x^{2}}{\mathrm{e}^{x}}\Bigr)' = \frac{(x^{2})' \cdot \mathrm{e}^{x} 
    - x^{2}\cdot (\mathrm{e}^{x} )'}{(\mathrm{e}^{x})^{2}} 
    = \frac{2x \cdot \mathrm{e}^{x} - x^{2}\cdot \mathrm{e}^{x}}{(\mathrm{e}^{x}) ^{2}} 
    = \frac{\mathrm{e}^{x} (2x-x^{2})}{(\mathrm{e}^{x})^2} = 
    \frac{2x - x^{2}}{\mathrm{e}^{x}} $
  \end{example}
  \begin{example}
    $\Bigl( \frac{\ln{x}}{x}\Bigr)' = \frac{(\ln{x} )' \cdot x- \ln{x} \cdot (x)'}{x^{2}} 
    = \frac{\frac{1}{x} \cdot x - \ln{x} \cdot 1}{x^{2}} = \frac{1- \ln{x}}{x^{2}} $
  \end{example}


  \section{Ασκήσεις Παραγώγισης}

  \begin{exercise}  Να υπολογιστεί η παράγωγος της συνάρτησης 
    $f(x) = \sqrt{x} \cdot \ln{x} $.
    \begin{align*}
      (\sqrt{x} \cdot \ln{x} )' 
        &= (\sqrt{x} )'\cdot \ln{x} + \sqrt{x} \cdot (\ln{x}
        )' = \frac{1}{2 \sqrt{x} } \cdot \ln{x} + \sqrt{x} \cdot \frac{1}{x} =
        \frac{\ln{x}}{2 \sqrt{x}} + \frac{\sqrt{x}}{x} = 
        \frac{x \ln{x}+ 2 (\sqrt{x})^{2}}{2 x \sqrt{x} } \\
        &= \frac{x (\ln{x} + 2)}{2x \sqrt{x}} = \frac{\ln{x} +2}{2 \sqrt{x}}
    \end{align*}
  \end{exercise}

  \begin{exercise}  Να υπολογιστεί η παράγωγος της συνάρτησης 
    $ f(x) = \sqrt[3]{x^{2}} \cdot \tan{x} $
    \begin{align*}
      (\sqrt[3]{x^{2}} \cdot \tan{x} )' 
  &= (\sqrt[3]{x^{2}} )' \cdot \tan{x} +
  \sqrt[3]{x^{2}} \cdot (\tan{x} )' = (x^{\frac{2}{3}})' \cdot \tan{x} +
  \sqrt[3]{x^{2}} \cdot \frac{1}{\cos^{2}{x}} = \frac{2}{3} \cdot x^{\frac{2}{3}
  -1} \cdot \tan{x} + \frac{\sqrt[3]{x^{2}}}{\cos^{2}{x}} \\
  &= \frac{2}{3} \cdot x^{- \frac{1}{3} } \cdot \tan{x} + 
  \frac{\sqrt[3]{x^{2}}}{\cos^{2}{x}} = \frac{2 \tan{x}}{3 \sqrt[3]{x}} +
  \frac{\sqrt[3]{x^{2}}}{\cos^{2}{x}} 
    \end{align*}
  \end{exercise}

  \begin{exercise} Να υπολογιστεί η παράγωγος της συνάρτησης 
    $ f(x) = 2^{x} \cdot \cot{x} $
    \begin{align*}
      (2^{x}\cdot \cot{x})' 
      &= (2^{x})' \cdot \cot{x} + 2^{x} \cdot (\cot{x} )' =
      2^{x}\cdot \ln{2} \cdot \cot{x} + 2^{x}\cdot \Bigl(- \frac{1}{\sin^{2}{x}}\Bigr) = 
      2^{x} \ln{2} \cdot \cot{x} - \frac{2^{x}}{\sin^{2}{x}} 
    \end{align*}
  \end{exercise}

  \begin{exercise} Να υπολογιστεί η παράγωγος της συνάρτησης 
    $ f(x) = x^{3} \cdot \arcsin{x} $
    \[
      \Bigl(x^{3}\cdot \arcsin{x}\Bigr)' = (x^{3})' \cdot \arcsin{x} + x^{3} \cdot (\arcsin{x} )' = 
      3x^{2} \cdot \arcsin{x} + x^{3} \cdot \frac{1}{\sqrt{1-x^{2}}} = 3x^{2} \arcsin{x}
      + \frac{x^{3}}{\sqrt{1 - x^{2}}} 
    \]
  \end{exercise}

  \begin{exercise} Να υπολογιστεί η παράγωγος της συνάρτησης 
    $ f(x) = \frac{\sin{x}}{x^{3}} $
    \[
      \Bigl(\frac{\sin{x}}{x^{3}} \Bigr)' = \frac{(\sin{x} )' \cdot x^{3}- \sin{x} \cdot
      (x^{3})'}{(x^{3})^{2}} = \frac{\cos{x} \cdot x^{3}- \sin{x} \cdot 3x^{2}}{x^{6}} = 
      \frac{x^{2}(x \cos{x} - 3 \sin{x})}{x^{6}} = \frac{x \cos{x} - 3 \sin{x}}{x^{4}}
    \]  
  \end{exercise}

  \begin{exercise} Να υπολογιστεί η παράγωγος της συνάρτησης 
    $ f(x) = \frac{x^{2}+5}{\ln{x}} $
    \begin{align*}
      \Bigl(\frac{x^{2}+5}{\ln{x}} \Bigr)' 
    & = \frac{(x^{2}+5)'\cdot \ln{x} - (x^{2}+5)\cdot
      (\ln{x} )'}{(\ln{x} )^{2}} = \frac{2x \cdot \ln{x} - (x^{2}+5) \cdot
    \frac{1}{x}}{\ln^{2}{x}} = \frac{2x \cdot \ln{x}}{\ln^{2}{x}} - \frac{x^{2}+5}{x
  \ln^{2}{x}} = \frac{2x}{\ln{x}} - \frac{x^{2}+5}{x \ln^{2}{x}} 
    \end{align*}
  \end{exercise}

  \begin{exercise} Να υπολογιστεί η παράγωγος της συνάρτησης 
    $ f(x) = \frac{x \mathrm{e}^{x}}{5-x^{2}} $
    \begin{align*}
      \Bigl(\frac{x \mathrm{e}^{x}}{5-x^{2}} \Bigr)' 
    &= \frac{(x \mathrm{e}^{x} )'\cdot
    (5-x^{2})- x \mathrm{e}^{x} \cdot (5-x^{2})'}{(5-x^{2})^{2}} = \frac{[(x)'
    \mathrm{e}^{x} + x( \mathrm{e}^{x} )']\cdot (5-x^{2}) - x \mathrm{e}^{x}\cdot 
  (-2x)}{(5-x^{2})^{2}} \\ 
    &= \frac{(\mathrm{e}^{x} + x \mathrm{e}^{x})(5-x^{2})+2x^{2} 
    \mathrm{e}^{x}}{(5-x^{2})^{2}} 
    = \frac{\mathrm{e}^{x} (1+x)(5-x^{2})+2x^{2} \mathrm{e}^{x}}{(5-x^{2})^{2}} 
    = \frac{\mathrm{e}^{x} (5-x^{2}+5x-x^{3}+2x^{2})}{(5-x^{2})^{2}} \\
    &= \frac{\mathrm{e}^{x} (-x^{3}+x^{2}+5x+5)}{(5-x^{2})^{2}} 
    \end{align*}
  \end{exercise}


  \section{Παράγωγος Σύνθετων Συναρτήσεων}

  \subsection{Κανόνας Αλυσίδας - Τύπος Leibniz}

  \begin{prop}[\textcolor{Col1}{Κανόνας Αλυσίδας}]
    Αν η $g$ είναι παραγωγίσιμη συνάρτηση στο $x$ και η $f$ είναι παραγωγίσιμη στο $ g(x)
    $, τότε η \textbf{σύνθετη} συνάρτηση $ y= f(g(x)) $ είναι επίσης παραγωγίσιμη στο $x$ 
    και η παράγωγός της δίνεται από τη σχέση
    \begin{empheq}[box=\mathboxr]{equation}
      \label{eq:chain}
      \bigl[f(g(x))\bigr]' = f'(g(x)) \cdot g'(x)
    \end{empheq}
    % Με το συμβολισμό του Leibniz, αν $ y=f(u) $ και $ u=g(x) $ είναι παραγωγίσιμες
    % συναρτήσεις, τότε 
    % \begin{equation}\label{eq:leib}
    %   \dv{y}{x} = \dv{y}{u} \cdot \dv{u}{x} 
    % \end{equation} 
  \end{prop}

  \enlargethispage*{2\baselineskip}

  \section{Πίνακας Παραγώγων Σύνθετων Συναρτήσεων}

  \begin{center}
    \begin{Mytable}
      \renewcommand{\arraystretch}{2.4}
      \begin{tabular}{|c||c|}
        \TabCellHead Βασικές Συναρτήσεις & \TabCellHead Τριγωνομετρικές Συναρτήσεις \\[5pt] \hline
        -- & $ \bigl(\sin{g(x)}\bigr)' = \cos{g(x)} \cdot g'(x) $ \\[5pt] \hline
        -- & $ \bigl(\cos{g(x)}\bigr)' = - \sin{g(x)}\cdot g'(x) $ \\[5pt] \hline 
        $ \bigl(g(x)^{a}\bigr)' = a g(x)^{a-1}\cdot g'(x) $ & $ \bigl(\tan{g(x)}\bigr)' = \frac{1}{\cos^{2}{g(x)}} \cdot g'(x) $ \\[5pt] \hline
        $ \Bigl(\frac{1}{g(x)}\Bigr)' = - \frac{1}{g^{2}(x)}\cdot g'(x) $ & $
        \bigl(\cot{g(x)}\bigr)' = - \frac{1}{\sin^{2}{g(x)}} \cdot g'(x) $ \\[5pt] \hline
        $ \Bigl(\frac{1}{g^{2}(x)}\Bigr)' = - \frac{2}{g^{3}(x)} \cdot g'(x) $ & $
        \bigl(\arctan{g(x)}\bigr)' = \frac{1}{1 + g^{2}(x)} \cdot g'(x) $ \\[5pt] \hline
        $ \bigl(a^{g(x)}\bigr)' = a^{g(x)}\cdot \ln{a} \cdot g'(x) $ & $
        \bigl(\arccot{g(x)}\bigr)' = \frac{-1}{1 + g^{2}(x)} \cdot g'(x) $ \\[5pt] \hline
        $ \bigl(e^{g(x)}\bigr)' = e^{g(x)} \cdot g'(x) $ &  $ \bigl(\arcsin{g(x)}\bigr)' =
        \frac{1}{\sqrt{1 - g^{2}(x)}} \cdot g'(x) $ \\[5pt] \hline
        $ \bigl(\ln{g(x)}\bigr)' = \frac{1}{g(x)} \cdot g'(x) $ & $ \bigl(\arccos{g(x)}\bigr)'
        = \frac{-1}{\sqrt{1 - g^{2}(x)}} \cdot g'(x) $ \\[5pt] \hline
        $ \bigl(\sqrt[n]{g(x)^{m}}\bigr)' = \bigl(g(x)^{\frac{m}{n}}\bigr)'= \frac{m}{n}
        g(x)^{\frac{m}{n} -1} \cdot g'(x) $ & $ \bigl(\sinh{g(x)}\bigr)' = \cosh{g(x)}
        \cdot g'(x) $ \\[5pt] \hline
        $ \bigl(\sqrt{g(x)}\bigr)' = \frac{1}{2 \sqrt{g(x)}} \cdot g'(x) $ & $
        \bigl(\cosh{g(x)}\bigr)' = \sinh{g(x)} \cdot g'(x) $ \\[5pt] \hline
      \end{tabular}
    \end{Mytable}
  \end{center}


  \subsection{Παραδείγματα}

  \begin{example}
    $ \bigl(\mathrm{e}^{3x^{2}}\bigr)' = \mathrm{e}^{3x^{2}} \cdot (3x^{2})' =
    \mathrm{e}^{3x^{2}} \cdot 6x = 6x \mathrm{e}^{3x^{2}} $
  \end{example}
  \begin{example}
    $ \bigl(\sin{(\sqrt{x})} \bigr)' = \cos{(\sqrt{x})} \cdot (\sqrt{x})' =
    \cos{(\sqrt{x})} \cdot \frac{1}{2 \sqrt{x}} = \frac{\cos{(\sqrt{x})}}{2 \sqrt{x}} $
  \end{example}
  \begin{example}
    $ \bigl(\tan{(2x-3)}\bigr)' = \frac{1}{\cos^{2}{(2x-3)}} \cdot (2x-3)' =
    \frac{1}{\cos^{2}{(2x-3)}} \cdot 2 = \frac{2}{\cos^{2}{(2x-3)}} $
  \end{example}
  \begin{example}
    $ \bigl(\sqrt{x^{3}+2x}\bigr)' = \frac{1}{2 \sqrt{x^{3}+2x}} 
    \cdot \bigl(x^{3}+2x\bigr)' = \frac{3x^{2}+2}{2 \sqrt{x^{3}+2x}} $
  \end{example}
  \begin{example}
    $ \bigl((x^{3}+1)^{3}\bigr)' = 3 (x^{3}+1)^{2} \cdot (x^{3}+1)' = 3 (x^{3}+1)^{2}
    \cdot 3x^{2} = 9 x^{2} (x^{3}+1)^{2} $
  \end{example}
  \begin{example}
    $ (\ln^{2}x)' = [(\ln{x} )^{2}]' = 2 \ln{x} \cdot (\ln{x} )' = 2 \ln{x} \cdot
    \frac{1}{x} = \frac{2 \ln{x}}{x} $
  \end{example}
  \begin{example}
    $ \bigl(\sin^{2}{(3x)}\bigr)' \!\! = 2 \sin{(3x)} \bigl(\sin{(3x)}\bigr)' 
    \! = 2 \sin{(3x)} \cos{(3x)} (3x)' 
    \! = 2 \sin{(3x)} \cos{(3x)} \cdot 3 
    \! = 6 \sin{(3x)} \cos{(3x)} \!\!\!\!\! $
  \end{example}
  \begin{example}
    $ \Bigl(\frac{1}{\ln{x}} \Bigr)' = - \frac{1}{(\ln{x} )^{2}} \cdot (\ln{x} )' 
    = - \frac{1}{\ln^{2}{x}} \cdot \frac{1}{x} = - \frac{1}{x \ln^{2}{x}} $ 
  \end{example}
  \begin{example}
    $ \Bigl(\frac{1}{\sqrt{x}}\Bigr)' = - \frac{1}{(\sqrt{x} )^{2}} \cdot (\sqrt{x})' = 
    = - \frac{1}{x} \cdot \frac{1}{2 \sqrt{x}} = - \frac{1}{2 x \sqrt{x}} $
  \end{example}
  \begin{example}
    $ \Bigl(\frac{1}{x^{2}+1} \Bigr)' = - \frac{1}{(x^{2}+1)^{2}} \cdot (x^{2}+1)' = 
    - \frac{1}{(x^{2}+1)^{2}} \cdot 2x = - \frac{2x}{(x^{2}+1)^{2}} $ 
  \end{example}
  \begin{exercise}
    Έστω η σύνθετη συνάρτηση $ y= \ln{\bigl(\sqrt{x^{2}+1}\bigr)} $. 
    Να υπολογιστεί η παράγωγος της συνάρτησης.  
  \end{exercise}
  \begin{solution}
    Χρησιμοποιώντας τη σχέση~\eqref{eq:chain}, έχουμε:
    \[
      \Bigl[\ln{\bigl(\sqrt{x^{2}+1}\bigr)} \Bigr]' = \frac{1}{\sqrt{x^{2}+1}} \cdot 
      \bigl(\sqrt{x^{2}+1} \bigr)' = \frac{1}{\sqrt{x^{2}+1}} \cdot 
      \frac{1}{2 \sqrt{x^{2}+1}} \cdot (x^{2}+1)' = \frac{1}{2 
      \bigl(\sqrt{x^{2}+1}\bigr)^{2}} \cdot 2x = \frac{x}{x^{2}+1}
    \]  
    % Αν θέλουμε να χρησιμοποιήσουμε τη σχέση~\eqref{eq:leib}, τότε \textbf{αναλύουμε} 
    % τη σύνθετη συνάρτηση, σε απλούστερες, ξεκινώντας από <<μέσα>> και πηγαίνοντας 
    % διαδοχικά προς τα <<έξω>>. Έτσι, θέτουμε $ u=x^{2}+1 $ και στη συνέχεια
    % $v= \sqrt{x^{2}+1} = \sqrt{u} $ και άρα η συνάρτηση μας, γίνεται

    % \twocolumnsiderrr{
    %   \[
    %     y= \ln{\bigl(\sqrt{x^{2}+1}\bigr)} \longrightarrow 
    %     \begin{cases}
    %       y= \ln{v} \\
    %       v= \sqrt{u}  \\
    %       u= x^{2}+1  
    %     \end{cases}
    %   \]
    %   Οπότε, για την παράγωγο, σύμφωνα με τον τύπο του Leibniz, θα ισχύει
    %   \[
    %     \dv{y}{x} = \dv{y}{v} \cdot \dv{v}{u} \cdot \dv{u}{x} = \frac{1}{v} \cdot
    %     \frac{1}{2 \sqrt{u}} \cdot 2x = \frac{1}{\sqrt{x^{2}+1}} \cdot \frac{1}{2
    %     \sqrt{x^{2}+1}} \cdot 2x = \frac{x}{x^{2}+1}
    %   \] 
    % }{
    %   \begin{tikzpicture}
    %     \node (a) at (0,0) {$y$} ;
    %     \node (b) at (0,-1) {$v$} ;
    %     \node (c) at (0,-2) {$u$} ;
    %     \node (d) at (0,-3) {$x$} ;
    %     \draw (a) -- node[right,Col1]{$\dv{y}{v}$} (b) ;
    %     \draw (b) -- node[right,Col1]{$\dv{v}{u}$} (c) ;
    %     \draw (c) -- node[right,Col1]{$\dv{u}{x}$} (d) ;
    %   \end{tikzpicture}       
    % }
  \end{solution}



\end{document}


