\input{preamble_ask.tex}
\input{definitions_ask.tex}
\input{myboxes.tex}
\input{tikz}

\geometry{left=1.9cm,right=1.9cm}

\newcommand{\twocolumnsidescc}[2]{\begin{minipage}[c]{0.48\linewidth}\raggedright
        #1
        \end{minipage}\hfill\begin{minipage}[c]{0.48\linewidth}\raggedright
        #2
    \end{minipage}
}

\geometry{top=2.5cm}
\everymath{\displaystyle}
\pagestyle{askhseis}

\begin{document}

\setcounter{chapter}{1}

\begin{center}
  \minibox{\large \bfseries \textcolor{Col1}{Διωνυμικό Ανάπτυγμα}}
\end{center}

\vspace{\baselineskip}



\section*{Διωνυμικοί Συντελεστές}


\begin{dfn}
  Έστω $ n,k $ ακέραιοι με $ 0 \leq k \leq n $. Ορίζουμε τους 
  \textcolor{Col1}{διωνυμικούς συντελεστές} με τον εξής τρόπο: 
  \[
    \binom{n}{k} = \frac{n!}{k!(n-k)!} = \frac{n(n-1)\cdots (n-k+1)}{k!}, \quad
    \text{αν } k \neq 0,n 
    \quad \text{και} \quad \binom{n}{0} = \binom{n}{n} = 1 
  \] 
\end{dfn}

\subsection*{Ιδιότητες Διωνυμικών Συντελεστών}

\begin{enumerate}
  \item \label{it:idiii} 
    $ \bm{\binom{n+1}{k} = \binom{n}{k-1} + \binom{n}{k}} $
    \begin{proof}
      \begin{align*}
        \binom{n}{k-1} + \binom{n}{k} 
        &= \frac{n!}{(k-1)!(n-k+1)!} + \frac{n!}{k!(n-k)!} 
        = \frac{kn!}{k!(n+1-k)!} + \frac{(n+1-k)n!}{k!(n+1-k)!} \\
        &= n! \left[\frac{k}{k!(n+1-k)!} + \frac{n+1-k}{k!(n+1-k)!}\right] 
        = \frac{n!(n+1)}{k!(n+1-k)!} = \frac{(n+1)!}{k!(n+1-k)!} = 
        \binom{n+1}{k}
      \end{align*} 
    \end{proof}

  \item \textbf{Οι διωνυμικοί συντελεστές είναι φυσικοί αριθμοί}.
    \begin{proof}
    \item {}
      Θα δείξουμε με Μαθηματική Επαγωγή, ότι $ \binom{n}{k} \in \mathbb{N} $,
      για κάθε $ n \in \mathbb{N} $ με $ k \leq n $.
      \begin{myitemize}
        \item Για $ n=1 $, προφανώς, $ \binom{1}{1} \in \mathbb{N} $
        \item Έστω ότι $ \binom{n}{k} \in \mathbb{N}, \; \text{για κάθε } k \leq n $
        \item Θα δείξουμε ότι $ \binom{n+1}{k} \in \mathbb{N}, \; \text{για κάθε } 
          k \leq n+1$. Πράγματι, από την ιδιότητα~\ref{it:idiii} έχουμε
          \[
            \binom{n+1}{k} = \binom{n}{k-1} + \binom{n}{k}, \; \text{για κάθε } 
            k \leq n
          \]
          επομένως από την υπόθεση της επαγωγής, $ \textstyle{\binom{n+1}{k}} \in
          \mathbb{N}, \; \text{για κάθε } k \leq n $. Όμως και 
          $ \textstyle{\binom{n+1}{n+1} \in \mathbb{N}} $, άρα 
          \[
            \binom{n+1}{k} \in \mathbb{N}, \text{για κάθε } k \leq n+1 
          \] 
      \end{myitemize}
    \end{proof}
  % \item $ \binom{n}{k} = \binom{n}{n-k}, \quad 0 \leq k \leq n $
\end{enumerate}

\enlargethispage{2\baselineskip}

\subsection*{Τρίγωνο του Pascal}

Οι \textcolor{Col1}{διωνυμικοί συντελεστές} δίνονται και από το 
\textbf{τρίγωνο του Pascal}, με το συντελεστή $ \textstyle{\binom{n}{k}} $ να 
βρίσκεται στη $ (n+1) $ γραμμή στην $ (k+1) $ θέση .
Το τρίγωνο του Pascal έχει τις παρακάτω δύο ιδιότητες:
\begin{enumerate}[i)]\label{en:pasc}
  \item Ο πρώτος και ο τελευταίος αριθμός κάθε γραμμής είναι $1$.
  \item \label{it:pascii} Οποιοσδήποτε άλλος αριθμός σε κάθε γραμμή, είναι το 
    άθροισμα των 2 αριθμών που βρίσκονται ακριβώς από πάνω του στην αμέσως 
    προηγούμενη γραμμή.
\end{enumerate}
\twocolumnsidescc{
\begin{gather*}
  (a+b)^{0}={\color{Col1}1} \\
  (a+b)^{1}={\color{Col1}1}a+b{\color{Col1}1} \\
  (a+b)^{2}={\color{Col1}1}a^{2}+{\color{Col1}2}ab
  +b^{2}{\color{Col1}1} \\
  (a+b)^{3}={\color{Col1}1}a^{3}+{\color{Col1}3}a^{2}b
  +{\color{Col1}3}ab^{2}+b^{3}{\color{Col1}1} \\
  (a+b)^{4}={\color{Col1}1}a^{4}+{\color{Col1}4}a^{3}b
  +{\color{Col1}6}a^{2}b^{2}+{\color{Col1}4}ab^{3}
  +b^{4}{\color{Col1}1} 
\end{gather*}
}{
\begin{myitemize}
  \item Η ιδlότητα~\ref{it:pascii} του Τριγώνου Pascal, είναι ακριβώς η
    ιδιότητα~\ref{it:idiii} των διωνυμικών συντελεστών. 
    \[ \binom{n+1}{k} = \binom{n}{k-1} + \binom{n}{k} \]
  \item Προφανώς $ \binom{n}{k} = \binom{n}{n-k}, \quad 0 \leq k \leq n $
\end{myitemize}
}



\section*{Διωνυμικό Θεώρημα}

Έστω$ a,b \in \mathbb{R} $ και $ n \in \mathbb{N} $. Τότε
\[
  \boxed{(a+b)^{n} = a^{n} + \binom{n}{1} a^{n-1}b + 
    \binom{n}{2} a^{n-2}b^{2} + \binom{n}{3} a^{n-3}b^{3} + \cdots + 
  \binom{n}{n-1} ab^{n-1} + b^{n} = \sum_{k=0}^{n} \binom{n}{k} a^{n-k}b^{k}}
\]
όπου οι διωνυμικοί συντελεστές δίνονται από τον τύπο
$
\binom{n}{k} = \frac{n!}{k!(n-k)!}
$ 

Με τη βοήθεια του Διωνυμικού θεωρήματος αποδεικνύονται πολύ εύκολα τα παρακάτω 
αθροίσματα των διωυυμικών συντελεστών.
\begin{prop}
\item {}
  \begin{enumerate}
    \item $ \sum_{k=0}^{n} \binom{n}{k} = \binom{n}{0} +\cdots+ \binom{n}{n} =  2^{n} $. 
      Πράγματι, $ 2^{n}=(1+1)^{n} =
      \sum_{k=0}^{n} \binom{n}{k} $
    \item $ \sum_{k=0}^{n} (-1)^k \binom{n}{k} = \binom{n}{0} - \binom{n}{1} +\cdots
      +(-1)^{n} \binom{n}{n} = 0 $. Πράγματι, $ 0 = (1+(-1))^{n} = 
      \sum_{k=0}^{n} (-1)^{k}\binom{n}{k} $
    \item $ \sum_{k \; \text{περιττός}} \binom{n}{k} = 2^{n-1} $. 
      Πράγματι με αφαίρεση της ii) από την i) έχουμε: $ 2 \sum_{k \; \text{περιττός}} 
      \binom{n}{k} = 2^{n} \Leftrightarrow \sum_{k \; \text{περιττός}}
      \binom{n}{k} = 2^{n-1} $
    \item $ \sum_{k \; \text{άρτιος}} \binom{n}{k} = 2^{n-1} $. 
      Πράγματι με πρόσθεση των i) και ii) έχουμε: $ 2 \sum_{k \; \text{άρτιος}}
      \binom{n}{k} = 2^{n} \Leftrightarrow \sum_{k \text{περιττός}} \binom{n}{k}
      = 2^{n-1} $
  \end{enumerate}
\end{prop}



\section*{Διωνυμική Σειρά}


Έστω $ a,b \in \mathbb{R} $ με $ \abs{a} > \abs{b} $ και $ p \in \mathbb{R} $, όχι
φυσικός και $ k \in \mathbb{N} $. Τότε
\[
  \boxed{(a+b)^{p} = a^{p} + \binom{p}{1} a^{p-1}b + \binom{p}{2} a^{p-2}b^{2} +
    \binom{p}{3} a^{p-3}b^{3} + \cdots + \binom{p}{k} a^{p-k}b^{k} + \cdots =
  \sum_{k=0}^{\infty} \binom{p}{k} a^{p-k}b^{k}}
\] 
ή ισοδύναμα
\[
  (a+b)^{p} = a^{p}\left[1 + \binom{p}{1} \left(\frac{b}{a}\right)+ \binom{p}{2} 
    \left(\frac{b}{a}\right)^{2}+ \binom{p}{3} \left(\frac{b}{ a}\right)^{3}+ \cdots + 
  \binom{p}{k} \left(\frac{b}{a}\right)^{k}+ \cdots\right]
\] 
όπου οι διωνυμικοί συντελεστές δίνονται από τον τύπο
$
\binom{p}{k} = \frac{p(p-1)(p-2)\cdots (p-(k-1))}{k!} 
$ με $ \binom{p}{0} \overset{\text{ορ.}}{=} 1 $.
\begin{rem}
\item {}
  Για $ p \in \mathbb{R} $ το διωνυμικό ανάπτυγμά είναι \textbf{σειρά}
  και μάλιστα συγκλίνουσα, αν $ \abs{\frac{b}{a}} < 1 $. 

  Ισχύει ότι αν $ \abs{\frac{b}{a}} < < 1  $ τότε  
  $ (a+b)^{p} \approx a^{p}\left(1+p\left(\frac{b}{a}\right)\right)  $ 
\end{rem}
\[
  (1+x)^p = 1 + px + \frac{p(p-1)}{2!} x^2 + \cdots + \frac{p(p-1)\cdots (p-k+1)}{k!}
  x^n + \cdots   
\] 
\begin{rem}
\item {}
  \begin{myitemize}
    \item Αν $ p>0 $ αλλά όχι ακέραιος, τότε η σειρά συγκλίνει (απολύτως) για $ -1 \leq x
      \leq 1 $
    \item Αν $-1<p<0$, τότε η σειρά συγκλίνει για $ -1 < x \leq 1 $
    \item Αν $p \leq -1$, τότε η σειρά συγκλίνει για $ -1 < x < 1 $
  \end{myitemize}
\end{rem}



\end{document}
