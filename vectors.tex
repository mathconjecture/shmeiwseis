\input{preamble_ask.tex}
\input{definitions_ask.tex}
\input{tikz.tex}
\input{myboxes.tex}

\usepackage[font={color=Col1},labelfont={bf},hypcap=false]{caption}
\pagestyle{vangelis}
\everymath{\displaystyle}


\newcommand{\twocolumnsidesll}[2]{\begin{minipage}[c]{0.60\linewidth}
        #1
        \end{minipage}\hfill\begin{minipage}[c]{0.35\linewidth}
        #2
    \end{minipage}
}


\newcommand{\twocolumnsidesrr}[2]{\begin{minipage}[t]{0.40\linewidth}
        #1
        \end{minipage}\hfill\begin{minipage}[t]{0.55\linewidth}
        #2
    \end{minipage}
}

\newcommand{\twocolumnsidescc}[2]{\begin{minipage}[c]{0.50\linewidth}
        #1
        \end{minipage}\hfill\begin{minipage}[c]{0.45\linewidth}
        #2
    \end{minipage}
}





\begin{document}

\section*{Η Έννοια του Διανύσματος}

\twocolumnsidesll{
  Ένα \textbf{προσανατολισμένο} ευθύγραμμο τμήμα, δηλαδή ένα ευθύγραμμο τμήμα με 
  ορισμένη αρχή και τέλος (ή πέρας), ονομάζεται \textcolor{Col1}{διάνυσμα}. Το διάνυσμα 
  με αρχή $ A $ και τέλος $ B $ συμβολίζεται με $ \mathbf{AB} $. Το μήκος του 
  ευθύγραμμου τμήματος $ AB $, ονομάζεται \textcolor{Col1}{μέτρο} του διανύσματος $ 
  \mathbf{AB} $ και συμβολίζεται με $ \norm{\mathbf{AB}} $. Δυο \textbf{παράλληλα} 
  διανύσματα με την ίδια φορά, ονομάζονται \textbf{ομόρροπα,} ενώ αν έχουν αντίθετη φορά 
  ονομάζονται \textbf{αντίρροπα}.
}{
  \begin{tikzpicture}
    \draw (0,0) -- coordinate[pos=0.1] (a) coordinate[pos=0.9] (b) (2,2) ;
    \fill (a) node[left=5pt] {$A$} circle (1.5pt) ;
    \fill (b) node[left=5pt] {$B$} circle (1.5pt) ;
    \draw[ultra thick,-latex,Col1] (a) -- (b) ;
    \begin{scope}[xshift=35pt]
      \draw (0,0) -- coordinate[pos=0.2] (a) coordinate[pos=0.8] (b) (2,2) ;
      \draw[ultra thick,-latex,Col2] (a) -- (b) ;
    \end{scope}
    \begin{scope}[xshift=50pt]
      \draw (0,0) -- coordinate[pos=0.3] (a) coordinate[pos=0.7] (b) (2,2) ;
      \draw[ultra thick,-latex,Col2] (a) -- (b) ;
    \end{scope}
    \begin{scope}[xshift=80pt]
      \draw (0,0) -- coordinate[pos=0.1] (a) coordinate[pos=0.7] (b) (2,2) ;
      \draw[ultra thick,-latex,Col2] (a) -- (b) ;
    \end{scope}
    \begin{scope}[xshift=95pt]
      \draw (0,0) -- coordinate[pos=0.2] (a) coordinate[pos=0.9] (b) (2,2) ;
      \draw[ultra thick,latex-,Col2] (a) -- (b) ;
    \end{scope}
  \end{tikzpicture}
}

\vspace{\baselineskip}
\twocolumnsides{
  \subsection*{Ίσα και Αντίθετα Διανύσματα}
  Δύο \textbf{ομόρροπα} διανύσματα με ίσα μέτρα, ονομάζονται \textcolor{Col1}{ίσα}, 
  ενώ δύο \textbf{αντίρροπα} διανύσματα με ίσα μέτρα, ονομάζονται 
  \textcolor{Col1}{αντίθετα}. 
  \begin{myitemize}
    \item $ \mathbf{a} = \mathbf{b} \Leftrightarrow \mathbf{a} \uparrow\uparrow 
      \mathbf{b} $ και $ \norm{\mathbf{a}} = \norm{\mathbf{b}} $
    \item $ \mathbf{a} = -\mathbf{b} \Leftrightarrow \mathbf{a} \uparrow 
      \downarrow \mathbf{b} $ και $ \norm{\mathbf{a}} = \norm{\mathbf{b}} $
  \end{myitemize}
  \subsection*{Παράλληλα Διανύσματα}
  \begin{myitemize}
    \item $ \mathbf{a} \parallel \mathbf{b} \Leftrightarrow \exists \lambda \in
      \mathbb{R} \; \text{ώστε} \; \mathbf{a} = \lambda \mathbf{b}, \; \mathbf{b} \neq
      \vec{0} $
  \end{myitemize}
}{
  \subsection*{Γωνία Διανυσμάτων}
  Αν $ \mathbf{a}=\mathbf{OA} \neq \mathbf{0}$ και 
  $ \mathbf{b}=\mathbf{OB} \neq \mathbf{0} $, τότε η \textbf{κυρτή} γωνία 
  $ \phi =  A\widehat{O}B $, ονομάζεται \textcolor{Col1}{γωνία} των $ \mathbf{a} $ και 
  $ \mathbf{b} $ και συμβολίζεται με $ \phi = (\widehat{\mathbf{a}, \mathbf{b}}) $. 
  Ισχύει $ 0 \leq \phi \leq \pi $.

  \twocolumnsidescc{
    \begin{myitemize}
      \item $ \mathbf{a} \uparrow\uparrow \mathbf{b} \Leftrightarrow \phi = 0 $
      \item $ \mathbf{a} \uparrow\downarrow \mathbf{b} \Leftrightarrow \phi = \pi $
      \item $ (\widehat{\mathbf{a}, \mathbf{b}}) + (\widehat{- \mathbf{a}, \mathbf{b}}) 
        = \pi $
    \end{myitemize}
  }{
    \begin{center}
      \begin{tikzpicture}[scale=0.8]
        \draw (2,0) -- coordinate[pos=0.2] (a) (0,0) coordinate (o) -- 
          coordinate[pos=0.8] (b) (2.0,1.5) ; 
        \draw (o) -- coordinate[pos=0.8] (c) (-2,0) ;
        \node[Col1,below] at (o) {$O$} ; 
        \draw (0,0) pic[fill=Col1!50,draw,-latex,"$\phi$",angle eccentricity=1.5] 
          {angle=a--o--b} ;
        \draw (o) edge[-latex,ultra thick,Col1] node[midway,below] {$ \mathbf{a} $} (a) ;
        \draw (o) edge[-latex,ultra thick,Col1] node[midway,above] {$ \mathbf{b} $} (b) ;
        \draw (o) edge[-latex,ultra thick,Col1] node[midway,below] {$ \mathbf{-a} $} 
          (c) ;
        \fill[Col1] (a) node[below] {$A$} circle (1.5pt) ;
        \fill[Col1] (b) node[above left,xshift=3pt] {$B$} circle (1.5pt) ;
      \end{tikzpicture}
    \end{center}
  }
}

\vspace{\baselineskip}

\twocolumnsides{
  \subsection*{Άθροισμα - Διαφορά Διανυσμάτων}
  \twocolumnsides{
    \begin{tikzpicture}[>=latex]
      \coordinate (o) at (0,0) ;
      \coordinate (a) at (2.3,0.8) ;
      \coordinate (b) at (2.6,2.2) ;
      \coordinate (c) at (0.3,1.4) ;
      \draw[->,Col2,very thick] (o) to node[midway,left] {$ \mathbf{b}$} (c) ;
      \draw[->,thick] (c) to node[midway,above] {$ \mathbf{a}$} (b) ;
      \draw[->,Col2,very thick] (o) to node[midway,below] {$ \mathbf{a} $} (a) ;
      \draw[->,thick] (a) to (b) ;
      \draw[->,thick] (a) to node[midway,right] {$ \mathbf{b} $} (b) ;
      \draw[->,Col1,very thick] (o) to node[midway,below,sloped] 
        {$ \mathbf{a}+ \mathbf{b} $} node[midway,above,sloped] 
        {$ \mathbf{b}+ \mathbf{a} $} (b) ;
    \end{tikzpicture}
  }{
    \begin{tikzpicture}[>=latex]
      \coordinate (o) at (0,0) ;
      \coordinate (a) at (2.3,0.8) ;
      \coordinate (b) at (2.6,2.2) ;
      \coordinate (c) at (0.3,1.4) ;
      \draw[->,Col2,very thick] (o) to node[midway,left] {$ \mathbf{b}$} (c) ;
      \draw (c) to (b) ;
      \draw[->,Col2,very thick] (o) to node[midway,below] {$ \mathbf{a} $} (a) ;
      \draw (a) to (b) ;
      \draw[->,Col1,very thick] (c) to node[midway,above,sloped] 
        {$ \mathbf{a}- \mathbf{b} $} (a) ;
    \end{tikzpicture}
  }
}{
  \subsection*{Πολλαπλασιασμός Αριθμού με Διάνυσμα}
  \begin{tikzpicture}[>=latex]
    \coordinate (o) at (0,0) ;
    \coordinate (a) at (0.8,0.8) ;
    \draw[->,ultra thick] (o) to node[midway,left=2pt] {$ \mathbf{a}$} (a) ;
    \begin{scope}[xshift=30pt,scale=2]
      \coordinate (o) at (0,0) ;
      \coordinate (a) at (0.8,0.8) ;
      \draw[->,Col1,ultra thick] (o) to node[midway,left=2pt] {$ 2\mathbf{a}$} (a) ;
    \end{scope}
    \begin{scope}[xshift=70pt,scale=0.5]
      \coordinate (o) at (0,0) ;
      \coordinate (a) at (0.8,0.8) ;
      \draw[->,Col1,ultra thick] (o) to node[midway,left=2pt] 
        {$ \mathsmaller{\frac{1}{2}} \mathbf{a}$} (a) ;
    \end{scope}
    \begin{scope}[xshift=120pt]
      \coordinate (o) at (0.8,0.8) ;
      \coordinate (a) at (0,0) ;
      \draw[->,Col2,ultra thick] (o) to node[midway,left=2pt,yshift=5pt] 
        {$ -\mathbf{a}$} (a) ;
    \end{scope}
    \begin{scope}[xshift=170pt,scale=1.5]
      \coordinate (o) at (0.8,0.8) ;
      \coordinate (a) at (0,0) ;
      \draw[->,Col2,ultra thick] (o) to node[midway,left=2pt,yshift=5pt] 
        {$ -1.5\mathbf{a}$} (a) ;
    \end{scope}
  \end{tikzpicture}
}

\vspace{\baselineskip}

\twocolumnsides{
  \subsection*{Τριγωνική Ανισότητα} 
  \begin{myitemize}[leftmargin=*]
    \item Αν $ \mathbf{a} \nparallel \mathbf{b} $ τότε $\abs{\norm{\mathbf{a}} - \norm{\mathbf{b}}} < \norm{\mathbf{a}+ \mathbf{b}} <
      \norm{\mathbf{a}} + \norm{\mathbf{b}}$
      \begin{center}
        \begin{tikzpicture}[>=latex]
          \draw (0,0) coordinate (o) -- (2.0,1.0) coordinate (a) -- (3,0) coordinate 
            (b) -- cycle ;
          \draw[->,ultra thick,Col2] (o) -- (a) node[midway,above] {$ \mathbf{a}$} ;
          \draw[->,ultra thick,Col2] (a) -- (b) node[midway,right,yshift=4pt] 
            {$ \mathbf{b}$} ;
          \draw[->,ultra thick,Col1] (o) -- (b) node[midway,below] {$ \mathbf{a+b}$} ;
        \end{tikzpicture}
      \end{center}
    \item Αν $ \mathbf{a} \upuparrows \mathbf{b} $ τότε $\abs{\norm{\mathbf{a}} - \norm{\mathbf{b}}} < \norm{\mathbf{a}+ \mathbf{b}} =
      \norm{\mathbf{a}} + \norm{\mathbf{b}}$
      \begin{center}
        \begin{tikzpicture}[>=latex]
          \coordinate (o) at (0,0) ; 
          \coordinate (a) at (2,0) ; 
          \coordinate (b) at (3.5,0) ; 
          \draw (o) edge[->,Col2,ultra thick] node[midway,above] {$ \mathbf{a}$} (a) ;
          \draw (a) edge[->,Col2,ultra thick] node[midway,above] {$ \mathbf{b}$} (b) ;
          \draw (0,-0.6) edge[->,Col1,ultra thick] node[midway,above] {$ \mathbf{a+b}$} 
            (3.5,-0.6) ;
        \end{tikzpicture}
      \end{center}
    \item Αν $ \mathbf{a} \uparrow\downarrow \mathbf{b} $ τότε $\abs{\norm{\mathbf{a}} - \norm{\mathbf{b}}} = \norm{\mathbf{a}+ \mathbf{b}} <
      \norm{\mathbf{a}} + \norm{\mathbf{b}}$
      \begin{center}
        \begin{tikzpicture}[>=latex]
          \coordinate (o) at (0,0) ; 
          \coordinate (a) at (2,0) ; 
          \coordinate (b) at (3.5,0) ; 
          \draw (o) edge[->,Col2,ultra thick] node[midway,above] {$ \mathbf{a}$} (b) ;
          \draw (3.5,-.5) edge[->,Col2,ultra thick] node[midway,above] {$ \mathbf{b}$} (2,-.5) ;
          \draw (0,-1.1) edge[->,Col1,ultra thick] node[midway,above] {$ \mathbf{a+b}$} 
            (2,-1.1) ;
        \end{tikzpicture}
      \end{center}
  \end{myitemize}
}{
  \subsection*{Γραμμική Ανεξαρτησία}
  \begin{myitemize}[leftmargin=*]
    \item \textcolor{Col1}{Γραμμικός συνδυασμός} των διανυσμάτων $ \mathbf{v_{1}},
      \mathbf{v_{2}}, \ldots, \mathbf{v_{n}} $ 
      ονομάζεται κάθε διάνυσμα της μορφής 
      $\lambda_{1} \mathbf{v_{1}}+ \lambda_{2} \mathbf{v_{2}} + \cdots
      \lambda _{n}\mathbf{v_{n}} $ με 
      $ \lambda_{1}, \lambda_{2}, \ldots, \lambda_{n} \in \mathbb{R} $ 
    \item Τα διανύσματα $ \mathbf{v_{1}}, \mathbf{v_{2}}, \ldots, \mathbf{v_{n}} $ 
      ονομάζονται \textcolor{Col1}{γραμμικώς ανεξάρτητα}, αν 
      οποτεδήποτε $\lambda_{1} \mathbf{v_{1}} + \lambda_{2}
      \mathbf{v_{2}} + \cdots + \lambda_{n} \mathbf{v_{n}} \! = \mathbf{0} $ τότε
      $ \lambda_{1} = \cdots = \lambda_{n} \! = 0$. 
    \item Το $ \mathbf{0} $ είναι γραμμικώς εξαρτημένο.
    \item Αν $ \mathbf{a} \neq \mathbf{0} $ τότε $ \mathbf{a}$ 
      είναι γραμμικώς ανεξάρτητο.
    \item Αν $ \mathbf{a} \nparallel \mathbf{b} $ τότε $\mathbf{a}, \mathbf{b} $ 
      είναι γραμμικώς ανεξάρτητα.
    \item Αν $ \mathbf{a} \parallel \mathbf{b} $ τότε $\mathbf{a}, \mathbf{b} $ 
      είναι γραμμικώς εξαρτημένα.
    \item Τρία διανύσματα του $ \mathbb{R}^{3} $ είναι γραμμικώς ανεξάρτητα $
      \Leftrightarrow $ δεν είναι συνεπίπεδα.
    \item Τα $ \mathbf{v_{1}}, \mathbf{v_{2}}, \ldots, \mathbf{v_{n}} $ είναι
      γραμμικώς εξαρτημένα $ \Leftrightarrow $ τουλάχιστον ένα από αυτά είναι γραμμικός 
      συνδυασμός των υπολοίπων.
  \end{myitemize}
}

\twocolumnsidesll{
  \section*{Διανύσματα στον $ \mathbb{R}^{3}$}
  Έστω $ \mathbf{u} \in \mathbb{R}^{3} $. Τότε υπάρχουν \textbf{μοναδικοί} αριθμοί, 
  $ u_{1}, u_{2}, u_{3} \in \mathbb{R} $, ώστε 
  $\mathbf{u} =  u_{1}\,\mathbf{i} + u_{2}\,\mathbf{j} + u_{3}\,\mathbf{k} $.
  Τα διανύσματα $ u_{1}\, \mathbf{i} $, $ u_{2}\, \mathbf{j}, u_{3}\, \mathbf{k} $ 
  ονομάζονται \textcolor{Col1}{συνιστώσες} του $ \mathbf{u} $ 
  ενώ οι συντελεστές $ u_{1}, u_{2}, u_{3} $ ονομάζονται 
  \textcolor{Col1}{συντεταγμένες} του $ \mathbf{u} $ και γράφουμε 
  $ \mathbf{u} = (u_{1}, u_{2}, u_{3}) $.
}{
  \begin{center}
    \begin{tikzpicture}[>=latex]
      \coordinate (o) at (0,0,0) ;
      \coordinate (x) at (0,0,2.5) ;
      \coordinate (y) at (2.5,0,0) ;
      \coordinate (z) at (0,2.5,0) ;
      \draw[axis,->] (o) -- (x) node[left] {$x$} ;
      \draw[axis,->] (o) -- (y) node[below] {$y$} ;
      \draw[axis,->] (o) -- (z) node[left] {$z$} ;
      \coordinate (u) at (1.8,2.2,1.8) ;
      \node (vu) at (u)[above right,Col1] {$\mathbf{u}$} ;
      \draw[very thick,->,Col1] (o) -- node[pos=0.7,left,black,xshift=-1pt] 
        {$\norm{\mathbf{u}}$} (u) ;
      \coordinate (pr) at (1.7,0,1.7) ;
      \draw[ultra thin] (u) -- (pr) ;
      \draw[dashed] (xyz cs:z=1.7) coordinate (u1) 
        node[left,yshift=3pt] {$u_{1}$} -- (pr) ;
      \draw[dashed] (xyz cs:x=1.7) coordinate (u2) node[above] {$u_{2}$} -- (pr) ;
      \draw[dashed] (xyz cs:y=2.1) coordinate (u3) node[left] {$u_{3}$} -- (u) ;
      \coordinate (i) at (0,0,0.6) ;
      \coordinate (j) at (0.6,0,0) ;
      \coordinate (k) at (0,0.6,0) ;
      \draw[thick,->] (o) -- (i) node[left,yshift=3pt] {$\mathbf{i}$} ;
      \draw[thick,->] (o) -- (j) node[above] {$\mathbf{j}$} ;
      \draw[thick,->] (o) -- (k) node[left] {$\mathbf{k}$} ;
      \draw[thin,Col1!50] (o) -- (pr) ;
      \draw pic[draw,scale=0.5] {right angle=pr--u1--o} ;
      \draw pic[draw,scale=0.5] {right angle=pr--u2--o} ;
      \draw pic[draw,scale=0.5] {right angle=u--u3--o} ;
      \draw pic[draw,scale=0.5] {right angle=u--pr--o} ;
    \end{tikzpicture}
  \end{center}
}

\twocolumnsidesrr{
  \subsection*{Μέτρο Διανύσματος}
  \begin{myitemize}
    \item $ \norm{\mathbf{u}} = \sqrt{u_{1}^{2}+ u_{2}^{2} + u_{3}^{2}} \geq 0 $
  \end{myitemize}
  \subsection*{Μοναδιαίο Διάνυσμα}
  \begin{myitemize}
    \item $ \widehat{\mathbf{u}} = \frac{1}{\norm{\mathbf{u}}} \mathbf{u} $
  \end{myitemize}
}{
  \subsection*{Ισότητα Διανυσμάτων}
  Έστω $ \mathbf{u} = (a_{1}, a_{2}, a_{3}) $ και $ \mathbf{v}= (b_{1}, b_{2} , b_{3}) $ 
  διανύσματα του $ \mathbb{R}^{3} $
  \begin{myitemize}
    \item $\mathbf{u} = \mathbf{v} \Leftrightarrow a_{1}= b_{1},\; a_{2}= b_{2}, 
      \; a_{3}= b_{3} $
  \end{myitemize}
  \subsection*{Πράξεις Διανυσματών}
  \begin{myitemize}[leftmargin=*]
    \item $ \mathbf{u}+ \mathbf{v} = (a_{1}, a_{2}, a_{3}) + (b_{1}, b_{2},
      b_{3}) = (a_{1}+ b_{1}, a_{2}+ b_{2}, a_{3}+ b_{3}) $
    \item $ \mathbf{u}- \mathbf{v} = (a_{1}, a_{2}, a_{3}) - (b_{1}, b_{2},
      b_{3}) = (a_{1}- b_{1}, a_{2}- b_{2}, a_{3}- b_{3}) $
    \item $ \lambda \mathbf{u} = \lambda (a_{1}, a_{2}, a_{3}) = (\lambda a_{1}, \lambda
      a_{2}, \lambda a_{3}) $
  \end{myitemize}

}

\end{document}
