\input{preamble_ask.tex}
\input{definitions_ask.tex}
\input{tikz.tex}
\input{myboxes.tex}

\pagestyle{vangelis}
\everymath{\displaystyle}


\newcommand{\twocolumnsidesll}[2]{\begin{minipage}[c]{0.60\linewidth}
        #1
        \end{minipage}\hfill\begin{minipage}[c]{0.35\linewidth}
        #2
    \end{minipage}
}



\begin{document}

\section*{Η Έννοια του Διανύσματος}

\twocolumnsidesll{
Ένα \textbf{προσανατολισμένο} ευθύγραμμο τμήμα, δηλαδή ένα ευθύγραμμο τμήμα με ορισμένη 
αρχή και τέλος (ή πέρας), ονομάζεται \textcolor{Col1}{διάνυσμα}. Το διάνυσμα με αρχή 
$ A $ και τέλος $ B $ συμβολίζεται με $ \mathbf{AB} $. Το μήκος του ευθύγραμμου τμήματος 
$ AB $, ονομάζεται \textcolor{Col1}{μέτρο} του διανύσματος $ \mathbf{AB} $ και 
συμβολίζεται με $ \norm{\mathbf{AB}} $. Δυο παράλληλα διανύσματα με την ίδια φορά,
ονομάζονται \textbf{ομόρροπα,} ενώ αν έχουν αντίθετη φορά ονομάζονται \textbf{αντίρροπα}.
}{
  \begin{tikzpicture}
    \draw (0,0) -- coordinate[pos=0.1] (a) coordinate[pos=0.9] (b) (2,2) ;
    \fill (a) node[left=5pt] {$A$} circle (1.5pt) ;
    \fill (b) node[left=5pt] {$B$} circle (1.5pt) ;
    \draw[ultra thick,-latex,Col1] (a) -- (b) ;
    \begin{scope}[xshift=35pt]
    \draw (0,0) -- coordinate[pos=0.2] (a) coordinate[pos=0.8] (b) (2,2) ;
    \draw[ultra thick,-latex,Col1] (a) -- (b) ;
    \end{scope}
    \begin{scope}[xshift=50pt]
    \draw (0,0) -- coordinate[pos=0.3] (a) coordinate[pos=0.7] (b) (2,2) ;
    \draw[ultra thick,-latex,Col1] (a) -- (b) ;
    \end{scope}
    \begin{scope}[xshift=80pt]
    \draw (0,0) -- coordinate[pos=0.1] (a) coordinate[pos=0.7] (b) (2,2) ;
    \draw[ultra thick,-latex,Col1] (a) -- (b) ;
    \end{scope}
    \begin{scope}[xshift=95pt]
    \draw (0,0) -- coordinate[pos=0.2] (a) coordinate[pos=0.9] (b) (2,2) ;
    \draw[ultra thick,latex-,Col1] (a) -- (b) ;
    \end{scope}
  \end{tikzpicture}
}

\twocolumnsidesll{
Έστω τα διανύσματα του $ \mathbb{R}^{3} $
\begin{myitemize}
  \item $ \mathbf{u} = (u_{1}, u_{2}, u_{3}) \overset{\text{ή}}{=} u_{1} \mathbf{i} +
    u_{2} \mathbf{j} + u_{3} \mathbf{k} $
  \item $ \mathbf{v} = (v_{1}, v_{2}, v_{3}) $
  \item $ \mathbf{w} = (w_{1}, w_{2}, w_{3}) $
\end{myitemize}
}{
\begin{tikzpicture}[>=latex,scale=0.9]
  \coordinate (o) at (0,0,0) ;
  \coordinate (x) at (0,0,2.5) ;
  \coordinate (y) at (2.5,0,0) ;
  \coordinate (z) at (0,2.5,0) ;
  \draw[axis,->] (o) -- (x) node[left] {$x$} ;
  \draw[axis,->] (o) -- (y) node[below] {$y$} ;
  \draw[axis,->] (o) -- (z) node[left] {$z$} ;
  \coordinate (u) at (2.0,2.2,1.8) ;
  \node (vu) at (u)[above right,Col1] {$\mathbf{u}$} ;
  \draw[very thick,->,Col1] (o) -- node[pos=0.7,left,black,xshift=-1pt] 
    {$\norm{\mathbf{u}}$} (u) ;
  \coordinate (pr) at (2.0,0,1.7) ;
  \draw[dashed] (u) -- (pr) ;
  \draw[dashed] (xyz cs:z=1.7) node[left,yshift=3pt] {$u_{1}$} -- (pr) ;
  \draw[dashed] (xyz cs:x=2.0) node[above] {$u_{2}$} -- (pr) ;
  \draw[dashed] (xyz cs:y=2.0) node[left] {$u_{3}$} -- (u) ;
  \coordinate (i) at (0,0,0.6) ;
  \coordinate (j) at (0.6,0,0) ;
  \coordinate (k) at (0,0.6,0) ;
  \draw[thick,->] (o) -- (i) node[left,yshift=3pt] {$\mathbf{i}$} ;
  \draw[thick,->] (o) -- (j) node[above] {$\mathbf{j}$} ;
  \draw[thick,->] (o) -- (k) node[left] {$\mathbf{k}$} ;
\end{tikzpicture}
}

\subsection*{Μέτρο Διανύσματος}

\begin{myitemize}
  \item $ \norm{\mathbf{u}} = \sqrt{u_{1}^{2} + u_{2}^{2} + u_{3}^{2}} $
\end{myitemize}

\subsection*{Μοναδιαίο Διάνυσμ α}

\begin{myitemize}
  \item $ \hat{\mathbf{u}} = \frac{1}{\norm{\mathbf{u}}} \cdot \mathbf{u} $
    \end{myitemize}
 

\end{document}
