\input{preamble_ask.tex}
\input{definitions_ask.tex}

\renewcommand{\qedsymbol}{}

\pagestyle{vangelis}


\begin{document}

\begin{center}
  \minibox{\large\bfseries \textcolor{Col1}{Ορθογωνιότητα Συναρτήσεων}}
\end{center}

\vspace{\baselineskip}



\section*{Ορισμοί}

\begin{dfn}
  Έστω δυο πραγματικές συναρτήσεις $ f(x), g(x) $ ορισμένες σε ένα διάστημα 
  $ [a,b] \subseteq \mathbb{R} $. Ορίζουμε ως \textcolor{Col1}{εσωτερικό γινόμενο} 
  των $ f(x) $ και $ g(x) $ και το συμβολίζουμε με $ <f,g> $ το ολοκλήρωμα
  \[
    <f,g> = \int _{a}^{b} w(x) f(x)g(x) \,{dx} 
  \] 
  όπου $ w(x) $ γνωστή \textbf{θετική} συνάρτηση, $ \forall x \in [a,b] $, η οποία 
  ονομάζεται \textcolor{Col1}{συνάρτηση βάρους}.
\end{dfn}

\begin{rem}
  Τα όρια ολοκλήρωσης στο παραπάνω ολοκλήρωμα μπορεί να είναι και $ \infty $.
\end{rem}

\begin{dfn}
  Δυο πραγματικές συναρτήσεις $ f(x), g(x) $ είναι \textcolor{Col1}{ορθογώνιες} 
  στο διάστημα $ [a,b] \subseteq \mathbb{R} $ ως προς τη συνάρτηση βάρους $ w(x) $, 
  αν το εσωτερικό τους γινόμενο είναι \textbf{μηδέν}, δηλαδή:
  \[
    \int _{a}^{b} w(x)f(x)g(x) \,{dx} = 0
  \] 
\end{dfn}

\begin{dfn}
  Μια ακολουθία πραγματικών συναρτήσεων $ \{ f_{n}(x) \}_{n=1}^{\infty} $ αποτελεί 
  ένα \textcolor{Col1}{σύστημα ορθογώνιων συναρτήσεων} στο διάστημα $ [a,b] $, 
  ως προς τη συνάρτηση βάρους $ w(x) $, αν τα μέλη της είναι 
  \textbf{ανά δύο ορθογώνια}, δηλαδή:
  \[
    \int _{a}^{b} w(x) f_{n}(x)f_{m}(x) \,{dx} = 0, \quad \forall n \neq m, \; 
    \text{όπου} \;  n,m = 1,2,3, \ldots
  \] 
\end{dfn}

\begin{rem}
  Οι παρακάτω ταυτότητες θα είναι χρήσιμες στον υπολογισμό των ολοκληρωμάτων:
  \begin{align*}
    \cos{a} \cos{b} &= \frac{1}{2} [\cos{(a-b)} + \cos{(a+b)}] \\ 
    \sin{a} \sin{b} &= \frac{1}{2} [\cos{(a-b)} - \cos{(a+b)}] \\
    \sin{a} \cos{b} &= \frac{1}{2} [\sin{(a-b)} + \sin{(a+b)}]
  \end{align*} 
\end{rem}


\section*{Ορθογωνιότητα των \ensuremath{\boldsymbol {\sin{(nx)}, \; \cos{(nx)}}}}

\begin{prop}
  Το σύστημα των συναρτήσεων $ \{1, \sin{(nx), \cos{(nx)} } \} _{n=1}^{\infty} $ 
  είναι ορθογώνιο στο διάστημα $ \color{Col1}{[- \pi , \pi ]} $, ως προς τη 
  συνάρτηση βάρους $ w(x)=1 $.
\end{prop}
\begin{proof}
\item {}
  Εξετάζουμε ως προς την ορθογωνιότητα \textbf{ανά δύο} τις συναρτήσεις.
  \[
    \int _{- \pi}^{\pi} 1 \cdot \sin{(nx)} \,{dx} 
    =  \left[\frac{- \cos{(nx)}}{n} \right]_{- \pi}^{\pi} 
    = - \frac{1}{n} \left[\cos{(nx)} \right]_{- \pi}^{\pi} 
    = - \frac{1}{n} [\cos{(n \pi)} - \cos{(- n \pi)}] 
    = - \frac{1}{n} [\cos{(n \pi)}- \cos{(n \pi)}] = 0 
  \] 
  \[
    \int _{-\pi}^{\pi} 1 \cdot \cos{(nx)} \,{dx} 
    =  \left[\frac{\sin{(nx)}}{n} \right]_{- \pi}^{\pi} 
    = \frac{1}{n} \left[\sin{(nx)} \right]_{- \pi}^{\pi} 
    = \frac{1}{n} [ \sin{(n \pi)} - \sin{(- n \pi)}] =  \frac{2}{n} \sin{(n \pi)} = 0 
  \]
  \begin{myitemize}
    \item Αν $ n \neq m $, τότε
      \begin{align*}
        \int _{-\pi}^{\pi} \sin{(nx)} \sin{(mx)} \,{dx} 
            &=\frac{1}{2} \int_{-\pi}^{\pi} \left[\cos{[(n-m)x]} - \cos{[(n+m)x]}\right]
            \,{dx} 
            = \frac{1}{2} \left[\frac{\sin{[(n-m)x]}}{n-m} - 
            \frac{\sin{[(n+m)x]}}{n+m} \right]_{-\pi}^{\pi} \\
            &= \frac{1}{\cancel{2}} \cdot \cancel{2}  \left[{\frac{\sin{[(n-m)
                  \pi]}}{n-m}} - \frac{\sin{[(n+m)
            \pi]}}{n+m}\right] = 1 \cdot 0 = 0  
      \end{align*} 
      \begin{align*}
        \int _{-\pi}^{\pi} \cos{(nx)} \cos{(mx)} \,{dx} 
            &= \frac{1}{2} \int _{-\pi}^{\pi} \left[\cos{[(n-m)x]} + 
              \cos{[(n+m)x]} \right] \,{dx} 
              = \frac{1}{2} \left[\frac{\sin{[(n-m)x]}}{n-m} 
            + \frac{\sin{[(n+m)x]}}{n+m} \right]_{-\pi}^{\pi} \\
            &= \frac{1}{\cancel{2}} \cdot \cancel{2} \left[ 
              \frac{\sin{[(n-m) \pi]}}{n-m} + \frac{\sin{[(n+m) \pi]}}{n+m}\right] 
              = 1 \cdot 0 = 0  
      \end{align*}
      \begin{align*}
        \int _{-\pi}^{\pi} \sin{(nx)} \cos{(mx)} \,{dx} 
            &= \frac{1}{2} \int _{-\pi}^{\pi} \left[\sin{[(n-m)x]} + 
              \sin{[(n+m)x]} \right] \,{dx} 
              = \frac{1}{2} \left[-\frac{\cos{[(n-m)x]}}{n-m} - 
              \frac{\cos{[(n+m)x]}}{n+m} \right]_{-\pi}^{\pi} \\
            &=  - \frac{1}{2} \left[\cancel{\frac{\cos{[(n-m) \pi]}}{n-m}} + 
              \bcancel{\frac{\cos{[(n+m) \pi]}}{n+m}} - \cancel{\frac{\cos{[(n-m)
            (-\pi)]}}{n-m}} - \bcancel{\frac{\cos{[(n+m) (-\pi)]}}{n+m}} \right] 
            = - \frac{1}{2} \cdot 0 = 0  
      \end{align*}
    \item Αν $ n = m $, τότε
      \begin{align*}
        \int _{-\pi}^{\pi} \cos^{2}(nx) \,{dx} = 
        \int _{-\pi}^{\pi} \frac{1+ \cos{(2nx)}}{2}
        \,{dx} = 
        \frac{1}{2} \left[x + \frac{\sin{(2nx)}}{2n} \right]_{-\pi}^{\pi} = \pi
      \end{align*}
      \begin{align*}
        \int _{-\pi}^{\pi} \sin^{2}(2nx) \,{dx} = 
        \int _{-\pi}^{\pi} \frac{1- \cos{(2nx)}}{2}
        \,{dx} = 
        \frac{1}{2} \left[x - \frac{\sin{(2nx)}}{2n} \right]_{-\pi}^{\pi} = \pi
      \end{align*}
      \begin{align*}
        \int _{-\pi}^{\pi} \sin{(nx)} \cos{(nx)} \,{dx} 
            &= \frac{1}{2} \int _{-\pi}^{\pi} \sin{(2nx)} \,{dx} 
            = \frac{1}{2} \left[-\frac{\cos{(2nx)}}{2n}\right] _{-\pi}^{\pi} 
            =  - \frac{1}{4n} \cdot 0 = 0  
      \end{align*}
  \end{myitemize}
\end{proof} 

\begin{prop}
  Το σύστημα των πραγματικών συναρτήσεων 
  $ \{ 1, \sin{(nx)}, \cos{(nx)} \}_{n=1}^{\infty} $ 
  είναι ορθογώνιο στο $ \color{Col1}{[0, \pi]} $, ως προς τη συνάρτηση βάρους 
  $ w(x)=1 $.
\end{prop}
\begin{proof}
  Ομοίως, με τη διαφορά ότι αν $ n=m $, τότε 
  \begin{align*}
    \int _{0}^{\pi} \cos^{2}(nx) \,{dx} = 
    \int _{0}^{\pi} \frac{1+ \cos{(2nx)}}{2} \,{dx} = 
    \frac{1}{2} \left[x + \frac{\sin{(2nx)}}{2n} \right]_{0}^{\pi} = \frac{\pi}{2}
  \end{align*}
  \begin{align*}
    \int _{0}^{\pi} \sin^{2}(nx) \,{dx} = 
    \int _{0}^{\pi} \frac{1- \cos{(2nx)}}{2} \,{dx} = 
    \frac{1}{2} \left[x - \frac{\sin{(2nx)}}{2n} \right]_{-\pi}^{\pi} = \frac{\pi}{2}
  \end{align*}
\end{proof}


\section*{Ορθογωνιότητα των \ensuremath{\boldsymbol{\sin{(\frac{n \pi x}{L})}, \; 
\cos{(\frac{n \pi x}{L})}}}}

\begin{prop}
  Το σύστημα των πραγματικών συναρτήσεων 
  $ \{ 1, \sin{(\frac{n \pi x}{L})}, \cos{(\frac{n \pi x}{L})} \}_{n=1}^{\infty} $ 
  είναι ορθογώνιο στο $ \color{Col1}{[-L,L]} $, με $ L>0 $, ως προς τη συνάρτηση 
  βάρους $ w(x)=1 $.
\end{prop}
\begin{proof}
  \[
    \int _{-L}^{L} \sin{\left(\frac{n \pi x}{L}\right)} \,{dx} =  
    \left[\frac{- \cos{\left(\frac{n \pi x}{L}\right)}}{\frac{n \pi }{L}} 
    \right]_{-L}^{L} 
    = - \frac{L}{n \pi} \left[\cos{\left(\frac{n \pi x}{L}\right)} \right]_{-L}^{L} = 
    - \frac{L}{n \pi} [ \cos{(n \pi)} - \cos{(- n \pi)}] = - \frac{L}{n \pi } 
    [ \cos{(n \pi)}- \cos{(n \pi)}] = 0 
  \] 
  \[
    \int _{-L}^{L} \cos{\left(\frac{n \pi x}{L}\right)} \,{dx} =  
    \left[\frac{\sin{\left(\frac{n \pi x}{L}\right)}}{\frac{n \pi }{L}} \right]_{-L}^{L} 
    = \frac{L}{n \pi} \left[\sin{\left(\frac{n \pi x}{L}\right)} \right]_{-L}^{L} = 
    \frac{L}{n \pi} [ \sin{(n \pi)} - \sin{(- n \pi)}] =  \frac{2L}{n \pi } 
    \sin{(n \pi)} = 0 
  \]
  \begin{myitemize}
    \item Αν $ n \neq m $, τότε
      \begin{align*}
        \int _{-L}^{L} \sin{\left(\frac{n \pi x }{L}\right)} 
        &\sin{\left(\frac{m \pi x}{L}\right)} \,{dx} 
        = \frac{1}{2} \int _{-L}^{L} \left[\cos{\left(\frac{(n-m) \pi x}{L}\right)} 
        - \cos{\left(\frac{(n+m) \pi x}{L}\right)} \right] \,{dx} \\
        &= \frac{1}{2} \left[\frac{\sin{\left(\frac{(n-m) \pi x}{L}\right)}}
          {\frac{(n-m) \pi}{L}} - \frac{\sin{\left(\frac{(n+m) \pi x}{L}\right)}}
        {\frac{(n+m) \pi }{L}}\right]_{-L}^{L} 
        = \frac{L}{2 \pi} \left[\frac{\sin{\left(\frac{(n-m) \pi
          x}{L}\right)}}{n-m} - 
        \frac{\sin{\left(\frac{(n+m) \pi x}{L}\right)}}{n+m}\right]_{-L}^{L} \\ 
        &= \frac{L}{\cancel{2}\pi} \cdot \cancel{2}
        \left[ \frac{\sin{[(n-m) \pi]}}{n-m} - \frac{\sin{[(n+m)
        \pi]}}{n+m}\right] = \frac{L}{\pi} \cdot 0 = 0
      \end{align*} 
      \begin{align*}
        \int _{-L}^{L} \cos{\left(\frac{n \pi x }{L}\right)} 
        &\cos{\left(\frac{m \pi x}{L}\right)} \,{dx} 
        = \frac{1}{2} \int _{-L}^{L} \left[\cos{\left(\frac{(n-m) \pi x}{L}\right)} 
        + \cos{\left(\frac{(n+m) \pi x}{L}\right)} \right] \,{dx} \\
        &= \frac{1}{2} \left[\frac{\sin{\left(\frac{(n-m) \pi x}{L}\right)}}
          {\frac{(n-m) \pi}{L}} + \frac{\sin{\left(\frac{(n+m) \pi x}{L}\right)}}
        {\frac{(n+m) \pi }{L}}\right]_{-L}^{L} 
        = \frac{L}{2 \pi} \left[\frac{\sin{\left(\frac{(n-m) \pi
          x}{L}\right)}}{n-m} + 
        \frac{\sin{\left(\frac{(n+m) \pi x}{L}\right)}}{n+m}\right]_{-L}^{L} \\ 
        &= \frac{L}{\cancel{2}\pi} \cdot \cancel{2}
        \left[ \frac{\sin{[(n-m) \pi]}}{n-m} + \frac{\sin{[(n+m)
        \pi]}}{n+m}\right] = \frac{L}{\pi} \cdot 0 = 0
      \end{align*}
      \begin{align*}
        \int _{-L}^{L} \sin{\left(\frac{n \pi x }{L}\right)} 
        &\cos{\left(\frac{m \pi x}{L}\right)} \,{dx} 
        = \frac{1}{2} \int _{-L}^{L} \left[\sin{\left(\frac{(n+m) \pi x}{L}\right)} 
        + \sin{\left(\frac{(n-m) \pi x}{L}\right)} \right] \,{dx} \\
        &= \frac{1}{2} \left[-\frac{\cos{\left(\frac{(n+m) \pi x}{L}\right)}}
          {\frac{(n+m) \pi}{L}} - \frac{\left(\cos{\frac{(n-m) \pi x}{L}}\right)}
        {\frac{(n-m) \pi }{L}}\right]_{-L}^{L}
        =-\frac{L}{2 \pi} \left[\frac{\cos{\left(\frac{(n+m) \pi x}{L}\right)}}{n+m} + 
        \frac{\cos{\left(\frac{(n-m) \pi x}{L}\right)}}{n-m}\right]_{-L}^{L} 
        = -\frac{L}{2\pi}\cdot 0 = 0
      \end{align*}
    \item Αν $ n = m $, τότε
      \begin{align*}
        \int _{-L}^{L} \cos^{2}\left(\frac{n \pi x}{L}\right) \,{dx} = 
        \int _{-L}^{L} \frac{1+ \cos{\left( 2 \frac{n \pi x}{L} \right)}}{2}
        \,{dx} = 
        \frac{1}{2} \left[x + \frac{\sin{\left(2 \frac{n \pi x}{L}\right)}}
        {\frac{2 n \pi}{L}} \right]_{-L}^{L} = L
      \end{align*}
      \begin{align*}
        \int _{-L}^{L} \sin^{2}\left(\frac{n \pi x}{L}\right) \,{dx} = 
        \int _{-L}^{L} \frac{1- \cos{\left( 2 \frac{n \pi x}{L} \right)}}{2}
        \,{dx} = 
        \frac{1}{2} \left[x - \frac{\sin{\left(2 \frac{n \pi x}{L}\right)}}
        {\frac{2 n \pi}{L}} \right]_{-L}^{L} = L
      \end{align*}
      \begin{align*}
        \int _{-L}^{L} \sin{\left(\frac{n \pi x }{L}\right)} 
        \cos{\left(\frac{n \pi x}{L}\right)} \,{dx} 
            &= \frac{1}{2} \int _{-L}^{L} \sin{\left(\frac{2 n \pi x}{L}\right)} \,{dx} 
            = \frac{1}{2} \left[-\frac{\cos{\left(\frac{2 n \pi x}{L}\right)}}
            {\frac{2n \pi}{L}} \right]_{-L}^{L} 
            = -\frac{L}{4 n \pi} \left[\cos{\left(\frac{2n \pi x}{L}\right)} 
            \right]_{-L}^{L} = -\frac{L}{4n\pi}\cdot 0 = 0
      \end{align*}
  \end{myitemize}
\end{proof}
\begin{prop}
  Το σύστημα των πραγματικών συναρτήσεων 
  $ \{ 1, \sin{\left(\frac{n \pi x}{L}\right)}, 
  \cos{\left(\frac{n \pi x}{L}\right)} \}_{n=1}^{\infty} $ 
  είναι ορθογώνιο στο $ \color{Col1}{[0,L]} $, με $ L>0 $, ως προς τη συνάρτηση βάρους 
  $ w(x)=1 $.
\end{prop}
\begin{proof}
  Ομοίως, με τη διαφορά ότι αν $ n=m $, τότε 
  \begin{align*}
    \int _{0}^{L} \cos^{2}\left(\frac{n \pi x}{L}\right) \,{dx} = 
    \int _{0}^{L} \frac{1+ \cos{\left( 2 \frac{n \pi x}{L} \right)}}{2}
    \,{dx} = 
    \frac{1}{2} \left[x + \frac{\sin{\left(2 \frac{n \pi x}{L}\right)}}
    {\frac{2 n \pi}{L}} \right]_{0}^{L} = \frac{L}{2}
  \end{align*}
  \begin{align*}
    \int _{0}^{L} \sin^{2}\left(\frac{n \pi x}{L}\right) \,{dx} = 
    \int _{0}^{L} \frac{1- \cos{\left( 2 \frac{n \pi x}{L} \right)}}{2}
    \,{dx} = 
    \frac{1}{2} \left[x - \frac{\sin{\left(2 \frac{n \pi x}{L}\right)}}
    {\frac{2 n \pi}{L}} \right]_{-L}^{L} = \frac{L}{2}
  \end{align*}
\end{proof}

\section*{Ορθογωνιότητα άλλων συναρτήσεων}

\begin{prop}
  Το σύστημα των συναρτήσεων \textcolor{Col1}{Bessel 1ου είδους}
  $ \left\{ J_{k}\left(\frac{\rho _{k,n}}{a}x \right)\right\}_{n=1}^{\infty} $ 
  είναι ορθογώνιο στο διάστημα $ \color{Col1}{[0,a]} $, ως προς τη συνάρτηση βάρους 
  $ w(x)=x $.
  \[
    \int _{0}^{a} x J_{k}\left(\frac{\rho _{k,n}}{a}x \right)
    J_{k}\left(\frac{\rho_{k,m}}{a}x\right) \,{dx} = 
    \begin{cases}
      0, & n \neq m \\
      \frac{a^{2}}{2} J_{k+1}^{2}(\rho_{k,n}), & n=m
    \end{cases}
   \] 
  όπου οι συναρτήσεις Bessel 1ου είδους δίνονται από τον τύπο:
  \[
    J_{\kappa}(x) = \sum_{n=0}^{\infty} \frac{(-1)^{n}}{n!
    \Gamma(n+\kappa+1)}\left(\frac{x}{2}\right)^{2n+\kappa} 
  \]
\end{prop}

\begin{prop}
  Το σύστημα των πολυωνύμων \textcolor{Col1}{Legendre} $ \{P_{n}(x)\}_{n=1}^{\infty} $, 
  είναι ορθογώνιο στο διάστημα $ \textcolor{Col1}{[-1,1]} $, ως προς τη συνάρτηση 
  βάρους $ w(x)=1 $.
  \[
    \int _{-1}^{1} P_{n}(x)P_{m}(x) \,{dx} = 
    \begin{cases}  
      0, & n \neq m \\
      \frac{2}{2n+1}, & n=m
    \end{cases}
  \] 
  όπου τα πολυώνυμα Legendre δίνονται από τον τύπο:
  \[
    P_{n}(x) = \frac{1}{2^{n} n!} \dv[n]{}{x} {(x^{2}-1)^{n}}, 
    \quad n=0,1,2,\ldots
  \]
\end{prop}

\begin{prop}
  Το σύστημα των πολυωνύμων \textcolor{Col1}{Laquerre} $ \{L_{n}(t)\}_{n=1}^{\infty} $, 
  είναι ορθογώνιο στο διάστημα $ \textcolor{Col1}{[0,+\infty]} $, ως προς τη συνάρτηση 
  βάρους $ w(t)= \mathrm{e}^{-t} $.
  \[
    \int _{-1}^{1} \mathrm{e}^{-t}  L_{n}(t)L_{m}(t) \,{dt} = 
    \begin{cases}  
      0, & n \neq m \\
      1, & n=m
    \end{cases}
  \] 
  όπου τα πολυώνυμα Laquerre δίνονται από τον τύπο:
  \[
    L_{n}(t) = \frac{\mathrm{e}^{t}}{n!} \dv[n]{}{t}{(t^{n} \mathrm{e}^{-t})}  
    \quad n=0,1,2,\ldots
  \] 
\end{prop}
\end{document}
