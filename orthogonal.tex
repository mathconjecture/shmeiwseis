\documentclass[a4paper,12pt]{article}
\usepackage{etex}
%%%%%%%%%%%%%%%%%%%%%%%%%%%%%%%%%%%%%%
% Babel language package
\usepackage[english,greek]{babel}
% Inputenc font encoding
\usepackage[utf8]{inputenc}
%%%%%%%%%%%%%%%%%%%%%%%%%%%%%%%%%%%%%%

%%%%% math packages %%%%%%%%%%%%%%%%%%
\usepackage{amsmath}
\usepackage{amssymb}
\usepackage{amsfonts}
\usepackage{amsthm}
\usepackage{proof}

\usepackage{physics}

%%%%%%% symbols packages %%%%%%%%%%%%%%
\usepackage{dsfont}
\usepackage{stmaryrd}
%%%%%%%%%%%%%%%%%%%%%%%%%%%%%%%%%%%%%%%


%%%%%% graphicx %%%%%%%%%%%%%%%%%%%%%%%
\usepackage{graphicx}
\usepackage{color}
%\usepackage{xypic}
\usepackage[all]{xy}
\usepackage{calc}
%%%%%%%%%%%%%%%%%%%%%%%%%%%%%%%%%%%%%%%

\usepackage{enumerate}

\usepackage{fancyhdr}
%%%%% header and footer rule %%%%%%%%%
\setlength{\headheight}{14pt}
\renewcommand{\headrulewidth}{0pt}
\renewcommand{\footrulewidth}{0pt}
\fancypagestyle{plain}{\fancyhf{}
\fancyhead{}
\lfoot{}
\rfoot{\small \thepage}}
\fancypagestyle{vangelis}{\fancyhf{}
\rhead{\small \leftmark}
\lhead{\small }
\lfoot{}
\rfoot{\small \thepage}}
%%%%%%%%%%%%%%%%%%%%%%%%%%%%%%%%%%%%%%%

\usepackage{hyperref}
\usepackage{url}
%%%%%%% hyperref settings %%%%%%%%%%%%
\hypersetup{pdfpagemode=UseOutlines,hidelinks,
bookmarksopen=true,
pdfdisplaydoctitle=true,
pdfstartview=Fit,
unicode=true,
pdfpagelayout=OneColumn,
}
%%%%%%%%%%%%%%%%%%%%%%%%%%%%%%%%%%%%%%



\usepackage{geometry}
\geometry{left=25.63mm,right=25.63mm,top=36.25mm,bottom=36.25mm,footskip=24.16mm,headsep=24.16mm}

%\usepackage[explicit]{titlesec}
%%%%%% titlesec settings %%%%%%%%%%%%%
%\titleformat{\chapter}[block]{\LARGE\sc\bfseries}{\thechapter.}{1ex}{#1}
%\titlespacing*{\chapter}{0cm}{0cm}{36pt}[0ex]
%\titleformat{\section}[block]{\Large\bfseries}{\thesection.}{1ex}{#1}
%\titlespacing*{\section}{0cm}{34.56pt}{17.28pt}[0ex]
%\titleformat{\subsection}[block]{\large\bfseries{\thesubsection.}{1ex}{#1}
%\titlespacing*{\subsection}{0pt}{28.80pt}{14.40pt}[0ex]
%%%%%%%%%%%%%%%%%%%%%%%%%%%%%%%%%%%%%%

%%%%%%%%% My Theorems %%%%%%%%%%%%%%%%%%
\newtheorem{thm}{Θεώρημα}[section]
\newtheorem{cor}[thm]{Πόρισμα}
\newtheorem{lem}[thm]{λήμμα}
\theoremstyle{definition}
\newtheorem{dfn}{Ορισμός}[section]
\newtheorem{dfns}[dfn]{Ορισμοί}
\theoremstyle{remark}
\newtheorem{remark}{Παρατήρηση}[section]
\newtheorem{remarks}[remark]{Παρατηρήσεις}
%%%%%%%%%%%%%%%%%%%%%%%%%%%%%%%%%%%%%%%




\newcommand{\vect}[2]{(#1_1,\ldots, #1_#2)}
%%%%%%% nesting newcommands $$$$$$$$$$$$$$$$$$$
\newcommand{\function}[1]{\newcommand{\nvec}[2]{#1(##1_1,\ldots, ##1_##2)}}

\newcommand{\linode}[2]{#1_n(x)#2^{(n)}+#1_{n-1}(x)#2^{(n-1)}+\cdots +#1_0(x)#2=g(x)}

\newcommand{\vecoffun}[3]{#1_0(#2),\ldots ,#1_#3(#2)}







\begin{document}


\chapter{Ορθογωνιότητα}

\begin{dfn}
    Έστω δυο πραγματικές συναρτήσεις $ f(x), g(x) $ ορισμένες σ᾽ ένα διάστημα 
    $ [a,b] \subseteq \mathbb{R} $. Ορίζουμε ως \textcolor{Col1}{εσωτερικό γινόμενο} 
    των $ f(x) $ και $ g(x) $ και το συμβολίζουμε με $ <f,g> $ το ολοκλήρωμα
    \[
        <f,g> = \int _{a}^{b} w(x) f(x)g(x) \,{dx} 
    \] 
    όπου $ w(x) $ γνωστή θετική συνάρτηση, $ \forall x \in [a,b] $, η οποία 
    ονομάζεται \textcolor{Col1}{συνάρτηση βάρους}.
\end{dfn}

\begin{rem}
    Τα όρια ολοκλήρωσης στο παραπάνω ολοκλήρωμα μπορεί να είναι και $ \infty $.
\end{rem}

\begin{dfn}
    Δυο πραγματικές συναρτήσεις $ f(x), g(x) $ είναι \textcolor{Col1}{ορθογώνιες} 
    στο διάστημα $ [a,b] \subseteq \mathbb{R} $ ως προς τη συνάρτηση βάρους $ w(x) $, 
    αν το εσωτερικό του γινόμενο είναι μηδέν, δηλαδή:
    \[
        \int _{a}^{b} w(x)f(x)g(x) \,{dx} = 0
    \] 
\end{dfn}

\begin{dfn}
    Μια ακολουθία πραγματικών συναρτήσεων $ \{ f_{n}(x) \}_{n=1}^{\infty} $ αποτελεί 
    ένα σύστημα ορθογώνιων συναρτήσεων στο διάστημα $ [a,b] $, ως προς τη 
    συνάρτηση βάρους $ w(x) $, αν τα μέλη της είναι ανα δύο ορθογώνια, δηλαδή:
    \[
        \int _{a}^{b} w(x) f_{n}(x)f_{m}(x) \,{dx} = 0 \quad \forall n \neq m, \; n,m = 
        1,2,3, \ldots
    \] 
\end{dfn}

\begin{prop}
    Το σύστημα των συναρτήσεων $ \{ \cos{(nx)} \}_{n=1}^{\infty} $ είναι ορθογώνιο 
    στο διάστημα $ [-\pi, \pi] $, ως προς τη συνάρτηση βάρους $ w(x)=1 $.

    \begin{proof}
    \item {}
        \begin{myitemize}
        \item Αν $ n \neq m $, τότε:
            \begin{align*}
                \int _{-\pi}^{\pi} \cos{nx} \cdot \cos{mx} \,{dx} &= \frac{1}{2} 
                \int _{-\pi}^{\pi} [\cos{(nx+mx)} + \cos{(nx - mx)}] \,{dx} = 
                \frac{1}{2} 
                \left[\frac{\sin{(n+m)x}}{n+m} + \frac{\sin{(n-m)x}}{n-m}
                \right]_{-\pi}^{\pi} = 0 \\
            \end{align*} 

        \item Αν $ n = m $, τότε:
            \begin{align*}
                \int _{-\pi}^{\pi} \cos^{2}(nx) \,{dx} = 
                \int _{-\pi}^{\pi} \frac{1+ \cos{2nx}}{2}
                \,{dx} = 
                \frac{1}{2} \left[x + \frac{\sin{2nx}}{2n}\right]_{-\pi}^{\pi} = \pi 
            \end{align*}
        \end{myitemize}
    \end{proof}
\end{prop}


\begin{prop}
    Το σύστημα των συναρτήσεων $ \{ \sin{(nx)} \} _{n=1}^{\infty} $ είναι ορθογώνιο
    στο διάστημα $ [0, \pi] $, ως προς τη συνάρτηση βάρους $ w(x)=1 $.
\end{prop}

\begin{proof}
\item {}
    Ομοίως
\end{proof}

\begin{prop}
    Το σύστημα των συναρτήσεων Bessel 1ου είδους $ \{ J_{n}(x) \} _{n=1}^{\infty} $ 
    είναι ορθογώνιο στο διάστημα $ [0,a] $, ως προς τη συνάρτηση βάρους $ w(x)=x $.
\end{prop}

\begin{prop}
    Το σύστημα των πραγματικών συναρτήσεων 
    $ \{ 1, \sin{\frac{n \pi x}{L}, \cos{\frac{n \pi x}{L} } } \}_{n=1}^{\infty} $ 
    είναι ορθογώνιο στο $ [-L,L] $, με $ L>0 $, ως προς τη συνάρτηση βάρους 
    $ w(x)=1 $.
\end{prop}

\begin{proof}
\item {}
        \[
            \int _{-L}^{L} \sin{\frac{n \pi x}{L}} \,{dx} =  
            \left[\frac{- \cos{\frac{n \pi x}{L}}}{\frac{n \pi }{L}} \right]_{-L}^{L} 
            = - \frac{L}{n \pi} \left[\cos{\frac{n \pi x}{L}} \right]_{-L}^{L} = 
            - \frac{L}{n \pi} [ \cos{n \pi} - \cos{(- n \pi)}] = - \frac{L}{n \pi } 
            [ \cos{n \pi}- \cos{n \pi}] = 0 
        \] 

        \[
            \int _{-L}^{L} \cos{\frac{n \pi x}{L}} \,{dx} =  
            \left[\frac{\sin{\frac{n \pi x}{L}}}{\frac{n \pi }{L}} \right]_{-L}^{L} 
            = \frac{L}{n \pi} \left[\sin{\frac{n \pi x}{L}} \right]_{-L}^{L} = 
            \frac{L}{n \pi} [ \sin{n \pi} - \sin{(- n \pi)}] =  \frac{2L}{n \pi } 
            \sin{n \pi} = 0 
        \]

        \begin{align*}
            \int _{-L}^{L} \sin{\frac{n \pi x }{L}} \sin{\frac{m \pi x}{L}} \,{dx} 
            &= \frac{1}{2} \int _{-L}^{L} \left[\cos{\frac{(n-m) \pi x}{L}} - 
            \cos{\frac{(n+m) \pi x}{L}} \right] \,{dx} = 
            \frac{1}{2} \left[\frac{\sin{\frac{(n-m) \pi x}{L}}}
                {\frac{(n-m) \pi}{L}} - \frac{\sin{\frac{(n+m) \pi x}{L}}}
            {\frac{(n+m) \pi x}{L}}\right]_{-L}^{L} \\ 
            &= \frac{L}{2 \pi} \left[\frac{1}{n-m} \sin{\frac{(n-m) \pi x}{L} - 
            \frac{1}{n+m} \sin{\frac{(n+m) \pi x}{L}}}\right]_{-L}^{L} \\ 
            &= \frac{L}{2 \pi}
            [\frac{1}{n-m} \sin{(n-m) \pi} - \frac{1}{n+m} \sin{(n+m) \pi}] = 
\end{align*} 
\end{proof}

\end{document}
