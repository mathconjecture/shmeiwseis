\documentclass[a4paper,12pt]{article}
\usepackage{etex}
%%%%%%%%%%%%%%%%%%%%%%%%%%%%%%%%%%%%%%
% Babel language package
\usepackage[english,greek]{babel}
% Inputenc font encoding
\usepackage[utf8]{inputenc}
%%%%%%%%%%%%%%%%%%%%%%%%%%%%%%%%%%%%%%

%%%%% math packages %%%%%%%%%%%%%%%%%%
\usepackage{amsmath}
\usepackage{amssymb}
\usepackage{amsfonts}
\usepackage{amsthm}
\usepackage{proof}

\usepackage{physics}

%%%%%%% symbols packages %%%%%%%%%%%%%%
\usepackage{dsfont}
\usepackage{stmaryrd}
%%%%%%%%%%%%%%%%%%%%%%%%%%%%%%%%%%%%%%%


%%%%%% graphicx %%%%%%%%%%%%%%%%%%%%%%%
\usepackage{graphicx}
\usepackage{color}
%\usepackage{xypic}
\usepackage[all]{xy}
\usepackage{calc}
%%%%%%%%%%%%%%%%%%%%%%%%%%%%%%%%%%%%%%%

\usepackage{enumerate}

\usepackage{fancyhdr}
%%%%% header and footer rule %%%%%%%%%
\setlength{\headheight}{14pt}
\renewcommand{\headrulewidth}{0pt}
\renewcommand{\footrulewidth}{0pt}
\fancypagestyle{plain}{\fancyhf{}
\fancyhead{}
\lfoot{}
\rfoot{\small \thepage}}
\fancypagestyle{vangelis}{\fancyhf{}
\rhead{\small \leftmark}
\lhead{\small }
\lfoot{}
\rfoot{\small \thepage}}
%%%%%%%%%%%%%%%%%%%%%%%%%%%%%%%%%%%%%%%

\usepackage{hyperref}
\usepackage{url}
%%%%%%% hyperref settings %%%%%%%%%%%%
\hypersetup{pdfpagemode=UseOutlines,hidelinks,
bookmarksopen=true,
pdfdisplaydoctitle=true,
pdfstartview=Fit,
unicode=true,
pdfpagelayout=OneColumn,
}
%%%%%%%%%%%%%%%%%%%%%%%%%%%%%%%%%%%%%%



\usepackage{geometry}
\geometry{left=25.63mm,right=25.63mm,top=36.25mm,bottom=36.25mm,footskip=24.16mm,headsep=24.16mm}

%\usepackage[explicit]{titlesec}
%%%%%% titlesec settings %%%%%%%%%%%%%
%\titleformat{\chapter}[block]{\LARGE\sc\bfseries}{\thechapter.}{1ex}{#1}
%\titlespacing*{\chapter}{0cm}{0cm}{36pt}[0ex]
%\titleformat{\section}[block]{\Large\bfseries}{\thesection.}{1ex}{#1}
%\titlespacing*{\section}{0cm}{34.56pt}{17.28pt}[0ex]
%\titleformat{\subsection}[block]{\large\bfseries{\thesubsection.}{1ex}{#1}
%\titlespacing*{\subsection}{0pt}{28.80pt}{14.40pt}[0ex]
%%%%%%%%%%%%%%%%%%%%%%%%%%%%%%%%%%%%%%

%%%%%%%%% My Theorems %%%%%%%%%%%%%%%%%%
\newtheorem{thm}{Θεώρημα}[section]
\newtheorem{cor}[thm]{Πόρισμα}
\newtheorem{lem}[thm]{λήμμα}
\theoremstyle{definition}
\newtheorem{dfn}{Ορισμός}[section]
\newtheorem{dfns}[dfn]{Ορισμοί}
\theoremstyle{remark}
\newtheorem{remark}{Παρατήρηση}[section]
\newtheorem{remarks}[remark]{Παρατηρήσεις}
%%%%%%%%%%%%%%%%%%%%%%%%%%%%%%%%%%%%%%%




\newcommand{\vect}[2]{(#1_1,\ldots, #1_#2)}
%%%%%%% nesting newcommands $$$$$$$$$$$$$$$$$$$
\newcommand{\function}[1]{\newcommand{\nvec}[2]{#1(##1_1,\ldots, ##1_##2)}}

\newcommand{\linode}[2]{#1_n(x)#2^{(n)}+#1_{n-1}(x)#2^{(n-1)}+\cdots +#1_0(x)#2=g(x)}

\newcommand{\vecoffun}[3]{#1_0(#2),\ldots ,#1_#3(#2)}



\let\vec\mathbf

\pagestyle{vangelis}

\begin{document}

\chapter{Γραμμική Ανεξαρτησία, Βάση, Διάσταση}

\section{Γραμμική ανεξαρτησία}

\begin{dfn}
    Έστω $ V $ ένας διανυσματικός χώρος επί του σώματος $ \mathbb{K} $ και έστω 
    $ \mathbf{v} \in V $. Λέμε ότι το διάνυσμα $ \mathbf{v}$ είναι 
    \textcolor{Col2}{γραμμικός συνδυασμός} των διανυσμάτων 
    $ \mathbf{u_{1}}, \mathbf{u_{2}}, \ldots \mathbf{u}_{n} $, αν υπάρχουν 
    $ \lambda _{1}, \lambda _{2}, \ldots, \lambda _{n} \in \mathbb{K} $ τέτοιοι ώστε 
    \[
        \mathbf{v} = \lambda _{1} \mathbf{u_{1}}+ \lambda_{2} \mathbf{u_{2}}+ 
        \cdots \lambda _{k} \mathbf{u}_{k} \Leftrightarrow \mathbf{v} = 
        \sum_{i=1}^{k} \lambda _{i} \mathbf{u}_{i} 
    \]
    Τα στοιχεία $ \lambda _{1}, \lambda _{2}, \ldots, \lambda _{k} $ ονομάζονται 
    \textcolor{Col2}{συντελεστές} τους γραμμικού συνδυασμόυ.
\end{dfn}

\begin{thm}
    Έστω $ V $ ένας διανυσματικός χώρος επί του σώματος $ \mathbb{K} $ και έστω 
    $ S = \{ \mathbf{u_{1}}, \mathbf{u_{2}}, \ldots, \mathbf{u}_{k}\} \subseteq V $.
    Θέτουμε $ W $ να είναι το σύνολο όλων των γραμμικών συνδυασμών των στοιχείων του 
    $S$, με συντελεστές από το σώμα $ \mathbb{K} $.  Δηλαδή 
    \[
        W = \{ w \in V \; : \; \exists \lambda _{1}, \ldots, \lambda _{k} \in 
            \mathbb{K} \; \text{ώστε} \; \mathbf{w} = \lambda _{1} 
            \mathbf{u_{1}}+ \cdots \lambda _{k} \mathbf{u}_{k}\} 
    \] 
    Τότε
    \begin{enumerate}[i)]
        \item $ S \subseteq W $
        \item $ W \leq V $ 
        \item $ W $ είναι ο μικρότερος υπόχωρος του $V$ που περιέχει το $S$.
    \end{enumerate}
\end{thm}
\begin{proof}
\item {}
    \begin{enumerate}[i)]
        \item Πράγματι, έστω $i$ με $ 1 \leq i \leq k $. Τότε  έχουμε
            \[ \mathbf{u}_{i} = 0 \mathbf{u}_{1}+ 0 \mathbf{u_{2}} + 
                \cdots + 0 \mathbf{u}_{i-1} + 1 \mathbf{u}_{i} + 0 \mathbf{u}_{i+1} + 
            \cdots + \mathbf{u}_{k}  \]
        \item Καταρχάς $ \mathbf{0} \in W $, γιατί $ \mathbf{0} = 0 \mathbf{u}_{1} + 
            0\mathbf{u_{2}} + \cdots 0 \mathbf{u}_{k}$.

            Έστω 
            \[
                \left. 
                    \begin{matrix}
                        \mathbf{w_{1}} \in W \\
                        \mathbf{w_{2}} \in W 
                    \end{matrix}
                \right\}
                \Rightarrow 
                \left. 
                    \begin{matrix}
                        \exists \lambda _{1}, \ldots, \lambda _{k} \in \mathbb{K} \; 
                        \text{ώστε} \; \mathbf{w_{1}} = \comb{u}{k} \\
                        \exists \mu _{1}, \ldots, \mu _{k} \in \mathbb{K} \; 
                        \text{ώστε} \; \mathbf{w_{2}} = \combc{u}{\mu}{k}
                    \end{matrix}
                \right\}
                \Rightarrow 
            \]
            \[
                \mathbf{w}_{1} + \mathbf{w_{2}} = (\comb{u}{k}) + (\combc{u}{\mu}{k}) 
                = (\lambda _{1}+ \mu _{1}) \mathbf{u_{1}}+ \cdots + (\lambda _{k}+ 
                \mu _{k}) \mathbf{u}_{k} \in W 
            \] 

            Έστω $ \mathbf{w} \in W $ και $ \lambda \in \mathbb{K} $, τότε 
            $ \mathbf{w} = \comb{u}{k} $ και έχουμε
            \[ \lambda \mathbf{w} = \lambda (\comb{u}{k}) =
                \lambda (\lambda _{1} \mathbf{u_{1}}) + \cdots \lambda (\lambda _{k}) 
                \mathbf{u}_{k} = (\lambda \lambda _{1}) 
                \mathbf{u_{1}} + \cdots + (\lambda \lambda _{k}) \mathbf{u}_{k} \in W 
            \]
        \item Έστω $ U \leq V $, τέτοιος ώστε $ S \subseteq U $. Θα δείξουμε ότι 
            $ W \leq $. Πράγματι
            \[
                \mathbf{w} \in W \Rightarrow \exists \lambda _{1}, 
                \cdots \lambda _{k} \in \mathbb{K} \; \text{ώστε} \; \mathbf{w} = 
                \comb{u}{k} 
            \] 
            Όμως $ S \subseteq U \Rightarrow \mathbf{u}_{1}, \ldots \mathbf{u}_{k} 
            \in U \overset{U \leq V}{ \Rightarrow } \lambda _{1} \mathbf{u_{1}}, 
            \ldots, \lambda \mathbf{u}_{k} \in U \overset{U \leq V}{ \Rightarrow } 
            \mathbf{w} = \comb{u}{k} \in U $. Άρα $ W \leq U $.
    \end{enumerate}
\end{proof}

\begin{dfn}
    Έστω $ V $ ένας διανυσματικός χώρος επί του $ \mathbb{K} $ και έστω $ S = 
    \{ \mathbf{u}_{1}, \ldots, \mathbf{u}_{k} \} \subseteq V$. Τότε ο υπόχωρος $ W $ 
    του προηγούμενου θεωϱηματος ονομάζεται ο υπόχωρος του $V$ που 
    \textcolor{Col2}{παράγεται} από το $ S $, ή γραμμική θήκη του $S$ και 
    συμβολίζεται με $ W = < S > $ ή $ W = \Span \{ v_{1},\ldots,v_{n}  \}  $.  
    Λέμε επίσης ότι το σύνολο $S$ \textcolor{Col2}{παράγει} τον υπόχωρο $W$. 
    Επειδή $S$  πεπερασμένο λέμε ότι $W$ είναι \textcolor{Col2}{πεπερασμένα 
    γεννώμενος} υπόχωρος του  $V$ και τα στοιχεία 
    $ \mathbf{u_{1}}, \ldots, \mathbf{u_{n}} $ του $S$ λέγονται 
    \textcolor{Col2}{γεννήτορες} του $W$.
\end{dfn}

\begin{prop}
\item {}
    Αν $ S \subseteq V $ και $ W_{i}, \; i \in I $ 
    υπόχωροι του $V$ που περιέχουν το $S$, τότε $ < S > = \bigcap_{i \in I} W_{i} $. 
\end{prop}
\begin{proof}
\item {}
    Έστω $ S \subseteq W_{i} $. Τότε $ S \subseteq \bigcap_{i \in I} W_{i} $. Επειδή 
    η τομή $  \bigcap_{i \in I} W_{i} $ είναι υπόχωρος και $ < S >  $ είναι ο 
    μικρότερος υπόχωρος που περιέχει το $S$, έχουμε ότι 
    $ < S > \leq \bigcap_{i \in I} W_{i}   $. Όμως και $ < S > = W_{i}  $ για κάποιο 
    $i$, άρα και $ \bigcap_{i \in Ι} W_{i} \subseteq < S >  $, οπότε προκύπτει το 
    ζητούμενο.
\end{proof}

\begin{rem}
\item {}
    \begin{enumerate}
        \item Ορίζουμε $ < \emptyset > = \{ \mathbf{0} \}  $. 
        \item Ο παραπάνω ορισμός γενικεύεται και για άπειρα σύνολα $ S \subseteq V $. 
            Στην περίπτωση αυτή 
            \[ 
                < S > = \{ \mathbf{v} \in V \; : \; \exists \lambda _{1}, 
                    \ldots, \lambda _{k} \in \mathbb{K} \; \text{και} \; 
                    \mathbf{u_{1}}, \ldots \mathbf{u}_{k} \in V \; 
                    \text{ώστε} \; \mathbf{v} = \lambda _{1} \mathbf{u_{1}} + 
                \cdots + \lambda _{k} \mathbf{u}_{k}\}.
            \]
    \end{enumerate}
\end{rem}


\begin{examples}
\item {}
    \begin{enumerate}
        \item Έστω ο $ \mathbb{R} $-χώρος $ \mathbb{R}^{n} $ και για $ i= 1,\ldots,n $
            έστω $ \mathbf{e_{i}} $ το διάνυσμα του $ \mathbb{R}^{n} $ το οποίο έχει 
            1 στην $ i $-θέση και 0 στις υπόλοιπες. Τότε $ \mathbb{R}^{n} = 
            < \mathbf{e_{1}}, \mathbf{e_{2}}, \ldots \mathbf{e_{n}} >  $. 
            Πράγματι, έστω $ \mathbf{u} = (u_{1},u_{2},\ldots,u_{n}) \in 
            \mathbb{R}^{n} $. Τότε 
            \begin{align*}
                \mathbf{u} = (u_{1},u_{2},\ldots,u_{n}) 
                &=  u_{1} (1,0,\ldots,0) + u_{2} (0,1,\ldots,0) + 
                \cdots + u_{n} (0,0,\ldots,1) \\
                &= u_{1} \mathbf{e_{1}} + u_{2} \mathbf{e_{2}} + \cdots + u_{n} 
                \mathbf{e_{n}} \in < \mathbf{e_{1}}, \mathbf{e_{2}}, 
                \ldots, \mathbf{e_{n}} >  
            \end{align*} 
            Άρα $ \mathbb{R}^{n} = < \mathbf{e_{1}}, \mathbf{e_{2}}, 
            \ldots, \mathbf{e_{n}} >  $.
        \item Έστω $ A \in \textbf{M}_{m \times n}(\mathbb{R}) $ και έστω 
            $ \mathbf{r}_{1}, \ldots \mathbf{r}_{m} $ οι $m$ - γραμμές του $A$.  
            Ο υπόχωρος $ < \mathbf{r}_{1}, \ldots, \mathbf{r}_{m} >  $ του 
            $ \mathbb{R}^{n} $ που παράγεται από τις γραμμές του $A$ ονομάζεται 
            χώρος γραμμών του $A$ και συμβολίζεται με $ R(A) $ ή $ \Gamma_{A} $.
        \item Έστω $ A \in \textbf{M}_{m \times n}(\mathbb{R}) $ και έστω 
            $ \mathbf{c}_{1}, \ldots \mathbf{c}_{n} $ οι $n$ - στήλες του $A$.  
            Ο υπόχωρος $ < \mathbf{c}_{1}, \ldots, \mathbf{c}_{n} >  $ του 
            $ \mathbb{R}^{m} $ που παράγεται από τις στήλες του $A$ ονομάζεται 
            χώρος στήλών του $A$ και συμβολίζεται με $ C(A) $ ή $ \Sigma_{A} $.
        \item Έστω $ \mathbf{P_{n}}(\mathbb{R}) $ ο $ \mathbb{R} $ - χώρος των 
            πολυωνύμων βαθμού $ \leq n $ και έστω $ p_{0} = 1, p_{1}=x, p_{2}=x^{2}, 
            \ldots, p_{n}=x^{n} $. Τότε $ \mathbf{P_{n}}(\mathbb{R}) = 
            < p_{0}, p_{1}, \ldots, p_{n} > $.
    \end{enumerate}
\end{examples}

\begin{thm}
    Έστω $V$ ένας διανυσματικός χώρος επί του $ \mathbb{K} $ και έστω $ S, S' $ δύο 
    μη-κενά, πεπερασμένα υποσύνολα του $V$. Τότε 
    \[ < S > = < S' > \Leftrightarrow S \subseteq < S' >  \; \text{και} \; 
     S' \subseteq < S >  \]
\end{thm}
\begin{proof}
\item {}
    \begin{description}
        \item[($\Rightarrow$)] Έστω ότι $ < S > = < S' >   $. Γνωρίζουμε από το προηγούμενο θεώρημα
            ότι $ S \subseteq < S >  $ και $ S' \subseteq < S' >  $. Άρα από την 
            υπόθεση έχουμε ότι και $ S \subseteq < S' >  $ και $ S' \subseteq < S >  $.
        \item[($\Leftarrow$)] Έστω ότι $ S \subseteq < S' >  $ και $ S' \subseteq < S >  $. Επειδή $
            < S >  $ είναι ο μικρότερος υπόχωρος του $V$ που περιέχει το $S$ και 
            $ S \subseteq < S' >  $ έπεται ότι $ < S > \leq < S' >   $. Ομοίως 
            $ < S' > \leq < S >   $, οπότε $ < S > = < S' >   $.
             \end{description}
\end{proof}

\begin{rem}
    Από το θεώρημα καταλαβαίνουμε ότι είναι δυνατόν περισσότερα από ένα υποσύνολα του 
    $V$ να παϱάγουν τον ίδιο υπόχωρο του $V$. Για παράδειγμα, για τον 
    $ \mathbb{R} $- χώρο $ \mathbb{R} $ έχουμε ότι $ \mathbb{R} = < x > , 
    \; \forall x \in  \mathbb{R} \setminus \{ 0 \} $.
\end{rem}

\section{Γραμμική Ανεξαρτησία}

\begin{dfn}
    Έστω $ (V,+,\cdot) $ ένας $ \mathbb{K} $- χώρος και έστω 
    $ S = \{ \mathbf{v_{1}}, \ldots, \mathbf{v}_{n} \} \subseteq V $. Λέμε ότι 
    τα διανύσματα $ \mathbf{v_{1}}, \ldots, \mathbf{v_{n}} $ είναι 
    \textcolor{Col2}{γραμμικώς ανεξάρτητα} ή ότι το σύνολο $ S $ είναι 
    \textcolor{Col2}{γραμμικώς ανεξάρτητο}, αν οποτεδήποτε έχουμε
    \[
        \comb{v}{k} = \mathbf{0} \; \text{τότε} \; \lambda _{1} = 
        \lambda _{2} = \cdots = \lambda _{n} = 0
    \]
    Αν τα διανύσματα $ \mathbf{v_{1}}, \ldots, \mathbf{v_{n}} $ δεν είναι γραμμικώς 
    ανεξάρτητα τότε λέγονται \textcolor{Col2}{γραμμικώς εξαρτημένα} και ισχύει 
    ότι υπάρχει μη τετριμμένος γραμμικός συνδυασμός στοιχείων του $S$ που είναι 
    ίσος με $ \mathbf{0} $, δηλαδή
    \[
        \exists  \lambda _{1}, \ldots, \lambda _{k} \in \mathbb{K} \; 
        \text{όχι όλα μηδέν, ώστε} \; \comb{u}{k} = \mathbf{0}
    \]
\end{dfn}

\begin{examples}
\item {}
    \begin{enumerate}
        \item Στον $ \mathbb{R}^{2} $ δύο διανύσματα $ \mathbf{u} $ και $ \mathbf{v} $ 
            είναι γραμμικώς ανεξάρτητα αν και μόνον αν δεν είναι παράλληλα. Πράγματι 
            έστω ότι $ \mathbf{u}, \mathbf{v} $ είναι γραμμικώς εξαρτημένα. Τότε 
            \[
                \exists \lambda _{1}, \lambda _{2} \in \mathbb{R} \; 
                \text{όχι και τα δύο μηδέν, ώστε} 
                \; \lambda_{1} \mathbf{u} + \lambda_{2} \mathbf{v} = \mathbf{0} 
            \]
            Έτσι, αν έστω ότι $ \lambda_{1} \neq 0 $, έχουμε 
            \[
                \mathbf{u} = - \frac{\lambda _{2}}{\lambda _{1}} \mathbf{v} \quad  
                \text{(παράλληλα)}
            \] 
    \end{enumerate}
\end{examples}

\begin{prop}
   Το $ \emptyset $ είναι γραμμικώς ανεξάρτητο. 
\end{prop}
\begin{proof}
    Έστω ότι το $ \emptyset $ είναι γραμμικώς εξαρτημένο. Άρα υπάρχει μη τετριμμένος 
    γραμμικός συνδυασμός στοιχείων του $ \emptyset $ που να είναι 
    $ \mathbf{0} $. Άτοπο, γιατί το $ \emptyset $ δεν έχει στοιχεία.
\end{proof}

\begin{prop}
    Κάθε μη-μηδενικό διάνυσμα ένος $ \mathbb{K} $- χώρου $V$ είναι γραμμικώς 
    ανεξάρτητο.
\end{prop}
\begin{proof}
    Πράγματι. Έστω $ \mathbf{u} \in V $ και $ \lambda \in \mathbb{K} $ με 
    $ \lambda \mathbf{u} = \mathbf{0} $. Τότε από την πρόταση~\ref{prop:prod}  
    έχουμε $ \lambda = 0 $, αφού $ \mathbf{u} \neq \mathbf{0} $.
\end{proof}

\begin{prop}
    Έστω $V$ ένας $ \mathbb{K} $- χώρος και 
    $ S = \{ \mathbf{v_{1}}, \mathbf{v_{2}}, \ldots, \mathbf{v_{n}}  \} \subseteq V $.
    Αν $ \mathbf{0} \in S $ τότε το $S$ είναι γραμμικώς εξαρτημένο. Συγκεκριμένα 
    $ \{ \mathbf{0} \} $ είναι γραμμικώς εξαρτημένο.
\end{prop}
\begin{proof}
    Πράγματι, έστω ότι $ \mathbf{u}_{i} = \mathbf{0} $, για κάποιο $ i $ με 
    $ 1 \leq i \leq k $. Τότε
    \[
        0 \mathbf{u_{1}}+ \cdots + 0 \mathbf{u}_{i-1} + 1 \mathbf{u}_{i} + 0 
        \mathbf{u}_{i+1} + \cdots + 0 \mathbf{u}_{k} = \mathbf{0}  
     \]
     είναι ένας γραμμικός συνδυασμός στοιχειών του $S$ που είναι $ \mathbf{0} $, 
     χωρίς να μηδέν όλοι οι συντελεστές.
\end{proof}



\begin{examples}
\item {}
    \begin{enumerate}
        \item 
            Έστω $ \mathbb{R} $- χώρος $ \mathbb{R}^{n} $. Τότε οποιαδήποτε $ k $ 
            το πλήθος διανύσματα του $ \mathbb{R}^{n} $ με $ k >n $ είναι 
            γραμμικώς εξαρτημένα. Πράγματι έστω $ k>n $ και έστω $k$ διανύσματα του 
            $ \mathbb{R}^{n} $: 
            $ \mathbf{u}_{1} = (u_{11}, u_{12}, \ldots, v_{1n}), 
            \mathbf{u_{2}}=(u_{21},u_{22},\ldots,u_{2n}), \ldots, 
            \mathbf{u}_{k}= (u_{k1}, u_{k2}, \ldots, u_{kn}) $. Θα δείξουμε 
            ότι υπάρχουν 
            $ \lambda _{1}, \lambda _{2}, \ldots, \lambda _{k} $ όχι όλοι μηδέν, 
            ώστε
            \[ 
                \lambda _{1} \mathbf{u_{1}}+ \lambda _{2} \mathbf{u_{2}}+\cdots+ 
                \lambda _{k} \mathbf{u}_{k} = \mathbf{0} 
            \] 
            Πράγματι 
            \[
                \lambda_{1} 
                \begin{pmatrix*} 
                    u_{11} \\ u_{12} \\ \vdots \\ u_{1n} 
                \end{pmatrix*} + 
                \lambda _{2} 
                \begin{pmatrix*} 
                    u_{21} \\ u_{22} \\ \vdots \\ u_{2n} 
                \end{pmatrix*} + \cdots + 
                \lambda _{k} 
                \begin{pmatrix*} 
                    u_{k1} \\ u_{k2} \\ \vdots \\ u_{kn} 
                \end{pmatrix*} 
                = \begin{pmatrix*} 0 \\ 0 \\ \vdots \\ 0 \end{pmatrix*} 
            \]
            αν το παρακάτω $ n \times k $ ομογενές σύστημα με αγνώστους τα 
            $ \lambda _{1}, \lambda _{2}, \ldots, \lambda _{k} $ 
            \begin{equation*}
                % no need to use "align*" env.
                \setlength\arraycolsep{1.5pt} % default value: 5pt
                \left.
                    \begin{array}{ccc ccc c @{\extracolsep{2.5pt}}c
                        @{\extracolsep{2.5pt}}c}
                        u_{11} \lambda _{1} & + & u_{21} \lambda _{2} & + & 
                        \cdots & + & u_{k1}\lambda_{k} & = & 0 \\
                        u_{12}\lambda_{1} & + & u_{22}\lambda_{2} & + & \cdots & 
                        + & u_{k2}\lambda_{k} & = & 0 \\
                        \vdots & & \vdots & & \ddots & &  \vdots & &  \vdots \\
                        u_{1n}\lambda_{1} & + & u_{2n}\lambda_{2} & + & \cdots & 
                        + & u_{kn}\lambda_{k} & = & 0 \\
                    \end{array}
                \right\} _{n \times k}
            \end{equation*}    
            έχει τουλάχιστον μια μη μηδενική λύση, το οποίο προφανώς ισχύει 
            (από γνωστό θεώρημα), αφού λόγω ότι $ n<k $ το σύστημα έχει 
            περισσότερους αγνώστους από εξισώσεις και λόγω ότι είναι και 
            ομογενές θα έχει άπειρες λύσεις.
        \item Έστω ο $ \mathbb{R} $-χώρος $ \mathbb{R}^{n} $ και για $ i= 1,\ldots,n $
            έστω $ \mathbf{e_{i}} $ το διάνυσμα του $ \mathbb{R}^{n} $ το οποίο έχει 
            1 στην $ i $-θέση και 0 στις υπόλοιπες. Τότε τα διανύσματα 
            $ \mathbf{e_{1}}, \mathbf{e_{2}}, \ldots, \mathbf{e_{n}} $ είναι 
            γραμμικώς ανεξάρτητα. Πράγματι, έστω 
            $ \lambda _{1}, \lambda _{2}, \ldots, \lambda _{n} \in \mathbb{R} $ 
            τέτοια ώστε 
            \[
                \lambda _{1} \mathbf{e_{1}}+ \lambda _{2} \mathbf{e_{2}} 
                + \cdots + \lambda _{n} \mathbf{e_{n}}= \mathbf{0}  
            \]
            Τότε έχουμε
            \[
                \lambda _{1} 
                \begin{pmatrix*} 1 \\ 0 \\ \vdots \\ 0 \end{pmatrix*} 
                + \lambda _{2} 
                \begin{pmatrix*} 0 \\ 1 \\ \vdots \\ 0 \end{pmatrix*}
                + \cdots + \lambda _{n} 
                \begin{pmatrix*} 0 \\ 0 \\ \vdots \\ 1 \end{pmatrix*}
                = 
                \begin{pmatrix*} 0 \\ 0 \\ \vdots \\ 0 \end{pmatrix*} 
                \Leftrightarrow 
                \begin{pmatrix*} \lambda _{1} \\ \lambda _{2} \\ \vdots \\ 
                \lambda _{n} \end{pmatrix*} = 
                \begin{pmatrix*} 0 \\ 0 \\ \vdots \\ 0 \end{pmatrix*}
                \Leftrightarrow 
                \lambda _{1} = \lambda _{2} = \dots = \lambda _{n} = 0 
            \]
            Δηλαδή τα διανύσματα $ \mathbf{e_{1}}, \mathbf{e_{2}}, \ldots, 
        \mathbf{e_{n}} $ είναι γραμμικώς ανεξάρτητα.
        \item Έστω $ \mathbb{R} $- χώρος $ \mathbf{P_{n}}(\mathbb{R}), \; n 
            \in \mathbb{N} $. Τότε το σύνολο $ \{1,x,x^{2}, \ldots,x^{n} \} $ 
            είναι γραμμικώς ανεξάρτητο. Πράγματι, έστω (προς άτοπο) ότι το 
            $S$ είναι γραμμικώς εξαρτημένο. Τότε υπάρχουν 
            $ \lambda _{0}, \lambda _{1}, \ldots, \lambda _{n} \in \mathbb{R} $, 
            όχι όλοι μηδέν, ώστε
            \[
                \lambda _{0} \cdot 1 + \lambda _{1} \cdot x + \lambda _{2} 
                \cdot x^{2} + \cdots + \lambda _{n} \cdot x^{n} = \mathbf{0}, \; 
                \forall x \in \mathbb{R} \quad \text{(ισότητα πολυωννύμων)}   
            \]
            Δηλαδή, κάθε πραγματικός αριθμός είναι ρίζα του μη μηδενικού πολυωνύμου 
            $ \lambda _{0} \cdot 1 + \lambda _{1} \cdot x + \lambda _{2} 
            \cdot x^{2} + \cdots + \lambda _{n} \cdot x^{n} $. Άτοπο, γιατί από 
            το θεμελιώδες θεώρημα της Άλγεβρας, το παραπάνω πολυώνυμο έχει το 
            πολύ $ n $ το πλήθος διακεκριμένες ρίζες.
    \end{enumerate}
\end{examples}

\begin{thm}\label{thm:s1s2depend}
    Έστω $V$ ένας $ \mathbb{K} $- χώρος και έστω 
    $ S = \{ \mathbf{u}_{1}, \ldots, \mathbf{u_{n}} \} $ μη μηδενικό σύνολο 
    διανυσμάτων του $V$. Τότε το $S$ είναι γραμμικώς εξαρτημένο αν και μόνον αν 
    κάποιο από τα διανύσματα του $S$ είναι γραμμικώς συνδυασμός των υπολοίπων. 
\end{thm}
\begin{proof}
\item {}
    \begin{description}
        \item [($ \Rightarrow $)] Έστω ότι το $S$ είναι γραμμικώς εξαρτημένο. Τότε 
            υπάρχουν $ \lambda _{1}, \ldots, \lambda_{n} \in \mathbb{K} $ όχι 
            όλα μηδέν, ώστε 
            \[
                \lambda _{1} \mathbf{u_{1}} + \cdots + \lambda _{n} \mathbf{u}_{n} = 
                \mathbf{0}  
            \] 
            Έστω ότι $ \lambda _{i} \neq 0 $ με $ 1 \leq i \leq n $. Τότε έχουμε
            \begin{align*}
                \lambda _{1} \mathbf{u}_{1} + \cdots + \lambda _{i-1} 
                \mathbf{u}_{i-1} + \lambda _{i} \mathbf{u}_{i} + \lambda _{i+1} 
                \mathbf{u}_{i+1} + \cdots + \lambda _{n}
                \mathbf{u}_{n} = \mathbf{0} \Leftrightarrow  \\
                \mathbf{u}_{i} = - \frac{\lambda _{1}}{\lambda _{i}} \mathbf{u}_{1} +
                \cdots - \frac{\lambda _{i-1}}{\lambda _{i}} \mathbf{u}_{i-1} - 
                \frac{\lambda _{i+1}}{\lambda _{i}} \mathbf{u}_{i+1} - 
                \cdots - \frac{\lambda _{n}}{\lambda _{i}} \mathbf{u}_{n} 
             \end{align*} 
             Δηλαδή το διάνυσμα $ \mathbf{u}_{i} $ είναι γραμμικός συνδυασμός των 
             υπολόιπων.
         \item [($\Leftarrow$)] 
             Έστω ότι για κάποιο $ i $ με $ 1 \leq i \leq n $ το διάνυσμα 
             $ \mathbf{u}_{i} $ είναι γραμμικός συνδυασμός των υπολοίπων διανυσμάτων.
             Άρα υπάρχουν $ \lambda _{1}, \ldots, \lambda _{n} \in \mathbb{K} $ 
             ώστε 
             \begin{gather*}
                 \mathbf{u}_{i} = \lambda _{1} \mathbf{u}_{1} + \cdots + 
                 \lambda _{n} \mathbf{u}_{n} \Leftrightarrow \\
                 \mathbf{u}_{i} = \lambda _{1} \mathbf{u_{1}} + \cdots + 
                 \lambda _{i-1}
                 \mathbf{u}_{i-1} + \lambda _{i+1} \mathbf{u}_{i+1} + \cdots + 
                 \lambda _{n} \mathbf{u}_{n} \Leftrightarrow \\
                 \lambda _{1} \mathbf{u_{1}} + \cdots + \lambda _{i-1}
                 \mathbf{u}_{i-1} + (-1) \cdot \mathbf{u}_{i} + \lambda _{i+1} 
                 \mathbf{u}_{i+1} + \cdots + \lambda _{n} \mathbf{u}_{n} = \mathbf{0}
             \end{gather*}
             Επομένως υπάρχει μη τετριμμένος γραμμικός συνδυασμός των στοιχειών του 
             $S$ που ειναι ο μηδενικός, άρα το $S$ είναι γραμμικώς εξαρτημένο.
    \end{description}
\end{proof}


\begin{thm}
    Έστω $V$ ένας $ \mathbb{K} $- χώρος και έστω $ S_{1}, S_{2} $ δύο μη κενά, 
    πεπερασμένα υποσύνολα του $V$ με $ S_{1} \subseteq S_{2} $. 
    Τότε, αν $ S_{1} $ είναι γραμμικώς εξαρτημένο, τότε και το $ S_{2} $ 
    είναι γραμμικώς εξαρτημένο.
\end{thm}
\begin{proof}
    Έστω $ S_{1} $ γραμμικώς εξαρτημένο και έστω ότι $ S_{1} = \{ \mathbf{u_{1}},
        \ldots, \mathbf{u}_{k} \} $ και έστω $ S_{2} = \{ \mathbf{u_{1}}, \ldots, 
    \mathbf{u}_{k}, \mathbf{u}_{k+1}, \ldots, \mathbf{u}_{n} \} $. 
    Εφόσον το $ S_{1} $ ειναι γραμμικώς εξαρτημένο, τότε για κάποιο $ i $ με 
    $ 1 \leq i \leq k $ το διάνυσμα $ \mathbf{u}_{i} $ γράφεται 
    ως γραμμικός συνδυασμός των υπολοίπων, δηλαδή υπάρχουν $ \lambda _{1}, \ldots, 
    \lambda _{k} \in \mathbb{K} $ ώστε
    \[
        \mathbf{u}_{i} = \lambda _{1} \mathbf{u_{1}}+ \cdots \lambda _{i-1} 
        \mathbf{u}_{i-1} + \lambda _{i+1} \mathbf{u}_{i+1} + \cdots + 
        \lambda _{k} \mathbf{u}_{k}
    \] 
    Όμως τότε έχουμε 
    \[
        \mathbf{u}_{i} = \lambda _{1} \mathbf{u_{1}}+ \cdots \lambda _{i-1} 
        \mathbf{u}_{i-1} + \lambda _{i+1} \mathbf{u}_{i+1} + \cdots + 
        \lambda _{k} \mathbf{u}_{k} + 0 \mathbf{u}_{k+1} + \cdots + 0 \mathbf{u}_{n}
    \] 
    Δηλαδή το $ S_{2} $ είναι γραμμικώς εξαρτημένο.
\end{proof}

\section{Βάση και Διάσταση Διανυσματικού Χώρου}

\begin{dfn}
    Έστω $ V $ ένας $ \mathbb{K} $- χώρος πεπερασμένης διάστασης.

Ένα υποσύνολο $ B = \{ \mathbf{u}_{1} , \ldots, \mathbf{u}_{n} \} \subseteq V $ 
ονομάζεται  {\color {Col2} $ \mathbb{K} $- βάση} του $V$, ή απλώς 
\textcolor{Col2}{βάση} του $V$, αν 
\begin{enumerate}[i)]
    \item Το σύνολο $B$ παράγει το χώρο $V$, δηλαδή $ V = < B >  $ 
    \item Το $B$ είναι γραμμικώς ανεξάρτητο υποσύνολο του $V$
\end{enumerate}
\end{dfn}

\begin{rems}
\item {}
    \begin{enumerate}
        \item Αν $B$ είναι μια βάση ενός διανυσματικού χώρου $V$ τότε 
            $ \mathbf{0}_{V} \not \in B $, γιατί αλλιώς το $B$ θα ήταν γραμμικώς 
            εξαρτημένο.
        \item $ \emptyset $ είναι βάση του τετριμμένου χώρου $ \{ \mathbf{0}_{V} \} $.
    \end{enumerate}
\end{rems}

\begin{examples}
\item {}
    \begin{enumerate}
        \item Έστω ο $ \mathbb{R} $- χώρος $ \mathbb{R}^{n} $. Τότε το σύνολο 
            $ B = \{ \mathbf{e_{1}}, \ldots, \mathbf{e_{n}} \} $ είναι βάση του και 
            ονομάζεται η συνήθης ή η κανονική βάση του $ \mathbb{R}^{n} $.
        \item Έστω ο $ \mathbb{R} $- χώρος $ \mathbf{P_{n}}(\mathbb{R}) $. Τότε το 
            σύνολο $ B = \{ 1, x, x^{2}, \ldots, x^{n} \} $ είναι βάση του και 
            ονομάζεται η συνήθης ή η κανονική βάση του $ \mathbf{P_{n}}(\mathbb{R}) $.
        \item Έστω ο $ \mathbb{R} $- χώρος $ \textbf{M}_{m \times n}(\mathbb{R}) $. 
            Τότε το σύνολο 
            $ B = \{ E_{ij} \in \textbf{M}_{m \times n}(\mathbb{R}) \; : \; 1 \leq i
            \leq m \; \text{και} \; 1 \leq j \leq n\}  $, όπου $ E _{ij} $ είναι ο 
            $ m \times n $ πίνακας ο οποίος έχει 1 στην $ ij $- θέση και 0 στις άλλες 
            είναι βάση του και ονομάζεται η συνήθης ή η κανονική βάση του 
            $ \textbf{M}_{m \times n}(\mathbb{R}) $.
    \end{enumerate}
\end{examples}

\begin{thm}[Χαρακτηρισμός των Βάσεων]
\item {}
    Έστω $V$ ένας $ \mathbb{K} $- χώρος πεπερασμένης διάστασης και έστω 
    $ B = \{ \mathbf{u_{1}}, \ldots, \mathbf{u}_{n} \} $ ένα υποσύνολο του V. Τότε 
    το $B$ είναι βάση του $V$ αν και μόνον αν κάθε διάνυσμα του $V$ γράφεται 
    κατά μοναδικό τϱόπο ως γραμμικός συνδυασμός των στοιχείων του $B$.
\end{thm}
\begin{proof}
\item {}
    \begin{description}
        \item [($ \Rightarrow $)]
            Έστω ότι το σύνολο $B$ είναι βάση του $V$ και έστω ότι για το 
            τυχαίο διάνυσμα $ \mathbf{v} \in V $, έχουμε ότι 
            \begin{align*} 
                \mathbf{v} &= \lambda _{1} \mathbf{u_{1}}+ \cdots + \lambda _{n} 
                \mathbf{u}_{n}, \quad \text{για κάποια} \;  \lambda _{1}, \ldots, 
                \lambda _{n} \in \mathbb{K}  \\
                \mathbf{v} &= \mu_{1} \mathbf{u_{1}}+ \cdots + \mu _{n} 
                \mathbf{u}_{n}, \quad \text{για κάποια} \;  \mu _{1}, \ldots, 
                \mu _{k} \in \mathbb{K}
            \end{align*}
            Τότε 
            \begin{gather*}
                \lambda _{1} \mathbf{u_{1}}+ \cdots + \lambda _{n} \mathbf{u}_{n} = 
                \mu_{1} \mathbf{u_{1}}+ \cdots + \mu _{n} \mathbf{u}_{n} 
                \Leftrightarrow \\
                (\lambda _{1}- \mu _{1}) \mathbf{u}_{1} + \cdots + (\lambda _{n} - 
                \mu _{n}) \mathbf{u}_{n} = \mathbf{0}
            \end{gather*}
            Όμως το σύνολο $B$ είναι βάση του $V$, άρα γραμμικώς ανεξάρτητο, άρα 
            \[
                \lambda _{1} = \mu _{1}, \ldots, \lambda _{n} = \mu _{n} 
            \] 
            και επομένως ο γραμμικός συνδυασμός είναι μοναδικός.
        \item [($ \Leftarrow $)] 
            Έστω ότι κάθε $ \mathbf{v} \in V $ γράφεται ως γραμμικώς συνδυασμός 
            στοιχείων του $B$ κατά μοναδικό τρόπο. Άρα το $B$ παράγει το χώρο $V$. 
            Το $B$ είναι επίσης γραμμικώς ανεξάρτητο υποσύνολο του $V$, γιατί 
            \[
                \comb{\mathbf{e}}{n} \mathbf{0} \Rightarrow \lambda _{i} = 0, \; 
                \forall i=1,\ldots, n
            \] 
            αφού επίσης $ \mathbf{0} = 0 \mathbf{e_{1}}+\cdots + 0 \mathbf{e}_{n} $ 
            και από υπόθεση το $ \mathbf{0} $ γράφεται με μοναδικό τρόπο ως γραμμικός 
            συνδυασμός στοιχείων του $B$.
    \end{description} 
\end{proof}

\begin{rem}
    Αν $B = \{ \mathbf{u}_{1}, \ldots, \mathbf{u_{n}} \} $ είναι μια διατεταγμένη 
    βάση ενός $ \mathbb{K} $ - χώρου $V$, τότε από το προηγούμενο θεώρημα έχουμε ότι 
    $ \forall \mathbf{u} \in V, \; \exists $ μοναδικό διάνυσμα 
    $ (u_{1}, \ldots, u_{n}) \in \mathbb{K}^{n} $ τέτοιο ώστε 
    $ \mathbf{u} = u_{1} \mathbf{u_{1}} + \cdots u_{n} \mathbf{u}_{n} $. 
    Το διάνυσμα $ (u_{1}, \ldots, u_{n}) \in \mathbb{K}^{n} $ λέγεται 
    \textcolor{Col2}{διάνυσμα συντεταγμένων} του $ \mathbf{u} $ και γράφουμε 
    \[ 
        [\mathbf{u}]_{B} = (u_{1}, \ldots, u_{n})
    \] 
    και  
    οι συντελεστές $ u_{1}, \ldots, u_{n} \in \mathbb{K} $ λέγονται 
    \textcolor{Col2}{συντεταγμένες} του διανύσματος $ \mathbf{u} $ ως προς τη βάση $B$.
\end{rem}

\begin{lem}
    Έστω $V$ ένας $ \mathbb{K} $- χώρος πεπερασμένης διάστασης και έστω 
    $ S = \{ \mathbf{u_{1}}, \ldots, \mathbf{u_{n}} \} $ ένα σύνολο γεννητόρων του 
    $V$. Τότε 
    \begin{enumerate}[i)]
        \item Αν $ \mathbf{w} \in V $ τότε το σύνολο $ S \cup \{ \mathbf{w} \} $ είναι 
            γραμμικώς εξαρτημένο και ισχύει $ V = < S \cup \{ \mathbf{w} \} >  $.
        \item Αν για κάποιο $ i $ με $ 1 \leq i \leq n $ το διάνυσμα 
            $ \mathbf{u}_{i} $ είναι γραμμικός συνδυασμός των υπολοίπων διανυσμάτων 
            του $S$, τότε το σύνολο $ S - \{ \mathbf{u}_{i} \} $ παράγει τον $V$. 
    \end{enumerate}
\end{lem}
\begin{proof}
\item {}
    \begin{enumerate}[i)]
        \item Αφού $S$ είναι ένα σύνολο γεννητόρων του $V$, έχουμε ότι 
            $ V = < S >  $.
            Άρα $ \mathbf{w} \in V \Rightarrow \mathbf{w} \in < S >  $, άρα το 
            $ \mathbf{w} $ είναι γραμμικός συνδυασμός των στοιχείων του $S$, 
            άρα σύμφωνα με το θεώρημα~\ref{thm:s1s2depend}, το σύνολο 
            $ S \cup \{ \mathbf{w} \} $ θα είναι γραμμικώς εξαρτημένο.
            Έστω τώρα  $ \mathbf{v} \in V = < S >  $. Τότε υπάρχουν 
            $ \lambda _{1}, \ldots, \lambda _{n} \in \mathbb{K} $ ώστε 
            $ \mathbf{v} = \lambda _{1} \mathbf{u}_{1}+ \cdots + \lambda _{n} 
            \mathbf{u}_{n} = \lambda _{1} \mathbf{u}_{1}+ \cdots + \lambda _{n} 
            \mathbf{u}_{n} + 0 \mathbf{w} \in < S  \cup \{ \mathbf{w}  \} > $. Άρα 
            $ V = < S \cup \{ \mathbf{w} \} >  $.
        \item Έστω ότι για κάποιο $ i $ με $ 1 \leq i \leq n $ υπάρχουν 
            $ \lambda _{1}, \ldots, \lambda _{n} \in \mathbb{K} $ ώστε 
            \begin{equation*}
                \label{eq:combi}
                \mathbf{u}_{i} = \lambda _{1} \mathbf{u_{1}} + \cdots \lambda _{i-1}
                \mathbf{u}_{i-1} + \lambda _{i+1} \mathbf{u}_{i+1} + \cdots + 
                \lambda _{n} \mathbf{u}_{n} 
            \end{equation*}
            Έστω $ \mathbf{v} \in V = < S > $. Τότε υπάρχουν $ \mu _{1}, \ldots, 
            \mu _{n} \in \mathbb{K} $ ώστε 
            \begin{align*} 
                \mathbf{v} &= 
                \mu _{1} \mathbf{u}_{1} + \cdots + \mu _{i} \mathbf{u}_{i} + \cdots + 
                \mu _{n} \mathbf{u}_{n} \\ 
                           &\overset{\ref{eq:combi}}{=} 
                           \mu _{1} \mathbf{u}_{1} + \cdots + \mu _{i} (\lambda _{1} 
                           \mathbf{u_{1}} + \cdots \lambda _{i-1}
                           \mathbf{u}_{i-1} + \lambda _{i+1} \mathbf{u}_{i+1} + 
                           \cdots + \lambda _{n} \mathbf{u}_{n}) + \cdots + \mu _{n} 
                           \mathbf{u}_{n} \\
                           &=(\mu _{1}+ \mu _{i}\lambda_{1}) \mathbf{u}_{1} + \cdots + 
                           (\mu _{i} \lambda _{i-1}) \mathbf{u}_{i-1} + (\mu _{i} 
                           \lambda _{i+1}) \mathbf{u}_{i+1} + \cdots + (\mu _{n} + 
                           \mu _{i} \lambda _{n}) \mathbf{u}_{n} \in 
                           < S - \{ \mathbf{u}_{i} \} >  
            \end{align*}
            Άρα $ v = < S - \{ \mathbf{u}_{i} \} >  $
    \end{enumerate}
\end{proof}

\begin{cor}
    Έστω $V$ ένας $ \mathbb{K} $- χώρος και έστω $ S_{1}, S_{2} $ μή κενά, 
    πεπερασμένα υποσύνολα του $V$, με $ S_{1} \subseteq S_{2} $. Τότε αν 
    $ S_{2} $ είναι γραμμικώς ανεξάρτητο τότε και το $ S_{1} $ είναι γραμμικώς 
    ανεξάρτητο.
\end{cor}

\begin{prop}
    Αν το σύνολο $ S = \{ \mathbf{u}_{1}, \ldots, \mathbf{u_{n}} \} $ παράγει 
    το χώρο $V$, τότε κάποιο υποσύνολο του $S$ είναι βάση του $V$.
\end{prop}

\begin{thm}
    Έστω $ B = \{ \mathbf{e_{1}}, \ldots, \mathbf{e_{n}} \} $ μια βάση του $V$. 
    Τότε 
    \begin{enumerate}[i)]
        \item Οποιαδήποτε $ n+1 $ στοιχεία του $V$ είναι γραμμικώς εξαρτημένα.
        \item Όλες οι βάσεις του χώρου $V$ έχουν το ίδιο πλήθος στοιχείων.
    \end{enumerate}
\end{thm}

\begin{dfn}
    Όπως είδαμε, αν ένας $ \mathbb{K} $ - χώρος $V$ έχει μια βάση με 
    $ n $ στοιχεία, τότε και κάθε άλλη βάση του, θα έχει επίσης $n$ στοιχεία. 
    Ο αριθμός $n$ λέγεται \textcolor{Col2}{διάσταση} του χώρου $V$ επί του 
    $ \mathbb{K} $ και γράφουμε $ \dim V = n $.
\end{dfn}

\begin{prop}
\item {}
    Έστω $V$ ένας $ \mathbb{K} $ - χώρος διάστασης $n$. Τότε
    \begin{enumerate}[i)]
        \item $n$ γραμμικώς ανεξάρτητα στοιχεία του $V$, αποτελούν βάση αυτού.
        \item Οποιαδήποτε $ n-1 $ στοιχεία του $V$ δεν παράγουν τον χώρο $V$.
        \item Οποιαδήποτε $ m < n $ γραμμικώς ανεξάρτητα στοιχεία του $V$, 
            συμπληρώνονται με $ n-m $ άλλα στοιχεία, ώστε να προκύψει μια 
            βάση του χώρου.
    \end{enumerate}
\end{prop}

\begin{prop}
    Έστω $V$ ένας διανυσματικός χώρος πεπερασμένης διάστασης. Έστω $ W $ ένας 
    υπόχωρος του $V$ με βάση $ \{ \mathbf{e_{1}}, \ldots, \mathbf{e_{m}} \} $. 
    Τότε υπάρχει βάση του χώρου $ V $ της μορφής 
    $ \{ \mathbf{e_{1}}, \ldots, \mathbf{e_{m}}, \mathbf{e_{m+1}}, \ldots, 
    \mathbf{e_{n}} \} $, με $ n \geq m $.
\end{prop}
\begin{proof}

\end{proof}

\begin{cor}
    Αν $W$ υπόχωρος του $V$ και $ \dim W = dim V $, τότε $ W=v $.
\end{cor}
\begin{proof}
    Γιατί, αν $ \{ \mathbf{e_{1}}, \ldots, \mathbf{e_{m}}\}  $ είναι βάση του 
    $ W $, θα είναι και βάση του $V$. Έτσι ο 
    $ V = < \mathbf{e_{1}}, \ldots, \mathbf{e}_{m} > = W$.
\end{proof}

\begin{prop}
    Εστω $ A = (a_{ij}) \in \textbf{M}_{n}(\mathbb{R}) $. Τότε 
    \begin{enumerate}[i)]
        \item Οι στήλες $ \mathbf{c}_{1}, \ldots, \mathbf{c}_{n} $ είναι γραμμικώς 
            εξαρτημένα διανύσματα του $ \mathbb{R}^{n} $ αν $ \abs{A} = 0 $.
        \item Οι γραμμές $ \mathbf{r}_{1}, \ldots, \mathbf{r}_{n} $ είναι γραμμικώς 
            εξαρτημένα διανύσματα του $ \mathbb{R}^{n} $ αν $ \abs{A} = 0 $.
    \end{enumerate}
\end{prop}
\begin{proof}

\end{proof}

\begin{rem}
    Από το θεώρημα έπεται ότι για να ελέγξουμε αν $ n $ το πλήθος διανύσματα 
    του $ \mathbb{R}^{n} $ είναι γραμμικώς ανεξάρτητα, τότε αρκεί να ελέγξω την 
    ορίζουσα του πίνακα $A$ που έχει τα διανύσματα αυτά ως στήλες ή ως γραμμές. 
    Αν $ \abs{A} = 0 $, τότε θα είναι γραμμικώς εξαρτημένα.
\end{rem}

\end{document}
