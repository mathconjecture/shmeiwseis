\input{preamble.tex}
\newcommand{\vect}[2]{(#1_1,\ldots, #1_#2)}
%%%%%%% nesting newcommands $$$$$$$$$$$$$$$$$$$
\newcommand{\function}[1]{\newcommand{\nvec}[2]{#1(##1_1,\ldots, ##1_##2)}}

\newcommand{\linode}[2]{#1_n(x)#2^{(n)}+#1_{n-1}(x)#2^{(n-1)}+\cdots +#1_0(x)#2=g(x)}

\newcommand{\vecoffun}[3]{#1_0(#2),\ldots ,#1_#3(#2)}



\begin{document}



\chapter{Ακρότατα Υπό Συνθήκη, Δεσμευμένα Ακρότατα}

\section{Πολλαπλασιαστές Lagrange}

\subsection{Ακρότατα υπό μία συνθήκη}

\begin{thm}
    Έστω $ f(x,y,z) $ συνάρτηση τριών μεταβλητών, ορισμένη σε ένα ανοιχτό 
    υποσύνολο $A$ του $ \mathbb{R}^{3} $ και έστω ότι οι παράγωγοι 2ης τάξης είναι 
    συνεχείς στο $A$. Υποθέτουμε ότι για την συνάρτηση $ \phi $ οι μερικές παράγωγοι 
    1ης τάξης είναι συνεχείς στο $A$ και ικανοποιείται η συνθήκη 
    \begin{equation}
        \label{eq:constr1}
        \phi (x,y,z) = 0
    \end{equation}
    Θεωρούμε τη συνάρτηση Lagrange που ορίζεται από τον τύπο
    \[
        L(x,y,z, \lambda) = f(x,y,z) + \lambda \phi (x,y,z) 
    \] 
    και τις ορίζουσες
    \[
        \abs{\overline{H}_{1}} = 
        \begin{vmatrix}
            L_{xx} & L_{xy} & L_{xz} & {\phi}_{x} \\
            L_{yx} & L_{yy} & L_{yz} & {\phi}_{y} \\
            L_{zx} & L_{zy} & L_{zz} & {\phi}_{z} \\
            {\phi}_{x} & {\phi}_{y} & {\phi}_{z} & 0
        \end{vmatrix}, \quad 
        \abs{\overline{H}_{2}} = 
        \begin{vmatrix}
            L_{yy} & L_{yz} & {\phi}_{y} \\
            L_{zy} & L_{zz} & {\phi}_{z} \\
            {\phi}_{y} & {\phi}_{z} & 0
        \end{vmatrix}, \quad 
    \] 
    Υποθέτουμε ότι $ (x_{0}, y_{0}, z_{0}, \lambda) $ είναι μια 
    λύση του συστήματος 
    \[
        \left.
            \begin{matrix}
                L_{x} = 0 \\
                L_{y} = 0 \\
                L_{z} = 0 \\
                L_{\lambda} = 0 \\
            \end{matrix}
        \right\} 
    \]
    Τότε:
    \begin{myitemize}
        \item Αν $ \abs{\overline{H}_{1}} < 0 $ και $ \abs{\overline{H}_{2}} < 0 $ 
            στο σημείο $ (x_{0}, y_{0}, z_{0}, \lambda) $, τότε η $f$ 
            παρουσιάζει τοπικό ελάχιστο στο $ (x_{0}, y_{0}, z_{0}) $ υπό τη
            συνθήκη του περιορισμού~\eqref{eq:constr1}.
        \item Αν $ \abs{\overline{H}_{1}} < 0 $ και $ \abs{\overline{H}_{2}} > 0 $ 
            στο σημείο $ (x_{0}, y_{0}, z_{0}, \lambda) $, τότε η $f$ 
            παρουσιάζει τοπικό μέγιστο στο $ (x_{0}, y_{0}, z_{0}) $ υπό τη
            συνθήκη του περιορισμού~\eqref{eq:constr1}.
    \end{myitemize}
\end{thm}


\subsection{Ακρότατα υπό δύο συνθήκες}


\begin{thm}
    Έστω $ f(x,y,z) $ συνάρτηση τριών μεταβλητών, ορισμένη σε ένα ανοιχτό 
    υποσύνολο $A$ του $ \mathbb{R}^{3} $ και έστω ότι οι παράγωγοι 2ης τάξης είναι 
    συνεχείς στο $A$. Υποθέτουμε ότι για τις συναρτήσεις $ \phi _{1} $ και 
    $ \phi _{2} $ οι μερικές παράγωγοι 1ης τάξης είναι συνεχείς στο $A$ και 
    ικανοποιούνται οι συνθήκες (δεσμεύεσεις) 
    \[
        \begin{cases}\label{eq:constr2}
            \phi_{1} (x,y,z) = 0 \\
            \phi_{2} (x,y,z) = 0 
        \end{cases} 
    \] 
    Θεωρούμε τη συνάρτηση Lagrange που ορίζεται από τον τύπο
    \[
        L(x,y,z, \lambda _{1}, \lambda _{2}) = f(x,y,z) + \lambda _{1} 
        \phi _{1}(x,y,z) + \lambda _{2} \phi _{2}(x,y,z) 
    \] 
    και την ορίζουσα 
    \[
        \abs{\overline{H}} = 
        \begin{vmatrix}
            L_{xx} & L_{xy} & L_{xz} & {\phi _{1}}_{x} & {\phi _{2}}_{x} \\
            L_{yx} & L_{yy} & L_{yz} & {\phi _{1}}_{y} & {\phi _{2}}_{y} \\
            L_{zx} & L_{zy} & L_{zz} & {\phi _{1}}_{z} & {\phi _{2}}_{z} \\
            {\phi _{1}}_{x} & {\phi _{1}}_{y} & {\phi _{1}}_{z} & 0 & 0 \\         
            {\phi _{2}}_{x} & {\phi _{2}}_{y} & {\phi _{2}}_{z} & 0 & 0 \\         
        \end{vmatrix} 
    \] 
    Υποθέτουμε ότι $ (x_{0}, y_{0}, z_{0}, \lambda _{1}, \lambda _{2}) $ είναι μια 
    λύση του συστήματος 
    \[
        \left.
            \begin{matrix}
                L_{x} = 0 \\
                L_{y} = 0 \\
                L_{z} = 0 \\
                L_{\lambda _{1}} = 0 \\
                L_{\lambda _{2}} = 0 
            \end{matrix}
        \right\} 
    \]
    Τότε:
    \begin{myitemize}
        \item Αν $ \abs{\overline{H}} > 0 $ στο σημείο 
            $ (x_{0}, y_{0}, z_{0}, \lambda _{1}, \lambda _{2}) $, τότε η $f$ 
            παρουσιάζει τοπικό ελάχιστο στο $ (x_{0}, y_{0}, z_{0}) $ υπό τις 
            συνθήκες του περιορισμού~\eqref{eq:constr2}.
        \item Αν $ \abs{\overline{H}} < 0 $ στο σημείο 
            $ (x_{0}, y_{0}, z_{0}, \lambda _{1}, \lambda _{2}) $, τότε η $f$ 
            παρουσιάζει τοπικό μέγιστο στο $ (x_{0}, y_{0}, z_{0}) $ υπό τις 
            συνθήκες του περιορισμού~\eqref{eq:constr2}.
    \end{myitemize}
\end{thm}

\end{document}

\end{document}
