\input{preamble.tex}
\newcommand{\vect}[2]{(#1_1,\ldots, #1_#2)}
%%%%%%% nesting newcommands $$$$$$$$$$$$$$$$$$$
\newcommand{\function}[1]{\newcommand{\nvec}[2]{#1(##1_1,\ldots, ##1_##2)}}

\newcommand{\linode}[2]{#1_n(x)#2^{(n)}+#1_{n-1}(x)#2^{(n-1)}+\cdots +#1_0(x)#2=g(x)}

\newcommand{\vecoffun}[3]{#1_0(#2),\ldots ,#1_#3(#2)}


\input{tikz.tex}
\input{myboxes.tex}


\begin{document}

\chapter*{Επιφανειακό Ολοκλήρωμα Ιου Είδους}

\setcounter{chapter}{1}

\section{Ορισμός}

\begin{dfn}
  Έστω $S$ μια \textbf{απλή}, και \textbf{λεία} επιφάνεια του $ \mathbb{R}^{3} $, 
  με παραμετρική εξίσωση  
  $ 
  \mathbf{r}(u,v) = x(u,v) \mathbf{i} + y(u,v) \mathbf{j} + z(u,v) \mathbf{k}, 
  $
  όπου $ (u,v) \in D \subseteq \mathbb{R}^{2} $
  και $ f(x,y,z) $ μια \textbf{συνεχής}, πραγματική συνάρτηση, ορισμένη στην επιφάνεια 
  $S$.  Ορίζουμε το \textcolor{Col1}{επιφανειακό ολοκλήρωμα Ιου είδους} της πραγματικής 
  συνάρτησης $ f(x,y,z) $ πάνω στην επιφάνεια $S$,
  \[
    \iint_{S}f(x,y,z) \, dS = \iint_{D} f(\mathbf{r}(u,v)) \norm{\mathbf{r}_{u} \times
    \mathbf{r}_{v}} \, dudv   
  \] 
  όπου $ f(\mathbf{r}(u,v)) = f(x(u,v),y(u,v),z(u,v)) $ 
\end{dfn}
%todo να εξηγήσω το στοιχειώδες εμβαδό dS πάνω στην επφιάνεια

\section{Μέθοδος Υπολογισμού του Επιφανειακού Ολοκληρώματος Ιου είδους}

\subsection*{Η επιφάνεια $S$ δίνεται σε διανυσματική παραμετρική μορφή}

Αν η επιφάνεια $S$ ορίζεται από τη διανυσματική παραμετρική εξίσωση 
$ \mathbf{r}(u,v) = x(u,v)\mathbf{i}+y(u,v)\mathbf{j}+z(u,v)\mathbf{k} $, όπου 
$ (u,v) \in D \subseteq \mathbb{R}^{2} $, τότε:
\[
  \iint_{S} f(x,y,z) \,{dS} = \iint_{D} f(x(u,v),y(u,v),z(u,v))
  \norm{\mathbf{r}_{u} \times \mathbf{r}_{v}} \,{du}{dv}
\]

\subsection*{Η επιφάνεια $S$ δίνεται ως συνάρτηση 2 μεταβλητών}

\begin{enumerate}
  \item Αν η επιφάνεια $S$ είναι συνάρτηση της μορφής $ z=z(x,y) $ και $D$ είναι η 
    προβολή της $S$ στο επίπεδο $ xy $, τότε: 
    \[
      \iint_{S} f(x,y,z) \,{dS} = \iint_{D} f(x,y,z(x,y)) \sqrt{1+(z_{x})^{2}+
      (z_{y})^{2}} \,{dx}{dy} 
    \] 

  \item Αν η επιφάνεια $S$ είναι συνάρτηση της μορφής $ y=y(x,z) $ και $D$ είναι η 
    προβολή της $S$ στο επίπεδο $ xz $, τότε: 
    \[
      \iint_{S} f(x,y,z) \,{dS} = \iint_{D} f(x,y(x,z),z) \sqrt{1+(y_{x})^{2}+
      (y_{z})^{2}} \,{dx}{dz} 
    \] 
  \item Αν η επιφάνεια $S$ είναι συνάρτηση της μορφής $ x=x(y,z) $ και $D$ είναι η 
    προβολή της $S$ στο επίπεδο $ yz $, τότε: 
    \[
      \iint_{S} f(x,y,z) \,{dS} = \iint_{D} f(x(y,z),y,z) \sqrt{1+(x_{y})^{2}+
      (x_{z})^{2}} \,{dy}{dz} 
    \] 
\end{enumerate}

\subsection*{Η επιφάνεια $S$ δίνεται εξίσωση της μορφής g(x,y,z)=0}

Αν η επιφάνεια $S$ δίνεται από την εξίσωση $ g(x,y,z) = 0 $ και $D$ είναι η προβολή της 
$S$ στο επίπεδο $ xy $, τότε:
\[
  \iint_{S} f(x,y,z) \,{dS} = \iint_{D} f(x,y,z(x,y)) \frac{\sqrt{(g_{x})^{2}+
  (g_{y})^{2}+(g_{z})^{2}}}{\abs{g_{z}}} \,{dx}{dy} 
\]
Γενικότερα, αν η επιφάνεια $S$ δίνεται από την εξίσωση $ g(x,y,z) = 0 $, όπου $ g
$ είναι μια συνεχώς διαφορίσιμη συνάρτηση, 
και $D$ είναι η προβολή της $S$ στο επίπεδο που βρίσκεται από κάτω της, τότε:
\[
  \iint_{S} f(x,y,z) \,{dS} = \iint_{D} f(x,y,z)
  \frac{\norm{\grad{g}}}{\abs{\grad(g) \cdot \mathbf{p}}} \,{dA} 
\]
όπου $ \mathbf{p} $ είναι ένα μοναδιαίο κάθετο διάνυσμα στην περιοχή $D$ και $
\grad(g)\cdot \mathbf{p} \neq 0 $.  


\section{Ιδιότητες}

\begin{enumerate}
  \item Αν η επιφάνεια $S$ είναι ένωση των τμηματικά λείων επιφανειών $ S_{1}, S_{2},
    \ldots, S_{n} $, που δεν έχουν κοινά εσωτερικά σημεία, τότε:
    \[
      \iint_{S} f \,{dS} = \iint_{S_{1}} f \,{dS} + \iint_{S_{2}} f \,{dS} +
      \cdots + \iint_{S_{n}} f \,{dS} 
    \] 
  \item Το επιφανειακό ολοκλήρωμα Ιου είδους, είναι ανεξάρτητο από την παραμετρική
    παράσταση της επιφάνειας.
  \item Επειδή το επιφανειακό ολοκλήρωμα Ιου είδους, ανάγεται σε διπλό ολοκλήρωμα, 
    ισχύουν ανάλογες ιδιότητες με αυτές των διπλών ολοκληρωμάτων.
\end{enumerate}

\section{Εφαρμογές}

\subsection*{Μέση Τιμή Συνάρτησης πάνω σε μία επιφάνεια}

Αν $S$ είναι μια κανονική επιφάνεια του $ \mathbb{R}^{3} $ και $ f(x,y,z) $ μια 
συνεχής συνάρτηση ορισμένη πάνω στην επιφάνεια $S$, τότε ονομάζουμε μέση τιμή της 
συνάρτησης $f$ πάνω στην επιφάνεια $S$, τον πραγματικό αριθμό
\[
  M(f,S) = \frac{1}{E(S)} \iint_{S} f(x,y,z) \,{dS} 
\] 
όπου $ E(S) $ είναι το εμβαδό της επιφάνειας $S$.

\subsection*{Εμβαδό Επιφάνειας}

Αν η επιφάνειας $S$ δίνεται σε διανυσματική παραμετρική μορφή 
$ \mathbf{r} = \mathbf{r}(u,v) $, τότε το εμβαδό της δίνεται από τον τύπο:
\[
  E = \iint_{S} \,{dS} = \iint_{D} \norm{\mathbf{r}_{u} \times \mathbf{r}_{v}} \,{du}
  {dv} 
 \]

 Αν η επιφάνειας $S$ δίνεται ως συνάρτηση της μορφής $ z=z(x,y) $, τότε το εμβαδό της 
 δίνεται από τον τύπο:
\[
  E = \iint_{S} \,{dS} = \iint_{D} \sqrt{1 + (z_{x})^{2}+(z_{y})^{2}} \,{dx} {dy} 
 \]

 Αν η επιφάνειας $S$ δίνεται ως εξίσωση της μορφής $ g(x,y,z) = 0 $, τότε το εμβαδό της 
 δίνεται από τον τύπο:
\[
  E = \iint_{S} \,{dS} = \iint_{D} \frac{\sqrt{(g_{x})^{2} + (g_{y})^{2}+(g_{z})^{2}}}
  {\abs{g_{z}}} \,{dx} {dy} 
 \]

\subsection*{Μάζα Επιφάνειας}

Αν $ \delta (x,y,z) $ είναι η επιφανειακή πυκνότητα, τότε η μάζα που κατανέμεται πάνω 
στην επιφάνεια $S$, δίνεται από τον τύπο:
\[
  m = \iint_{S} \delta (x,y,z) \,{dS}  
 \] 

\subsection*{Κέντρο Μάζας}

Οι συντεταγμένες του κέντρου μάζας μιας υλικής επιφάνειας $S$ δίνονται από τους 
τύπους: 
\[
  \overline{x} = \frac{\iint_{S} x \delta (x,y,z) \,{dS}}{\iint_{S} \delta (x,y,z) \,{dS}
  }, \qquad 
  \overline{y} = \frac{\iint_{S} y \delta (x,y,z) \,{dS}}{\iint_{S} \delta (x,y,z) \,{dS}
  }, \qquad
  \overline{z} = \frac{\iint_{S} z \delta (x,y,z) \,{dS}}{\iint_{S} \delta (x,y,z) \,{dS}
}, \qquad
\]
Αν η επιφάνεια είναι ομογενής, τότε συνήθως θέτουμε $ \delta (x,y,z)=1 $.

\subsection*{Ροπές Αδράνειας}

Οι ροπές αδράνειας μιας επιφάνειας $S$ ως προς τους άξονες $ x,y $ και $z$, δίνονται 
από τους τύπους:
\[
  I_{x} = \iint_{S} (y^{2}+z^{2}) \delta (x,y,z)\,{dS}, \qquad
  I_{y} = \iint_{S} (x^{2}+z^{2}) \delta (x,y,z)\,{dS}, \qquad
  I_{z} = \iint_{S} (x^{2}+y^{2}) \delta (x,y,z)\,{dS}
 \] 








\end{document}

