\documentclass[a4paper,12pt]{article}
\usepackage{etex}
%%%%%%%%%%%%%%%%%%%%%%%%%%%%%%%%%%%%%%
% Babel language package
\usepackage[english,greek]{babel}
% Inputenc font encoding
\usepackage[utf8]{inputenc}
%%%%%%%%%%%%%%%%%%%%%%%%%%%%%%%%%%%%%%

%%%%% math packages %%%%%%%%%%%%%%%%%%
\usepackage{amsmath}
\usepackage{amssymb}
\usepackage{amsfonts}
\usepackage{amsthm}
\usepackage{proof}

\usepackage{physics}

%%%%%%% symbols packages %%%%%%%%%%%%%%
\usepackage{dsfont}
\usepackage{stmaryrd}
%%%%%%%%%%%%%%%%%%%%%%%%%%%%%%%%%%%%%%%


%%%%%% graphicx %%%%%%%%%%%%%%%%%%%%%%%
\usepackage{graphicx}
\usepackage{color}
%\usepackage{xypic}
\usepackage[all]{xy}
\usepackage{calc}
%%%%%%%%%%%%%%%%%%%%%%%%%%%%%%%%%%%%%%%

\usepackage{enumerate}

\usepackage{fancyhdr}
%%%%% header and footer rule %%%%%%%%%
\setlength{\headheight}{14pt}
\renewcommand{\headrulewidth}{0pt}
\renewcommand{\footrulewidth}{0pt}
\fancypagestyle{plain}{\fancyhf{}
\fancyhead{}
\lfoot{}
\rfoot{\small \thepage}}
\fancypagestyle{vangelis}{\fancyhf{}
\rhead{\small \leftmark}
\lhead{\small }
\lfoot{}
\rfoot{\small \thepage}}
%%%%%%%%%%%%%%%%%%%%%%%%%%%%%%%%%%%%%%%

\usepackage{hyperref}
\usepackage{url}
%%%%%%% hyperref settings %%%%%%%%%%%%
\hypersetup{pdfpagemode=UseOutlines,hidelinks,
bookmarksopen=true,
pdfdisplaydoctitle=true,
pdfstartview=Fit,
unicode=true,
pdfpagelayout=OneColumn,
}
%%%%%%%%%%%%%%%%%%%%%%%%%%%%%%%%%%%%%%



\usepackage{geometry}
\geometry{left=25.63mm,right=25.63mm,top=36.25mm,bottom=36.25mm,footskip=24.16mm,headsep=24.16mm}

%\usepackage[explicit]{titlesec}
%%%%%% titlesec settings %%%%%%%%%%%%%
%\titleformat{\chapter}[block]{\LARGE\sc\bfseries}{\thechapter.}{1ex}{#1}
%\titlespacing*{\chapter}{0cm}{0cm}{36pt}[0ex]
%\titleformat{\section}[block]{\Large\bfseries}{\thesection.}{1ex}{#1}
%\titlespacing*{\section}{0cm}{34.56pt}{17.28pt}[0ex]
%\titleformat{\subsection}[block]{\large\bfseries{\thesubsection.}{1ex}{#1}
%\titlespacing*{\subsection}{0pt}{28.80pt}{14.40pt}[0ex]
%%%%%%%%%%%%%%%%%%%%%%%%%%%%%%%%%%%%%%

%%%%%%%%% My Theorems %%%%%%%%%%%%%%%%%%
\newtheorem{thm}{Θεώρημα}[section]
\newtheorem{cor}[thm]{Πόρισμα}
\newtheorem{lem}[thm]{λήμμα}
\theoremstyle{definition}
\newtheorem{dfn}{Ορισμός}[section]
\newtheorem{dfns}[dfn]{Ορισμοί}
\theoremstyle{remark}
\newtheorem{remark}{Παρατήρηση}[section]
\newtheorem{remarks}[remark]{Παρατηρήσεις}
%%%%%%%%%%%%%%%%%%%%%%%%%%%%%%%%%%%%%%%




\newcommand{\vect}[2]{(#1_1,\ldots, #1_#2)}
%%%%%%% nesting newcommands $$$$$$$$$$$$$$$$$$$
\newcommand{\function}[1]{\newcommand{\nvec}[2]{#1(##1_1,\ldots, ##1_##2)}}

\newcommand{\linode}[2]{#1_n(x)#2^{(n)}+#1_{n-1}(x)#2^{(n-1)}+\cdots +#1_0(x)#2=g(x)}

\newcommand{\vecoffun}[3]{#1_0(#2),\ldots ,#1_#3(#2)}




\pagestyle{vangelis}
\everymath{\displaystyle}

\renewcommand{\qedsymbol}{}

\begin{document}

\chapter{Μετασχηματισμός Laplace}

\vspace{\baselineskip}


\begin{dfn}
  Μια συνάρτηση $ f(t) $ λέμε ότι είναι \textcolor{Col1}{εκθετικού τύπου $a$}, όπου 
  $a>0$, στο $ 0 \leq t < \infty $, αν υπάρχει $ M>0 $, ώστε για κάθε $ t \geq t_{0} $ 
  να ισχύει $ \abs{f(t)} \leq M \mathrm{e}^{at} $.
\end{dfn}

\begin{prop}
  Κάθε φραγμένη συνάρτηση είναι εκθετικού τύπου 0.
\end{prop}
\begin{proof}
  Προφανώς
  \[
    \abs{f(t)} \leq M = M \mathrm{e}^{0t}
  \] 
\end{proof}

\begin{example}
  Η βηματική συνάρτηση $ H(t) $ είναι εκθετικού τύπου 0.
  \begin{proof}
    \[
      \abs{H(t)} \leq 1 = \mathrm{e}^{0t} 
    \] 
  \end{proof}
\end{example}

\begin{prop}
  Κάθε πολυωνυμική συνάρτηση είναι εκθετικού τύπου $a$, για κάθε $ a>0 $.
\end{prop}
\begin{proof}
  Πράγματι
  \[
    \mathrm{e}^{at} = \sum_{n=0}^{\infty} \frac{(at)^{n}}{n!} = \sum_{n=0}^{\infty} 
    \frac{a^{n}}{n!} t^{n} \geq \frac{a^{n}}{n!} t^{n} \Rightarrow t^{n} \leq
    \frac{n!}{a^{n}} \mathrm{e}^{at}
  \]
\end{proof}

\begin{example}
  Η συνάρτηση $ f(t) = \mathrm{e}^{5t} \sin{2t} $ είναι εκθετικού τύπου 5.
  \begin{proof}
    \[
      \abs{f(t)} = \abs{\mathrm{e}^{5t} \sin{2t}} = \abs{\mathrm{e}^{5t}} \cdot 
      \abs{\sin{2t}} \leq \mathrm{e}^{5t} \cdot 1 = \mathrm{e}^{5t}
    \] 
  \end{proof}
\end{example}


\begin{dfn}
  Έστω $ f(t) $ συνάρτηση, ορισμένη για $ t \geq 0 $. Τότε ο μετασχηματισμός Laplace 
  της $f$, είναι
  \[
    \mathcal{L} \left[f(t)\right] = F(s) = \int _{0}^{+\infty} f(t)\cdot \mathrm{e}^{-st} 
    \,{dt}, \quad s>0
  \] 
  οποτεδήποτε το γενικευμένο ολοκλήρωμα συγκλίνει.
\end{dfn}

\begin{example}
  $ \mathcal{L}\left[1\right] = \frac{1}{s}, \quad s > 0 $
\end{example}
\begin{proof}
  \[
    \mathcal{L}\left[1\right] = \int_{0}^{+\infty} 1\cdot \mathrm{e}^{-st} \,{dt} =  
    \int_{0}^{+\infty} \mathrm{e}^{-st} \,{dt} = \eval{\frac{\mathrm{e}^{-st}}{-s}}
    _{0}^{+\infty} = \lim_{t \to \infty} \cancelto{0}{\frac{\mathrm{e}^{-st}}{-s}} 
    + \frac{1}{s} = \frac{1}{s} , \quad s>0 
   \] 
\end{proof}

\begin{example}
  $ \mathcal{L}\left[t\right] = \frac{1}{s^{2}}, \quad s>0 $
\end{example}
\begin{proof}
  \[
    \mathcal{L}\left[t\right] = \int_{0}^{+\infty} t \cdot \mathrm{e}^{-st} \,{dt} = 
    \left[-\frac{t \mathrm{e}^{-st}}{s} - \frac{\mathrm{e}^{-st}}{s^{2}}\right]
    _{0}^{+\infty} = \lim_{t \to \infty} 
    \cancelto{0}{\left(- \frac{t\mathrm{e}^{-st}}{s} -
    \frac{\mathrm{e}^{-st}}{s^{2}}\right)} + \frac{1}{s^{2}} = \frac{1}{s^{2}} , 
    \quad s>0
   \] 
\end{proof}

\begin{example}
  $ \mathcal{L}\left[\mathrm{e}^{at}\right] = \frac{1}{s-a}, \quad s > a $ 
  \begin{proof}
    \[
      \mathcal{L}\left[\mathrm{e}^{at}\right] = \int_{0}^{+\infty} \mathrm{e}^{at} 
      \mathrm{e}^{-st} \,{dt} = \int_{0}^{+\infty} \mathrm{e}^{-(s-a)t} \,{dt} =
      \eval{\frac{\mathrm{e}^{-(s-a)t}}{-(s-a)}} _{0}^{+\infty} = 
      \lim_{t \to \infty} \cancelto{0}{\frac{\mathrm{e}^{-(s-a)}}{-(s-a)}}
      + \frac{1}{s-a} = \frac{1}{s-a}, \quad s > a
    \]
  \end{proof}
\end{example}

\begin{example}
  $ \mathcal{L}\left[\sin{at}\right] = \frac{a}{s^{2}+a^{2}} $
\end{example}
\begin{proof}
  \begin{align*}
    \mathcal{L}\left[\sin{at}\right] &= \int_{0}^{+\infty} \sin{at} \cdot
    \mathrm{e}^{-st} \,{dt} = \left[- \frac{\mathrm{e}^{-st}}{s^{2}+a^{2}} 
    (a \cos{at} + s \sin{at})\right] _{0}^{+\infty} = \lim_{t \to \infty} 
    \cancelto{0}{\left[- \frac{\mathrm{e}^{-st}}{s^{2}+a^{2}} (a \cos{at} + s
    \sin{at})\right]} + \frac{a}{s^{2}+a^{2}} \\
                                     &= \frac{a}{s^{2}+a^{2}} , \quad s>0
  \end{align*} 
\end{proof}

\end{document}
