\input{preamble_ask.tex}
\input{definitions_ask.tex}

\geometry{left=15.63mm,right=15.63mm,top=30.25mm,bottom=36.25mm,
footskip=24.16mm,headsep=24.16mm}

\pagestyle{vangelis}
\everymath{\displaystyle}

\renewcommand{\qedsymbol}{}

\newtheoremstyle{myex}
{\topsep}{\topsep}%
{}{}%
{\color{Col1}\bfseries}{}%
{\baselineskip}{}%



\begin{document}

\chapter*{Μετασχηματισμός Laplace}

\vspace{\baselineskip}


\section*{Ορισμός και Παραδείγματα}

\begin{dfn}
  Έστω $ f(t) $ συνάρτηση, ορισμένη για $ t \geq 0 $. Τότε ο μετασχηματισμός Laplace 
  της $f$, είναι
  \begin{equation}\label{eq:laplace}
    \mathcal{L} \left[f(t)\right] = F(s) = \int _{0}^{+\infty} f(t) 
    \mathrm{e}^{-st} \,{dt}, \quad s>0
  \end{equation} 
  οποτεδήποτε το γενικευμένο ολοκλήρωμα συγκλίνει.
\end{dfn}

\begin{example}
  $ \boldsymbol{\mathcal{L}\left[1\right] = \frac{1}{s}, \quad s > 0} $
\end{example}
\begin{proof}
  \[
    \mathcal{L}\left[1\right] = \int_{0}^{+\infty} 1\cdot \mathrm{e}^{-st} \,{dt} =  
    \int_{0}^{+\infty} \mathrm{e}^{-st} \,{dt} = \eval{\frac{\mathrm{e}^{-st}}{-s}}
    _{0}^{+\infty} = \lim_{t \to \infty} \cancelto{0}{\frac{\mathrm{e}^{-st}}{-s}} 
    + \frac{1}{s} = \frac{1}{s} , \quad s>0 
  \] 
\end{proof}

\begin{example}
  $ \boldsymbol{\mathcal{L}\left[t\right] = \frac{1}{s^{2}}, \quad s>0} $
\end{example}
\begin{proof}
  \[
    \mathcal{L}\left[t\right] = \int_{0}^{+\infty} t \cdot \mathrm{e}^{-st} \,{dt} = 
    \left[-\frac{t \mathrm{e}^{-st}}{s} - \frac{\mathrm{e}^{-st}}{s^{2}}\right]
    _{0}^{+\infty} = \lim_{t \to \infty} 
    \cancelto{0}{\left(- \frac{t\mathrm{e}^{-st}}{s} -
    \frac{\mathrm{e}^{-st}}{s^{2}}\right)} + \frac{1}{s^{2}} = \frac{1}{s^{2}} , 
    \quad s>0
  \] 
\end{proof}

\begin{example}
  $ \boldsymbol{\mathcal{L}\left[\mathrm{e}^{at}\right] = \frac{1}{s-a}, \quad s > a} $ 
\end{example}
  \begin{proof}
    \[
      \mathcal{L}\left[\mathrm{e}^{at}\right] = \int_{0}^{+\infty} \mathrm{e}^{at} 
      \mathrm{e}^{-st} \,{dt} = \int_{0}^{+\infty} \mathrm{e}^{-(s-a)t} \,{dt} =
      \eval{\frac{\mathrm{e}^{-(s-a)t}}{-(s-a)}} _{0}^{+\infty} = 
      \lim_{t \to \infty} \cancelto{0}{\frac{\mathrm{e}^{-(s-a)}}{-(s-a)}}
      + \frac{1}{s-a} = \frac{1}{s-a}, \quad s > a
    \]
  \end{proof}

\begin{example}
  $ \boldsymbol{\mathcal{L}\left[\sin{at}\right] = \frac{a}{s^{2}+a^{2}}} $
\end{example}
\begin{proof}
  \begin{align*}
    \mathcal{L}\left[\sin{at}\right] &= \int_{0}^{+\infty} \sin{at} \cdot
    \mathrm{e}^{-st} \,{dt} = {\left[- \frac{\mathrm{e}^{-st}}{s^{2}+a^{2}} 
  (a \cos{at} + s \sin{at})\right]} _{0}^{+\infty} = \lim_{t \to \infty} 
    \cancelto{0}{\left[- \frac{\mathrm{e}^{-st}}{s^{2}+a^{2}} (a \cos{at} + s
    \sin{at})\right]} + \frac{a}{s^{2}+a^{2}} \\
  \end{align*} 
  \begin{description}
    \item [Β Τρόπος:]
      \begin{align*}
        \mathcal{L}\left[\sin{at}\right] = \mathcal{L}\left(\frac{\mathrm{e}^{iat} -
        \mathrm{e}^{-iat}}{2i}\right) = \frac{1}{2i} 
        \left(\mathcal{L}[\mathrm{e}^{iat}] - \mathcal{L}[\mathrm{e}^{-iat}]\right) = 
        \frac{1}{2i} \left(\frac{1}{s-ia} - \frac{1}{s+ia}\right) = 
        \frac{1}{2i} \left(\frac{2ia}{s^{2}+a^{2}}\right) = \frac{a}{s^{2}+a^{2}}
      \end{align*}
  \end{description}
\end{proof}

\begin{example}
  $ \boldsymbol{\mathcal{L}\left[\sinh{at}\right] = \frac{a}{s^{2}-a^{2}}} $
\end{example}
\begin{proof}
  \begin{align*}
    \mathcal{L}\left[\sin{at}\right] = \mathcal{L}\left(\frac{\mathrm{e}^{at} -
    \mathrm{e}^{-at}}{2}\right) = \frac{1}{2} 
    \left(\mathcal{L}[\mathrm{e}^{at}] - \mathcal{L}[\mathrm{e}^{-at}]\right) = 
    \frac{1}{2} \left(\frac{1}{s-a} - \frac{1}{s+a}\right) = 
    \frac{1}{2} \left(\frac{2a}{s^{2}-a^{2}}\right) = \frac{a}{s^{2}-a^{2}}
  \end{align*}
\end{proof}

\begin{example}\label{ex:nonex}
  Η συνάρτηση $ \boldsymbol{f(t)= \mathrm{e}^{t^{2}}} $ δεν έχει μετασχηματισμό Laplace.
\end{example}
\begin{proof}
\item {}
  Αν $ s \leq 0 $ τότε $ t^{2}-st \geq 0 \Rightarrow
  \mathrm{e}^{t^{2}-st} \geq 1 $, άρα 
  \[
    \mathcal{L}\left[\mathrm{e}^{t^{2}}\right] = \int_{0}^{+\infty} 
    \mathrm{e}^{t^{2}}  \mathrm{e}^{-st}\,{dt} =
    \int_{0}^{+\infty} \mathrm{e}^{t^{2}-st} \,{dt} \geq \int_{0}^{+\infty} \,{dt}, 
    \quad \forall t \geq 0
  \] 
  Όμως
  \[
    \int_{0}^{+\infty} \,{dt} = {\left[t \strut\right]}_{0}^{+\infty} = 
    \lim_{t \to +\infty} t - 0 = +\infty, \quad \text{αποκλίνει}
  \]
  Άρα από το κριτήριο σύγκρισης και το αρχικό ολοκλήρωμα αποκλίνει.

  Αν $ s \geq 0 $ τότε $ t^{2}-st \geq 0 \Leftrightarrow t(t-s) \geq 0
  \Leftrightarrow t \geq s $, οπότε $ \mathrm{e}^{t^{2}-st} \geq 1 \Leftrightarrow t
  \geq s $, επομένως 
  \[
    \mathcal{L}\left[\mathrm{e}^{t^{2}}\right] = \int_{0}^{+\infty} 
    \mathrm{e}^{t^{2}}  \mathrm{e}^{-st}\,{dt} = \int_{0}^{+\infty} 
    \mathrm{e}^{t^{2}-st} \,{dt} \geq \int_{s}^{+\infty} \,{dt}, \quad \forall t \geq s
  \]
  όμως
  \[
    \int_{s}^{+\infty} \,{dt} = 
    {\left[t \strut\right]}_{s}^{+\infty} = \lim_{t \to +\infty} t - s = +\infty
  \] 
  Άρα από το κριτήριο σύγκρισης και το αρχικό ολοκλήρωμα αποκλίνει.  
  Άρα, η $ f(t) = t^{2} $ δεν έχει μετασχηματισμό Laplace.
\end{proof}


\section*{Ικανές και Αναγκαίες συνθήκες ύπαρξης του μετασχηματισμού Laplace}

Από το παράδειγμα~\ref{ex:nonex} καταλαβαίνουμε ότι δεν έχουν όλες οι συναρτήσεις 
μετασχηματισμό Laplace. Στο επόμενο θεώρημα, μετά από κάποιους χρήσιμους ορισμούς,  
διατυπώνουμε \textbf{ικανές} συνθήκες για την ύπαρξη του μετασχηματισμού Laplace 
μιας συνάρτησης. 

\begin{dfn}
  Μια συνάρτηση $ f(t) $ λέμε ότι είναι \textcolor{Col1}{εκθετικού τύπου $a$}, όπου 
  $a>0$, στο $ 0 \leq t < \infty $, αν υπάρχει $ M>0$ και $ t_{0}>0 $, ώστε για κάθε 
  $ t \geq t_{0} $ να ισχύει $ \abs{f(t)} \leq M \mathrm{e}^{at} $.
\end{dfn}

\begin{rem}
  Ουσιαστικά, μια συνάρτηση εκθετικού τύπου $a$, δεν αυξάνει ταχύτερα από την 
  συνάρτηση $ M \mathrm{e}^{at} $ καθώς $ t \to \infty $. Αυτό σημαίνει ότι η 
  γραφική της παράσταση, περιέχεται στην περιοχή που βρίσκεται ανάμεσα στις 
  γραφικές παραστάσεις των συναρτήσεων 
  $ \pm M \mathrm{e}^{at}$, για κάθε $ t \geq t_{0} $.
\end{rem}

\begin{prop}
  Κάθε φραγμένη συνάρτηση είναι εκθετικού τύπου 0.
\end{prop}
\begin{proof}
  Προφανώς
  \[
    \abs{f(t)} \leq M = M \mathrm{e}^{0t}
  \] 
\end{proof}

\begin{example}
  Η Βηματική συνάρτηση $ \boldsymbol{H(t)} $ είναι εκθετικού τύπου 0.
\end{example}
  \begin{proof}
    \[
      \abs{H(t)} \leq 1 = \mathrm{e}^{0t} 
    \] 
  \end{proof}

\begin{prop}
  Κάθε πολυωνυμική συνάρτηση είναι εκθετικού τύπου $a$, για κάθε $ a>0 $.
\end{prop}
\begin{proof}
  Πράγματι
  \[
  \mathrm{e}^{at} = \sum_{n=0}^{\infty} \frac{{(at)}^{n}}{n!} = \sum_{n=0}^{\infty} 
    \frac{a^{n}}{n!} t^{n} \geq \frac{a^{n}}{n!} t^{n} \Rightarrow t^{n} \leq
    \frac{n!}{a^{n}} \mathrm{e}^{at}
  \]
\end{proof}

\begin{example}
  Η συνάρτηση $ \boldsymbol{f(t) = \mathrm{e}^{5t} \sin{2t}} $ είναι εκθετικού τύπου 5.
\end{example}
  \begin{proof}
    \[
      \abs{f(t)} = \abs{\mathrm{e}^{5t} \sin{2t}} = \abs{\mathrm{e}^{5t}} \cdot 
      \abs{\sin{2t}} \leq \mathrm{e}^{5t} \cdot 1 = \mathrm{e}^{5t}
    \] 
  \end{proof}

\begin{rem}
  Ένας, τρόπος για να ελέγχουμε αν μια συνάρτηση είναι εκθετικού τύπου $a$, είναι 
  υπολογίζοντας το όριο
  \[
    \lim_{t \to \infty} \frac{\abs{f(t)}}{\mathrm{e}^{at}} 
  \] 
  Αν αυτό το όριο είναι πεπερασμένο για κάποια τιμή του $a$, τότε η συνάρτηση $ f(t)
  $ θα είναι εκθετικού τύπου $a$, αλλιώς, αν αυτό το όριο είναι άπειρο για κάθε $a$, 
  τότε η $f$ δεν είναι εκθετικού τύπου.
\end{rem}

\begin{example}
  Η συνάρτηση του παραδείγματος~\ref{ex:nonex}, δεν είναι εκθετικού τύπου, αφού
  \[
    \lim_{t \to \infty} \frac{|\mathrm{e}^{t^{2}}|}{\mathrm{e}^{at}} = 
    \lim_{t \to \infty} \frac{\mathrm{e}^{t^{2}}}{\mathrm{e}^{at}} = 
    \lim_{t \to \infty} \mathrm{e}^{t(t-a)} = \infty, \quad \forall a
  \]
\end{example}

\begin{dfn}
  Μια συνάρτηση $ f \colon [a,b] \to \mathbb{R} $ ονομάζεται 
  \textcolor{Col1}{τμηματικά συνεχής}, αν το σύνολο των σημείων 
  ασυνέχειας είναι πεπερασμένο και \textbf{σε κάθε} σημείο ασυνέχειας της $f$ 
  υπάρχουν τα πλευρικά όρια και είναι πεπερασμένα.
\end{dfn}

\begin{rem}
  Προφανώς αν μια συνάρτηση είναι συνεχής, είναι και τμηματικά συνεχής.
\end{rem}

\begin{thm}\label{thm:sygklisi}
  Έστω $f$ μια συνάρτηση, που είναι \textbf{τμηματικά συνεχής} και 
  \textbf{εκθετικού τύπου $a$}. Τότε ο μετασχηματισμός Laplace 
  $ \mathcal{L}\left[f(t)\right] = F(s) $ της $ f(t) $, όπως ορίστηκε 
  στη σχέση~\eqref{eq:laplace}, \textbf{υπάρχει} για κάθε $ s>a $.
\end{thm}
\begin{proof}
  ισχύει ότι
  \[
    \mathcal{L}\left[f(t)\right] = \int_{0}^{+\infty} f(t) \mathrm{e}^{-st} \,{dt} 
    \leq \abs{\int_{0}^{+\infty} f(t)  \mathrm{e}^{-st} \,{dt}} \leq 
    \int_{0}^{+\infty} \abs{f(t)}  \mathrm{e}^{-st} \,{dt} 
  \] 
  επομένως, αρκεί να δείξουμε ότι 
  \[
    \int_{0}^{+\infty} \abs{f(t)} \mathrm{e}^{-st} \,{dt} < \infty .
  \] 
  Πράγματι
  \[
    \int_{0}^{+\infty} \abs{f(t)}  \mathrm{e}^{-st} \,{dt}  = \int_{0}^{+\infty}
    \abs{f(t)} \mathrm{e}^{-st} \mathrm{e}^{-at} \mathrm{e}^{at} \,{dt} = 
    \int_{0}^{+\infty} \abs{f(t)}  \mathrm{e}^{-at}  \mathrm{e}^{-(s-a)t} 
    \,{dt} 
  \] 
  και επειδή $f$ είναι εκθετικού τύπου $a$ έχουμε ότι
  \[
    \int_{0}^{+\infty} \abs{f(t)} \mathrm{e}^{-st} \,{dt} \leq M \int_{0}^{+\infty}
    \mathrm{e}^{-(s-a)t} \,{dt} = M
    {\left[\frac{\mathrm{e}^{-(s-a)t}}{-(s-a)}\right]}_{0}^{+\infty} = \frac{M}{s-a} < 
    \infty , \quad \text{αν} \quad s-a>0 \Leftrightarrow s>a
  \] 
\end{proof}

\begin{rem}
  Άμεση συνέπεια του παραπάνω θεωρήματος είναι ότι η μεταβλητή $s$ είναι \textbf{θετική},
  γεγονός που τις περισσότερες φορές συμπεριλαμβάνεται στον ορισμό του μετασχηματισμού 
  Laplace.
\end{rem}

\begin{cor}\label{cor:4}
  Αν ισχύουν οι συνθήκες του θεωρήματος~\ref{thm:sygklisi}, τότε
  \begin{equation}\label{eq:anagkaia}
    \lim_{s \to \infty} F(s) = 0 
  \end{equation} 
\end{cor}
\begin{proof}
  Πράγματι, από την απόδειξη του θεωρήματος, έχουμε
  \[
    \lim_{s \to \infty} \abs{F(s)} \leq \lim_{s \to \infty} \frac{M}{s-a} = 0
  \] 
  από όπου, με το κριτήριο παρεμβολής, έπεται ότι
  \[
    \lim_{s \to \infty} F(s) = 0 
  \]
\end{proof}
\begin{rem}
  Άμεση συνέπεια του πορίσματος~\ref{cor:4} είναι ότι η συνθήκη~\eqref{eq:anagkaia} 
  είναι \textbf{αναγκαία} για την ύπαρξη του μετασχηματισμού Laplace μιας συνάρτησης. 
  Αυτό σημαίνει ότι αν συνάρτηση $F(s)$ δεν ικανοποιεί τη συνθήκη αυτή, τότε δε 
  μπορεί να είναι ο μετασχηματισμός Laplace καμίας συνεχούς ή τμηματικά συνεχούς 
  συνάρτησης.
\end{rem}

\section*{Μετασχηματισμός Laplace περιοδικής συνάρτησης}

\begin{prop}
  Έστω $ f(t) $ μια περιοδική συνάρτηση, θετικής περιόδου $T$, για την οποία υπάρχει 
  ο μετασχηματισμός Laplace. Να δειχθεί ότι 
  \[
    \mathcal{L}\left[f(t)\right] = \frac{1}{1- \mathrm{e}^{sT}} \int _{0}^{T}
    \mathrm{e}^{-st}f(t) \,{dt}  
  \] 
\end{prop}
  \begin{proof}
    Από τον ορισμό του μετασχηματισμού Laplace
    \begin{align*}
      \mathcal{L}\left[f(t)\right] 
      &= \int_{0}^{+\infty} \mathrm{e}^{-st} f(t) \,{dt} \\ 
      &= \int_{0}^{T} \mathrm{e}^{-st} f(t) \,{dt} + \int _{T}^{2T} \mathrm{e}^{-st}f(t)
      \,{dt} + \int _{2T}^{3T} \mathrm{e}^{-st} f(t)\,{dt} + \cdots \\ 
      &= \int _{0}^{T} \mathrm{e}^{-st} f(t)\,{dt} + \int _{0}^{T} \mathrm{e}^{-s(u+T)}
      f(u+T) \,{du} + \int _{0}^{T} \mathrm{e}^{-s(u+2T)} f(u+2T) \,{du} + \cdots \\
      \intertext{λαμβάνοντας όμως υπόψιν ότι $f$ περιοδική με περίοδο $T>0$, παίρνουμε} 
      &= \int _{0}^{T} \mathrm{e}^{-st} f(t) \,{dt} + \int _{0}^{T} \mathrm{e}^{-s(u+T)}
      f(u) \,{du} + \int _{0}^{T} \mathrm{e}^{-s(u+2T)} f(u) \,{du} + \cdots \\
      &= \int _{0}^{T} \mathrm{e}^{-st} f(t) \,{dt} + \mathrm{e}^{-sT} \int _{0}^{T} 
      \mathrm{e}^{-st} f(t) \,{dt} + \mathrm{e}^{-2sT} \int _{0}^{T} \mathrm{e}^{-st}
      f(t) \,{dt} + \cdots \\
      &= \left[1+ \mathrm{e}^{-sT} + \mathrm{e}^{{(-sT)}^{2}} + \cdots\right] 
      \int _{0}^{T} \mathrm{e}^{-st} f(t) \,{dt} \\
      \intertext{και χρησιμοποιώντας τον τύπο αθροίσματος γεωμετρικής σειράς με 
        λόγο $ 0< \mathrm{e}^{-st} <1$, προκύπτει}
      &= \frac{1}{1- \mathrm{e}^{-st}} \int _{0}^{T} \mathrm{e}^{-st} f(t) \,{dt} 
    \end{align*}
  \end{proof}


\end{document}
