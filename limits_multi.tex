\documentclass[a4paper,12pt]{article}
\usepackage{etex}
%%%%%%%%%%%%%%%%%%%%%%%%%%%%%%%%%%%%%%
% Babel language package
\usepackage[english,greek]{babel}
% Inputenc font encoding
\usepackage[utf8]{inputenc}
%%%%%%%%%%%%%%%%%%%%%%%%%%%%%%%%%%%%%%

%%%%% math packages %%%%%%%%%%%%%%%%%%
\usepackage{amsmath}
\usepackage{amssymb}
\usepackage{amsfonts}
\usepackage{amsthm}
\usepackage{proof}

\usepackage{physics}

%%%%%%% symbols packages %%%%%%%%%%%%%%
\usepackage{dsfont}
\usepackage{stmaryrd}
%%%%%%%%%%%%%%%%%%%%%%%%%%%%%%%%%%%%%%%


%%%%%% graphicx %%%%%%%%%%%%%%%%%%%%%%%
\usepackage{graphicx}
\usepackage{color}
%\usepackage{xypic}
\usepackage[all]{xy}
\usepackage{calc}
%%%%%%%%%%%%%%%%%%%%%%%%%%%%%%%%%%%%%%%

\usepackage{enumerate}

\usepackage{fancyhdr}
%%%%% header and footer rule %%%%%%%%%
\setlength{\headheight}{14pt}
\renewcommand{\headrulewidth}{0pt}
\renewcommand{\footrulewidth}{0pt}
\fancypagestyle{plain}{\fancyhf{}
\fancyhead{}
\lfoot{}
\rfoot{\small \thepage}}
\fancypagestyle{vangelis}{\fancyhf{}
\rhead{\small \leftmark}
\lhead{\small }
\lfoot{}
\rfoot{\small \thepage}}
%%%%%%%%%%%%%%%%%%%%%%%%%%%%%%%%%%%%%%%

\usepackage{hyperref}
\usepackage{url}
%%%%%%% hyperref settings %%%%%%%%%%%%
\hypersetup{pdfpagemode=UseOutlines,hidelinks,
bookmarksopen=true,
pdfdisplaydoctitle=true,
pdfstartview=Fit,
unicode=true,
pdfpagelayout=OneColumn,
}
%%%%%%%%%%%%%%%%%%%%%%%%%%%%%%%%%%%%%%



\usepackage{geometry}
\geometry{left=25.63mm,right=25.63mm,top=36.25mm,bottom=36.25mm,footskip=24.16mm,headsep=24.16mm}

%\usepackage[explicit]{titlesec}
%%%%%% titlesec settings %%%%%%%%%%%%%
%\titleformat{\chapter}[block]{\LARGE\sc\bfseries}{\thechapter.}{1ex}{#1}
%\titlespacing*{\chapter}{0cm}{0cm}{36pt}[0ex]
%\titleformat{\section}[block]{\Large\bfseries}{\thesection.}{1ex}{#1}
%\titlespacing*{\section}{0cm}{34.56pt}{17.28pt}[0ex]
%\titleformat{\subsection}[block]{\large\bfseries{\thesubsection.}{1ex}{#1}
%\titlespacing*{\subsection}{0pt}{28.80pt}{14.40pt}[0ex]
%%%%%%%%%%%%%%%%%%%%%%%%%%%%%%%%%%%%%%

%%%%%%%%% My Theorems %%%%%%%%%%%%%%%%%%
\newtheorem{thm}{Θεώρημα}[section]
\newtheorem{cor}[thm]{Πόρισμα}
\newtheorem{lem}[thm]{λήμμα}
\theoremstyle{definition}
\newtheorem{dfn}{Ορισμός}[section]
\newtheorem{dfns}[dfn]{Ορισμοί}
\theoremstyle{remark}
\newtheorem{remark}{Παρατήρηση}[section]
\newtheorem{remarks}[remark]{Παρατηρήσεις}
%%%%%%%%%%%%%%%%%%%%%%%%%%%%%%%%%%%%%%%




\newcommand{\vect}[2]{(#1_1,\ldots, #1_#2)}
%%%%%%% nesting newcommands $$$$$$$$$$$$$$$$$$$
\newcommand{\function}[1]{\newcommand{\nvec}[2]{#1(##1_1,\ldots, ##1_##2)}}

\newcommand{\linode}[2]{#1_n(x)#2^{(n)}+#1_{n-1}(x)#2^{(n-1)}+\cdots +#1_0(x)#2=g(x)}

\newcommand{\vecoffun}[3]{#1_0(#2),\ldots ,#1_#3(#2)}



\pagestyle{vangelis}
\everymath{\displaystyle}


\begin{document}

\chapter{Όρια Συναρτήσεων Πολλών Μεταβλητών}

\section{Ορισμός}

\vspace{\baselineskip}

\mydfn{%
    Έστω $ f \colon A \subseteq \mathbb{R}^{2} \to \mathbb{R} $.

    Λέμε ότι η $f$ έχει όριο τον αριθμό $ l \in \mathbb{R} $ στο σημείο 
    $ (x_{0}, y_{0}) $ και το συμβολίζουμε με \[ \lim\limits_{(x,y)\to (x_{0},y_{0})} 
        f(x,y) = l \quad \text{ή} \quad \lim_{\substack{x\to x_{0} \\ y\to y_{0}}} 
    f(x,y)=l, \] αν 
    \[
        \forall \varepsilon > 0, \; \exists \delta _{1} > 0, \; \exists \delta _2 > 0 
        \; : \; \forall (x,y) \in A \; \text{με} \; 0 < \abs{x - x_{0}} < \delta _{1} 
        \; \text{και} \; 0 < \abs{y - y_{0}} < \delta _{2} \Rightarrow \abs{f(x,y)-l} 
        < \varepsilon 
\]}

\begin{examples}
\item {}
    \begin{enumerate}
        \item Να δείξετε ότι $ \lim\limits_{\substack{x\to 0 \\y \to 0}} 
            \frac{xy}{x^{2}+y^{2}+1} = 0 $.
    \end{enumerate}

    \begin{proof}
    \item {}
        \begin{description}
            \item [Δοκιμή:]
                \[
                    \abs{\frac{xy}{x^{2}+y^{2}+1}-0} < \varepsilon \Leftrightarrow 
                    \frac{\abs{xy}}{\abs{x^{2}+y^{2}+1}} < \varepsilon 
                \] 
                Όμως ισχύει ότι 
                \[
                    \frac{\abs{xy}}{\abs{x^{2}+y^{2}+1}} < \abs{xy}, \quad 
                    \forall (x,y) \in \mathbb{R}^{2}
                \] 
                Οπότε αρκεί
                \[
                    \abs{xy} < \varepsilon \Leftrightarrow \abs{x} \abs{y} < \varepsilon 
                    \Leftrightarrow \abs{x} \abs{y} < \sqrt{\varepsilon} 
                    \sqrt{\varepsilon}
                \] 
                Έστω $ \varepsilon > 0 $. Επιλέγουμε 

                $ \delta _{1} = \delta _{2} = \sqrt{\varepsilon} $ και έχουμε:

                \[
                    \forall (x,y) \in \mathbb{R}^{2} \; \text{με} \; 0 < \abs{x-0} = 
                    \abs{x} < \sqrt{\varepsilon} = \delta _{1} \; \text{και} \; 
                    0 < \abs{y-0} = \abs{y} < \sqrt{\varepsilon} = 
                    \delta _{2} \Rightarrow  
                    \abs{\frac{xy}{x^{2}+y^{2}+1}-0} < \varepsilon
                \] 
        \end{description}
    \end{proof}
\end{examples}

\section{Βασικά Θεωρήματα}

\mythm{%
    Έστω $ \lim\limits_{(x,y)\to (x_{0}, y_{0})} f(x,y) = l_{1} $ και 
    $ \lim\limits_{(x,y)\to (x_{0}, y_{0})} g(x,y) = l_{2} $. Τότε:
    \begin{enumerate}
        \item $ \lim\limits_{(x,y)\to (x_{0}, y_{0})} [f(x,y)+g(x,y)] = l_{1}+l_{2} $
        \item $ \lim\limits_{(x,y)\to (x_{0}, y_{0})} [a f(x,y)] = al_{1} $
        \item $ \lim\limits_{(x,y)\to (x_{0}, y_{0})} [f(x,y)\cdot g(x,y)] = l_{1}\cdot
            l_{2} $
        \item $ \lim\limits_{(x,y)\to (x_{0}, y_{0})} \frac{f(x,y)}{g(x,y)} = 
            \frac{l_{1}}{l_{2}}, \; l_{2} \neq 0 $
        \item $ \lim\limits_{(x,y)\to (x_{0}, y_{0})} [f(x,y)]^{a} = l^{a}, \; 
            a \in \mathbb{Q}^{*} $
    \end{enumerate}
}

\mythm{%
    \begin{minipage}{0.3\textwidth}
        \begin{enumerate}[i)]
            \item $ \abs{f(x,y)} \leq \abs{g(x,y)} $ \hfill \tikzmark{a}
            \item $ \lim\limits_{(x,y)\to (x_{0}, y_{0})} g(x,y) = 0 $ \hfill \tikzmark{b}
        \end{enumerate}
    \end{minipage}

    \mybrace{a}{b}[$ \lim\limits_{(x,y)\to (x_{0}, y_{0})} f(x,y) = 0 $]
}

\begin{example}
    Να υπολογίσετε το όριο $ \lim\limits_{(x,y)\to (0, 0)} xy \sin{\frac{1}{x}} $. 

    \begin{proof}
    \item {}
        \begin{minipage}{0.4\textwidth}
            \begin{enumerate}[i)]
                \item $ \abs{xy \sin{\frac{1}{x}}} = \abs{xy} 
                    \abs{\sin{\frac{1}{x}}} \leq \abs{xy} \cdot 1 = \abs{xy} $ 
                    \hfill \tikzmark{a}
                \item $ \lim\limits_{(x,y)\to (0, 0)} xy = 0$ \hfill \tikzmark{b}
            \end{enumerate}
        \end{minipage}

        \mybrace{a}{b}[$ \lim\limits_{(x,y)\to (0, 0)} xy 
        \sin{\frac{1}{x}} = 0$]
    \end{proof}
\end{example}

\mythm{%
    \begin{minipage}{0.3\textwidth}
        \begin{enumerate}[i)]
            \item $ \abs{g(x,y)} \leq M \in \mathbb{R} $ \hfill \tikzmark{a}
            \item $ \lim\limits_{(x,y)\to (x_{0}, y_{0})} f(x,y) = 0 $ \hfill \tikzmark{b}
        \end{enumerate}
    \end{minipage}

    \mybrace{a}{b}[$ \lim\limits_{(x,y)\to (x_{0}, y_{0})} [f(x,y)\cdot g(x,y)] = 0 $]
}

\begin{example}
    Να υπολογίσετε το όριο $ \lim\limits_{(x,y)\to (0, 0)} (x^{2}+y^{2}) 
    \sin{\frac{1}{y}} $.  

    \begin{proof}
    \item {}
        \begin{minipage}{0.4\textwidth}
            \begin{enumerate}[i)]
                \item $ \abs{\sin{\frac{1}{y}}} \leq 1 $ \hfill \tikzmark{a}
                \item $ \lim\limits_{(x,y)\to (0, 0)} (x^{2}+y^{2}) = 0+0=0 $ 
                    \hfill \tikzmark{b}
            \end{enumerate}
        \end{minipage}
        \mybrace{a}{b}[$ \lim\limits_{(x,y)\to (0, 0)} (x^{2}+y^{2}) \sin{\frac{1}{y}} 
        = 0$]
    \end{proof}
\end{example}


\section{Διαδοχικά ή Επάλληλα όρια}

Αν για κάθε $ x \neq x_{0}, \; \exists $ το $ \lim_{y \to y_{0}} f(x,y) $, τότε το όριο 
αυτό είναι συνάρτηση του $x$, έστω $ \lim_{y \to y_{0}} f(x,y) = g(x) $ και 
αν υπάρχει και το $ \lim_{x \to x_{0}} g(x) $, ορίζουμε: 

\[
    L_{1} = \lim_{x \to x_{0}} \left(\lim_{y \to y_{0}} f(x,y)\right) 
    \quad \text{και ομοίως} \quad
    L_{2} = \lim_{y \to y_{0}} \left(\lim_{x \to x_{0}} f(x,y)\right) 
 \] 

Έστω ότι $ \lim\limits_{(x,y)\to (x_{0}, y_{0})} f(x,y) = L \in \mathbb{R} $.

Ισχύουν οι παρακάτω περιπτώσεις:
\begin{myitemize}
\item Αν υπάρχουν τα όρια $ L_{1} $ και $ L_{2} $ και $ L_{1} \neq L_{2} $, τότε
    δεν υπάρχει το $ L $
\item Αν υπάρχουν τα όρια $ L_{1} $ και $ L_{2} $ και υπάρχει το $ L $, τότε
    $ L_{1}=L_{2}=L $
\item Αν υπάρχουν τα $ L_{1} $ και $ L_{2} $, τότε δεν εξασφαλίζεται η ύπαρξη του 
    $ L $ ακόμη και αν $ L_{1}=L_{2} $.
\item Η ύπαρξη του $ L_{1} $ δεν εξασφαλίζει την ύπαρξη του $ L_{2} $ και αντιστρόφως.
\item Η ύπαρξη του $ L $ δεν εξασφαλίζει την ύπαρξη των $ L_{1} $ και $ L_{2} $.
\end{myitemize}

\begin{examples}
\item {}  
    \begin{enumerate}
        \item Να υπολογιστεί το όριο $ \lim\limits_{(x,y)\to (0, 0)}
            \frac{x^{2}-y^{2}+x^{3}+y^{3}}{x^{2}+y^{2}} $

            \begin{proof}
                \begin{align*}
                    L_{1} &= \lim_{x \to 0} \left(\lim_{y \to 0}
                    \frac{x^{2}-y^{2}+x^{3}+y^{3}}{x^{2}+y^{2}}\right) = 
                    \lim_{x \to 0} \frac{x^{2}+x^{3}}{x^{2}} = \lim_{x \to 0} (1+x) = 1
                    \intertext{και}
                        L_{2} &= \lim_{y \to 0} \left( \lim_{x \to 0} 
                        \frac{x^{2}-y^{2}+x^{3}+y^{3}} {x^{2}+y^{2}}\right) = 
                        \lim_{y \to 0} \frac{y^{3}-y^{2}}{y^{2}} = 
                        \lim_{y \to 0} (y - 1) = -1
                        \end{align*} 
                 Επομένως αφού $ L_{1} \neq L_{2} \Rightarrow \not \exists L $ 
            \end{proof}

        \item Να υπολογιστεί το όριο $ \lim\limits_{(x,y)\to (0, 0)} 
            \frac{xy}{x^{2}+y^{2}} $.

            \begin{proof}
                \begin{align*}
                    L_{1} &= \lim_{x \to 0} \left(\lim_{y \to 0} 
                    \frac{xy}{x^{2}+y^{2}}\right) = \lim_{x \to 0} \frac{0}{x^{2}} = 
                    \lim_{x \to 0} 0 = 0
                    \intertext{και}
                        L_{2} &= \lim_{y \to 0} \left(\lim_{x \to 0} 
                        \frac{xy}{x^{2}+y^{2}}\right) = \lim_{y \to 0} 
                        \frac{0}{y^{2}} = \lim_{y \to 0} 0 = 0
                        \end{align*}
            \end{proof}
            Παρατηρούμε ότι $ L_{1} = L_{2} $, όμως δεν γνωρίζουμε αν υπάρχει το $L$, 
            οπότε δεν μπορούμε να ισχυριστούμε ότι $ L=0 $.

            \begin{description}
                \item [Β᾽ Τρόπος: (Καμπύλες)]
                    \[
                        \lim\limits_{(x,y)\to (0, 0)} \frac{xy}{x^{2}+y^{2}} 
                        \xlongequal[\lambda \in \mathbb{R}]{y= \lambda x} \lim_{x \to 0}
                        \frac{\lambda x^{2}}{x^{2}+ \lambda^{2} x^{2}} = \lim_{x \to 0} 
                        \frac{\lambda x^{2}}{x^{2}(\lambda ^{2}+1)} = \lim_{x \to 0} 
                        \frac{\lambda}{\lambda ^{2}+1} 
                    \]
                    Οπότε δεν υπάρχει το όριο, γιατί εξαρτάται από την τιμή του 
                    $ \lambda $ και άρα από την καμπύλη προσέγγισης του σημείου 
                    $ (0,0) $.
            \end{description}
    \end{enumerate}
\end{examples}


\section{Υπολογισμός Ορίων μέσω καμπυλών}

Χρησιμοποιούμε καμπύλες της μορφής 
\[ 
    x - x_{0} = \lambda (y- y_{0})^{n} \quad \text{ή} 
    \quad y- y_{0} = \lambda (x- x_{0})^{n}, \quad l \in \mathbb{R}, \; n \in 
    \mathbb{N}  
\]

\begin{rem}
    Δυστυχώς ο τρόπος αυτός δεν εξασφαλίζει την ύπαρξη του ορίου, όταν αυτό υπάρχει, 
    όμως χρησιμοποείται για να δείξουμε ότι ένα όριο, δεν υπάρχει.
\end{rem}

\begin{examples}
\item {}
    \begin{enumerate}
        \item Να υπολογιστεί το όριο $ \lim\limits_{\substack{x\to 0 \\y \to 0}}
            \frac{x^{2}y}{x^{4}+y^{2}} $.

            \begin{proof}
                \[
                    \lim\limits_{\substack{x\to 0 \\y \to 0}} 
                    \frac{x^{2}y}{x^{2}+y^{2}} 
                    \xlongequal[\lambda \in \mathbb{R}]{y= \lambda x^{2}} 
                    \lim_{x \to 0} \frac{x^{2} \lambda x^{2}}
                    {x^{4}+\lambda^{2}x^{4}} = \lim_{x\to 0} 
                    \frac{\lambda x^{4}}{x^{4}(1+\lambda ^{2})} = 
                    \lim_{x \to 0} \frac{\lambda}{\lambda ^{2}+1} 
                \] 
                Οπότε δεν υπάρχει το όριο, γιατί εξαρτάται από την τιμή του $ \lambda $ 
                και άρα από την καμπύλη προσέγγισης του σημείου $ (0,0) $.
            \end{proof}

        \item Να υπολογιστεί το όριο $ \lim\limits_{(x,y)\to (0, 0)} \frac{x^{2}-y^{4}}
            {2x^{2}+y^{4}} $. 

            \begin{proof}
                \[
                    \lim\limits_{(x,y)\to (0, 0)} \frac{x^{2}-y^{4}}
                    {2x^{2}+y^{4}} \xlongequal[\lambda \in \mathbb{R}]
                    {x= \lambda y^{2}} \lim_{y \to 0} 
                    \frac{\lambda ^{2}y^{4}-y^{4}}{2 \lambda ^{2}y^{4}+y^{4}} = 
                    \lim_{y \to 0} \frac{y^{4}(\lambda ^{2}-1)}
                    {y^{4}(2 \lambda ^{2}+1)} = 
                    \frac{\lambda ^{2}-1}{2 \lambda ^{2}+1} 
                \] 
                Οπότε δεν υπάρχει το όριο, γιατί εξαρτάται από την τιμή του $ \lambda $ 
                και άρα από την καμπύλη προσέγγισης του σημείου $ (0,0) $.
            \end{proof}
    \end{enumerate}
\end{examples}

\begin{example}[Θέμα Εξετάσεων]
    Να υπολογιστεί το όριο $ \lim\limits_{(x,y)\to (0, 0)} \frac{x^{3}y^{3}}{x^{3}y^{3}+ (x-y)^{3}} $ 
    \begin{proof}
    \item {}
        \begin{align*}
            \lim\limits_{\substack{x \to 0\\ y \to 0}} 
            \frac{x^{3}y^{3}}{x^{3}y^{3} + (x-y)^{3}} 
    &\xlongequal[\lambda \in \mathbb{R}]{y = \lambda x} \lim_{x \to 0} 
    \frac{x^{3} \lambda ^{3} x^{3}}{x^{3} \lambda ^{3} x^{3} + (x - \lambda x)^{3}} 
    = \lim_{x \to 0} \frac{\lambda ^{3}x^{6}}{x^{3}
    (\lambda ^{3} x^{3} + (1- \lambda)^{3})} \\
    &= \lim_{x \to 0} \frac{\lambda ^{3} x^{3}}{\lambda ^{3}x^{3} + (1- \lambda)^{3}} 
    = 
    \begin{cases} 
        0, & \lambda \neq 1 \\ 
        1, & \lambda = 1 
    \end{cases} 
        \end{align*} 
        Οπότε δεν υπάρχει το όριο, γιατί εξαρτάται από την τιμή του 
        $ \lambda $ και άρα από την καμπύλη προσέγγισης του σημείου 
        $ (0,0) $.
    \end{proof}
\end{example}

\section{Υπολογισμός Ορίων με χρήση Πολικών Συντεταγμένων}

Η μέθοδος αυτή βασίζεται στη χρήση πολικών συντεταγμένων με κέντρο το σημείο 
$ (x_{0}, y_{0}) $, στο οποίο θέλουμε να υπολογίσουμε το όριο, και μας εξασφαλίζει 
την ύπαρξη του ορίου, όταν αυτό υπάρχει.

\begin{rem}
\item {}
    \begin{myitemize}
    \item Ο μετασχηματισμός των Πολικών Συντεταγμένων, είναι:
        \[
            x - x_{0} = r \cos{\theta} \quad \text{και} \quad y - y_{0} = 
            r \sin{\theta}, \; r \geq 0, \; \theta \in [0, 2 \pi]
        \] 
        και ισχύει ότι 
        \[
            r = \sqrt{(x- x_{0})^{2}+(y- y_{0})^{2}} \quad \text{και} \quad \theta = 
            \frac{y}{x}.
        \] 
        Ο μετασχηματισμός του ορίου, τότε είναι 
        \[
            \lim\limits_{(x,y)\to (x_{0}, y_{0})} f(x,y) = 
            \lim_{r \to 0} f(x(r, \theta ), y(r, \theta)) = L
        \] 


    \item Συνήθως η μέθοδος αυτή εφαρμόζεται όταν το σημείο στο οποίο θέλουμε να 
        υπολογίσουμε το όριο είναι το $ (0,0) $ και όταν η συνάρτηση είναι ρητή ή 
        περιέχει παραστάσεις της μορφής $ x^{2}+y^{2} $. Αν το όριο βγεί 
        πραγματικός αριθμός ανεξάρτητος του $\theta$, τότε το όριο υπάρχει 
    \end{myitemize}
\end{rem}

\begin{examples}
\item {}
    \begin{enumerate}
        \item 
            \begin{align*} \lim\limits_{(x,y)\to (0, 0)} 
                \frac{x^{2}-2y^{2}+2x^{3}-y^{3}}{x^{2}+y^{2}} 
               &= \lim_{r \to 0} \frac{r^{2} \cos^{2}{\theta} - 2r^{2} \sin^{2}{\theta 
               +2r^{3} \cos^{3}{\theta - r^{3} \sin^{3}{\theta}}}}{r^{2}} \\ 
               &= \lim_{r \to 0} \frac{r^{2}(\cos^{2}{\theta} - 2 \sin^{2}{\theta} +2r 
               \cos^{3}{\theta} - r \sin^{3}{\theta})}{r^{2}} = 
               \cos^{2}{\theta} - 2 \sin^{2}{\theta}, \; \forall \theta \in [0, 2 \pi]
            \end{align*}
            Επομένως δεν υπάρχει το όριο γιατί εξαρτάται από το $\theta$.

        \item 
            \begin{align*}
                \lim\limits_{(x,y)\to (0, 0)} \frac{x^{3}y}{x^{2}+y^{2}} &= 
                \lim_{r \to 0} \frac{r^{3} \cos^{3}{\theta} \cdot r 
                    \sin{\theta}}{r^{2}} = \lim_{r \to 0} \frac{r^{4} \sin{\theta} 
                \cos^{3}{\theta}}{r^{2}} = 
                \lim_{r \to 0} r^{2} \sin{\theta} \cos^{3}{\theta} = 0
            \end{align*}
            Επομένως το όριο είναι ανεξάρτητο του $\theta$, άρα 
            \[
                \lim\limits_{(x,y)\to (0,0)} \frac{x^{3}y}{x^{2}+y^{2}} = 0
            \] 

        \item 
            \begin{align*}
                \lim\limits_{(x,y)\to (0, 0)} \frac{\sqrt{x^{2}+y^{2}+1}-1}
                {x^{2}+y^{2}} = \lim_{r \to 0} \frac{\sqrt{r^{2}+1} -1}{r^{2}} 
                \overset{\left(\frac{0}{0}\right)}{\underset{\text{LH}}{=}} 
                \lim_{r \to 0} \frac{\frac{r}{\sqrt{r^{2}+1}}}{2r} = \frac{1}{2}
            \end{align*}
            Επομένως το όριο είναι ανεξάρτητο του $\theta$, άρα 
            \[
                \lim\limits_{(x,y)\to (0, 0)} 
                \frac{\sqrt{x^{2}+y^{2}+1} -1}{x^{2}+y^{2}} = \frac{1}{2} 
            \] 

        \item 
            \[ 
                \lim\limits_{(x,y)\to (0, 0)} \frac{x^{3}+y^{3}}{x^{2}+y^{2}} = \lim_{r \to 0}
                \frac{r^{3} \cos^{3}{\theta} + r^{3} \sin^{3}{\theta}}{r^{2} \cos^{2}{\theta} +
                r^{2} \sin^{2}{\theta}} = \lim_{r \to 0} 
                r(\cos^{3}{\theta} + \sin^{3}{\theta}) = 0
            \] 
            Επομένως το όριο είναι ανεξάρτητο του $\theta$ 
            άρα 
            \[
                \lim\limits_{(x,y)\to (0, 0)} \frac{x^{3}+y^{3}}{x^{2}+y^{2}} = 0 
             \] 
    \end{enumerate}
\end{examples}

\section{Άλλα Όρια}

\begin{enumerate}
    \item Να υπολογιστούν τα παρακάτω όρια
        \begin{enumerate}[i)]
            \item 
        \end{enumerate}
\end{enumerate}

\end{document}
