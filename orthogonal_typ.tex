\documentclass[a4paper,12pt]{article}
\usepackage{etex}
%%%%%%%%%%%%%%%%%%%%%%%%%%%%%%%%%%%%%%
% Babel language package
\usepackage[english,greek]{babel}
% Inputenc font encoding
\usepackage[utf8]{inputenc}
%%%%%%%%%%%%%%%%%%%%%%%%%%%%%%%%%%%%%%

%%%%% math packages %%%%%%%%%%%%%%%%%%
\usepackage{amsmath}
\usepackage{amssymb}
\usepackage{amsfonts}
\usepackage{amsthm}
\usepackage{proof}

\usepackage{physics}

%%%%%%% symbols packages %%%%%%%%%%%%%%
\usepackage{dsfont}
\usepackage{stmaryrd}
%%%%%%%%%%%%%%%%%%%%%%%%%%%%%%%%%%%%%%%


%%%%%% graphicx %%%%%%%%%%%%%%%%%%%%%%%
\usepackage{graphicx}
\usepackage{color}
%\usepackage{xypic}
\usepackage[all]{xy}
\usepackage{calc}
%%%%%%%%%%%%%%%%%%%%%%%%%%%%%%%%%%%%%%%

\usepackage{enumerate}

\usepackage{fancyhdr}
%%%%% header and footer rule %%%%%%%%%
\setlength{\headheight}{14pt}
\renewcommand{\headrulewidth}{0pt}
\renewcommand{\footrulewidth}{0pt}
\fancypagestyle{plain}{\fancyhf{}
\fancyhead{}
\lfoot{}
\rfoot{\small \thepage}}
\fancypagestyle{vangelis}{\fancyhf{}
\rhead{\small \leftmark}
\lhead{\small }
\lfoot{}
\rfoot{\small \thepage}}
%%%%%%%%%%%%%%%%%%%%%%%%%%%%%%%%%%%%%%%

\usepackage{hyperref}
\usepackage{url}
%%%%%%% hyperref settings %%%%%%%%%%%%
\hypersetup{pdfpagemode=UseOutlines,hidelinks,
bookmarksopen=true,
pdfdisplaydoctitle=true,
pdfstartview=Fit,
unicode=true,
pdfpagelayout=OneColumn,
}
%%%%%%%%%%%%%%%%%%%%%%%%%%%%%%%%%%%%%%



\usepackage{geometry}
\geometry{left=25.63mm,right=25.63mm,top=36.25mm,bottom=36.25mm,footskip=24.16mm,headsep=24.16mm}

%\usepackage[explicit]{titlesec}
%%%%%% titlesec settings %%%%%%%%%%%%%
%\titleformat{\chapter}[block]{\LARGE\sc\bfseries}{\thechapter.}{1ex}{#1}
%\titlespacing*{\chapter}{0cm}{0cm}{36pt}[0ex]
%\titleformat{\section}[block]{\Large\bfseries}{\thesection.}{1ex}{#1}
%\titlespacing*{\section}{0cm}{34.56pt}{17.28pt}[0ex]
%\titleformat{\subsection}[block]{\large\bfseries{\thesubsection.}{1ex}{#1}
%\titlespacing*{\subsection}{0pt}{28.80pt}{14.40pt}[0ex]
%%%%%%%%%%%%%%%%%%%%%%%%%%%%%%%%%%%%%%

%%%%%%%%% My Theorems %%%%%%%%%%%%%%%%%%
\newtheorem{thm}{Θεώρημα}[section]
\newtheorem{cor}[thm]{Πόρισμα}
\newtheorem{lem}[thm]{λήμμα}
\theoremstyle{definition}
\newtheorem{dfn}{Ορισμός}[section]
\newtheorem{dfns}[dfn]{Ορισμοί}
\theoremstyle{remark}
\newtheorem{remark}{Παρατήρηση}[section]
\newtheorem{remarks}[remark]{Παρατηρήσεις}
%%%%%%%%%%%%%%%%%%%%%%%%%%%%%%%%%%%%%%%




\input{definitions_ask.tex}

\pagestyle{vangelis}
\everymath{\displaystyle}

\begin{document*}


% \begin{center}
%   \minibox{\Large\bfseries \textcolor{Col1}{Τύποι Ορθογωνιότητας}}
% \end{center}


\chapter*{Σειρές Fourier}


\section*{Σειρά Fourier συνάρτησης με περίοδο $ 2 \pi $}

\begin{thm}
  Έστω $ f \colon \mathbb{R} \to \mathbb{R} $ μια \textbf{περιοδική} συνάρτηση με 
  περίοδο $ 2 \pi $ και \textbf{τμηματικά λεία}. Τότε η σειρά Fourier της $f$ 
  συγκλίνει για κάθε $ x \in \mathbb{R} $, στην τιμή
  \begin{align*}
    f(x) &= \frac{a_{0}}{2} + \sum_{n=1}^{\infty} [a_{n} \cos{nx} + b_{n} \sin{nx}], 
    \quad \forall x \; \text{όπου $x$ είναι σημείο \textbf{συνέχειας} της $f$} \\
    \frac{f(x_{0}^{-}) + f(x_{0}^{+})}{2} &= \frac{a_{0}}{2} + \sum_{n=1}^{\infty} 
    [a_{n} \cos{nx} + b_{n} \sin{nx}], 
    \quad \forall x_{0} \; \text{όπου $x_{0}$ είναι σημείο \textbf{ασυνέχειας} της $f$}\\
  \end{align*} 
  όπου οι συντελεστές fourier δίνονται από τις σχέσεις:
  \[
  \boxed{a_{n} = \frac{1}{\pi} \int _{- \pi }^{\pi} f(x) \cos{nx} \,{dx}, \quad
  n = \textcolor{Col1}{0},1,2,3, \ldots} \quad \text{και} \quad 
  \boxed{b_{n} = \frac{1}{\pi} \int _{- \pi }^{\pi} f(x) \sin{nx} \,{dx}, \quad n =
  1,2,3, \ldots}
   \] 
\end{thm}


\section*{Σειρά Fourier συνάρτησης με περίοδο $ 2 L$}

\begin{thm}
  Έστω $ f \colon \mathbb{R} \to \mathbb{R} $ μια \textbf{περιοδική} συνάρτηση με 
  περίοδο $ 2 L $ και \textbf{τμηματικά λεία}. Τότε η σειρά Fourier της $f$ 
  συγκλίνει για κάθε $ x \in \mathbb{R} $, στην τιμή
  \begin{align*}
    f(x) &= \frac{a_{0}}{2} + \sum_{n=1}^{\infty} \left[a_{n} 
    \cos{\left(\frac{n \pi x}{L}\right)} + b_{n} \sin{\left(\frac{n \pi x}{L} \right)}
  \right], 
    \quad \forall x \; \text{όπου $x$ είναι σημείο \textbf{συνέχειας} της $f$} \\
    \frac{f(x_{0}^{-}) + f(x_{0}^{+})}{2} &= \frac{a_{0}}{2} + \sum_{n=1}^{\infty} 
    \left[a_{n} \cos{\left(\frac{n \pi x}{L}\right)} + b_{n} 
    \sin{\left(\frac{n \pi x}{L}\right)}\right], 
    \quad \forall x_{0} \; \text{όπου $x_{0}$ είναι σημείο \textbf{ασυνέχειας} της $f$}\\
  \end{align*} 
  όπου οι συντελεστές fourier δίνονται από τις σχέσεις:
  \[
    \boxed{a_{n} = \frac{1}{L} \int _{- L }^{L} f(x) \cos{\left(\frac{n \pi x}{L}\right)} \,{dx}, \quad
  n = \textcolor{Col1}{0},1,2,3, \ldots} \quad \text{και} \quad 
  \boxed{b_{n} = \frac{1}{L} \int _{- L }^{L} f(x) \sin{\left(\frac{n \pi x}{L}\right)} \,{dx}, \quad n =
  1,2,3, \ldots}
   \] 
\end{thm}

\begin{rem}
  $ f({x_{0}}^{-}) = \lim\limits_{x \to {x_{0}}^{-}} f(x) $ και 
  $ f({x_{0}}^{+}) = \lim\limits_{x \to {x_{0}}^{+}} f(x) $ 
\end{rem}


\chapter*{Ημιτονική και Συνημιτονική σειρά Fourier}

\section*{Συνημιτονική σειρά Fourier συνάρτησης με περίοδο $ 2 \pi $}

\begin{thm}
  Έστω $ f \colon \mathbb{R} \to \mathbb{R} $ μια \textbf{\textcolor{Col1}{άρτια}} 
  \textbf{περιοδική} συνάρτηση με περίοδο $ 2 \pi $ και \textbf{τμηματικά λεία}. 
  Τότε η σειρά Fourier της $f$ συγκλίνει για κάθε $ x \in \mathbb{R} $, στην τιμή
  \begin{align*}
    f(x) &= \frac{a_{0}}{2} + \sum_{n=1}^{\infty} a_{n} 
    \cos{(n x)}, 
    \quad \forall x \; \text{όπου $x$ είναι σημείο \textbf{συνέχειας} της $f$} \\
    \frac{f(x_{0}^{-}) + f(x_{0}^{+})}{2} &= \frac{a_{0}}{2} + \sum_{n=1}^{\infty} 
    a_{n} \cos{(nx)},
    \quad \forall x_{0} \; \text{όπου $x_{0}$ είναι σημείο \textbf{ασυνέχειας} της $f$}\\
  \end{align*} 
  όπου οι συντελεστές fourier δίνονται από τις σχέσεις:
  \[
    \boxed{a_{n} = \frac{2}{\pi} \int _{0}^{\pi} f(x) \cos{(nx)} 
    \,{dx}, \quad n = \textcolor{Col1}{0},1,2,3, \ldots} \quad \text{και} \quad 
    \boxed{b_{n} = 0 \quad n = 1,2,3, \ldots}
    \] 
  \end{thm}

\section*{Ημιτονική σειρά Fourier συνάρτησης με περίοδο $ 2 \pi $}

\begin{thm}
  Έστω $ f \colon \mathbb{R} \to \mathbb{R} $ μια \textbf{\textcolor{Col1}{περιττή}} 
  \textbf{περιοδική} συνάρτηση με περίοδο $ 2 \pi $ και \textbf{τμηματικά λεία}. 
  Τότε η σειρά Fourier της $f$ συγκλίνει για κάθε $ x \in \mathbb{R} $, στην τιμή
  \begin{align*}
    f(x) &= \sum_{n=1}^{\infty} b_{n} \sin{(nx)}, 
    \quad \forall x \; \text{όπου $x$ είναι σημείο \textbf{συνέχειας} της $f$} \\
    \frac{f(x_{0}^{-}) + f(x_{0}^{+})}{2} &= \sum_{n=1}^{\infty} 
    b_{n} \sin{(nx)},
    \quad \forall x_{0} \; \text{όπου $x_{0}$ είναι σημείο \textbf{ασυνέχειας} της $f$}\\
  \end{align*} 
  όπου οι συντελεστές fourier δίνονται από τις σχέσεις:
  \[
    \boxed{b_{n} = \frac{2}{\pi} \int _{0}^{\pi} f(x) \sin{(nx)} 
    \,{dx}, \quad n = 1,2,3, \ldots} \quad \text{και} \quad 
    \boxed{a_{n} = 0 \quad n = 0,1,2,3, \ldots}
    \] 
  \end{thm}


\section*{Συνημιτονική σειρά Fourier συνάρτησης με περίοδο $ 2 L $}

\begin{thm}
  Έστω $ f \colon \mathbb{R} \to \mathbb{R} $ μια \textbf{\textcolor{Col1}{άρτια}} 
  \textbf{περιοδική} συνάρτηση με περίοδο $ 2 L $ και \textbf{τμηματικά λεία}. 
  Τότε η σειρά Fourier της $f$ συγκλίνει για κάθε $ x \in \mathbb{R} $, στην τιμή
  \begin{align*}
    f(x) &= \frac{a_{0}}{2} + \sum_{n=1}^{\infty} a_{n} 
    \cos{\left(\frac{n \pi x}{L}\right)}, 
    \quad \forall x \; \text{όπου $x$ είναι σημείο \textbf{συνέχειας} της $f$} \\
    \frac{f(x_{0}^{-}) + f(x_{0}^{+})}{2} &= \frac{a_{0}}{2} + \sum_{n=1}^{\infty} 
    a_{n} \cos{\left(\frac{n \pi x}{L}\right)},
    \quad \forall x_{0} \; \text{όπου $x_{0}$ είναι σημείο \textbf{ασυνέχειας} της $f$}\\
  \end{align*} 
  όπου οι συντελεστές fourier δίνονται από τις σχέσεις:
  \[
    \boxed{a_{n} = \frac{2}{L} \int _{0}^{L} f(x) \cos{\left(\frac{n \pi x}{L}\right)} 
    \,{dx}, \quad n = \textcolor{Col1}{0},1,2,3, \ldots} \quad \text{και} \quad 
    \boxed{b_{n} = 0 \quad n = 1,2,3, \ldots}
    \] 
  \end{thm}

\section*{Ημιτονική σειρά Fourier συνάρτησης με περίοδο $ 2 L $}

\begin{thm}
  Έστω $ f \colon \mathbb{R} \to \mathbb{R} $ μια \textbf{\textcolor{Col1}{περιττή}} 
  \textbf{περιοδική} συνάρτηση με περίοδο $ 2 L $ και \textbf{τμηματικά λεία}. 
  Τότε η σειρά Fourier της $f$ συγκλίνει για κάθε $ x \in \mathbb{R} $, στην τιμή
  \begin{align*}
    f(x) &= \sum_{n=1}^{\infty} b_{n} \sin{\left(\frac{n \pi x}{L}\right)}, 
    \quad \forall x \; \text{όπου $x$ είναι σημείο \textbf{συνέχειας} της $f$} \\
    \frac{f(x_{0}^{-}) + f(x_{0}^{+})}{2} &= \sum_{n=1}^{\infty} 
    b_{n} \sin{\left(\frac{n \pi x}{L}\right)},
    \quad \forall x_{0} \; \text{όπου $x_{0}$ είναι σημείο \textbf{ασυνέχειας} της $f$}\\
  \end{align*} 
  όπου οι συντελεστές fourier δίνονται από τις σχέσεις:
  \[
    \boxed{b_{n} = \frac{2}{L} \int _{0}^{L} f(x) \sin{\left(\frac{n \pi x}{L}\right)} 
    \,{dx}, \quad n = 1,2,3, \ldots} \quad \text{και} \quad 
    \boxed{a_{n} = 0 \quad n = 0,1,2,3, \ldots}
    \] 
  \end{thm}



\chapter*{Σχέσεις ορθογωνιότητας}

\twocolumnsides{
  \section*{Ορθογωνιότητα στο $ [- \pi, \pi] $}
\renewcommand{\arraystretch}{4.5}
 \begin{tabular}{l}
   $\int _{-\pi}^{\pi} \cos{(nx)} \cos{(mx)} \,{dx} = 
  \begin{cases}
    0, & n \neq m \\
    \pi, &n=m \neq 0 \\
    2\pi, &n=m=0 
  \end{cases} $ \\
  $ \int _{-\pi}^{\pi} \sin{(nx)} \sin{(mx)} \,{dx} = 
  \begin{cases}
    0, & n \neq m \\
    \pi, &n=m \neq 0 \\
  \end{cases} $ \\
  $ \int _{-\pi}^{\pi} \sin{(nx)} \cos{(mx)} \,{dx} = 0, \quad 
  \forall n,m $ \\
 \end{tabular} 
  }{
  \section*{Ορθογωνιότητα στο $ [0, \pi] $}
\renewcommand{\arraystretch}{4.5}
\begin{tabular}{l}
  $ \int _{0}^{\pi} \cos{(nx)} \cos{(mx)} \,{dx} = 
  \begin{cases}
    0, & n \neq m \\
    {\pi}/{2}, &n=m \neq 0 \\
    \pi, &n=m=0 
  \end{cases} $ \\
  $ \int _{0}^{\pi} \sin{(nx)} \sin{(mx)} \,{dx} = 
  \begin{cases}
    0, & n \neq m \\
    {\pi}/{2}, &n=m \neq 0 \\
  \end{cases} $ \\
  $ \int _{0}^{\pi} \sin{(nx)} \cos{(mx)} \,{dx} = 0, \quad 
  \forall n,m $
\end{tabular}
}

\vspace{2\baselineskip}

\twocolumnsides{
  \section*{Ορθογωνιότητα στο $ [- L, L] $}
\renewcommand{\arraystretch}{4.5}
 \begin{tabular}{l}
  $\int _{-L}^{L} \cos{\left(\frac{n \pi x}{L}\right)} 
  \cos{\left(\frac{m \pi x}{L}\right)} \,{dx} = 
  \begin{cases}
    0, & n \neq m \\
    L, &n=m \neq 0 \\
    2L, &n=m=0 
  \end{cases} $ \\
  $ \int _{-L}^{L} \sin{\left(\frac{n \pi x}{L}\right)} 
  \sin{\left(\frac{m \pi x}{L}\right)} \,{dx} = 
  \begin{cases}
    0, & n \neq m \\
    L, &n=m \neq 0 \\
  \end{cases} $ \\
  $ \int _{-L}^{L} \sin{\left(\frac{n \pi x}{L}\right)} 
  \cos{\left(\frac{m \pi x}{L}\right)} \,{dx} = 0, \quad 
  \forall n,m $ \\
 \end{tabular} 
  }{
  \section*{Ορθογωνιότητα στο $ [0, L] $}
\renewcommand{\arraystretch}{4.5}
\begin{tabular}{l}
  $ \int _{0}^{L} \cos{\left(\frac{n \pi x}{L}\right)} 
  \cos{\left(\frac{m \pi x}{L}\right)} \,{dx} = 
  \begin{cases}
    0, & n \neq m \\
    {L}/{2}, &n=m \neq 0 \\
    L, &n=m=0 
  \end{cases} $ \\
  $ \int _{0}^{L} \sin{\left(\frac{n \pi x}{L}\right)} 
  \sin{\left(\frac{m \pi x}{L}\right)} \,{dx} = 
  \begin{cases}
    0, & n \neq m \\
    {L}/{2}, &n=m \neq 0 \\
  \end{cases} $ \\
  $ \int _{0}^{L} \sin{\frac{n \pi x}{L}} \cos{\frac{m \pi x}{L}} \,{dx} = 0, \quad 
  \forall n,m $
\end{tabular}
}



\end{document*}
