\input{preamble_ask.tex}
\input{definitions_ask.tex}

\geometry{top=1cm}

\everymath{\displaystyle}
\pagestyle{askhseis}

\begin{document}


\chapter*{Ρυθμός Μεταβολής}

\section*{Μεθοδολογία επίλυσης Προβλημάτων}

\begin{enumerate}
  \item Διαβάζουμε προσεκτικά το πρόβλημα.
  \item Σχεδιάζουμε σχήμα, αν είναι εύκολο.
  \item Αντιστοιχίζουμε μεταβλητές στις ποσότητες του προβλήματος που είναι συναρτήσεις 
    του χρόνου.
  \item Εκφράζουμε τα κατάλληλα δεδομένα του προβλήματος και το ζητούμενο ρυθμό 
    μεταβολής με τη βοήθεια παραγώγων.
  \item Βρίσκουμε μια σχέση που συνδέει μεταξύ τους τις ποσότητες του προβλήματος. 
    Αν είναι απαραίτητο απαλοίφουμε κάποια απο τις μεταβλητές χρησιμοποιώντας 
    κατάλληλες σχέσεις.
  \item Παραγωγίζουμε με τον κανόνα αλυσίδας και τα δύο μέλη της σχέσης αυτής ως 
    προς $t$.
  \item Αντικαθιστούμε τα δεδομένα του προβλήματος στην σχέση που προκύπτει και λύνουμε 
    ως προς το ζητούμενο ρυθμό μεταβολής.
\end{enumerate}


\section*{Προβλήματα στο ρυθμό μεταβολής}

\begin{enumerate}
  \item Φουσκώνουμε με αέρα ένα σφαιρικό μπαλόνι, έτσι ώστε ο όγκος του να αυξάνει 
    με ρυθμό $100$ \si{cm^{3}/s}. Πόσο γρήγορα αυξάνει η ακτίνα του μπαλονιού όταν η 
    διάμετρος του είναι $50$ \si{cm}?

    \hfill Απ: $\frac{1}{25\pi}\si{cm\per s}$

  \item Μια σκάλα μήκους $10$ \si{m} είναι ακουμπησμένη σε ένα κατακόρυφο τοίχο. 
    Εάν το κάτω μέρος (βάση) της σκάλας ολισθαίνει οριζόντια, απομακρυνόμενη από 
    τον τοίχο με ρυθμό $1$ \si{m\per s}, πόσο γρήγορα η κορυφή της σκάλας πέφτει 
    (ολισθαίνοντας κατακόρυφα στον τοίχο), όταν η βάση της σκάλας απέχει 
    $6$ \si{cm} από τον τοίχο?

    \hfill Απ: $-\SI[quotient-mode=fraction]{3/4}{m/s}$

  \item Μια δεξαμενή νερού έχει το σχήμα ενός ανάποδου κυκλικού κώνου με ακτίνα βάσης 
    $2$ \si{m} και ύψος $4$ \si{m}. Αν στη δεξαμενή εισέρχεται ποσότητα νερού με ρυθμό 
    $2$ \si{m^{3}/min}, να υπολογίσετε το ρυθμό με τον οποίο ανέρχεται η στάθμη του 
    νερού στο εσωτερικό της δεξαμενής, όταν το νερό έχει βάθος $3$ \si{m}.

    \hfill Απ: $\frac{8}{9\pi}$

  \item Αυτοκίνητο $A$ ταξιδεύει δυτικά με ταχύτητα $50$ \si{km\per h} και αυτοκίνητο 
    $B$ ταξιδεύει βόρεια με ταχύτητα $60$ \si{km/h}. Και τα δύο αυτοκίνητα κινούνται 
    προς την διαστάυρωση των δύο δρόμων. Με τι ρυθμό τα δύο αυτοκίνητα πλησιάζουν το ένα 
    το άλλο, όταν το $A$ βρίσκεται $0.3$ \si{km} και το $B$ $0.4$ \si{km} από τη 
    διασταύρωση?

    \hfill Απ: $\SI{-78}{km/h}$

  \item Ένας άνδρας περπατά σε ευθύγραμμο μονοπάτι με ταχύτητα \SI{4}{m/s}. Μια δέσμη 
    leiser η οποία είναι τοποθετημένη σε απόσταση \SI{20}{m} από το μονοπάτι, τον 
    σημαδεύει συνεχώς. Με ποιο ρυθμό περιστρέφεται η δέσμη leiser οταν ο άνδρας 
    βρίσκεται σε απόσταση \SI{15}{m} από εκείνο το σημείο στο μονοπάτι το οποίο απέχει 
    ελάχιστη απόσταση από την πηγή leiser?

    \hfill Απ: $\SI{0.128}{rad/s}$

  \item Πλοίο Α ξεκινάει από ένα λιμάνι στις 12 μ.μ. και κατευθύνεται δυτικά
    με ταχύτητα  \SI{9}{km/h}. Πλοίο Β ξεκινάει από το ίδιο λιμάνι στη 1
    μ.μ. και κατευθύνεται νότια με ταχύτητα \SI{12}{km/h}. Με τι ρυθμό
    απομακρύνονται μεταξύ τους τα δύο πλοία στις 3 μ.μ.?

    \hfill Απ: $\SI{14,70}{km/h}$
\end{enumerate}


\end{document}
