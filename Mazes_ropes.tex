\input{../preamble/preamble.tex}
\newcommand{\vect}[2]{(#1_1,\ldots, #1_#2)}
%%%%%%% nesting newcommands $$$$$$$$$$$$$$$$$$$
\newcommand{\function}[1]{\newcommand{\nvec}[2]{#1(##1_1,\ldots, ##1_##2)}}

\newcommand{\linode}[2]{#1_n(x)#2^{(n)}+#1_{n-1}(x)#2^{(n-1)}+\cdots +#1_0(x)#2=g(x)}

\newcommand{\vecoffun}[3]{#1_0(#2),\ldots ,#1_#3(#2)}





\everymath{\displaystyle}

\begin{document}


\begin{center}
\fbox{\large\bfseries Εφαρμογές Διπλού Ολοκληρώματος}
\end{center}

\vspace{\baselineskip}

Έστω ότι το χωρίο $D$ είναι ένα υλικό φύλλο  και ότι η πυκνότητα της μάζας σε κάθε σημείο του $(x,y)$ δίνεται από την συνεχή συνάρτηση $\delta=\delta(x,y)$.

Η συνολική μάζα $M$ που κατανέμεται στο χωρίο $D$ δίνεται από τον τύπο:

\[
   \text{\bfseries Μάζα:} \quad M=\iint_{D}\delta(x,y)\,dxdy
\]

Οι στατικές (πρώτες) ροπές ως προς τους άξονες $x$ και $y$, δίνονται αντίστοιχα από τους τύπους:

\[
  \begin{tabular}{>{\bfseries}l<{}>{$}l<{$}}
    Στατικές Ροπές: & M_{x}=\iint_{D}y\delta(x,y)\,dxdy \\
      &  M_{y}=\iint_{D}x\delta(x,y)\,dxdy \\
  \end{tabular}
\]

Οι συντεταγμένες $(\overline{x},\overline{y})$ του κέντρου μάζας $K$ του χωρίου $D$ δίνονται από τους τύπους:

\[
  \text{\bfseries Κέντρο Μάζας:} \quad \overline{x}=\frac{M_{y}}{M},\; \overline{y}=\frac{M_{x}}{M}
\]

Οι ροπές αδράνειας (δεύτερες ροπές) ως προς τους άξονες $x$ και $y$, η ροπή αδράνειας ως προς τυχαία ευθεία $L$ με $r(x,y)$ να είναι η απόσταση του $(x,y)$ από την $L$ και η πολική ροπή ως προς την αρχή των αξόνων, δίνονται αντίστοιχα από τους τύπους:

\[
  \begin{tabular}{>{\bfseries}l<{}>{$}l<{$}}
     Ροπές Αδράνειας: & Ι_{x}=\iint_{D}y^{2}\delta(x,y)\,dxdy \\
      &  Ι_{y}=\iint_{D}x^{2}\delta(x,y)\,dxdy \\
      &  I_{L}=\iint_{D}r^{2}(x,y)\delta(x,y)\,dxdy \\
      &  I_{o}=\iint_{D}(x^{2}+y^{2})\delta(x,y)\,dxdy =  I_{x}+I_{y}
  \end{tabular}
\]

\newpage

\begin{center}
\fbox{\large\bfseries Εφαρμογές Τριπλού Ολοκληρώματος}
\end{center}

\vspace{\baselineskip}

Έστω ότι το στερεό χωρίο $V$ είναι ένα υλικό σώμα και ότι η πυκνότητα της μάζας σε κάθε σημείο του $(x,y,z)$ δίνεται από την συνεχή συνάρτηση $\delta=\delta(x,y,z)$.

Η συνολική μάζα $M$ που κατανέμεται στο χωρίο $V$ δίνεται από τον τύπο:

\[
   \text{\bfseries Μάζα:} \quad M=\iiint_{V}\delta(x,y,z)\,dxdydz
\]

Οι στατικές (πρώτες) ροπές ως προς τα επίπεδα συντεταγμένων $Oxy$, $Oxz$ και $Oyz$ δίνονται αντίστοιχα από τους τύπους:

\[
  \begin{tabular}{>{\bfseries}l<{}>{$}l<{$}}
    Στατικές Ροπές: & M_{xy}=\iiint_{V}z\delta(x,y,z)\,dxdydz \\
      &  M_{xz}=\iiint_{V}y\delta(x,y,z)\,dxdydz \\
      & M_{yz}=\iiint_{V}x\delta(x,y,z)\,dxdydz
  \end{tabular}
\]

Οι συντεταγμένες $(\overline{x},\overline{y},\overline{z})$ του κέντρου μάζας $K$ του χωρίου $V$ δίνονται από τους τύπους:

\[
  \text{\bfseries Κέντρο Μάζας:} \quad \overline{x}=\frac{M_{yz}}{M},\; \overline{y}=\frac{M_{xz}}{M},\;
  \overline{z}=\frac{M_{xy}}{M}
\]

Οι ροπές αδράνειας (δεύτερες ροπές) ως προς τα επίπεδα συντεταγμένων $Oxy$, $Oxz$ και $Oyz$, τους άξονες $x$, $y$ και $z$ και η ροπή αδράνειας και η ροπή αδράνειας ως προς την αρχή των αξόνων, δίνονται αντίστοιχα από τους τύπους:

\[
  \begin{tabular}{>{\bfseries}l<{}>{$}l<{$}}
     Ροπές Αδράνειας: & Ι_{xy}=\iiint_{V}z^{2}\delta(x,y,z)\,dxdydz \\
      &  Ι_{xz}=\iiint_{V}y^{2}\delta(x,y,z)\,dxdydz \\
      &  Ι_{yz}=\iiint_{V}x^{2}\delta(x,y,z)\,dxdydz \\
      &  Ι_{x}=\iiint_{V}(y^{2}+z^{2})\delta(x,y,z)\,dxdydz = I_{xy}+I_{xz} \\
      &  Ι_{y}=\iiint_{V}(x^{2}+z^{2})\delta(x,y,z)\,dxdydz = I_{xy}+I{yz} \\
      &  Ι_{z}=\iiint_{V}(x^{2}+y^{2})\delta(x,y,z)\,dxdydz = I_{xz}+I_{yz} \\
      &  I_{o}=\iiint_{V}(x^{2}+y^{2}+z^{2})\delta(x,y,z)\,dxdydz =  I_{xy}+I_{xz}+I_{yz}
  \end{tabular}
\]


\newpage

\begin{center}
\fbox{\large\bfseries Εφαρμογές Επικαμπύλιου Ολοκληρώματος}
\end{center}

\vspace{\baselineskip}

Έστω ότι $c$ είναι μια υλική καμπύλη και ότι η πυκνότητα της μάζας σε κάθε σημείο της $(x,y,z)$ δίνεται από την συνεχή συνάρτηση $\delta=\delta(x,y,z)$.

Η συνολική μάζα $M$ που κατανέμεται στην καμπύλη $c$ δίνεται από τον τύπο:

\[
   \text{\bfseries Μάζα:} \quad M=\int_{c}\delta(x,y,z)\,ds
\]

Οι (πρώτες) ροπές ως προς τα επίπεδα συντεταγμένων $Oxy$, $Oxz$ και $Oyz$, δίνονται αντίστοιχα από τους τύπους:

\[
  \begin{tabular}{>{\bfseries}l<{}>{$}l<{$}}
    Πρώτες Ροπές: & M_{xy}=\int_{c}z\delta(x,y,z)\,ds \\
      &  M_{xz}=\int_{c}y\delta(x,y,z)\,ds \\
      & M_{yz}=\int_{c}x\delta(x,y,z)\,ds
  \end{tabular}
\]

Οι συντεταγμένες $(\overline{x},\overline{y}, \overline{z})$ του κέντρου μάζας $K$ της καμπύλης $c$ δίνονται από τους τύπους:

\[
  \text{\bfseries Κέντρο Μάζας:} \quad \overline{x}=\frac{M_{yz}}{M},\; \overline{y}=\frac{M_{xz}}{M},\; \overline{z}=\frac{M_{xy}}{M}
\]

Οι ροπές αδράνειας (δεύτερες ροπές) ως προς τους άξονες $x$, $y$ και $z$, δίνονται αντίστοιχα από τους τύπους:

\[
  \begin{tabular}{>{\bfseries}l<{}>{$}l<{$}}
     Ροπές Αδράνειας: & Ι_{x}=\int_{c}(y^{2}+z^{2})\delta(x,y,z)\,ds \\
      &  Ι_{y}=\int_{c}(x^{2}+z^{2})\delta(x,y,z)\,ds \\
      &  I_{z}=\int_{c}(x^{2}+y^{2})\delta(x,y,z)\,ds \\
  \end{tabular}
\]

\end{document}
