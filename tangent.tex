\documentclass[a4paper,12pt]{article}
\usepackage{etex}
%%%%%%%%%%%%%%%%%%%%%%%%%%%%%%%%%%%%%%
% Babel language package
\usepackage[english,greek]{babel}
% Inputenc font encoding
\usepackage[utf8]{inputenc}
%%%%%%%%%%%%%%%%%%%%%%%%%%%%%%%%%%%%%%

%%%%% math packages %%%%%%%%%%%%%%%%%%
\usepackage{amsmath}
\usepackage{amssymb}
\usepackage{amsfonts}
\usepackage{amsthm}
\usepackage{proof}

\usepackage{physics}

%%%%%%% symbols packages %%%%%%%%%%%%%%
\usepackage{dsfont}
\usepackage{stmaryrd}
%%%%%%%%%%%%%%%%%%%%%%%%%%%%%%%%%%%%%%%


%%%%%% graphicx %%%%%%%%%%%%%%%%%%%%%%%
\usepackage{graphicx}
\usepackage{color}
%\usepackage{xypic}
\usepackage[all]{xy}
\usepackage{calc}
%%%%%%%%%%%%%%%%%%%%%%%%%%%%%%%%%%%%%%%

\usepackage{enumerate}

\usepackage{fancyhdr}
%%%%% header and footer rule %%%%%%%%%
\setlength{\headheight}{14pt}
\renewcommand{\headrulewidth}{0pt}
\renewcommand{\footrulewidth}{0pt}
\fancypagestyle{plain}{\fancyhf{}
\fancyhead{}
\lfoot{}
\rfoot{\small \thepage}}
\fancypagestyle{vangelis}{\fancyhf{}
\rhead{\small \leftmark}
\lhead{\small }
\lfoot{}
\rfoot{\small \thepage}}
%%%%%%%%%%%%%%%%%%%%%%%%%%%%%%%%%%%%%%%

\usepackage{hyperref}
\usepackage{url}
%%%%%%% hyperref settings %%%%%%%%%%%%
\hypersetup{pdfpagemode=UseOutlines,hidelinks,
bookmarksopen=true,
pdfdisplaydoctitle=true,
pdfstartview=Fit,
unicode=true,
pdfpagelayout=OneColumn,
}
%%%%%%%%%%%%%%%%%%%%%%%%%%%%%%%%%%%%%%



\usepackage{geometry}
\geometry{left=25.63mm,right=25.63mm,top=36.25mm,bottom=36.25mm,footskip=24.16mm,headsep=24.16mm}

%\usepackage[explicit]{titlesec}
%%%%%% titlesec settings %%%%%%%%%%%%%
%\titleformat{\chapter}[block]{\LARGE\sc\bfseries}{\thechapter.}{1ex}{#1}
%\titlespacing*{\chapter}{0cm}{0cm}{36pt}[0ex]
%\titleformat{\section}[block]{\Large\bfseries}{\thesection.}{1ex}{#1}
%\titlespacing*{\section}{0cm}{34.56pt}{17.28pt}[0ex]
%\titleformat{\subsection}[block]{\large\bfseries{\thesubsection.}{1ex}{#1}
%\titlespacing*{\subsection}{0pt}{28.80pt}{14.40pt}[0ex]
%%%%%%%%%%%%%%%%%%%%%%%%%%%%%%%%%%%%%%

%%%%%%%%% My Theorems %%%%%%%%%%%%%%%%%%
\newtheorem{thm}{Θεώρημα}[section]
\newtheorem{cor}[thm]{Πόρισμα}
\newtheorem{lem}[thm]{λήμμα}
\theoremstyle{definition}
\newtheorem{dfn}{Ορισμός}[section]
\newtheorem{dfns}[dfn]{Ορισμοί}
\theoremstyle{remark}
\newtheorem{remark}{Παρατήρηση}[section]
\newtheorem{remarks}[remark]{Παρατηρήσεις}
%%%%%%%%%%%%%%%%%%%%%%%%%%%%%%%%%%%%%%%




\newcommand{\vect}[2]{(#1_1,\ldots, #1_#2)}
%%%%%%% nesting newcommands $$$$$$$$$$$$$$$$$$$
\newcommand{\function}[1]{\newcommand{\nvec}[2]{#1(##1_1,\ldots, ##1_##2)}}

\newcommand{\linode}[2]{#1_n(x)#2^{(n)}+#1_{n-1}(x)#2^{(n-1)}+\cdots +#1_0(x)#2=g(x)}

\newcommand{\vecoffun}[3]{#1_0(#2),\ldots ,#1_#3(#2)}




\pagestyle{vangelis}
\everymath{\displaystyle}


\begin{document}


\chapter{Εφαπτόμενο Επίπεδο - Κάθετη Ευθεία}

\section{Εφαπτόμενο Επίπεδο}

% Η εξίσωση ενός επιπέδου που διέρχεται από το σημείο $ P_{0}(x_{0}, y_{0}, z_{0}) $ 
% και όπου το διάνυσμα $ \mathbf{n} = (a,b,c) $ είναι κάθετο στο επίπεδο, είναι:
% \[
%   a(x- x_{0}) + b (y- y_{0}) + c (z - z_{0})
% \] 

\begin{myitemize}
  \item Αν $S: F(x,y,z)=0$ επιφάνεια τότε η διανυσματική εξίσωση του εφαπτόμενου 
    επιπέδου της $S$ στο σημείο της $ P(x_{0}, y_{0}, z_{0}) $ είναι
    \begin{equation}\label{eq:tan}
      (\mathbf{r} - \mathbf{r_{0}})\cdot \grad F(P_{0})= \mathbf{0} 
    \end{equation} 
    όπου $ \mathbf{r}=x \mathbf{i}+y \mathbf{j}+z \mathbf{k} $ και 
    $ \mathbf{r_{0}}=x_{0} \mathbf{i}+ y _{0} \mathbf{j}+ z_{0} \mathbf{k} $.

    Σε καρτεσιανές συντεταγμένες η εξίσωση ~\eqref{eq:tan} γράφεται:
    \begin{equation}\label{eq:tan2}
      \boxed{		
        \pdv{F}{x}\eval_{P_{0}} (x-x_{0}) + \pdv{F}{y}\eval_{P_{0}} (y-y_{0}) +
        \pdv{F}{z}\eval_{P_{0}} (z-z_{0}) =0 
      } 
    \end{equation}

  \item Αν $ S: z=f(x,y) $ επιφάνεια τότε θέτουμε $ F(x,y,z) =  f(x,y) - z = 0 $. 
    Οπότε σε αυτήν την περίπτωση η εξίσωση ~\eqref{eq:tan2} γράφεται
    \[
      \boxed{
        \pdv{f}{x}\eval_{P_{0}} (x - x_{0}) + 
        \pdv{f}{y}\eval_{P_{0}} (y - y_{0}) - (z - z_{0}) = 0 
      }
    \] 
\end{myitemize}

\begin{rem}
  Αν $ z=f(x,y) $ και $ \grad f(x_{0}, y_{0}) = 0 $, τότε το εφαπτόμενο επίπεδο 
  της επιφάνειας $S$ στο σημείο $ (x_{0}, y_{0}, f(x_{0}, y_{0})) $ είναι 
  παράλληλο στο επίπεδο $ xy $, γιατί τότε ένα κάθετο διάνυσμα στο επίπεδο είναι 
  \[
    \mathbf{n} = \pdv{f}{x} \left(x_{0}, y_{0}\right)  \mathbf{i} + 
    \pdv{f}{y} \left(x_{0}, y_{0}\right) \mathbf{j} - \mathbf{k} = - \mathbf{k}
  \] 
  το οποίο είναι παράλληλο στον άξονα $z$. Επομένως όταν 
  $ \grad f(x_{0}, y_{0}) = 0 $ το εφαπτόμενο επίπεδο επίπεδο της επιφάνειας στο 
  $ (x_{0}, y_{0}, f(x_{0}, y_{0})) $ είναι το οριζόντιο επίπεδο με εξίσωση 
  $ z = z_{0} $.
\end{rem}

\section{Κάθετη Ευθεία}

\begin{myitemize}
  \item Αν $S: F(x,y,z)=0$ τότε ένα κάθετο διάνυσμα στην επιφάνεια $S$ 
    στο σημείο $ P(x_{0}, y_{0}, z_{0}) $ είναι
    \[
      \grad F{(P_{0})} = \left(\pdv{F}{x}, \pdv{F}{y}, \pdv{F}{z}\right)_{P_{0}}
    \]
  \item Αν $S: F(x,y,z)=0$ τότε η διανυσματική εξίσωση της κάθετης ευθείας της 
    επιφάνειας $S$ στο σημείο της $ P(x_{0}, y_{0}, z_{0}) $ είναι
    \begin{equation}\label{eq:kath}
      (\mathbf{r} - \mathbf{r_{0}}) \times \grad F(P_{0}) = \mathbf{0}
    \end{equation} 
    όπου $ \mathbf{r}=x \mathbf{i}+y \mathbf{j}+z \mathbf{k} $ και 
    $ r_{0}=x_{0} \mathbf{i}+ y _{0} \mathbf{j}+ z_{0} \mathbf{k} $.

    Πιο συγκεκριμένα 
    \[
      \boxed{\mathbf{r}(t) = (x_{0}, y_{0}, z_{0}) + t \grad F(x_{0}, y_{0}, z_{0})}
    \] 

    Σε καρτεσιανές συντεταγμένες η εξίσωση ~\eqref{eq:kath} γράφεται:
    \begin{equation}\label{eq:kath2}
      \boxed{ 
        \frac{x- x_{0}}{\eval{\pdv{F}{x}} _{P_{0}}} = 
        \frac{y- y_{0}}{\eval{\pdv{F}{y}} _{P_{0}}} = 
        \frac{z- z_{0}}{\eval{\pdv{F}{z}} _{P_{0}}}  
      }
    \end{equation}

  \item Αν $ S: z=f(x,y) $ τότε θέτουμε $ F(x,y,z) =  f(x,y) - z = 0 $. Οπότε σε
    αυτήν την περίπτωση η εξίσωση ~\eqref{eq:kath2} γράφεται
    \[
      \boxed{
        \frac{x- x_{0}}{\eval{\pdv{f}{x}} _{P_{0}}} = 
        \frac{y- y_{0}}{\eval{\pdv{f}{y}} _{P_{0}}} = 
        \frac{z- z_{0}}{-1} 
      }
    \] 
\end{myitemize}


\end{document}
