\input{$HOME/Desktop/preamble/preamble.tex}
\newcommand{\vect}[2]{(#1_1,\ldots, #1_#2)}
%%%%%%% nesting newcommands $$$$$$$$$$$$$$$$$$$
\newcommand{\function}[1]{\newcommand{\nvec}[2]{#1(##1_1,\ldots, ##1_##2)}}

\newcommand{\linode}[2]{#1_n(x)#2^{(n)}+#1_{n-1}(x)#2^{(n-1)}+\cdots +#1_0(x)#2=g(x)}

\newcommand{\vecoffun}[3]{#1_0(#2),\ldots ,#1_#3(#2)}





\everymath{\displaystyle}


\begin{document}

\begin{center}
\fbox{\large\bf Εφαπτόμενο Επίπεδο μιας επιφάνειας }
\end{center}

\vspace{\baselineskip}

\begin{itemize}
	\item Αν $S: F(x,y,z)=0$ τότε η διανυσματική εξίσωση του εφαπτόμενου επιπέδου της $S$ στο σημείο
		της $ P(x_{0}, y_{0}, z_{0}) $ είναι
		\begin{equation}\label{eq:tan}
			(\mathbf{r} - \mathbf{r}_{0})\cdot \grad F(P_{0})=0 
		\end{equation} 
		όπου $ \mathbf{r}=x \mathbf{i}+y \mathbf{j}+z \mathbf{k} $ και $ r_{0}=x_{0} \mathbf{i}+ y
	_{0} \mathbf{j}+ z_{0} \mathbf{k} $.

	Σε καρτεσιανές συντεταγμένες η εξίσωση ~\eqref{eq:tan} γράφεται:

	\[
		\boxed{		
			\pdv{F}{x}\eval_{P_{0}} (x-x_{0}) + \pdv{F}{y}\eval_{P_{0}} (y-y_{0}) +
			\pdv{F}{z}\eval_{P_{0}} (z-z_{0}) =0 
		}
	\]
\item Ένα κάθετο διάνυσμα στην επιφάνεια $ S $ στο σημείο $ P_{0} $ είναι το διάνυσμα της κλίσης της
	συνάρτησης 
	\[
		\grad F{(P_{0})} = \left(\pdv{f}{x}, \pdv{f}{y}, \pdv{f}{z}\right)_{P_{0}}
	\] 

\item Αν $ S: z=f(x,y) $ τότε θέτουμε $ F(x,y,z) = z - f(x,y) = 0 $. Οπότε σε
	αυτήν την περίπτωση η εξίσωση ~\eqref{eq:tan} γράφεται
	\[
		\boxed{
			-\pdv{f}{x}\eval_{P_{0}} (x - x_{0}) - \pdv{f}{y}\eval_{P_{0}} (y - y_{0}) + (z - z_{0})
		}
	\] 
\end{itemize}

\end{document}
