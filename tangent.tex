\input{preamble_ask.tex}
\input{definitions_ask.tex}

\geometry{top=1cm}

\pagestyle{vangelis}
\everymath{\displaystyle}
\setcounter{chapter}{1}

\begin{document}


\chapter*{Εφαπτόμενο Επίπεδο - Κάθετη Ευθεία}

\section*{Εφαπτόμενο Επίπεδο}

% Η εξίσωση ενός επιπέδου που διέρχεται από το σημείο $ P_{0}(x_{0}, y_{0}, z_{0}) $ 
% και όπου το διάνυσμα $ \mathbf{n} = (a,b,c) $ είναι κάθετο στο επίπεδο, είναι:
% \[
%   a(x- x_{0}) + b (y- y_{0}) + c (z - z_{0})
% \] 

\begin{myitemize}
  \item Αν $S: F(x,y,z)=0$ επιφάνεια, τότε η διανυσματική εξίσωση του 
    εφαπτόμενου επιπέδου της στο σημείο $ P(x_{0}, y_{0}, z_{0}) $ είναι
    \begin{equation}\label{eq:tan}
      (\mathbf{r} - \mathbf{r_{0}})\cdot \grad F(P_{0})= \mathbf{0} 
    \end{equation} 
    όπου $ \mathbf{r}=x \mathbf{i}+y \mathbf{j}+z \mathbf{k} $ και 
    $ \mathbf{r_{0}}=x_{0} \mathbf{i}+ y _{0} \mathbf{j}+ z_{0} \mathbf{k} $.

    Σε καρτεσιανές συντεταγμένες η εξίσωση ~\eqref{eq:tan} γράφεται:
    \begin{equation}\label{eq:tan2}
      \boxed{		
        \pdv{F}{x}\eval_{P_{0}} (x-x_{0}) + \pdv{F}{y}\eval_{P_{0}} (y-y_{0}) +
        \pdv{F}{z}\eval_{P_{0}} (z-z_{0}) =0 
      } 
    \end{equation}

  \item Αν $ S: z=f(x,y) $ επιφάνεια τότε θέτουμε $ F(x,y,z) =  f(x,y) - z = 0 $. 
    Οπότε σε αυτήν την περίπτωση η εξίσωση ~\eqref{eq:tan2} γράφεται
    \[
      \boxed{
        \pdv{f}{x}\eval_{P_{0}} (x - x_{0}) + 
        \pdv{f}{y}\eval_{P_{0}} (y - y_{0}) - (z - z_{0}) = 0 
      }
    \] 
\end{myitemize}

\begin{rem}
  Αν $ z=f(x,y) $ και $ \grad f(x_{0}, y_{0}) = 0 $, τότε το εφαπτόμενο επίπεδο 
  της επιφάνειας $S$ στο σημείο $ (x_{0}, y_{0}, f(x_{0}, y_{0})) $ είναι 
  παράλληλο στο επίπεδο $ xy $, γιατί τότε ένα κάθετο διάνυσμα στο επίπεδο είναι 
  \[
    \mathbf{n} = \pdv{f}{x} \left(x_{0}, y_{0}\right)  \mathbf{i} + 
    \pdv{f}{y} \left(x_{0}, y_{0}\right) \mathbf{j} - \mathbf{k} = - \mathbf{k}
  \] 
  το οποίο είναι παράλληλο στον άξονα $z$. Επομένως όταν 
  $ \grad f(x_{0}, y_{0}) = 0 $ το εφαπτόμενο επίπεδο επίπεδο της επιφάνειας στο 
  $ (x_{0}, y_{0}, f(x_{0}, y_{0})) $ είναι το οριζόντιο επίπεδο με εξίσωση 
  $ z = z_{0} $.
\end{rem}

\section*{Κάθετη Ευθεία}

\begin{myitemize}
  \item Αν $S: F(x,y,z)=0$ τότε ένα κάθετο διάνυσμα στην επιφάνεια $S$ 
    στο σημείο $ P(x_{0}, y_{0}, z_{0}) $ είναι
    \[
      \grad F{(P_{0})} = \left(\pdv{F}{x}, \pdv{F}{y}, \pdv{F}{z}\right)_{P_{0}}
    \]
  \item Αν $S: F(x,y,z)=0$ τότε η διανυσματική εξίσωση της κάθετης ευθείας της 
    επιφάνειας $S$ στο σημείο της $ P(x_{0}, y_{0}, z_{0}) $ είναι
    \begin{equation}\label{eq:kath}
      (\mathbf{r} - \mathbf{r_{0}}) \times \grad F(P_{0}) = \mathbf{0}
    \end{equation} 
    όπου $ \mathbf{r}=x \mathbf{i}+y \mathbf{j}+z \mathbf{k} $ και 
    $ r_{0}=x_{0} \mathbf{i}+ y _{0} \mathbf{j}+ z_{0} \mathbf{k} $.

    Πιο συγκεκριμένα 
    \[
      \boxed{\mathbf{r}(t) = (x_{0}, y_{0}, z_{0}) + t \grad F(x_{0}, y_{0}, z_{0})}
    \] 

    Σε καρτεσιανές συντεταγμένες η εξίσωση ~\eqref{eq:kath} γράφεται:
    \begin{equation}\label{eq:kath2}
      \boxed{ 
        \frac{x- x_{0}}{\eval{\pdv{F}{x}} _{P_{0}}} = 
        \frac{y- y_{0}}{\eval{\pdv{F}{y}} _{P_{0}}} = 
        \frac{z- z_{0}}{\eval{\pdv{F}{z}} _{P_{0}}}  
      }
    \end{equation}

  \item Αν $ S: z=f(x,y) $ τότε θέτουμε $ F(x,y,z) =  f(x,y) - z = 0 $. Οπότε σε
    αυτήν την περίπτωση η εξίσωση ~\eqref{eq:kath2} γράφεται
    \[
      \boxed{
        \frac{x- x_{0}}{\eval{\pdv{f}{x}} _{P_{0}}} = 
        \frac{y- y_{0}}{\eval{\pdv{f}{y}} _{P_{0}}} = 
        \frac{z- z_{0}}{-1} 
      }
    \] 
\end{myitemize}


\end{document}
