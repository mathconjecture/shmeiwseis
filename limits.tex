\input{preamble_ask.tex}
\input{definitions_ask.tex}
\input{tikz.tex}
\input{myboxes.tex}

\geometry{top=1.5cm}

\input{insbox}
\usepackage{caption}

\newcommand{\twocolumnsidesch}[2]{\begin{minipage}[c]{0.45\linewidth}\raggedright
    #1
    \end{minipage}\hfill\begin{minipage}[c]{0.45\linewidth}\raggedright
    #2
\end{minipage}}

\everymath{\displaystyle}
\pagestyle{askhseis}

\usetikzlibrary{decorations.markings}

\tikzset{
  set arrow inside/.code={\pgfqkeys{/tikz/arrow inside}{#1}},
  set arrow inside={end/.initial=stealth, opt/.initial=},
  /pgf/decoration/Mark/.style={
    mark/.expanded=at position #1 with
    {
      \noexpand\arrow[\pgfkeysvalueof{/tikz/arrow inside/opt}]{\pgfkeysvalueof{/tikz/arrow inside/end}}
    }
  },
  arrow inside/.style 2 args={
    set arrow inside={#1},
    postaction={
      decorate,decoration={
        markings,Mark/.list={#2}
      }
    }
  },
}


\begin{document}


\chapter*{Όριο Συνάρτησης}

\section*{Η έννοια του ορίου}


\InsertBoxL{2}{\parbox[b][7\baselineskip][c]{0.33\textwidth}{
    \begin{tikzpicture}
      \draw[->,blue!55,>=stealth] (-2,0) -- (3,0) node[below] {$x$} ;
      \draw[->,blue!55,>=stealth] (0,-0.5) -- (0,2.5) node [left] {$y$} ;
      \draw[color=Col1,samples=200,domain=-1.4:1.4,thick] plot (\x, {\x*\x}) 
        node[right] (f) {$y=f(x)$} [arrow inside={}{0.75}] ;
      \path[color=Col1,samples=200,domain=1.4:-1.4,thick] plot (\x, {\x*\x}) 
        [arrow inside={}{0.165}] ;
      \node[draw,Col1,circle,inner sep=1.5pt,fill=white] (a) at (1,1) {} ;
      \coordinate (b) at (1,0) ;
      \coordinate (c) at (0,1) ;
      \fill[Col1] (b) circle (1pt) node[below] {$ x_{0}$} ;
      \fill[Col1] (c) circle (1pt) node[left] {$a$} ;
      \draw[dashed] (a) -- (b) ;
      \draw[dashed] (a) -- (c) ;
      \node at (b) [Col2,below left,yshift=-2pt,xshift=-3pt] {$\rightarrow$} ;
      \node at (b) [Col2,below right,yshift=-2pt,xshift=3pt] {$\leftarrow$} ;
      \node at (c) [Col2,above left,yshift=2pt] {$\downarrow$} ;
      \node at (c) [Col2,below left,yshift=-2pt] {$\uparrow$} ;
    \end{tikzpicture}
}}

Έστω συνάρτηση $ f(x) $, η οποία είναι ορισμένη για τις διάφορες τιμές του $x$,
\textbf{κοντά} στο $ x_{0} $. Αυτό σημαίνει, ότι η $f$ είναι ορισμένη σε μια 
ανοιχτή \textbf{περιοχή} του $ x_{0} $, εκτός ίσως από το ίδιο το $ x_{0} $. 

Παρατηρούμε ότι για τις διάφροες τιμές του $x$ \textcolor{Col1}{κοντά} στο $ x_{0} $, 
οι τιμές του $y$, δηλαδή της $ f(x) $, είναι \textcolor{Col1}{κοντά} στο $a$. Μάλιστα, 
φαίνεται ότι καθώς το $x$ πλησιάζει \textbf{οσοδήποτε} κοντά στο $ x_{0} $ (και από τις 
δύο πλευρές), οι τιμές της $ f(x) $ πλησιάζουν \textbf{οσοδήποτε} κοντά στην τιμή $ a $. 
Αυτό, το εκράζουμε λέγοντας ότι το \textcolor{Col1}{όριο} της $ f(x) $, καθώς το $x$ τείνει στο $ x_{0} $, είναι ίσο με $a$ και γράφουμε
\begin{empheq}[box=\mathboxg]{equation*}
  \lim_{x \to x_{0}} f(x) = a \quad \text{ή} \quad f(x) \to a \; \text{καθώς} \; x \to
  x_{0}
\end{empheq}

\begin{rem}
  Για τον υπολογισμό του ορίου, κάποιας συνάρτησης, καθώς το $ x $ τείνει στο $ x_{0} $, 
  εξετάζουμε τις τιμές της συνάρτησης κοντά στο $ x_{0} $ και ποτέ στο ίδιο το 
  $ x_{0} $. Μάλιστα, η συνάρτηση, δεν είναι απαραίτητο, να είναι ορισμένη στο $ x_{0} $.
  Το μόνο που μας ενδιαφέρει είναι η συμπεριφορά της συνάρτησης, \textbf{κοντά} στο 
  $ x_{0} $.
\end{rem}


\threecolumnsides{
  \begin{tikzpicture}
    \draw[->,blue!55,>=stealth] (-1,0) -- (3,0) node[right] {$x$} ;
    \draw[->,blue!55,>=stealth] (0,-0.7) -- (0,2.5) node [above] {$y$} ;
    \draw[color=Col1,samples=200,domain=-0.6:1.4,thick] plot (\x, {\x*\x}) 
      node[right] (f) {$y=f(x)$} [arrow inside={}{0.60}] ;
    \path[color=Col1,samples=200,domain=1.4:-0.6,thick] plot (\x, {\x*\x}) 
      [arrow inside={}{0.27}] ;
    \coordinate (b) at (1,0) ;
    \coordinate (c) at (0,1) ;
    \coordinate (p) at (1,1) ;
    \fill[Col1] (b) circle (1pt) node[below] {$ x_{0} $} ;
    \fill[Col1] (c) circle (1pt) node[left=1pt] {$a$} ;
    \draw[dashed] (c) -| (b);
    \node at (2,1) {$(a)$} ;
  \end{tikzpicture}
}{
  \begin{tikzpicture}[]
    \draw[->,blue!55,>=stealth] (-1,0) -- (3,0) node[right] {$x$} ;
    \draw[->,blue!55,>=stealth] (0,-0.7) -- (0,2.5) node [above] {$y$} ;
    \draw[color=Col1,samples=200,domain=-0.6:1.4,thick] plot (\x, {\x*\x}) 
      node[right] (f) {$y=f(x)$} [arrow inside={}{0.60}] ;
    \path[color=Col1,samples=200,domain=1.4:-0.6,thick] plot (\x, {\x*\x}) 
      [arrow inside={}{0.27}] ;
    \coordinate (b) at (1,0) ;
    \coordinate (c) at (0,1) ;
    \coordinate (p) at (1,1) ;
    \fill[Col1] (b) circle (1pt) node[below] {$ x_{0} $} ;
    \fill[Col1] (c) circle (1pt) node[left=1pt] {$a$} ;
    \draw[dashed] (c) -| (b);
    \node[draw,Col1,circle,inner sep=1.5pt,fill=white] (p) at (1,1) {} ;
    \node at (2,1) {$(b)$} ;
  \end{tikzpicture}
}{
  \begin{tikzpicture}
    \draw[->,blue!55,>=stealth] (-1,0) -- (3,0) node[right] {$x$} ;
    \draw[->,blue!55,>=stealth] (0,-0.7) -- (0,2.5) node [above] {$y$} ;
    \draw[color=Col1,samples=200,domain=-0.6:1.4,thick] plot (\x, {\x*\x}) 
      node[right] (f) {$y=f(x)$} [arrow inside={}{0.60}] ;
    \path[color=Col1,samples=200,domain=1.4:-0.6,thick] plot (\x, {\x*\x}) 
      [arrow inside={}{0.27}] ;
    \coordinate (b) at (1,0) ;
    \coordinate (c) at (0,1) ;
    \coordinate (e) at (0,2) ;
    \coordinate (d) at (1,2) ;
    \coordinate (p) at (1,1) ;
    \fill[Col1] (b) circle (1pt) node[below] {$ x_{0} $} ;
    \fill[Col1] (c) circle (1pt) node[left=1pt] {$a$} ;
    \fill[Col1] (e) circle (1pt) node[left=1pt] {$b$} ;
    \fill[Col1] (d) circle (2pt) ;
    \fill[Col1] (e) circle (1pt) ;
    \draw[dashed] (c) -| (b);
    \node[draw,Col1,circle,inner sep=1.5pt,fill=white] (p) at (1,1) {} ;
    \node at (2,1) {$(c)$} ;
  \end{tikzpicture}
}

\begin{myitemize}
  \item Στην περίπτωση $ (a) $ παρατηρούμε ότι $ f(x_{0}) = a $ και 
    $ \lim_{x \to x_{0}} f(x) = a $.
  \item Στην περίπτωση $ (b) $ παρατηρούμε ότι $ f(x_{0}) $ δεν είναι ορισμένο, αλλά
    $ \lim_{x \to x_{0}} f(x) = a $.
  \item Στην περίπτωση $ (c) $ παρατηρούμε ότι $ f(x_{0}) = b $, όμως 
    $ \lim_{x \to x_{0}} f(x) = a $.
\end{myitemize}


\section*{Πλευρικά Όρια Συνάρτησης}


\InsertBoxL{2}{\parbox[b][8\baselineskip][c]{0.33\textwidth}{
\begin{tikzpicture}
  \draw[->,blue!55,>=stealth] (-2,0) -- (3,0) node[right] {$x$} ;
  \draw[->,blue!55,>=stealth] (0,-0.5) -- (0,2.8) node [left] {$y$} ;
  \draw[color=Col1,samples=200,domain=-1.35:1,thick] plot (\x, {\x*\x+1})
    node[right=10pt] (f) {$y=f(x)$} [arrow inside={}{0.90}] ;
  % \draw[color=Col1,samples=200,domain=2.5:1,thick] plot (\x, {-\x+1.5}) 
  %   [arrow inside={}{0.85}] ;
  \node[draw,Col1,circle,inner sep=1.5pt,fill=white] (b) at (1,2) {} ;
  \node[draw,Col1,circle,inner sep=1.5pt,fill=white] (c) at (1,0.5) {} ;
  \coordinate (e) at (0,0.5) ;
  \coordinate (g) at (0,2) ;
  \coordinate (d) at (1,0) ;
  \node (k) at (d) [Col1,below right] {$ x_{0} $} ;
  \draw[dashed] (b) -- (c) ;
  \draw[dashed] (c) -- (d) ;
  \draw[dashed] (a) -- (d) ;
  % \draw[dashed] (a) -- (h) ;
  \draw[dashed] (b) -- (g) ;
  \draw[dashed] (c) -- (e) ;
  \node at (d) [above left,yshift=-2pt] {$\rightarrow$} ;
  \node at (d) [Col2,above right,yshift=-2pt] {$\leftarrow$} ;
  \node at (e) [Col2,above left] {$\downarrow$} ;
  \node at (g) [below left,yshift=-3pt] {$\uparrow$} ;
  \fill[Col1] (e) circle (1pt) node[below left] {$a$} ;
  \fill[Col1] (g) circle (1pt) node[left] {$b$} ;
  \fill[Col1] (d) circle (1pt) ;
  \coordinate (k) at (2.5,1.5) ;
  \begin{scope}[decoration={markings,mark=at position 0.85 with {\arrow{stealth}}}]
  \draw[color=Col1,thick,postaction={decorate}] (k) -- (c) ;
  \end{scope}
\end{tikzpicture}
}}

Στο διπλανό σχήμα, παρατηρούμε ότι η συνάρτηση $f$ πλησιάζει την τιμή $b$, καθώς το 
$x$ τείνει στο $ x_{0} $ από \textcolor{Col1}{αριστερά} αλλά η $f$ πλησιάζει την τιμή 
$a$, καθώς το $x$ τείνει στο $ x_{0} $ από \textcolor{Col1}{δεξιά}. Συμβολίζουμε
\begin{empheq}[box=\mathboxg]{equation*}
  \lim_{x \to x_{0}^{-}} f(x) = b \quad \text{και} \quad \lim_{x \to x_{0}^{+}} f(x) = a 
\end{empheq}
Ο συμβολισμός $ x \to x_{0}^{-} $ διαβάζεται ως «$x$ τείνει στο $ x_{0} $ από τα 
αριστερά» και σημαίνει ότι μας ενδιαφέρουν οι τιμές του $x$ κοντά στο $ x_{0} $, 
που είναι \textbf{μικρότερες} από το $ x_{0} $. Αντίστοιχα, 
ο συμβολισμός $ x \to x_{0}^{+} $ διαβάζεται ως «$x$ τείνει στο $ x_{0} $ από τα 
δεξιά» και σημαίνει ότι μας ενδιαφέρουν οι τιμές του $x$ κοντά στο $ x_{0} $, που
είναι \textbf{μεγαλύτερες} από το $ x_{0} $. 

\begin{prop}
 Το όριο μιας συνάρτησης υπάρχει, αν και μόνον αν, υπάρχουν τα πλευρικά όρια και είναι
  ίσα. Δηλαδή
\begin{empheq}[box=\mathboxr]{equation*}
    \lim_{x \to x_{0}} f(x) = a \Leftrightarrow 
    \lim_{x \to x_{0}^{-}} f(x) = a \quad \text{και} \quad \lim_{x \to x_{0}^{+}} f(x) 
    = a
    \end{empheq}
\end{prop}


\section*{Ιδιότητες των Ορίων}


\begin{prop}
  Έστω $ c, x_{0} \in \mathbb{R} $, $ \lim_{x\to x_{0}} f(x) = a \in \mathbb{R}$ και 
  $ \lim_{x\to x_{0}} g(x) = b \in \mathbb{R}$.
  \begin{empheq}[box=\mathboxg]{equation*}
    \twocolumnsidesc{
      \begin{enumerate}
        \item $ \lim_{x \to x_{0}} c = c, \; \forall a \in \mathbb{R} $
        \item $ \lim_{x\to x_{0}} x = x_{0} $
        \item $ \lim_{x\to x_{0}} (f(x) \pm g(x)) = a \pm b $
      \end{enumerate}
    }{
      \begin{enumerate}[start=4]
        \item $ \lim_{x\to x_{0}} (c \cdot f(x)) = c\cdot a, \; \forall c \in 
          \mathbb{R} $
        \item $ \lim_{x\to x_{0}} (f(x) \cdot g(x)) = a\cdot b $
        \item $ \lim_{x\to x_{0}} \left(\frac{f(x)}{g(x)}\right) = \frac{a}{b}, 
          \;  b\neq 0 $
      \end{enumerate}
    }
  \end{empheq}
\end{prop}

\begin{example}
  $ \lim_{x \to -3} 5 = 5 $
\end{example}

\begin{example}
  $ \lim_{x \to 2} x = 2 $
\end{example}

\begin{example}
  $ \lim_{x \to -1} \Bigl(\frac{1}{x} - x^{3}\Bigr) = \lim_{x \to -1} \frac{1}{x}
  - \lim_{x \to -1} x^{3} = \frac{1}{-1} - (-1)^{3} = -1-(-1)=-1+1=0  $
\end{example}

\begin{example}
  $ \lim_{x \to 4} \bigl(3 \sqrt{x}\bigr) = 3 \lim_{x \to 4} \sqrt{x} = 3 \cdot \sqrt{4} 
  = 3\cdot 2 = 6$
\end{example}

\begin{example}
  $ \lim_{x \to 1} \bigl(\sqrt{x} \cdot \ln{x}\bigr) = \lim_{x \to 1} \sqrt{x} \cdot \lim_{x \to 1}
  \ln{x} = \sqrt{1} \cdot \ln{1} = 1 \cdot 0 = 0 $
\end{example}

\begin{example}
  $ \lim_{x \to -1} \Bigl(\frac{\mathrm{e}^{x}}{x}\Bigr) = \frac{\lim\limits_{x \to -1}
    \mathrm{e}^{x}}{\lim\limits_{x \to -1}x} = \frac{\mathrm{e}^{-1}}{-1} = -
    \mathrm{e}^{-1} = - \frac{1}{\mathrm{e}} $  
  \end{example}

\begin{prop}
  Αν $ \lim_{x \to x_{0}} f(x) = a $ και $ n \in \mathbb{N} - \{ 0 \} $, τότε 
  \begin{empheq}[box=\mathboxg]{equation*}
    \lim_{x \to x_{0}} \bigl[f(x)\bigr]^{n} = a^{n}. 
  \end{empheq}
\end{prop}

\begin{example}
  $ \lim_{x \to \pi} \cos^{4}{x} = \bigl(\lim_{x \to \pi} \cos{x} \bigr)^{4} =
  (\cos{\pi} )^{4} = (-1)^{4}=1 $
  \end{example}

\begin{example} 
  $\lim_{x \to -1} (2x^{2}-1)^{4} = \bigl[\lim_{x \to -1}
  (2x^{2}-1)\bigr]^{4} = \bigl(2(-1)^{2}-1\bigr)^{4} = 1^{4} = 1 $ 
\end{example}

\begin{prop}
  Αν $ \lim_{x \to x_{0}} f(x) = a \in \mathbb{R} - {\{ 0 \}} $, τότε υπάρχει 
  περιοχή του $ x_{0} $ ώστε οι τιμές της $ f(x) $ να είναι \textbf{ομόσημες} του $a$ 
  για κάθε $x$ σε αυτή την περιοχή, δηλαδή:
  \begin{myitemize}
    \item Αν $ a > 0 $ τότε $ f(x)>0 $ 
    \item Αν $ a < 0 $ τότε $ f(x)<0 $ 
  \end{myitemize}
\end{prop}

\begin{prop}
\item {}
  \begin{myitemize}
    \item Αν $ \lim_{x \to x_{0}} f(x) = a \in \mathbb{R} $ και $ f(x) \geq 0 $ για κάθε 
      $x$ σε μια περιοχή του $ x_{0} $, τότε $ a \geq 0 $ 
    \item Αν $ \lim_{x \to x_{0}} f(x) = a \in \mathbb{R} $ και $ \lim_{x \to x_{0}} g(x)
      = b \in \mathbb{R}$ και $ f(x) \geq g(x) $ για κάθε $x$ σε μια περιοχή του 
      $ x_{0} $, τότε $ a \geq b $ 
  \end{myitemize}
\end{prop}

\begin{prop}
  Αν $ \lim_{x \to x_{0}} f(x) = a \in \mathbb{R} $ και $ f(x) \geq 0 $ για κάθε $x$ σε 
  μια περιοχή του $ x_{0} $, τότε
  \begin{empheq}[box=\mathboxg]{equation*}
    \lim_{x\to x_{0}} \sqrt[n]{f(x)} = \sqrt[n]{a}, \; n \in \mathbb{N}, \; n \geq 2 
  \end{empheq}
\end{prop}

\begin{example}
  $ \lim_{x \to -2} \sqrt{4x^{2}-3} = \sqrt{\lim_{x \to -2} (4x^{2}-3)} =
  \sqrt{4(-2)^{2}-3} = \sqrt{16-3} = \sqrt{13} $    
\end{example}

\begin{example}
  $ \lim_{x \to 1} \sqrt[3]{x-2} = \sqrt[3]{\lim_{x \to 1} (x-2)} = \sqrt[3]{1-2} =
  \sqrt[3]{-1} = -1 $
\end{example}

\begin{prop}
  Αν $ p(x) = a_{n}x^{n} + a_{n-1}x^{n-1} + \cdots + a_{1}x + a_{0} $,
  \textbf{πολυώνυμο}, τότε 
  $ \lim_{x \to x_{0}} p(x) = a_{n} x_{0}^{n} + a_{n-1} x_{0}^{n-1} + \cdots + a_{1}
  x_{0} $, δηλαδή
  \begin{empheq}[box=\mathboxg]{equation*}
    \lim_{x \to x_{0}} p(x) = p(x_{0})
  \end{empheq}
\end{prop}

\begin{prop}
  Αν $ p(x) $, $ q(x) $ πολυώνυμα, με $ q(x_{0}) \neq 0 $, τότε 
  \begin{empheq}[box=\mathboxg]{equation*}
    \lim_{x \to x_{0}} \frac{p(x)}{q(x)} = \frac{p(x_{0})}{q(x_{0})}
  \end{empheq}
\end{prop}

\begin{example}
  $ \lim_{x \to 2} (x^{3}+4x^{2}-3) = 2^{3}+ 4 \cdot 2^{2} -3 = 8+16-3 = 21 $  
\end{example}

\begin{example}
  $ \lim_{x \to 1} \frac{x^{4}+x^{2}-1}{x^{2}+5} = \lim_{x \to 1}
  \frac{1^{4}+1^{2}-1}{1^{2}+5} = \frac{1}{6} $
\end{example}

\begin{example}
  $ \lim_{x \to 1} \frac{x^{2}-3x+2}{x^{2}-1} \overset{(\frac{0}{0})}{=} \lim_{x \to 1}
  \frac{(x-2)\cancel{(x-1)}}{\cancel{(x-1)}(x+1)} = \lim_{x \to 1} \frac{x-2}{x+1} = 
  \frac{1-2}{1+1} = - \frac{1}{2} $
  \end{example}

\subsection*{Χρήσιμες Προτάσεις}

\begin{prop}
  $
  \lim_{x \to x_{0}} f(x) = l \Leftrightarrow \lim_{x \to x_{0}} (f(x)-l)=0
  \Leftrightarrow \lim_{x \to x_{0}} \abs{f(x)-l} = 0 
  $
\end{prop}

\begin{prop}
  $ \lim_{x \to x_{0}} f(x)=l \Leftrightarrow \lim_{x \to x_{0}} (-f(x)) = -l $
\end{prop}

\begin{prop}
  Αν $ \lim_{x \to x_{0}} f(x) = l \in \mathbb{R} $ τότε 
  $ \lim_{x \to x_{0}} \abs{f(x)} = \abs{l} $ 
\end{prop}

\begin{example}
  $ \lim_{x \to -2} \abs{-x^{3}+2x-7} = \abs{\lim_{x \to -2} (-x^{3+2x-7})} = \abs{-3}
  = 3$ 
\end{example}

\begin{rem}
  Προσοχή, το αντίστροφο της προηγούμενης πρότασης, δεν ισχύει πάντα. Για παράδειγμα, 
  αν θεωρήσουμε τη συνάρτηση $ f(x) = \frac{\abs{x}}{x} $, τότε έχουμε:
  \begin{myitemize}
    \item $ \lim_{x \to 0} \abs{f(x)} = \lim_{x \to 0} \abs{\frac{\abs{x}}{x}} =
      \lim_{x \to 0} \frac{\abs{x}}{\abs{x}} = 1 $
    \item $ \lim_{x \to 0} f(x) = \lim_{x \to 0} \frac{\abs{x}}{x} $, το οποίο, δεν
      υπάρχει, γιατί: $\lim_{x \to 0^{-}} f(x) = \lim_{x \to 0^{-}} \frac{-x}{x} = -1$ ενώ $\lim_{x \to 0^{+}} f(x) = \lim_{x \to 0^{+}} \frac{x}{x} = 1 $

  \end{myitemize}
\end{rem}

\begin{prop}
  $ \lim_{x \to x_{0}} f(x) = 0 \Leftrightarrow \lim_{x \to x_{0}} \abs{f(x)} = 0 $
\end{prop}

\begin{prop}
  Αν μια συνάρτηση $ f $ έχει όριο στο σημείο $ x_{0} $, τότε αυτό είναι
  \textbf{μοναδικό}.
\end{prop}

\begin{prop}
\item {}
  \begin{minipage}[t]{8.0 cm}
    \begin{myitemize}
      \item $ h(x) \leq f(x) \leq g(x), \; \forall x$ σε μια περιοχή του $ x_{0} $
        \hfill\tikzmark{a}
      \item $ \lim_{x \to x_{0}} h(x) = \lim_{x \to x_{0}} g(x) = l \in \mathbb{R} $
        \hfill\tikzmark{b}
    \end{myitemize}
  \end{minipage}
  \mybrace{a}{b}[ $ \lim_{x \to x_{0}} f(x) = l $ ]
\end{prop}

\begin{example}
  Να δείξετε ότι $ \lim_{x \to 0} \left(x^{2} \sin{\frac{1}{x}}\right) = 0  $ 
\end{example}
\begin{solution}
  Έχουμε, ότι
  \[
    - x^{2} \leq x^{2} \sin{\frac{1}{x}} \leq x^{2}, \; \forall x \in \mathbb{R}^{*}
  \]
  και επειδή $ \lim_{x \to 0} (-x^{2}) = \lim_{x \to 0} x^{2} = 0 $, σύμφωνα με το 
  Κριτήριο Παρεμβολής, προκύπτει ότι 
  \[
    \lim_{x \to 0} \left(x^{2} \sin{\frac{1}{x}}\right)  = 0
  \]
\end{solution}

\subsection*{Απροσδιόριστες Μορφές}


\subsection*{Παραδείγματα}

%todo να προσθεσω ένα παραδειγμα με απειρο δια απειρο

\begin{example}
  Να υπολογιστεί το όριο $ \lim_{x \to 1} \frac{\ln{x}}{x-1} $
\end{example}
\begin{solution}
  $ \lim_{x \to 1} \frac{\ln{x}}{x-1}
  \overset{\left(\frac{0}{0}\right)}{\underset{\mathrm{LH}}{=}}  \lim_{x \to 1} 
  \frac{(\ln{x} )'}{(x-1)'} = \lim_{x \to 1} \frac{\frac{1}{x}}{1} = 
  \lim_{x \to 1} \frac{1}{x} = 1 $
\end{solution}

\begin{example}
  Να υπολογιστεί το όριο $ \lim_{x \to \infty} \frac{\mathrm{e}^{x}}{x^{2}} $
\end{example}
\begin{solution}
  $ \lim_{x \to \infty} \frac{\mathrm{e}^{x}}{x^{2}}
  \overset{\left(\frac{\infty}{\infty}\right)}{\underset{\mathrm{LH}}{=}} \lim_{x \to
  \infty} \frac{(\mathrm{e}^{x} )'}{(x^{2})'} = \lim_{x \to \infty}
  \frac{\mathrm{e}^{x}}{2x}
  \overset{\left(\frac{\infty}{\infty}\right)}{\underset{\mathrm{LH}}{=}} 
  \lim_{x \to \infty} \frac{(\mathrm{e}^{x})'}{(2x)'} = \lim_{x \to \infty}
  \frac{\mathrm{e}^{x}}{2} = \infty
  $ 
\end{solution}

\begin{example}
  Να υπολογιστεί το όριο $ \lim_{x \to \infty} \frac{\ln{x}}{\sqrt{x}} $
\end{example}
\begin{solution}
  $ \lim_{x \to \infty} \frac{\ln{x}}{\sqrt{x}}
  \overset{\left(\frac{\infty}{\infty}\right)}{\underset{\mathrm{LH}}{=}} 
  \lim_{x \to \infty} \frac{(\ln{x} )'}{(\sqrt{x} )'} = \lim_{x \to \infty}
  \frac{\frac{1}{x}}{\frac{1}{2 \sqrt{x}}} = \lim_{x \to \infty} 
  \frac{2 \sqrt{x} }{x} = \lim_{x \to \infty} \frac{2}{\sqrt{x}} = 0
  $
\end{solution}

\begin{example}
  Να υπολογιστεί το όριο $ \lim_{x \to 0^{+}} (x \ln{x}) $
\end{example}
\begin{solution}
  $ \lim_{x \to 0^{+}} (x \ln{x}) \overset{(0 \cdot \infty)}{=} \lim_{x \to 0^{+}}
  \frac{\ln{x}}{\frac{1}{x}}
  \overset{\left(\frac{\infty}{\infty}\right)}{\underset{\mathrm{LH}}{=}} 
  \lim_{x \to 0^{+}} \frac{(\ln{x} )'}{(\frac{1}{x} )'} = \lim_{x \to 0^{+}}
  \frac{\frac{1}{x}}{- \frac{1}{x{2}}} = \lim_{x \to 0^{+}} \Bigl(-\frac{x^{2}}{x}\Bigr) 
  = \lim_{x \to 0^{+}} (-x) = 0 $
\end{solution}

\begin{example}
  Να υπολογιστεί το όριο $ \lim_{x \to \infty} (\mathrm{e}^{x} -x) $
\end{example}
\begin{solution}
  $ \lim_{x \to \infty} (\mathrm{e}^{x} - x) = \lim_{x \to \infty} x(\mathrm{e}^{x}
  - 1) = \lim_{x \to \infty} x \cdot \lim_{x \to \infty} \Bigl(\frac{\mathrm{e}^{x}}{x}
  -1\Bigr) $
  \begin{myitemize}
    \item $ \lim_{x \to \infty} x = \infty $
    \item $ \lim_{x \to \infty} \left(\frac{\mathrm{e}^{x}}{x} -1\right) = 
      \lim_{x \to \infty} \frac{\mathrm{e}^{x}}{x} -1
      \overset{\left(\frac{\infty}{\infty}\right)}{\underset{\mathrm{LH}}{=}} 
      \lim_{x \to \infty} \frac{(\mathrm{e}^{x} )'}{(x)'} -1 = \lim_{x \to \infty}
      \frac{\mathrm{e}^{x}}{1} -1 = \lim_{x \to \infty} \mathrm{e}^{x} -1 = \infty- 1 
      = \infty$
  \end{myitemize}
  Επομένως $ \lim_{x \to \infty} x \cdot \lim_{x \to \infty}
  \Bigl(\frac{\mathrm{e}^{x}}{x} -1 \Bigr) = \infty\cdot \infty = \infty$ 
\end{solution}

\begin{example}
  Να υπολογιστεί το όριο $ \lim_{x \to 1^{+}} \left(\frac{1}{\ln{x}} -
  \frac{1}{x-1}\right) $
\end{example}
\begin{solution}
  \begin{align*}
    \lim_{x \to 1^{+}} \frac{x-1- \ln{x}}{(x-1) \ln{x}})
    &\overset{\left(\frac{0}{0}\right)}{\underset{\mathrm{LH}}{=}} \lim_{x \to 1^{+}} 
    \frac{(x-1- \ln{x})'}{[(x-1) \ln{x}]'} = \lim_{x \to 1^{+}} \frac{1-
    \frac{1}{x}}{\ln{x} + (x-1) \cdot \frac{1}{x}} = \lim_{x \to 1^{+}} 
    \frac{x-1}{ x \ln{x} + x-1} \\
    &\overset{\left(\frac{0}{0}\right)}{\underset{\mathrm{LH}}{=}} \lim_{x \to 1^{+}} 
    \frac{(x-1)'}{(x \ln{x} +x-1)'} = \lim_{x \to 1^{+}} \frac{1}{\ln{x} +x \cdot
    \frac{1}{x}+1} = \lim_{x \to 1^{+}} \frac{1}{\ln{x}+2} = \frac{1}{2}
  \end{align*}
\end{solution}

\begin{example}
  Να υπολογιστεί το όριο $ \lim_{x \to 0^{+}} x^{x} $
\end{example}
\begin{solution}
  Παρατηρούμε ότι το όριο $ \lim_{x \to 0^{+}} x^{x} $ είναι απροσδιόριστο, της μορφής 
  $ 0^{0} $ . Επομένως:
  \begin{myitemize}
    \item \textbf{Λογαριθμίζουμε:} $ \ln{x^{x}} = x \ln{x} $
    \item \textbf{Όριο Λογαριθμισμένης:} $ \lim_{x \to 0^{+}} x \ln{x} \overset{(0 \cdot
      \infty)}{=} \lim_{x \to 0^{+}} \frac{\ln{x}}{\frac{1}{x}}
      \overset{\left(\frac{\infty}{\infty}\right)}{\underset{\mathrm{LH}}{=}}  
      \lim_{x \to 0^{+}} \frac{(\ln{x} )'}{(\frac{1}{x} )'} = \lim_{x \to 0^{+}}
      \frac{\frac{1}{x}}{- \frac{1}{x^{2}}} = \lim_{x \to 0^{+}} (-x) = 0 $
    \item \textbf{Ζητούμενο όριο:} $ \lim_{x \to 0^{+}} x^{x} = \lim_{x \to 0^{+}} = 
      \mathrm{e}^{0} = 1 $ 
  \end{myitemize}
\end{solution}

\begin{example}
  Να υπολογιστεί το όριο $ \lim_{x \to \infty} \Bigl(1+ \frac{1}{x}\Bigr)^{x} $
\end{example}
\begin{solution}
  Παρατηρούμε ότι το όριο $ \lim_{x \to \infty} \Bigl(1+ \frac{1}{x}\Bigr)^{x} $ 
  είναι απροσδιόριστο, της μορφής $ 1^{\infty} $ . Επομένως:
  \begin{myitemize}
    \item \textbf{Λογαριθμίζουμε:} $ \ln{\Bigl(1+ \frac{1}{x}\Bigr)^{x}} = 
      x \ln{\Bigl(1 + \frac{1}{x}\Bigr)} $
    \item \textbf{Όριο Λογαριθμισμένης:} $ \lim_{x \to \infty}\Bigl[x \ln{\Bigl(1+
      \frac{1}{x}\Bigr)}\Bigr] = \lim_{x \to \infty} x \cdot \lim_{x \to \infty} 
      \ln{\Bigl(1+ \frac{1}{x}\Bigr)}$
      \begin{myitemize}
        \item $ \lim_{x \to \infty} x = \infty $
        \item $ \lim_{x \to \infty} \ln{\Bigl(1+ \frac{1}{x}\Bigr)} = \ln{1} = 0 $, 
          γιατί $ \lim_{x \to \infty} \Bigl(1+ \frac{1}{x}\Bigr) = 
          1 + \lim_{x \to \infty} \frac{1}{x} = 1+0=1 $
      \end{myitemize}
    \item \textbf{Ζητούμενο όριο:} $ \lim_{x \to \infty} 
      \Bigl(1+ \frac{1}{x}\Bigr)^{x} = \mathrm{e}^{1} = \mathrm{e} $
  \end{myitemize}    
\end{solution}


\end{document}
