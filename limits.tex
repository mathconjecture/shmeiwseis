\documentclass[a4paper,12pt]{article}
\usepackage{etex}
%%%%%%%%%%%%%%%%%%%%%%%%%%%%%%%%%%%%%%
% Babel language package
\usepackage[english,greek]{babel}
% Inputenc font encoding
\usepackage[utf8]{inputenc}
%%%%%%%%%%%%%%%%%%%%%%%%%%%%%%%%%%%%%%

%%%%% math packages %%%%%%%%%%%%%%%%%%
\usepackage{amsmath}
\usepackage{amssymb}
\usepackage{amsfonts}
\usepackage{amsthm}
\usepackage{proof}

\usepackage{physics}

%%%%%%% symbols packages %%%%%%%%%%%%%%
\usepackage{dsfont}
\usepackage{stmaryrd}
%%%%%%%%%%%%%%%%%%%%%%%%%%%%%%%%%%%%%%%


%%%%%% graphicx %%%%%%%%%%%%%%%%%%%%%%%
\usepackage{graphicx}
\usepackage{color}
%\usepackage{xypic}
\usepackage[all]{xy}
\usepackage{calc}
%%%%%%%%%%%%%%%%%%%%%%%%%%%%%%%%%%%%%%%

\usepackage{enumerate}

\usepackage{fancyhdr}
%%%%% header and footer rule %%%%%%%%%
\setlength{\headheight}{14pt}
\renewcommand{\headrulewidth}{0pt}
\renewcommand{\footrulewidth}{0pt}
\fancypagestyle{plain}{\fancyhf{}
\fancyhead{}
\lfoot{}
\rfoot{\small \thepage}}
\fancypagestyle{vangelis}{\fancyhf{}
\rhead{\small \leftmark}
\lhead{\small }
\lfoot{}
\rfoot{\small \thepage}}
%%%%%%%%%%%%%%%%%%%%%%%%%%%%%%%%%%%%%%%

\usepackage{hyperref}
\usepackage{url}
%%%%%%% hyperref settings %%%%%%%%%%%%
\hypersetup{pdfpagemode=UseOutlines,hidelinks,
bookmarksopen=true,
pdfdisplaydoctitle=true,
pdfstartview=Fit,
unicode=true,
pdfpagelayout=OneColumn,
}
%%%%%%%%%%%%%%%%%%%%%%%%%%%%%%%%%%%%%%



\usepackage{geometry}
\geometry{left=25.63mm,right=25.63mm,top=36.25mm,bottom=36.25mm,footskip=24.16mm,headsep=24.16mm}

%\usepackage[explicit]{titlesec}
%%%%%% titlesec settings %%%%%%%%%%%%%
%\titleformat{\chapter}[block]{\LARGE\sc\bfseries}{\thechapter.}{1ex}{#1}
%\titlespacing*{\chapter}{0cm}{0cm}{36pt}[0ex]
%\titleformat{\section}[block]{\Large\bfseries}{\thesection.}{1ex}{#1}
%\titlespacing*{\section}{0cm}{34.56pt}{17.28pt}[0ex]
%\titleformat{\subsection}[block]{\large\bfseries{\thesubsection.}{1ex}{#1}
%\titlespacing*{\subsection}{0pt}{28.80pt}{14.40pt}[0ex]
%%%%%%%%%%%%%%%%%%%%%%%%%%%%%%%%%%%%%%

%%%%%%%%% My Theorems %%%%%%%%%%%%%%%%%%
\newtheorem{thm}{Θεώρημα}[section]
\newtheorem{cor}[thm]{Πόρισμα}
\newtheorem{lem}[thm]{λήμμα}
\theoremstyle{definition}
\newtheorem{dfn}{Ορισμός}[section]
\newtheorem{dfns}[dfn]{Ορισμοί}
\theoremstyle{remark}
\newtheorem{remark}{Παρατήρηση}[section]
\newtheorem{remarks}[remark]{Παρατηρήσεις}
%%%%%%%%%%%%%%%%%%%%%%%%%%%%%%%%%%%%%%%




\newcommand{\vect}[2]{(#1_1,\ldots, #1_#2)}
%%%%%%% nesting newcommands $$$$$$$$$$$$$$$$$$$
\newcommand{\function}[1]{\newcommand{\nvec}[2]{#1(##1_1,\ldots, ##1_##2)}}

\newcommand{\linode}[2]{#1_n(x)#2^{(n)}+#1_{n-1}(x)#2^{(n-1)}+\cdots +#1_0(x)#2=g(x)}

\newcommand{\vecoffun}[3]{#1_0(#2),\ldots ,#1_#3(#2)}




\everymath{\displaystyle}



\begin{document}



\chapter{Όριο συνάρτησης}

\begin{dfn}
	Έστω συνάρτηση $ y=f(x) $ η οποία είναι ορισμένη σε μια περιοχή $\delta$ του σημείου
	$x_{0}$, πιθανώς εξαιρουμένου του ίδιου του σημείου $x_{0}$. Θα λέμε ότι ο
	αριθμός $ l \in \mathbb{R} $ είναι το \textbf{όριο} της $ f(x) $ καθώς το $x$ τείνει
	στο $x_{0}$ και θα το συμβολίζουμε με $ \lim_{x\to x_{0}} f(x) = l $ αν για
	κάθε θετικό αριθμό $ \varepsilon $ υπάρχει θετικός αριθμός $ \delta = \delta
	(\varepsilon)$ τέτοιον ώστε αν $ \abs{x-x_{0}} < \delta $ τότε $ \abs{f(x)-
	l} < \varepsilon $.
\end{dfn}

Σε πιο συνεκτική μορφή ο παραπάνω ορισμός γράφεται ως

\[
	\lim_{x\to x_{0}} = l \overset{\text{ορ}}{\Leftrightarrow} 	\forall \varepsilon > 0,\, \exists \delta = \delta(\varepsilon) > 0 \, : \, \abs{x-x_{0}} < \delta
	\Rightarrow \abs{f(x) - l} < \varepsilon.
\] 

\begin{rems}
	\begin{enumerate}
		\mbox{}
		\item Όταν $ l = \pm \infty $ τότε λέμε ότι το όριο δεν υπάρχει
		\item Για το $ x_{0} $ ισχύει ότι $ x_{0} \in \mathbb{R} $ ή ότι $ x_{0}
			= \pm \infty $.
		\item Ορίζουμε τα πλευρικά όρια $ \lim_{x\to x_{0}^{-}} = l_1 $ και $
			\lim_{x\to x_{0}^{+}} = l_{2} $. Το όριο υπάρχει μόνο όταν $ l_{1},
			l_{2} \in \mathbb{R}$ και $ l_{1} = l_{2} $.
	\end{enumerate}
\end{rems}	

\section{Ιδιότητες των Ορίων}


Έστω $ k, x_{0} \in \mathbb{R} $, $ \lim_{x\to x_{0}} f(x) = l$ και $ \lim_{x\to x_{0}} g(x) = m$.


\begin{enumerate}
	\item $ \lim_{x\to x_{0}} k = k $, $ \forall k \in \mathbb{R} $
	\item $ \lim_{x\to x_{0}} x = x_{0} $
	\item $ \lim_{x\to x_{0}} f(x) \pm g(x) = l \pm m $
	\item $ \lim_{x\to x_{0}} k \cdot f(x) = k\cdot l $
	\item $ \lim_{x\to x_{0}} (f(x) \cdot g(x)) = l\cdot m $
	\item $ \lim_{x\to x_{0}} \left(\frac{f(x)}{g(x)}\right) = \frac{l}{m} $, $ m\neq 0 $
	\item $ \lim_{x\to x_{0}} (f(x))^{n} = l^{n} $ , $ n\in \mathbb{N} $ 
	\item $ \lim_{x\to x_{0}} \sqrt[n]{f(x)} = \sqrt[n]{l} $, $ n \in \mathbb{N^{*}} $
\end{enumerate}

	
\end{document}
