\input{preamble_ask.tex}
\input{definitions_ask.tex}
\input{tikz.tex}
\input{myboxes.tex}

% \setcounter{chapter}{1}

\geometry{top=2.0cm,left=1.5cm,right=1.5cm}

\input{insbox}

\newcommand{\twocolumnsidescc}[2]{\begin{minipage}[c]{0.30\linewidth}\raggedright
        #1
        \end{minipage}\quad\begin{minipage}[c]{0.30\linewidth}\raggedright
        #2
    \end{minipage}
}

\newcommand{\twocolumnsidesr}[2]{\begin{minipage}[t]{0.38\linewidth}\raggedright
        #1
        \end{minipage}\hfill\begin{minipage}[t]{0.60\linewidth}\raggedright
        #2
    \end{minipage}
}

\newcommand{\twocolumnsidesrr}[2]{\begin{minipage}[t]{0.45\linewidth}\raggedright
        #1
        \end{minipage}\hfill\begin{minipage}[t]{0.51\linewidth}\raggedright
        #2
    \end{minipage}
}

\newcommand{\twocolumnsidesll}[2]{\begin{minipage}[b]{0.68\linewidth}\raggedright
        #1
        \end{minipage}\hfill\begin{minipage}[b]{0.28\linewidth}\raggedright
        #2
    \end{minipage}
}



\everymath{\displaystyle}
\pagestyle{askhseis}

\usetikzlibrary{decorations.markings}

\tikzset{
  set arrow inside/.code={\pgfqkeys{/tikz/arrow inside}{#1}},
  set arrow inside={end/.initial=stealth, opt/.initial=},
  /pgf/decoration/Mark/.style={
    mark/.expanded=at position #1 with
    {
      \noexpand\arrow[\pgfkeysvalueof{/tikz/arrow inside/opt}]{\pgfkeysvalueof{/tikz/arrow inside/end}}
    }
  },
  arrow inside/.style 2 args={
    set arrow inside={#1},
    postaction={
      decorate,decoration={
        markings,Mark/.list={#2}
      }
    }
  },
}

\usetikzlibrary{shadings,patterns,arrows.meta,angles,quotes,shapes,
decorations.pathmorphing, decorations.shapes,decorations.text}

\tikzset{vangelis/.style = {ultra thick,magenta,dashed}, mystyle/.style={line width=3pt,blue!55,double,rounded corners}}

\pgfplotsset{
  myaxis/.style={axis lines=center,axis line style={thick,blue!50,-stealth},tick label
  style={font=\small,blue!50},xtick=\empty,ytick=\empty,tick style={blue!50}},
  myplot/.style={Col1!75,ultra thick,samples=500,no marks,smooth,},
  dashed lines/.style={dashed,blue!50,ultra thin},
  polax/.style={grid=none},
}


\begin{document}


\chapter*{Όριο Συνάρτησης}


\section*{Η έννοια του ορίου}

\InsertBoxL{2}{\parbox[b][7\baselineskip][c]{0.33\textwidth}{
    \begin{tikzpicture}
      \draw[->,blue!55,>=stealth] (-2,0) -- (3,0) node[below] {$x$} ;
      \draw[->,blue!55,>=stealth] (0,-0.5) -- (0,2.5) node [left] {$y$} ;
      \draw[color=Col1,samples=200,domain=-1.4:1.4,thick] plot (\x, {\x*\x}) 
        node[right] (f) {$y=f(x)$} [arrow inside={}{0.75}] ;
      \path[color=Col1,samples=200,domain=1.4:-1.4,thick] plot (\x, {\x*\x}) 
        [arrow inside={}{0.165}] ;
      \node[draw,Col1,circle,inner sep=1.5pt,fill=white] (a) at (1,1) {} ;
      \coordinate (b) at (1,0) ;
      \coordinate (c) at (0,1) ;
      \fill[Col1] (b) circle (1pt) node[below] {$ x_{0}$} ;
      \fill[Col1] (c) circle (1pt) node[left] {$a$} ;
      \draw[dashed] (a) -- (b) ;
      \draw[dashed] (a) -- (c) ;
      \node at (b) [Col2,below left,yshift=-2pt,xshift=-3pt] {$\rightarrow$} ;
      \node at (b) [Col2,below right,yshift=-2pt,xshift=3pt] {$\leftarrow$} ;
      \node at (c) [Col2,above left,yshift=2pt] {$\downarrow$} ;
      \node at (c) [Col2,below left,yshift=-2pt] {$\uparrow$} ;
    \end{tikzpicture}
}}

Έστω συνάρτηση $ f(x) $, η οποία είναι ορισμένη για τις διάφορες τιμές του $x$,
\textbf{κοντά} στο $ x_{0} $. Αυτό σημαίνει, ότι η $f$ είναι ορισμένη σε μια 
ανοιχτή \textbf{περιοχή} του $ x_{0} $, εκτός ίσως από το ίδιο το $ x_{0} $. 

Παρατηρούμε ότι για τις διάφορες τιμές του $x$ \textcolor{Col1}{κοντά} στο $ x_{0} $, 
οι τιμές του $y$, δηλαδή της $ f(x) $, είναι \textcolor{Col1}{κοντά} στο $a$. Μάλιστα, 
φαίνεται ότι καθώς το $x$ πλησιάζει \textbf{οσοδήποτε} κοντά στο $ x_{0} $ (και από τις 
δύο πλευρές), οι τιμές της $ f(x) $ πλησιάζουν \textbf{οσοδήποτε} κοντά στην τιμή $ a $. 
Αυτό, το εκφράζουμε λέγοντας ότι το \textcolor{Col1}{όριο} της $ f(x) $, καθώς το $x$ τείνει στο $ x_{0} $, είναι ίσο με $a$ και γράφουμε
\begin{empheq}[box=\mathboxg]{equation*}
  \lim_{x \to x_{0}} f(x) = a \quad \text{ή} \quad f(x) \to a \; \text{καθώς} \; x \to
  x_{0}
\end{empheq}

\begin{rem}
  Για τον υπολογισμό του ορίου, κάποιας συνάρτησης, καθώς το $ x $ τείνει στο $ x_{0} $, 
  εξετάζουμε τις τιμές της συνάρτησης κοντά στο $ x_{0} $ και ποτέ στο ίδιο το 
  $ x_{0} $. Μάλιστα, η συνάρτηση, δεν είναι απαραίτητο, να είναι ορισμένη στο $ x_{0} $.
  Το μόνο που μας ενδιαφέρει είναι η συμπεριφορά της συνάρτησης, \textbf{κοντά} στο 
  $ x_{0} $.
\end{rem}

\threecolumnsides{
  \begin{tikzpicture}
    \draw[->,blue!55,>=stealth] (-1,0) -- (3,0) node[right] {$x$} ;
    \draw[->,blue!55,>=stealth] (0,-0.7) -- (0,2.5) node [above] {$y$} ;
    \draw[color=Col1,samples=200,domain=-0.6:1.4,thick] plot (\x, {\x*\x}) 
      node[right] (f) {$y=f(x)$} [arrow inside={}{0.60}] ;
    \path[color=Col1,samples=200,domain=1.4:-0.6,thick] plot (\x, {\x*\x}) 
      [arrow inside={}{0.27}] ;
    \coordinate (b) at (1,0) ;
    \coordinate (c) at (0,1) ;
    \coordinate (p) at (1,1) ;
    \fill[Col1] (b) circle (1pt) node[below] {$ x_{0} $} ;
    \fill[Col1] (c) circle (1pt) node[left=1pt] {$a$} ;
    \draw[dashed] (c) -| (b);
    \node at (2,1) {$(a)$} ;
  \end{tikzpicture}
}{
  \begin{tikzpicture}
    \draw[->,blue!55,>=stealth] (-1,0) -- (3,0) node[right] {$x$} ;
    \draw[->,blue!55,>=stealth] (0,-0.7) -- (0,2.5) node [above] {$y$} ;
    \draw[color=Col1,samples=200,domain=-0.6:1.4,thick] plot (\x, {\x*\x}) 
      node[right] (f) {$y=f(x)$} [arrow inside={}{0.60}] ;
    \path[color=Col1,samples=200,domain=1.4:-0.6,thick] plot (\x, {\x*\x}) 
      [arrow inside={}{0.27}] ;
    \coordinate (b) at (1,0) ;
    \coordinate (c) at (0,1) ;
    \coordinate (e) at (0,2) ;
    \coordinate (d) at (1,2) ;
    \coordinate (p) at (1,1) ;
    \fill[Col1] (b) circle (1pt) node[below] {$ x_{0} $} ;
    \fill[Col1] (c) circle (1pt) node[left=1pt] {$a$} ;
    \fill[Col1] (e) circle (1pt) node[left=1pt] {$b$} ;
    \fill[Col1] (d) circle (2pt) ;
    \fill[Col1] (e) circle (1pt) ;
    \draw[dashed] (c) -| (b);
    \node[draw,Col1,circle,inner sep=1.5pt,fill=white] (p) at (1,1) {} ;
    \node at (2,1) {$(b)$} ;
  \end{tikzpicture}
}{
  \begin{tikzpicture}[]
    \draw[->,blue!55,>=stealth] (-1,0) -- (3,0) node[right] {$x$} ;
    \draw[->,blue!55,>=stealth] (0,-0.7) -- (0,2.5) node [above] {$y$} ;
    \draw[color=Col1,samples=200,domain=-0.6:1.4,thick] plot (\x, {\x*\x}) 
      node[right] (f) {$y=f(x)$} [arrow inside={}{0.60}] ;
    \path[color=Col1,samples=200,domain=1.4:-0.6,thick] plot (\x, {\x*\x}) 
      [arrow inside={}{0.27}] ;
    \coordinate (b) at (1,0) ;
    \coordinate (c) at (0,1) ;
    \coordinate (p) at (1,1) ;
    \fill[Col1] (b) circle (1pt) node[below] {$ x_{0} $} ;
    \fill[Col1] (c) circle (1pt) node[left=1pt] {$a$} ;
    \draw[dashed] (c) -| (b);
    \node[draw,Col1,circle,inner sep=1.5pt,fill=white] (p) at (1,1) {} ;
    \node at (2,1) {$(c)$} ;
  \end{tikzpicture}
}

\begin{myitemize}
  \item Στην περίπτωση $ (a) $ παρατηρούμε ότι $ f(x_{0}) = a $ και 
    $ \lim_{x \to x_{0}} f(x) = a $.
  \item Στην περίπτωση $ (b) $ παρατηρούμε ότι $ f(x_{0}) = b $, όμως 
    $ \lim_{x \to x_{0}} f(x) = a $.
  \item Στην περίπτωση $ (c) $ παρατηρούμε ότι $ f(x_{0}) $ δεν είναι ορισμένο, αλλά
    $ \lim_{x \to x_{0}} f(x) = a $.
\end{myitemize}


\section*{Πλευρικά Όρια Συνάρτησης}

\InsertBoxL{2}{\parbox[b][7.5\baselineskip][c]{0.33\textwidth}{
    \begin{tikzpicture}
      \draw[->,blue!55,>=stealth] (-2,0) -- (3,0) node[right] {$x$} ;
      \draw[->,blue!55,>=stealth] (0,-0.5) -- (0,2.8) node [left] {$y$} ;
      \draw[color=Col1,samples=200,domain=-1.35:1,thick] plot (\x, {\x*\x+1})
        node[right=10pt] (f) {$y=f(x)$} [arrow inside={}{0.90}] ;
      % \draw[color=Col1,samples=200,domain=2.5:1,thick] plot (\x, {-\x+1.5}) 
      %   [arrow inside={}{0.85}] ;
      \node[draw,Col1,circle,inner sep=1.5pt,fill=white] (b) at (1,2) {} ;
      \node[draw,Col1,circle,inner sep=1.5pt,fill=white] (c) at (1,0.5) {} ;
      \coordinate (e) at (0,0.5) ;
      \coordinate (g) at (0,2) ;
      \coordinate (d) at (1,0) ;
      \node (k) at (d) [Col1,below right] {$ x_{0} $} ;
      \draw[dashed] (b) -- (c) ;
      \draw[dashed] (c) -- (d) ;
      \draw[dashed] (a) -- (d) ;
      % \draw[dashed] (a) -- (h) ;
      \draw[dashed] (b) -- (g) ;
      \draw[dashed] (c) -- (e) ;
      \node at (d) [above left,yshift=-2pt] {$\rightarrow$} ;
      \node at (d) [Col2,above right,yshift=-2pt] {$\leftarrow$} ;
      \node at (e) [Col2,above left] {$\downarrow$} ;
      \node at (g) [below left,yshift=-3pt] {$\uparrow$} ;
      \fill[Col1] (e) circle (1pt) node[below left] {$a$} ;
      \fill[Col1] (g) circle (1pt) node[left] {$b$} ;
      \fill[Col1] (d) circle (1pt) ;
      \coordinate (k) at (2.5,1.5) ;
      \begin{scope}[decoration={markings,mark=at position 0.85 with {\arrow{stealth}}}]
        \draw[color=Col1,thick,postaction={decorate}] (k) -- (c) ;
      \end{scope}
    \end{tikzpicture}
}}

Στο διπλανό σχήμα, παρατηρούμε ότι η συνάρτηση $f$ πλησιάζει την τιμή $b$, καθώς το 
$x$ τείνει στο $ x_{0} $ από \textcolor{Col1}{αριστερά} αλλά η $f$ πλησιάζει την τιμή 
$a$, καθώς το $x$ τείνει στο $ x_{0} $ από \textcolor{Col1}{δεξιά}. Συμβολίζουμε
\begin{empheq}[box=\mathboxg]{equation*}
  \lim_{x \to x_{0}^{-}} f(x) = b \quad \text{και} \quad \lim_{x \to x_{0}^{+}} f(x) = a 
\end{empheq}
Ο συμβολισμός $ x \to x_{0}^{-} $ διαβάζεται ως «$x$ τείνει στο $ x_{0} $ από τα 
αριστερά» και σημαίνει ότι μας ενδιαφέρουν οι τιμές του $x$ κοντά στο $ x_{0} $, 
που είναι \textbf{μικρότερες} από το $ x_{0} $. Αντίστοιχα, 
ο συμβολισμός $ x \to x_{0}^{+} $ διαβάζεται ως «$x$ τείνει στο $ x_{0} $ από τα 
δεξιά» και σημαίνει ότι μας ενδιαφέρουν οι τιμές του $x$ κοντά στο $ x_{0} $, που
είναι \textbf{μεγαλύτερες} από το $ x_{0} $. 

\begin{prop}
  Το όριο μιας συνάρτησης υπάρχει, αν και μόνον αν, υπάρχουν τα πλευρικά όρια και είναι
  ίσα. Δηλαδή
  \begin{empheq}[box=\mathboxr]{equation*}
    \lim_{x \to x_{0}} f(x) = a \Leftrightarrow 
    \lim_{x \to x_{0}^{-}} f(x) = a \quad \text{και} \quad \lim_{x \to x_{0}^{+}} f(x) 
    = a
  \end{empheq}
\end{prop}


\section*{Ιδιότητες των Ορίων}

\begin{prop}
  Έστω $ c, x_{0} \in \mathbb{R} $ και έστω ότι τα όρια 
  $ \smash{\lim_{x\to x_{0}} f(x)} $ και $ \smash{\lim_{x\to x_{0}} g(x)} $,
  \textbf{υπάρχουν}. Έστω επίσης, $ p(x), q(x) $ πολυώνυμα.
  \begin{empheq}[box=\mathboxg]{equation*}
    \twocolumnsidesc{
      \begin{enumerate}
        \item $ \lim_{x \to x_{0}} c = c, \; \forall a \in \mathbb{R} $
        \item $ \lim_{x\to x_{0}} x = x_{0} $
        \item $ \lim_{x\to x_{0}} (f(x) \pm g(x)) = \lim_{x \to x_{0}} f(x) \pm
          \lim_{x \to x_{0}} g(x) $
        \item $ \lim_{x\to x_{0}} (c \cdot f(x)) = c\cdot \lim_{x \to x_{0}} f(x), \; 
          \forall c \in \mathbb{R} $
        \item $ \lim_{x\to x_{0}} (f(x) \cdot g(x)) = \lim_{x \to x_{0}} f(x) \cdot 
          \lim_{x \to x_{0}} g(x) $
        \item $ \lim_{x\to x_{0}} \Bigl(\frac{f(x)}{g(x)}\Bigr) = 
          \frac{\lim\limits_{x \to x_{0}} f(x)}{\lim\limits_{x \to x_{0}} g(x)}, 
          \;  \lim_{x \to x_{0}} g(x) \neq 0 $
      \end{enumerate}
    }{
      \begin{enumerate}[start=7]
        \item $ \lim_{x \to x_{0}} p(x) = p(x_{0}) $
        \item $ \lim_{x \to x_{0}} \frac{p(x)}{q(x)} = \frac{p(x_{0})}{q(x_{0})} , \;
          q(x_{0}) \neq 0 $
        \item $ \lim_{x \to x_{0}} \abs{f(x)} = \Bigl|\lim_{x \to x_{0}} f(x)\Bigr| $
        \item $ \lim_{x \to x_{0}} [f(x)]^{n} = 
          \Bigl[\lim_{x \to x_{0}} f(x)\Bigr]^{n}, \; n \in \mathbb{N}- \{ 0 \} $ 
        \item  $ \lim_{x \to x_{0}} \sqrt[n]{f(x)} = 
          \sqrt[n]{\lim_{x \to x_{0}} f(x)}, \; n \in \mathbb{N}, n \geq 2 $ 
      \end{enumerate}
    }
  \end{empheq}
\end{prop}

\begin{rem}
  Στην ιδιότητα 11, όταν $ n $ \textbf{άρτιος}, προϋποθέτουμε ότι 
  $ f(x) \geq 0, \; \forall x $, σε μια περιοχή του $ x_{0} $.
\end{rem}

\begin{rem}
  Προσοχή! Το αντίστροφο της ιδιότητας 9, δεν ισχύει πάντα. Για παράδειγμα, 
  έστω $ \smash{f(x) = \frac{\abs{x}}{x}} $. Τότε,
  \begin{myitemize}
    \item $ \lim_{x \to 0} \abs{f(x)} = \lim_{x \to 0} \abs{\frac{\abs{x}}{x}} =
      \lim_{x \to 0} \frac{\abs{x}}{\abs{x}} = 1 $
    \item $ \lim_{x \to 0} f(x) = \lim_{x \to 0} \frac{\abs{x}}{x} $, δεν
      υπάρχει, γιατί: $\lim_{x \to 0^{-}} f(x) = \lim_{x \to 0^{-}} \frac{-x}{x} = -1$ ενώ $\lim_{x \to 0^{+}} f(x) = \lim_{x \to 0^{+}} \frac{x}{x} = 1 $
  \end{myitemize}
\end{rem}

\begin{rem}
  Προσοχή! Το αντίστροφο της ιδιότητας 10 δεν ισχύει πάντα. Αν θεωρήσουμε την
  ίδια συνάρτηση, έχουμε 
  \[
    \lim_{x \to 0} \Bigl(\frac{\abs{x}}{x}\Bigr)^{2} = \lim_{x \to 0}
    \frac{x^{2}}{x^{2}} =1 \quad \text{όμως} \quad \lim_{x \to 0} f(x) = \lim_{x \to 0}
    \frac{\abs{x}}{x} \quad \text{δεν υπάρχει}
  \]
\end{rem}


\section*{Παραδείγματα}

\begin{example}
  $ \lim_{x \to -3} 5 = 5 $
\end{example}

\begin{example}
  $ \lim_{x \to 2} x = 2 $
\end{example}

\begin{example}
  $ \lim_{x \to -1} \Bigl(\frac{1}{x} - x^{3}\Bigr) = \lim_{x \to -1} \frac{1}{x}
  - \lim_{x \to -1} x^{3} = \frac{1}{-1} - (-1)^{3} = -1-(-1)=-1+1=0  $
\end{example}

\begin{example}
  $ \lim_{x \to 4} \bigl(3 \sqrt{x}\bigr) = 3 \lim_{x \to 4} \sqrt{x} = 3 \cdot \sqrt{4} 
  = 3\cdot 2 = 6$
\end{example}

\begin{example}
  $ \lim_{x \to 1} \bigl(\sqrt{x} \cdot \ln{x}\bigr) = \lim_{x \to 1} \sqrt{x} \cdot \lim_{x \to 1}
  \ln{x} = \sqrt{1} \cdot \ln{1} = 1 \cdot 0 = 0 $
\end{example}

\begin{example}
  $ \lim_{x \to -1} \Bigl(\frac{\mathrm{e}^{x}}{x}\Bigr) = \frac{\lim\limits_{x \to -1}
  \mathrm{e}^{x}}{\lim\limits_{x \to -1}x} = \frac{\mathrm{e}^{-1}}{-1} = -
  \mathrm{e}^{-1} = - \frac{1}{\mathrm{e}} $  
\end{example}

\begin{example}
  $ \lim_{x \to 2} (x^{3}+4x^{2}-3) = 2^{3}+ 4 \cdot 2^{2} -3 = 8+16-3 = 21 $  
\end{example}

\begin{example}
  $ \lim_{x \to 1} \frac{x^{4}+x^{2}-1}{x^{2}+5} = \lim_{x \to 1}
  \frac{1^{4}+1^{2}-1}{1^{2}+5} = \frac{1}{6} $
\end{example}

\begin{example}
  $ \lim_{x \to -2} \abs{-x^{3}+2x-7} = \abs{\lim_{x \to -2} (-x^{3}+2x-7)} = \abs{-3}
  = 3$ 
\end{example}

\begin{example}
  $ \lim_{x \to \pi} \cos^{4}{x} = \bigl(\lim_{x \to \pi} \cos{x} \bigr)^{4} =
  (\cos{\pi} )^{4} = (-1)^{4}=1 $
\end{example}

\begin{example} 
  $\lim_{x \to -1} (2x^{2}-1)^{4} = \bigl[\lim_{x \to -1}
  (2x^{2}-1)\bigr]^{4} = \bigl(2(-1)^{2}-1\bigr)^{4} = 1^{4} = 1 $ 
\end{example}

\begin{example}
  $ \lim_{x \to -2} \sqrt{4x^{2}-3} = \sqrt{\lim_{x \to -2} (4x^{2}-3)} =
  \sqrt{4(-2)^{2}-3} = \sqrt{16-3} = \sqrt{13} $    
\end{example}

\begin{example}
  $ \lim_{x \to 1} \sqrt[3]{x-2} = \sqrt[3]{\lim_{x \to 1} (x-2)} = \sqrt[3]{1-2} =
  \sqrt[3]{-1} = -1 $
\end{example}


\section*{Όρια με παρονομαστή 0}

\begin{prop}
\item {}
  \begin{myitemize}
    \item Αν $ \lim_{x \to x_{0}} g(x) = 0 $ με $ g(x)>0 $ σε μια περιοχή του $ x_{0} $ 
      τότε $ \lim_{x \to x_{0}} \frac{1}{g(x)} = + \infty $
    \item Αν $ \lim_{x \to x_{0}} g(x) = 0 $ με $ g(x)<0 $ σε μια περιοχή του $ x_{0} $ 
      τότε $ \lim_{x \to x_{0}} \frac{1}{g(x)} = - \infty $
  \end{myitemize}
\end{prop}

\begin{rem}
  Η παραπάνω πρόταση ισχύει και όταν $ x_{0} = + \infty $ ή $ - \infty $.
\end{rem}

\twocolumnsidesll{
\begin{example}
  $ \lim_{x \to 2^{-}} \frac{1}{x-2} = - \infty $, γιατί αφού $ x \to 2^{-} 
  \Rightarrow x < 2 $, άρα $ x-2<0 $ 
\end{example}
\begin{example}
  $ \lim_{x \to 2^{+}} \frac{1}{x-2} = + \infty $, γιατί αφού $ x \to 2^{+} 
  \Rightarrow x > 2 $, άρα $ x-2>0 $ 
\end{example}
}{
  \smash{\begin{tikzpicture}[scale=0.6]
    \draw[-stealth,blue!50] (-1.5,0) -- (4.5,0) node[below] {$x$};
    \draw[-stealth,blue!50] (0,-2.5) -- (0,2.5) node[left] {$y$};
    \draw[Col2] (2,-2.5) -- (2,2.5) ;
    \fill[Col2] (2,0) node[below right,Col1] {$2$} circle (2pt) ;
    \draw[domain=2.4:4.5,smooth,variable=\x,Col1!75,very thick] 
      plot (\x,{1/(\x-2)}) node[above left,xshift=10pt,yshift=10pt] 
      {\small$\frac{1}{x-2}$} ;
    \draw[domain=-1.3:1.6,smooth,variable=\x,Col1!75,very thick] plot (\x,{1/(\x-2)});
  \end{tikzpicture}
}}

\vspace{\baselineskip}

\twocolumnsidesr{
  \section*{Όρια στο άπειρο}
  \begin{prop}Έστω $ n \in \mathbb{N} $. Τότε
    \begin{mybox2}
    \item {}
      \begin{myitemize}
        \item $ \lim_{x \to +\infty} x^{n} = + \infty, \; n \in \mathbb{N} $ 
        \item $ \lim_{x \to -\infty} x^{n} = 
          \begin{cases}
            + \infty, & n \; \text{άρτιος} \\ 
            - \infty, & n \; \text{περιττός} \\ 
          \end{cases} $ 
        \item $ \lim_{x \to \pm \infty} \frac{1}{x^{n}} = 0 $
      \end{myitemize}
    \end{mybox2}
  \end{prop}
}{
  \section*{Πράξεις στο $ \overline{\mathbb{R}}=\mathbb{R} \cup \{+\infty, -\infty \}$}
  \begin{mybox3}
    \begin{myitemize}
      \item $ - \infty< \infty $ και $ - \infty < a < + \infty, \; \forall a \in 
        \mathbb{R} $
      \item $ + \infty + (+ \infty) = + \infty $ και $ - \infty + (- \infty) = - \infty $
      \item $ + \infty \pm a = + \infty $ και $ - \infty \pm a = - \infty, \; \forall a 
        \in \mathbb{R} $ 
    \end{myitemize}
  \end{mybox3} 
  \twocolumnsidesrr{
    \begin{mybox3}
      \begin{myitemize}
        \item $ (+ \infty)\cdot (+ \infty) = + \infty $
        \item $ (- \infty)\cdot (- \infty) = + \infty $
        \item $ (+ \infty)\cdot (- \infty) = - \infty $
        \item $ \frac{a}{\pm \infty} = 0, \; \forall a \in \mathbb{R} $
      \end{myitemize}
    \end{mybox3}
  }{
    \begin{mybox3}
      \begin{myitemize}
        \item $ a \cdot (+ \infty) = 
          \begin{cases}
            + \infty, & a>0 \\
            - \infty, & a<0 
          \end{cases}$ 
        \item $ a \cdot (- \infty) = 
          \begin{cases}
            - \infty, & a>0 \\
            + \infty, & a<0 
          \end{cases}$ 
      \end{myitemize}
    \end{mybox3}
}}


\section*{Όρια Ρητών Συναρτήσεων στο άπειρο}

\begin{prop}
  Αν $ p(x) = a_{n} x^{n} + a_{n-1}x^{n-1}+\cdots + a_{1}x + a_{0}, \; a_{n} \neq 0  $ 
  και $ q(x) = b_{m} x^{m} + b_{m-1}x^{m-1}+\cdots + b_{1}x + b_{0}, \; b_{m} \neq 0 $, 
  τότε, ισχύει 
  \begin{center}
    \twocolumnsidescc{
      \begin{empheq}[box=\mathboxg]{equation*}
        \lim_{x \to \pm\infty} p(x) = \lim_{x \to \pm \infty} a_{n}x^{n}
      \end{empheq}
    }{
      \begin{empheq}[box=\mathboxg]{equation*}
        \lim_{x \to \pm\infty} \frac{p(x)}{q(x)} = \lim_{x \to \pm \infty}
        \frac{a_{n}x^{n}}{b_{m}x^{m}}
      \end{empheq}
    }
  \end{center}
\end{prop}

\begin{example}
  $ \lim_{x \to \infty} (-2x^{3}+x^{2-1}) = \lim_{x \to \infty} (-2x^{3}) = -2
  \lim_{x \to \infty} x^{3} = -2 \cdot \infty = - \infty $
\end{example}
\begin{example}
  $ \lim_{x \to - \infty} \frac{2x^{2}-x+2}{x^{3}-x+1} = \lim_{x \to - \infty}
  \frac{2 x^{2}}{x^{3}} = \lim_{x \to - \infty} \frac{2}{x} = 0 $
\end{example}
\begin{example}
  $ \lim_{x \to \infty} \frac{-2x^{2}-x+1}{3x^{2}-2} = \lim_{x \to \infty}
  \frac{-2 x^{2}}{3 x^{2}} = \lim_{x \to \infty} \frac{-2}{3} = - 
  \frac{2}{3} $ 
\end{example}
\begin{example}
  $ \lim_{x \to - \infty} \frac{-3x^{5}+2x-1}{x^{2}+1} = \lim_{x \to - \infty}
  \frac{-3x^{5}}{x^{2}} = \lim_{x \to - \infty} (-3x^{3}) = -3 \lim_{x \to - \infty}
  x^{3} = -3 \cdot (- \infty) = + \infty $
\end{example}
\begin{example}
$ \lim_{x \to \infty} \sqrt{\frac{x^{2}+x-1}{x+2}} = \sqrt{\lim_{x \to \infty}
\frac{x^{2}+x-1}{x+2}} = \sqrt{\lim_{x \to \infty} \frac{x^{2}}{x}} = \sqrt{\lim_{x \to 
  \infty} x} = \infty $
  \end{example}


\newpage


\section*{Χρήσιμες Προτάσεις}

\begin{prop}
  Αν μια συνάρτηση $ f $ έχει όριο στο σημείο $ x_{0} $, τότε αυτό είναι
  \textbf{μοναδικό}.
\end{prop}


\begin{prop}
  Αν $ \lim_{x \to x_{0}} f(x) = a \in \mathbb{R} - {\{ 0 \}} $, τότε υπάρχει 
  περιοχή του $ x_{0} $ ώστε οι τιμές της $ f(x) $ να είναι \textbf{ομόσημες} του $a$ 
  για κάθε $x$ σε αυτή την περιοχή, δηλαδή:
  \begin{myitemize}
    \item Αν $ a > 0 $ τότε $ f(x)>0 $ 
    \item Αν $ a < 0 $ τότε $ f(x)<0 $ 
  \end{myitemize}
\end{prop}

\begin{prop}
\item {}
  \begin{myitemize}
    \item Αν $ \lim_{x \to x_{0}} f(x) = a \in \mathbb{R} $ και $ f(x) \geq 0 $ για κάθε 
      $x$ σε μια περιοχή του $ x_{0} $, τότε $ a \geq 0 $ 
    \item Αν $ \lim_{x \to x_{0}} f(x) = a \in \mathbb{R} $ και $ \lim_{x \to x_{0}} g(x)
      = b \in \mathbb{R}$ και $ f(x) \geq g(x) $ για κάθε $x$ σε μια περιοχή του 
      $ x_{0} $, τότε $ a \geq b $ 
  \end{myitemize}
\end{prop}

\begin{prop}
  Ισχύουν οι παρακάτω ισοδυναμίες
  \begin{empheq}[box=\mathboxg]{equation*}
    \twocolumnsidesc{
      \begin{enumerate}
        \item $ \lim_{x \to x_{0}} f(x) = a \Leftrightarrow \lim_{x \to x_{0}} (f(x)-a)=0 $
        \item $ \lim_{x \to x_{0}} f(x) = a \Leftrightarrow \lim_{x \to x_{0}} \abs{f(x)-a}=0 $
        \item $ \lim_{x \to x_{0}} f(x) = 0 \Leftrightarrow \lim_{x \to x_{0}} \abs{f(x)}=0 $ 
      \end{enumerate}
    }{
      \begin{enumerate}[start=4]
        \item $ \lim_{x \to x_{0}} f(x) = a \Leftrightarrow \lim_{x \to x_{0}} (-f(x))=-a $ 
        \item $ \lim_{x \to x_{0}} f(x) = a \Leftrightarrow \lim_{h \to 0} f(x_{0}+h)=a  $
        \item $ \lim_{x \to x_{0}} f(x) = a \Leftrightarrow \lim_{h \to 1} f(x_{0}\cdot
          h)=a, \; x_{0} \neq 0 $
      \end{enumerate}
    }
  \end{empheq}
\end{prop}


\begin{prop}[\textcolor{Col1}{Κριτήριο Παρεμβολής}]
\item {}
  \begin{minipage}[t]{8.0 cm}
    \begin{myitemize}
      \item $ h(x) \leq f(x) \leq g(x), \; \forall x$ σε μια περιοχή του $ x_{0} $
        \hfill\tikzmark{a}
      \item $ \lim_{x \to x_{0}} h(x) = \lim_{x \to x_{0}} g(x) = l \in \mathbb{R} $
        \hfill\tikzmark{b}
    \end{myitemize}
  \end{minipage}
  \mybrace{a}{b}[ $ \lim_{x \to x_{0}} f(x) = l $ ]
\end{prop}

\begin{rem}
  Η παραπάνω πρόταση ισχύει και όταν $ x_{0} = + \infty $ ή $ - \infty $.
\end{rem}

\begin{example}
  Να δείξετε ότι $ \lim_{x \to 0} \Bigl(x^{2} \sin{\frac{1}{x}}\Bigr) = 0  $ 
\end{example}
\begin{solution}
  Καταρχας, παρατηρούμε ότι δεν μπορούμε να εφαρμόσουμε τις ιδιότητες των ορίων, και να 
  «σπάσουμε» ενδεχομένως, το όριο, στο γινόμενο των ορίων, γιατί, το όριο 
  $ \lim_{x \to 0} \sin{1/x} $, δεν υπάρχει.
  Επομένως, έχουμε, ότι
  \[
    - x^{2} \leq x^{2} \sin{\frac{1}{x}} \leq x^{2}, \; \forall x \in \mathbb{R}^{*}
  \]
  και επειδή $ \lim_{x \to 0} (-x^{2}) = \lim_{x \to 0} x^{2} = 0 $, σύμφωνα με το 
  Κριτήριο Παρεμβολής, προκύπτει ότι 
  \[
    \lim_{x \to 0} \Bigl(x^{2} \sin{\frac{1}{x}}\Bigr)  = 0
  \]
\end{solution}

\begin{cor}
\item {}
  \begin{minipage}[t]{6.5 cm}
    \begin{myitemize}
      \item $ \abs{f(x)} \leq g(x) $ σε μια περιοχή του $ x_{0} $
        \hfill\tikzmark{a}
      \item $ \lim_{x \to x_{0}} g(x) = 0 $
        \hfill\tikzmark{b}
    \end{myitemize}
  \end{minipage}
  \mybrace{a}{b}[ $ \lim_{x \to x_{0}} f(x) = 0 $ ]
\end{cor}

\begin{example}
  Να δείξετε ότι $ \lim_{x \to 0} \Bigl(x \cos{\frac{1}{x}}\Bigr) = 0 $
\end{example}
\begin{solution}
  Έχουμε ότι
  \[
    \Bigl|x \cos{\frac{1}{x}}\Bigr| = \abs{x} \cdot 
    \Bigl|\cos{\frac{1}{x}}\Bigr| \leq \abs{x} \cdot 1 = 
    \abs{x}, \; \forall x \in \mathbb{R}
  \] 
  Όμως $ \lim_{x \to 0} \abs{x} = 0 $
  Επομένως από το πόρισμα του Κριτηρίου Παρεμβολής, έχουμε ότι 
  \[
    \lim_{x \to 0} \Bigl(x \cos{\frac{1}{x}}\Bigr) = 0 
  \] 
\end{solution}


\section*{Απροσδιόριστες Μορφές}


\subsection*{Όριο Ρητής συνάρτησης της μορφής $\Bigl(\frac{0}{0}\Bigr) $}

Μπορούμε να κάνουμε άρση της απροσδιοριστίας, παραγοντοποιώντας αριθμητή και παρονομαστή.

\begin{example}
  $ \lim_{x \to 1} \frac{x^{2}-3x+2}{x^{2}-1} \overset{(\frac{0}{0})}{=} \lim_{x \to 1}
  \frac{(x-2)\cancel{(x-1)}}{\cancel{(x-1)}(x+1)} = \lim_{x \to 1} \frac{x-2}{x+1} = 
  \frac{1-2}{1+1} = - \frac{1}{2} $
\end{example}


\subsection*{Όριο συνάρτησης $ \frac{f(x)}{g(x)} $ με ρίζα της μορφής $\Bigl(\frac{0}{0}\Bigr)$}

Μπορούμε να κάνουμε άρση της απροσδιοριστίας, πολλαπλασιάζοντας με την κατάλληλη 
συζυγή παράσταση.

\begin{center}
  \setcounter{chapter}{2}
  \begin{Mytable}
    \renewcommand{\arraystretch}{1.5}
    \begin{tabular}{|c|c||c|c|}
      \TabCellHead Ρίζα  &  \TabCellHead Συζυγής Παράσταση & 
      \TabCellHead Ρίζα  &  \TabCellHead Συζυγής Παράσταση \\ \hline
      $ \sqrt{a} $ & $ \sqrt{a} $ & $ a + \sqrt{b} $ & $ a - \sqrt{b} $ \\ \hline
      $ \sqrt{a} + \sqrt{b} $ & $ \sqrt{a} - \sqrt{b} $ &
      $ \sqrt{a} - \sqrt{b} $ & $ \sqrt{a} + \sqrt{b} $ \\ \hline 
      $ \sqrt[3]{a} $ & $ \sqrt[3]{a^{2}} $ &
      $ \sqrt[n]{a^{m}} $ & $ \sqrt[n]{a^{n-m}} $ \\ \hline 
      $ \sqrt[3]{a} - \sqrt[3]{b} $ & $ \sqrt[3]{a^{2}} + \sqrt[3]{ab} +
      \sqrt[3]{b^{2}} $ &
      $ \sqrt[3]{a} + \sqrt[3]{b} $ & $ \sqrt[3]{a^{2}} - \sqrt[3]{ab} +
      \sqrt[3]{b^{2}} $ \\ \hline
    \end{tabular}
  \end{Mytable}
\end{center}

\begin{example}
  \begin{align*}
    \lim_{x \to 0} \frac{\sqrt{x^{2}+4} - 2}{x^{2}} 
    &\overset{(\frac{0}{0})}{=}  
    \lim_{x \to 0} \frac{(\sqrt{x^{2}+4} -2)(\sqrt{x^{2}+4} +2)}{x^{2}(\sqrt{x^{2}+4} +2)}
    = \lim_{x \to 0} \frac{(\sqrt{x^{2}+4})^{2}-2^{2}}{x^{2}(\sqrt{x^{2}+4} +2)} \\ 
    &= \lim_{x \to 0} \frac{x^{2}+\cancel{4}-\cancel{4}}{x^{2}(\sqrt{x^{2}+4} +2)} =
    \lim_{x \to 0} \frac{\cancel{x}^{2}}{\cancel{x}^{2}(\sqrt{x^{2}+4} +2)} = \lim_{x \to 0} \frac{1}{\sqrt{x^{2}+4} +2} = \frac{1}{4}  
  \end{align*}
\end{example}

\begin{rem}
  Φυσικά, και για τις δύο προηγούμενες περιπτώσεις, όπου κάνουμε άρση της
  απροσδιοριστίας με αλγεβρικό τρόπο, θα μπορούσαμε να εφαρμόσουμε τον κανόνα 
  L'H\^{o}pital.
\end{rem}

\section*{Κανόνας L'H\^{o}pital}

\begin{thm}
  Έστω $ f, g $ παραγωγίσιμες συναρτήσεις, με $ g'(x) \neq 0 $ σε μια ανοιχτή περιοχή
  του $ x_{0} $ (εκτός ίσως από το ίδιο το $ x_{0} $). Έστω επίσης, ότι
  \begin{gather*}
    \lim_{x \to x_{0}} f(x) = 0 \quad \text{και} \quad \lim_{x \to x_{0}} g(x) = 0
    \qquad \color{Col1}{\text{Περίπτωση $ \Bigl(\frac{0}{0}\Bigr) $}} \\
    \text{ή} \\
    \lim_{x \to x_{0}} f(x) = \pm \infty \quad \text{και} \quad \lim_{x \to x_{0}} g(x)
    = \pm \infty \qquad \color{Col1}{\text{Περίπτωση $ \Bigl(\frac{\infty}{\infty}\Bigr) $}}
  \end{gather*}
  Τότε
  \begin{empheq}[box=\mathboxg]{equation*}
    \lim_{x \to x_{0}} \frac{f(x)}{g(x)} = \lim_{x \to x_{0}} \frac{f'(x)}{g'(x)} 
  \end{empheq}
  με την προϋπόθεση ότι υπάρχει το όριο $ \lim_{x \to x_{0}} \frac{f'(x)}{g'(x)} $, 
  και είναι πεπερασμένο ή άπειρο.
\end{thm}

\begin{rem}
\item {}
  \begin{enumerate}
    \item Αν $ x_{0} \in \mathbb{R} $, τότε ο κανόνας ισχύει, και όταν 
      $ x \to x_{0}^{-} $ ή $ x \to x_{0}^{+} $.
    \item Ο κανόνας ισχύει, ακόμη και όταν $ x_{0} = \infty $ ή $ x_{0} = - \infty $.
  \end{enumerate}
\end{rem}


\section*{Παραδείγματα}

\subsection*{Περίπτωση $ \Bigl(\frac{0}{0}\Bigr) $}

Σε αυτήν την περίπτωση, εφαρμόζουμε τον κανόνα L' H\^{o}pital.

\begin{example}
  Να υπολογιστεί το όριο $ \lim_{x \to 1} \frac{\ln{x}}{x-1} $
\end{example}
\begin{solution}
  $ \lim_{x \to 1} \frac{\ln{x}}{x-1}
  \overset{\left(\frac{0}{0}\right)}{\underset{\mathrm{LH}}{=}}  \lim_{x \to 1} 
  \frac{(\ln{x} )'}{(x-1)'} = \lim_{x \to 1} \frac{\frac{1}{x}}{1} = 
  \lim_{x \to 1} \frac{1}{x} = 1 $
\end{solution}

\begin{example}
  Να υπολογιστεί το όριο $ \lim_{x \to 0} \frac{3x- \sin{x}}{x} $
\end{example}
\begin{solution}
  $ \lim_{x \to 0} \frac{3x- \sin{x}}{x} \overset{(\frac{0}{0})}{=} \lim_{x \to 0}
  \frac{(3x- \sin{x} )'}{(x)'} = \lim_{x \to 0} \frac{3- \cos{x}}{1} = \frac{3- 1}{1}
  = 2 $
\end{solution}

\begin{example}
  Να υπολογιστεί το όριο $ \lim_{x \to 0} \frac{\sqrt{1+x} -1}{x} $
\end{example}
\begin{solution}
  $ \lim_{x \to 0} \frac{\sqrt{1+x} -1}{x} \overset{(\frac{0}{0})}{=} \lim_{x \to 0}
  \frac{(\sqrt{1+x} -1)'}{(x)'} = \lim_{x \to 0} \frac{\frac{1}{2 \sqrt{1+x}}}{1} = 
  \lim_{x \to 0} \frac{1}{2 \sqrt{1+x}} = \frac{1}{2} $
\end{solution}

\begin{example}
  Να υπολογιστεί το όριο $ \lim_{x \to 0} \frac{x- \sin{x}}{x^{3}} $
\end{example}
\begin{solution} 
  $\lim_{x \to 0} \frac{x - \sin{x}}{x^{3}}
  \overset{\left(\frac{0}{0}\right)}{\underset{\mathrm{LH}}{=}} 
  \lim_{x \to 0} \frac{1 - \cos{x}}{3x^{2}}
  \overset{\left(\frac{0}{0}\right)}{\underset{\mathrm{LH}}{=}} 
  \lim_{x \to 0} \frac{\sin{x}}{6x} 
  \overset{\left(\frac{0}{0}\right)}{\underset{\mathrm{LH}}{=}} 
  \lim_{x \to 0} \frac{\cos{x}}{6} = \frac{1}{6}$ 
\end{solution}


\subsection*{Περίπτωση $ \bigl(\frac{\infty}{\infty}\bigr) $}

Σε αυτήν την περίπτωση, εφαρμόζουμε τον κανόνα L' H\^{o}pital.

\begin{example}
  Να υπολογιστεί το όριο $ \lim_{x \to \infty} \frac{\mathrm{e}^{x}}{x^{2}} $
\end{example}
\begin{solution}
  $ \lim_{x \to \infty} \frac{\mathrm{e}^{x}}{x^{2}}
  \overset{\left(\frac{\infty}{\infty}\right)}{\underset{\mathrm{LH}}{=}} \lim_{x \to
  \infty} \frac{(\mathrm{e}^{x} )'}{(x^{2})'} = \lim_{x \to \infty}
  \frac{\mathrm{e}^{x}}{2x}
  \overset{\left(\frac{\infty}{\infty}\right)}{\underset{\mathrm{LH}}{=}} 
  \lim_{x \to \infty} \frac{(\mathrm{e}^{x})'}{(2x)'} = \lim_{x \to \infty}
  \frac{\mathrm{e}^{x}}{2} = \infty
  $ 
\end{solution}

\begin{example}
  Να υπολογιστεί το όριο $ \lim_{x \to \infty} \frac{\ln{x}}{\sqrt{x}} $
\end{example}
\begin{solution}
  $ \lim_{x \to \infty} \frac{\ln{x}}{\sqrt{x}}
  \overset{\left(\frac{\infty}{\infty}\right)}{\underset{\mathrm{LH}}{=}} 
  \lim_{x \to \infty} \frac{(\ln{x} )'}{(\sqrt{x} )'} = \lim_{x \to \infty}
  \frac{\frac{1}{x}}{\frac{1}{2 \sqrt{x}}} = \lim_{x \to \infty} 
  \frac{2 \sqrt{x} }{x} = \lim_{x \to \infty} \frac{2}{\sqrt{x}} = 0
  $
\end{solution}


\subsection*{Περίπτωση $ (0 \cdot \infty) $}

Αν το όριο $ \lim_{x \to x_{0}} \left(f(x)\cdot g(x)\right) $, όπου $ x_{0} \in
\mathbb{R} $ ή $ x_{0}= \pm \infty $, είναι απροσδιόριστο της μορφής $ (0 \cdot \pm
\infty) $, τότε μετατρέπουμε το γινόμενο $ f(x) \cdot g(x) $ σε πηλίκο, όπως παρακάτω:
\[
  f(x)\cdot g(x) = \frac{f(x)}{\frac{1}{g(x)}} \quad \text{ή} \quad
  f(x) \cdot g(x) = \frac{g(x)}{\frac{1}{f(x)}}  
\] 
και έτσι αναγόμστε σε όριο της μορφής $ (\frac{0}{0}) $ ή $
(\frac{\infty}{\infty}) $ όπου μπορούμε να εφαρμόσουμε τον κανόνα L' H\^{o}pital.

\begin{example}
  Να υπολογιστεί το όριο $ \lim_{x \to 0^{+}} (x \ln{x}) $
\end{example}
\begin{solution}
  $ \lim_{x \to 0^{+}} (x \ln{x}) \overset{(0 \cdot (-\infty))}{=} \lim_{x \to 0^{+}}
  \frac{\ln{x}}{\frac{1}{x}}
  \overset{\left(\frac{\infty}{\infty}\right)}{\underset{\mathrm{LH}}{=}} 
  \lim_{x \to 0^{+}} \frac{(\ln{x} )'}{(\frac{1}{x} )'} = \lim_{x \to 0^{+}}
  \frac{\frac{1}{x}}{- \frac{1}{x{2}}} = \lim_{x \to 0^{+}} \Bigl(-\frac{x^{2}}{x}\Bigr) 
  = \lim_{x \to 0^{+}} (-x) = 0 $
\end{solution}

\begin{example}
  Να υπολογιστεί το όριο $ \lim_{x \to \infty} (x \mathrm{e}^{-x}) $
\end{example}
\begin{solution}
  $ \lim_{x \to \infty} (x \mathrm{e}^{-x} ) \overset{(\infty\cdot 0)}{=} \lim_{x \to
  \infty} \frac{x}{\mathrm{e}^{x}}
  \overset{\left(\frac{\infty}{\infty}\right)}{\underset{\mathrm{LH}}{=}} 
  \lim_{x \to \infty} \frac{(x)'}{(\mathrm{e}^{x} )'} = \lim_{x \to \infty}
  \frac{1}{\mathrm{e}^{x}} = 0 $
\end{solution}


\subsection*{Περίπτωση $ (\infty - \infty)$}

Αν το όριο $ \lim_{x \to x_{0}} \left(f(x) \pm g(x)\right) $, όπου $ x_{0} \in
\mathbb{R} $ ή $ x_{0}= \pm \infty $, είναι απροσδιόριστο της μορφής $ (\infty-
\infty) $, τότε βγάζουμε κοινό παράγοντα μία εκ των δύο συναρτήσεων, όπως παρακάτω: 
\[
  f(x) \pm g(x) = f(x)\Bigl(1 \pm \frac{g(x)}{f(x)}\Bigr) \quad \text{ή} \quad f(x) \pm g(x) =
  g(x)\Bigl(\frac{f(x)}{g(x)} - 1\Bigr)
\] 
και έτσι αναγόμστε σε όριο της μορφής $ \Bigl(\frac{0}{0}\Bigr) $ ή $
\Bigl(\frac{\infty}{\infty}\Bigr) $ όπου μπορούμε να εφαρμόσουμε τον κανόνα L' H\^{o}pital.


\begin{example}
  Να υπολογιστεί το όριο $ \lim_{x \to \infty} (\mathrm{e}^{x} -x) $
\end{example}
\begin{solution}
  $ \lim_{x \to \infty} (\mathrm{e}^{x} - x) = \lim_{x \to \infty} x(\mathrm{e}^{x}
  - 1) = \lim_{x \to \infty} x \cdot \lim_{x \to \infty} \Bigl(\frac{\mathrm{e}^{x}}{x}
  -1\Bigr) $
  \begin{myitemize}
    \item $ \lim_{x \to \infty} x = \infty $
    \item $ \lim_{x \to \infty} \left(\frac{\mathrm{e}^{x}}{x} -1\right) = 
      \lim_{x \to \infty} \frac{\mathrm{e}^{x}}{x} -1
      \overset{\left(\frac{\infty}{\infty}\right)}{\underset{\mathrm{LH}}{=}} 
      \lim_{x \to \infty} \frac{(\mathrm{e}^{x} )'}{(x)'} -1 = \lim_{x \to \infty}
      \frac{\mathrm{e}^{x}}{1} -1 = \lim_{x \to \infty} \mathrm{e}^{x} -1 = \infty- 1 
      = \infty$
  \end{myitemize}
  Επομένως $ \lim_{x \to \infty} x \cdot \lim_{x \to \infty}
  \Bigl(\frac{\mathrm{e}^{x}}{x} -1 \Bigr) = \infty\cdot \infty = \infty$ 
\end{solution}

\begin{example}
  Να υπολογιστεί το όριο $ \lim_{x \to 1^{+}} \left(\frac{1}{\ln{x}} -
  \frac{1}{x-1}\right) $
\end{example}
\begin{solution} 
  $\lim_{x \to 1^{+}} \frac{x-1- \ln{x}}{(x-1) \ln{x}})
    \overset{\left(\frac{0}{0}\right)}{\underset{\mathrm{LH}}{=}} \!
    \lim_{x \to 1^{+}} \frac{1- \frac{1}{x}}{\ln{x} + (x-1) \frac{1}{x}} = \!
    \lim_{x \to 1^{+}} \frac{x-1}{ x \ln{x} + x-1} 
    \overset{\left(\frac{0}{0}\right)}{\underset{\mathrm{LH}}{=}} \!
    \lim_{x \to 1^{+}} \frac{1}{\ln{x} +x \frac{1}{x}+1} =
    \lim_{x \to 1^{+}} \frac{1}{\ln{x}+2} = \frac{1}{2}$ 
\end{solution}


\subsection*{Περιπτώσεις $ (1^{\infty}), \; (0^{0}), \; (\infty^{0}) $}

Αν το όριο $ \lim_{x \to x_{0}} f(x)^{g(x)} $, όπου $ x_{0} \in
\mathbb{R} $ ή $ x_{0}= \pm \infty $, είναι απροσδιόριστο της μορφής 
$ (1^{\infty}) $, $ (0^0) $ ή $ (\infty^{0}) $ 
τότε εφαρμόζουμε την παρακάτω διαδικασία, των 3 βημάτων:
\begin{myitemize}
  \item \textcolor{Col1}{Λογαριθμίζουμε τη συνάρτηση $ f(x)^{g(x)} $:} $ \ln{\bigl[f(x)^{g(x)}\bigr]} = g(x)\cdot \ln{f(x)} $
  \item \textcolor{Col1}{Υπολογίζουμε το όριο της λογαριθμισμένης συνάρτησης:} 
    $ \lim_{\smash{x \to x_{0}}} \left[g(x)\cdot \ln{f(x)}\right] = \ldots = a $
  \item \textcolor{Col1}{Υπολογίζουμε το ζητούμενο όριο:} 
    $ \lim_{x \to x_{0}} \smash{f(x)^{g(x)}} = \mathrm{e}^{a} $
\end{myitemize}


\begin{example}
  Να υπολογιστεί το όριο $ \lim_{x \to 0^{+}} x^{x} $
\end{example}
\begin{solution}
  Παρατηρούμε ότι το όριο $ \lim_{x \to 0^{+}} x^{x} $ είναι απροσδιόριστο, της μορφής 
  $ 0^{0} $ . Επομένως:
  \begin{myitemize}
    \item \textbf{Λογαριθμίζουμε:} $ \ln{x^{x}} = x \ln{x} $
    \item \textbf{Όριο Λογαριθμισμένης:} $ \lim_{x \to 0^{+}} x \ln{x} \overset{(0 \cdot
      \infty)}{=} \lim_{x \to 0^{+}} \frac{\ln{x}}{\frac{1}{x}}
      \overset{\left(\frac{\infty}{\infty}\right)}{\underset{\mathrm{LH}}{=}}  
      \lim_{x \to 0^{+}} \frac{(\ln{x} )'}{(\frac{1}{x} )'} = \lim_{x \to 0^{+}}
      \frac{\frac{1}{x}}{- \frac{1}{x^{2}}} = \lim_{x \to 0^{+}} (-x) = 0 $
    \item \textbf{Ζητούμενο όριο:} $ \lim_{x \to 0^{+}} x^{x} = \lim_{x \to 0^{+}} = 
      \mathrm{e}^{0} = 1 $ 
  \end{myitemize}
\end{solution}

\begin{example}
  Να υπολογιστεί το όριο $ \lim_{x \to \infty} \Bigl(1+ \frac{1}{x}\Bigr)^{x} $
\end{example}
\begin{solution}
  Παρατηρούμε ότι το όριο $ \lim_{x \to \infty} \Bigl(1+ \frac{1}{x}\Bigr)^{x} $ 
  είναι απροσδιόριστο, της μορφής $ 1^{\infty} $ . Επομένως:
  \begin{myitemize}
    \item \textbf{Λογαριθμίζουμε:} $ \ln{\Bigl(1+ \frac{1}{x}\Bigr)^{x}} = 
      x \ln{\Bigl(1 + \frac{1}{x}\Bigr)} $
    \item \textbf{Όριο Λογαριθμισμένης:} $ \lim_{x \to \infty}\Bigl[x \ln{\Bigl(1+
      \frac{1}{x}\Bigr)}\Bigr] = \lim_{x \to \infty} x \cdot \lim_{x \to \infty} 
      \ln{\Bigl(1+ \frac{1}{x}\Bigr)}$
      \begin{myitemize}
        \item $ \lim_{x \to \infty} x = \infty $
        \item $ \lim_{x \to \infty} \ln{\Bigl(1+ \frac{1}{x}\Bigr)} = \ln{1} = 0 $, 
          γιατί $ \lim_{x \to \infty} \Bigl(1+ \frac{1}{x}\Bigr) = 
          1 + \lim_{x \to \infty} \frac{1}{x} = 1+0=1 $
      \end{myitemize}
    \item \textbf{Ζητούμενο όριο:} $ \lim_{x \to \infty} 
      \Bigl(1+ \frac{1}{x}\Bigr)^{x} = \mathrm{e}^{1} = \mathrm{e} $
  \end{myitemize}    
\end{solution}




\end{document}
