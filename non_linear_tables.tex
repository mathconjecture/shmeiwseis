\documentclass[a4paper,12pt]{article}
\usepackage{etex}
%%%%%%%%%%%%%%%%%%%%%%%%%%%%%%%%%%%%%%
% Babel language package
\usepackage[english,greek]{babel}
% Inputenc font encoding
\usepackage[utf8]{inputenc}
%%%%%%%%%%%%%%%%%%%%%%%%%%%%%%%%%%%%%%

%%%%% math packages %%%%%%%%%%%%%%%%%%
\usepackage{amsmath}
\usepackage{amssymb}
\usepackage{amsfonts}
\usepackage{amsthm}
\usepackage{proof}

\usepackage{physics}

%%%%%%% symbols packages %%%%%%%%%%%%%%
\usepackage{dsfont}
\usepackage{stmaryrd}
%%%%%%%%%%%%%%%%%%%%%%%%%%%%%%%%%%%%%%%


%%%%%% graphicx %%%%%%%%%%%%%%%%%%%%%%%
\usepackage{graphicx}
\usepackage{color}
%\usepackage{xypic}
\usepackage[all]{xy}
\usepackage{calc}
%%%%%%%%%%%%%%%%%%%%%%%%%%%%%%%%%%%%%%%

\usepackage{enumerate}

\usepackage{fancyhdr}
%%%%% header and footer rule %%%%%%%%%
\setlength{\headheight}{14pt}
\renewcommand{\headrulewidth}{0pt}
\renewcommand{\footrulewidth}{0pt}
\fancypagestyle{plain}{\fancyhf{}
\fancyhead{}
\lfoot{}
\rfoot{\small \thepage}}
\fancypagestyle{vangelis}{\fancyhf{}
\rhead{\small \leftmark}
\lhead{\small }
\lfoot{}
\rfoot{\small \thepage}}
%%%%%%%%%%%%%%%%%%%%%%%%%%%%%%%%%%%%%%%

\usepackage{hyperref}
\usepackage{url}
%%%%%%% hyperref settings %%%%%%%%%%%%
\hypersetup{pdfpagemode=UseOutlines,hidelinks,
bookmarksopen=true,
pdfdisplaydoctitle=true,
pdfstartview=Fit,
unicode=true,
pdfpagelayout=OneColumn,
}
%%%%%%%%%%%%%%%%%%%%%%%%%%%%%%%%%%%%%%



\usepackage{geometry}
\geometry{left=25.63mm,right=25.63mm,top=36.25mm,bottom=36.25mm,footskip=24.16mm,headsep=24.16mm}

%\usepackage[explicit]{titlesec}
%%%%%% titlesec settings %%%%%%%%%%%%%
%\titleformat{\chapter}[block]{\LARGE\sc\bfseries}{\thechapter.}{1ex}{#1}
%\titlespacing*{\chapter}{0cm}{0cm}{36pt}[0ex]
%\titleformat{\section}[block]{\Large\bfseries}{\thesection.}{1ex}{#1}
%\titlespacing*{\section}{0cm}{34.56pt}{17.28pt}[0ex]
%\titleformat{\subsection}[block]{\large\bfseries{\thesubsection.}{1ex}{#1}
%\titlespacing*{\subsection}{0pt}{28.80pt}{14.40pt}[0ex]
%%%%%%%%%%%%%%%%%%%%%%%%%%%%%%%%%%%%%%

%%%%%%%%% My Theorems %%%%%%%%%%%%%%%%%%
\newtheorem{thm}{Θεώρημα}[section]
\newtheorem{cor}[thm]{Πόρισμα}
\newtheorem{lem}[thm]{λήμμα}
\theoremstyle{definition}
\newtheorem{dfn}{Ορισμός}[section]
\newtheorem{dfns}[dfn]{Ορισμοί}
\theoremstyle{remark}
\newtheorem{remark}{Παρατήρηση}[section]
\newtheorem{remarks}[remark]{Παρατηρήσεις}
%%%%%%%%%%%%%%%%%%%%%%%%%%%%%%%%%%%%%%%




\newcommand{\vect}[2]{(#1_1,\ldots, #1_#2)}
%%%%%%% nesting newcommands $$$$$$$$$$$$$$$$$$$
\newcommand{\function}[1]{\newcommand{\nvec}[2]{#1(##1_1,\ldots, ##1_##2)}}

\newcommand{\linode}[2]{#1_n(x)#2^{(n)}+#1_{n-1}(x)#2^{(n-1)}+\cdots +#1_0(x)#2=g(x)}

\newcommand{\vecoffun}[3]{#1_0(#2),\ldots ,#1_#3(#2)}





\begin{document}

    
\chapter{Ευστάθεια Σημείων Ισορροπίας}


\vspace{\baselineskip}

% \begin{Mytable}
%     \renewcommand{\arraystretch}{1.5}
%     \begin{tabular}{c|c|c} 
%     \TabRowHead \TabCellHead Ιδιοτιμές & \TabCellHead Κρίσιμο Σημείο & \TabCellHead 
%     Ευστάθεια 
%     \\ \hline
%     $ \boldsymbol{\lambda_{1} > \lambda_{2} > 0} $ & κόμβος & ασταθής  
%     \\ \hline
%     $ \lambda_{1} < \lambda_{2} < 0 $ & κόμβος & ασυμπτ. ευσταθής  
%     \\\hline
%     $ \lambda_{2} < 0 < \lambda_{1} $ & σάγμα & ασταθής  
%     \\\hline
%     $ \lambda_{1} = \lambda_{2} = \lambda > 0 $ & άστρο ή εκφυλ. κόμβος & ασταθής   
%     \\ \hline
%     $ \lambda_{1} = \lambda_{2} = \lambda < 0 $ & άστρο ή εκφυλ. κόμβος & 
%     ασυμπτ. ευσταθής   
%     \\\hline
%     $ \lambda _{1,2} = \lambda \pm i \mu $ & &  
%     \\\hline
%     $ \quad \lambda > 0 $ & εστία & ασταθής 
%     \\\hline
%     $ \quad \lambda < 0 $ & εστία & ασυμπτ. ευσταθής 
%     \\\hline
%     $ \quad \lambda = 0 $ & κέντρο & ευσταθές
% \end{tabular}   
% \end{Mytable}

\vspace{\baselineskip}


{\centering
\begin{Mytable}
    \renewcommand{\arraystretch}{1.5}
\begin{tabular}{c|c|c||c|c} 
    \TabRowHead & \multicolumn{2}{c||}{\TabCellHead \large Γραμμικά Συστήματα} &
   \multicolumn{2}{c}{\TabCellHead \large Μη-Γραμμικά Συστήματα}  
   \\ \hline
    \TabRowHead \TabCellHead Ιδιοτιμές & \TabCellHead Κρίσιμο Σημείο & \TabCellHead 
    Ευστάθεια & \TabCellHead Κρίσιμο Σημείο & \TabCellHead Ευστάθεια  
    \\ \hline  
    $ \lambda_{1} > \lambda_{2} > 0 $ & κόμβος & ασταθής & κόμβος & ασταθές  
    \\ \hline
    $ \lambda_{1} < \lambda_{2} < 0 $ & κόμβος & ασυμπτ. ευσταθής & κόμβος & ασταθές  
    \\ \hline
    $ \lambda_{2} < 0 < \lambda_{1} $ & σάγμα & ασταθής & σάγμα & ασταθές  
    \\ \hline
    $ \lambda_{1} = \lambda_{2} = \lambda > 0 $ & άστρο ή εκφυλ. κόμβος & ασταθής & 
    κόμβος ή σπείρα & ασταθές    \\ \hline
    $ \lambda_{1} = \lambda_{2} = \lambda < 0 $ & άστρο ή εκφυλ. κόμβος & 
    ασυμπτ. ευσταθής & κόμβος ή σπείρα & ασυμπτ. ευσταθές    
    \\\hline
    $ \lambda _{1,2} = \lambda \pm i \mu $ & & & &  
    \\ \hline
    $ \quad \lambda > 0 $ & εστία & ασταθής & σπείρα & ασταθές  
    \\ \hline
    $ \quad \lambda < 0 $ & εστία & ασυμπτ. ευσταθής & σπείρα & ασυμπτ. ευσταθές  
    \\ \hline
    $ \quad \lambda = 0 $ & κέντρο & ευσταθές & κέντρο ή σπείρα & ? 
\end{tabular}   
\end{Mytable}}

\newpage

\chapter{Ταξινόμηση Πορτραίτων Φάσης}

\vspace{2\baselineskip}

\begin{comment}
:Title: Poincare Diagram, Classification of Phase Portraits
:Features: 
:Tags: Arcs;Foreach;Markings;Diagrams;Plots;Mathematics
:Author: Gernot Salzer
:Slug: poincare

The solutions of a system of linear differential equations can be
classified according to the trace and the determinant of the
coefficient matrix. This diagram show schematically the different
types of solutions.

Originally published on TeX.SX, tex.stackexchange.com/a/347401, 6 Jan 2017
Based on a manual drawing by Douglas R. Hundley,
http://people.whitman.edu/~hundledr/courses/M244/Poincare.pdf

You may use the code without any restrictions; no rights reserved. 
\end{comment}
\usetikzlibrary{decorations.markings}

\tikzset
 {every pin/.style = {pin edge = {Col2,<-},Col2}, % pins are arrows from label to point
  > = stealth,                            % arrow tips look like stealth bombers
  flow/.style =    % everything marked as "flow" will be decorated with an arrow
   {decoration = {markings, mark=at position #1 with {\arrow{>}}},
    postaction = {decorate}
   },
  flow/.default = 0.5,          % default position of the arrow is in the middle
  main/.style = {line width=1pt,Col2}                    % thick lines for main graph
 }

% \newtemplate[Scaling, default 0.18]{\NameOfTemplate}{Caption}{Code}
%
% Typesets Code and stores it in the box \NameOfTemplate.
% This way we avoid nested tikzpictures when inserting the templates into the
% main picture, since nesting is not guaranteed to work.
\newcommand\newtemplate[4][0.18]%
 {\newsavebox#2%
  \savebox#2%
   {\begin{tabular}{@{}c@{}}
      \begin{tikzpicture}[Col2,scale=#1]
      #4
      \end{tikzpicture}\\[-1ex]
      \templatecaption{#3}\\[-1ex]
    \end{tabular}%
   }%
 }
\newcommand\template[1]{\usebox{#1}}             % use the Code stored in box #1
\newcommand\templatecaption[1]{{\sffamily\scriptsize#1}}       % typeset caption
\renewcommand\Tr{\mathop{\mathrm{Tr}}}

\newtemplate[0.2]\sink{\textcolor{Col1}{ευσταθής κόμβος}}%
 {\foreach \sx in {+,-}                   % for right/left half do:
   {\draw[flow] (\sx4,0) -- (0,0);        %   draw half of horizontal axis
    \draw[flow] (0,\sx4) -- (0,0);        %   draw half of vertical axis
    \foreach \sy in {+,-}                 %   for upper/lower quadrant do:
      \foreach \a/\b in {2/1,3/0.44}      %     draw two half-parabolas
        \draw[flow,domain=\sx\a:0] plot (\x, {\sy\b*\x*\x});
   }
 }

 \newtemplate[0.2]\source{\textcolor{Col1}{ασταθής κόμβος}}%
 {\foreach \sx in {+,-}                   % for right/left half do:
   {\draw[flow] (0,0) -- (\sx4,0);        %   draw half of horizontal axis
    \draw[flow] (0,0) -- (0,\sx4);        %   draw half of vertical axis
    \foreach \sy in {+,-}                 %   for upper/lower quadrant do:
      \foreach \a/\b in {2/1,3/0.44}      %     draw two half-parabolas
        \draw[flow,domain=0:\sx\a] plot (\x, {\sy\b*\x*\x});
   }
 }

 \newtemplate[0.2]\spiralsink{\textcolor{Col1}{ευσταθής εστία}}%
 {\draw (-4,0) -- (4,0);                  % draw horizontal axis
  \draw (0,-4) -- (0,4);                  % draw vertical axis
  \draw [samples=100,smooth,domain=27:7]  % draw spiral
       plot ({\x r}:{0.005*\x*\x});       % Using "flow" here gives "Dimension
  \def\x{26}                              %        too large", so we draw a tiny
  \draw[->] ({\x r}:{0.005*\x*\x}) -- +(0.01,-0.01);%     tangent for the arrow.
 }

 \newtemplate[0.2]\spiralsource{\textcolor{Col1}{ασταθής εστία}}%
 {\draw (-4,0) -- (4,0);                  % draw horizontal axis
  \draw (0,-4) -- (0,4);                  % draw vertical axis
  \draw [samples=100,smooth,domain=10:28] % draw spiral
       plot ({-\x r}:{0.005*\x*\x});      % Using "flow" here gives "Dimension
  \def\x{27.5}                            %        too large", so we draw a tiny
  \draw[<-] ({-\x r}:{0.005*\x*\x}) -- +(0.01,-0.008);%   tangent for the arrow.
 }

 \newtemplate[0.17]\centre{\textcolor{Col1}{κέντρο}}% British spelling since \center is in use
 {\draw (-4,0) -- (4,0);                  % draw horizontal axis
  \draw (0,-4) -- (0,4);                  % draw vertical axis
  \foreach \r in {1,2,3}                  % draw three circles
    \draw[flow=0.63] (\r,0) arc (0:-360:\r cm);
 }

 \newtemplate\saddle{\textcolor{Col1}{σάγμα}}%
 {\foreach \sx in {+,-}                   % for right/left half do:
   {\draw[flow] (\sx4,0) -- (0,0);        %   draw half of horizontal axis
    \draw[flow] (0,0) -- (0,\sx4);        %   draw half of vertical axis
    \foreach \sy in {+,-}                 %   for upper/lower quadrant do:
      \foreach \a/\b/\c/\d in {2.8/0.3/0.7/0.6, 3.9/0.4/1.3/1.1}
        \draw[flow] (\sx\a,\sy\b)         %     draw two bent lines
          .. controls (\sx\c,\sy\d) and (\sx\d,\sy\c)
          .. (\sx\b,\sy\a);
   }
 }

 \newtemplate[0.2]\degensink{\textcolor{Col1}{ευσταθ. εκφ. κόμβος}}%
 {\draw (0,-4) -- (0,4);                  % draw vertical axis
  \foreach \s in {+,-}                    % for upper/lower half do:
   {\draw[flow] (\s4,0) -- (0,0);         %   draw half of horizontal axis
    \foreach \a/\b/\c/\d in {3.5/4/1.5/1, 2.5/2/1/0.8}
      \draw[flow] (\s-3.5,\s\a)           %   draw two bent lines
        .. controls (\s\b,\s\c) and (\s\b,\s\d)
        .. (0,0);
   }
 }

 \newtemplate[0.2]\degensource{\textcolor{Col1}{ασταθ. εκφ. κόμβος}}%
 {\draw (0,-4) -- (0,4);                  % draw vertical axis
  \foreach \s in {+,-}                    % for upper/lower half do:
   {\draw[flow] (0,0) -- (\s4,0);         %   draw half of horizontal axis
    \foreach \a/\b/\c/\d in {3.5/4/1.5/1, 2.5/2/1/0.8}
      \draw[flow] (0,0)                   %   draw two bent lines
        .. controls (\s\b,\s\d) and (\s\b,\s\c)
        .. (\s-3.5,\s\a);
   }
 }

    \centering
\begin{tikzpicture}[scale=1.2,line cap=round,line join=round]
    % MAIN DIAGRAM
    \draw [main,->] (0,-0.3) -- (0,4.7)                            % vertical axis
        node [label={[above]$\scriptstyle\det A$}] {}
        node [label={[above,yshift=0.8cm,Col1]%
            { \minibox[c]{\large Ταξινόμηση Πορτραίτων Φάσης \\ 
        \small\color{Col2} $ \Delta = Tr^{2}(A) - 4\det(A) $}}}] {};
    \draw [main,->] (-5,0) -- (5,0)                              % horizontal axis
        node [label={[right,yshift=-0.5ex]$\scriptstyle\Tr A$}] {}; 
    \draw [main, domain=-4:4,Col1] plot (\x, {0.25*\x*\x});                % main graph
    \node at (-4,4) [pin={[above,Col1]$\scriptstyle\Delta=0$}] {};
    \node at ( 4,4) [pin={[above,align=left,Col1]%
        {$\scriptstyle\Delta=0$}}] {};
    % TEMPLATES describing areas
    \node at ( 0  ,-1.4) {\template\saddle};
    \node at (-4  , 1  ) {\template\sink};
    \node at ( 4  , 1  ) {\template\source}; 
    \node at (-1.8, 3.7) {\template\spiralsink};
    \node at ( 1.8, 3.7) {\template\spiralsource};
    % TEMPLATES labeling lines and points
    \node at ( 0  , 1.2) [pin={[draw,right,xshift=0.3cm]%
        \template\centre}] {};
    % \node at (-3  , 0  ) [pin={[draw,below,yshift=-1cm]%
        % \template\stablefp}] {};
    % \node at ( 3  , 0  ) [pin={[draw,below,yshift=-1cm]%
        % \template\unstablefp}] {};
    \node at (-3.5,{0.25*3.5*3.5}) [pin={[draw,left,xshift=-1.15cm,yshift=-0.3cm]%
        \template\degensink}] {};
    \node at ( 3.5,{0.25*3.5*3.5}) [pin={[draw,right,xshift=0.9cm,yshift=-0.3cm]%
        \template\degensource}] {};
    % \node at ( 0  , 1  ) [pin={[left=1em]%
        % \templatecaption{$\Delta < 0$}\\[-1ex]}] {};
    \node at (0,1) [above left=1em,xshift=-1em,Col1] {$\scriptstyle\Delta<0$} ;
    \node at (-3,0) [below=2em,xshift=-1em,Col1] {$\scriptstyle\Delta>0$} ;
    \node at (3,0) [below=2em,xshift=-1em,Col1] {$\scriptstyle\Delta>0$} ;

\end{tikzpicture}



\end{document}
