\documentclass[a4paper,12pt]{article}
\usepackage{etex}
%%%%%%%%%%%%%%%%%%%%%%%%%%%%%%%%%%%%%%
% Babel language package
\usepackage[english,greek]{babel}
% Inputenc font encoding
\usepackage[utf8]{inputenc}
%%%%%%%%%%%%%%%%%%%%%%%%%%%%%%%%%%%%%%

%%%%% math packages %%%%%%%%%%%%%%%%%%
\usepackage{amsmath}
\usepackage{amssymb}
\usepackage{amsfonts}
\usepackage{amsthm}
\usepackage{proof}

\usepackage{physics}

%%%%%%% symbols packages %%%%%%%%%%%%%%
\usepackage{dsfont}
\usepackage{stmaryrd}
%%%%%%%%%%%%%%%%%%%%%%%%%%%%%%%%%%%%%%%


%%%%%% graphicx %%%%%%%%%%%%%%%%%%%%%%%
\usepackage{graphicx}
\usepackage{color}
%\usepackage{xypic}
\usepackage[all]{xy}
\usepackage{calc}
%%%%%%%%%%%%%%%%%%%%%%%%%%%%%%%%%%%%%%%

\usepackage{enumerate}

\usepackage{fancyhdr}
%%%%% header and footer rule %%%%%%%%%
\setlength{\headheight}{14pt}
\renewcommand{\headrulewidth}{0pt}
\renewcommand{\footrulewidth}{0pt}
\fancypagestyle{plain}{\fancyhf{}
\fancyhead{}
\lfoot{}
\rfoot{\small \thepage}}
\fancypagestyle{vangelis}{\fancyhf{}
\rhead{\small \leftmark}
\lhead{\small }
\lfoot{}
\rfoot{\small \thepage}}
%%%%%%%%%%%%%%%%%%%%%%%%%%%%%%%%%%%%%%%

\usepackage{hyperref}
\usepackage{url}
%%%%%%% hyperref settings %%%%%%%%%%%%
\hypersetup{pdfpagemode=UseOutlines,hidelinks,
bookmarksopen=true,
pdfdisplaydoctitle=true,
pdfstartview=Fit,
unicode=true,
pdfpagelayout=OneColumn,
}
%%%%%%%%%%%%%%%%%%%%%%%%%%%%%%%%%%%%%%



\usepackage{geometry}
\geometry{left=25.63mm,right=25.63mm,top=36.25mm,bottom=36.25mm,footskip=24.16mm,headsep=24.16mm}

%\usepackage[explicit]{titlesec}
%%%%%% titlesec settings %%%%%%%%%%%%%
%\titleformat{\chapter}[block]{\LARGE\sc\bfseries}{\thechapter.}{1ex}{#1}
%\titlespacing*{\chapter}{0cm}{0cm}{36pt}[0ex]
%\titleformat{\section}[block]{\Large\bfseries}{\thesection.}{1ex}{#1}
%\titlespacing*{\section}{0cm}{34.56pt}{17.28pt}[0ex]
%\titleformat{\subsection}[block]{\large\bfseries{\thesubsection.}{1ex}{#1}
%\titlespacing*{\subsection}{0pt}{28.80pt}{14.40pt}[0ex]
%%%%%%%%%%%%%%%%%%%%%%%%%%%%%%%%%%%%%%

%%%%%%%%% My Theorems %%%%%%%%%%%%%%%%%%
\newtheorem{thm}{Θεώρημα}[section]
\newtheorem{cor}[thm]{Πόρισμα}
\newtheorem{lem}[thm]{λήμμα}
\theoremstyle{definition}
\newtheorem{dfn}{Ορισμός}[section]
\newtheorem{dfns}[dfn]{Ορισμοί}
\theoremstyle{remark}
\newtheorem{remark}{Παρατήρηση}[section]
\newtheorem{remarks}[remark]{Παρατηρήσεις}
%%%%%%%%%%%%%%%%%%%%%%%%%%%%%%%%%%%%%%%




\newcommand{\vect}[2]{(#1_1,\ldots, #1_#2)}
%%%%%%% nesting newcommands $$$$$$$$$$$$$$$$$$$
\newcommand{\function}[1]{\newcommand{\nvec}[2]{#1(##1_1,\ldots, ##1_##2)}}

\newcommand{\linode}[2]{#1_n(x)#2^{(n)}+#1_{n-1}(x)#2^{(n-1)}+\cdots +#1_0(x)#2=g(x)}

\newcommand{\vecoffun}[3]{#1_0(#2),\ldots ,#1_#3(#2)}




\pagestyle{vangelis}


\begin{document}

\chapter{Γραμμικοί Μετασχηματισμοί}

\section{Ορισμός - Παραδείγματα}

\begin{dfn}
  Έστω $ (V, \oplus _{V}, \odot _{V}) $ και $ (W, \oplus _{W}, \odot _{W}) $ δύο 
  $ \mathbb{K} - $χώροι. Μια συνάρτηση $ T \colon V \to W $ λέγεται 
  \textcolor{Col1}{γραμμικός μετασχηματισμός} αν ικανοποιεί τις παρακάτω συνθήκες:
  \begin{enumerate}[i)]
    \item $ T(\mathbf{x} \oplus _{V} \mathbf{y}) = T(\mathbf{x}) \oplus _{W}
      T(\mathbf{y}), \quad \forall \mathbf{x}, \mathbf{y}  \in V $
    \item $ T(\lambda \odot _{V} \mathbf{x}) = \lambda \odot _{W} T(\mathbf{x}), 
      \quad \forall \lambda \in \mathbb{K} $ και $ \forall \mathbf{x} \in V $
  \end{enumerate}
  Αν $T$ είναι ένας γραμμικός μετασχηματισμός από τον χώρο $V$ στον $V$ τότε λέμε 
  ότι ο $T$ είναι ένας \textcolor{Col1}{γραμμικός τελεστής} επί του $V$. 
\end{dfn}

\begin{rem}
  Συνήθως, αν δεν υπάρχει σύγχυση, γράφουμε απλώς $ T(\mathbf{x}+ \mathbf{y}) =
  T(\mathbf{x}) + T(\mathbf{y}) $ και 
  $ T(\lambda \mathbf{x}) = \lambda T(\mathbf{x}) $, για τις δυο συνθήκες του ορισμού.
\end{rem}

\begin{prop}
  Έστω $ V $ και $W$ δύο $ \mathbb{K}- $χώροι. Τότε 
  \begin{enumerate}[i)]
    \item Αν $ T \colon V \to W $ είναι ένας γραμμικός μετασχηματισμός, τότε 
      $ T(\mathbf{0}_{V}) = \mathbf{0}_{W} $
    \item $ T \colon V \to W $ γραμμικός μετασχηματισμός $ \Leftrightarrow T(k
      \mathbf{x} + \lambda \mathbf{y}) = = k T(\mathbf{x}) + \lambda T(\mathbf{y}) $
  \end{enumerate}
\end{prop}

\begin{proof}
\item 
  \begin{enumerate}[i)]
    \item 
      $ 
      \left.
        \begin{matrix*}[l]
          T(\mathbf{0}_{V}) = T(\mathbf{0}_{V}+ \mathbf{0}_{V}) = 
          T(\mathbf{0}_{V}) + T(\mathbf{0}_{V})  \\ 
          T(\mathbf{0}_{V}) = T(\mathbf{0}_{V}) + \mathbf{0}_{W}
        \end{matrix*} 
      \right\}
      \Rightarrow \cancel{T(\mathbf{0}_{V})} + T(\mathbf{0}_{V}) =
      \cancel{T(\mathbf{0}_{V})} + \mathbf{0}_{W} \Rightarrow  
      T(\mathbf{0}_{V}) = \mathbf{0}_{W} $
    \item 
  \end{enumerate}
\end{proof}

\begin{rem}
  Αν $ T \colon V \to W $ είναι ένας γραμμικός μετασχηματισμός, τότε για κάθε 
  $ n \in \mathbb{N} $, $ \mathbf{x}_{i} \in V $ και $ \lambda _{i} \in \mathbb{K} $, 
  $ i = 1,\ldots, n $ ισχύει:
  \[
    T\left(\sum_{i=1}^{n} \lambda _{i} \mathbf{x}_{i}\right) = \sum_{i=1}^{n} 
    \lambda _{i} T(\mathbf{x}_{i}) 
  \] 
\end{rem}

\begin{example}[Ταυτοτικός]
  Έστω $V$ ένας $ \mathbb{K} - $χώρος και έστω $ T \colon V \to V $ η ταυτοτική 
  συνάρτηση επί του $V$ με 
  \[
    T(\mathbf{x}) = \mathbf{x}, \quad \mathbf{x} \in V   
  \]
  Τότε η $T$ είναι γραμμικός μετασχηματισμός και λέγεται 
  \textcolor{Col1}{ταυτοτικός} μετασχηματισμός επί του $V$ και συμβολίζεται με $ I_{V} $.
\end{example}

\begin{example}[Μηδενικός]
  Έστω $V$ και $W$ δυο $ \mathbb{K}- $χώροι και έστω $ T \colon V \to W $ με 
  \[
    T(\mathbf{x}) = \mathbf{0}_{W}, \quad \forall \mathbf{x} \in V
  \]
  Τότε η $T$ είναι γραμμικός μετασχηματισμός και λέγεται 
  \textcolor{Col1}{μηδενικός} μετασχηματισμός επί του $V$ και συμβολίζεται με $ T_{0} $.
\end{example}

\begin{example}[Προβολή επί του άξονα $x$]
  Έστω ο $ \mathbb{R}- $χώρος $ \mathbb{R}^{2} $ με τις συνήθεις πράξεις και έστω 
  $ T \colon \mathbb{R}^{2} \to \mathbb{R}^{2} $ με 
  \[
    T(\mathbf{x}, \mathbf{y}) = (x, 0) 
  \] 
  Τότε η $T$ είναι γραμμικός μετασχηματισμός και λέγεται \textcolor{Col1}{προβολή} επί 
  του άξονα $x$.
\end{example}

\begin{example}[Συμμετρία ως προς άξονα $x$]
  Έστω ο $ \mathbb{R}- $χώρος $ \mathbb{R}^{2} $ με τις συνήθεις πράξεις και έστω 
  $ T \colon \mathbb{R}^{2} \to \mathbb{R}^{2} $ με 
  \[
    T(\mathbf{x}, \mathbf{y}) = (x, -y) 
  \] 
  Τότε η $T$ είναι γραμμικός μετασχηματισμός και λέγεται \textcolor{Col1}{συμμετριά} ως 
  προς τον άξονα $x$.
\end{example}

\begin{example}
  Έστω ο $ \mathbb{R}- $χώρος $ \mathbb{R}^{2} $ με τις συνήθεις πράξεις και έστω 
  $ T \colon \mathbb{R}^{2} \to \mathbb{R}^{2} $ με 
  \[
    T(\mathbf{x}, \mathbf{y}) = (x,2x+y) 
  \] 
  Τότε η $T$ είναι γραμμικός μετασχηματισμός, αφού, αν $ \mathbf{x}=(x_{1}, y_{1}) $ 
  και $ \mathbf{y}=(x_{2}, y_{2}) $, τότε $ \mathbf{x}+ \mathbf{y} = 
  (x_{1}+ x_{2}, y_{1}+ y_{2}) $ και $ \lambda \mathbf{x} = 
  (\lambda x_{1}, \lambda x_{2}) $ έχουμε:
  \begin{align*}
    T(\mathbf{x}+ \mathbf{y}) &= T(x_{1}+ x_{2}, y_{1}+ y_{2}) \\
                              &= (x_{1}+ x_{2}, 2(x_{1}+ x_{2})+ y_{1}+ y_{2}) \\
                              &= (x_{1}, 2 x_{1}+ y_{1}) + (x_{2}, 2 x_{2}+ y_{2}) \\
                              &= T(\mathbf{x}) + T(\mathbf{y})
                              \intertext{και}
    T(\lambda \mathbf{x}) &= T(\lambda x_{1}, \lambda y_{1}) \\
                          &= (\lambda x_{1}, 2 \lambda x_{1}+ \lambda y_{1}) \\
                          &= \lambda (x_{1}, 2 x_{1}+ y_{1}) \\
                          &= \lambda T(\mathbf{x})
  \end{align*}

\end{example}

\begin{example}
  Έστω ο $ \mathbb{R}- $χώρος $ \mathbb{R}^{2} $ με τις συνήθεις πράξεις και έστω 
  $ T \colon \mathbb{R}^{2} \to \mathbb{R}^{2} $ με 
  \[
    T(\mathbf{x}, \mathbf{y}) = (x+1, y) 
  \] 
  Τότε η $T$ δεν είναι γραμμικός μετασχηματισμός, αφού $ T(0,0) = (1,0) \neq (0,0) $ 
\end{example}

\begin{example}[Αριστερόστροφη Περιστροφή]
  Έστω $ \theta \in [0,2 \pi) $. Ορίζουμε $ T_{\theta} \colon \mathbb{R}^{2} \to
  \mathbb{R}^{2}  $ με 
  \[
    T_{\theta}(x,y) = (x \cos{\theta} - y \sin{\theta}, x \sin{\theta}, y \cos{\theta}) 
  \] 
  Τότε η $ T_{\theta} $ είναι γραμμικός μετασχηματισμός και γεωμετρικά στρέφει ένα 
  διάνυσμα $ \mathbf{u} \in \mathbb{R}^{2} $ κατά γωνία $\theta$ με αντι-ωρολογιακή 
  φορά.
\end{example}

\begin{rem}[Εξήγηση για τον τύπο της $\theta$]
  Έστω $\theta \in [0, 2 \pi)$, και έστω τα διανύσματα βάσης του $ \mathbb{R}^{2} $ 
  $ \mathbf{e_{1}} = (1,0) $ και $ \mathbf{e_{2}} = (0,1) $. Τότε έχουμε:
  \begin{align*}
    T_{\theta}(\mathbf{e_{1}}) &= T_{\theta}(1,0) = (\cos{\theta}, \sin{\theta}) \\
    T_{\theta}(\mathbf{e_{2}}) &= T_{\theta}(0,1) = (- \sin{\theta} , \cos{\theta})
  \end{align*}
  Έστω, τώρα $ \mathbf{u} = (x,y) \in \mathbb{R}^{2} $. Τότε 
  \begin{align*}
    T_{\theta}(\mathbf{u}) = T_{\theta} (x,y) 
    &=  T_{\theta} (x \mathbf{e_{1}}+ y \mathbf{e_{2}}) \\
    &= T_{\theta }(x \mathbf{e_{1}}) + T_{\theta }(y \mathbf{e_{2}}) \\
    &= x (\cos{\theta}, \sin{\theta}) + y (- \sin{\theta} , \cos{\theta}) \\
    &= (x \cos{\theta} - y \sin{\theta} , x \sin{\theta} + y \cos{\theta}) \\
  \end{align*} 
\end{rem}

\begin{example}
  Έστω ο $ \mathbb{K}- $χώρος $ \textbf{M}_{m \times n}(\mathbb{K}) $. Ορίζουμε 
  $ T \colon \textbf{M}_{m \times n}(\mathbb{K}) \to 
  \textbf{M}_{m \times n}(\mathbb{K}) $ με 
  \[
    T(A) = A^{T}, \quad \forall A \in \textbf{M}_{m \times n}(\mathbb{K})
  \] 
  Τότε η $T$ είναι γραμμικός μετασχηματισμός, αφού
  \begin{align*}
    T(kA+ \lambda B) &= {(kA + \lambda B)}^{T} \\
                     &= {(kA)}^{T} + {(\lambda B)}^{T} \\
                     &= k A^{T} + \lambda B^{T} \\
                     &= k T(A) + \lambda T(B)
  \end{align*} 
\end{example}

\begin{example}
  Έστω ο $ \mathbb{R}- $χώρος $ \mathbb{R}^{2} $ και έστω $ T \colon \mathbb{R}^{2} \to
  \mathbb{R}^{2} $ με 
  \[
    T(x,y) = (\abs{x} , 0) 
  \] 
  Τότε η $T$ δεν είναι γραμμικός μετασχηματισμός, αφού 
  \[
    \left.
      \begin{aligned}
        T((-1)(1,0)) &= T(-1,0) = (\abs{(-1)},0) = (1,0) \\
        (-1) T(1,0) &= (-1) (\abs{1} , 0) = (-1) (1,0) = (-1,0)
      \end{aligned}
    \right\} \Rightarrow Τ((-1)(1,0)) \neq (-1)T(1,0)
  \]
\end{example}

\begin{example}
  Έστω οι $ \mathbb{R} - $χώροι $ \textbf{P}_{n}(\mathbb{R}) $ και $
  \textbf{P}_{n-1}(\mathbb{R}) $ και έστω $ T \colon \textbf{P}_{n}(\mathbb{R}) \to 
  \textbf{P}_{n-1}(\mathbb{R}) $ με 
  $ T(p) = p' $
  Τότε η $T$ είναι γραμμικός μετασχηματισμός, αφού 
  \[
    T(kp + \lambda q) = (kp+ \lambda q)' = kp' + \lambda q' = k T(p) + \lambda T(q)
  \] 
\end{example}

\begin{example}
  Έστω οι $ \mathbb{R} - $χώροι $ C(\mathbb{R}) $ και $ \mathbb{R} $, όπου 
  $ C(\mathbb{R}) $ είναι ο χώρος των συνεχών συναρτήσεων από το $ \mathbb{R} $ στο 
  $ \mathbb{R} $. Έστω $ a,b \in \mathbb{R} $ με $ a<b $. Ορίζουμε 
  $ T \colon C(\mathbb{R}) \to \mathbb{R}$ με 
  \[
    T(f) = \int _{a}^{b} f(x) \,{dx}, \quad \forall f \in C(\mathbb{R}) 
  \] 
  Τότε από τη γραμμική ιδιότητα του ορισμένου ολοκληρώματος, προκύπτει ότι 
  η $T$ είναι γραμμικός μετασχηματισμός.
\end{example}


\section{Πυρήνας και Εικόνα Γραμμικού Μετασχηματισμού}

\begin{dfn}
  Έστω $V$, $K$ δυο $ \mathbb{K}- $χώροι και έστω $ T \colon V \to W $ ένας 
  γραμμικός μετασχηματισμός. Τότε 
  \begin{myitemize}
    \item Το σύνολο 
      $ \ker(T) = \{ \mathbf{x} \in V \; : \; T(\mathbf{x}) = \mathbf{0}_{W} \} $ 
      λέγεται \textcolor{Col1}{πυρήνας} του γραμμικού μετασχηματισμού $T$.
    \item Το σύνολο 
      $ \im(T) = \{ T(\mathbf{x}) \; : \; \mathbf{x} \in V \} $ 
      λέγεται \textcolor{Col1}{εικόνα} του γραμμικού μετασχηματισμού $T$.
  \end{myitemize}
\end{dfn}

\begin{example}
  Έστω $V$ και $W$ δυο $ \mathbb{K}- $χώροι. Τότε για τον ταυτοτικό μετασχηματισμό 
  ισχύει 
  \[
    \ker(I_{V}) = \{ \mathbf{0}_{V} \} \quad \text{και} \quad \im(I_{V}) = V 
  \]
  και για τον μηδενικό μετασχηματισμό ισχύει
  \[
    \ker(T_{0}) = V  \quad \text{και} \quad \im(T_{0}) = \{ \mathbf{0}_{W} \}   
  \] 
\end{example}

\begin{example}
  Έστω οι $ \mathbb{R}- $χώροι $ \mathbb{R}^{3} $ και $ \mathbb{R}^{2} $ και έστω 
  $ T \colon \mathbb{R}^{3} \to \mathbb{R}^{2} $ με $ T(x,y,z) = (x-y,2z) $. Η 
  $T$ είναι γραμμικός μετασχηματισμός και έχουμε:
  \begin{align*}
    \ker(T) = \{(x,y,z)\in \mathbb{R}^{3} \; : \; T(x,y,z) = (0,0) \} \\
    &= \{(x,y,z)\in \mathbb{R}^{3} \; : \; (x-y,2z)=(0,0) \} \\
    &= \{(x,y,z)\in \mathbb{R}^{3} \; : \; x-y=0 \; \text{και} \; 2z=0 \} \\
    &= \{(x,y,z)\in \mathbb{R}^{3} \; : \; x=y \; \text{και} \; z=0 \} \\
    &= \{(x,x,0)\in \mathbb{R}^{3} \; : \; x \in \mathbb{R} \} \\
    &= < (1,1,0) >  
  \end{align*}
  ενώ για την εικόνα της $T$ έχουμε ότι $ \im(T) = \mathbb{R}^{2}  $, αφού αν 
  $ (a,b) \in \mathbb{R}^{2} $, τότε
  \[
    T(2a,a, b/2)= (2a-a,2(b/2)) = (a,b)
  \] 
\end{example}

\begin{thm}
  Έστω $V$ και $W$  δύο  $ \mathbb{K}- $χώροι  και έστω  $ T \colon V \to W $  ένας 
  γραμμικός μετασχηματισμός. Τότε ισχύουν τα παρακάτω:
  \begin{myitemize}
    \item Ο πυρήνας $ \ker(T)  $ του $T$ είναι υπόχωρος του $V$. 
    \item Η εικόνα $ \im(T)  $ του $T$ είναι υπόχωρος του $W$.
  \end{myitemize}
\end{thm}

\begin{proof}

\end{proof}

\begin{thm}
  Έστω $V$ και $W$ δύο $ \mathbb{K}- $χώροι και έστω $ T \colon V \to W $ ένας 
  γραμμικός μετασχηματισμός. Αν $ B = \{ \mathbf{v_{1}}, \ldots, \mathbf{v_{n}} \} $ 
  είναι μια βάση για τον $V$, τότε
  \[
    \im(T) = < T(\mathbf{v_{1}}), \ldots, T(\mathbf{v_{n}}) >  
  \]
  δηλαδή, $ \im(T) = < T(B) > $.
\end{thm}

\begin{proof}

\end{proof}

\begin{example}
  Έστω οι $ \mathbb{R} - $χώροι $ \textbf{P}_{2}(\mathbb{R}) $ και 
  $ \textbf{M}_{2}(\mathbb{R}) $. Ορίζουμε $ T \colon \textbf{P}_{2}(\mathbb{R}) \to
  \textbf{M}_{2}(\mathbb{R}) $ με 
  \[
    T(p) = 
    \begin{pmatrix*}
      p(1)-p(2) & 0 \\
      0 & p(0)
    \end{pmatrix*}, \quad p \in \textbf{P}_{2}(\mathbb{R})
  \] 
  Η $T$ είναι γραμμικός μετασχηματισμός και για να βρούμε βάση και τη διάσταση του 
  $ \im(T) $, θεωρούμε τη βάση $ B = \{ 1,x,x^{2} \} $ του $
  \textbf{P}_{2}(\mathbb{R}) $. Τότε από το προηγούμενο θεώρημα έχουμε
  \begin{align*}
    \im(T) = < T(B) > &= < T(1), T(x), T(x^{2}) > \\
                      &= < 
                      \begin{pmatrix*}[r]
                        0 & 0 \\
                        0 & 1
                      \end{pmatrix*}, 
                      \begin{pmatrix*}[r]
                        -1 & 0 \\
                        0 & 0 
                      \end{pmatrix*}, 
                      \begin{pmatrix*}[r]
                        -3 & 0 \\
                        0 & 0
                      \end{pmatrix*}
                      > \\
                      & = 
                      < 
                      \begin{pmatrix*}[r]
                        0 & 0 \\
                        0 & 1
                      \end{pmatrix*}
                      , 
                      \begin{pmatrix*}[r]
                        -1 & 0 \\
                        0 & 0
                      \end{pmatrix*}
                      >  
  \end{align*}
  γιατί προφανώς $ 
  \begin{pmatrix*}[r]
    -3 & 0 \\
    0 & 0
  \end{pmatrix*} = -3 
  \begin{pmatrix*}[r]
    -1 & 0 \\
    0 & 0 
  \end{pmatrix*}
  $. Οπότε το σύνολο $ <  
  \begin{pmatrix*}[r]
    0 & 0 \\
    0 & 1
  \end{pmatrix*}
  , 
  \begin{pmatrix*}[r]
    -1 & 0 \\
    0 & 0
  \end{pmatrix*}
  > $ 
  παράγει το χώρο $ \im(T) $ και επειδή τα στοιχεία του είναι 
  προφανώς γραμμικώς ανεξάρτητα, αποτελούν βάση του $ \im(T) $.
  Άρα $ \dim \im(T) = 2 $
\end{example}

\begin{rem}
  Το προηγούμενο παράδειγμα δείχνει ότι αν $ B = \{ \mathbf{v_{1}}, \ldots, 
  \mathbf{v_{n}} \} $ είναι μια βάση για έναν $ \mathbb{K}- $χώρο και 
  $ T \colon V \to W $ είναι ένας γραμμικός μετασχηματισμός, τότε δεν ισχύει 
  απαραίτητα ότι το σύνολο $ T(B) = 
  \{ T(\mathbf{v_{1}}), \ldots, T(\mathbf{v_{n}}) \}$ είναι μια βάση για τον $ \im(T) $.
\end{rem}

\begin{prop}
  Το σύνολο $ T(B) $ θα είναι μια βάση για τον $ \im(T) $ αν επιπλέον η $T$ είναι 
  $ 1-1 $.
\end{prop}

\begin{proof}

\end{proof}

\begin{dfn}
  Έστω $V$ και $W$ δύο $ \mathbb{K} - $χώροι και έστω $ T \colon V \to W $ ένας 
  γραμμικός μετασχηματισμός. Αν οι υπόχωροι $ \ker(T) $ και $ \im(T) $ των $V$ και $W$ 
  αντίστοιχα είναι πεπερασμένης διάστασης, τότε ορίζουμε 
  \begin{myitemize}
    \item την \textcolor{Col1}{μηδενικότητα} του $T$ να είναι η διάσταση 
      $ \dim \ker(T) $ του $ \ker(T) $ 
    \item τον \textcolor{Col1}{βαθμό} του $T$ να είναι η διάσταση $ \dim \im(T) $ του 
      $ \im(T) $.
  \end{myitemize}
\end{dfn}

\begin{rem}
\item {}
  \begin{myitemize}
    \item Τη μηδενικότητα ενός γραμμικού μετασχηματισμού $T$, συνήθως τη συμβολίζουμε με 
      $ \Null(T) $ ή απλώς $ \n(T) $. Πολλές φορές θα χρησιμοποιούμε και το 
      $ \dim \ker(T) $.
    \item Το βαθμό ενός γραμμικού μετασχηματισμού $T$, συνήθως τον συμβολίζουμε με 
      $ \rank(T) $ ή απλώς $ \R(T) $.
  \end{myitemize}
\end{rem}

\begin{thm}[Θεώρημα Διάστασης]
  Έστω $ V $ και $W$ δύο $ \mathbb{K}- $χώροι και έστω $ T \colon V \to W $ ένας 
  γραμμικός μετασχηματισμός. Αν ο $V$ είναι πεπερασμένης διάστασης, τότε 
  \[
    \Null(T) + \rank(T) = \dim (V) 
  \]
  ή με άλλα λόγια
  \begin{center}
    μηδενικότητα του $T$ $ + $ βαθμός του $T$ $ = \dim (V) $ 
  \end{center}
\end{thm}

\begin{proof}

\end{proof}

\begin{thm}
  Έστω $ V $ και $W$ δύο $ \mathbb{K}- $χώροι και έστω $ T \colon V \to W $ ένας 
  γραμμικός μετασχηματισμός. Τότε ισχύει ότι 
  \[
    T \quad \text{είναι} \quad 1-1 \Leftrightarrow \ker(T) = \{ \mathbf{0}_{V} \} 
  \] 
\end{thm}

\begin{proof}
\end{proof}

\begin{thm}
  Έστω $ V $ και $W$ δύο $ \mathbb{K}- $χώροι και έστω $ T \colon V \to W $ ένας 
  γραμμικός μετασχηματισμός. Αν $ \dim(V) = \dim(W) = n < \infty $ τότε οι 
  παρακάτω προτάσεις είναι ισοδύναμες.
  \begin{enumerate}[i)]
    \item $T$ είναι $ 1-1 $
    \item $T$ είναι επί
    \item $ \rank(T) = \dim(V) $
  \end{enumerate}
\end{thm}

\begin{proof}

\end{proof}

\begin{rem}
  Άρα από το προηγούμενο θεώρημα συμπεραίνουμε ότι αν $ T $ γραμμικός μετασχηματισμός 
  μεταξύ ίδιων χώρων, τότε αν είναι $ 1-1 $ θα είναι και επί και αντιστρόφως, αν είναι 
  επί θα είναι και $ 1-1 $.
\end{rem}

\begin{thm}
  Έστω $V$ και $W$ δυο $ \mathbb{K}- $χώροι. Υποθέτουμε ότι το σύνολο 
  $ \{ \mathbf{v_{1}}, \ldots, \mathbf{v_{n}}\} $ είναι μια βάση για τον $V$. 
  Για οποιαδήποτε $ \mathbf{w_{1}}, \ldots \mathbf{w_{n}} $ στον $W$ υπάρχει 
  ακριβώς ένας γραμμικός μετασχηματισμός $ T \colon V \to W $ τέτοιος ώστε
  \[
    T(\mathbf{v}_{i}) = \mathbf{w}_{i}, \quad \forall i \in \{ 1,2, \ldots \}
  \] 
\end{thm}

\begin{proof}

\end{proof}

\begin{cor}
  Έστω $V$ και $W$ δυο $ \mathbb{K}- $χώροι. Υποθέτουμε ότι ο $V$ έχει μια πεπερασμένη 
  βάση, έστω την $ B = \{ \mathbf{v_{1}}, \ldots, \mathbf{v_{n}} \} $. Αν 
  $ S,T \colon V \to W$ είναι δυο γραμμικοί μετασχηματισμοί και 
  $ S(\mathbf{v}_{i}) = T(\mathbf{v}_{i}) $ για $ i=1,2,\ldots,n $ τότε $ S=T $.
\end{cor}

\begin{rem}
  Δηλαδή δυο γραμμικοί γραμμικοί μετασχηματισμοί ταυτίζονται αν ταυτίζονται στα στοιχεία
  μιας βάσης.
\end{rem}

\begin{example}
  Έστω $ T \colon \mathbb{R}^{2} \to \mathbb{R}^{2} $ ο γραμμικός μετασχηματισμός με 
  τύπο $ T(a_{1}, a_{2}) = (2 a_{2}- a_{1}, 3 a_{1}) $. 
  Υποθέτουμε ότι $ U \colon \mathbb{R}^{2} \to \mathbb{R}^{2} $ είναι γραμμικός. Αν 
  γνωρίζουμε ότι $ U(1,2) = (3,3) $ και $ U(1,1) = (1,3) $, τότε $ U=T $. 

  Πράγματι, αυτό προκύπτει από το γεγονός ότι $ \{ (1,2), (3,3) \}  $ είναι μια βάση 
  του $ \mathbb{R}^{2} $ και από το ότι $ T(1,2)=(3,3) = U(1,2) $ και 
  $ T(1,1) = (1,3) = U(1,1) $. Τότε σύμφωνα με το πόρισμα, αφού οι δυο μετασχηματισμοί 
  ταυτίζονται στα στοιχεία μιας βάσης, είναι ίσοι.
\end{example}


 \end{document}
