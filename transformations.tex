\input{preamble/preamble.tex}
\newcommand{\vect}[2]{(#1_1,\ldots, #1_#2)}
%%%%%%% nesting newcommands $$$$$$$$$$$$$$$$$$$
\newcommand{\function}[1]{\newcommand{\nvec}[2]{#1(##1_1,\ldots, ##1_##2)}}

\newcommand{\linode}[2]{#1_n(x)#2^{(n)}+#1_{n-1}(x)#2^{(n-1)}+\cdots +#1_0(x)#2=g(x)}

\newcommand{\vecoffun}[3]{#1_0(#2),\ldots ,#1_#3(#2)}




\pagestyle{vangelis}


\begin{document}

\chapter{Γραμμικοί Μετασχηματισμοί}

\begin{dfn}
  Έστω $ (V, \oplus _{V}, \odot _{V}) $ και $ (W, \oplus _{W}, \odot _{W}) $ 
  δύο $ \mathbb{K} - $χώροι. Μια συνάρτηση $ T \colon V \to W $ λέγεται 
  \textcolor{Col1}{γραμμικός μετασχηματισμός} αν ικανοποιεί τις παρακάτω συνθήκες:
  \begin{enumerate}[i)]
    \item $ T(\mathbf{x} \oplus _{V} \mathbf{y}) = T(\mathbf{x}) \oplus _{W}
      T(\mathbf{y}), \quad \forall \mathbf{x}, \mathbf{y}  \in V $
    \item $ T(\lambda \odot _{V} \mathbf{x}) = \lambda \odot _{W} T(\mathbf{x}), 
      \quad \forall \lambda \in \mathbb{K} $ και $ \forall \mathbf{x} \in V $
  \end{enumerate}
  Αν $T$ είναι ένας γραμμικός μετασχηματισμός από τον χώρο $V$ στον $V$ τότε λέμε 
  ότι ο $T$ είναι ένας \textcolor{Col1}{γραμμικός τελεστής} επί του $V$. 
\end{dfn}

\begin{rem}
  Συνήθως, αν δεν υπάρχει σύγχυση, γράφουμε απλώς $ T(\mathbf{x}+ \mathbf{y}) =
  T(\mathbf{x}) + T(\mathbf{y}) $ και 
  $ T(\lambda \mathbf{x}) = \lambda T(\mathbf{x}) $, για τις δυο συνθήκες του ορισμού.
\end{rem}

\begin{prop}
  Έστω $ V $ και $W$ δύο $ \mathbb{K}- $χώροι. Τότε 
  \begin{enumerate}[i)]
    \item Αν $ T \colon V \to W $ είναι ένας γραμμικός μετασχηματισμός, τότε 
      $ T(\mathbf{0}_{V}) = \mathbf{0}_{W} $
    \item $ T \colon V \to W $ γραμμικός μετασχηματισμός $ \Leftrightarrow T(k
      \mathbf{x} + \lambda \mathbf{y}) = = k T(\mathbf{x}) + \lambda T(\mathbf{y}) $
  \end{enumerate}
\end{prop}

\begin{proof}
\item 
  \begin{enumerate}[i)]
    \item 
      $ 
      \left.
        \begin{matrix*}[l]
          T(\mathbf{0}_{V}) = T(\mathbf{0}_{V}+ \mathbf{0}_{V}) = 
          T(\mathbf{0}_{V}) + T(\mathbf{0}_{V})  \\ 
          T(\mathbf{0}_{V}) = T(\mathbf{0}_{V}) + \mathbf{0}_{W}
        \end{matrix*} 
      \right\}
      \Rightarrow \cancel{T(\mathbf{0}_{V})} + T(\mathbf{0}_{V}) =
      \cancel{T(\mathbf{0}_{V})} + \mathbf{0}_{W} \Rightarrow  
      T(\mathbf{0}_{V}) = \mathbf{0}_{W} $
    \item 
  \end{enumerate}
\end{proof}

\begin{rem}
  Αν $ T \colon V \to W $ είναι ένας γραμμικός μετασχηματισμός, τότε για κάθε 
  $ n \in \mathbb{N} $, $ \mathbf{x}_{i} \in V $ και $ \lambda _{i} \in \mathbb{K} $, 
  $ i = 1,\ldots, n $ ισχύει:
  \[
    T\left(\sum_{i=1}^{n} \lambda _{i} \mathbf{x}_{i}\right) = \sum_{i=1}^{n} 
    \lambda _{i} T(\mathbf{x}_{i}) 
  \] 
\end{rem}

\begin{example}[Ταυτοτικός]
  Έστω $V$ ένας $ \mathbb{K} - $χώρος και έστω $ T \colon V \to V $ η ταυτοτική 
  συνάρτηση επί του $V$ με 
  \[
    T(\mathbf{x}) = \mathbf{x}, \quad \mathbf{x} \in V   
  \]
  Τότε η $T$ είναι γραμμικός μετασχηματισμός και λέγεται 
  \textcolor{Col1}{ταυτοτικός} μετασχηματισμός επί του $V$ και συμβολίζεται με $ I_{V} $.
\end{example}

\begin{example}[Μηδενικός]
  Έστω $V$ και $W$ δυο $ \mathbb{K}- $χώροι και έστω $ T \colon V \to W $ με 
  \[
    T(\mathbf{x}) = \mathbf{0}_{W}, \quad \forall \mathbf{x} \in V
  \]
  Τότε η $T$ είναι γραμμικός μετασχηματισμός και λέγεται 
  \textcolor{Col1}{μηδενικός} μετασχηματισμός επί του $V$ και συμβολίζεται με $ T_{0} $.
\end{example}

\begin{example}[Προβολή επί του άξονα $x$]
  Έστω ο $ \mathbb{R}- $χώρος $ \mathbb{R}^{2} $ με τις συνήθεις πράξεις και έστω 
  $ T \colon \mathbb{R}^{2} \to \mathbb{R}^{2} $ με 
  \[
    T(\mathbf{x}, \mathbf{y}) = (x, 0) 
  \] 
  Τότε η $T$ είναι γραμμικός μετασχηματισμός και λέγεται \textcolor{Col1}{προβολή} επί 
  του άξονα $x$.
\end{example}

\begin{example}[Συμμετρία ως προς άξονα $x$]
  Έστω ο $ \mathbb{R}- $χώρος $ \mathbb{R}^{2} $ με τις συνήθεις πράξεις και έστω 
  $ T \colon \mathbb{R}^{2} \to \mathbb{R}^{2} $ με 
  \[
    T(\mathbf{x}, \mathbf{y}) = (x, -y) 
  \] 
  Τότε η $T$ είναι γραμμικός μετασχηματισμός και λέγεται \textcolor{Col1}{συμμετριά} ως 
  προς τον άξονα $x$.
\end{example}

\begin{example}
  Έστω ο $ \mathbb{R}- $χώρος $ \mathbb{R}^{2} $ με τις συνήθεις πράξεις και έστω 
  $ T \colon \mathbb{R}^{2} \to \mathbb{R}^{2} $ με 
  \[
    T(\mathbf{x}, \mathbf{y}) = (x+1, y) 
  \] 
  Τότε η $T$ δεν είναι γραμμικός μετασχηματισμός, αφού $ T(0,0) = (1,0) \neq (0,0) $ 
\end{example}

\begin{example}[Αριστερόστροφη Περιστροφή]
  Έστω $ \theta \in [0,2 \pi) $. Ορίζουμε $ T_{\theta} \colon \mathbb{R}^{2} \to
  \mathbb{R}^{2}  $ με 
  \[
    T_{\theta}(x,y) = (x \cos{\theta} - y \sin{\theta}, x \sin{\theta}, y \cos{\theta}) 
  \] 
  Τότε η $ T_{\theta} $ είναι γραμμικός μετασχηματισμός και γεωμετρικά στρέφει ένα 
  διάνυσμα $ \mathbf{u} \in \mathbb{R}^{2} $ κατά γωνία $\theta$ με αντι-ωρολογιακή 
  φορά.
\end{example}

\begin{rem}[Εξήγηση για τον τύπο της $\theta$]
  Έστω $\theta \in [0, 2 \pi)$, και έστω τα διανύσματα βάσης του $ \mathbb{R}^{2} $ 
  $ \mathbf{e_{1}} = (1,0) $ και $ \mathbf{e_{2}} = (0,1) $. Τότε έχουμε:
  \begin{align*}
    T_{\theta}(\mathbf{e_{1}}) &= T_{\theta}(1,0) = (\cos{\theta}, \sin{\theta}) \\
    T_{\theta}(\mathbf{e_{2}}) &= T_{\theta}(0,1) = (- \sin{\theta} , \cos{\theta})
  \end{align*}
  Έστω, τώρα $ \mathbf{u} = (x,y) \in \mathbb{R}^{2} $. Τότε 
  \begin{align*}
    T_{\theta}(\mathbf{u}) = T_{\theta} (x,y) 
    &=  T_{\theta} (x \mathbf{e_{1}}+ y \mathbf{e_{2}}) \\
    &= T_{\theta }(x \mathbf{e_{1}}) + T_{\theta }(y \mathbf{e_{2}}) \\
    &= x (\cos{\theta}, \sin{\theta}) + y (- \sin{\theta} , \cos{\theta}) \\
    &= (x \cos{\theta} - y \sin{\theta} , x \sin{\theta} + y \cos{\theta}) \\
  \end{align*} 
\end{rem}

\begin{example}
  Έστω ο $ \mathbb{K}- $χώρος $ \textbf{M}_{m \times n}(\mathbb{K}) $. Ορίζουμε 
  $ T \colon \textbf{M}_{m \times n}(\mathbb{K}) \to 
  \textbf{M}_{m \times n}(\mathbb{K}) $ με 
  \[
    T(A) = A^{T}, \quad \forall A \in \textbf{M}_{m \times n}(\mathbb{K})
  \] 
  Τότε η $T$ είναι γραμμικός μετασχηματισμός, αφού
  \begin{align*}
    T(kA+ \lambda B) &= (kA + \lambda B)^{T} \\
                     &= (kA)^{T} + (\lambda B)^{T} \\
                     &= k A^{T} + \lambda B^{T} \\
                     &= k T(A) + \lambda T(B)
  \end{align*} 
\end{example}

\begin{example}
  Έστω ο $ \mathbb{R}- $χώρος $ \mathbb{R}^{2} $ και έστω $ T \colon \mathbb{R}^{2} \to
  \mathbb{R}^{2} $ με 
  \[
    T(x,y) = (\abs{x} , 0) 
  \] 
  Τότε η $T$ δεν είναι γραμμικός μετασχηματισμός, αφού 
  \[
    \left.
      \begin{aligned}
        T((-1)(1,0)) &= T(-1,0) = (\abs{(-1)},0) = (1,0) \\
        (-1) T(1,0) &= (-1) (\abs{1} , 0) = (-1) (1,0) = (-1,0)
      \end{aligned}
    \right\} \Rightarrow Τ((-1)(1,0)) \neq (-1)T(1,0)
  \]
\end{example}

\begin{example}
  Έστω οι $ \mathbb{R} - $χώροι $ \textbf{P}_{n}(\mathbb{R}) $ και $
  \textbf{P}_{n-1}(\mathbb{R}) $ και έστω $ T \colon \textbf{P}_{n}(\mathbb{R}) \to 
  \textbf{P}_{n-1}(\mathbb{R}) $ με 
  $ T(p) = p' $
  Τότε η $T$ είναι γραμμικός μετασχηματισμός, αφού 
  \[
    T(kp + \lambda q) = (kp+ \lambda q)' = kp' + \lambda q' = k T(p) + \lambda T(q)
  \] 
\end{example}

\begin{example}
  Έστω οι $ \mathbb{R} - $χώροι $ C(\mathbb{R}) $ και $ \mathbb{R} $, όπου 
  $ C(\mathbb{R}) $ είναι ο χώρος των συνεχών συναρτήσεων από το $ \mathbb{R} $ στο 
  $ \mathbb{R} $. Έστω $ a,b \in \mathbb{R} $ με $ a<b $. Ορίζουμε 
  $ T \colon C(\mathbb{R}) \to \mathbb{R}$ με 
  \[
    T(f) = \int _{a}^{b} f(x) \,{dx}, \quad \forall f \in C(\mathbb{R}) 
  \] 
  Τότε από τη γραμμική ιδιότητα του ορισμένου ολοκληρώματος, προκύπτει ότι 
  η $T$ είναι γραμμικός μετασχηματισμός.
\end{example}

 \end{document}
