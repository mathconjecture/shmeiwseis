\input{preamble/preamble.tex}
\newcommand{\vect}[2]{(#1_1,\ldots, #1_#2)}
%%%%%%% nesting newcommands $$$$$$$$$$$$$$$$$$$
\newcommand{\function}[1]{\newcommand{\nvec}[2]{#1(##1_1,\ldots, ##1_##2)}}

\newcommand{\linode}[2]{#1_n(x)#2^{(n)}+#1_{n-1}(x)#2^{(n-1)}+\cdots +#1_0(x)#2=g(x)}

\newcommand{\vecoffun}[3]{#1_0(#2),\ldots ,#1_#3(#2)}







\begin{document}

\chapter{Γραμμικοί Μετασχηματισμοί}

\begin{dfn}
    Έστω $ (V, \oplus _{v}, \odot _{v}) $ και $ (W, \oplus _{w}, \odot _{w}) $ 
    δύο $ \mathbb{K} - $χώροι. Μια συνάρτηση $ T \colon V \to W $ λέγεται 
    \textcolor{Col1}{γραμμικός μετασχηματισμός} αν ικανοποιεί τις παρακάτω συνθήκες:
    \begin{enumerate}[i)]
        \item $ T(x \oplus _{V} y) = T(x) \oplus _{W} T(y), \quad \forall x,y \in V $
        \item $ T(\lambda \odot _{V} x) = \lambda \odot _{W} T(x), \quad \forall \lambda 
            \in \mathbb{K} $ και $ \forall x \in V $
    \end{enumerate}
    Αν $T$ είναι ένας γραμμικός μετασχηματισμός από τον χώρο $V$ στον $V$ τότε λέμε 
    ότι ο $T$ είναι ένας \textcolor{Col1}{γραμμικός τελεστής} επί του $V$. 
\end{dfn}

\begin{rem}
    Συνήθως, αν δεν υπάρχει σύγχυση, γράφουμε απλώς $ T(x+y) = T(x) + T(y) $ και 
    $ T(\lambda x) = \lambda T(x) $, για τις δυο συνθήκες του ορισμού.
\end{rem}

\begin{rem}
    Έστω $ V $ και $W$ δύο $ \mathbb{K}- $χώροι 
\end{rem}

\end{document}
