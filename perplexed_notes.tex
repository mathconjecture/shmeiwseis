\documentclass[a4paper,12pt]{article}
\usepackage{etex}
%%%%%%%%%%%%%%%%%%%%%%%%%%%%%%%%%%%%%%
% Babel language package
\usepackage[english,greek]{babel}
% Inputenc font encoding
\usepackage[utf8]{inputenc}
%%%%%%%%%%%%%%%%%%%%%%%%%%%%%%%%%%%%%%

%%%%% math packages %%%%%%%%%%%%%%%%%%
\usepackage{amsmath}
\usepackage{amssymb}
\usepackage{amsfonts}
\usepackage{amsthm}
\usepackage{proof}

\usepackage{physics}

%%%%%%% symbols packages %%%%%%%%%%%%%%
\usepackage{dsfont}
\usepackage{stmaryrd}
%%%%%%%%%%%%%%%%%%%%%%%%%%%%%%%%%%%%%%%


%%%%%% graphicx %%%%%%%%%%%%%%%%%%%%%%%
\usepackage{graphicx}
\usepackage{color}
%\usepackage{xypic}
\usepackage[all]{xy}
\usepackage{calc}
%%%%%%%%%%%%%%%%%%%%%%%%%%%%%%%%%%%%%%%

\usepackage{enumerate}

\usepackage{fancyhdr}
%%%%% header and footer rule %%%%%%%%%
\setlength{\headheight}{14pt}
\renewcommand{\headrulewidth}{0pt}
\renewcommand{\footrulewidth}{0pt}
\fancypagestyle{plain}{\fancyhf{}
\fancyhead{}
\lfoot{}
\rfoot{\small \thepage}}
\fancypagestyle{vangelis}{\fancyhf{}
\rhead{\small \leftmark}
\lhead{\small }
\lfoot{}
\rfoot{\small \thepage}}
%%%%%%%%%%%%%%%%%%%%%%%%%%%%%%%%%%%%%%%

\usepackage{hyperref}
\usepackage{url}
%%%%%%% hyperref settings %%%%%%%%%%%%
\hypersetup{pdfpagemode=UseOutlines,hidelinks,
bookmarksopen=true,
pdfdisplaydoctitle=true,
pdfstartview=Fit,
unicode=true,
pdfpagelayout=OneColumn,
}
%%%%%%%%%%%%%%%%%%%%%%%%%%%%%%%%%%%%%%



\usepackage{geometry}
\geometry{left=25.63mm,right=25.63mm,top=36.25mm,bottom=36.25mm,footskip=24.16mm,headsep=24.16mm}

%\usepackage[explicit]{titlesec}
%%%%%% titlesec settings %%%%%%%%%%%%%
%\titleformat{\chapter}[block]{\LARGE\sc\bfseries}{\thechapter.}{1ex}{#1}
%\titlespacing*{\chapter}{0cm}{0cm}{36pt}[0ex]
%\titleformat{\section}[block]{\Large\bfseries}{\thesection.}{1ex}{#1}
%\titlespacing*{\section}{0cm}{34.56pt}{17.28pt}[0ex]
%\titleformat{\subsection}[block]{\large\bfseries{\thesubsection.}{1ex}{#1}
%\titlespacing*{\subsection}{0pt}{28.80pt}{14.40pt}[0ex]
%%%%%%%%%%%%%%%%%%%%%%%%%%%%%%%%%%%%%%

%%%%%%%%% My Theorems %%%%%%%%%%%%%%%%%%
\newtheorem{thm}{Θεώρημα}[section]
\newtheorem{cor}[thm]{Πόρισμα}
\newtheorem{lem}[thm]{λήμμα}
\theoremstyle{definition}
\newtheorem{dfn}{Ορισμός}[section]
\newtheorem{dfns}[dfn]{Ορισμοί}
\theoremstyle{remark}
\newtheorem{remark}{Παρατήρηση}[section]
\newtheorem{remarks}[remark]{Παρατηρήσεις}
%%%%%%%%%%%%%%%%%%%%%%%%%%%%%%%%%%%%%%%




\newcommand{\vect}[2]{(#1_1,\ldots, #1_#2)}
%%%%%%% nesting newcommands $$$$$$$$$$$$$$$$$$$
\newcommand{\function}[1]{\newcommand{\nvec}[2]{#1(##1_1,\ldots, ##1_##2)}}

\newcommand{\linode}[2]{#1_n(x)#2^{(n)}+#1_{n-1}(x)#2^{(n-1)}+\cdots +#1_0(x)#2=g(x)}

\newcommand{\vecoffun}[3]{#1_0(#2),\ldots ,#1_#3(#2)}



% \usepackage{qtree}
%\usepackage{booktabs}

%\renewcommand{\baselinestretch}{1.2}

\everymath{\displaystyle}

\begin{document}

\chapter{Πεπλεγμένες Συναρτήσεις}


\section{Θεωρήματα Πεπλεγμένων συναρτήσεων}

\vspace{\baselineskip}

\subsection{Η εξίσωση $ F(x,y) = 0 $}

Έστω $ F(x,y) = 0 $, όπου $ F\colon D \to \mathbb{R} $ μια συνάρτηση με πεδίο
ορισμού ένα ανοιχτό υποσύνολο $D$ του $\mathbb{R}^{2}$ και $ (x_0,y_0) $ ένα εσωτερικό σημείο του $D$.
Αν 

\begin{enumerate}[(i)]
	\item $F(x_0,y_0) = 0$ 
	\item $ F_x, F_y$ συνεχείς σε περιοχή του σημείου $ (x_0,y_0) $ 
	\item $ F_y(x_0,y_0) \neq 0 $
\end{enumerate}
τότε υπάρχει μοναδική συνάρτηση $ y=y(x) $ ορισμένη στο $ I_0 \subseteq \mathbb{R} $  τέτοια ώστε:
\begin{enumerate}[(i)]
	\item $y_0 = y(x_0)$
	\item $F(x,y(x)) = 0 \quad \forall x \in I_0$
	\item $ \dv{y}{x} = - \frac{F_x}{F_y} \quad \forall x \in I_0  $
\end{enumerate}

\begin{rem}
	Ο παραπάνω τύπος για την παράγωγο $ \dv{y}{x} $ προκύπτει ως λύση της εξίσωσης
	\[
	\pdv{F}{x} + \pdv{F}{y}\dv{y}{x} = 0 
	\] 
	η οποία προκύπτει με παραγώγιση της $ F(x,y) = 0$, αν θεωρήσουμε ότι $ y=y(x) $.
	
\end{rem}

\subsection{Η εξίσωση $ F(x,y,z) = 0 $}


Έστω $ F(x,y,z) = 0 $, όπου $F\colon D \to \mathbb{R}$ μια συνάρτηση με πεδίο ορισμού ένα ανοικτό
υποσύνολο $ D $ του $ \mathbb{R}^{3}  $ και $ (x_0,y_0,z_0) $ ένα εσωτερικό σημείο του $ D $. Αν
\begin{enumerate}[(i)]
	\item $ F(x_0,y_0,z_0) $
	\item $ F_x, F_y, F_z $ συνεχείς σε περιοχή του σημείου $ (x_0,y_0,z_0) $
	\item $ F_z(x_0,y_0,z_0) \neq 0 $
\end{enumerate}
τότε υπάρχει μοναδική συνάρτηση, $ z=z(x,y) $ ορισμένη στο $ D_0 \subseteq \mathbb{R}^{2} $ τέτοια
ώστε:
\begin{enumerate}[(i)]
	\item $ z_0 = z(x_0,y_0) $
	\item $ F(x,y,z(x,y)) = 0  \quad \forall (x,y)\in  D_0 $
	\item $ \pdv{z}{x} = - \frac{F_x}{F_z} $, $ \pdv{z}{y} = - \frac{F_y}{Fz}\quad \forall (x,y) \in
		D_0$
\end{enumerate}

\begin{rem}
	Οι παραπάνω τύποι για τις παραγώγους $ \pdv{z}{x}, \pdv{z}{y} $ προκύπτουν ως λύσεις των 
	εξισώσεων  
	\begin{align*}	
	\pdv{F}{x} + \pdv{F}{z}\pdv{z}{x} &= 0 \\
	\pdv{F}{y} + \pdv{F}{z}\pdv{z}{x} &= 0 
\end{align*}
	οι οποίες προκύπτουν με παραγώγιση της $ F(x,y,z) = 0 $, ως προς $x$ και $y$ αντίστοιχα και  αν θεωρήσουμε ότι $ z=z(x,y) $.
	
\end{rem}

\subsection{Σύστημα Εξισώσεων}

\subsubsection{1η περίπτωση}

Έστω το σύστημα εξισώσεων $(\Sigma):
	\begin{cases}
		F(x,y,z) = 0  \\
		G(x,y,z) = 0
	\end{cases}$
	και έστω το σημείο $ P_0(x_0,y_0,z_0) $, όπου ισχύει:
	\begin{enumerate}[(i)]
		\item  \begin{tabular}{l}
				$F(x_0,y_0,z_0) = 0$ \\
				$G(x_0,y_0,z_0) = 0$
			\end{tabular}
		\item Οι συναρτήσεις $ F, G $ είναι $ C^{1} $ τάξης 
		\item $ \left.\pdv{(F,G)}{(y,z)}\right|_{P_0} \neq 0 $ 
	\end{enumerate}
	τότε υπάρχουν μοναδικές συναρτήσεις $ y = y(x) $ και $ z = z(x) $ τάξης $ C^{1} $ τέτοιες
	ώστε:
	\begin{enumerate}[(i)]
		\item \begin{tabular}{l}
				$ y_0 = y(x_0) $ \\
				$ z_0 = z(x_0) $
			\end{tabular}
		\item \begin{tabular}{l}
				$ F(x,y(x),z(x)) = 0 $ \\
				$ G(x,y(x),z(x)) = 0 $
		\end{tabular}
	\item $ \dv{y}{x} = - \frac{\pdv{(F,G)}{(x,z)}}{\pdv{(F,G)}{(y,z)}} $, $ \dv{z}{x} = -
		\frac{\pdv{(F,G)}{(y,x)}}{\pdv{(F,G)}{(y,z)}} $
	\end{enumerate}

\begin{rem}
	Οι παραπάνω τύποι για τις τιμές των μερικών παραγώγων $ \dv{y}{x}, \dv{z}{x}$ προκύπτουν ως
	λύσεις του ακόλουθου $ 2 \times 2 $ συστήματος εξισώσεων οι οποίες προκύπτουν με παραγώγιση των
	εξισώσεων του συστήματος $ (\Sigma) $, ως προς $x$.
	\renewcommand{\arraystretch}{2}
	\[
	\begin{aligned}
		(\Sigma_x): \left\{\begin{tabular}{l}
	$\pdv{F}{x} + \pdv{F}{y}\dv{y}{x} + \pdv{F}{z}\dv{z}{x} = 0$ \\
	$\pdv{G}{x} + \pdv{G}{y}\dv{y}{x} + \pdv{G}{z}\dv{z}{x} = 0$
	\end{tabular}
	\right.
\end{aligned}
	\]
\end{rem}

\subsubsection{2η περίπτωση}

Έστω το σύστημα εξισώσεων $(\Sigma):
	\begin{cases}
		F(x,y,z,w) = 0  \\
		G(x,y,z,w) = 0
	\end{cases}$
	και έστω το σημείο $ P_0(x_0,y_0,z_0,w_0) $, όπου ισχύει:
	\begin{enumerate}[(i)]
		\item  \begin{tabular}{l}
				$F(x_0,y_0,z_0,w_0) = 0$ \\
				$G(x_0,y_0,z_0,w_0) = 0$
			\end{tabular}
		\item Οι συναρτήσεις $ F, G $ είναι $ C^{1} $ τάξης 
		\item $ \left.\pdv{(F,G)}{(z,w)}\right|_{P_0} \neq 0 $ 
	\end{enumerate}
	τότε υπάρχουν μοναδικές συναρτήσεις $ z = z(x,y) $ και $ w = w(x,y) $ τάξης $ C^{1} $ τέτοιες
	ώστε:
	\begin{enumerate}[(i)]
		\item \begin{tabular}{l}
				$ z_0 = z(x_0,y_0) $ \\
				$ w_0 = w(x_0,y_0) $
			\end{tabular}
		\item \begin{tabular}{l}
				$ F(x,y,z(x,y), w(x,y) = 0 $ \\
			$ G(x,y,z(x,y), w(x,y)) = 0 $
		\end{tabular}
	\item $ \pdv{z}{x} = - \frac{\pdv{(F,G)}{(x,w)}}{\pdv{(F,G)}{(x,w)}} $, $ \pdv{z}{y} = -
		\frac{\pdv{(F,G)}{(y,w)}}{\pdv{(F,G)}{(z,w)}} $, $ \pdv{w}{x} = -
		\frac{\pdv{(F,G)}{(z,x)}}{\pdv{(F,G)}{(z,w)}} $, $ \pdv{w}{y} = -
		\frac{\pdv{(F,G)}{(z,y)}}{\pdv{(F,G)}{(z,w)}} $
	\end{enumerate}

\begin{rem}
	Οι παραπάνω τύποι για τις τιμές των μερικών παραγώγων $ \pdv{z}{x}, \pdv{z}{y}, \pdv{w}{x},
	\pdv{w}{y} $ προκύπτουν ως λύσεις των ακόλουθων $ 2 \times 2 $ γραμμικών συστημάτων τα οποία προκύπτουν με παραγώγιση των εξισώσεων του συστήματος $ (\Sigma) $, ως προς $x$ και $y$
αντίαντίστοιχα.
	\renewcommand{\arraystretch}{2}
	\[
	\begin{aligned}
		(\Sigma_x): \left\{\begin{tabular}{l}
	$\pdv{F}{x} + \pdv{F}{z}\pdv{z}{x} + \pdv{F}{w}\pdv{w}{x} = 0$ \\
	$\pdv{G}{x} + \pdv{G}{z}\pdv{z}{x} + \pdv{G}{w}\pdv{w}{x} = 0$
	\end{tabular}
	\right.
&\quad \text{και} \quad&
		(\Sigma_y): \left\{\begin{tabular}{l}
	$\pdv{F}{y} + \pdv{F}{z}\pdv{z}{y} + \pdv{F}{w}\pdv{w}{y} = 0$ \\
	$\pdv{G}{y} + \pdv{G}{z}\pdv{z}{y} + \pdv{G}{w}\pdv{w}{y} = 0$
	\end{tabular}
	\right.
\end{aligned}
	\]

\end{rem}

     \section{Ιακωβιανές Ορίζουσες}

     \subsection{Ορισμός}

     Έστω $ \begin{cases} f^{1}=f^{1}(x_{1},\ldots,x_{n}) \\
     f^{2}=f^{2}(x_{1},\ldots,x_{n}) \\
 \vdots \\
 f^{n} = f^{n}(x_{1}\ldots,x_{n}) 
 \end{cases} $, τότε η Ιακωβιανή ορίζουσα, είναι $ J = \pdv{(f^{1},\ldots,f^{n})}{(x_{1}\ldots,x_{n})} = \begin{vmatrix}
 f^{1}_{x_{1}} & f^{1}_{x_{2}} & \cdots & f^{1}_{x_{n}} \\
 f^{2}_{x_{1}} & f^{2}_{x_{2}} & \cdots & f^{2}_{x_{n}} \\
 \vdots & \vdots & \cdots & \vdots \\
 f^{n}_{x_{1}} & f^{n}_{x_{2}} & \cdots & f^{n}_{x_{n}} \\
 \end{vmatrix}$

 Η κύρια χρησιμότητά τους είναι στην εύρεση των μερικών 
 παραγώγων πεπλεγμένων συναρτήσεων, όπως εξηγείται στον 
 παρακάτω γενικό κανόνα.

\subsection{Γενικός Κανόνας}

Όταν ζητάμε την μερική Παράγωγο μιας εξαρτημένης μεταβλητής 
$ (\text{Ε.Μ.$^{*}$}) $, ως προς κάποια ανεξάρτητη 
μεταβλητή $ (\text{Α.Μ.$^{*}$}) $, τότε αυτή είναι 
ίση με μειον το πηλίκο της Ιακωβιανής ορίζουσας 
των Πεπλεγμένων συναρτήσων 
ως προς τις εξαρτημένες μεταβλητές όπου όμως έχουμε αντικαταστήσει την εξαρτημένη μεταβλητή 
$ (\text{Ε.Μ.$^{*}$}) $ με την ανεξάρτητη μεταβλητή 
$ (\text{Α.Μ.$^{*}$}) $ προς την Ιακωβιανή ορίζουσα των 
Πεπλεγμένων συναρτήσεων ως προς τις εξαρτημένες μεταβλητές.

\[
    \pdv{(\text{Ε.Μ.}^{*})}{(\text{Α.Μ.}^{*})} = - 
    \frac{J \; (\text{Πεπλ. ως προς E.M.}^{*} 
    \to A.M.^{*})}{J \; (\text{Πεπλεγμένων ως προς E.M.)}} 
 \] 

 \begin{example}
 \item {}
     \begin{enumerate}
         \item Έστω το σύστημα
             $ \begin{cases}
                 F(u,v,w,x,y)  = 0 \\
                 G(u,v,w,x,y)  = 0 \\
                 H(u,v,w,x,y)  = 0
             \end{cases} $. Τότε έχουμε Ε.Μ.:3 (όσες και οι 
             εξισώσεις) και Α.Μ.:2 (οι υπόλοιπες). 
             Οπότε:
             \[
             \left.\pdv{u}{x}\right)_{y} = - \frac{\pdv{(F,G,H)}{(x,v,w)}}{\pdv{(F,G,H)}{(u,v,w)}}  \quad
             \text{και} \quad \left.\pdv{w}{y}\right)_{x} = - \frac{\pdv{(F,G,H)}{(u,v,y)}}{\pdv{(F,G,H)}{(u,v,w)}} 
              \] 
              και
              \[
                  \left. \pdv{v}{y} \right)_{x} = - 
                  \frac{\pdv{(F,G,H)}{(u,y,w)}}{\pdv{(F,G,H)}{(u,v,w)}} \quad \text{και} \quad \left.
                  \pdv{x}{v} \right)_{w} = - 
                  \frac{\pdv{(F,G,H)}{(u,v,y)}}{\pdv{(F,G,H)}{(u,x,y)}} \quad \text{και} \quad \left.
                  \pdv{w}{u} \right)_{y} = - 
                  \frac{\pdv{(F,G,H)}{(v,u,x)}}{\pdv{(F,G,H)}{ (v,w,x)}} 
               \] 
     \end{enumerate}

 \subsection{Θεωρήματα για Ιακωβιανές Ορίζουσες}
 
 \begin{enumerate}
     \item Μια ικανή και αναγκαία συνθήκη ώστε το σύστημα, 
        \[
             \begin{cases}
                 F(u,v,x,y,z) = 0 \\
                 G(u,v,x,y,z) = 0
             \end{cases}
         \]
         να μπορεί να λυθεί, για παράδειγμα, ως προς 
         $u$ και $v$, είναι η Ιακωβιανή Ορίζουσα
         \[
             J = \pdv{(F,G)}{(u,v)} \neq 0 
          \] 

      \item Ο κανόνας που χρησιμοποιούμε για την εύρεση 
          Ιακωβιανών οριζουσών σύνθετων συναρτήσεων είναι 
          ίδιος με τον κανόνα αλυσίδας μερικών παραγώγων.

          Έστω 
          \[
              \begin{cases} x=x(u,v) \\
              y=y(u,v)\end{cases} \quad \text{και} \quad 
              \begin{cases} u = u(r,s) \\
              v=v(r,s) \end{cases} 
           \] 
           Τότε
           \[
               J = \pdv{(x,y)}{(r,s)} = \pdv{(x,y)}{(u,v)} \cdot \pdv{(u,v)}{(r,s)}  
            \] 
        \item Αν $ \begin{cases} u=u(x,y) \\ v=v(x,y) \end{cases} $ τότε μια ικανή και αναγκαία
            συνθήκη ώστε να υπάρχει μια συναρτησιακή σχέση μεταξύ των $ u $ και $v $ είναι η
            ικανωβιανή ορίζουσα $ J = \pdv{(u,v)}{(x,y)} = 0 $. Δηλαδή:
            \[
                F(u,v)=0 \Leftrightarrow J = \pdv{(u,v)}{(x,y)} = 0 
             \] 
 \end{enumerate}
 \end{example}

\end{document}
