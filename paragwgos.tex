\input{preamble_ask.tex}
\input{definitions_ask.tex}
\input{tikz.tex}
\input{myboxes.tex}

\geometry{top=2.0cm,left=1.5cm,right=1.5cm}

\input{insbox}
\usepackage[font={color=Col1},labelfont={bf},hypcap=false]{caption}


\everymath{\displaystyle}
\pagestyle{vangelis}


\tikzset{tangent/.style=
  {decoration=
    {markings, mark=at position #1 with
      {\coordinate (tangent point-\pgfkeysvalueof{/pgf/decoration/mark info/sequence number}) at (0pt,0pt);
        \coordinate (tangent unit vector-\pgfkeysvalueof{/pgf/decoration/mark info/sequence number}) at (1,0pt);
        \coordinate (tangent orthogonal unit vector-\pgfkeysvalueof{/pgf/decoration/mark info/sequence number}) at (0pt,1);
      }
    },
  postaction=decorate},
  use tangent/.style={shift=(tangent point-#1),
    x=(tangent unit vector-#1),
  y=(tangent orthogonal unit vector-#1)},
use tangent/.default=1}


\newcommand{\twocolumnsiderr}[2]{\begin{minipage}[c]{0.30\linewidth}
    #1
    \end{minipage}\hfill\begin{minipage}[c]{0.68\linewidth}
    #2
  \end{minipage}
}

\newcommand{\twocolumnsidell}[2]{\begin{minipage}[c]{0.68\linewidth}
    #1
    \end{minipage}\hfill\begin{minipage}[c]{0.30\linewidth}
    #2
  \end{minipage}
}




\begin{document}



\chapter*{Παράγωγος}


\section*{Η Έννοια της Παραγώγου}

Τα προβλήματα. της εύρεσης της εφαπτομένης μιας καμπύλης και του υπολογισμού της 
ταχύτητας ενός κινητού σώματος, ανάγονται στον υπολογισμό του ίδιου τύπου ορίου, 
όπως θα δούμε στη συνέχεια. Αυτό το όριο, ονομάζεται \textbf{παράγωγος} και ερμηνεύεται 
ως ο ρυθμός μεταβολής κάποιου μεγέθους.

\subsection*{Εφαπτομένη Καμπύλης}

Για τον υπολογισμό της εφαπτομένης μιας καμπύλης με εξίσωση $ y=f(x) $, στο 
σημείο $P( x_{0}, f(x_{0})) $, θεωρούμε ένα γειτονικό σημείο $ Q(x,f(x)) $, με 
$ x \neq x_{0} $ και υπολογίζουμε την κλίση $\lambda$ της \textbf{τέμνουσας} $ PQ $
(Σχήμα~\ref{fig:slope_sec})

\twocolumnsiderr{
  \begin{tikzpicture}[scale=0.8]
    \draw[-stealth,axis] (-0.5,0) -- (5,0) node[below] (x) {\smaller$x$} ;
    \draw[-stealth,axis] (0,-0.5) -- (0,3.7) node[left] (y) {\smaller$y$} ;
    \coordinate (o) at (0,0) ;
    \begin{scope}[yshift=-15pt]
      \draw[graph,tangent=0.28,Col1] (0.2,0.0) to[out=80, in=180] (2.2,3.0) coordinate 
        (q) to[out=0,in=-180] node[above=20pt,pos=0.7,xshift=7pt] {$y=f(x)$} (5,0.7) ;
      \fill[use tangent] (0,0) coordinate (p) circle (1.5pt);
      \fill (p) node[above,xshift=-2pt] {$P$} circle (1.5pt);
      \fill (q) node[above] {$Q$} circle (1.5pt);
      \draw[dashed] (o|-p) node[left,Col2] {\smaller$f(x_{0})$} -- (p) -- (o-|p)
        node[below,Col2] {\smaller$x_{0}$} ;
      \draw[dashed] (o|-q) node[left,Col2] {\smaller$f(x)$} -- (q) -- (o-|q) 
        node[below,Col2] {\smaller$x$} ;
      \coordinate (a) at (p-|q) ; 
      \draw[decorate,decoration={brace,amplitude=3pt,raise=3pt,mirror},Col2] (p) -- (a) ;
      \draw (a) -- (q) node[midway,xshift=5pt,yshift=-2pt,pin={[pin
        edge={black,latex-},Col2]75:\smaller$f(x)-f(x_{0})$}]{} ;
      \draw[decorate,decoration={brace,amplitude=3pt,raise=3pt,mirror},Col2] (a) -- (q) ;
      \draw[Col1] pic["$\phi$",fill=Col1!50,angle eccentricity=1.4] {angle={a--p--q}} ;
      \draw (p) -- (q) ;
      \draw (p) -- (a) node[below=3pt,midway,Col2] {\smaller$x- x_{0}$} ;
    \end{scope}
  \end{tikzpicture}
  \captionof{figure}{Κλίση τέμνουσας}
  \label{fig:slope_sec}
}{
  \[
    \lambda _{PQ} = \tan{\phi} = \frac{f(x)-f(x_{0})}{x- x_{0}} 
  \] 
  Στη συνέχεια, αφήνουμε το σημείο $Q$ να πλησιάσει το $P$, κατά μήκος της 
  καμπύλης, επιτρέποντας στο $x$ να πλησιάσει το $ x_{0} $. Αν η κλίση $\lambda_{PQ}$ 
  τείνει σε κάποιον αριθμό, έστω $ \lambda $, τότε ορίζουμε την εφαπτομένη στο σημείο 
  $P$ να είναι η ευθεία που διέρχεται από το $P$ και έχει κλίση $ \lambda $. Με άλλα 
  λόγια, η εφαπτομένη της καμπύλης στο σημείο $P$, είναι το όριο (οριακή θέση) της 
  τέμνουσας $ PQ $, καθώς το $Q$ τείνει στο $P$ (Σχήμα~\ref{fig:lim_sec}).
}

\twocolumnsiderr{
  \begin{tikzpicture}[scale=0.8]
    \draw[-stealth,axis] (-0.5,0) -- (5,0) node[below] (x) {\smaller$x$} ;
    \draw[-stealth,axis] (0,-0.5) -- (0,3.7) node[left] (y) {\smaller$y$} ;
    \coordinate (o) at (0,0) ;
    \begin{scope}[yshift=-15pt]
      \draw[thick,tangent=0.28,Col2] (0.2,0.0) to[out=80, in=180] (2.2,3.0) coordinate 
        (q) to[out=0,in=-180] coordinate[pos=0.23] (q1) coordinate[pos=0.4] (q2) (5,0.7) ;
      \fill[use tangent] (0,0) coordinate (p) circle (1.5pt);
      \fill (p) node[above,xshift=-2pt] {$P$} circle (1.5pt);
      \fill (q) node[above=3pt] {\smaller$Q$} circle (1.5pt);
      \fill (q1) node[above right] {\smaller$Q$} circle (1.5pt);
      \fill (q2) node[above right] {\smaller$Q$} circle (1.5pt);
      \draw[very thick,use tangent,Col1] (-1.8,0) -- (2,0) node[right] {$\varepsilon$} ;
      \draw[shorten <=-1.3cm,blue!50] (p) -- ($ (p)!4cm!(q) $) ;
      \draw[shorten <=-1.3cm,blue!50] (p) -- ($ (p)!4cm!(q1) $) ;
      \draw[shorten <=-1.3cm,blue!50] (p) -- ($ (p)!4cm!(q2) $) ;
      \node  at (q) [above] (f) {} ;
      \node  at (p) [above right=20pt,yshift=4pt] (g) {} ;
      \node  at (q1) [above] (a) {} ;
      \node  at (q1) [right,xshift=1pt] (d) {} ;
      \node at (q) [above right,xshift=3pt,yshift=-1pt] (b) {} ;
      \node  at (q2) [above,xshift=1pt] (c) {} ;
      \node  at (q2) [right] (e) {} ;
      \draw (a.center) edge[-stealth] (b.center) ;
      \draw (c.center) edge[-stealth] (d.center) ;
      \draw (f.center) edge[-stealth] (g.center) ;
      \node at (o|-p) [left,Col2]{\phantom{\smaller$f(x_{0})$}} ;
      \draw[dashed] (p) -- (o-|p) node[below,Col2] {\smaller$x_{0}$} ;
      \draw[dashed] (q2) -- (o-|q2) node[below,Col2] {\smaller$x$} ;
      \node at (o-|q2) [below left=2pt] (r1) {} ;
      \node at (o-|q1) [below left=2pt] (r2) {} ;
      \node at (o-|q)[below left=2pt] (r3) {} ;
      \node at (o-|p)[below right=2pt] (r4) {} ;
      \path[draw,Col2] (r1.center) edge[-stealth] (r2.center) ;
      \path[draw,Col2] (r2.center) edge[-stealth,xshift=-6pt] (r3.center) ;
      \path[draw,Col2] (r3.center) edge[-stealth,xshift=-6pt] (r4.east) ;
    \end{scope}
  \end{tikzpicture}
  \captionof{figure}{Όριο τέμνουσας}
  \label{fig:lim_sec}
}{
  \begin{dfn}
    Η \textcolor{Col1}{εφαπτομένη} της καμπύλης $ y=f(x) $ στο σημείο 
    $ P(x_{0}, f(x_{0})) $, είναι η ευθεία που διέρχεται από το $P$ με κλίση 
    \begin{empheq}[box=\mathboxg]{equation}
      \label{eq:slope}
      \lambda = \lim_{x \to x_{0}} \frac{f(x)-f(x_{0})}{x- x_{0}} 
    \end{empheq}
    υπό την προϋπόθεση ότι το όριο, υπάρχει. Τότε η εξίσωσή της εφαπτομένης, δίνεται 
    από την σχέση:
    \begin{empheq}[box=\mathboxr]{equation*}
      \textcolor{Col1}{\varepsilon} \colon y- f(x_{0}) = \lambda (x- x_{0})
    \end{empheq}
  \end{dfn}
}

\subsection*{Ταχύτητα σώματος}

Έστω, ένα σώμα το οποίο κινείται \textbf{ευθύγραμμα}, σύμφωνα με την εξίσωση 
κίνησης, $ x=f(t) $, όπου $x$ είναι η απομάκρυνση του σώματος από την αρχή του
συστήματος συντεταγμένων, κατά τη χρονική στιγμή $t$. Η συνάρτηση $f$ που περιγράφει 
την κίνηση του σώματος, λέγεται 
\textbf{συνάρτηση θέσης} του σώματος. 

\twocolumnsiderr{
  \begin{tikzpicture}[scale=0.8]
    \draw[-latex] (0,0) -- coordinate[pos=0.5] (a1) coordinate[pos=0.90] (a2) 
    (6.0,0) node [below] {$x$} ;
    \node at (a1) [above,xshift=2pt] {$P$} ;
    \node at (a2) [above] {$Q$} ;
    \fill (0.5,0) node[below] (o) {$0$} circle (1.5pt) ;
    \coordinate (b) at (0.5,-0.8) ;
    \coordinate (b') at (0.5,-1.5) ;
    \coordinate (b1) at ($(a1)-(0,0.8)$) ;
    \coordinate (b2) at ($(a2)-(0,1.5)$) ;
    \fill[Col1] (a1) circle (1.5pt) ;
    \fill[Col1] (a2) circle (1.5pt) ;
    \draw[decorate,decoration={brace,amplitude=4pt,raise=5pt,mirror},Col2] (a1) -- (a2) ;
    \draw (a1) -- (a2) node[below=10pt,midway,Col2] {\smaller$f(t)- f(t_{0})$} ;
    \draw (a1) node[pin={[pin edge={black,latex-},Col2,align=center]100:\small{θέση τη
      χρον.} \\ \small{στιγμή $ t_{0} $}}]{} ;
    \draw (a2) node[pin={[pin edge={black,latex-},Col2,align=center]100:\small{θέση τη
      χρον.} \\ \small{στιγμή $ t $}}]{} ;
    \draw[very thick,Col1] (a1) -- (a2) ;
    \draw[{Stealth}-{Stealth}] (b) node {$|$} -- node[fill=white]{\smaller$f(t_{0})$} (b1)
      node {$|$} ;
    \draw[{Stealth}-{Stealth}] (b') node {$|$} -- node[fill=white]{\smaller$f(t)$} (b2) 
      node {$|$} ;
    % \draw[||{Stealth}-{Stealth}||] (0, 1) -- (1, 1);    %% || will produce thicker pipe
  \end{tikzpicture}
}{
  \vspace*{0.5\baselineskip}
  Κατά τη χρονική διάρκεια από $ t_{0} $ έως $ t $, 
  η μεταβολή της θέσης του σώματος (μετατόπιση), από τη θέση $ P $ στη θέση $ Q $,  
  είναι $ f(t)-f(t_{0}) $ και άρα η μέση ταχύτητα του σώματος είναι
  \[
    \text{μέση ταχύτητα} = \frac{\text{μετατόπιση}}{\text{χρόνος}} = 
    \frac{f(t)-f(t_{0})}{t- t_{0}}  
  \] 
  που είναι ίση με την κλίση της \textbf{τέμνουσας} $ PQ $ (Σχήμα~\ref{fig:speed}).
}

\twocolumnsiderr{
  \begin{tikzpicture}[scale=0.8]
    \draw[-stealth,axis] (-0.5,0) -- (5,0) node[below] (x) {\smaller$t$} ;
    \draw[-stealth,axis] (0,-0.5) -- (0,4) node[left] (y) {\smaller$x$} ;
    \coordinate (o) at (0,0) ;
    \node (0) at (0.25, 0.25) {};
    \node (1) at (0.75, 1.25) {};
    \node (2) at (1.25, 0.75) {};
    \node (p) at (1.95, 2.175) {};
    \node (q) at (3.175, 3.25) {};
    \node (5) at (3.575, 3.225) {};
    \draw [thick,Col2,in=180, out=90, looseness=0.75] (0.center) to (1.center);
    \draw [thick,Col2,in=-180, out=0] (1.center) to (2.center);
    \draw [thick,Col2,in=-120, out=0, looseness=0.75] (2.center) to (p.center);
    \draw [thick,Col2,bend left,tangent=0.15] (p.center) to (q.center);
    \draw [thick,Col2,in=165, out=0, looseness=0.75] (q.center) to (5.center);
    \draw[very thick,use tangent,Col1] (-1.1,0) -- (1.2,0) ;
    \fill[use tangent] (0,0) coordinate (p) circle (1.5pt);
    \fill (p) node[above left,xshift=3pt] {$P$} circle (1.5pt);
    \fill (q) node[above] {$Q$} circle (1.5pt);
    \draw[dashed,Col2] (o) -- (o|-p) node[left] {\smaller$f(t_{0})$} -- (p) -- (o-|p)
      node[below,Col2] {\smaller$t_{0}$} ;
    \draw (p.center) -- (q.center) ;
    \draw[dashed,Col2] (o) -- (o|-q) node[left]{\smaller$f(t)$} -- (q) -- (o-|q) 
      node[below,Col2] {\smaller$t$} ;
    \coordinate (a) at (p-|q) ; 
    \draw (p) -- (a) node[below=3pt,midway,Col2] {\smaller$t- t_{0}$} ;
    \draw[decorate,decoration={brace,amplitude=3pt,raise=3pt,mirror},Col2] (p) -- (a) ;
    \draw (a) -- (q.center) node[midway,xshift=5pt,pin={[pin edge={black,latex-},Col2]-85:\smaller$\!\!\!\!f(t)\!-\!f(t_{0})$}]{} ;
    \draw[decorate,decoration={brace,amplitude=3pt,raise=3pt,mirror},Col2] (a) -- (q) ;
    \node[Col2,xshift=-2pt] at (1,2) {$x=f(t)$} ;
  \end{tikzpicture}
  \captionof{figure}{ταχύτητα σώματος}
  \label{fig:speed}
}{
  Στη συνέχεια, υποθέτουμε ότι υπολογίζουμε τη μέση ταχύτητα του σώματος, σε όλο και 
  μικρότερα χρονικά διαστήματα. Με άλλα λόγια, επιτρέπουμε στο $t$ να πλησιάσει το 
  $ t_{0} $, και ορίζουμε την \textcolor{Col1}{ταχύτητα} (ή στιγμιαία ταχύτητα) του 
  σώματος, τη χρονική στιγμή $ t_{0} $ να είναι το \textbf{όριο} της μέσης ταχύτητας
  \begin{empheq}[box=\mathboxg]{equation}
    \label{eq:speed}
    v(t_{0}) = \lim_{t \to t_{0}} \frac{f(t)-f(t_{0})}{t- t_{0}}  
  \end{empheq}
  Αυτό σημαίνει, ότι η ταχύτητα του σώματος, τη χρονική στιγμή $ t_{0} $ είναι ίση με 
  την \textbf{κλίση της εφαπτομένης} στο σημείο $P$, όπως φαίνεται από τις 
  εξισώσεις~\eqref{eq:slope} και~\eqref{eq:speed}.
}


\section*{Ορισμός Παραγώγου}

Είδαμε, ότι ο ίδιος τύπος ορίου, εμφανίζεται κατά τον υπολογισμό της εφαπτομένης 
καμπύλης και της ταχύτητας ενός σώματος. Ακριβέστερα, όρια της μορφής 
\[
  \lim_{x \to x_{0}} \frac{f(x)-f(x_{0})}{x- x_{0}} 
\] 
εμφανίζονται, όπως θα δούμε, κατά τον υπολογισμό του \textcolor{Col1}{ρυθμού μεταβολής} 
κάποιου μεγέθους.

\begin{dfn}
  Η \textcolor{Col1}{παράγωγος} μιας συνάρτησης $f(x)$ στο σημείο $ x_{0} $, 
  συμβολίζεται με $ f'(x_{0}) $ και είναι ίση με 
  \begin{empheq}[box=\mathboxr]{equation*}
    f'(x_{0}) = \lim_{x \to x_{0}} \frac{f(x)-f(x_{0})}{x- x_{0}}
  \end{empheq}
  οποτεδήποτε αυτό το όριο υπάρχει. Τότε λέμε ότι η συνάρτηση $f$ είναι
  \textcolor{Col1}{παραγωγίσιμη} στο $ x_{0} $.
\end{dfn}

\begin{rem}
  Αν θέσουμε $ x- x_{0}=h $ τότε έχουμε $ x = x_{0}+h $ και $ h \to 0 $ καθώς $ x \to
  x_{0} $, οπότε με αντικατάσταση, το παραπάνω όριο, γράφεται στην ισοδύναμη μορφή
  \begin{empheq}[box=\mathboxr]{equation*}
    f'(x_{0}) = \lim_{h \to 0} \frac{f(x_{0}+h)-f(x_{0})}{h}
  \end{empheq}
\end{rem}


\subsection*{Ρυθμός Μεταβολής}

Έστω $y$ κάποια ποσότητα η οποία εξαρτάται από κάποια άλλη ποσότητα $x$. 
Επομένως το $y$ είναι συνάρτηση του $x$ και γράφουμε $ y=f(x) $.

\twocolumnsiderr{
\begin{tikzpicture}[scale=0.8]
    \draw[-stealth,axis] (-0.5,0) -- (4.5,0) node[below] (x) {\smaller$x$} ;
    \draw[-stealth,axis] (0,-0.5) -- (0,4) node[left] (y) {\smaller$y$} ;
    \coordinate (O) at (0,0) ;
    \draw[thick,tangent=0.30,Col2] (0.8,0.8) to[out=90, in=180] coordinate[pos=0.7] (p1) 
      (3.5,3.5) node[right]{$f(x)$} ;
    \filldraw[use tangent] (0,0) node (p) {} circle (1.5pt);
    \draw[very thick,use tangent,Col1!50] (-1.7,0) node (a) {} -- (1.8,0) node (d) {} ;
    \draw[dashed,Col2] (O|-p) node[left]{\smaller$f(x_{1})$} -- (p) -- (O-|p)
      node[below]{\smaller$x_{1}$};
    \filldraw (p1) circle (1.5pt);
    \node at (p1) [above right,yshift=3pt] {$Q$} ;
    \node at (p) [above left,xshift=3pt] {$P$} ;
    % \draw[dashed] (O|-p1) node[left]{\smaller$f(x_{2})$} -- (p1) -- (O-|p1) 
    %   node[below]{\smaller$x_{2}$};
    \draw[dashed,Col2] (O|-p) -- (p) ;
    \node at (p-|p1) (x){} ;
    \draw (p.center) -- (x.center) ;
    \draw (x.center) -- (p1) ;
    \draw (p.center) -- (p1) ;
    \draw[dashed,Col2] (x) -- (O-|x) node[below] {\smaller$x_{2}$} ;
    \draw[dashed,Col2] (O|-p1) node[left] {\smaller$f(x_{2})$} -- (p1) ;
    \draw[decorate,decoration={brace,amplitude=3pt,raise=4pt,mirror}] (p.center) -- (x)
      node[midway,below=5pt,Col2] {\smaller{$\Delta x$}} ;
    \draw[decorate,decoration={brace,amplitude=3pt,raise=4pt}] (p1.center) -- (x)
      node[midway,right=5pt,Col2] {\smaller{$\Delta y$}} ;
  \end{tikzpicture}
  \captionof{figure}{ρυθμός μεταβολής}
  \label{fig:rate}
}{
  Καθώς το $x$ μεταβάλλεται από την τιμή $ x_{1} $ στην τιμή $ x_{2} $, λέμε ότι η 
  μεταβολή του $x$, είναι $\Delta x = x_{2}- x_{1} $ και η αντίστοιχη μεταβολή του $y$ 
  είναι $ \Delta y = f(x_{2}) - f(x_{1})$. Τότε, ο λόγος
  \[
    \frac{\Delta y}{\Delta x} = \frac{f(x_{2} )-f(x_{1})}{x_{2}- x_{1}}
  \] 
  λέγεται \textcolor{Col1}{μέσος ρυθμός μεταβολής} του $y$ ως προς $x$, επί του 
  διαστήματος $ [x_{1}, x_{2}] $ και αναπαρίσταται γεωμετρικά από την κλίση της 
  τέμνουσας $ PQ $, όπως φαίνεται στο σχήμα~\ref{fig:rate}.
}

\vspace{0.5\baselineskip}
Κατά αναλογία με τη διαδικασία του υπολογισμού της ταχύτητας ενός σώματος, που είδαμε 
προηγουμένως, θεωρούμε το μέσο ρυθμό μεταβολής σε όλο και μικρότερα διαστήματα,
επιτρέποντας το $ x_{2} $ να πλησιάσει το $ x_{1} $, και άρα η ποσότητα $ \Delta x $ 
να πλησιάσει το 0. Το όριο, αυτών των μέσων ρυθμών μεταβολής, καθώς το $ x_{2} $ τείνει 
στο $ x_{1} $, ονομάζεται (στιγμιαίος) \textcolor{Col1}{ρυθμός μεταβολής} του $y$ ως 
προς $x$ στο σημείο $ x= x_{1} $, και αναπαρίσταται γεωμετρικά από την \textbf{κλίση της 
εφαπτομένης} της καμπύλης $ y=f(x) $ στο σημείο $ P(x_{1}, f(x_{1})) $. Δηλαδή
\[
  \text{ρυθμός μεταβολής του $y$ ως προς $x$} = \lim_{\Delta x \to 0} \frac{\Delta
  y}{\Delta x} = \lim_{x_{2} \to x_{1}} \frac{f(x_{2} )-f(x_{1})}{x_{2}- x_{1}} 
\] 
Αναγνωρίζουμε αυτό το όριο! Είναι η παράγωγος της συνάρτησης $f$ στο σημείο $ x_{1} $.



\end{document}
