\input{preamble2.tex}
\input{definitions2.tex}
\input{tikz.tex}
\input{myboxes.tex}

\everymath{\displaystyle}
\pagestyle{vangelis}

%adds a rectangle background box behind every picture
 % \tikzset{every picture/.append style={background rectangle/.style={fill=blue!05,rounded
 %   corners},show background rectangle}}


\usepackage[font={color=Col1},labelfont={bf},hypcap=false]{caption}


\geometry{top=2.0cm,left=1.5cm,right=1.5cm}

\input{insbox}



%paragwgos
\newcommand{\twocolumnsidesr}[2]{\begin{minipage}[t]{0.36\linewidth}
        #1
        \end{minipage}\hfill\begin{minipage}[t]{0.60\linewidth}
        #2
    \end{minipage}
}

\newcommand{\twocolumnsiderrp}[2]{\begin{minipage}[c]{0.30\linewidth}
    #1
    \end{minipage}\hfill\begin{minipage}[c]{0.68\linewidth}
    #2
  \end{minipage}
}

\newcommand{\twocolumnsidell}[2]{\begin{minipage}[c]{0.68\linewidth}
    #1
    \end{minipage}\hfill\begin{minipage}[c]{0.30\linewidth}
    #2
  \end{minipage}
}

\newcommand{\twocolumnsiderrr}[2]{\begin{minipage}[c]{0.80\linewidth}
    #1
    \end{minipage}\hfill\begin{minipage}[c]{0.15\linewidth}
    #2
  \end{minipage}
}

\newcommand{\threecolumnside}[3]{\begin{minipage}[t]{0.30\linewidth}\raggedright
        #1
        \end{minipage}\hfill\begin{minipage}[t]{0.30\linewidth}\raggedright
        #2
        \end{minipage}\hfill\begin{minipage}[t]{0.30\linewidth}\raggedright
        #3
    \end{minipage}
}

\newcommand{\twocolumnsidesrr}[2]{\begin{minipage}[t]{0.35\linewidth}
    #1
    \end{minipage}\hfill\begin{minipage}[t]{0.60\linewidth}
    #2
  \end{minipage}
}


\newcommand{\twocolumnsiderr}[2]{\begin{minipage}[c]{0.30\linewidth}
    #1
    \end{minipage}\hfill\begin{minipage}[c]{0.68\linewidth}
    #2
  \end{minipage}
}

\newcommand{\twocolumnsidesss}[2]{\begin{minipage}[c]{0.48\linewidth}\raggedright
    #1
    \end{minipage}\hfill\begin{minipage}[c]{0.48\linewidth}\raggedright
    #2
  \end{minipage}
}

\tikzset{
  set arrow inside/.code={\pgfqkeys{/tikz/arrow inside}{#1}},
  set arrow inside={end/.initial=stealth, opt/.initial=},
  /pgf/decoration/Mark/.style={
    mark/.expanded=at position #1 with
    {
      \noexpand\arrow[\pgfkeysvalueof{/tikz/arrow inside/opt}]{\pgfkeysvalueof{/tikz/arrow inside/end}}
    }
  },
  arrow inside/.style 2 args={
    set arrow inside={#1},
    postaction={
      decorate,decoration={
        markings,Mark/.list={#2}
      }
    }
  },
}

\tikzset{vangelis/.style = {ultra thick,magenta,dashed}, mystyle/.style={line width=3pt,blue!55,double,rounded corners}}

\pgfplotsset{
  myaxis/.style={axis lines=center,axis line style={thick,blue!50,-stealth},tick label
  style={font=\small,blue!50},xtick=\empty,ytick=\empty,tick style={blue!50}},
  myplot/.style={Col1!75,ultra thick,samples=500,no marks,smooth,},
  dashed lines/.style={dashed,blue!50,ultra thin},
  polax/.style={grid=none},
}

\tikzset{mygraph/.pic={\begin{scope}[scale=0.7]
    \draw [in=-180, out=90, looseness=0.75] (2.3,1) to (3.55,2.75);
    \draw [in=180, out=0] (3.55,2.75) to (4.475,2);
    \draw [in=-90, out=0] (4.475,2) to (4.925,2.5);
    \end{scope}
}}

\begin{document}


\chapter{Παράγωγος}

\section{Η Έννοια της Παραγώγου}

Τα προβλήματα, της εύρεσης της εφαπτομένης μιας καμπύλης και του υπολογισμού της 
ταχύτητας ενός κινητού σώματος, ανάγονται στον υπολογισμό του ίδιου τύπου ορίου, 
όπως θα δούμε στη συνέχεια. Αυτό το όριο, ονομάζεται \textbf{παράγωγος} και ερμηνεύεται 
ως ο ρυθμός μεταβολής κάποιου μεγέθους.

\subsection{Εφαπτομένη Καμπύλης}

Για τον υπολογισμό της εφαπτομένης μιας καμπύλης με εξίσωση $ y=f(x) $, στο 
σημείο $P( x_{0}, f(x_{0})) $, θεωρούμε ένα γειτονικό σημείο $ Q(x,f(x)) $, με 
$ x \neq x_{0} $ και υπολογίζουμε την κλίση $\lambda$ της \textbf{τέμνουσας} $ PQ $
(Σχήμα~\ref{fig:slope_sec})

\twocolumnsiderrp{
  \begin{tikzpicture}[scale=0.8]
    \draw[-stealth,axis] (-0.5,0) -- (5,0) node[below] (x) {\smaller$x$} ;
    \draw[-stealth,axis] (0,-0.5) -- (0,3.7) node[left] (y) {\smaller$y$} ;
    \coordinate (o) at (0,0) ;
    \begin{scope}[yshift=-15pt]
      \draw[graph,tangent=0.28,Col1] (0.2,0.0) to[out=80, in=180] (2.2,3.0) coordinate 
        (q) to[out=0,in=-180] node[above=20pt,pos=0.7,xshift=7pt] {$y=f(x)$} (5,0.7) ;
      \fill[use tangent] (0,0) coordinate (p) circle (1.5pt);
      \fill (p) node[above,xshift=-2pt] {$P$} circle (1.5pt);
      \fill (q) node[above] {$Q$} circle (1.5pt);
      \draw[dashed] (o|-p) node[left,Col2] {\smaller$f(x_{0})$} -- (p) -- (o-|p)
        node[below,Col2] {\smaller$x_{0}$} ;
      \draw[dashed] (o|-q) node[left,Col2] {\smaller$f(x)$} -- (q) -- (o-|q) 
        node[below,Col2] {\smaller$x$} ;
      \coordinate (a) at (p-|q) ; 
      \draw[decorate,decoration={brace,amplitude=3pt,raise=3pt,mirror},Col2] (p) -- (a) ;
      \draw (a) -- (q) node[midway,xshift=5pt,yshift=-2pt,pin={[pin
        edge={black,latex-},Col2]75:\smaller$f(x)-f(x_{0})$}]{} ;
      \draw[decorate,decoration={brace,amplitude=3pt,raise=3pt,mirror},Col2] (a) -- (q) ;
      \draw[Col1] pic["$\phi$",fill=Col1!50,angle eccentricity=1.4] {angle={a--p--q}} ;
      \draw (p) -- (q) ;
      \draw (p) -- (a) node[below=3pt,midway,Col2] {\smaller$x- x_{0}$} ;
    \end{scope}
  \end{tikzpicture}
  \captionof{figure}{Κλίση τέμνουσας}
  \label{fig:slope_sec}
}{
  \[
    \lambda _{PQ} = \tan{\phi} = \frac{f(x)-f(x_{0})}{x- x_{0}} 
  \] 
  Στη συνέχεια, αφήνουμε το σημείο $Q$ να πλησιάσει το $P$, κατά μήκος της 
  καμπύλης, επιτρέποντας στο $x$ να πλησιάσει το $ x_{0} $. Αν η κλίση $\lambda_{PQ}$ 
  τείνει σε κάποιον αριθμό, έστω $ \lambda $, τότε ορίζουμε την εφαπτομένη στο σημείο 
  $P$ να είναι η ευθεία που διέρχεται από το $P$ και έχει κλίση $ \lambda $. Με άλλα 
  λόγια, η εφαπτομένη της καμπύλης στο σημείο $P$, είναι το όριο (οριακή θέση) της 
  τέμνουσας $ PQ $, καθώς το $Q$ τείνει στο $P$ (Σχήμα~\ref{fig:lim_sec}).
}

\twocolumnsiderrp{
  \begin{tikzpicture}[scale=0.8]
    \draw[-stealth,axis] (-0.5,0) -- (5,0) node[below] (x) {\smaller$x$} ;
    \draw[-stealth,axis] (0,-0.5) -- (0,3.7) node[left] (y) {\smaller$y$} ;
    \coordinate (o) at (0,0) ;
    \begin{scope}[yshift=-15pt]
      \draw[thick,tangent=0.28,Col2] (0.2,0.0) to[out=80, in=180] (2.2,3.0) coordinate 
        (q) to[out=0,in=-180] coordinate[pos=0.23] (q1) coordinate[pos=0.4] (q2) (5,0.7) ;
      \fill[use tangent] (0,0) coordinate (p) circle (1.5pt);
      \fill (p) node[above,xshift=-2pt] {$P$} circle (1.5pt);
      \fill (q) node[above=3pt] {\smaller$Q$} circle (1.5pt);
      \fill (q1) node[above right] {\smaller$Q$} circle (1.5pt);
      \fill (q2) node[above right] {\smaller$Q$} circle (1.5pt);
      \draw[very thick,use tangent,Col1] (-1.8,0) -- (2,0) node[right] {$\varepsilon$} ;
      \draw[shorten <=-1.3cm,blue!50] (p) -- ($ (p)!4cm!(q) $) ;
      \draw[shorten <=-1.3cm,blue!50] (p) -- ($ (p)!4cm!(q1) $) ;
      \draw[shorten <=-1.3cm,blue!50] (p) -- ($ (p)!4cm!(q2) $) ;
      \node at (q) [above] (f) {} ;
      \node at (p) [above right=20pt,yshift=4pt] (g) {} ;
      \node at (q1) [above] (a) {} ;
      \node at (q1) [right,xshift=1pt] (d) {} ;
      \node at (q) [above right,xshift=3pt,yshift=-1pt] (b) {} ;
      \node at (q2) [above,xshift=1pt] (c) {} ;
      \node at (q2) [right] (e) {} ;
      \draw (a.center) edge[-stealth] (b.center) ;
      \draw (c.center) edge[-stealth] (d.center) ;
      \draw (f.center) edge[-stealth] (g.center) ;
      \node at (o|-p) [left,Col2]{\phantom{\smaller$f(x_{0})$}} ;
      \draw[dashed] (p) -- (o-|p) node[below,Col2] {\smaller$x_{0}$} ;
      \draw[dashed] (q2) -- (o-|q2) node[below,Col2] {\smaller$x$} ;
      \node at (o-|q2) [below left=2pt] (r1) {} ;
      \node at (o-|q1) [below left=2pt] (r2) {} ;
      \node at (o-|q)[below left=2pt] (r3) {} ;
      \node at (o-|p)[below right=2pt] (r4) {} ;
      \path[draw,Col2] (r1.center) edge[-stealth] (r2.center) ;
      \path[draw,Col2] (r2.center) edge[-stealth,xshift=-6pt] (r3.center) ;
      \path[draw,Col2] (r3.center) edge[-stealth,xshift=-6pt] (r4.east) ;
    \end{scope}
  \end{tikzpicture}
  \captionof{figure}{Όριο τέμνουσας}
  \label{fig:lim_sec}
}{
  \begin{dfn}
    Η \textcolor{Col1}{εφαπτομένη} της καμπύλης $ y=f(x) $ στο σημείο 
    $ P(x_{0}, f(x_{0})) $, είναι η ευθεία που διέρχεται από το $P$ με κλίση 
    \begin{empheq}[box=\mathboxg]{equation}
      \label{eq:slope}
      \lambda = \lim_{x \to x_{0}} \frac{f(x)-f(x_{0})}{x- x_{0}} 
    \end{empheq}
    υπό την προϋπόθεση ότι το όριο, υπάρχει. Τότε η εξίσωσή της εφαπτομένης, δίνεται 
    από την σχέση:
    \begin{empheq}[box=\mathboxr]{equation*}
      \textcolor{Col1}{\varepsilon} \colon y- f(x_{0}) = \lambda (x- x_{0})
    \end{empheq}
  \end{dfn}
}

\subsection{Ταχύτητα σώματος}

Έστω, ένα σώμα το οποίο κινείται \textbf{ευθύγραμμα}, σύμφωνα με την εξίσωση 
κίνησης, $ x=f(t) $, όπου $x$ είναι η απομάκρυνση του σώματος από την αρχή του
συστήματος συντεταγμένων, κατά τη χρονική στιγμή $t$. Η συνάρτηση $f$ που περιγράφει 
την κίνηση του σώματος, λέγεται 
\textbf{συνάρτηση θέσης} του σώματος. 

\twocolumnsiderrp{
  \begin{tikzpicture}[scale=0.8]
    \draw[-latex] (0,0) -- coordinate[pos=0.5] (a1) coordinate[pos=0.90] (a2) 
    (6.0,0) node [below] {$x$} ;
    \node at (a1) [above,xshift=2pt] {$P$} ;
    \node at (a2) [above] {$Q$} ;
    \fill (0.5,0) node[below] (o) {$0$} circle (1.5pt) ;
    \coordinate (b) at (0.5,-0.8) ;
    \coordinate (b') at (0.5,-1.5) ;
    \coordinate (b1) at ($(a1)-(0,0.8)$) ;
    \coordinate (b2) at ($(a2)-(0,1.5)$) ;
    \fill[Col1] (a1) circle (1.5pt) ;
    \fill[Col1] (a2) circle (1.5pt) ;
    \draw[decorate,decoration={brace,amplitude=4pt,raise=5pt,mirror},Col2] (a1) -- (a2) ;
    \draw (a1) -- (a2) node[below=10pt,midway,Col2] {\smaller$f(t)- f(t_{0})$} ;
    \draw (a1) node[pin={[pin edge={black,latex-},Col2,align=center]100:\small{θέση τη
      χρον.} \\ \small{στιγμή $ t_{0} $}}]{} ;
    \draw (a2) node[pin={[pin edge={black,latex-},Col2,align=center]100:\small{θέση τη
      χρον.} \\ \small{στιγμή $ t $}}]{} ;
    \draw[very thick,Col1] (a1) -- (a2) ;
    \draw[{Stealth}-{Stealth}] (b) node {$|$} -- node[fill=white]{\smaller$f(t_{0})$} (b1)
      node {$|$} ;
    \draw[{Stealth}-{Stealth}] (b') node {$|$} -- node[fill=white]{\smaller$f(t)$} (b2) 
      node {$|$} ;
    % \draw[||{Stealth}-{Stealth}||] (0, 1) -- (1, 1);    %% || will produce thicker pipe
  \end{tikzpicture}
}{
  \vspace*{0.5\baselineskip}
  Κατά τη χρονική διάρκεια από $ t_{0} $ έως $ t $, 
  η μεταβολή της θέσης του σώματος (μετατόπιση), από τη θέση $ P $ στη θέση $ Q $,  
  είναι $ f(t)-f(t_{0}) $ και άρα η μέση ταχύτητα του σώματος είναι
  \[
    \text{μέση ταχύτητα} = \frac{\text{μετατόπιση}}{\text{χρόνος}} = 
    \frac{f(t)-f(t_{0})}{t- t_{0}}  
  \] 
  που είναι ίση με την κλίση της \textbf{τέμνουσας} $ PQ $ (Σχήμα~\ref{fig:speed}).
}

\twocolumnsiderrp{
  \begin{tikzpicture}[scale=0.8]
    \draw[-stealth,axis] (-0.5,0) -- (5,0) node[below] (x) {\smaller$t$} ;
    \draw[-stealth,axis] (0,-0.5) -- (0,4) node[left] (y) {\smaller$x$} ;
    \coordinate (o) at (0,0) ;
    \node (0) at (0.25, 0.25) {};
    \node (1) at (0.75, 1.25) {};
    \node (2) at (1.25, 0.75) {};
    \node (p) at (1.95, 2.175) {};
    \node (q) at (3.175, 3.25) {};
    \node (5) at (3.575, 3.225) {};
    \draw [thick,Col2,in=180, out=90, looseness=0.75] (0.center) to (1.center);
    \draw [thick,Col2,in=-180, out=0] (1.center) to (2.center);
    \draw [thick,Col2,in=-120, out=0, looseness=0.75] (2.center) to (p.center);
    \draw [thick,Col2,bend left,tangent=0.15] (p.center) to (q.center);
    \draw [thick,Col2,in=165, out=0, looseness=0.75] (q.center) to (5.center);
    \draw[very thick,use tangent,Col1] (-1.1,0) -- (1.2,0) ;
    \fill[use tangent] (0,0) coordinate (p) circle (1.5pt);
    \fill (p) node[above left,xshift=3pt] {$P$} circle (1.5pt);
    \fill (q) node[above] {$Q$} circle (1.5pt);
    \draw[dashed,Col2] (o) -- (o|-p) node[left] {\smaller$f(t_{0})$} -- (p) -- (o-|p)
      node[below,Col2] {\smaller$t_{0}$} ;
    \draw (p.center) -- (q.center) ;
    \draw[dashed,Col2] (o) -- (o|-q) node[left]{\smaller$f(t)$} -- (q) -- (o-|q) 
      node[below,Col2] {\smaller$t$} ;
    \coordinate (a) at (p-|q) ; 
    \draw (p) -- (a) node[below=3pt,midway,Col2] {\smaller$t- t_{0}$} ;
    \draw[decorate,decoration={brace,amplitude=3pt,raise=3pt,mirror},Col2] (p) -- (a) ;
    \draw (a) -- (q.center) node[midway,xshift=5pt,pin={[pin edge={black,latex-},Col2]-85:\smaller$\!\!\!\!f(t)\!-\!f(t_{0})$}]{} ;
    \draw[decorate,decoration={brace,amplitude=3pt,raise=3pt,mirror},Col2] (a) -- (q) ;
    \node[Col2,xshift=-2pt] at (1,2) {$x=f(t)$} ;
  \end{tikzpicture}
  \captionof{figure}{ταχύτητα σώματος}
  \label{fig:speed}
}{
  Στη συνέχεια, υποθέτουμε ότι υπολογίζουμε τη μέση ταχύτητα του σώματος, σε όλο και 
  μικρότερα χρονικά διαστήματα. Με άλλα λόγια, επιτρέπουμε στο $t$ να πλησιάσει το 
  $ t_{0} $, και ορίζουμε την \textcolor{Col1}{ταχύτητα} (ή στιγμιαία ταχύτητα) του 
  σώματος, τη χρονική στιγμή $ t_{0} $ να είναι το \textbf{όριο} της μέσης ταχύτητας
  \begin{empheq}[box=\mathboxg]{equation}
    \label{eq:speed}
    v(t_{0}) = \lim_{t \to t_{0}} \frac{f(t)-f(t_{0})}{t- t_{0}}  
  \end{empheq}
  Αυτό σημαίνει, ότι η ταχύτητα του σώματος, τη χρονική στιγμή $ t_{0} $ είναι ίση με 
  την \textbf{κλίση της εφαπτομένης} στο σημείο $P$, όπως φαίνεται από τις 
  εξισώσεις~\eqref{eq:slope} και~\eqref{eq:speed}.
}


\section{Ορισμός Παραγώγου}

Είδαμε, ότι ο ίδιος τύπος ορίου, εμφανίζεται κατά τον υπολογισμό της εφαπτομένης 
καμπύλης και της ταχύτητας ενός σώματος. Ακριβέστερα, όρια της μορφής 
\[
  \lim_{x \to x_{0}} \frac{f(x)-f(x_{0})}{x- x_{0}} 
\] 
εμφανίζονται, όπως θα δούμε, κατά τον υπολογισμό του \textcolor{Col1}{ρυθμού μεταβολής} 
κάποιου μεγέθους.

\begin{dfn}
  Η \textcolor{Col1}{παράγωγος} μιας συνάρτησης $f(x)$ στο σημείο $ x_{0} $, 
  συμβολίζεται με $ f'(x_{0}) $ και είναι ίση με 
  \begin{empheq}[box=\mathboxr]{equation*}
    f'(x_{0}) = \lim_{x \to x_{0}} \frac{f(x)-f(x_{0})}{x- x_{0}}
  \end{empheq}
  οποτεδήποτε αυτό το όριο υπάρχει. Τότε λέμε ότι η συνάρτηση $f$ είναι
  \textcolor{Col1}{παραγωγίσιμη} στο $ x_{0} $.
\end{dfn}

\begin{rem}
  Αν θέσουμε $ x- x_{0}=h $ τότε έχουμε $ x = x_{0}+h $ και $ h \to 0 $ καθώς $ x \to
  x_{0} $, οπότε με αντικατάσταση, το παραπάνω όριο, γράφεται στην ισοδύναμη μορφή
  \begin{empheq}[box=\mathboxr]{equation*}
    f'(x_{0}) = \lim_{h \to 0} \frac{f(x_{0}+h)-f(x_{0})}{h}
  \end{empheq}
\end{rem}


\subsection{Ρυθμός Μεταβολής}

Έστω ότι η μεταβλητή $y$ είναι συνάρτηση του $x$, δηλαδή $ y=f(x) $.

\twocolumnsiderrp{
\begin{tikzpicture}[scale=0.8]
    \draw[-stealth,axis] (-0.5,0) -- (4.5,0) node[below] (x) {\smaller$x$} ;
    \draw[-stealth,axis] (0,-0.5) -- (0,4) node[left] (y) {\smaller$y$} ;
    \coordinate (O) at (0,0) ;
    \draw[thick,tangent=0.30,Col2] (0.8,0.8) to[out=90, in=180] coordinate[pos=0.7] (p1) 
      (3.5,3.5) node[right]{$f(x)$} ;
    \filldraw[use tangent] (0,0) node (p) {} circle (1.5pt);
    \draw[very thick,use tangent,Col1!50] (-1.7,0) node (a) {} -- (1.8,0) node (d) {} ;
    \draw[dashed,Col2] (O|-p) node[left]{\smaller$f(x_{1})$} -- (p) -- (O-|p)
      node[below]{\smaller$x_{1}$};
    \filldraw (p1) circle (1.5pt);
    \node at (p1) [above right,yshift=3pt] {$Q$} ;
    \node at (p) [above left,xshift=3pt] {$P$} ;
    % \draw[dashed] (O|-p1) node[left]{\smaller$f(x_{2})$} -- (p1) -- (O-|p1) 
    %   node[below]{\smaller$x_{2}$};
    \draw[dashed,Col2] (O|-p) -- (p) ;
    \node at (p-|p1) (x){} ;
    \draw (p.center) -- (x.center) ;
    \draw (x.center) -- (p1) ;
    \draw (p.center) -- (p1) ;
    \draw[dashed,Col2] (x) -- (O-|x) node[below] {\smaller$x_{2}$} ;
    \draw[dashed,Col2] (O|-p1) node[left] {\smaller$f(x_{2})$} -- (p1) ;
    \draw[decorate,decoration={brace,amplitude=3pt,raise=4pt,mirror}] (p.center) -- (x)
      node[midway,below=5pt,Col2] {\smaller{$\Delta x$}} ;
    \draw[decorate,decoration={brace,amplitude=3pt,raise=4pt}] (p1.center) -- (x)
      node[midway,right=5pt,Col2] {\smaller{$\Delta y$}} ;
  \end{tikzpicture}
  \captionof{figure}{ρυθμός μεταβολής}
  \label{fig:rate}
}{
  Καθώς το $x$ μεταβάλλεται από την τιμή $ x_{1} $ στην τιμή $ x_{2} $, λέμε ότι η 
  μεταβολή του $x$, είναι $\Delta x = x_{2}- x_{1} $ και η αντίστοιχη μεταβολή του $y$ 
  είναι $ \Delta y = f(x_{2}) - f(x_{1})$. Τότε, ο λόγος
  \[
    \frac{\Delta y}{\Delta x} = \frac{f(x_{2} )-f(x_{1})}{x_{2}- x_{1}}
  \] 
  λέγεται \textcolor{Col1}{μέσος ρυθμός μεταβολής} του $y$ ως προς $x$, επί του 
  διαστήματος $ [x_{1}, x_{2}] $ και αναπαρίσταται γεωμετρικά από την κλίση της 
  τέμνουσας $ PQ $, όπως φαίνεται στο σχήμα~\ref{fig:rate}.
}

\vspace{0.5\baselineskip}
Κατά αναλογία με τη διαδικασία του υπολογισμού της ταχύτητας ενός σώματος, που είδαμε 
προηγουμένως, θεωρούμε το μέσο ρυθμό μεταβολής σε όλο και μικρότερα διαστήματα,
επιτρέποντας το $ x_{2} $ να πλησιάσει το $ x_{1} $, και άρα η ποσότητα $ \Delta x $ 
να πλησιάσει το 0. Το όριο, αυτών των μέσων ρυθμών μεταβολής, καθώς το $ x_{2} $ τείνει 
στο $ x_{1} $, ονομάζεται (στιγμιαίος) \textcolor{Col1}{ρυθμός μεταβολής} του $y$ ως 
προς $x$ στο σημείο $ x= x_{1} $, και αναπαρίσταται γεωμετρικά από την \textbf{κλίση της 
εφαπτομένης} της καμπύλης $ y=f(x) $ στο σημείο $ P(x_{1}, f(x_{1})) $. Δηλαδή
\[
  \text{ρυθμός μεταβολής του $y$ ως προς $x$} = \lim_{\Delta x \to 0} \frac{\Delta
  y}{\Delta x} = \lim_{x_{2} \to x_{1}} \frac{f(x_{2} )-f(x_{1})}{x_{2}- x_{1}} 
\] 
Αναγνωρίζουμε αυτό το όριο! Είναι η παράγωγος της συνάρτησης $f$ στο σημείο $ x_{1} $.





\section{Παράγωγοι Βασικών Συναρτήσεων}

Με τη βοήθεια του ορισμού της παραγώγου, που είδαμε σε προηγούμενη ενότητα,
αποδεικνύονται οι τύποι για τις παραγώγους των βασικών συναρτήσεων, στον πίνακα που 
ακολουθεί.

\begin{center}
\begin{Mytable}
  \renewcommand{\arraystretch}{2.0}
  \begin{tabular}{|c||c|}
    \TabCellHead Βασικές Συναρτήσεις & \TabCellHead Τριγωνομετρικές Συναρτήσεις \\[4pt] \hline
    $ (c)' = 0, \quad c \in \mathbb{R} $ & $ (\sin{x})' = \cos{x} $ \\[4pt] \hline
    $ (x)' = 1 $ & $ (\cos{x})' = - \sin{x} $ \\[4pt] \hline 
    $ (x^{a})' = a x^{a-1} $ & $ (\tan{x})' = \frac{1}{\cos^{2}{x}}$ \\[4pt] \hline
    $ \Bigl(\frac{1}{x}\Bigr)' = - \frac{1}{x^{2}} $ & $ (\cot{x})' = - \frac{1}{\sin^{2}{x}}  $ \\[4pt] \hline
    $ \Bigl(\frac{1}{x^{2}}\Bigr)' = - \frac{2}{x^{3}} $ & $ (\arctan{x})' = \frac{1}{1 + x^{2}} $ \\[4pt] \hline
    $ (a^{x})' = a^{x}\cdot \ln{a} $ & $ (\arccot{x})' = \frac{-1}{1 + x^{2}} $ \\[4pt] \hline
    $ (e^{x})' = e^{x} $ &  $ (\arcsin{x})' = \frac{1}{\sqrt{1 - x^{2}}} $ \\[4pt] \hline
    $ (\ln{x})' = \frac{1}{x} $ & $ (\arccos{x})' = \frac{-1}{\sqrt{1 - x^{2}}} $ \\[4pt] \hline
    $ (\sqrt[n]{x^{m}})' = (x^{\frac{m}{n}})'= \frac{m}{n} x^{\frac{m}{n} -1} $ & $ (\sinh{x})' = \cosh{x} $ \\[4pt] \hline
    $ (\sqrt{x})' = \frac{1}{2 \sqrt{x}} $ & $ (\cosh{x})' = \sinh{x} $ \\[4pt] \hline
  \end{tabular}
\end{Mytable}
\end{center}

\subsection{Παραδείγματα}

\twocolumnsidesr{
  \begin{example}
    $ (3)'= 0 $
  \end{example}
  \begin{example}
    $ (x^{3})' = 3x^{2} $
  \end{example}
  \begin{example}
    $ (3^{x})' = 3^{x} \ln{3} $
  \end{example}
  \begin{example}
    $ \Bigl(\frac{1}{x^{4}}\Bigr)'\!\! =\! (x^{-4})'\!\! =\! -4x^{-5} \!\!\! $
  \end{example}
}{
  \begin{example}
    $ (x^{2/3})' = \frac{2}{3}x^{\frac{2}{3} - 1} = \frac{2}{3} x^{- \frac{1}{3}} = 
    \frac{2}{3} \frac{1}{x^{\frac{1}{3}}} = \frac{2}{3 \sqrt[3]{x}}$
  \end{example}
  \begin{example}
    $ ( \sqrt[3]{x} )' = (x^{\frac{1}{3} })' = \frac{1}{3} x^{\frac{1}{3} - 1} =
    \frac{1}{3} x^{-\frac{2}{3} } = \frac{1}{3 x^{\frac{2}{3}}} = \frac{1}{3
    \sqrt[3]{x^{2}}} $
  \end{example}
  \begin{example}
    $ \bigl(\sqrt[4]{x^{3}}\bigr)' \!= \bigl(x^{\frac{3}{4}}\bigr)' 
    \!= \frac{3}{4} x^{\frac{3}{4} -1} = \frac{3}{4} x^{-\frac{1}{4}} = \frac{3}{4}
    \frac{1}{x^{\frac{1}{4}}} = \frac{3}{4 \sqrt[4]{x}} $
  \end{example}
}


\section{Κανόνες Παραγώγισης}

Οι προτάσεις του, επόμενου πίνακα, γνωστές ως \textcolor{Col1}{κανόνες παραγώγισης}, 
σε συνδυασμό με τους τύπους του προηγούμενου πίνακα, μας δίνουν τη δυνατότητα να
υπολογίζουμε τις παραγώγους, πιο πολύπλοκων συναρτήσεων.

\begin{center}
  \begin{Mytable}
    \renewcommand{\arraystretch}{2.0}
    \begin{tabular}{|c|c|}
      \TabCellHead Αθροίσματος & $ (f(x)+g(x))' = f'(x)+ g'(x) $ \\[4pt] \hline 
      \TabCellHead Σταθεράς & $ (a f(x))' = a f'(x)$ \\[4pt] \hline
      \TabCellHead Γινομένου & $ (f(x)\cdot g(x))' = f'(x)\cdot g(x) 
      + f(x) g'(x) $ \\[4pt] \hline
      \TabCellHead Πηλίκου & $ \Bigl(\frac{f(x)}{g(x)}\Bigr)' = \frac{f'(x)\cdot
      g(x) - f(x) g'(x)}{g^{2}(x)} $ \\[4pt] \hline
      \end{tabular}
    \end{Mytable}
  \end{center}


  \section{Παραδείγματα}

  \subsection{Κανόνας Αθροίσματος}
  \begin{example}
    $( \cos{x} + \sqrt{x})' = (\cos{x} )' + (\sqrt{x} )' = - \sin{x} + \frac{1}{2
    \sqrt{x}} $
  \end{example}
  \begin{example}
    $ \Bigl(\frac{1}{x} - \tan{x} \Bigr)' = \Bigl(\frac{1}{x}\Bigr)' - (\tan{x} )' 
    = - \frac{1}{x^{2}} - \frac{1}{\cos^{2}{x}} $
  \end{example}


  \subsection{Κανόνας Σταθεράς}
  \begin{example}
    $( 3 \ln{x})' = 3 \cdot (\ln{x} )' = 3 \cdot \frac{1}{x} = \frac{3}{x} $
  \end{example}
  \begin{example}
    $\Bigl(\frac{2}{x}\Bigr)' = 2 \cdot \Bigl(\frac{1}{x}\Bigr)' = 2 \cdot
    \Bigl(-\frac{1}{x^{2}}\Bigr) = -\frac{2}{x^{2}} $
  \end{example}

  \subsection{Κανόνας Γινομένου}
  \begin{example}
    $ (x^{2}\cdot \ln{x})' = (x^{2})' \cdot \ln{x} + x^{2}\cdot (\ln{x})' = 2x \cdot
    \ln{x} + x^{2} \cdot \frac{1}{x} = 2x \ln{x} + x = x(2 \ln{x} +1) $
  \end{example}
  \begin{example}
    $ (\mathrm{e}^{x} \cdot \sin{x})' = (\mathrm{e}^{x} )'\cdot \sin{x} +
    \mathrm{e}^{x} \cdot (\sin{x} )' = \mathrm{e}^{x} \cdot \sin{x} + \mathrm{e}^{x}
    \cos{x} = \mathrm{e}^{x} \cdot (\sin{x} + \cos{x} ) $
  \end{example}

  \subsection{Κανόνας Πηλίκου}
  \begin{example}
    $ \Bigl(\frac{x^{2}}{\mathrm{e}^{x}}\Bigr)' = \frac{(x^{2})' \cdot \mathrm{e}^{x} 
    - x^{2}\cdot (\mathrm{e}^{x} )'}{(\mathrm{e}^{x})^{2}} 
    = \frac{2x \cdot \mathrm{e}^{x} - x^{2}\cdot \mathrm{e}^{x}}{(\mathrm{e}^{x}) ^{2}} 
    = \frac{\mathrm{e}^{x} (2x-x^{2})}{(\mathrm{e}^{x})^2} = 
    \frac{2x - x^{2}}{\mathrm{e}^{x}} $
  \end{example}
  \begin{example}
    $\Bigl( \frac{\ln{x}}{x}\Bigr)' = \frac{(\ln{x} )' \cdot x- \ln{x} \cdot (x)'}{x^{2}} 
    = \frac{\frac{1}{x} \cdot x - \ln{x} \cdot 1}{x^{2}} = \frac{1- \ln{x}}{x^{2}} $
  \end{example}


  \section{Ασκήσεις Παραγώγισης}

  \begin{exercise}  Να υπολογιστεί η παράγωγος της συνάρτησης 
    $f(x) = \sqrt{x} \cdot \ln{x} $.
    \begin{align*}
      (\sqrt{x} \cdot \ln{x} )' 
        &= (\sqrt{x} )'\cdot \ln{x} + \sqrt{x} \cdot (\ln{x}
        )' = \frac{1}{2 \sqrt{x} } \cdot \ln{x} + \sqrt{x} \cdot \frac{1}{x} =
        \frac{\ln{x}}{2 \sqrt{x}} + \frac{\sqrt{x}}{x} = 
        \frac{x \ln{x}+ 2 (\sqrt{x})^{2}}{2 x \sqrt{x} } \\
        &= \frac{x (\ln{x} + 2)}{2x \sqrt{x}} = \frac{\ln{x} +2}{2 \sqrt{x}}
    \end{align*}
  \end{exercise}

  \begin{exercise}  Να υπολογιστεί η παράγωγος της συνάρτησης 
    $ f(x) = \sqrt[3]{x^{2}} \cdot \tan{x} $
    \begin{align*}
      (\sqrt[3]{x^{2}} \cdot \tan{x} )' 
  &= (\sqrt[3]{x^{2}} )' \cdot \tan{x} +
  \sqrt[3]{x^{2}} \cdot (\tan{x} )' = (x^{\frac{2}{3}})' \cdot \tan{x} +
  \sqrt[3]{x^{2}} \cdot \frac{1}{\cos^{2}{x}} = \frac{2}{3} \cdot x^{\frac{2}{3}
  -1} \cdot \tan{x} + \frac{\sqrt[3]{x^{2}}}{\cos^{2}{x}} \\
  &= \frac{2}{3} \cdot x^{- \frac{1}{3} } \cdot \tan{x} + 
  \frac{\sqrt[3]{x^{2}}}{\cos^{2}{x}} = \frac{2 \tan{x}}{3 \sqrt[3]{x}} +
  \frac{\sqrt[3]{x^{2}}}{\cos^{2}{x}} 
    \end{align*}
  \end{exercise}

  \begin{exercise} Να υπολογιστεί η παράγωγος της συνάρτησης 
    $ f(x) = 2^{x} \cdot \cot{x} $
    \begin{align*}
      (2^{x}\cdot \cot{x})' 
      &= (2^{x})' \cdot \cot{x} + 2^{x} \cdot (\cot{x} )' =
      2^{x}\cdot \ln{2} \cdot \cot{x} + 2^{x}\cdot \Bigl(- \frac{1}{\sin^{2}{x}}\Bigr) = 
      2^{x} \ln{2} \cdot \cot{x} - \frac{2^{x}}{\sin^{2}{x}} 
    \end{align*}
  \end{exercise}

  \begin{exercise} Να υπολογιστεί η παράγωγος της συνάρτησης 
    $ f(x) = x^{3} \cdot \arcsin{x} $
    \[
      \Bigl(x^{3}\cdot \arcsin{x}\Bigr)' = (x^{3})' \cdot \arcsin{x} + x^{3} \cdot (\arcsin{x} )' = 
      3x^{2} \cdot \arcsin{x} + x^{3} \cdot \frac{1}{\sqrt{1-x^{2}}} = 3x^{2} \arcsin{x}
      + \frac{x^{3}}{\sqrt{1 - x^{2}}} 
    \]
  \end{exercise}

  \begin{exercise} Να υπολογιστεί η παράγωγος της συνάρτησης 
    $ f(x) = \frac{\sin{x}}{x^{3}} $
    \[
      \Bigl(\frac{\sin{x}}{x^{3}} \Bigr)' = \frac{(\sin{x} )' \cdot x^{3}- \sin{x} \cdot
      (x^{3})'}{(x^{3})^{2}} = \frac{\cos{x} \cdot x^{3}- \sin{x} \cdot 3x^{2}}{x^{6}} = 
      \frac{x^{2}(x \cos{x} - 3 \sin{x})}{x^{6}} = \frac{x \cos{x} - 3 \sin{x}}{x^{4}}
    \]  
  \end{exercise}

  \begin{exercise} Να υπολογιστεί η παράγωγος της συνάρτησης 
    $ f(x) = \frac{x^{2}+5}{\ln{x}} $
    \begin{align*}
      \Bigl(\frac{x^{2}+5}{\ln{x}} \Bigr)' 
    & = \frac{(x^{2}+5)'\cdot \ln{x} - (x^{2}+5)\cdot
      (\ln{x} )'}{(\ln{x} )^{2}} = \frac{2x \cdot \ln{x} - (x^{2}+5) \cdot
    \frac{1}{x}}{\ln^{2}{x}} = \frac{2x \cdot \ln{x}}{\ln^{2}{x}} - \frac{x^{2}+5}{x
  \ln^{2}{x}} = \frac{2x}{\ln{x}} - \frac{x^{2}+5}{x \ln^{2}{x}} 
    \end{align*}
  \end{exercise}

  \begin{exercise} Να υπολογιστεί η παράγωγος της συνάρτησης 
    $ f(x) = \frac{x \mathrm{e}^{x}}{5-x^{2}} $
    \begin{align*}
      \Bigl(\frac{x \mathrm{e}^{x}}{5-x^{2}} \Bigr)' 
    &= \frac{(x \mathrm{e}^{x} )'\cdot
    (5-x^{2})- x \mathrm{e}^{x} \cdot (5-x^{2})'}{(5-x^{2})^{2}} = \frac{[(x)'
    \mathrm{e}^{x} + x( \mathrm{e}^{x} )']\cdot (5-x^{2}) - x \mathrm{e}^{x}\cdot 
  (-2x)}{(5-x^{2})^{2}} \\ 
    &= \frac{(\mathrm{e}^{x} + x \mathrm{e}^{x})(5-x^{2})+2x^{2} 
    \mathrm{e}^{x}}{(5-x^{2})^{2}} 
    = \frac{\mathrm{e}^{x} (1+x)(5-x^{2})+2x^{2} \mathrm{e}^{x}}{(5-x^{2})^{2}} 
    = \frac{\mathrm{e}^{x} (5-x^{2}+5x-x^{3}+2x^{2})}{(5-x^{2})^{2}} \\
    &= \frac{\mathrm{e}^{x} (-x^{3}+x^{2}+5x+5)}{(5-x^{2})^{2}} 
    \end{align*}
  \end{exercise}



  \section{Συμβολισμός Leibniz}

  Αν $ y= f(x) $ είναι κάποια συνάρτηση, τότε για την παράγωγό της, χρησιμοποιούμε 
  συνήθως τους συμβολισμούς,
  \[
    y' = f'(x) = \color{Col1}{\dv{\color{black}{y}}{x}} =
    \color{Col1}{\dv{\color{black}{f}}{x}} =
    \color{Col1}{\dv{}{x}} \color{black}{f (x)} 
  \] 
  όπου με το σύμβολο $\color{Col1}{\mathrm{d}/\mathrm{d}x}$ δηλώνουμε την πράξη της 
  \textbf{παραγώγισης ως προς $x$}. Για το συμβολισμό της παραγώγου, σε κάποια 
  \textbf{συγκεκριμένη τιμή} $ x= a $, γράφουμε
  \[
    f'(a) = \eval{\dv{y}{x}}_{x= a} = \eval{\dv{f}{x}}_{x=a} 
  \] 
  \begin{example}
    Να υπολογιστεί η παράγωγος της συνάρτησης $ y= x^{3} $, όταν $ x=2 $.
  \end{example}
  \begin{solution}
    $ f'(2) = \dv{}{x}\eval{(x^{3})}_{x=2} = \eval{3x^{2}}_{x=2} = 3\cdot (2)^{2} = 3\cdot
    4 = 12 $
  \end{solution}
  \begin{rem}
    Ο συμβολισμός $ y' $ και $ f' $ για την παράγωγο μιας συνάρτησης, συναντάται στις
    σημειώσεις του \textbf{Νεύτωνα}, ενώ ο συμβολισμός $ \mathrm{d}y/\mathrm{d}x $ και 
    $ \mathrm{d}f/\mathrm{d}x $, στις αντίστοιχες σημειώσεις του \textbf{Leibniz}, όπου 
    την ίδια περίοδο, αλλά ανεξάρτητα ο ένας από τον άλλον, εργάστηκαν πάνω στις ιδέες του 
    διαφορικού λογισμού.
  \end{rem}
  \begin{rem}
    Αν η ανεξάρτητη μεταβλητή είναι ο \textbf{χρόνος}, αν δηλαδή, για παράδειγμα 
    $ y=f(t) $, τότε για την παράγωγο χρησιμοποιούμε αρκετά συχνά και τους συμβολισμούς
    \[
      \dot y = \dot f
    \] 
    όπου με την τελεία, δηλώνουμε την πράξη της \textbf{παραγώγισης ως προς $t$}. 
  \end{rem}


  \section{Παράγωγοι Ανώτερης τάξης}

  Αν $f$ είναι μια \textbf{παραγωγίσιμη} συνάρτηση, τότε η παράγωγός της $ f' $, είναι 
  επίσης μία συνάρτηση, η οποία αν ξαναπαραγωγιστεί, συμβολίζεται με $ (f')' = f'' $ 
  και λέγεται \textcolor{Col1}{2η παράγωγος} ή \textcolor{Col1}{παράγωγος 2ης τάξης}, της 
  συνάρτησης $f$. Με το συμβολισμό του Leibniz γράφουμε αντίστοιχα, για τη 2η παράγωγο
  \[
    \dv{}{x} {\Bigl(\dv{f}{x}\Bigr)} = \color{Col1}{\dv[2]{\color{black}{f}}{x}}  
  \] 
  Με ανάλογο τρόπο, προκύπτουν και οι υπόλοιπες παράγωγοι, \textbf{ανώτερης τάξης}, 
  της συνάρτησης $f$. Για τη $n$-οστή παράγωγο της συνάρτησης $f$, γράφουμε
  \[
    f^{(n)}(x) = \dv[n]{f}{x} %, \quad \text{όπου} \; n=2,3,4,\ldots
  \] 
  \begin{example}
    Να υπολογιστούν η 2η και 3η παράγωγος για τη συνάρτηση $ y=x^{4}+3x^{2}-1 $.
  \end{example}
  \begin{solution}
    Αρχικά, υπολογίζουμε την 1η παράγωγο.
    \[
      f'(x) = (x^{4}+3x^{2}-1)' = 4x^{3}+6x 
    \] 
    Υπολογίζουμε τη 2η παράγωγο, ξαναπαραγωγίζοντας την 1η παράγωγο.
    \[
      f''(x) = (4x^{3}+6x)' = 12x^{2}+6 
    \] 
    Επομένως, για την 3η παράγωγο, έχουμε
    \[
      f'''(x) = (12x^{2}+6)' = 24x 
    \]
  \end{solution}


  \section{Παράγωγος Σύνθετων Συναρτήσεων}

  \subsection{Κανόνας Αλυσίδας - Τύπος Leibniz}

  \begin{prop}[\textcolor{Col1}{Κανόνας Αλυσίδας}]
    Αν η $g$ είναι παραγωγίσιμη συνάρτηση στο $x$ και η $f$ είναι παραγωγίσιμη στο $ g(x)
    $, τότε η \textbf{σύνθετη} συνάρτηση $ y= f(g(x)) $ είναι επίσης παραγωγίσιμη στο $x$ 
    και η παράγωγός της δίνεται από τη σχέση
    \begin{empheq}[box=\mathboxr]{equation}
      \label{eq:chain}
      \bigl[f(g(x))\bigr]' = f'(g(x)) \cdot g'(x)
    \end{empheq}
    Με το συμβολισμό του Leibniz, αν $ y=f(u) $ και $ u=g(x) $ είναι παραγωγίσιμες
    συναρτήσεις, τότε 
    \begin{equation}\label{eq:leib}
      \dv{y}{x} = \dv{y}{u} \cdot \dv{u}{x} 
    \end{equation} 
  \end{prop}


  \section{Πίνακας Παραγώγων Σύνθετων Συναρτήσεων}

  \begin{center}
    \begin{Mytable}
      \renewcommand{\arraystretch}{2.4}
      \begin{tabular}{|c||c|}
        \TabCellHead Βασικές Συναρτήσεις & \TabCellHead Τριγωνομετρικές Συναρτήσεις \\[5pt] \hline
        -- & $ \bigl(\sin{g(x)}\bigr)' = \cos{g(x)} \cdot g'(x) $ \\[5pt] \hline
        -- & $ \bigl(\cos{g(x)}\bigr)' = - \sin{g(x)}\cdot g'(x) $ \\[5pt] \hline 
        $ \bigl(g(x)^{a}\bigr)' = a g(x)^{a-1}\cdot g'(x) $ & $ \bigl(\tan{g(x)}\bigr)' = \frac{1}{\cos^{2}{g(x)}} \cdot g'(x) $ \\[5pt] \hline
        $ \Bigl(\frac{1}{g(x)}\Bigr)' = - \frac{1}{g^{2}(x)}\cdot g'(x) $ & $
        \bigl(\cot{g(x)}\bigr)' = - \frac{1}{\sin^{2}{g(x)}} \cdot g'(x) $ \\[5pt] \hline
        $ \Bigl(\frac{1}{g^{2}(x)}\Bigr)' = - \frac{2}{g^{3}(x)} \cdot g'(x) $ & $
        \bigl(\arctan{g(x)}\bigr)' = \frac{1}{1 + g^{2}(x)} \cdot g'(x) $ \\[5pt] \hline
        $ \bigl(a^{g(x)}\bigr)' = a^{g(x)}\cdot \ln{a} \cdot g'(x) $ & $
        \bigl(\arccot{g(x)}\bigr)' = \frac{-1}{1 + g^{2}(x)} \cdot g'(x) $ \\[5pt] \hline
        $ \bigl(e^{g(x)}\bigr)' = e^{g(x)} \cdot g'(x) $ &  $ \bigl(\arcsin{g(x)}\bigr)' =
        \frac{1}{\sqrt{1 - g^{2}(x)}} \cdot g'(x) $ \\[5pt] \hline
        $ \bigl(\ln{g(x)}\bigr)' = \frac{1}{g(x)} \cdot g'(x) $ & $ \bigl(\arccos{g(x)}\bigr)'
        = \frac{-1}{\sqrt{1 - g^{2}(x)}} \cdot g'(x) $ \\[5pt] \hline
        $ \bigl(\sqrt[n]{g(x)^{m}}\bigr)' = \bigl(g(x)^{\frac{m}{n}}\bigr)'= \frac{m}{n}
        g(x)^{\frac{m}{n} -1} \cdot g'(x) $ & $ \bigl(\sinh{g(x)}\bigr)' = \cosh{g(x)}
        \cdot g'(x) $ \\[5pt] \hline
        $ \bigl(\sqrt{g(x)}\bigr)' = \frac{1}{2 \sqrt{g(x)}} \cdot g'(x) $ & $
        \bigl(\cosh{g(x)}\bigr)' = \sinh{g(x)} \cdot g'(x) $ \\[5pt] \hline
      \end{tabular}
    \end{Mytable}
  \end{center}


  \subsection{Παραδείγματα}

  \begin{example}
    $ \bigl(\mathrm{e}^{3x^{2}}\bigr)' = \mathrm{e}^{3x^{2}} \cdot (3x^{2})' =
    \mathrm{e}^{3x^{2}} \cdot 6x = 6x \mathrm{e}^{3x^{2}} $
  \end{example}
  \begin{example}
    $ \bigl(\sin{(\sqrt{x})} \bigr)' = \cos{(\sqrt{x})} \cdot (\sqrt{x})' =
    \cos{(\sqrt{x})} \cdot \frac{1}{2 \sqrt{x}} = \frac{\cos{(\sqrt{x})}}{2 \sqrt{x}} $
  \end{example}
  \begin{example}
    $ \bigl(\tan{(2x-3)}\bigr)' = \frac{1}{\cos^{2}{(2x-3)}} \cdot (2x-3)' =
    \frac{1}{\cos^{2}{(2x-3)}} \cdot 2 = \frac{2}{\cos^{2}{(2x-3)}} $
  \end{example}
  \begin{example}
    $ \bigl(\sqrt{x^{3}+2x}\bigr)' = \frac{1}{2 \sqrt{x^{3}+2x}} 
    \cdot \bigl(x^{3}+2x\bigr)' = \frac{3x^{2}+2}{2 \sqrt{x^{3}+2x}} $
  \end{example}
  \begin{example}
    $ \bigl((x^{3}+1)^{3}\bigr)' = 3 (x^{3}+1)^{2} \cdot (x^{3}+1)' = 3 (x^{3}+1)^{2}
    \cdot 3x^{2} = 9 x^{2} (x^{3}+1)^{2} $
  \end{example}
  \begin{example}
    $ (\ln^{2}x)' = [(\ln{x} )^{2}]' = 2 \ln{x} \cdot (\ln{x} )' = 2 \ln{x} \cdot
    \frac{1}{x} = \frac{2 \ln{x}}{x} $
  \end{example}
  \begin{example}
    $ \bigl(\sin^{2}{(3x)}\bigr)' \!\! = 2 \sin{(3x)} \bigl(\sin{(3x)}\bigr)' 
    \! = 2 \sin{(3x)} \cos{(3x)} (3x)' 
    \! = 2 \sin{(3x)} \cos{(3x)} \cdot 3 
    \! = 6 \sin{(3x)} \cos{(3x)} \!\!\!\!\! $
  \end{example}
  \begin{example}
    $ \Bigl(\frac{1}{\ln{x}} \Bigr)' = - \frac{1}{(\ln{x} )^{2}} \cdot (\ln{x} )' 
    = - \frac{1}{\ln^{2}{x}} \cdot \frac{1}{x} = - \frac{1}{x \ln^{2}{x}} $ 
  \end{example}
  \begin{example}
    $ \Bigl(\frac{1}{\sqrt{x}}\Bigr)' = - \frac{1}{(\sqrt{x} )^{2}} \cdot (\sqrt{x})' = 
    = - \frac{1}{x} \cdot \frac{1}{2 \sqrt{x}} = - \frac{1}{2 x \sqrt{x}} $
  \end{example}
  \begin{example}
    $ \Bigl(\frac{1}{x^{2}+1} \Bigr)' = - \frac{1}{(x^{2}+1)^{2}} \cdot (x^{2}+1)' = 
    - \frac{1}{(x^{2}+1)^{2}} \cdot 2x = - \frac{2x}{(x^{2}+1)^{2}} $ 
  \end{example}
  \begin{exercise}
    Έστω η σύνθετη συνάρτηση $ y= \ln{\bigl(\sqrt{x^{2}+1}\bigr)} $. 
    Να υπολογιστεί η παράγωγος της συνάρτησης.  
  \end{exercise}
  \begin{solution}
    Χρησιμοποιώντας τη σχέση~\eqref{eq:chain}, έχουμε:
    \[
      \Bigl[\ln{\bigl(\sqrt{x^{2}+1}\bigr)} \Bigr]' = \frac{1}{\sqrt{x^{2}+1}} \cdot 
      \bigl(\sqrt{x^{2}+1} \bigr)' = \frac{1}{\sqrt{x^{2}+1}} \cdot 
      \frac{1}{2 \sqrt{x^{2}+1}} \cdot (x^{2}+1)' = \frac{1}{2 
      \bigl(\sqrt{x^{2}+1}\bigr)^{2}} \cdot 2x = \frac{x}{x^{2}+1}
    \]  
    Αν θέλουμε να χρησιμοποιήσουμε τη σχέση~\eqref{eq:leib}, τότε \textbf{αναλύουμε} 
    τη σύνθετη συνάρτηση, σε απλούστερες, ξεκινώντας από <<μέσα>> και πηγαίνοντας 
    διαδοχικά προς τα <<έξω>>. Έτσι, θέτουμε $ u=x^{2}+1 $ και στη συνέχεια
    $v= \sqrt{x^{2}+1} = \sqrt{u} $ και άρα η συνάρτηση μας, γίνεται

    \twocolumnsiderrr{
      \[
        y= \ln{\bigl(\sqrt{x^{2}+1}\bigr)} \longrightarrow 
        \begin{cases}
          y= \ln{v} \\
          v= \sqrt{u}  \\
          u= x^{2}+1  
        \end{cases}
      \]
      Οπότε, για την παράγωγο, σύμφωνα με τον τύπο του Leibniz, θα ισχύει
      \[
        \dv{y}{x} = \dv{y}{v} \cdot \dv{v}{u} \cdot \dv{u}{x} = \frac{1}{v} \cdot
        \frac{1}{2 \sqrt{u}} \cdot 2x = \frac{1}{\sqrt{x^{2}+1}} \cdot \frac{1}{2
        \sqrt{x^{2}+1}} \cdot 2x = \frac{x}{x^{2}+1}
      \] 
    }{
      \begin{tikzpicture}
        \node (a) at (0,0) {$y$} ;
        \node (b) at (0,-1) {$v$} ;
        \node (c) at (0,-2) {$u$} ;
        \node (d) at (0,-3) {$x$} ;
        \draw (a) -- node[right,Col1]{$\dv{y}{v}$} (b) ;
        \draw (b) -- node[right,Col1]{$\dv{v}{u}$} (c) ;
        \draw (c) -- node[right,Col1]{$\dv{u}{x}$} (d) ;
      \end{tikzpicture}       
    }
  \end{solution}

\end{document}
