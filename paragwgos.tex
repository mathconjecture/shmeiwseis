\input{preamble_ask.tex}
\input{definitions_ask.tex}
\input{tikz.tex}
\input{myboxes.tex}

\geometry{top=2.0cm,left=1.5cm,right=1.5cm}

\input{insbox}



\everymath{\displaystyle}
\pagestyle{vangelis}


\tikzset{tangent/.style=
  {decoration=
    {markings, mark=at position #1 with
      {\coordinate (tangent point-\pgfkeysvalueof{/pgf/decoration/mark info/sequence number}) at (0pt,0pt);
        \coordinate (tangent unit vector-\pgfkeysvalueof{/pgf/decoration/mark info/sequence number}) at (1,0pt);
        \coordinate (tangent orthogonal unit vector-\pgfkeysvalueof{/pgf/decoration/mark info/sequence number}) at (0pt,1);
      }
    },
  postaction=decorate},
  use tangent/.style={shift=(tangent point-#1),
    x=(tangent unit vector-#1),
  y=(tangent orthogonal unit vector-#1)},
  use tangent/.default=1}


\newcommand{\twocolumnsiderr}[2]{\begin{minipage}[c]{0.30\linewidth}\raggedright
        #1
       \end{minipage}\hfill\begin{minipage}[c]{0.68\linewidth}\raggedright
        #2
    \end{minipage}
}

\newcommand{\twocolumnsidell}[2]{\begin{minipage}[c]{0.68\linewidth}\raggedright
        #1
        \end{minipage}\hfill\begin{minipage}[c]{0.30\linewidth}\raggedright
        #2
    \end{minipage}
}




\begin{document}



\chapter*{Παράγωγος}


\section*{Η Έννοια της Παραγώγου}

Το πρόβλημα της εύρεσης της εφαπτομένης μιας καμπύλης και του υπολογισμού της 
ταχύτητας ενός κινητού σώματος, ανάγονται στον υπολογισμό του ίδιου τύπου ορίου, 
όπως θα δούμε στη συνέχεια. Αυτό το όριο, ονομάζεται \textbf{παράγωγος} και ερμηνεύεται 
ως ρυθμός μεταβολής κάποιου μεγέθους.

\subsection*{Εφαπτομένη Καμπύλης}

Για τον υπολογισμό της εφαπτομένης μιας καμπύλης με εξίσωση $ y=f(x) $, στο 
σημείο $P( x_{0}, f(x_{0})) $, θεωρούμε ένα γειτονικό σημείο $ Q(x,f(x)) $, με 
$ x \neq x_{0} $ και υπολογίζουμε την κλίση της τέμνουσας $ PQ $

\twocolumnsiderr{
  \begin{tikzpicture}[scale=0.8]
    \draw[-stealth,axis] (-0.5,0) -- (5,0) node[below] (x) {\smaller$x$} ;
    \draw[-stealth,axis] (0,-0.5) -- (0,3.7) node[left] (y) {\smaller$y$} ;
    \coordinate (o) at (0,0) ;
    \begin{scope}[yshift=-15pt]
      \draw[graph,tangent=0.28,Col1] (0.2,0.0) to[out=80, in=180] (2.2,3.0) coordinate 
        (q) to[out=0,in=-180] node[above=20pt,pos=0.7,xshift=7pt] {$y=f(x)$} (5,0.7) ;
      \fill[use tangent] (0,0) coordinate (p) circle (1.5pt);
      \fill (p) node[above,xshift=-2pt] {$P$} circle (1.5pt);
      \fill (q) node[above] {$Q$} circle (1.5pt);
      \draw[dashed] (o|-p) node[left,Col2] {\smaller$f(x_{0})$} -- (p) -- (o-|p)
        node[below,Col2] {\smaller$x_{0}$} ;
      \draw (p) -- (q) ;
      \draw[dashed] (o|-q) node[left,Col2] {\smaller$f(x)$} -- (q) -- (o-|q) 
        node[below,Col2] {\smaller$x$} ;
      \draw (p) -- (q) ;
      \coordinate (a) at (p-|q) ; 
      \draw (p) -- (a) node[below=3pt,midway,Col2] {\smaller$x- x_{0}$} ;
      \draw[decorate,decoration={brace,amplitude=3pt,raise=3pt,mirror},Col2] (p) -- (a) ;
      \draw (a) -- (q) node[midway,xshift=5pt,yshift=-2pt,pin={[pin
        edge={black,latex-},Col2]75:\smaller$f(x)-f(x_{0})$}]{} ;
      \draw[decorate,decoration={brace,amplitude=3pt,raise=3pt,mirror},Col2] (a) -- (q) ;
    \end{scope}
  \end{tikzpicture}
}{
  \[
    \lambda _{PQ} = \frac{f(x)-f(x_{0})}{x- x_{0}} 
  \] 
  Στη συνέχεια, επιτρέπουμε στο σημείο $Q$ να πλησιάσει το $P$, κατά μήκος της 
  καμπύλης, επιτρέποντας στο $x$ να πλησιάσει το $ x_{0} $. Αν η κλίση $\lambda_{PQ}$ 
  πλησιάζει κάποιον αριθμό $ \lambda $, τότε ορίζουμε την εφαπτομένη στο σημείο $P$ να 
  είναι η ευθεία που διέρχεται από το $P$ με κλίση $ \lambda $. Με άλλα λόγια, η 
  εφαπτομένη της καμπύλης στο σημείο $P$, είναι το όριο της κλίσης της τέμνουσας $ PQ $, 
  καθώς το $Q$ τείνει στο $P$.
}

\vspace{\baselineskip}

\twocolumnsiderr{
  \begin{tikzpicture}[scale=0.8]
    \draw[-stealth,axis] (-0.5,0) -- (5,0) node[below] (x) {\smaller$x$} ;
    \draw[-stealth,axis] (0,-0.5) -- (0,3.7) node[left] (y) {\smaller$y$} ;
    \coordinate (o) at (0,0) ;
    \begin{scope}[yshift=-15pt]
      \draw[thick,tangent=0.28,Col2] (0.2,0.0) to[out=80, in=180] (2.2,3.0) coordinate 
        (q) to[out=0,in=-180] coordinate[pos=0.23] (q1) coordinate[pos=0.4] (q2) (5,0.7) ;
      \fill[use tangent] (0,0) coordinate (p) circle (1.5pt);
      \fill (p) node[above,xshift=-2pt] {$P$} circle (1.5pt);
      \fill (q) node[above=3pt] {\smaller$Q$} circle (1.5pt);
      \fill (q1) node[above right] {\smaller$Q$} circle (1.5pt);
      \fill (q2) node[above right] {\smaller$Q$} circle (1.5pt);
      \draw[very thick,use tangent,Col1] (-1.8,0) -- (2,0) ;
      \draw[shorten <=-1.3cm,blue!50] (p) -- ($ (p)!4cm!(q) $) ;
      \draw[shorten <=-1.3cm,blue!50] (p) -- ($ (p)!4cm!(q1) $) ;
      \draw[shorten <=-1.3cm,blue!50] (p) -- ($ (p)!4cm!(q2) $) ;
      \node  at (q) [above] (f) {} ;
      \node  at (p) [above right=20pt,yshift=4pt] (g) {} ;
      \node  at (q1) [above] (a) {} ;
      \node  at (q1) [right,xshift=1pt] (d) {} ;
      \node at (q) [above right,xshift=3pt,yshift=-1pt] (b) {} ;
      \node  at (q2) [above,xshift=1pt] (c) {} ;
      \node  at (q2) [right] (e) {} ;
      \draw (a.center) edge[-stealth] (b.center) ;
      \draw (c.center) edge[-stealth] (d.center) ;
      \draw (f.center) edge[-stealth] (g.center) ;
      \node at (o|-p) [left,Col2]{\phantom{\smaller$f(x_{0})$}} ;
      \draw[dashed] (p) -- (o-|p) node[below,Col2] {\smaller$x_{0}$} ;
      \draw[dashed] (q2) -- (o-|q2) node[below,Col2] {\smaller$x$} ;
      \node at (o-|q2) [below left=2pt] (r1) {} ;
      \node at (o-|q1) [below left=2pt] (r2) {} ;
      \node at (o-|q)[below left=2pt] (r3) {} ;
      \node at (o-|p)[below right=2pt] (r4) {} ;
      \path[draw,Col2] (r1.center) edge[-stealth] (r2.center) ;
      \path[draw,Col2] (r2.center) edge[-stealth,xshift=-6pt] (r3.center) ;
      \path[draw,Col2] (r3.center) edge[-stealth,xshift=-6pt] (r4.east) ;
    \end{scope}
  \end{tikzpicture}
}{
  \begin{dfn}
    Η \textcolor{Col1}{εφαπτομένη} της καμπύλης $ y=f(x) $ στο σημείο 
    $ P(x_{0}, f(x_{0})) $, είναι η ευθεία που διέρχεται από το $P$ με κλίση 
    \[
      \lambda = \lim_{x \to x_{0}} \frac{f(x)-f(x_{0})}{x- x_{0}} 
    \]
  \end{dfn}
}



\subsection*{Ταχύτητα κινητού σώματος}

%todo να το ξαναγράψω... δες έννοιες, θεση, μετατοπιση, κλπ

Έστω, ένα σώμα το οποίο κινείται σε ευθεία γραμμή, σύμφωνα με την εξίσωση κίνησης, 
$ s=f(t) $, όπου $s$ είναι η απομάκρυνση του σώματος από την αρχή, τη χρονική στιγμή
$t$. Η συνάρτηση $f$ που περιγράφει την κίνηση του σώματος, λέγεται 
\textbf{συνάρτηση θέσης}. Κατά τη χρονική διάρκεια από $ t_{0} $ έως $ t $, η μεταβολή 
της θέσης του σώματος είναι $ f(t)-f(t_{0}) $ και η μέση ταχύτητα του σώματος είναι
\[
  \text{μέση ταχύτητα} = \frac{\text{μετατόπιση}}{\text{χρόνος}} = 
  \frac{f(t)-f(t_{0})}{t- t_{0}}  
\] 
που είναι ίση με την κλίση της τέμνουσας $ PQ $.

Στη συνέχεια, υποθέτουμε ότι υπολογίζουμε τη μέση ταχύτητα του σώματος, σε όλα και 
μικρότερα χρονικά διαστήματα. Με άλλα λόγια, επιτρέπουμε στο $t$ να πλησιάσει το 
$ t_{0} $, και ορίζουμε την ταχύτητα (ή στιγμιαία ταχύτητα) του σώματος, $ v(t_{0}) $, 
τη χρονική στιγμή $ t_{0} $ να είναι το όριο της μέσης ταχύτητας
\[
  v(t_{0}) = \lim_{t \to t_{0}} \frac{f(t)-f(t_{0})}{t- t_{0}}  
\] 
Αυτό σημαίνει, ότι η ταχύτητα του σώματος, τη χρονική στιγμή $ t_{0} $ είναι ίση με την 
κλίση της εφαπτομένης στο σημείο $P$.


\subsection*{Παράγωγος}

Είδαμε, ότι ο ίδιος τύπος ορίου, εμφανίζεται κατά τον υπολογισμό της εφαπτομένης 
καμπύλης και της ταχύτητας ενός σώματος. Πιο συγκεκριμένα, όρια της μορφής 
\[
  \lim_{x \to x_{0}} \frac{f(x)-f(x_{0})}{x- x_{0}} 
\] 
εμφανίζονται, κατά τον υπολογισμό του ρυθμού μεταβολής κάποιου μεγέθους.

\begin{dfn}
  Η παράγωγος μιας συνάρτησης $f$ στο σημείο $ x_{0} $, συμβολίζεται με $ f'(x_{0}) $ 
και ορίζεται να είναι το όριο
  \[
    f'(x_{0}) = \lim_{x \to x_{0}} \frac{f(x)-f(x_{0})}{x- x_{0}}
  \] 
  οποτεδήποτε αυτό το όριο υπάρχει.
\end{dfn}

\begin{rem}
  Αν θέσουμε $ x- x_{0}=h $ τότε έχουμε $ x = x_{0}+h $ και $ h \to 0 $ καθώς $ x \to
  x_{0} $, οπότε με αντικατάσταση, το παραπάνω όριο, γράφεται στη μορφή
  \[
    f'(x_{0}) = \lim_{h \to 0} \frac{f(x_{0}+h)-f(x_{0})}{h}
  \] 
\end{rem}


\end{document}




