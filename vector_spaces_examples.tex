\documentclass[a4paper,12pt]{article}
\usepackage{etex}
%%%%%%%%%%%%%%%%%%%%%%%%%%%%%%%%%%%%%%
% Babel language package
\usepackage[english,greek]{babel}
% Inputenc font encoding
\usepackage[utf8]{inputenc}
%%%%%%%%%%%%%%%%%%%%%%%%%%%%%%%%%%%%%%

%%%%% math packages %%%%%%%%%%%%%%%%%%
\usepackage{amsmath}
\usepackage{amssymb}
\usepackage{amsfonts}
\usepackage{amsthm}
\usepackage{proof}

\usepackage{physics}

%%%%%%% symbols packages %%%%%%%%%%%%%%
\usepackage{dsfont}
\usepackage{stmaryrd}
%%%%%%%%%%%%%%%%%%%%%%%%%%%%%%%%%%%%%%%


%%%%%% graphicx %%%%%%%%%%%%%%%%%%%%%%%
\usepackage{graphicx}
\usepackage{color}
%\usepackage{xypic}
\usepackage[all]{xy}
\usepackage{calc}
%%%%%%%%%%%%%%%%%%%%%%%%%%%%%%%%%%%%%%%

\usepackage{enumerate}

\usepackage{fancyhdr}
%%%%% header and footer rule %%%%%%%%%
\setlength{\headheight}{14pt}
\renewcommand{\headrulewidth}{0pt}
\renewcommand{\footrulewidth}{0pt}
\fancypagestyle{plain}{\fancyhf{}
\fancyhead{}
\lfoot{}
\rfoot{\small \thepage}}
\fancypagestyle{vangelis}{\fancyhf{}
\rhead{\small \leftmark}
\lhead{\small }
\lfoot{}
\rfoot{\small \thepage}}
%%%%%%%%%%%%%%%%%%%%%%%%%%%%%%%%%%%%%%%

\usepackage{hyperref}
\usepackage{url}
%%%%%%% hyperref settings %%%%%%%%%%%%
\hypersetup{pdfpagemode=UseOutlines,hidelinks,
bookmarksopen=true,
pdfdisplaydoctitle=true,
pdfstartview=Fit,
unicode=true,
pdfpagelayout=OneColumn,
}
%%%%%%%%%%%%%%%%%%%%%%%%%%%%%%%%%%%%%%



\usepackage{geometry}
\geometry{left=25.63mm,right=25.63mm,top=36.25mm,bottom=36.25mm,footskip=24.16mm,headsep=24.16mm}

%\usepackage[explicit]{titlesec}
%%%%%% titlesec settings %%%%%%%%%%%%%
%\titleformat{\chapter}[block]{\LARGE\sc\bfseries}{\thechapter.}{1ex}{#1}
%\titlespacing*{\chapter}{0cm}{0cm}{36pt}[0ex]
%\titleformat{\section}[block]{\Large\bfseries}{\thesection.}{1ex}{#1}
%\titlespacing*{\section}{0cm}{34.56pt}{17.28pt}[0ex]
%\titleformat{\subsection}[block]{\large\bfseries{\thesubsection.}{1ex}{#1}
%\titlespacing*{\subsection}{0pt}{28.80pt}{14.40pt}[0ex]
%%%%%%%%%%%%%%%%%%%%%%%%%%%%%%%%%%%%%%

%%%%%%%%% My Theorems %%%%%%%%%%%%%%%%%%
\newtheorem{thm}{Θεώρημα}[section]
\newtheorem{cor}[thm]{Πόρισμα}
\newtheorem{lem}[thm]{λήμμα}
\theoremstyle{definition}
\newtheorem{dfn}{Ορισμός}[section]
\newtheorem{dfns}[dfn]{Ορισμοί}
\theoremstyle{remark}
\newtheorem{remark}{Παρατήρηση}[section]
\newtheorem{remarks}[remark]{Παρατηρήσεις}
%%%%%%%%%%%%%%%%%%%%%%%%%%%%%%%%%%%%%%%




\newcommand{\vect}[2]{(#1_1,\ldots, #1_#2)}
%%%%%%% nesting newcommands $$$$$$$$$$$$$$$$$$$
\newcommand{\function}[1]{\newcommand{\nvec}[2]{#1(##1_1,\ldots, ##1_##2)}}

\newcommand{\linode}[2]{#1_n(x)#2^{(n)}+#1_{n-1}(x)#2^{(n-1)}+\cdots +#1_0(x)#2=g(x)}

\newcommand{\vecoffun}[3]{#1_0(#2),\ldots ,#1_#3(#2)}



% \pagestyle{vangelis}

\begin{document}

\chapter{Ασκήσεις--Παραδείγματα}

\section{Διανυσματικοί Χώροι}


%todo να γράψω λυσεις και να μπουν στις σημειωσεις μου ως παραδείγματα



\begin{exercise}
    Να δείξετε ότι το υποσύνολο $ W = \{(x,y,z)\in \mathbb{R}^{3} \; : 
    \; 2x-3y+z=0 \} $ είναι υπόχωρος του $ \mathbb{R}^{3} $.
\end{exercise}

\begin{solution}
\item {}
    Έστω 
    \begin{align*}
        \mathbf{w}_{1} = (x_{1}, y_{1}, z_{1}) \in W 
        \Rightarrow  2 x_{1} - 3 y_{1} + z_{1} = 0 \\
        \mathbf{w_{2}} = (x_{2}, y_{2}, z_{2}) \in W 
        \Rightarrow 2 x_{2} - 3 y_{2} + z_{2} = 0
    \end{align*} 
    Άρα 
    \[
        \mathbf{w_{1}}+ \mathbf{w_{2}} = (x_{1}, y_{1}, z_{1}) + 
        (x_{2}, y_{2}, z_{2}) = (x_{1}+ y_{1}, x_{2}+ y_{2}+ z_{1} + z_{2})
    \] 
    Εξετάζουμε αν $ \mathbf{w_{1}}+ \mathbf{w_{2}} \in W $. Έχουμε:
    \[
        2 (x_{1}+ y_{1}) -3 (x_{2}+ y_{2}) + (z_{1}+ z_{2}) = 
        \underbrace{2 x_{1} - 3 y_{1} + z_{1}}_{=0} + 
        \underbrace{2 x_{2}- 3 y_{2} + z_{2}}_{=0} = 0 + 0 = 0 
    \] 
\end{solution}

\begin{example}
  \begin{enumerate}[i)]
    \item 
  Να βρεθεί βάση και διάσταση του υπόχωρου $W$ του $ \mathbb{R}^{4} $ που παράγεται 
  από τα διανύσματα $ \mathbf{u_{1}} = (1,2,-2,3) $, $ \mathbf{u_{2}} = (5,6,-4,13) $ 
  και $ \mathbf{u_{3}} = (2,0,2,4) $. 
\item 
  Να επεκτείνετε τη βάση που θα βρείτε στο ερώτημα $ (i) $ σε βάση του $ \mathbb{R}^{4} $.
  \end{enumerate}

\end{example}

\end{document}
