\input{preamble.tex}
\newcommand{\vect}[2]{(#1_1,\ldots, #1_#2)}
%%%%%%% nesting newcommands $$$$$$$$$$$$$$$$$$$
\newcommand{\function}[1]{\newcommand{\nvec}[2]{#1(##1_1,\ldots, ##1_##2)}}

\newcommand{\linode}[2]{#1_n(x)#2^{(n)}+#1_{n-1}(x)#2^{(n-1)}+\cdots +#1_0(x)#2=g(x)}

\newcommand{\vecoffun}[3]{#1_0(#2),\ldots ,#1_#3(#2)}



% \pagestyle{vangelis}

\begin{document}

\chapter{φδσ}

\section{Διανυσματικοί Χώροι}


%todo να γράψω λυσεις και να μπουν στις σημειωσεις μου ως παραδείγματα



\begin{exercise}
    Να δείξετε ότι το υποσύνολο $ W = \{(x,y,z)\in \mathbb{R}^{3} \; : 
    \; 2x-3y+z=0 \} $ είναι υπόχωρος του $ \mathbb{R}^{3} $.
\end{exercise}

\begin{solution}
\item {}
    Έστω 
    \begin{align*}
        \mathbf{w}_{1} = (x_{1}, y_{1}, z_{1}) \in W 
        \Rightarrow  2 x_{1} - 3 y_{1} + z_{1} = 0 \\
        \mathbf{w_{2}} = (x_{2}, y_{2}, z_{2}) \in W 
        \Rightarrow 2 x_{2} - 3 y_{2} + z_{2} = 0
    \end{align*} 
    Άρα 
    \[
        \mathbf{w_{1}}+ \mathbf{w_{2}} = (x_{1}, y_{1}, z_{1}) + 
        (x_{2}, y_{2}, z_{2}) = (x_{1}+ y_{1}, x_{2}+ y_{2}+ z_{1} + z_{2})
    \] 
    Εξετάζουμε αν $ \mathbf{w_{1}}+ \mathbf{w_{2}} \in W $. Έχουμε:
    \[
        2 (x_{1}+ y_{1}) -3 (x_{2}+ y_{2}) + (z_{1}+ z_{2}) = 
        \underbrace{2 x_{1} - 3 y_{1} + z_{1}}_{=0} + 
        \underbrace{2 x_{2}- 3 y_{2} + z_{2}}_{=0} = 0 + 0 = 0 
    \] 
\end{solution}

\begin{example}
  \begin{enumerate}[i)]
    \item 
  Να βρεθεί βάση και διάσταση του υπόχωρου $W$ του $ \mathbb{R}^{4} $ που παράγεται 
  από τα διανύσματα $ \mathbf{u_{1}} = (1,2,-2,3) $, $ \mathbf{u_{2}} = (5,6,-4,13) $ 
  και $ \mathbf{u_{3}} = (2,0,2,4) $. 
\item 
  Να επεκτείνετε τη βάση που θα βρείτε στο ερώτημα $ (i) $ σε βάση του $ \mathbb{R}^{4} $.
  \end{enumerate}

\end{example}

\end{document}
