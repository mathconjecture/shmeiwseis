\input{preamble_typ.tex}
\input{definitions_typ.tex}

\geometry{left=12.63mm,right=12.63mm,top=30.25mm, bottom=36.25mm,footskip=24.16mm,
headsep=24.16mm}

\pagestyle{vangelis}
\everymath{\displaystyle}

\renewcommand{\qedsymbol}{}


\begin{document}

\chapter{Ομοιόμορφη Συνέχεια}

Έστω $A \subseteq \mathbb{R} $ και έστω $ f \colon A \to \mathbb{R} $.  

\section{Συνέχεια}

\begin{dfn}
  $f$ \textbf{συνεχής} στο $A \Leftrightarrow \textcolor{Col1}{\forall x_{0} \in A}, \; 
  \forall \varepsilon > 0, \; \exists \delta > 0, \; \forall x \in A \quad 
  \left[\strut\abs{x - x_{0}} < \delta \Rightarrow \abs{f(x)-f(x_{0})} < 
  \varepsilon\right] $ 
\end{dfn}

\begin{dfn}
  $f$ \textbf{όχι συνεχής} στο $A \Leftrightarrow 
  \textcolor{Col1}{\exists x_{0} \in A}, \; \exists
  \varepsilon > 0, \; \forall \delta > 0, \; \exists x \in A \quad \left[\strut\abs{x - 
  x_{0}} < \delta \Rightarrow \abs{f(x)-f(x_{0})} < \varepsilon\right] $ 
\end{dfn}

\section{Ομοιόμορφη Συνέχεια}

\begin{dfn}
  $f$ \textbf{ομοιόμορφα συν.} στο 
  $A \Leftrightarrow \forall \varepsilon > 0, \; 
  \exists \delta > 0, \; \textcolor{Col1}{\forall x_{0} \in A}, \; \forall x \in A 
  \quad \left[\strut\abs{x - x_{0}} < \delta \Rightarrow \abs{f(x)-f(x_{0})} < 
  \varepsilon\right] $ 
\end{dfn}

\begin{dfn}
  $f$ \textbf{όχι ομοιόμορφα συν.} στο $A \Leftrightarrow 
  \exists \varepsilon > 0, \forall \delta > 0, \; 
  \textcolor{Col1}{\exists x_{0} \in A}, \; \exists x \in A \quad 
  \left[\strut\abs{x - x_{0}} < \delta \Rightarrow \abs{f(x)- f(x_{0})} < 
  \varepsilon\right] $ 
\end{dfn}

Η τυπική διαφορά στους παραπάνω ορισμούς, είναι η σειρά των ποσοδεικτών. Ουσιαστικά όμως,  

\twocolumnsides{
  \section{Όταν αποδεικνύουμε συνέχεια στο A}
  \begin{myitemize}
    \item επιλέγουμε $ x_{0} \in A $
    \item Επιλέγουμε $ \varepsilon > 0 $
    \item Βρίσκουμε $ \delta = \delta (x_{0}, \varepsilon) $
    \item Επιλέγουμε $ x \in A $ με $ \abs{x- x_{0}} < \delta $ 
    \item αποδεικνύουμε ότι $ \abs{f(x) - f(x_{0})} < \varepsilon $.
  \end{myitemize}}{
  \section{Όταν αποδεικνύουμε ομοιόμορφη συνέχεια στο A} 
  \begin{myitemize}
    \item επιλέγουμε $ \varepsilon > 0 $
    \item Βρίσκουμε $ \delta = \delta (\varepsilon) $
    \item Επιλέγουμε $ x_{0} \in A $
    \item Επιλέγουμε $ x \in A $ με $ \abs{x- x_{0}} < \delta $ 
    \item αποδεικνύουμε ότι $ \abs{f(x) - f(x_{0})} < \varepsilon $.
  \end{myitemize}
}

\vspace{\baselineskip}

\section{Παρατηρήσεις}

Από την παραπάνω ανάλυση, είναι προφανές ότι μια ομοιόμορφα συνεχής συνάρτηση, θα είναι 
και συνεχής, γιατί αν βρούμε κάποιο $ \delta > 0 $ που δουλεύει για όλα τα $ x_{0} \in A
$, τότε μπορούμε να βρούμε $ \delta $ (το ίδιο $\delta$) που δουλεύει και για κάθε 
$ x_{0} \in A $ ξεχωριστά.

\vspace{\baselineskip}

\section{Παραδείγματα}

\begin{example}
  Έστω $ A = \mathbb{R} $ και $ f \colon A \to \mathbb{R} $ με $ f(x)=3x+7 $. Τότε 
  η $f$ είναι ομοιόμορφα συνεχής στο $A$.
\end{example}
\begin{proof}
  Έστω $ \varepsilon >0 $. $ \exists \delta = \frac{\varepsilon}{3} >0 $, ώστε 
  $ \forall x_{0} \in A $, και $ \forall x \in A $ με $ \abs{x - x_{0}} < \delta $ να 
  ισχύει
  \[
    \abs{3x+7-(3x_{0}+7)} = \abs{3(x-x_{0})} = 3 \abs{x- x_{0}} < 3\cdot 
    \frac{\varepsilon}{3} = \varepsilon
  \]
\end{proof}
\end{document}


