\input{preamble_ask.tex}
\input{definitions_ask.tex}

\pgfdeclarelayer{bg}
\pgfdeclarelayer{fg}
\pgfsetlayers{bg,main,fg}

\renewcommand{\qedsymbol}{}

\pagestyle{vangelis}


\begin{document}

\begin{center}
  \minibox{\large\bfseries \textcolor{Col1}{Αλλαγή Μεταβλητών στο Διπλό Ολοκλήρωμα}}
\end{center}

\vspace{\baselineskip}


\section*{Γενική Μεθοδολογία}

\begin{enumerate}
  \item Επιλέγουμε \textbf{κατάλληλες} εξισώσεις μετασχηματισμού ανάλογα με το σχήμα του 
    χωρίου 
    $D$. 
    \[
      \left.
        \begin{matrix}
          x=x(u,v) \\
          y=y(u,v)
        \end{matrix} 
      \right\} \quad \text{και} \quad J = \pdv{(x,y)}{(u,v)} = 
      \begin{vmatrix*}[r]
        x_{u} & x_{v} \\
        y_{u} & y_{v}
      \end{vmatrix*}
    \] 
  \item Μετασχηματίζουμε τη συνάρτηση $ f(x,y) $, θέτοντας όπου $ x=x(u,v) $ και όπου 
    $ y=y(u,v) $
    \[
      g(u,v) = f(x(u,v),y(u,v)) 
    \] 
  \item Μετασχηματίζουμε το χωρίο $D$, μετασχηματίζοντας συνήθως τις 
    \textbf{καμπύλες} που αποτελούν το \textcolor{Col1}{σύνορο} του $D$, 
    θέτοντας όπου $ x=x(u,v) $ και όπου $ y=y(u,v) $ και στη συνέχεια σχεδιάζουμε το 
    νέο χωρίο $ D^{*} $ στο επίπεδο $ uv $.  
  \item Χρησιμοποιούμε τον τύπο αλλαγής μεταβλητών για το διπλό ολοκλήρωμα:
    \[
      \iint_{D} f(x,y) \,{dx}{dy} = \iint_{D^{*}} g(u,v) \abs{\pdv{(x,y)}{(u,v)}}
      \,{du}{dv} 
    \] 
    όπου $ \abs{\pdv{(x,y)}{(u,v)}} $ είναι η \textbf{απόλυτη τιμή} της Ιακωβιανής 
    ορίζουσας του μετασχηματισμού.
\end{enumerate}

\section*{Επιλογή Κατάλληλου Μετασχηματισμού}

\begin{myitemize}[leftmargin=*]
  \item Όταν το χωρίο $D$ είναι \textbf{κυκλικός δίσκος}, ή μέρος κυκλικού δίσκου, τότε
    μετασχηματίζουμε σε \textcolor{Col1}{πολικές} συντεταγμένες:
    \[
      \left.
        \begin{matrix}
          x=r \cos{\theta} \\
          y=r \sin{\theta}
        \end{matrix} 
      \right\} \quad \text{και} \quad J = \pdv{(x,y)}{(r, \theta)} = 
      \begin{vmatrix*}[r]
        x_{r} & x_{\theta} \\
        y_{r} & y_{\theta} 
      \end{vmatrix*} = 
      \begin{vmatrix*}[r]
        \cos{\theta} & -r \sin{\theta} \\
        \sin{\theta} & r \cos{\theta}
      \end{vmatrix*} = r
    \] 
  \item Όταν το χωρίο $D$ είναι \textbf{ελλειπτικός δίσκος}, ή μέρος ελλειπτικού δίσκου, 
    τότε μετασχηματίζουμε σε \textcolor{Col1}{ελλειπτικές} συντεταγμένες:
    \[
      \left.
        \begin{matrix}
          x=ar \cos{\theta} \\
          y=br \sin{\theta}
        \end{matrix} 
      \right\} \quad \text{και} \quad J = \pdv{(x,y)}{(r, \theta)} =  
      \begin{vmatrix*}[r]
        x_{r} & x_{\theta} \\
        y_{r} & y_{\theta} 
      \end{vmatrix*} = 
      \begin{vmatrix*}[r]
        a\cos{\theta} & -ar \sin{\theta} \\
        b\sin{\theta} & br \cos{\theta}
      \end{vmatrix*} = ab\,r 
    \] 
    όπου $a$ και $b$ είναι οι άξονες της έλλειψης με εξίσωση
    $ \frac{x^{2}}{a^{2}} + \frac{y^{2}}{b^{2}} =1 $.

  \item Όταν μια παράσταση των $ x $ και $ y $ εμφανίζεται στην ολοκληρωτέα συνάρτηση 
    και ταυτόχρονα στο σύνορο του $D$, τότε τη θέτουμε ίση με $ u $ και συνήθως 
    θέτουμε $ v = x $ ή $ v=y $ ή οποιαδήποτε άλλη απλή παράσταση των $ x $ και 
    $ y $, με την προϋπόθεση ότι $ J \neq 0 $. 
  \item Όταν μια παράσταση των $ x $ και $ y $ εμφανίζεται δύο φορές στο σύνορο του $D$, 
    τότε τη θέτουμε ίση με $ u $ και συνήθως θέτουμε $ v = x $ ή $ v=y $ ή οποιαδήποτε 
    απλή παράσταση των $ x $ και $ y $ με την προϋπόθεση ότι $ J \neq 0 $. 
\end{myitemize}

\begin{rem}
  Οι ελλειπτικές συντεταγμένες, είναι \textbf{σύνθεση} ενός γραμμικού μετασχηματισμού 
  που μετασχηματίζει τον ελλειπτικό δίσκο σε κυκλικό και των πολικών συντεταγμένων, 
  όπως φαίνεται στο παρακάτω σχήμα.
\end{rem}

\vspace{\baselineskip}

\begin{tikzpicture}[scale=0.8]
  \begin{pgfonlayer}{bg}
    \draw[-latex,Col2] (-3,0) -- (3,0) node[right]{$x$} ;
    \draw[-latex,Col2] (0,-2) -- (0,2) node[left]{$y$} ;
    \draw[-latex,Col2] (6,0) -- (10,0) node [right]{$u$} ;
    \draw[-latex,Col2] (8,-2) -- (8,2) node [left]{$v$} ;
    \draw[-latex,Col2] (13,0) -- (17,0) node [right]{$r$} ;
    \draw[-latex,Col2] (15,-2) -- (15,2) node [left]{$\theta$} ;
  \end{pgfonlayer}
  \coordinate (el) at (1,1);
  \node[Col1] at (el) [above,xshift=15pt]
    {$\frac{x^{2}}{a^{2}}+\frac{y^{2}}{b^{2}}=1$} ;
  \coordinate (cl) at (9,1);
  \node[Col1] at (cl) [above,xshift=8pt] {\small$u^{2}+v^{2}=1$} ;
  \coordinate (o1) at (0, 0) {};
  \coordinate (o2) at (8, 0) {};
  \coordinate (o3) at (15, 0) {};
  \coordinate (a) at (2,0) ;
  \coordinate (b) at (0,1) ;
  \node at (a) [below right] {$a$} ;
  \node at (b) [above left] {$b$} ;
  \coordinate (u) at (10,0) ;
  \coordinate (v) at (8,2) ;
  \coordinate (c) at (9, 0) ;
  \node at (c) [below right] {$1$} ;
  \coordinate (r) at (16,0) ;
  \coordinate (th) at (15,1.4) ;
  \node  at (r) [below] {$1$};
  \node  at (th) [left] {$2 \pi$};
  %arrows
  \coordinate (8) at (3, 0.5) ;
  \coordinate (9) at (6, 0.5) ;
  \coordinate (18) at (10, 0.5) ;
  \coordinate (23) at (13, 0.5) ;
  \coordinate (n1) at (3, -0.5) ;
  \coordinate (n2) at (13, -0.5) ;
  %shapes and fillings
  \fill[blue!25,opacity=0.5] (o3) rectangle (th-|r) ;
  \draw[Col1,very thick] (o3) rectangle (th-|r) ;
  \fill[blue!25,opacity=0.5] (o1) ellipse (2 and 1);
  \draw[Col1,very thick] (o1) ellipse (2 and 1);
  \fill[blue!25,opacity=0.5] (o2) circle (1);
  \draw[Col1,very thick] (o2) circle (1);
  \draw [in=150, out=30, looseness=0.75,blue!50,thin] (8.center) edge[-latex] 
    node[blue!75,midway,above=13pt,xshift=6pt] {$\frac{x}{a}=u \Rightarrow x=au$} 
    node[blue!75,midway,above,xshift=5pt] {$\frac{y}{b}=v \Rightarrow y=bv$} (9.center) ;
  \draw [in=150, out=30, looseness=0.75,blue!50,thin] (18.center) edge[-latex] 
    node[blue!75,midway,above=10pt,xshift=5pt] {$u=r \cos{\theta}$} 
    node[blue!75,midway,above,xshift=5pt] {$v=r \sin{\theta}$} (23.center);
  \draw [in=-120, out=-60, looseness=0.75,blue!50,thin] (n1.center) edge[-latex] 
    node[blue!75,midway,below,xshift=5pt] {$x=ar \cos{\theta}$} 
    node[blue!75,midway,below=10pt,xshift=5pt] {$y=br \sin{\theta}$} (n2.center);
\end{tikzpicture}

\enlargethispage{\baselineskip}



\end{document}
