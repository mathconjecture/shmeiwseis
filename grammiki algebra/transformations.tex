\documentclass[a4paper,12pt]{article}
\usepackage{etex}
%%%%%%%%%%%%%%%%%%%%%%%%%%%%%%%%%%%%%%
% Babel language package
\usepackage[english,greek]{babel}
% Inputenc font encoding
\usepackage[utf8]{inputenc}
%%%%%%%%%%%%%%%%%%%%%%%%%%%%%%%%%%%%%%

%%%%% math packages %%%%%%%%%%%%%%%%%%
\usepackage{amsmath}
\usepackage{amssymb}
\usepackage{amsfonts}
\usepackage{amsthm}
\usepackage{proof}

\usepackage{physics}

%%%%%%% symbols packages %%%%%%%%%%%%%%
\usepackage{dsfont}
\usepackage{stmaryrd}
%%%%%%%%%%%%%%%%%%%%%%%%%%%%%%%%%%%%%%%


%%%%%% graphicx %%%%%%%%%%%%%%%%%%%%%%%
\usepackage{graphicx}
\usepackage{color}
%\usepackage{xypic}
\usepackage[all]{xy}
\usepackage{calc}
%%%%%%%%%%%%%%%%%%%%%%%%%%%%%%%%%%%%%%%

\usepackage{enumerate}

\usepackage{fancyhdr}
%%%%% header and footer rule %%%%%%%%%
\setlength{\headheight}{14pt}
\renewcommand{\headrulewidth}{0pt}
\renewcommand{\footrulewidth}{0pt}
\fancypagestyle{plain}{\fancyhf{}
\fancyhead{}
\lfoot{}
\rfoot{\small \thepage}}
\fancypagestyle{vangelis}{\fancyhf{}
\rhead{\small \leftmark}
\lhead{\small }
\lfoot{}
\rfoot{\small \thepage}}
%%%%%%%%%%%%%%%%%%%%%%%%%%%%%%%%%%%%%%%

\usepackage{hyperref}
\usepackage{url}
%%%%%%% hyperref settings %%%%%%%%%%%%
\hypersetup{pdfpagemode=UseOutlines,hidelinks,
bookmarksopen=true,
pdfdisplaydoctitle=true,
pdfstartview=Fit,
unicode=true,
pdfpagelayout=OneColumn,
}
%%%%%%%%%%%%%%%%%%%%%%%%%%%%%%%%%%%%%%



\usepackage{geometry}
\geometry{left=25.63mm,right=25.63mm,top=36.25mm,bottom=36.25mm,footskip=24.16mm,headsep=24.16mm}

%\usepackage[explicit]{titlesec}
%%%%%% titlesec settings %%%%%%%%%%%%%
%\titleformat{\chapter}[block]{\LARGE\sc\bfseries}{\thechapter.}{1ex}{#1}
%\titlespacing*{\chapter}{0cm}{0cm}{36pt}[0ex]
%\titleformat{\section}[block]{\Large\bfseries}{\thesection.}{1ex}{#1}
%\titlespacing*{\section}{0cm}{34.56pt}{17.28pt}[0ex]
%\titleformat{\subsection}[block]{\large\bfseries{\thesubsection.}{1ex}{#1}
%\titlespacing*{\subsection}{0pt}{28.80pt}{14.40pt}[0ex]
%%%%%%%%%%%%%%%%%%%%%%%%%%%%%%%%%%%%%%

%%%%%%%%% My Theorems %%%%%%%%%%%%%%%%%%
\newtheorem{thm}{Θεώρημα}[section]
\newtheorem{cor}[thm]{Πόρισμα}
\newtheorem{lem}[thm]{λήμμα}
\theoremstyle{definition}
\newtheorem{dfn}{Ορισμός}[section]
\newtheorem{dfns}[dfn]{Ορισμοί}
\theoremstyle{remark}
\newtheorem{remark}{Παρατήρηση}[section]
\newtheorem{remarks}[remark]{Παρατηρήσεις}
%%%%%%%%%%%%%%%%%%%%%%%%%%%%%%%%%%%%%%%




\newcommand{\vect}[2]{(#1_1,\ldots, #1_#2)}
%%%%%%% nesting newcommands $$$$$$$$$$$$$$$$$$$
\newcommand{\function}[1]{\newcommand{\nvec}[2]{#1(##1_1,\ldots, ##1_##2)}}

\newcommand{\linode}[2]{#1_n(x)#2^{(n)}+#1_{n-1}(x)#2^{(n-1)}+\cdots +#1_0(x)#2=g(x)}

\newcommand{\vecoffun}[3]{#1_0(#2),\ldots ,#1_#3(#2)}





\pagestyle{vangelis}


\begin{document}

\chapter{Γραμμικοί Μετασχηματισμοί}

\section{Ορισμός--Παραδείγματα}


\begin{dfn}
  Έστω $ (V, \oplus _{V}, \odot _{V}) $ και $ (W, \oplus _{W}, \odot _{W}) $ δύο 
  $ \mathbb{K} - $χώροι. Μια συνάρτηση $ T \colon V \to W $ λέγεται 
  \textcolor{Col1}{γραμμικός μετασχηματισμός} αν ικανοποιεί τις παρακάτω συνθήκες:
  \begin{enumerate}[(i)]
    \item $ T(\mathbf{x} \oplus _{V} \mathbf{y}) = T(\mathbf{x}) \oplus _{W}
      T(\mathbf{y}), \quad \forall \mathbf{x}, \mathbf{y}  \in V $
    \item $ T(\lambda \odot _{V} \mathbf{x}) = \lambda \odot _{W} T(\mathbf{x}), 
      \quad \forall \lambda \in \mathbb{K} $ και $ \forall \mathbf{x} \in V $
  \end{enumerate}
  Αν $T$ είναι ένας γραμμικός μετασχηματισμός από τον χώρο $V$ στον $V$ τότε λέμε 
  ότι ο $T$ είναι ένας \textcolor{Col1}{γραμμικός τελεστής} επί του $V$. 
\end{dfn}

\begin{rem}
\item {}
  \begin{myitemize}
    \item Συνήθως, αν δεν υπάρχει σύγχυση, γράφουμε απλώς 
          $ T(\mathbf{x}+ \mathbf{y}) = T(\mathbf{x}) + T(\mathbf{y}) $ και 
          $ T(\lambda \mathbf{x}) = \lambda T(\mathbf{x}) $
    \item Προφανώς $T$ γραμμικός $ \Leftrightarrow  T(k \mathbf{x} + \lambda 
      \mathbf{y}) =k T(\mathbf{x}) + \lambda T(\mathbf{y}) $
  \end{myitemize}
\end{rem}

\begin{rem}
    Από τη 2η σχέση του ορισμού, προκύπτει ότι
      \begin{center}
        αν $ \lambda = 0 $, τότε $ T(\mathbf{0}_{V}) = \mathbf{0}_{W} $ \\
        αν $ \lambda = -1 $, τότε $T(- \mathbf{x}) = - T(\mathbf{x}) $
      \end{center}
      Επομένως αν για μια συνάρτηση δεν ισχύει κάποια από τις παραπανω σχέσεις, τότε 
      δεν είναι γραμμική.
\end{rem}

% \begin{rem}
%   Αν $ T \colon V \to W $ είναι ένας γραμμικός μετασχηματισμός, τότε για κάθε 
%   $ n \in \mathbb{N} $, $ \mathbf{x}_{i} \in V $ και $ \lambda _{i} \in \mathbb{K} $, 
%   $ i = 1,\ldots, n $ ισχύει:
%   \[
%     T\left(\sum_{i=1}^{n} \lambda _{i} \mathbf{x}_{i}\right) = \sum_{i=1}^{n} 
%     \lambda _{i} T(\mathbf{x}_{i}) 
%   \] 
% \end{rem}

\begin{example}[Ταυτοτικός]
  Έστω $V$ ένας $ \mathbb{K} - $χώρος και έστω $ I_{V} \colon V \to V $ η ταυτοτική 
  συνάρτηση επί του $V$ με 
  \[
    I_{V}(\mathbf{x}) = \mathbf{x}, \quad \mathbf{x} \in V   
  \]
  Τότε η $I_{V}$ είναι γραμμικός μετασχηματισμός και λέγεται 
  \textcolor{Col1}{ταυτοτικός} μετασχηματισμός επί του $V$. 
\end{example}

\begin{example}[Μηδενικός]
  Έστω $V$ και $W$ δυο $ \mathbb{K}- $χώροι και έστω $ T_{0} \colon V \to W $ με 
  \[
    T_{0}(\mathbf{x}) = \mathbf{0}_{W}, \quad \forall \mathbf{x} \in V
  \]
  Τότε η $T_{0}$ είναι γραμμικός μετασχηματισμός και λέγεται 
  \textcolor{Col1}{μηδενικός} μετασχηματισμός επί του $V$.
\end{example}

\begin{example}[Προβολή επί του άξονα $x$]
  Έστω ο $ \mathbb{R}- $χώρος $ \mathbb{R}^{2} $ με τις συνήθεις πράξεις και έστω 
  $ T \colon \mathbb{R}^{2} \to \mathbb{R}^{2} $ με 
  \[
    T(\mathbf{x}, \mathbf{y}) = (x, 0) 
  \] 
  Τότε η $T$ είναι γραμμικός μετασχηματισμός και λέγεται \textcolor{Col1}{προβολή} επί 
  του άξονα $x$.
\end{example}

\begin{example}[Συμμετρία ως προς άξονα $x$]
  Έστω ο $ \mathbb{R}- $χώρος $ \mathbb{R}^{2} $ με τις συνήθεις πράξεις και έστω 
  $ T \colon \mathbb{R}^{2} \to \mathbb{R}^{2} $ με 
  \[
    T(\mathbf{x}, \mathbf{y}) = (x, -y) 
  \] 
  Τότε η $T$ είναι γραμμικός μετασχηματισμός και λέγεται \textcolor{Col1}{συμμετρία} ως 
  προς τον άξονα $x$.
\end{example}

\begin{example}
  Έστω ο $ \mathbb{R}- $χώρος $ \mathbb{R}^{2} $ με τις συνήθεις πράξεις και έστω 
  $ T \colon \mathbb{R}^{2} \to \mathbb{R}^{2} $ με 
  \[
    T(\mathbf{x}, \mathbf{y}) = (x,2x+y) 
  \] 
  Τότε η $T$ είναι γραμμικός μετασχηματισμός, αφού, αν $ \mathbf{x}=(x_{1}, y_{1}) $ 
  και $ \mathbf{y}=(x_{2}, y_{2}) $, τότε \[ \mathbf{x}+ \mathbf{y} = 
    (x_{1}+ x_{2}, y_{1}+ y_{2}) \quad \text{και} \quad  \lambda \mathbf{x} = 
  (\lambda x_{1}, \lambda x_{2}) \] και έχουμε:
\[
     \begin{aligned}
      T(\mathbf{x}+ \mathbf{y}) &= T(x_{1}+ x_{2}, y_{1}+ y_{2}) \\
                                &= (x_{1}+ x_{2}, 2(x_{1}+ x_{2})+ y_{1}+ y_{2}) \\
                                &= (x_{1}, 2 x_{1}+ y_{1}) + (x_{2}, 2 x_{2}+ y_{2}) \\
                                &= T(\mathbf{x}) + T(\mathbf{y}) 
    \end{aligned}
    \qquad \text{και} \qquad
    \begin{aligned}
      T(\lambda \mathbf{x}) &= T(\lambda x_{1}, \lambda y_{1}) \\
                            &= (\lambda x_{1}, 2 \lambda x_{1}+ \lambda y_{1}) \\
                            &= \lambda (x_{1}, 2 x_{1}+ y_{1}) \\
                            &= \lambda T(\mathbf{x})
    \end{aligned}
  \]
\end{example}

\begin{example}
  Έστω ο $ \mathbb{R}- $χώρος $ \mathbb{R}^{2} $ με τις συνήθεις πράξεις και έστω 
  $ T \colon \mathbb{R}^{2} \to \mathbb{R}^{2} $ με 
  \[
    T(\mathbf{x}, \mathbf{y}) = (x+1, y) 
  \] 
  Τότε η $T$ δεν είναι γραμμικός μετασχηματισμός, αφού $ T(0,0) = (1,0) \neq (0,0) $ 
\end{example}

\begin{example}[Αριστερόστροφη Περιστροφή]
  Έστω $ \theta \in [0,2 \pi) $. Ορίζουμε $ T_{\theta} \colon \mathbb{R}^{2} \to
  \mathbb{R}^{2}  $ με 
  \[
    T_{\theta}(x,y) = (x \cos{\theta} - y \sin{\theta}, x \sin{\theta}, y \cos{\theta}) 
  \] 
  Τότε η $ T_{\theta} $ είναι γραμμικός μετασχηματισμός και γεωμετρικά στρέφει ένα 
  διάνυσμα $ \mathbf{u} \in \mathbb{R}^{2} $ κατά γωνία $\theta$ με αντι-ωρολογιακή 
  φορά.
\end{example}

\begin{rem}[Εξήγηση για τον τύπο της $\theta$]
  Έστω $\theta \in [0, 2 \pi)$, και έστω τα διανύσματα βάσης του $ \mathbb{R}^{2} $ 
  $ \mathbf{e_{1}} = (1,0) $ και $ \mathbf{e_{2}} = (0,1) $ (όπως θα δούμε παρακάτω μια 
  γραμμική συνάρτηση ορίζεται στα στοιχεία μιας βάσης). Τότε έχουμε:
  \begin{align*}
    T_{\theta}(\mathbf{e_{1}}) &= T_{\theta}(1,0) = (\cos{\theta}, \sin{\theta}) \\
    T_{\theta}(\mathbf{e_{2}}) &= T_{\theta}(0,1) = (- \sin{\theta} , \cos{\theta})
  \end{align*}
  Έστω, τώρα $ \mathbf{u} = (x,y) \in \mathbb{R}^{2} $. Τότε 
  \begin{align*}
    T_{\theta}(\mathbf{u}) = T_{\theta} (x,y) 
    &=  T_{\theta} (x \mathbf{e_{1}}+ y \mathbf{e_{2}}) \\
    &= xT_{\theta }(\mathbf{e_{1}}) + yT_{\theta }(\mathbf{e_{2}}) \\
    &= x (\cos{\theta}, \sin{\theta}) + y (- \sin{\theta} , \cos{\theta}) \\
    &= (x \cos{\theta} - y \sin{\theta} , x \sin{\theta} + y \cos{\theta}) \\
  \end{align*} 
\end{rem}

\begin{example}
  Έστω ο $ \mathbb{K}- $χώρος $ \textbf{M}_{m \times n}(\mathbb{K}) $. Ορίζουμε 
  $ T \colon \textbf{M}_{m \times n}(\mathbb{K}) \to 
  \textbf{M}_{m \times n}(\mathbb{K}) $ με 
  \[
    T(A) = A^{T}, \quad \forall A \in \textbf{M}_{m \times n}(\mathbb{K})
  \] 
  Τότε η $T$ είναι γραμμικός μετασχηματισμός, αφού
  \begin{align*}
    T(kA+ \lambda B) &= {(kA + \lambda B)}^{T} \\
                     &= {(kA)}^{T} + {(\lambda B)}^{T} \\
                     &= k A^{T} + \lambda B^{T} \\
                     &= k T(A) + \lambda T(B)
  \end{align*} 
\end{example}

\begin{example}
  Έστω ο $ \mathbb{R}- $χώρος $ \mathbb{R}^{2} $ και έστω $ T \colon \mathbb{R}^{2} \to
  \mathbb{R}^{2} $ με 
  \[
    T(x,y) = (\abs{x} , 0) 
  \] 
  Τότε η $T$ δεν είναι γραμμικός μετασχηματισμός, αφού 
  \[
    \left.
      \begin{aligned}
        T((-1)(1,0)) &= T(-1,0) = (\abs{(-1)},0) = (1,0) \\
        (-1) T(1,0) &= (-1) (\abs{1} , 0) = (-1) (1,0) = (-1,0)
      \end{aligned}
    \right\} \Rightarrow Τ((-1)(1,0)) \neq (-1)T(1,0)
  \]
\end{example}

\begin{example}
  Έστω οι $ \mathbb{R} - $χώροι $ \textbf{P}_{n}(\mathbb{R}) $ και $
  \textbf{P}_{n-1}(\mathbb{R}) $ και έστω $ T \colon \textbf{P}_{n}(\mathbb{R}) \to 
  \textbf{P}_{n-1}(\mathbb{R}) $ με 
  $ T(p) = p' $
  Τότε η $T$ είναι γραμμικός μετασχηματισμός, αφού 
  \[
    T(kp + \lambda q) = (kp+ \lambda q)' = kp' + \lambda q' = k T(p) + \lambda T(q)
  \] 
\end{example}

\begin{example}
  Έστω οι $ \mathbb{R} - $χώροι $ C(\mathbb{R}) $ και $ \mathbb{R} $, όπου 
  $ C(\mathbb{R}) $ είναι ο χώρος των συνεχών συναρτήσεων από το $ \mathbb{R} $ στο 
  $ \mathbb{R} $. Έστω $ a,b \in \mathbb{R} $ με $ a<b $. Ορίζουμε 
  $ T \colon C(\mathbb{R}) \to \mathbb{R}$ με 
  \[
    T(f) = \int _{a}^{b} f(x) \,{dx}, \quad \forall f \in C(\mathbb{R}) 
  \] 
  Τότε από τη γραμμική ιδιότητα του ορισμένου ολοκληρώματος, προκύπτει ότι 
  η $T$ είναι γραμμικός μετασχηματισμός.
\end{example}

\begin{prop}
  Έστω  $ T \colon E \to V $ και $ S \colon V \to W $ δυο γραμμικοί μετασχηματισμοί. 
  Τότε και η σύνθεση $ S \circ T $ είναι επίσης γραμμικός μετασχηματισμός.
\end{prop}
\begin{proof}
  Αν $ \mathbf{x}, \mathbf{y} \in V $ και $ \lambda \in \mathbb{K} $ τότε
  \begin{align*}
    (S \circ T) (\mathbf{x}+ \mathbf{y}) &= S(T(\mathbf{x}+ \mathbf{y})) =
    S(T(\mathbf{x})+T(\mathbf{y})) = S(T(\mathbf{x})) + S(T(\mathbf{y})) = 
    (S \circ T) (\mathbf{x}) + (S \circ T) (\mathbf{y}) \\
    (S \circ T)(\lambda \mathbf{x}) &= S(T(\lambda \mathbf{x})) = 
    S(\lambda T(\mathbf{x})) = \lambda S(T(\mathbf{x})) = \lambda (S \circ T) 
    (\mathbf{x})
  \end{align*}
\end{proof}

\begin{prop}
  Αν ο γραμμικός μετασχηματισμός $ T \colon V \to W $ αντιστρέφεται τότε ο αντίστροφος 
  μετασχηματισμός $ T^{-1} $ είναι επίσης γραμμικός.
\end{prop}
\begin{proof}
  Έστω $ \mathbf{a}, \mathbf{b} \in W $ και $ \lambda \in \mathbb{K} $. Τότε 
  $ \mathbf{a} + \mathbf{b}, \lambda \mathbf{a} \in W $ και 
  $ \mathbf{a}= T(\mathbf{x}) $, $ \mathbf{b} = T(\mathbf{y}) $, για κάποια
  $ \mathbf{x}, \mathbf{y} \in V $. 
  Έχουμε
  \begin{equation*}
    \mathbf{a}+ \mathbf{b} = T(\mathbf{x}) + T(\mathbf{y}) = T(\mathbf{x}+ \mathbf{y})
    \quad \text{και} \quad
    \lambda \mathbf{a} = \lambda T(\mathbf{x}) = T(\lambda \mathbf{x})
  \end{equation*}
  επομένως
  \begin{align*}
    T^{-1} (\mathbf{a}+ \mathbf{b}) &= \mathbf{x} + \mathbf{y} = T^{-1} (\mathbf{a}) 
    + T^{-1} (\mathbf{b}) \\
    T^{-1} (\lambda \mathbf{a}) &= \lambda \mathbf{x} = \lambda T^{-1} (\mathbf{a})
   \end{align*} 
\end{proof}


\section{Πυρήνας και Εικόνα Γραμμικού Μετασχηματισμού}

\begin{dfn}
  Έστω $V$, $K$ δυο $ \mathbb{K}- $χώροι και έστω $ T \colon V \to W $ ένας 
  γραμμικός μετασχηματισμός. Τότε 
  το σύνολο των στοιχείων $ \mathbf{x} \in V $, που απεικονίζονται  
  στο μηδέν του χώρου $W$ λέγεται \textcolor{Col1}{πυρήνας} του $T$ και συμβολίζεται με 
  $ \ker(T) $.  Το σύνολο $T(V)$ των εικόνων των στοιχείων του $V$ λέγεται 
  \textcolor{Col1}{εικόνα} του $T$ και συμβολίζεται με $ \im(T) $. Δηλαδή
  \begin{myitemize}
    \item $ \ker(T) = \{ \mathbf{x} \in V \; : \; T(\mathbf{x}) = \mathbf{0}_{W} \} $ 
    \item $ \im(T) = \{ T(\mathbf{x}) \; : \; \mathbf{x} \in V \} $ 
  \end{myitemize}
\end{dfn}

\begin{rem}
  Ισχύει ότι $ \mathbf{0}_{V} \in \ker(T) $, και $ \mathbf{0}_{W} \in \im(T)
  $, γιατί η $T$ είναι γραμμική και επομένως ισχύει ότι 
  $ T(\mathbf{0}_{V}) = \mathbf{0}_{W} $.
\end{rem}

\begin{example}
  Έστω $V$ και $W$ δυο $ \mathbb{K}- $χώροι. Τότε για τον ταυτοτικό μετασχηματισμό 
  ισχύει 
  \[
    \ker(I_{V}) = \{ \mathbf{0}_{V} \} \quad \text{και} \quad \im(I_{V}) = V 
  \]
  και για τον μηδενικό μετασχηματισμό ισχύει
  \[
    \ker(T_{0}) = V  \quad \text{και} \quad \im(T_{0}) = \{ \mathbf{0}_{W} \}   
  \] 
\end{example}

\begin{example}
  Έστω οι $ \mathbb{R}- $χώροι $ \mathbb{R}^{3} $ και $ \mathbb{R}^{2} $ και έστω 
  $ T \colon \mathbb{R}^{3} \to \mathbb{R}^{2} $ με $ T(x,y,z) = (x-y,2z) $. Η 
  $T$ είναι γραμμικός μετασχηματισμός (δείξτε το) και έχουμε:
  \begin{align*}
    \ker(T) &= \{(x,y,z)\in \mathbb{R}^{3} \; : \; T(x,y,z) = (0,0) \} \\
            &= \{(x,y,z)\in \mathbb{R}^{3} \; : \; (x-y,2z)=(0,0) \} \\
            &= \{(x,y,z)\in \mathbb{R}^{3} \; : \; x-y=0 \; \text{και} \; 2z=0 \} \\
            &= \{(x,y,z)\in \mathbb{R}^{3} \; : \; x=y \; \text{και} \; z=0 \} \\
            &= \{(x,x,0)\in \mathbb{R}^{3} \; : \; x \in \mathbb{R} \} \\
            &= < (1,1,0) >  
  \end{align*}
  Μία βάση για τον $ \ker(T) $ είναι η $ B = \{ (1,1,0) \} $, αφού το διάνυσμα 
  $ (1,1,0) $ είναι γραμμικώς ανεξάρτητο, ως μη μηδενικό. Μπορούμε να επεκτείνουμε 
  αυτή τη βάση, σε μια βάση για τον $ \mathbb{R}^{3} $, ως εξής:
  \[
    B_{\mathbb{R}^{3}} = \{ (1,1,0), (0,1,0), (0,0,1) \}  
  \]
  ενώ για την εικόνα της $T$ έχουμε, ότι άν  
  $ (a,b) \in \mathbb{R}^{2} $, τότε ζητάμε $ (x,y,z) \in \mathbb{R}^{3} $ ώστε 
  \begin{align*}
    T(x,y,z) &= (a,b) \Leftrightarrow \\ 
    (x-y,2z) &= (a,b) \Leftrightarrow \\
    x= y+a \quad &\text{και} \quad z = b/2
   \end{align*} 
   Επομένως, για κάθε $ (a,b) \in \mathbb{R}^{2} $, έχουμε ότι 
  \[
    T(2a,a, b/2)= (2a-a,2(b/2)) = (a,b)
  \] 
  'Αρα $ \im(T) = \mathbb{R}^{2} $
\end{example}

\begin{thm}
  Έστω $V$ και $W$  δύο  $ \mathbb{K}- $χώροι  και έστω  $ T \colon V \to W $  ένας 
  γραμμικός μετασχηματισμός. Τότε ισχύουν τα παρακάτω:
  \begin{enumerate}[(i)]
    \item Ο πυρήνας $ \ker(T)  $ του $T$ είναι υπόχωρος του $V$. 
    \item Η εικόνα $ \im(T)  $ του $T$ είναι υπόχωρος του $W$.
  \end{enumerate}
\end{thm}

\begin{proof}
\item {}
  \begin{description}
    \item [(i)] 
      Έστω, $ \mathbf{x}, \mathbf{y} \in \ker(T) $ και $ \lambda \in \mathbb{K} $. 
      Τότε $ T(\mathbf{x}) = T(\mathbf{y}) = \mathbf{0}_{W} $. 
      Έχουμε
      \begin{gather*}
        T(\mathbf{x}+ \mathbf{y}) = T(\mathbf{x}) + T(\mathbf{y}) = 
        \mathbf{0}_{W} + \mathbf{0}_{W} = \mathbf{0}_{W} \Rightarrow 
        \mathbf{x}+ \mathbf{y} \in \ker(T) \\
        T(\lambda \mathbf{x}) = \lambda T(\mathbf{x}) = \lambda \mathbf{0}_{W} = 
        \mathbf{0}_{W} \Rightarrow \lambda \mathbf{x} \in \ker(T) 
      \end{gather*}
    \item [(ii)] 
      Έστω, $ \mathbf{x}, \mathbf{y} \in \im(T) $ και $ \lambda \in \mathbb{K} $. 
      Τότε $ \mathbf{x} = T(\mathbf{a}) $ και $ \mathbf{y} = T(\mathbf{b}) $, με 
      $ \mathbf{a}, \mathbf{b} \in V $. Οπότε
      \begin{gather*}
        \mathbf{x} + \mathbf{y} = T(\mathbf{a}) + T(\mathbf{b}) = T(\mathbf{a}+
        \mathbf{b}) \in \im(T) \\
        \lambda \mathbf{x} = \lambda T(\mathbf{a}) = T(\lambda \mathbf{a}) \in \im(T)
      \end{gather*}
  \end{description}
\end{proof}

\begin{thm}\label{thm:imt}
  Έστω $V$ και $W$ δύο $ \mathbb{K}- $χώροι και έστω $ T \colon V \to W $ ένας 
  γραμμικός μετασχηματισμός. Αν $ B = \{ \mathbf{v_{1}}, \ldots, \mathbf{v_{n}} \} $ 
  είναι μια βάση για τον $V$, τότε
  \[
    \im(T) = < T(\mathbf{v_{1}}), \ldots, T(\mathbf{v_{n}}) >  
  \]
  δηλαδή, $ \im(T) = < T(B) > $.
\end{thm}

\begin{proof}
  Είναι προφανές ότι $ T(\mathbf{v}_{i}) \in \im(T), \quad \forall i = 1,\ldots, n $. 
  Αφού $ \im(T) \leq W $ και $ < T(\mathbf{v_{1}}), \ldots, T(\mathbf{v_{n}}) > $ είναι 
  ο μικρότερος υπόχωρος του $W$ που περιέχει τα διανύσματα $ T(\mathbf{v}_{i}), \quad i =
  1,\ldots, n$, έπεται ότι 
  \begin{equation}\label{eq:im1}
    < T(\mathbf{v_{1}}), \ldots, T(\mathbf{v_{n}}) > \subseteq \im(T)  
  \end{equation} 
  Έστω $ \mathbf{y} \in \im(T) $. Τότε $ \mathbf{y}= T(\mathbf{x}) $ για κάποιο $
  \mathbf{x} \in V $. Αφού το σύνολο $B$ είναι μια βάση του $V$, υπάρχουν $ x_{1}, 
  \ldots, x_{n} \in \mathbb{K} $, τέτοιοι ώστε
  \[
    \mathbf{x} = x_{1} \mathbf{v_{1}} + \cdots x_{n} \mathbf{v_{n}}
  \] 
  Οπότε, έχουμε
  \begin{align*}
    \mathbf{y} = T(\mathbf{x})  
    &= T(x_{1} \mathbf{v_{1}}+ x_{2} \mathbf{v_{2}}+ 
    \cdots + x_{n} \mathbf{v}_{n}) \\ 
    &= x_{1}T(\mathbf{v_{1}}) + x_{2}T(\mathbf{v_{2}}) + \cdots + x_{n} 
    T(\mathbf{v_{n}}) \in  < T(\mathbf{v_{1}}), \ldots, T(\mathbf{v_{n}}) >\\ 
  \end{align*}
  δηλαδή
  \begin{equation}\label{eq:im2} 
    \im(T) \subseteq < T(\mathbf{v_{1}}), \ldots, T(\mathbf{v_{n}}) > 
  \end{equation} 
  Οπότε από τις σχέσεις~\eqref{eq:im1} και~\eqref{eq:im2} έχουμε
  \[
    \im(T) = < T(\mathbf{v_{1}}), \ldots, T(\mathbf{v_{n}}) >  
  \]
\end{proof}

\begin{rem}
  Επομένως σύμφωνα με το θεώρημα~\ref{thm:imt}, ένας γραμμικός μετασχηματισμός 
  $ T \colon V \to W $ ορίζεται από τις τιμές που παίρνει στα στοιχεία μιας βάσης 
  $ B $ του χώρου $V$, με την έννοια ότι η εικόνα του τυχαίου στοιχείου 
  $ \mathbf{v} \in V $ είναι γραμμικός συνδυασμός των εικόνων των στοιχείων του 
  $B$ μέσω του μετασχηματισμού $T$. 
  Πράγματι, αν
  $ B = \{ \mathbf{e}_{1}, \ldots, \mathbf{e_{n}} \} $, 
  τότε για το τυχαίο $ \mathbf{v} \in V $ έχουμε ότι 
  $ \mathbf{v}= x_{1} \mathbf{e_{1}} + x_{2} \mathbf{e_{2}} +  
  \cdots + x_{n} \mathbf{e_{n}} $ με $ x_{i} \in \mathbb{K} $. Συνεπώς
  \begin{align*}
    T(\mathbf{v}) & = T(x_{1} \mathbf{e_{1}}+ x_{2} \mathbf{e_{2}}+ \cdots + x_{n}
    \mathbf{e}_{n}) \\ 
                  &= x_{1}T(\mathbf{e_{1}}) + x_{2}T(\mathbf{e_{2}}) + \cdots + x_{n} 
                  T(\mathbf{e_{n}})
  \end{align*}
  όπου τα $ T(\mathbf{e}_{i}) \in W $, είναι οι τιμές του $T$ στα 
  στοιχεία της βάσης $B$.
\end{rem}

\begin{rem}
  Αν ο $V$ δεν έχει πεπερασμένη διάσταση, τότε το στοιχείο $ \mathbf{v} \in V $ 
  θα γράφεται ως γραμμικός συνδυασμός πεπερασμένου πλήθους στοιχειών της βάσης του, 
  έστω $ \mathbf{e_{1}}, \mathbf{e_{2}}, \ldots, \mathbf{e_{n}} $ και η απόδειξη 
  συνεχίζει όπως παραπάνω.
\end{rem}

\begin{example}
  Έστω οι $ \mathbb{R} - $χώροι $ \textbf{P}_{2}(\mathbb{R}) $ και 
  $ \textbf{M}_{2}(\mathbb{R}) $. Ορίζουμε $ T \colon \textbf{P}_{2}(\mathbb{R}) \to
  \textbf{M}_{2}(\mathbb{R}) $ με 
  \[
    T(p) = 
    \begin{pmatrix*}
      p(1)-p(2) & 0 \\
      0 & p(0)
    \end{pmatrix*}, \quad p \in \textbf{P}_{2}(\mathbb{R})
  \] 
  Η $T$ είναι γραμμικός μετασχηματισμός (δείξτε το) και για να βρούμε βάση και 
  τη διάσταση του 
  $ \im(T) $, θεωρούμε τη βάση $ B = \{ 1,x,x^{2} \} $ του $
  \textbf{P}_{2}(\mathbb{R}) $. Τότε από το προηγούμενο θεώρημα έχουμε
  \begin{align*}
    \im(T) = 
    &= < T(1), T(x), T(x^{2}) > \\
    &= < 
    \begin{pmatrix*}[r]
      0 & 0 \\
      0 & 1
    \end{pmatrix*}, 
    \begin{pmatrix*}[r]
      -1 & 0 \\
      0 & 0 
    \end{pmatrix*}, 
    \begin{pmatrix*}[r]
      -3 & 0 \\
      0 & 0
    \end{pmatrix*}
    > \\
    & = 
    < 
    \begin{pmatrix*}[r]
      0 & 0 \\
      0 & 1
    \end{pmatrix*}
    , 
    \begin{pmatrix*}[r]
      -1 & 0 \\
      0 & 0
    \end{pmatrix*}>
  \end{align*}
  γιατί προφανώς $ 
  \begin{pmatrix*}[r]
    -3 & 0 \\
    0 & 0
  \end{pmatrix*} = -3 
  \begin{pmatrix*}[r]
    -1 & 0 \\
    0 & 0 
  \end{pmatrix*}
  $. Οπότε το σύνολο $ <  
  \begin{pmatrix*}[r]
    0 & 0 \\
    0 & 1
  \end{pmatrix*}
  , 
  \begin{pmatrix*}[r]
    -1 & 0 \\
    0 & 0
  \end{pmatrix*}
  > $ 
  παράγει το χώρο $ \im(T) $ και επειδή τα στοιχεία του είναι 
  προφανώς γραμμικώς ανεξάρτητα, αποτελούν βάση του $ \im(T) $.
  Άρα $ \dim \im(T) = 2 $
\end{example}

\begin{rem}
  Το προηγούμενο παράδειγμα δείχνει ότι αν $ B = \{ \mathbf{v_{1}}, \ldots, 
  \mathbf{v_{n}} \} $ είναι μια βάση για έναν $ \mathbb{K}- $χώρο και 
  $ T \colon V \to W $ είναι ένας γραμμικός μετασχηματισμός, τότε δεν ισχύει 
  απαραίτητα ότι και το σύνολο $ T(B) = \{ T(\mathbf{v_{1}}), 
  \ldots, T(\mathbf{v_{n}}) \}$ είναι μια βάση για τον $ \im(T) $.
\end{rem}

\begin{prop}
  Το σύνολο $ T(B) $ θα είναι μια βάση για τον $ \im(T) $ αν επιπλέον η $T$ είναι 
  $ 1-1 $.
\end{prop}

\begin{proof}
  Πράγματι, έστω ότι ο $T$ είναι $ 1-1 $. Για να δείξουμε ότι το σύνολο $T(B) $ είναι 
  βάση, αρκεί να δείξουμε ότι το σύνολο $ T(B) $ είναι 
  γραμμικώς ανεξάρτητο, αφού το ότι παράγει τον $ \im(T) $, είναι ακριβώς το 
  αποτέλεσμα του προηγούμενου θεωρήματος. Επομένως, έστω $ \lambda _{1}, \ldots, 
  \lambda _{n} \in \mathbb{K} $ τέτοια ώστε 
  \begin{align*}
    \lambda _{1} T(\mathbf{v}_{1}) + \cdots, \lambda _{n} T(\mathbf{v_{n}}) &= 
    \mathbf{0}_{W} \Rightarrow \\
    T(\lambda _{1} \mathbf{v}_{1} + \cdots, \lambda _{n} \mathbf{v_{n}}) &=
    T(\mathbf{0}_{V}) \overset{1-1}{\Rightarrow} \\
    \lambda _{1} \mathbf{v}_{1} + \cdots, \lambda _{n} \mathbf{v_{n}} &= \mathbf{0}_{V} 
    \overset{\text{Β \; Βάση}}{\Rightarrow} \\ 
    \lambda _{1} = \cdots = \lambda _{n} &= 0
  \end{align*} 
\end{proof}

\begin{dfn}
  Έστω $V$ και $W$ δύο $ \mathbb{K} - $χώροι και έστω $ T \colon V \to W $ ένας 
  γραμμικός μετασχηματισμός. Αν οι υπόχωροι $ \ker(T) $ και $ \im(T) $ των $V$ και $W$ 
  αντίστοιχα είναι πεπερασμένης διάστασης, τότε ορίζουμε 
  \begin{myitemize}
    \item την \textcolor{Col1}{μηδενικότητα} του $T$ να είναι η διάσταση 
      $ \dim \ker(T) $ του πυρήνα του μετασχηματισμού $T$.
    \item τον \textcolor{Col1}{βαθμό} του $T$ να είναι η διάσταση $ \dim \im(T) $ της 
     εικόνας του μετασχηματισμού $ T $.
  \end{myitemize}
\end{dfn}

\begin{rem}
\item {}
  \begin{myitemize}
    \item Τη μηδενικότητα ενός γραμμικού μετασχηματισμού $T$, συνήθως τη συμβολίζουμε με 
      $ \Null(T) $ ή απλώς $ \n(T) $. Πολλές φορές θα χρησιμοποιούμε και το 
      $ \dim \ker(T) $.
    \item Το βαθμό ενός γραμμικού μετασχηματισμού $T$, συνήθως τον συμβολίζουμε με 
      $ \rank(T) $ ή απλώς $ \R(T) $.
  \end{myitemize}
\end{rem}

\begin{thm}[Θεώρημα Διάστασης]
  Έστω $ V $ και $W$ δύο $ \mathbb{K}- $χώροι και έστω $ T \colon V \to W $ ένας 
  γραμμικός μετασχηματισμός. Αν ο $V$ είναι πεπερασμένης διάστασης, τότε 
  \begin{align*}
    \dim(\ker(T)) + \dim(\im(T)) = \dim(V) 
    \intertext{ή ισοδύναμα}
    \Null(T) + \rank(T) = \dim (V) 
  \end{align*}
\end{thm}

\begin{proof}
  Υποθέτουμε ότι $ \dim(V) = n $, $ n(T) = k $ και ότι $ \{ \mathbf{x_{1}},
  \mathbf{x_{2}}, \ldots, \mathbf{x}_{k} \} $ είναι μια βάση για τον $ \ker(T) $. 
  Επεκτείνουμε το $ \{ \mathbf{x_{1}}, \ldots, \mathbf{x}_{k} \} $ σε μια βάση 
  $ \beta = \{ \mathbf{x_{1}}, \ldots, \mathbf{x}_{k}, \mathbf{x}_{k+1}, \ldots,
  \mathbf{x}_{n} \} $ του $V$.

  Ισχυριζόμαστε ότι το σύνολο 
  $ S = \{ T(\mathbf{x}_{k+1}), \ldots, T(\mathbf{x}_{n}) \} $ είναι μια βάση για τον 
  $ \im(T) $.

  Καταρχάς αποδεικνύουμε ότι το $S$ παράγει τον $ \im(T) $. Εφόσον το σύνολο $ \beta $ 
  έιναι μια βάση για τον $V$, από προηγούμενο θεώρημα έχουμε ότι 
  \begin{align*}
    \im(T) = < T(\beta) >&= < T(\mathbf{x_{1}}), \ldots, T(\mathbf{x}_{k}),
    T(\mathbf{x}_{k+1}), \ldots, T(\mathbf{x}_{n}) > \\ 
                         &= < T(\mathbf{x}_{k+1}), \ldots, T(\mathbf{x}_{n}) > \\
                         &= < S > 
  \end{align*}
  αφού $ T(\mathbf{x}_{i}) = 0, \quad \forall i = 1,\ldots, k $, διότι $ \mathbf{x}_{i}
  \in \ker(T) $. 

  Άρα το $S$ παράγει τον $ \im(T) $. Τώρα αποδεικνύουμε ότι το $S$ είναι γραμμικά 
  ανεξάρτητο. Έστω $ \lambda _{k+1}, \ldots, \lambda _{n} \in \mathbb{K} $, τέτοια ώστε 
  \begin{align*} 
    \lambda _{k+1} T(\mathbf{x}_{k+1}) + \cdots + \lambda _{n} T(\mathbf{x}_{n}) =
    \mathbf{0}_{W} \Rightarrow \\
    T(\lambda _{k+1} \mathbf{x}_{k+1} + \cdots + \lambda _{n} \mathbf{x}_{n}) = 
    \mathbf{0}_{W} \Rightarrow \\
    \lambda _{k+1} \mathbf{x}_{k+1} + \cdots + \lambda _{n} \mathbf{x}_{n} \in \ker(T) 
  \end{align*} 
  Επειδή $ \{ \mathbf{x_{1}}, \dots, \mathbf{x}_{n} \} $ είναι μια βάση για τον 
  $ \ker(T) $, υπάρχουν $ \mu _{1}, \ldots, \mu _{k} \in \mathbb{K} $ τέτοια ώστε 
  το στοιχείο $ \lambda _{k+1} \mathbf{x}_{k+1} + \cdots + \lambda _{n} \mathbf{x}_{n} $ 
  να γράφεται ως γραμμικός συνδυασμός των στοιχείων την βάσης, δηλαδή
  \begin{align*}
    \lambda _{k+1} \mathbf{x}_{k+1} + \cdots + \lambda _{n} \mathbf{x}_{n} = \mu _{1} 
    \mathbf{x_{1}} + \cdot + \mu _{k} \mathbf{x}_{k} \Rightarrow \\
    \mu _{1} \mathbf{x_{1}} + \cdots + \mu _{k} \mathbf{x}_{k} + -\lambda _{1}
    \mathbf{x}_{1} - \cdots - \lambda _{n} \mathbf{x}_{n} = \mathbf{0}_{V} \Rightarrow \\
    \mu _{i} = 0, \quad \forall i = 1, \ldots, k \quad \text{και} \quad \lambda _{i} = 
    0, \quad \forall i = k+1, \ldots, n
   \end{align*} 
   διότι το σύνολο $ \beta = \{ \mathbf{x}_{1}, \ldots, \mathbf{x}_{k}, \mathbf{x}_{k+1},
   \ldots, \mathbf{x}_{n}\}  $ είναι μια βάση για τον $V$. 

   Άρα το $S$ είναι γραμμικά ανεξάρτητο. Παρατηρείτε ότι το επιχείρημα δείχνει ότι 
   τα $ T(\mathbf{x}_{k+1}), \ldots, T(\mathbf{x}_{n}) $ είναι διάφορα ανά δύο. 

   Άρα $ r(T) = \abs{S} = n-k = \dim(V) - n(T) \Rightarrow n(T) + r(T) = \dim(V) $.
\end{proof}

\begin{thm}
  Έστω $ V $ και $W$ δύο $ \mathbb{K}- $χώροι και έστω $ T \colon V \to W $ ένας 
  γραμμικός μετασχηματισμός. Τότε ισχύει ότι 
  \[
    T \quad \text{είναι} \quad 1-1 \Leftrightarrow \ker(T) = \{ \mathbf{0}_{V} \} 
  \] 
\end{thm}

\begin{proof}
\item {}
  \begin{description}
    \item [($ \Rightarrow $)] Έστω ότι $T$ είναι $ 1-1 $. Αν $ \mathbf{x} \in \ker(T)
      \Rightarrow T(\mathbf{x}) = \mathbf{0}_{W} \Rightarrow T(\mathbf{x}) =
      T(\mathbf{0}_{V}) \overset{1-1}{\Rightarrow} \mathbf{x} = \mathbf{0}_{v} $. 
      Επομένως, $ \ker(T ) = \{\mathbf{0}_{V} \} $.
    \item [($ \Leftarrow $)] Έστω ότι $ \ker(T) = \{ \mathbf{0}_{V} \} $ και έστω 
      $ \mathbf{x}, \mathbf{y} \in V $ με $ T(\mathbf{x}) = T(\mathbf{y}) $. Θα 
      δείξουμε ότι $ \mathbf{x} = \mathbf{y} $. Πράγματι, έχουμε
      \[
        \mathbf{0}_{W} = T(\mathbf{x}) - T(\mathbf{y}) = T(\mathbf{x} - \mathbf{y}) 
      \]
      άρα $ \mathbf{x} - \mathbf{y} \in \ker(T) \Rightarrow \mathbf{x}- \mathbf{y} =
      \mathbf{0}_{V} \Rightarrow \mathbf{x} = \mathbf{y} $.
  \end{description}
\end{proof}

\begin{thm}\zlabel{thm:3one}
  Έστω $ V $ και $W$ δύο $ \mathbb{K}- $χώροι και έστω $ T \colon V \to W $ ένας 
  γραμμικός μετασχηματισμός. Αν $ \dim(V) = \dim(W) = n < \infty $ τότε οι 
  παρακάτω προτάσεις είναι ισοδύναμες.
  \begin{enumerate}[(i)]
    \item $T$ είναι $ 1-1 $
    \item $T$ είναι επί
    \item $ \rank(T) = \dim(V) $
  \end{enumerate}
\end{thm}

\begin{proof}
  Από το θεώρημα διάστασης, έχουμε 
  \[
    n(T) + r(T) = \dim(V) 
  \] 
  Τώρα χρησιμοποιώντας το προηγούμενο θεώρημα, έχουμε ότι 
  \[
    T \quad 1-1 \Leftrightarrow \ker(T) = \{ \mathbf{0}_{V} \} \Leftrightarrow n(T) = 0 
    \Leftrightarrow r(T) = \dim(V) \Leftrightarrow r(T) = \dim(W) \Leftrightarrow
    \im(T) = W \Leftrightarrow T \quad \text{επί} 
   \] 
\end{proof}

\begin{rem}
  Άρα από το προηγούμενο θεώρημα συμπεραίνουμε ότι αν $ T $ γραμμικός μετασχηματισμός 
  μεταξύ ίδιων χώρων, τότε αν είναι $ 1-1 $ θα είναι και επί και αντιστρόφως, αν είναι 
  επί θα είναι και $ 1-1 $.
\end{rem}

\begin{thm}
  Έστω $V$ και $W$ δυο $ \mathbb{K}- $χώροι. Υποθέτουμε ότι το σύνολο 
  $ \{ \mathbf{v_{1}}, \ldots, \mathbf{v_{n}}\} $ είναι μια βάση για τον $V$. 
  Για οποιαδήποτε $ \mathbf{w_{1}}, \ldots \mathbf{w_{n}} $ στον $W$ υπάρχει 
  ακριβώς ένας γραμμικός μετασχηματισμός $ T \colon V \to W $ τέτοιος ώστε
  \[
    T(\mathbf{v}_{i}) = \mathbf{w}_{i}, \quad \forall i \in \{ 1,2, \ldots \}
  \] 
\end{thm}

\begin{proof}
%todo απόδειξη του θεωρήματος μοναδικού Τ
\end{proof}

\begin{cor}
  Έστω $V$ και $W$ δυο $ \mathbb{K}- $χώροι. Υποθέτουμε ότι ο $V$ έχει μια πεπερασμένη 
  βάση, έστω την $ B = \{ \mathbf{v_{1}}, \ldots, \mathbf{v_{n}} \} $. Αν 
  $ S,T \colon V \to W$ είναι δυο γραμμικοί μετασχηματισμοί και 
  $ S(\mathbf{v}_{i}) = T(\mathbf{v}_{i}) $ για $ i=1,2,\ldots,n $ τότε $ S=T $.
\end{cor}

\begin{rem}
  Δηλαδή δυο γραμμικοί γραμμικοί μετασχηματισμοί ταυτίζονται αν ταυτίζονται στα στοιχεία
  μιας βάσης.
\end{rem}

\begin{example}
  Έστω $ T \colon \mathbb{R}^{2} \to \mathbb{R}^{2} $ ο γραμμικός μετασχηματισμός με 
  τύπο $ T(a_{1}, a_{2}) = (2 a_{2}- a_{1}, 3 a_{1}) $. 
  Υποθέτουμε ότι $ U \colon \mathbb{R}^{2} \to \mathbb{R}^{2} $ είναι γραμμικός. Αν 
  γνωρίζουμε ότι $ U(1,2) = (3,3) $ και $ U(1,1) = (1,3) $, τότε $ U=T $. 

  Πράγματι, αυτό προκύπτει από το γεγονός ότι $ \{ (1,2), (3,3) \}  $ είναι μια βάση 
  του $ \mathbb{R}^{2} $ και από το ότι $ T(1,2)=(3,3) = U(1,2) $ και 
  $ T(1,1) = (1,3) = U(1,1) $. Τότε σύμφωνα με το πόρισμα, αφού οι δυο μετασχηματισμοί 
  ταυτίζονται στα στοιχεία μιας βάσης, είναι ίσοι.
\end{example}

\begin{prop}
  Έστω $V$ και $W$ δυο $ \mathbb{K}- $χώροι, πεπερασμένης διάστασης και έστω 
  $ T \colon V \to W $ ένας γραμμικός μετασχηματισμός. Να δείξετε ότι 
  \begin{enumerate}[i)]
    \item $ \dim(V) < \dim(W) \Rightarrow T $ δεν είναι επί
    \item $ \dim(V) > \dim(W) \Rightarrow T $ δεν είναι $ 1-1 $ 
  \end{enumerate}
\end{prop}

\begin{proof}
  %todo να βρω τις αποδείξεις και να τις γραψω
\end{proof}


 \end{document}
