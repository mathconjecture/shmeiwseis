\documentclass[a4paper,table]{report}
\documentclass[a4paper,12pt]{article}
\usepackage{etex}
%%%%%%%%%%%%%%%%%%%%%%%%%%%%%%%%%%%%%%
% Babel language package
\usepackage[english,greek]{babel}
% Inputenc font encoding
\usepackage[utf8]{inputenc}
%%%%%%%%%%%%%%%%%%%%%%%%%%%%%%%%%%%%%%

%%%%% math packages %%%%%%%%%%%%%%%%%%
\usepackage{amsmath}
\usepackage{amssymb}
\usepackage{amsfonts}
\usepackage{amsthm}
\usepackage{proof}

\usepackage{physics}

%%%%%%% symbols packages %%%%%%%%%%%%%%
\usepackage{dsfont}
\usepackage{stmaryrd}
%%%%%%%%%%%%%%%%%%%%%%%%%%%%%%%%%%%%%%%


%%%%%% graphicx %%%%%%%%%%%%%%%%%%%%%%%
\usepackage{graphicx}
\usepackage{color}
%\usepackage{xypic}
\usepackage[all]{xy}
\usepackage{calc}
%%%%%%%%%%%%%%%%%%%%%%%%%%%%%%%%%%%%%%%

\usepackage{enumerate}

\usepackage{fancyhdr}
%%%%% header and footer rule %%%%%%%%%
\setlength{\headheight}{14pt}
\renewcommand{\headrulewidth}{0pt}
\renewcommand{\footrulewidth}{0pt}
\fancypagestyle{plain}{\fancyhf{}
\fancyhead{}
\lfoot{}
\rfoot{\small \thepage}}
\fancypagestyle{vangelis}{\fancyhf{}
\rhead{\small \leftmark}
\lhead{\small }
\lfoot{}
\rfoot{\small \thepage}}
%%%%%%%%%%%%%%%%%%%%%%%%%%%%%%%%%%%%%%%

\usepackage{hyperref}
\usepackage{url}
%%%%%%% hyperref settings %%%%%%%%%%%%
\hypersetup{pdfpagemode=UseOutlines,hidelinks,
bookmarksopen=true,
pdfdisplaydoctitle=true,
pdfstartview=Fit,
unicode=true,
pdfpagelayout=OneColumn,
}
%%%%%%%%%%%%%%%%%%%%%%%%%%%%%%%%%%%%%%



\usepackage{geometry}
\geometry{left=25.63mm,right=25.63mm,top=36.25mm,bottom=36.25mm,footskip=24.16mm,headsep=24.16mm}

%\usepackage[explicit]{titlesec}
%%%%%% titlesec settings %%%%%%%%%%%%%
%\titleformat{\chapter}[block]{\LARGE\sc\bfseries}{\thechapter.}{1ex}{#1}
%\titlespacing*{\chapter}{0cm}{0cm}{36pt}[0ex]
%\titleformat{\section}[block]{\Large\bfseries}{\thesection.}{1ex}{#1}
%\titlespacing*{\section}{0cm}{34.56pt}{17.28pt}[0ex]
%\titleformat{\subsection}[block]{\large\bfseries{\thesubsection.}{1ex}{#1}
%\titlespacing*{\subsection}{0pt}{28.80pt}{14.40pt}[0ex]
%%%%%%%%%%%%%%%%%%%%%%%%%%%%%%%%%%%%%%

%%%%%%%%% My Theorems %%%%%%%%%%%%%%%%%%
\newtheorem{thm}{Θεώρημα}[section]
\newtheorem{cor}[thm]{Πόρισμα}
\newtheorem{lem}[thm]{λήμμα}
\theoremstyle{definition}
\newtheorem{dfn}{Ορισμός}[section]
\newtheorem{dfns}[dfn]{Ορισμοί}
\theoremstyle{remark}
\newtheorem{remark}{Παρατήρηση}[section]
\newtheorem{remarks}[remark]{Παρατηρήσεις}
%%%%%%%%%%%%%%%%%%%%%%%%%%%%%%%%%%%%%%%




\newcommand{\vect}[2]{(#1_1,\ldots, #1_#2)}
%%%%%%% nesting newcommands $$$$$$$$$$$$$$$$$$$
\newcommand{\function}[1]{\newcommand{\nvec}[2]{#1(##1_1,\ldots, ##1_##2)}}

\newcommand{\linode}[2]{#1_n(x)#2^{(n)}+#1_{n-1}(x)#2^{(n-1)}+\cdots +#1_0(x)#2=g(x)}

\newcommand{\vecoffun}[3]{#1_0(#2),\ldots ,#1_#3(#2)}


\input{myboxes.tex}

\geometry{left=20.63mm,right=20.63mm,top=30.25mm,bottom=20.25mm,footskip=14.16mm,headsep=24.16mm}


\let\vec\mathbf

\pagestyle{vangelis}

\begin{document}


\chapter{Εσωτερικό Γινόμενο}

\section{Ορισμός του Εσωτερικού Γινομένου}

\begin{dfn}
  Έστω $V$ ένας διανυσματικός χώρος επί του σώματος $\mathbb{K}$, όπου $ \mathbb{K} =
  \mathbb{R} $ ή $ \mathbb{C} $. \textcolor{Col1}{Εσωτερικό γινόμενο}, στο διανυσματικό 
  χώρο $V$ ονομάζεται μια απεικόνιση 
  \[
    \langle \cdot , \cdot \rangle \colon V \times V \to \mathbb{K}
  \] 
  η οποία, $ \forall \mathbf{u}, \mathbf{v}, \mathbf{w} \in V $ και $ \forall a,b \in
  \mathbb{K} $ ικανοποιεί τα παρακάτω αξιώματα:
    \begin{description}
      \item [Γραμμική Ιδιότητα:] 
        \begin{enumerate}[(i)]
          \item []
          \item $ \langle \mathbf{u} + \mathbf{v}, \mathbf{w}\rangle = \langle 
            \mathbf{u}, \mathbf{w}\rangle + \langle \mathbf{v}, \mathbf{w}\rangle $
          \item $ \langle a \mathbf{u}, \mathbf{v}\rangle = a \langle \mathbf{u},
            \mathbf{v}\rangle $
        \end{enumerate}
      \item [Συμμετρική Ιδιότητα:] $ \langle \mathbf{u}, \mathbf{v}\rangle = \langle
        \mathbf{v} , \mathbf{u}\rangle^{*}, 
        \quad \forall \mathbf{u}, \mathbf{v} \in V$
      \item [Θετικά Ορισμένη Ιδιότητα] 
        $ \langle \mathbf{u}, \mathbf{u}\rangle \geq 0 $ και $ \langle \mathbf{u},
        \mathbf{u}\rangle = 0 \Leftrightarrow \mathbf{u} = \mathbf{0},
        \quad \forall \mathbf{u} \in V $
    \end{description}
\end{dfn}

\begin{rem}
\item {}
  \begin{myitemize}
    \item Με $ a^{*} $ συμβολίζουμε το \textbf{συζυγή}\footnote{Αν $ z=a+ib 
        \in \mathbb{C} $, τότε ο \textcolor{Col1}{συζυγής} του $ z $ είναι ο μιγαδικός 
      αριθμός $ z^{*} = a- ib $.} μιγαδικό αριθμό του $ a \in \mathbb{C} $. 
      Οπότε στην περίπτωση όπου $ \mathbb{K}= \mathbb{R} $, η σχέση $ \langle
      \mathbf{u}, \mathbf{v}\rangle = \langle \mathbf{v}, \mathbf{u}\rangle ^{*} $ 
      γίνεται $ \langle \mathbf{u}, \mathbf{v}\rangle = \langle \mathbf{v}, 
      \mathbf{u}\rangle $.
    \item Αν ορίζεται τέτοια απεικόνιση σε ένα διανυσματικό χώρο $V$, τότε ο χώρος 
      $V$ ονομάζεται \textcolor{Col1}{χώρος εσωτερικού γινομένου}, 
      \textbf{πραγματικός} αν $ \mathbb{K}= \mathbb{R} $ και \textbf{μιγαδικός} αν 
      $ \mathbb{K}= \mathbb{C} $.
    \item Αν επιπλέον $ \dim(V) < \infty $ τότε ο χώρος $ V $ ονομάζεται
      \textcolor{Col1}{Ευκλείδιος} αν $ \mathbb{K} = \mathbb{R} $ και
      \textcolor{Col1}{Μοναδιαίος} (Unitary) αν $ \mathbb{K}= \mathbb{C} $.
  \end{myitemize}
\end{rem}

\begin{rem}
  Στην περίπτωση όπου $ \mathbb{K}= \mathbb{C} $, τότε ένα εσωτερικό γινόμενο δεν μπορεί
  να είναι γραμμική απεικόνιση και ως προς τη 2η μεταβλητή, γιατί τότε, αν 
  $ \mathbf{u} \in V $, και $ \mathbf{u} \neq \mathbf{0} $ θα είχαμε 
  $ \langle \mathbf{u}, \mathbf{u}\rangle > 0 $, όμως $ \langle i \mathbf{u}, i 
  \mathbf{u}\rangle = i^{2} \langle \mathbf{u}, \mathbf{u}\rangle = - \langle 
  \mathbf{u}, \mathbf{u}\rangle < 0 $, άτοπο. Αποδεικνύεται όμως ότι 
  \[
    \langle \mathbf{u}, \mathbf{v}+ \mathbf{w}\rangle = \langle \mathbf{u},
    \mathbf{v}\rangle + \langle \mathbf{v}, \mathbf{w}\rangle \quad \text{και} \quad
    \langle \mathbf{u}, a \mathbf{v}\rangle = a^{*} \langle \mathbf{u}, 
    \mathbf{v}\rangle 
  \] 
  Οι σχέσεις αυτές μπορούν να αντικατασταθούν από τη σχέση 
  \[
    \langle \mathbf{u}, a \mathbf{v}+ b \mathbf{w}\rangle = a^{*} \langle \mathbf{u},
    \mathbf{v}\rangle + b^{*} \langle \mathbf{u}, \mathbf{w}\rangle 
  \] 
  Πράγματι, 
  \begin{align*}
    \langle \mathbf{u}, a \mathbf{v}+ b \mathbf{w}\rangle 
    &= \langle a \mathbf{v}+ b
    \mathbf{w}, \mathbf{u}\rangle ^{*} = \bigl(a \langle \mathbf{v}, \mathbf{u}\rangle + 
      b \langle \mathbf{w}, \mathbf{u}\rangle\bigr)^{*} = \bigl(a \langle \mathbf{v},
    \mathbf{u}\rangle\bigr)^{*} + (b \langle \mathbf{w}, \mathbf{u}\rangle)^{*} \\ 
    &= a^{*} \langle \mathbf{v}, \mathbf{u}\rangle^{*} + b^{*} \langle \mathbf{w},
    \mathbf{u}\rangle^{*} 
    = a^{*} \langle \mathbf{u}, \mathbf{v}\rangle + b^{*} \langle \mathbf{u}, 
    \mathbf{w}\rangle
  \end{align*} 
\end{rem}

\begin{rem}
  Ισχύει ότι $ \langle \mathbf{u}, \vec{0}\rangle = \langle \vec{0},
  \mathbf{u}\rangle = \vec{0}, \; \forall \mathbf{u} $ καθώς επίσης και ότι 
  $ \langle \mathbf{u}, \mathbf{u}\rangle \in \mathbb{R}, \; \forall \mathbf{u} $, 
  αφού 
  $ \langle \mathbf{u}, \mathbf{u}\rangle = \langle \mathbf{u},
  \mathbf{u}\rangle^{*}\footnote{Θυμόμαστε ότι $z \in \mathbb{R} \Leftrightarrow
  z=z^{*} $.} $.
\end{rem}


\section*{Παραδείγματα}

\begin{example}
    Αν $ V = \mathbb{R}^{n} $ με συντελεστές από το σώμα $ \mathbb{R} $ και 
      $ \mathbf{u} = (x_{1},\ldots,x_{n}) $, 
      $ (y_{1},\ldots,y_{n}) \in \mathbb{R}^{n} $,  
      τότε η πράξη
      \[
        \langle \mathbf{u}, \mathbf{v}\rangle = x_{1} y_{1} + x_{2} y_{2} +
        \cdots + x_{n} y_{n} = \sum_{i=1}^{n} x_{i} y_{i}
      \]
      είναι εσωτερικό γινόμενο στον 
      $ \mathbb{R}^{n} $ και ονομάζεται \textbf{κανονικό} εσωτερικό γινόμενο. 
      Έτσι, ο $ \mathbb{R}^{n} $ γίνεται \textbf{Ευκλείδιος} χώρος.
\end{example}
\begin{example}
    Αν $ V = \mathbb{C}^{n} $ με συντελεστές από το σώμα $ \mathbb{C} $ και 
      $ \mathbf{u} = (x_{1},\ldots,x_{n}) $, 
      $ (y_{1},\ldots,y_{n}) \in \mathbb{R}^{n} $,  
      τότε η πράξη
      \[
        \langle \mathbf{u}, \mathbf{v}\rangle = x_{1} \overline{y_{1}} + x_{2}
        \overline{y_{2}} + \cdots + x_{n} \overline{y_{n}} = 
        \sum_{i=1}^{n} x_{i} \overline{y_{i}} 
      \]
      είναι εσωτερικό γινόμενο στον 
      $ \mathbb{C}^{n} $ και ονομάζεται \textbf{κανονικό} εσωτερικό γινόμενο. 
      Έτσι, ο $ \mathbb{C}^{n} $ γίνεται \textbf{Μοναδιαίος} χώρος.
\end{example}

%todo να συνεχίσω σημειώσεις για εσωτερικό γινόμενο με παραδείγματα

\end{document}

