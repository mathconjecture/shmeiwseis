\documentclass[a4paper,12pt]{article}
\usepackage{etex}
%%%%%%%%%%%%%%%%%%%%%%%%%%%%%%%%%%%%%%
% Babel language package
\usepackage[english,greek]{babel}
% Inputenc font encoding
\usepackage[utf8]{inputenc}
%%%%%%%%%%%%%%%%%%%%%%%%%%%%%%%%%%%%%%

%%%%% math packages %%%%%%%%%%%%%%%%%%
\usepackage{amsmath}
\usepackage{amssymb}
\usepackage{amsfonts}
\usepackage{amsthm}
\usepackage{proof}

\usepackage{physics}

%%%%%%% symbols packages %%%%%%%%%%%%%%
\usepackage{dsfont}
\usepackage{stmaryrd}
%%%%%%%%%%%%%%%%%%%%%%%%%%%%%%%%%%%%%%%


%%%%%% graphicx %%%%%%%%%%%%%%%%%%%%%%%
\usepackage{graphicx}
\usepackage{color}
%\usepackage{xypic}
\usepackage[all]{xy}
\usepackage{calc}
%%%%%%%%%%%%%%%%%%%%%%%%%%%%%%%%%%%%%%%

\usepackage{enumerate}

\usepackage{fancyhdr}
%%%%% header and footer rule %%%%%%%%%
\setlength{\headheight}{14pt}
\renewcommand{\headrulewidth}{0pt}
\renewcommand{\footrulewidth}{0pt}
\fancypagestyle{plain}{\fancyhf{}
\fancyhead{}
\lfoot{}
\rfoot{\small \thepage}}
\fancypagestyle{vangelis}{\fancyhf{}
\rhead{\small \leftmark}
\lhead{\small }
\lfoot{}
\rfoot{\small \thepage}}
%%%%%%%%%%%%%%%%%%%%%%%%%%%%%%%%%%%%%%%

\usepackage{hyperref}
\usepackage{url}
%%%%%%% hyperref settings %%%%%%%%%%%%
\hypersetup{pdfpagemode=UseOutlines,hidelinks,
bookmarksopen=true,
pdfdisplaydoctitle=true,
pdfstartview=Fit,
unicode=true,
pdfpagelayout=OneColumn,
}
%%%%%%%%%%%%%%%%%%%%%%%%%%%%%%%%%%%%%%



\usepackage{geometry}
\geometry{left=25.63mm,right=25.63mm,top=36.25mm,bottom=36.25mm,footskip=24.16mm,headsep=24.16mm}

%\usepackage[explicit]{titlesec}
%%%%%% titlesec settings %%%%%%%%%%%%%
%\titleformat{\chapter}[block]{\LARGE\sc\bfseries}{\thechapter.}{1ex}{#1}
%\titlespacing*{\chapter}{0cm}{0cm}{36pt}[0ex]
%\titleformat{\section}[block]{\Large\bfseries}{\thesection.}{1ex}{#1}
%\titlespacing*{\section}{0cm}{34.56pt}{17.28pt}[0ex]
%\titleformat{\subsection}[block]{\large\bfseries{\thesubsection.}{1ex}{#1}
%\titlespacing*{\subsection}{0pt}{28.80pt}{14.40pt}[0ex]
%%%%%%%%%%%%%%%%%%%%%%%%%%%%%%%%%%%%%%

%%%%%%%%% My Theorems %%%%%%%%%%%%%%%%%%
\newtheorem{thm}{Θεώρημα}[section]
\newtheorem{cor}[thm]{Πόρισμα}
\newtheorem{lem}[thm]{λήμμα}
\theoremstyle{definition}
\newtheorem{dfn}{Ορισμός}[section]
\newtheorem{dfns}[dfn]{Ορισμοί}
\theoremstyle{remark}
\newtheorem{remark}{Παρατήρηση}[section]
\newtheorem{remarks}[remark]{Παρατηρήσεις}
%%%%%%%%%%%%%%%%%%%%%%%%%%%%%%%%%%%%%%%




\newcommand{\vect}[2]{(#1_1,\ldots, #1_#2)}
%%%%%%% nesting newcommands $$$$$$$$$$$$$$$$$$$
\newcommand{\function}[1]{\newcommand{\nvec}[2]{#1(##1_1,\ldots, ##1_##2)}}

\newcommand{\linode}[2]{#1_n(x)#2^{(n)}+#1_{n-1}(x)#2^{(n-1)}+\cdots +#1_0(x)#2=g(x)}

\newcommand{\vecoffun}[3]{#1_0(#2),\ldots ,#1_#3(#2)}



\zexternaldocument{transformations}

\pagestyle{vangelis}






\begin{document}

\chapter{Αναπαράσταση Γραμ. Μετασχ.}

\section{Ορισμός - Παραδείγματα}

\begin{dfn}
  Έστω  $V$  ένας  $ \mathbb{K}- $χώρος πεπερασμένης διάστασης. Μια
  \textcolor{Col1}{διατεταγμένη βάση} για τον $V$ είναι μια βάση εφοδιασμένη με μια 
  συγκεκριμένη διάταξη.
\end{dfn}

\begin{rem}
  Πολλές φορές αντί να γράφουμε $ \beta = (\mathbf{v_{1}}, \ldots, \mathbf{v_{n}}) $ για 
  μια διατεταγμένη βάση ενός $ \mathbb{K}- $χώρου $V$ θα γράφουμε 
  $ \beta = \{ \mathbf{v_{1}}, \ldots, \mathbf{v_{n}} \} $ και θα δηλώνουμε ότι η 
  $ \beta $ είναι διατεταγμένη βάση (με διάταξη αυτή που φαίνεται από την καταγραφή 
  των στοιχείων, δηλαδή ότι $ \mathbf{v_{1}} $ είναι το 1ο στοιχείο, 
  $ \mathbf{v_{2}} $ το 2ο κ.ο.κ.) 
\end{rem}


\begin{example}
Για παράδειγμα, στον $ \mathbb{R}^{3} $, η (συνήθης) βάση $ \beta = \{ \mathbf{e_{1}}, 
\mathbf{e_{2}},\mathbf{e_{3}}\} $ μπορεί να θεωρηθεί ως μια διατεταγμένη βάση. 
Επίσης, το σύνολο $ \gamma = \{ \mathbf{e_{2}}, \mathbf{e_{1}}, \mathbf{e_{3}} \} $ 
μπορεί να θεωθηθεί ως διατεταγμένη βάση. 
Τότε ισχύει ότι $ \beta \neq \gamma $ ως διατεταγμένες βάσεις. 
Ταυτίζονται όμως ως σύνολα.  
\end{example}

\begin{rem}
  Θυμάμαι ότι $ \mathbf{e_{1}} = (1,0,0), \mathbf{e_{2}}=(0,1,0), 
  \mathbf{e_{3}}=(0,0,1) $ 
\end{rem}

\begin{dfn}
  Έστω $V$ και $W$ δυο $ \mathbb{K}- $χώροι πεπερασμένης διάστασης. Έστω $ \beta = \{
  \mathbf{v_{1}}, \mathbf{v_{2}}, \ldots, \mathbf{v_{n}} \} $ μια διατεταγμένη βάση 
  του $V$ και έστω $ \gamma = \{ \mathbf{w_{1}}, \mathbf{w_{2}}, \ldots, 
  \mathbf{w_{n}} \} $ μια διατεταγμένη βάση του $W$. Έστω $ T \colon V \to W $ ένας 
  γραμμικός μετασχηματισμός. Τότε υπάρχουν μοναδικά $ a_{1j}, a_{2j}, \ldots, a_{mj} 
  \in \mathbb{K} $, με $ j = 1,2, \ldots, n $  τέτοια ώστε 
  \[
    T(\mathbf{v}_{j}) = \sum_{i=1}^{m} a_{ij} \mathbf{w}_{j}  
  \] 
  Ο $ m \times n $ πίνακας $A = (a_{ij}) $ λέγεται πίνακας 
  (\textcolor{Col1}{αναπαράσταση}) του γραμμικού μετασχηματισμού $T$ ως προς τις 
  διατεταγμένες βάσεις $ \beta $ και 
  $ \gamma $ και συμβολίζεται με $ [T]_{\beta}^{\gamma} $.
\end{dfn}

\begin{rem}
  Αν $ V=W $ και $ \beta = \gamma $ τότε γράφουμε $ A = [T]_{\beta} $ αντί για 
  $ [T]_{\beta}^{\beta} $.
\end{rem}

\begin{rem}
  Ουσιαστικά, ο πίνακας $ A = (a_{ij}) $ είναι ο $ m \times n $ πίνακας
    \begin{align*}
      A = [T]_{\beta}^{\gamma} = 
      \begin{pmatrix*}[c]
        \vrule & \vrule & & \vrule \\
        T(\mathbf{v_{1}}) & T(\mathbf{v_{2}}) & \cdots & T(\mathbf{v_{n}}) \\
        \vrule & \vrule & & \vrule
      \end{pmatrix*} \quad \text{\color{Col1} αναπαράσταση (του $T$)}
    \end{align*} 
    που περιέχει ως στήλες τα $ [T(\mathbf{v}_{j})]_{\gamma} $ 
    με $ j = 1,\ldots, n $, δηλαδή τις συντεταγμένες των  εικόνων μέσω του $T$, 
    των στοιχείων της βάσης $ \beta $ του $V$, εκφρασμένες ως προς τη βάση 
    $ \gamma $ του $W$. 
\end{rem}

\begin{rem}\label{rem:1}
    \label{rem:2} Αν $ S \colon V \to W $ είναι ένας γραμμικός μετασχηματισμός 
      τέτοιος ώστε 
      \[
        [S]_{\beta}^{\gamma} = [T]_{\beta}^{\gamma}
      \] 
      οπότε $ S(\mathbf{v}_{j}) = T(\mathbf{v}_{j}) $ για όλα τα $ j = 1,\ldots, n $ 
      τότε από γνωστό πόρισμα έπεται ότι $ S=T $. %todo να βρω και να γράψω το πορισμα
      Δηλαδή, γραμμικοί μετασχηματισμοί με ίδιο πίνακα–αναπαράσταση, ταυτίζονται.
\end{rem}

\begin{example}
  Έστω $ T \colon \mathbb{R}^{2} \to \mathbb{R}^{3} $ ο γραμμικός μετασχηματισμός με 
  τύπο 
  \[
    T(a_{1}, a_{2}) = (a_{1}+3 a_{2}, 0, 2 a_{1}- 4 a_{2}) 
  \] 
  Έστω $\beta$ και $\gamma$ οι συνήθεις διατεταγμένες βάσεις για τους $ \mathbb{R}^{2} $
  και $ \mathbb{R}^{3} $, αντίστοιχα (δηλαδή, $ \beta = \{ (1,0), (0,1) \} $ και 
  $ \gamma = \{ \underbrace{(1,0,0)}_{\mathbf{e_{1}}},
  \underbrace{(0,1,0)}_{\mathbf{e_{2}}}, \underbrace{(0,0,1)}_{\mathbf{e_{3}}} \} $). 
  Τότε
  \begin{align*}
    T(1,0) &= (1,0,2) = 1 \mathbf{e_{1}} + 0 \mathbf{e_{2}}+2 \mathbf{e_{3}} \\
    T(0,1) &= (3,0,-4) = 3 \mathbf{e_{1}}+ 0 \mathbf{e_{2}}- 4 \mathbf{e_{3}}
  \end{align*}
  άρα
  \[
    [T]_{\beta}^{\gamma} = 
    \begin{pmatrix*}[r]
      1 & 3 \\
      0 & 0 \\
      2 & -4
  \end{pmatrix*} 
  \] 
\end{example}

\begin{example}
  Έστω $ T \colon \textbf{P}_{3}(\mathbb{R}) \to \textbf{P}_{2}(\mathbb{R}) $ ο 
  γραμμικός μετασχηματισμός με τύπο 
  \[
    T(p(x)) = p'(x), \quad \forall x \in \mathbb{R} 
  \]
  Έστω $ \beta $ και $\gamma$ οι συνήθεις διατεταγμένες βάσεις για τους $
  \textbf{P}_{3}(\mathbb{R}) $ και $ \textbf{P}_{2}(\mathbb{R}) $, αντίστοιχα 
  (δηλαδή, $ \beta = \{ 1,x,x^{2},x^{3} \}$ και $\gamma = \{ 1,x,x^{2} \}$). 
  Τότε
  \begin{align*}
    T(1) = 0\cdot 1 +0x +0x^{2} \\
    T(x) = 1 \cdot 1 + 0x + 0x^{2} \\
    T(x^{2}) = 0 \cdot 1 + 2x +0x^{2} \\ 
    T(x^{3}) = 0 \cdot 1 + 0x + 3x^{2}
  \end{align*}
  άρα 
  \[
    [T]_{\beta}^{\gamma} = 
    \begin{pmatrix*}[r]
      0 & 1 & 0 & 0 \\
      0 & 0 & 2 & 0 \\
      0 & 0 & 0 & 3
    \end{pmatrix*}
  \] 
\end{example}

\begin{prop}
  Έστω $V$ και $W$ δυο $ \mathbb{K}- $χώροι, και έστω $ T,S \colon V \to W $ δύο 
  γραμμικοί μετασχηματισμοί. Έστω επίσης, $\lambda$ οποιοσδήποτε αριθμός στο 
  $ \mathbb{K} $. Τότε οι συναρτήσεις
  \begin{gather*}
    T+S \colon V \to W, \quad \text{με τύπο}\quad (T+S)(\mathbf{v}) = 
    T(\mathbf{v}) + S(\mathbf{v}), \quad \forall \mathbf{v} \in V \\
    \lambda T \colon V \to W, \quad \text{με τύπο}\quad (\lambda T)(\mathbf{v}) = 
    \lambda T(\mathbf{v}), \forall \mathbf{v} \in V
  \end{gather*}
  είναι γραμμικοί μετασχηματισμοί.
\end{prop}

\begin{proof}
  %todo
\end{proof}

\begin{dfn}
  Έστω $V$ και $W$ δύο $ \mathbb{K}- $χώροι. Τότε το σύνολο όλων των γραμμικών
  μετασχηματισμών από τον χώρο $V$ στο χώρο $W$, συμβολίζεται με $ \mathcal{L}(V,W) $. 
  Αν $ V=W $, τότε γράφουμε $ \mathcal{L}(V) $ αντί για $ \mathcal{L}(V,V) $.
\end{dfn}

\begin{thm}
  Έστω $V$ και $W$ δύο $ \mathbb{K}- $χώροι. Τότε το σύνολο $ \mathcal{L}(V,W) $ 
  με πράξεις όπως ορίστηκαν στην προηγούμενη πρόταση είναι ένας $ \mathbb{K}- $χώρος.
\end{thm}

\begin{proof}
  %todo   
\end{proof}

Αργότερα θα δούμε ότι μπορούμε να ταυτίσουμε τον $ \mathcal{L}(V,W) $ με τον $
\textbf{M}_{m \times n}(\mathbb{K}) $, όπου $n$ και $m$ είναι οι διαστάσεις των 
$V$ και $W$, αντίστοιχα. 

\begin{thm}
  Έστω $V$ και $W$ δύο $ \mathbb{K}- $χώροι πεπερασμένης διάστασης με διατεταγμένες 
  βάσεις $ \beta = \{ \mathbf{v_{1}}, \ldots, \mathbf{v_{n}} \} $ και $ \gamma = 
  \{ \mathbf{w_{1}}, \ldots, \mathbf{w_{m}} \} $ αντίστοιχα. Έστω $ T,S \in
  \mathcal{L}(V,W) $. Τότε ισχύουν τα παρακάτω:
  \begin{enumerate}[i)]
    \item $ [T+S]_{\beta}^{\gamma} = [T]_{\beta}^{\gamma} + [S]_{\beta}^{\gamma}$
    \item $ [\lambda T]_{\beta}^{\gamma} = \lambda [T]_{\beta}^{\gamma}, \quad \forall
      \lambda \in \mathbb{K} $
  \end{enumerate}
\end{thm}

\begin{proof}
  %todo   
\end{proof}

%todo (έχει δώσει 6 ασκήσεις για το σπίτι... να τις λύσω)

\begin{thm}
  Έστω $ V,W,Z $, τρεις $ \mathbb{K}- $χώροι και έστω $ T \colon V \to W $ και $ S
  \colon W \to Z $ δύο γραμμικοί μετασχηματισμοί. Τότε ή σύνθεση 
  $ S \circ T \colon V \to Z $ είναι επίσης, γραμμικός μετασχηματισμός.
\end{thm}

\begin{proof}
  %todo  
\end{proof}

\begin{thm}
  Έστω $V$ ένας $ \mathbb{K}- $χώρος και έστω $ T, S_{1}, S_{2} \in \mathcal{L}(V) $. 
  Τότε ισχούν τα παρακάτω:
  \begin{enumerate}[i)]
    \item $ T \circ (S_{1}+S_{2}) = (T \circ S_{1}) + (T \circ S_{2}) $ \quad και \quad $
      (S_{1}+S_{2}) \circ T = (S_{1} \circ T) + (S_{2} \circ T) $
    \item $ T \circ (S_{1} \circ S_{2}) = (T \circ S_{1}) \circ S_{2} $
    \item $ T \circ I_{V} = I_{V} \circ T = T $
    \item $ \lambda (S_{1} \circ S_{2}) = (\lambda S_{1}) \circ S_{2} = S_{1} \circ
      (\lambda S_{2}), \quad \forall \lambda \in \mathbb{K} $
  \end{enumerate}
\end{thm}

\begin{proof}
  %todo 
\end{proof}

\begin{thm}\label{thm:linthm}
  Έστω $ V,W,Z $ τρεις $ \mathbb{K}- $χώροι πεπερασμένης διάστασης με διατεταγμένες 
  βάσεις $ \alpha, \beta $ και $\gamma$, αντίστοιχα. Έστω $ T \in \mathcal{L}(V,W) $ 
  και $ S \in \mathcal{L}(W,Z) $. Τότε
  \[
    [S \circ T]_{\alpha}^{\gamma} = [S]_{\beta}^{\gamma} \cdot [T]_{\alpha}^{\beta}
  \] 
\end{thm}

\begin{proof}
  %todo 
\end{proof}

\begin{cor}
  Έστω $V$ ένας $ \mathbb{K}- $χώρος πεπερασμένης διάστασης με μια διατεταγμένη βάση 
  $\beta$. Έστω $ T,S \in \mathcal{L}(V) $ Τότε
  \[
    [S \circ T]_{\beta} = [S]_{\beta} \cdot [T]_{\beta}
  \] 
\end{cor}

\begin{thm}
  Έστω $V$ ένας $ \mathbb{K}- $χώρος πεπερασμένης διάστασης με μια διατεταγμένη βάση 
  $ \beta = \{ \mathbf{v_{1}}, \ldots, \mathbf{v_{n}} \} $. Τότε 
  ισχύει ότι 
  \[
    [I_{V}]_{\beta} = I_{n}
  \] 
  όπου $ I_{n} $ είναι ο ταυτοτικός πίνακας.
\end{thm}

\begin{proof}
  %todo  
\end{proof}

\begin{example}
  Έστω $ S \colon \textbf{P}_{3}(\mathbb{R}) \to \textbf{P}_{2}(\mathbb{R}) $ 
  και $ T \colon \textbf{P}_{2}(\mathbb{R}) \to \textbf{P}_{3}(\mathbb{R}) $ 
  με 
  \[
    S(p(x)) = p'(x) \quad \text{και} \quad T(p(x)) = \int_{0}^{x} p(t) \,{dt} 
  \] 
  Τότε οι $ S,T $ είναι γραμμικές. Έστω $ \alpha = \{ 1,x,x^{2},x^{3} \} $ και 
  $ \beta = \{ 1,x,x^{2} \} $ οι συνήθεις διατεταγμένες βάσεις των $
  \textbf{P}_{3}(\mathbb{R}) $ και $ \textbf{P}_{2}(\mathbb{R}) $ αντίστοιχα.
  Όπως είναι γνωστό, ισχύει ότι $ S \circ T = I_{\textbf{P}_{2}(\mathbb{R})}$. 
  Ας διαπιστώσουμε αυτό το αποτέλεσμα χρησιμοποιώντας το θεώρημα~\ref{thm:linthm}. 
  Έχουμε
  \[
    [S]_{\alpha}^{\beta} = 
    \begin{pmatrix*}[r]
      0 & 1 & 0 & 0 \\
      0 & 0 & 2 & 0 \\
      0 & 0 & 0 & 3
    \end{pmatrix*} \quad \text{και} \quad [T]_{\beta}^{\alpha} = 
    \begin{pmatrix*}[c]
      0 & 0 & 0 \\
      1 & 0 & 0 \\
      0 & 1/2 & 0 \\
      0 & 0 & 1/3
    \end{pmatrix*}
  \] 
\end{example}
Από την προηγούμενη θεωρία, έχουμε ότι 
\[
  [S \circ T]_{\beta} = [S]_{\alpha}^{\beta} \cdot [T]_{\beta}^{\alpha} = 
  \begin{pmatrix*}[r]
    1 & 0 & 0 \\
    0 & 1 & 0 \\
    0 & 0 & 1
  \end{pmatrix*} = [I_{\textbf{P}_{2}(\mathbb{R})}]_{\beta}
\]
Άρα, από την παρατήρηση~\ref{rem:1} παίρνουμε ότι 
$
S \circ T = I_{\textbf{P}_{2}(\mathbb{R})} 
$

\begin{dfn}
  Έστω $ A \in \textbf{M}_{m \times n}(\mathbb{K})  $. Ορίζουμε μια συνάρτηση 
  $ T_{A} \colon \mathbb{K}^{n} \to \mathbb{K}^{m} $ ως εξής:
  \[
    T_{A}(\mathbf{x}) = A \cdot \mathbf{x}, \quad \text{για κάθε διάνυσμα στήλη} 
    \quad  \mathbf{x} \in \mathbb{K}^{n}  
  \] 
  Η συνάρτηση $ T_{A} $ λέγεται \textcolor{Col1}{αριστερός--πολλαπλασιασμός
  μετασχηματισμός}.
\end{dfn}

Επομένως, αν 
\[
  A = 
  \begin{pmatrix}
    a_{11} & a_{12} & \cdots & a_{1n} \\
    a_{21} & a_{22} & \cdots & a_{2n} \\
    \vdots & \vdots & \ddots & \vdots \\
    a_{m1} & a_{m2} & \cdots & a_{mn} 
  \end{pmatrix} \quad \text{και} \quad \mathbf{x} = 
  \begin{pmatrix*}[r] x_{1} \\ x_{2} \\ \vdots \\ x_{n} \end{pmatrix*}
  \quad \text{τότε} \quad T_{A}(\mathbf{x}) = 
  \begin{pmatrix*}[c]
    a_{11} x_{1} + a_{12} x_{2} + \cdots + a_{1n}x_{n} \\
    a_{21} x_{1} + a_{22} x_{2} + \cdots + a_{2n}x_{n} \\ 
    \vdots \\
    a_{m1} x_{1} + a_{m2} x_{2} + \cdots + a_{mn}x_{n} \\ 
  \end{pmatrix*}_{m \times 1}
\] 

\begin{thm}
  Έστω $ A \in \textbf{M}_{m \times n}(\mathbb{K}) $. Τότε ο αριστερός-πολλαπλασιασμός
  μετασχηματισμός $ T_{A} \colon \mathbb{K}^{n} \to \mathbb{K}^{m} $ 
  είναι γραμμικός. Επιπλέον, αν $ B \in \textbf{M}_{m \times n}(\mathbb{K}) $ 
  και $ \beta , \gamma $ είναι οι συνήθεις διατεταγμένες βάσεις για τους 
  $ \mathbb{K}^{n} $ και $ \mathbb{K}^{m} $, αντίστοιχα, τότε έχουμε τις παρακάτω 
  ιδιότητες:
  \begin{enumerate}[i)]
    \item $ [T_{A}]_{\beta}^{\gamma} = A $
    \item $ T_{A} = T_{B} \Leftrightarrow A = B $ 
    \item $ T_{A+B} = T_{A} + T_{B} $
    \item $ T_{\lambda A} = \lambda T_{A}, \quad \forall \lambda \in \mathbb{K} $
    \item Αν $ S \colon \mathbb{K}^{n} \to \mathbb{K}^{m} $ είναι γραμμική, τότε 
      $ \exists ! \; C \in \textbf{M}_{m \times n}(\mathbb{K}) $ τέτοιος ώστε 
      $ S=T_{C} $.  Συγκεκριμένα, $ C = [S]_{\beta}^{\gamma} $.
    \item Αν $ E \in \textbf{M}_{n \times p}(\mathbb{K}) $, τότε $ T_{AE} = T_{A} 
      \circ T_{E} $ 
    \item Αν $ m=n $, τότε $ T_{I{n}} = I_{\mathbb{K}^{n}} $
  \end{enumerate}
\end{thm}

\begin{proof}
  %todo
\end{proof}

\begin{example}
  Έστω $ T \colon \mathbb{R}^{2} \to \mathbb{R}^{2} $ με 
  \[
    T \begin{pmatrix*}[r] x_{1} \\ x_{2} \end{pmatrix*} = 
    \begin{pmatrix*}[r] 2 x_{1}- x_{2} \\ x_{1}+ x_{2} \end{pmatrix*}
  \]  
  Τότε η $T$ είναι γραμμική αφού $ T = T_{A} $, όπου $ A = 
  \begin{pmatrix*}[r]
    2 & -1 \\
    1 & 1
  \end{pmatrix*} $
  Επιπλέον, ο πίνακας της $T$ ως προς τη συνήθη διατεταγμένη βάση $ \beta = \{ (1,0),
  (0,1) \} $ είναι ο $A$.
\end{example}

\begin{prop}
  Έστω $ A \in \textbf{M}_{m \times n}(\mathbb{K}) $. Τότε $ \im(T_{A}) = \Sigma _{A} $ 
  άρα $ \rank(T_{A}) = \rank(A) $
\end{prop}

\begin{proof}

\end{proof}

\begin{prop}
  Έστω $ A \in \textbf{M}_{m \times n}(\mathbb{K}) $. Τότε $ \ker(T_{A}) = N_{A} $ 
  (όπου $ N(A) $ είναι ο μηδενόχωρος του $A$) και άρα $ \dim(N(A)) = n - \rank(A) $
\end{prop}

\begin{proof}

\end{proof}

\begin{thm}
  Έστω $V$ και $W$ δυο $ \mathbb{K}- $χώροι πεπερασμένης διάστασης με διατεταγμένες 
  βάσεις $\beta$ και $\gamma$, αντίστοιχα, και έστω $ T \in \mathcal{L}(V,W) $. 
  Τότε, για κάθε $ \mathbf{x} \in V $, έχουμε 
  $ [T(\mathbf{x})]_{\gamma} = [T]_{\beta}^{\gamma} \cdot [\mathbf{x}]_{\beta} $
\end{thm}

\begin{proof}

\end{proof}

\begin{dfn}
  Έστω $V$ ένας $ \mathbb{K}- $χώρος πεπερασμένης διάστασης και έστω $ \beta = 
  \{ \mathbf{v_{1}}, \mathbf{v_{2}}, \ldots, \mathbf{v}_{n} \}$,
  $ \beta ' = \{ \mathbf{v_{1}}', \mathbf{v_{2}}', \ldots \mathbf{v_{n}}' \} $ 
  δυο διατεταγμένες βάσεις για τον $V$. Για $ j = 1,2, \ldots , n $, έστω ότι 
  \[
    [\mathbf{v}_{j}]_{\beta '} = 
    \begin{pmatrix*}[r] a_{1j} \\ a_{2j} \\ \vdots \\ a_{nj} \end{pmatrix*} 
  \] 
  (το διάνυσμα των συντεταγμένων του $ \mathbf{v}_{j} $ ως προς τη βάση $ \beta ' $, 
  συνεπώς $ \mathbf{v}_{j} = \sum_{i=1}^{n} a_{ij} \mathbf{v}_{i}' $). Τότε ο 
  πίνακας $ A = (a _{ij}) $ λέγεται πίνακας αλλαγής βάσης από τη $ \beta $ στη 
  $ \beta ' $.
\end{dfn}

\begin{thm}
  Έστω $V$ ένας $ \mathbb{K}- $χώρος πεπερασμένης διάστασης και έστω $\beta$ και 
  $ \beta ' $ δυο διατεταγμένες βάσεις για τον $V$. Τότε ο πίνακας αλλαγής βάσης από 
  την $\beta$ στη $\beta'$ είναι ο πίνακας του ταυτοτικού μετασχηματισμού $ I_{V} $ 
  ως προς τις βάσεις $\beta$ και $ \beta ' $.
\end{thm}

\begin{proof}

\end{proof}

\begin{rem}
  Δεν ισχύει εν γένει ότι $ [I_{V}]_{\beta }^{\beta '} = I_{n} $ (ενώ θυμηθείτε ότι 
  ισχύει πάντα $ [I_{V}]_{\beta } = I_{n} $). Για παράδειγμα, έστω ο $ \mathbb{R}- $ 
  χώρος $ \mathbb{R}^{2} $ και οι εξής διατεταγμένες βάσεις του, 
  \begin{align*}
    \beta = \{ \mathbf{f}_{1} = (1,0), \mathbf{f}_{2} = (1,1) \} \quad \text{και} 
    \quad \beta ' = \{ \mathbf{e}_{1} = (1,0), \mathbf{e_{2}} = (0,1) \} 
  \end{align*}
  Τότε
  \begin{gather*}
    I_{V}(\mathbf{f}_{1}) = \mathbf{f}_{1} = 1 \mathbf{e}_{1} + 0 \mathbf{e_{2}} \\
    I_{V}(\mathbf{f}_{2}) = \mathbf{f}_{2} = 1 \mathbf{e_{1}}+ 1 \mathbf{e_{2}}
  \end{gather*} 
  άρα 
  \[
    [I_{V}]_{\beta }^{\beta '} = 
    \begin{pmatrix*}[r]
      1 & 1 \\
      0 & 1
    \end{pmatrix*} \neq I_{2}
  \] 
\end{rem}

\begin{thm}\label{thm:3}
  Έστω $v$ ένας $ \mathbb{K}- $χώρος πεπερασμένης διάστασης και εστω $\beta$ και 
  $\beta '$ δυο διατεταγμένες βάσεις για τον $V$. Τότε, για κάθε $ \mathbf{u} \in V $, 
  ισχύει 
  \[
    [\mathbf{u}]_{\beta '} = [I_{V}]_{\beta }^{\beta '} \cdot [\mathbf{u}]_{\beta }
  \] 
\end{thm}

\begin{proof}
\end{proof}

\begin{thm}\label{thm:4}
  Έστω $V$ ένας $ \mathbb{K}- $χώρος πεπερασμένης διάστασης και εστω $\beta$ και 
  $\beta '$ δυο διατεταγμένες βάσεις για τον $V$. Ο πίνακας $ [I_{V}]_{\beta }^{\beta '}
  $ (αλλαγής βάσης από τη $\beta$ στη $\beta '$) είναι αντιστρέψιμος και ισχύει 
  \[
    \left({[I_{V}]_{\beta }^{\beta '}}\right)^{-1} = [I_{V}]_{\beta '}^{\beta}  
  \] 
  Δηλαδή ο αντίστροφος του πίνακα αλλαγής βάσης από τη $\beta$ στη $\beta '$ είναι 
  ο πίνακας αλλαγής βάσης από την $\beta '$ στη $\beta$.
\end{thm}

\begin{proof}

\end{proof}

\begin{example}
  Θεωρούμε τις εξής διατεταγμένες βάσεις του $ \mathbb{R}^{2} $:
  \begin{align*}
    \beta = \{ \mathbf{e_{1}}= (1,0), \mathbf{e_{2}}= (0,1) \} \\
    \beta ' = \{ \mathbf{f}_{1} = (1,1), \mathbf{f} _{2} = (2,1) \} 
  \end{align*}
  Να βρεθεί ο πίνακας αλλαγής βάσης από τη $ \beta $ στη $ \beta' $ καθώς και το 
  διάνυσμα των συντεταγμένων του $ \mathbf{u} = (2,4) $ ως προς τη $\beta '$.
\end{example}

\begin{solution}
  Έχουμε $ [I_{\mathbb{R}^{2}}]_{\beta }^{\beta'} = \left([I_{\mathbb{R}^{2}}]_{\beta
  '}^{\beta}\right)^{-1} $ (από το θεώρημα~\ref{thm:4}). Επειδή η $\beta$ είναι η 
  συνήθης βάση του $ \mathbb{R}^{2} $, προκύπτει άμεσα ότι 
  \[
    [I_{\mathbb{R}^{2}}]_{\beta '}^{\beta } = 
    \begin{pmatrix*}[r]
      1 & 2 \\
      1 & 1
    \end{pmatrix*}
  \] 
  Τώρα, 
  \[
    \begin{pmatrix*}[r]
      1 & 2 & \vrule &  1 & 0 \\
      \undermat{[I_{\mathbb{R}^{2}}]_{\beta '}^{\beta }}{1 & 1} & \vrule & 0 & 1 
    \end{pmatrix*} \sim 
    \begin{pmatrix*}[r]
      1 & 2 & \vrule & 1 & 0 \\
      0 & -1 & \vrule & -1 & 1
    \end{pmatrix*} \sim 
    \begin{pmatrix*}[r]
      1 & 2 & \vrule & 1 & 0 \\
      0 & 1 & \vrule & 1 & -1
    \end{pmatrix*} \sim 
    \begin{pmatrix*}[r]
      1 & 0 & \vrule & -1 & 2 \\
      0 & 1 & \vrule & \undermat{[I_{\mathbb{R}^{2}}]_{\beta}^{\beta'}}{1 & -1} 
    \end{pmatrix*}
  \]

  \vspace{2\baselineskip}

  Για το διάνυσμα $ [\mathbf{u}]_{\beta'} $ των συντεταγμένων του 
  $ \mathbf{u} $ ως προς τη βάση $ \beta ' $, από το θεώρημα~\ref{thm:3} 
  έχουμε ότι 
  \[
    [\mathbf{u}]_{\beta '} = [I_{\mathbb{R}^{2}}]_{\beta }^{\beta'} \cdot
    [\mathbf{u}]_{\beta} = 
    \begin{pmatrix*}[r]
      -1 & 2 \\
      1 & -1
    \end{pmatrix*} \cdot 
    \begin{pmatrix*}[r]
      2 \\
      4
    \end{pmatrix*} = 
    \begin{pmatrix*}[r]
      6 \\
      -2
    \end{pmatrix*}
  \] 
\end{solution}

\begin{example}
  Θεωρούμε τις εξής διατεταγμένες βάσεις του $ \mathbb{R}^{2} $: 
  \begin{align*}
    \beta &= \{ (1,1), (1,-1) \} \\
    \beta ' &= \{ (2,4), (2,1) \}
  \end{align*}
  Να βρεθεί ο πίνακας αλλαγής βάσης από τη $\beta$ στη $\beta'$.
\end{example}

\begin{solution}
  Εφόσον $\beta'$ είναι μια βάση για τον $ \mathbb{R}^{2} $, για το $ (1,1) $ 
  υπάρχουν μοναδικά $ a,b \in \mathbb{R} $ ώστε 
  \[
    (1,1) = a (2,4) + b(2,1) 
  \]
  Ομοίως, για το (1,-1) υπάρχουν μοναδικά $ c,d \in \mathbb{R} $ ώστε 
  \[
    (1,-1) = c(2,4) + d(2,1) 
  \] 
  Τότε
  \[
    [I_{\mathbb{R}^{2}}]_{\beta}^{\beta'} = 
    \begin{pmatrix*}[r]
      a & c \\
      b & d
    \end{pmatrix*}
  \] 
  Έχουμε λοιπόν να λύσουμε τα συστήματα 
  \[
    \left.
      \begin{matrix}
        2a + 3b = 1 \\
        4a + b = 1
      \end{matrix} 
    \right\} \quad \text{και} \quad 
    \left.
      \begin{matrix}
        3c + 3d = 1 \\
        4c + d = -1
      \end{matrix} 
    \right\} 
  \] 
  τα οποία λόγω της μοναδικής γραφής των (1,1) και (1,-1) θα έχουν μοναδική λύση, 
  πράγματι
  \[
    \begin{pmatrix*}[r]
      2 & 3 & \vrule & 1 & 1 \\
      4 & 1 & \vrule & 1 & -1
    \end{pmatrix*} \sim \cdots \sim
    \begin{pmatrix*}[r]
      1 & 0 & \vrule & 1/5 & -2/5 \\
      0 & 1 & \vrule & 1/5 & 3/5
    \end{pmatrix*}
  \] 
  Άρα, 
  \[
    [I_{\mathbb{R}^{2}}]_{\beta}^{\beta'} = 
    \begin{pmatrix*}[r]
      1/5 & -2/5 \\
      1/5 & 3/5
    \end{pmatrix*}
  \] 
\end{solution}

\begin{thm}
  Έστω $V$ και $W$ δυο $ \mathbb{K}- $χώροι πεπερασμένης διάστασης, και έστω 
  $\beta_{1}$, ${\beta_{1}}'$ δυο διατεταγμένες βάσεις για τον $V$ και έστω 
  $\beta_{2}$, ${{\beta}_{2}}'$ δυο διατεταγμένες βάσεις για τον $W$. Έστω 
  $ T \in \mathcal{L}(V,W) $. Τότε ισχύει: 
  \[
    [T]_{{\beta_{1}}'}^{{\beta_{2}}'} = [I_{W}]_{\beta_{2}}^{{\beta_{2}}'} \cdot 
    [T] _{\beta_{1}}^{\beta_{2}} \cdot [I_{V}]_{{\beta_{1}}'}^{\beta_{1}}
  \] 
\end{thm}

\begin{proof}

\end{proof}

\begin{cor}\label{cor:bas}
  Έστω $V$ ένας $ \mathbb{K}- $χώρος πεπερασμένης διάστασης και έστω $ \beta $ και 
  $ \beta' $ δυο διατεταγμένες βάσεις για τον $V$. Έστω $ T \in \mathcal{L}(V) $. 
  Τότε ισχύει
  \begin{align*}
    [T]_{\beta'} &= [T]_{\beta}^{\beta'} \cdot [T]_{\beta} \cdot [T]_{\beta'}^{\beta} & 
    (\text{Θεώρημα~\ref{thm:4}}) \\
                 &= [I_{V}]_{\beta }^{\beta'} \cdot [T]_{\beta } 
    \cdot \left([I_{V}]_{\beta }^{\beta'}\right)^{-1} &
  \end{align*} 
\end{cor}

\begin{rem}
  Υπενθυμίζουμε ότι αν $ A,B \in \textbf{M}_{m \times n}(\mathbb{K}) $ τότε ο 
  $B$ λέγεται όμοιος με τον $A$ αν υπάρχει αντιστρέψιμος πίνακας $ P \in
  \textbf{M}_{n}(\mathbb{K}) $ τέτοιος ώστε $ B = P \cdot A \cdot P^{-1} $. 
  Επομένως αν $V$, $ \beta , \beta' $ και $T$ είναι όπως στο Πόρισμα~\ref{cor:bas}, 
  τότε ο πίνακας $ [T]_{\beta'} $ είναι όμοιος προς τον $[T]_{\beta}$.
\end{rem}

\begin{dfn}
  Έστω $V$ και $W$ δυο $ \mathbb{K}- $χώροι και έστω $ T \in \mathcal{L}(V,W) $. 
  Μια συνάρτηση $ U \colon W \to V $, λέγεται ότι είναι ένας αντίστροφος του $T$ 
  αν 
  \[
    T \circ U = I_{W} \quad \text{και} \quad U \circ T = I_{V} 
  \]
  Αν ο $T$ έχει έναν αντίστροφο, τότε ο $T$ καλείται αντιστρέψιμος.
\end{dfn}

Σημειώνουμε ότι αν ο $T$ είναι αντιστρέψιμος, τότε ο αντίστροφος του $T$ είναι 
μοναδικός και συμβολίζεται με $ T^{-1} $. Ισχύουν:
\begin{enumerate}
  \item $ (T \circ U)^{-1} = U^{-1} \circ T^{-1} $
  \item $ (T^{-1})^{-1} = T $, \quad δηλαδή ο $T$ είναι επίσης αντιστρέψιμος
\end{enumerate}
Ακόμη συχνά χρησιμοποιούμε το γεγονός ότι μια συνάρτηση είναι αντιστρέψιμη αν και μόνον αν
είναι $ 1-1 $ και επί. Οπότε μπορούμε να αναδιατυπώσουμε το θεώρημα~\ref{thm:3one} 
ως εξής
\begin{enumerate}[resume]
  \item Έστω $ T \in \mathcal{L}(V,W) $, όπου $V$ και $W$ δύο $ \mathbb{K}- $χώροι 
    πεπεαρασμένης διάστασης με $ \dim(V) = \dim(W) $. Τότε 
    \[ T  \quad \text{αντιστρέψιμος}  \Leftrightarrow \rank(T) = \dim(V) \]
\end{enumerate}

\begin{thm}
  Έστω $V$ και $W$ δυο $ \mathbb{K}- $χώροι και έστω $ T \colon V \to W $ ένας 
  αντιστρέψιμος γραμμικός μετασχηματισμός. Τότε $ T^{-1} \in \mathcal{L}(V,W) $.
\end{thm}

\begin{proof}

\end{proof}

\begin{thm}
  Έστω $V$ και $W$ δυο $ \mathbb{K}- $χώροι πεπερασμένης διάστασης και έστω 
  $ T \in \mathcal{L}(V,W) $. Αν ο $T$ είναι αντιστρέψιμος, τότε 
  $ \dim(V) = \dim(W) $.
\end{thm}

\begin{proof}

\end{proof}

\begin{thm}
  Έστω $V$ και $W$ δυο $ \mathbb{K}- $χώροι πεπερασμένης διάστασης με διατεταγμένες 
  βάσεις $\beta$ και $\gamma$, αντίστοιχα. Έστω $ T \in \mathcal{L}(V,W) $. Τότε 
  ο $T$ είναι αντιστρέψιμος αν και μόνον αν ο πίνακας $ [T]_{\beta}^{\gamma } $ 
  είναι αντιστρέψιμος. Επιπλέον
  \[
    \left([T]_{\beta }^{\gamma }\right)^{-1} = [T^{-1}]_{\gamma }^{\beta} 
  \]
\end{thm}

\begin{proof}

\end{proof}


\begin{cor}
  Έστω $V$ ένας $ \mathbb{K}- $χώρος πεπερασμένης διάστασης με μια διατεταγμένη 
  βάση $ \beta $, και έστω $ T \colon V \to V $ γραμμικός. Τότε $T$ αντιστρέψιμος 
  αν και μόνον αν $ [T]_{\beta} $ είναι αντιστρέψιμος. Επιπλέον, $ [T^{-1}]_{\beta} = 
  ([T]_{\beta })^{-1}$.
\end{cor}

\begin{cor}
  Έστω $ A \in \textbf{M}_{n}(\mathbb{K})  $. Τότε ο $A$ είναι αντιστρέψιμος αν και 
  μόνον αν ο (αριστερός - πολλάπλασιασμός μετασχηματισμός) $ T_{A} $ είναι 
  αντιστρέψιμος. Επιπλέον, $ (T_{Α})^{-1} = T_{A^{-1}} $.
\end{cor}

\begin{dfn}
  Έστω $V$ και $W$ δύο $ \mathbb{K}- $χώροι. Λέμε ότι ο $V$ είναι ισόμορφος με τον 
  $W$ και το συμβολίζουμε με $ V \cong W $ αν υπάρχει γραμμικός μετασχηματισμός 
  $ T \colon V \to W $ ο οποίος είναι αντιστρέψιμος (δηλαδή $ 1-1 $ και επί). Ένας 
  τέτοιος γραμμικός μετασχηματισμός καλείται ισομορφισμός απο τον $V$ στον $W$.
\end{dfn}

\begin{rem}
  Κάθε $ \mathbb{K}- $χώρος είναι ισόμορφος με τον εαυτό του. (Ο ταυτοτικός 
  μετασχηματισμός $ I_{V} \colon V \to V $ ) είναι ένας ισομορφισμός από τον $V$ επί 
  του $V$.
\end{rem}

\begin{rem}
  Αν ο $ \mathbb{K}- $χωρος $V$ είαι ισόμορφος με τον $ \mathbb{K}- $χώρο $W$, τότε 
  και ο $W$ είναι ισόμορφος με τον $V$ (Αν $ T \colon V \to W $ είναι ένας ισομορφισμός 
  από τον $V$ στον $W$, τότε ο $ T^{-1} \colon W \to V $ είναι ένας ισομορφισμός από 
  τον $W$ στον $V$). 
\end{rem}

\begin{rem}
  Αν ο $ \mathbb{K}- $χώρος $V$ είναι ισόμορφος με τον $ \mathbb{K}- $χώρο $W$ και ο 
  $W$ είναι ισόμορφος με τον $ \mathbb{K}- $χώρο $Z$, τότε ο $V$ είναι ισόμορφος με 
  τον $Z$. (Αν $ T \colon V \to W $ και $ S \colon W\to Z $ είναι ισομορφισμοί, τότε 
  η σύνθεση $ S \circ T \colon V \to Z$ είναι ισομορφισμός από τον $V$ στον $ Z $ ).
\end{rem}

Επομέως από τις παραπάνω παρατηρήσεις η σχέση ισομορφισμού είναι μια σχέση 
ισοδυναμίας επί του συνόλου των $ \mathbb{K}- $χώρων.

\begin{thm}
  Έστω $V$ και $W$ δύο $ \mathbb{K}- $χώροι πεπερασμένης διάστασης. Τότε
\[
  V \quad \text{ισόμορφος με τον $W$} \Leftrightarrow \dim(V) = \dim(W)   
 \] 
\end{thm}

\begin{cor}
  Έστω $V$ ένας $ \mathbb{K}- $χώρος. Τότε ο $V$ είναι ισόμορφος με τον 
  $ \mathbb{K}^{n} $ αν και μόνον αν $ \dim(V) = n $ 
\end{cor}

\begin{rem}
  Αν $ V $ είναι ένας $ \mathbb{K}- $χώρος διάστασης $n$ με μια διατεταγμένη βάση 
  $\beta$, τότε ένας ισομορφισμός $ T \colon V \to \mathbb{K}^{n} $ είναι ο 
  \[
    T(\mathbf{v}) = [\mathbf{v}]_{\beta}, \quad \forall \mathbf{v} \in V
   \] 
   όπου $ [\mathbf{v}]_{\beta} $ είναι το διάνυσμα των συντεταγμένων του $ \mathbf{v} $ 
   ως προς τη βάση $\beta$. 
\end{rem}

\begin{thm}
  Έστω $V$ και $W$ δύο $ \mathbb{K}- $χώροι πεπερασμένων διαστάσεων, έστω $n$ και 
  $ m $ αντίστοιχα, και έστω $\beta$ και $\gamma$ διατεταγμένες βάσεις για τον $V$ και 
  $W$ αντίστοιχα. Τότε η συνάρτηση $ \phi \colon \mathcal{L}(V,W) \to 
  \textbf{M}_{m \times n}(\mathbb{K})  $ η οποία ορίζεται ως 
  $ \phi (T) = [T]_{beta}^{\gamma}, \quad \forall 
  T \in \mathcal{L}(V,W) $ είναι ένας ισομορφισμός.
\end{thm}

\begin{cor}
  Έστω $V$ και $W$ δύο $ \mathbb{K}- $χώροι πεπερασμένων διαστάσεων, με διαστάσεις 
  $ n $ και $ m $ αντίστοιχα. Τότε ο $ \mathbb{K}- $χώρος $ \mathcal{L}(V,W) $ είναι 
  πεπερασμένης διάστασης με διαστάση $ m \cdot n $.
\end{cor}

\end{document}
