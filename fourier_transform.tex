\documentclass[a4paper,12pt]{article}
\usepackage{etex}
%%%%%%%%%%%%%%%%%%%%%%%%%%%%%%%%%%%%%%
% Babel language package
\usepackage[english,greek]{babel}
% Inputenc font encoding
\usepackage[utf8]{inputenc}
%%%%%%%%%%%%%%%%%%%%%%%%%%%%%%%%%%%%%%

%%%%% math packages %%%%%%%%%%%%%%%%%%
\usepackage{amsmath}
\usepackage{amssymb}
\usepackage{amsfonts}
\usepackage{amsthm}
\usepackage{proof}

\usepackage{physics}

%%%%%%% symbols packages %%%%%%%%%%%%%%
\usepackage{dsfont}
\usepackage{stmaryrd}
%%%%%%%%%%%%%%%%%%%%%%%%%%%%%%%%%%%%%%%


%%%%%% graphicx %%%%%%%%%%%%%%%%%%%%%%%
\usepackage{graphicx}
\usepackage{color}
%\usepackage{xypic}
\usepackage[all]{xy}
\usepackage{calc}
%%%%%%%%%%%%%%%%%%%%%%%%%%%%%%%%%%%%%%%

\usepackage{enumerate}

\usepackage{fancyhdr}
%%%%% header and footer rule %%%%%%%%%
\setlength{\headheight}{14pt}
\renewcommand{\headrulewidth}{0pt}
\renewcommand{\footrulewidth}{0pt}
\fancypagestyle{plain}{\fancyhf{}
\fancyhead{}
\lfoot{}
\rfoot{\small \thepage}}
\fancypagestyle{vangelis}{\fancyhf{}
\rhead{\small \leftmark}
\lhead{\small }
\lfoot{}
\rfoot{\small \thepage}}
%%%%%%%%%%%%%%%%%%%%%%%%%%%%%%%%%%%%%%%

\usepackage{hyperref}
\usepackage{url}
%%%%%%% hyperref settings %%%%%%%%%%%%
\hypersetup{pdfpagemode=UseOutlines,hidelinks,
bookmarksopen=true,
pdfdisplaydoctitle=true,
pdfstartview=Fit,
unicode=true,
pdfpagelayout=OneColumn,
}
%%%%%%%%%%%%%%%%%%%%%%%%%%%%%%%%%%%%%%



\usepackage{geometry}
\geometry{left=25.63mm,right=25.63mm,top=36.25mm,bottom=36.25mm,footskip=24.16mm,headsep=24.16mm}

%\usepackage[explicit]{titlesec}
%%%%%% titlesec settings %%%%%%%%%%%%%
%\titleformat{\chapter}[block]{\LARGE\sc\bfseries}{\thechapter.}{1ex}{#1}
%\titlespacing*{\chapter}{0cm}{0cm}{36pt}[0ex]
%\titleformat{\section}[block]{\Large\bfseries}{\thesection.}{1ex}{#1}
%\titlespacing*{\section}{0cm}{34.56pt}{17.28pt}[0ex]
%\titleformat{\subsection}[block]{\large\bfseries{\thesubsection.}{1ex}{#1}
%\titlespacing*{\subsection}{0pt}{28.80pt}{14.40pt}[0ex]
%%%%%%%%%%%%%%%%%%%%%%%%%%%%%%%%%%%%%%

%%%%%%%%% My Theorems %%%%%%%%%%%%%%%%%%
\newtheorem{thm}{Θεώρημα}[section]
\newtheorem{cor}[thm]{Πόρισμα}
\newtheorem{lem}[thm]{λήμμα}
\theoremstyle{definition}
\newtheorem{dfn}{Ορισμός}[section]
\newtheorem{dfns}[dfn]{Ορισμοί}
\theoremstyle{remark}
\newtheorem{remark}{Παρατήρηση}[section]
\newtheorem{remarks}[remark]{Παρατηρήσεις}
%%%%%%%%%%%%%%%%%%%%%%%%%%%%%%%%%%%%%%%




\newcommand{\vect}[2]{(#1_1,\ldots, #1_#2)}
%%%%%%% nesting newcommands $$$$$$$$$$$$$$$$$$$
\newcommand{\function}[1]{\newcommand{\nvec}[2]{#1(##1_1,\ldots, ##1_##2)}}

\newcommand{\linode}[2]{#1_n(x)#2^{(n)}+#1_{n-1}(x)#2^{(n-1)}+\cdots +#1_0(x)#2=g(x)}

\newcommand{\vecoffun}[3]{#1_0(#2),\ldots ,#1_#3(#2)}


\input{tikz.tex}

\DeclareMathOperator{\Si}{Si}

% \everymath{\displaystyle}

\begin{document}

\setcounter{chapter}{1}

\chapter*{Μετασχηματισμός Fourier}

\begin{dfn}
  Έστω $ f \colon (- \infty, \infty) \to \mathbb{R} $, τμηματικά λεία σε κάθε πεπερασμένο
  διάστημα και απολύτως ολοκληρώσιμη στο $ (- \infty, \infty) $. Τότε
  \begin{align*}
    \mathcal{F}[f(t)] &= \hat{f} (w) = \frac{1}{\sqrt{2 \pi}} \int _{-\infty} ^{\infty} 
    f(t) \mathrm{e}^{-iwt} \,{dt}
    \quad \text{\textcolor{Col1}{Μετασχηματισμός Fourier}} 
    \intertext{και}
    \mathcal{F}^{-1}[\hat{f} (w)] &= f(t) = \frac{1}{\sqrt{2 \pi}}
    \int _{- \infty} ^{\infty} \hat{f} (w) \mathrm{e}^{iwt} \,{dw}=e
    \quad \text{\textcolor{Col1}{Αντίστροφος Μετασχηματισμός
    Fourier}} 
  \end{align*} 
\end{dfn}

\begin{example}
  Να βρεθεί ο Μετασχηματισμός Fourier της συνάρτησης 
  $ f(t) = 
  \begin{cases}
    1, & \abs{t} \leq a \\
    0, & \abs{t} > a
  \end{cases}$
\end{example}
\begin{solution}
  \[ 
    \mathcal{F}[f(t)] = \frac{1}{\sqrt{2 \pi}} \int _{a}^{a} \mathrm{e}^{-iwt} \,{dt} =
    \frac{1}{\sqrt{2 \pi}} \left[\frac{\mathrm{e}^{-iwt}}{-iw} \right]_{-a}^{a} = 
    \frac{1}{\sqrt{2 \pi}} \left(\frac{\mathrm{e}^{-iat} - \mathrm{e}^{iat}}{-iw}\right) 
    = \frac{2}{\sqrt{2 \pi} w} \left(\frac{\mathrm{e}^{iaw} -
    \mathrm{e}^{-iaw}}{2i}\right) = \sqrt{\frac{2}{\pi}} \frac{\sin{(aw)}}{w}
  \]
  \begin{align*}
    \mathcal{F^{-1}}[\hat{f} (w)] 
    &= \frac{1}{\sqrt{2 \pi}} \int _{\infty}^{\infty}
    \sqrt{\frac{2}{\pi}} \frac{\sin{(aw)}}{w} \mathrm{e}^{iwt} \,{dw} =
    \frac{1}{\sqrt{2 \pi}} \sqrt{\frac{2}{\pi}} \int _{- \infty}^{\infty}
    \frac{\sin{(aw)}}{w} \mathrm{e}^{iwt} \,{dw} \\ 
    &= \frac{1}{\pi} \int _{- \infty}^{\infty} \frac{sin(aw) \cos{(wt)}}{w} + i 
    \frac{\sin{(aw)} \sin{(wt)}}{w} \,{dw} 
  \end{align*} 
  και επειδή $ \int _{- \infty}^{\infty} \frac{\sin{(aw)} \sin{(wt)}}{w} \,{dw} = 0 $ 
  για κάθε $t$, αφού η συνάρτηση είναι περιττή, έχουμε:
  \begin{align*}
    f(t) = \mathcal{F^{-1}}[f(t)] &= \frac{1}{\pi} 
    \int _{-\infty}^{\infty}\frac{\sin{(aw)} \cos{(wt)}}{w}, \quad \forall t \neq \pm a 
    \\
    \frac{f(a^{-})+f(a^{+})}{2} = \frac{1}{2} = \mathcal{F^{-1}}[f(t)] &= \frac{1}{\pi} 
    \int _{-\infty}^{\infty}\frac{\sin{(aw)} \cos{(aw)}}{w}, \quad t= \pm a
  \end{align*} 
  Οπότε, παίρνουμε το ολοκλήρωμα:
  \[
    \int _{- \infty}^{\infty} \frac{\sin{(aw)} \cos{(wt)}}{w} \,{dw} = 
    \begin{cases}
      \pi, & \abs{t} < a \\
      \frac{\pi}{2}, & \abs{t} = a \\
      0, & \abs{t} > a
    \end{cases}
  \] 
\end{solution}

\begin{example}
  Να βρεθεί ο Μετασχηματισμός Fourier της συνάρτησης 
  $ f(t) = 
  \begin{cases}
    \mathrm{e}^{-kt}, & t \geq 0 \\
    0, & {t} < 0
  \end{cases}$, με $ k>0 $.
\end{example}
\begin{solution}
  \begin{align*}
    \mathcal{F}\left\{f(t)\right\} 
    &= \frac{1}{\sqrt{2 \pi}} \int_{-\infty}^{+\infty}
    \mathrm{e}^{-kt} \mathrm{e}^{-iwt} \,{dt} = \frac{1}{\sqrt{2 \pi}}
    \int_{0}^{+\infty} \mathrm{e}^{-(k+iw)t} \,{dt} = \frac{1}{\sqrt{2 \pi}} 
    \left[\frac{\mathrm{e}^{-(k+iw)t}}{-(k+iw)}\right]_{0}^{\infty} \\
    &= \frac{1}{\sqrt{2 \pi}} \frac{k-iw}{k^{2}+w^{2}} \left(\cancelto{0}{-\lim_{t \to
    \infty} \mathrm{e}^{-(k+iw)t}} + 1\right) = \frac{1}{\sqrt{2 \pi}} 
    \frac{k-iw}{k^{2}+w^{2}} 
  \end{align*} 
  Ο λόγος, για τον οποίο το παραπάνω όριο είναι ίσο με 0, είναι:
  \[
    \abs{\mathrm{e}^{-(k+iw)t}} = \abs{\mathrm{e}^{-kt} \cdot \mathrm{e}^{-iwt}} =
    \mathrm{e}^{-kt} \abs{\cos{(wt)} - i \sin{(wt)}} = \mathrm{e}^{-kt}
    \sqrt{\cos^{2}{(wt)} + \sin^{2}{(wt)}} = \mathrm{e}^{-kt} 
  \] 
  Άρα 
  \[
    \lim_{t \to \infty} \abs{\mathrm{e}^{-(k+iw)t}} = 
    \lim_{t \to \infty} \mathrm{e}^{-kt} \overset{k>0}{=} 0 \Rightarrow 
    \lim_{t \to \infty} \mathrm{e}^{-(k+iw)t} = 0
  \] 
  γιατί, από γνωστή πρόταση 
  $ \lim_{x \to \infty} \abs{f(x)} = 0 \Rightarrow \lim_{x \to \infty} f(x) = 0 $.
\end{solution}

\begin{example}
  Να βρεθεί ο Μετασχηματισμός Fourier της συνάρτησης 
  $ f(t) = \mathrm{e}^{-a \abs{t}} = 
  \begin{cases}
    \mathrm{e}^{-at}, & t \geq 0 \\
    \mathrm{e}^{at} , & {t} < 0
  \end{cases}$, με $ a>0 $.
\end{example}
\begin{solution}
  \begin{align*}
    \mathcal{F}\left\{f(t)\right\} 
    &= \frac{1}{\sqrt{2 \pi}} \int_{-\infty}^{+\infty} 
    \mathrm{e}^{-a \abs{t}} \mathrm{e}^{-i\omega t} \,{dt} = \frac{1}{\sqrt{2 \pi}} 
    \left(\int_{-\infty}^{0} \mathrm{e}^{at} \mathrm{e}^{-i\omega t} \,{dt} +
    \int_{0}^{+\infty} \mathrm{e}^{-at} \mathrm{e}^{-i\omega t} \,{dt} \right) \\
    &= \frac{1}{\sqrt{2 \pi}} \left(\int _{- \infty}^{0} \mathrm{e}^{(a-i\omega)t} 
      \,{dt} + \int_{0}^{+\infty} \mathrm{e}^{-(a+i\omega)t} \,{dt}\right) 
    = \frac{1}{\sqrt{2 \pi}} \left(\left[\frac{\mathrm{e}^{(a-i\omega)t}}{a-i\omega}
        \right]_{- \infty}^{0} + \left[\frac{\mathrm{e}^{-(a+i\omega)t}}{-(a+i\omega)}
    \right]_{0}^{\infty} \right) \\
    &= \frac{1}{\sqrt{2 \pi}} \left(\frac{1- \cancelto{0}{\lim\limits_{t \to - \infty}
          \mathrm{e}^{(a-i\omega)t}}}{a-i\omega} + \frac{\cancelto{0}{\lim\limits_{t \to \infty}
\mathrm{e}^{-(a+i\omega)t}} -1 }{-(a+i\omega)}  \right) \\
    &= \frac{1}{\sqrt{2 \pi}} \left(\frac{1}{a- i\omega} + \frac{1}{a+i \omega}\right) 
    = \frac{1}{\sqrt{2 \pi}} \frac{a+i\omega+a-i\omega}{a^{2}-(i\omega)^{2}} = 
    \frac{1}{\sqrt{2 \pi}} \frac{2a}{a^{2}+\omega^{2}} = \sqrt{\frac{2}{\pi}}
    \frac{a}{a^{2}+\omega^{2}} 
  \end{align*}
\end{solution}



\end{document}
