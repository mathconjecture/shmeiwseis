\documentclass[a4paper,12pt]{article}
\usepackage{etex}
%%%%%%%%%%%%%%%%%%%%%%%%%%%%%%%%%%%%%%
% Babel language package
\usepackage[english,greek]{babel}
% Inputenc font encoding
\usepackage[utf8]{inputenc}
%%%%%%%%%%%%%%%%%%%%%%%%%%%%%%%%%%%%%%

%%%%% math packages %%%%%%%%%%%%%%%%%%
\usepackage{amsmath}
\usepackage{amssymb}
\usepackage{amsfonts}
\usepackage{amsthm}
\usepackage{proof}

\usepackage{physics}

%%%%%%% symbols packages %%%%%%%%%%%%%%
\usepackage{dsfont}
\usepackage{stmaryrd}
%%%%%%%%%%%%%%%%%%%%%%%%%%%%%%%%%%%%%%%


%%%%%% graphicx %%%%%%%%%%%%%%%%%%%%%%%
\usepackage{graphicx}
\usepackage{color}
%\usepackage{xypic}
\usepackage[all]{xy}
\usepackage{calc}
%%%%%%%%%%%%%%%%%%%%%%%%%%%%%%%%%%%%%%%

\usepackage{enumerate}

\usepackage{fancyhdr}
%%%%% header and footer rule %%%%%%%%%
\setlength{\headheight}{14pt}
\renewcommand{\headrulewidth}{0pt}
\renewcommand{\footrulewidth}{0pt}
\fancypagestyle{plain}{\fancyhf{}
\fancyhead{}
\lfoot{}
\rfoot{\small \thepage}}
\fancypagestyle{vangelis}{\fancyhf{}
\rhead{\small \leftmark}
\lhead{\small }
\lfoot{}
\rfoot{\small \thepage}}
%%%%%%%%%%%%%%%%%%%%%%%%%%%%%%%%%%%%%%%

\usepackage{hyperref}
\usepackage{url}
%%%%%%% hyperref settings %%%%%%%%%%%%
\hypersetup{pdfpagemode=UseOutlines,hidelinks,
bookmarksopen=true,
pdfdisplaydoctitle=true,
pdfstartview=Fit,
unicode=true,
pdfpagelayout=OneColumn,
}
%%%%%%%%%%%%%%%%%%%%%%%%%%%%%%%%%%%%%%



\usepackage{geometry}
\geometry{left=25.63mm,right=25.63mm,top=36.25mm,bottom=36.25mm,footskip=24.16mm,headsep=24.16mm}

%\usepackage[explicit]{titlesec}
%%%%%% titlesec settings %%%%%%%%%%%%%
%\titleformat{\chapter}[block]{\LARGE\sc\bfseries}{\thechapter.}{1ex}{#1}
%\titlespacing*{\chapter}{0cm}{0cm}{36pt}[0ex]
%\titleformat{\section}[block]{\Large\bfseries}{\thesection.}{1ex}{#1}
%\titlespacing*{\section}{0cm}{34.56pt}{17.28pt}[0ex]
%\titleformat{\subsection}[block]{\large\bfseries{\thesubsection.}{1ex}{#1}
%\titlespacing*{\subsection}{0pt}{28.80pt}{14.40pt}[0ex]
%%%%%%%%%%%%%%%%%%%%%%%%%%%%%%%%%%%%%%

%%%%%%%%% My Theorems %%%%%%%%%%%%%%%%%%
\newtheorem{thm}{Θεώρημα}[section]
\newtheorem{cor}[thm]{Πόρισμα}
\newtheorem{lem}[thm]{λήμμα}
\theoremstyle{definition}
\newtheorem{dfn}{Ορισμός}[section]
\newtheorem{dfns}[dfn]{Ορισμοί}
\theoremstyle{remark}
\newtheorem{remark}{Παρατήρηση}[section]
\newtheorem{remarks}[remark]{Παρατηρήσεις}
%%%%%%%%%%%%%%%%%%%%%%%%%%%%%%%%%%%%%%%




\newcommand{\vect}[2]{(#1_1,\ldots, #1_#2)}
%%%%%%% nesting newcommands $$$$$$$$$$$$$$$$$$$
\newcommand{\function}[1]{\newcommand{\nvec}[2]{#1(##1_1,\ldots, ##1_##2)}}

\newcommand{\linode}[2]{#1_n(x)#2^{(n)}+#1_{n-1}(x)#2^{(n-1)}+\cdots +#1_0(x)#2=g(x)}

\newcommand{\vecoffun}[3]{#1_0(#2),\ldots ,#1_#3(#2)}



\pagestyle{vangelis}
\everymath{\displaystyle}


\begin{document}

\chapter{Συνέχεια Συναρτήσεων δυο Μεταβλητών}

\section{Ορισμός}

\mydfn{ Η συνάρτηση $ f \colon A \subseteq \mathbb{R}^{2} \to \mathbb{R} $ λέγεται συνεχής στο σημείο $ (x_{0}, y_{0}) \in A $ αν 
$ \lim\limits_{(x,y)\to (x_{0}, y_{0})} f(x,y) = f(x_{0}, y_{0}) $.}

\mydfn{Η συνάρτηση $ f \colon A \subseteq \mathbb{R}^{2} \to \mathbb{R} $ λέγεται συνεχής στο Α, αν
είναι συνεχής σε κάθε σημείο $ (x_{0}, y_{0}) \in A $.}

\begin{rem}
    Από τον παραπάνω ορισμό, καταλαβαίνουμε ότι μια συνάρτηση $ f(x,y) $ είναι συνεχής στο σημείο $
    (x_{0}, y_{0}) $, αν
    \begin{myitemize}
        \item Υπάρχει το $ \lim\limits_{\substack{x\to x_{0} \\y \to y_{0}}} f(x,y) $
        \item Υπάρχει το $ f(x_{0}, y_{0}) \in \mathbb{R} $
        \item $  \lim\limits_{\substack{x\to x_{0} \\y \to y_{0}}} f(x,y)= f(x_{0}, y_{0}) $
    \end{myitemize}
\end{rem}

\begin{examples}
\item {}
    \begin{enumerate}
        \item Να εξετάσετε αν οι παρακάτω συναρτήσεις μπορούν να οριστούν στην αρχή των αξόνων, ώστε να είναι συνεχείς σε αυτήν.
            \begin{enumerate}[i)]
                \item $ f(x,y) = \frac{x^{2}+y^{4}}{x^{4}+y^{2}} $
                    \begin{solution}
                    \item {}
                        Προφανώς η $f$ είναι συνεχής, ως ρητή σε κάθε σημείο $ (x_{0}, y_{0}) \neq (0,0) $.

                        Εξετάζουμε τη συνέχεια της $f$ στο σημείο $ (0,0) $. Έχουμε:
                        \[
                            \lim\limits_{\substack{x\to 0 \\y \to 0}} \frac{x^{2}+y^{4}}{x^{4}+y^{2}} = \lim_{r \to 0} \frac{r^{2}
                                \cos^{2}{\theta} + r^{4} \sin^{4}{\theta}}{r^{4} \cos^{4}{\theta} +
                            r^{2} \sin^{2}{\theta}} =
                            \lim_{r \to 0} \frac{\cos^{2}{\theta} + r^{2} \sin^{2}{\theta}}{r^{2}
                            \cos^{2}{\theta} + \sin^{2}{\theta}} =
                            \frac{\cos^{2}{\theta}}{\sin^{2}{\theta}}
                        \] 
                        Επομένως δεν υπάρχει το όριο της $f$ στο $ (0,0) $, γιατί εξαρτάται από το
                        $\theta$, άρα η $f$ δεν είναι 
                        συνεχής στο $ (0,0) $.

                    \end{solution}

                \item $ g(x,y) = \frac{x^{2}y^{2}}{x^{2}y^{2}+ (x+y)^{2}} $
                    \begin{solution}
                    \item {}
                        Προφανώς η $g$ είναι συνεχής, ως ρητή σε κάθε σημείο $ (x_{0}, y_{0}) \neq (0,0) $.

                        Εξετάζουμε τη συνέχεια της $g$ στο σημείο $ (0,0) $. Έχουμε:
                        \[
                            \lim\limits_{\substack{x\to 0 \\y \to 0}}
                            \frac{x^{2}y^{2}}{x^{2}y^{2}+(x+y)^{2}} = \lim_{r \to 0}
                            \frac{r^{4} \cos^{2}{\theta} \sin^{2}{\theta}}{r^{4} \cos^{2}{\theta}
                            \sin^{2}{\theta} + r^{2}(\cos{\theta} + \sin{\theta})^{2}} = 0
                        \] 
                        Επομένως, η $ g(x,y) $ μπορεί να οριστεί στην αρχή των αξόνων, ως $ g(0,0)
                        = 0 $ ώστε να είναι συνεχής.

                        Δηλαδή η συνάρτηση 
                        \[
                            g(x,y) = \begin{cases} \frac{x^{2}y^{2}}{x^{2}y^{2} + (x+y)^{2}}, & (x,y) \neq
                                (0,0) \\ 
                            0, & (x,y) = (0,0)\end{cases}  
                        \] είναι συνεχής.
                            \end{solution}
                    \end{enumerate}
                \item Να μελετηθεί πλήρως ως προς τη συνέχεια η συνάρτηση 
                    \[
                        f(x,y) = \begin{cases} \frac{xy}{x^{2}+y^{2}}, & (x,y) \neq (0,0) \\ 0, & (x,y) =
                        (0,0) \end{cases}  
                    \]
                    \begin{solution}
                    \item {}              
                        Αν $ (x,y) \neq (0,0) $ τότε η $f$ είναι συνεχής 
                        ως ρητή συνάρτηση.

                        Αν $ (x,y)=(0,0) $, εξετάζουμε το όριο της $f$: 
                        \[
                            \lim\limits_{\substack{x\to 0 \\y \to 0}} \frac{xy}{x^{2}+y^{2}} = \lim_{r \to 0}
                            \frac{r^{2} \cos{\theta} \sin{\theta} }{ r^{2}} = \lim_{r \to 0}(\cos{\theta}
                            \sin{\theta})
                        \] 
                        Επομένως δεν υπάρχει το όριο της $f$ στο σημείο $ (0,0) $, άρα η $f$ δεν είναι συνεχής
                        στο $ (0,0) $.
                    \end{solution}

                \item Να μελετηθεί πλήρως ως προς τη συνέχεια η συνάρτηση 
                    \[
                        g(x,y) = \begin{cases} y^{2}x \sin{\frac{1}{x^{2}+y^{2}}}, & (x,y) \neq (0,0) \\ 0, & (x,y)
                        = (0,0)\end{cases}  
                    \] 
                    \begin{solution}
                    \item {}
                        Αν $ (x,y) \neq (0,0) $ τότε η $g$ είναι συνεχής 
                        ως ρητή συνάρτηση.

                        Αν $ (x,y)=(0,0) $, εξετάζουμε το όριο της $g$. 
                        Έχουμε:
                        \[
                            \abs{y^{2}x \sin{\frac{1}{x^{2}+y^{2}}}} = \abs{y^{2}x}\cdot \abs{\sin{\frac{1}{x^{2}+y^{2}}}}
                            \leq \abs{y^{2}x} \cdot 1 = \abs{y^{2}x}  
                        \] 
                        και 
                        \[
                            \lim\limits_{\substack{x\to 0 \\y \to 0}} y^{2}x = 0 
                        \] 
                        επομένως από το θεώρημα, έχουμε ότι $ \lim\limits_{\substack{x\to 0 \\y \to 0}} y^{2}x
                        \sin{\frac{1}{x^{2}+y^{2}}} = 0 = f(0,0) $. Άρα η $f$ είναι συνεχής.
                    \end{solution}

                \item Να εξεταστεί πλήρως ως προς τη συνέχεια η συνάρτηση 
                \[
                    f(x,y) = \begin{cases} e^{-\frac{x^{2}}{y}}, & y \neq 0 \\ 0, & y= 0 \end{cases}  
                 \] 
                 \begin{solution}
                 \item {}
                    Για $ y \neq 0 $ η $f$ είναι συνεχής ως σύνθεση και πράξεις συνεχών συναρτήσεων. 

                    Για $ y =0 $, έχουμε:
                    \begin{myitemize}
                    \item Αν $ x \neq 0 $, τότε:
                        \[
                            \lim_{y \to 0^{+}} e^{- \frac{x^{2}}{y}} \overset{e^{- \infty}}{=} 0 
                         \] 
                         ενώ 
                         \[
                             \lim_{y \to 0^{-}} e^{- \frac{x^{2}}{y}} \overset{e^{\infty}}{=} \infty
                          \]
                      \item Αν $ x = 0 $, τότε: 
                          \[
                              f(0,y) = \begin{cases} 1, & y \neq 0 \\ 0, & y= 0 \end{cases} 
                           \] 
                           Άρα η $f$ δεν είναι συνεχής στον άξονα $x$, δηλαδή για $ y=0 $.
                    \end{myitemize}
                 \end{solution}

             \item Να δείξετε ότι η συνάρτηση 
                 \[
                     f(x,y) = \begin{cases} \frac{1 - \cos{\sqrt{xy}}}{x}, & x \neq 0 \\
                     \frac{y}{2}, & x=0 \end{cases} 
                  \] 
                  είναι συνεχής σε όλα τα σημεία του άξονα $y$.
                  \begin{solution}
                  \item {}
                      Έστω $ (0, y_{0}) $ ένα τυχαίο σημείο 
                      του άξονα $y$. 
                      \[
                          \lim\limits_{\substack{x\to 0 \\y \to y_0}} \frac{1 - \cos{\sqrt{xy}}}{x} =
                          \lim\limits_{\substack{x\to 0 \\y \to y_0}} \frac{2
                          \sin^{2}{\frac{\sqrt{xy}}{2}}}{x} = \lim\limits_{\substack{x\to 0 \\y
                      \to y_0}} \frac{y}{2} \frac{2\sin^{2}{\frac{\sqrt{xy}}{2}}}{\frac{xy}{2}} =
                      \lim\limits_{\substack{x\to 0 \\y \to y_0}} \frac{y}{2} \cdot
                      \lim\limits_{\substack{x\to 0 \\y \to y_0}} \left(\frac{\sin^{2}{\frac{\sqrt{xy} }{2}
                      }}{\frac{\sqrt{xy}}{2}} \right)^{2} = \frac{y_0}{2} \cdot (1)^{2} =
                      \frac{y_{0}}{2} 
                  \] 
                  και 
                  $ f(0, y_{0}) = \frac{y_{0}}{2} $.

                  Επομένως η $f$ είναι συνεχής σε κάθε σημείο του άξονα $y$.

                  \end{solution}
                    \end{enumerate}
                        \end{examples}


\end{document}
