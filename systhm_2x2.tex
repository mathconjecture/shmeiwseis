\documentclass[a4paper,table]{report}
\input{preamble_ask.tex}
\input{definitions_ask.tex}
\input{myboxes.tex}

\pgfdeclarelayer{bg}
\pgfdeclarelayer{fg}
\pgfsetlayers{bg,main,fg}

\renewcommand{\qedsymbol}{}

\pagestyle{askhseis}


\begin{document}

\begin{center}
  \minibox{\large\bfseries \textcolor{Col1}{Γραμμικά Συστήματα 2 x 2}}
\end{center}

\vspace{\baselineskip}


\section*{Μέθοδος Αντικατάστασης}

\begin{mybox2}
  \item {}
\begin{myitemize}
  \item Λύνουμε μία από τις εξισώσεις του συστήματος, ως προς έναν άγνωστο.
  \item Αντικαθιστούμε στην άλλη εξίσωση του συστήματος τον άγνωστο αυτόν και έτσι 
    προκύπτει μία νέα εξίσωση με έναν μόνο άγνωστο, την οποία και λύνουμε.
  \item Την τιμή του αγνώστου που βρήκαμε την αντικαθιστούμε στην ισότητα που προέκυψε
    στο πρώτο βήμα και βρίσκουμε τον άλλον άγνωστο.
\end{myitemize}
\end{mybox2}

\begin{example} 
  Να λυθεί το παρακάτω σύστημα με τη μέθοδο της αντικατάστασης.
  \[
    \left.
      \begin{matrix}
        2x+y=1 \\
        3x+2y=0
      \end{matrix} 
    \right\} 
  \] 
\end{example}
\begin{solution}
  Λύνουμε την 1η εξίσωση ως προς $y$
  \[
    2x+y=1 \Rightarrow y=1-2x 
  \]
  Αντικαθιστούμε την $ y=1-2x $ στη 2η εξίσωση του συστήματος και έχουμε:
  \[
    3x+2(1-2x) = 0 \Leftrightarrow 3x+2-4x=0 \Leftrightarrow -x+2=0 \Leftrightarrow x=2 
  \]
  Από την σχέση $ y= 1-2x $ για $ x=2 $ έχουμε:
  \[
    y=1-2\cdot 2 \Leftrightarrow y = 1-4 \Leftrightarrow y = -3 
  \] 
  Άρα το σύστημα έχει μοναδική λύση, την $ x=2 $ και $ y=-3 $.
\end{solution}


\section*{Μέθοδος Αντίθετων Συντελεστών}

\begin{mybox2}
\item {}
\begin{myitemize}
  \item Πολλαπλασιάζουμε τα δύο μέλη κάθε εξίσωσης με κατάλληλους αριθμούς, ώστε σε έναν
    από τους αγνώστους να προκύψουν αντίθετοι συντελεστές.
  \item Προσθέτουμε κατά μέλη τις δύο εξισώσεις που προέκυψαν και έτσι απαλοίφουμε τον
    άγνωστο με τους αντίθετους συντελεστές. Προκύπτει, έτσι μία εξίσωση με έναν άγνωστο 
    την οποία και λύνουμε.
  \item Αντικαθιστούμε την τιμή του αγνώστου που βρήκαμε σε μία από τις εξισώσεις του 
    συστήματος και βρίσκουμε και τον άλλον άγνωστο.
\end{myitemize}
\end{mybox2}

\begin{example}
  Να λυθεί το παρακάτω σύστημα με τη μέθοδο των αντίθετων συντελεστών.
  \[
     \left.
       \begin{matrix}
         2x+3y=4 \\
         3x+4y=5
       \end{matrix} 
     \right\} 
   \] 
\end{example}
\begin{solution}
  Θα προσπαθήσουμε να εμφανίσουμε αντίθετους συντελεστές στον άγνωστο $x$. 

  Γι᾽ αυτό πολλαπλασιάζουμε την 1η εξίσωση με 3 και τη 2η εξίσωση με -2, και έχουμε:
  \[
    \left.
      \begin{matrix}
        2x+3y=4 \\
        3x+4y=5 
      \end{matrix} 
    \right. 
    \left.
      \begin{matrix*}[l]
        \textcolor{Col1}{| \cdot (3)} \\
        \textcolor{Col1}{| \cdot (-2)}
      \end{matrix*} 
    \right\} \Leftrightarrow 
    \left.
      \begin{matrix}
        \cancel{6x}+9y=12 \\
        -\cancel{6x}-8y=-10
      \end{matrix} 
    \right\} \overset{(+)}{\Rightarrow} 
    y=2
  \]
  Η 1η εξίσωση για $ y=2 $ γίνεται:
  \[
    2x+3\cdot 2 = 4 \Leftrightarrow 2x+6=4 \Leftrightarrow 2x=-2 \Leftrightarrow
    x=-1 
  \]
  Άρα το σύστημα έχει μοναδική λύση, την $ x=-1 $ και $ y=2 $.
\end{solution}


%todo να προσθέσω παραδείγματα με απειρες λυσεις και αδυνατα

\end{document}


