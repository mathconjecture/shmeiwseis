\documentclass[a4paper,12pt]{article}
\usepackage{etex}
%%%%%%%%%%%%%%%%%%%%%%%%%%%%%%%%%%%%%%
% Babel language package
\usepackage[english,greek]{babel}
% Inputenc font encoding
\usepackage[utf8]{inputenc}
%%%%%%%%%%%%%%%%%%%%%%%%%%%%%%%%%%%%%%

%%%%% math packages %%%%%%%%%%%%%%%%%%
\usepackage{amsmath}
\usepackage{amssymb}
\usepackage{amsfonts}
\usepackage{amsthm}
\usepackage{proof}

\usepackage{physics}

%%%%%%% symbols packages %%%%%%%%%%%%%%
\usepackage{dsfont}
\usepackage{stmaryrd}
%%%%%%%%%%%%%%%%%%%%%%%%%%%%%%%%%%%%%%%


%%%%%% graphicx %%%%%%%%%%%%%%%%%%%%%%%
\usepackage{graphicx}
\usepackage{color}
%\usepackage{xypic}
\usepackage[all]{xy}
\usepackage{calc}
%%%%%%%%%%%%%%%%%%%%%%%%%%%%%%%%%%%%%%%

\usepackage{enumerate}

\usepackage{fancyhdr}
%%%%% header and footer rule %%%%%%%%%
\setlength{\headheight}{14pt}
\renewcommand{\headrulewidth}{0pt}
\renewcommand{\footrulewidth}{0pt}
\fancypagestyle{plain}{\fancyhf{}
\fancyhead{}
\lfoot{}
\rfoot{\small \thepage}}
\fancypagestyle{vangelis}{\fancyhf{}
\rhead{\small \leftmark}
\lhead{\small }
\lfoot{}
\rfoot{\small \thepage}}
%%%%%%%%%%%%%%%%%%%%%%%%%%%%%%%%%%%%%%%

\usepackage{hyperref}
\usepackage{url}
%%%%%%% hyperref settings %%%%%%%%%%%%
\hypersetup{pdfpagemode=UseOutlines,hidelinks,
bookmarksopen=true,
pdfdisplaydoctitle=true,
pdfstartview=Fit,
unicode=true,
pdfpagelayout=OneColumn,
}
%%%%%%%%%%%%%%%%%%%%%%%%%%%%%%%%%%%%%%



\usepackage{geometry}
\geometry{left=25.63mm,right=25.63mm,top=36.25mm,bottom=36.25mm,footskip=24.16mm,headsep=24.16mm}

%\usepackage[explicit]{titlesec}
%%%%%% titlesec settings %%%%%%%%%%%%%
%\titleformat{\chapter}[block]{\LARGE\sc\bfseries}{\thechapter.}{1ex}{#1}
%\titlespacing*{\chapter}{0cm}{0cm}{36pt}[0ex]
%\titleformat{\section}[block]{\Large\bfseries}{\thesection.}{1ex}{#1}
%\titlespacing*{\section}{0cm}{34.56pt}{17.28pt}[0ex]
%\titleformat{\subsection}[block]{\large\bfseries{\thesubsection.}{1ex}{#1}
%\titlespacing*{\subsection}{0pt}{28.80pt}{14.40pt}[0ex]
%%%%%%%%%%%%%%%%%%%%%%%%%%%%%%%%%%%%%%

%%%%%%%%% My Theorems %%%%%%%%%%%%%%%%%%
\newtheorem{thm}{Θεώρημα}[section]
\newtheorem{cor}[thm]{Πόρισμα}
\newtheorem{lem}[thm]{λήμμα}
\theoremstyle{definition}
\newtheorem{dfn}{Ορισμός}[section]
\newtheorem{dfns}[dfn]{Ορισμοί}
\theoremstyle{remark}
\newtheorem{remark}{Παρατήρηση}[section]
\newtheorem{remarks}[remark]{Παρατηρήσεις}
%%%%%%%%%%%%%%%%%%%%%%%%%%%%%%%%%%%%%%%




\newcommand{\vect}[2]{(#1_1,\ldots, #1_#2)}
%%%%%%% nesting newcommands $$$$$$$$$$$$$$$$$$$
\newcommand{\function}[1]{\newcommand{\nvec}[2]{#1(##1_1,\ldots, ##1_##2)}}

\newcommand{\linode}[2]{#1_n(x)#2^{(n)}+#1_{n-1}(x)#2^{(n-1)}+\cdots +#1_0(x)#2=g(x)}

\newcommand{\vecoffun}[3]{#1_0(#2),\ldots ,#1_#3(#2)}


\input{tikz.tex}
\input{myboxes.tex}


\begin{document}

\chapter*{Επιφανειακό Ολοκλήρωμα ΙΙου Είδους}

\setcounter{chapter}{1}

\section{Ορισμός}

Έστω $ S $ μια απλή, λεία και προσανατολισμένη επιφάνεια του $ \mathbb{R}^{3} $, 
με διανυσματική παραμετρική εξίσωση $ \mathbf{r}(u,v) =
x(u,v)\mathbf{i}+y(u,v)\mathbf{j}+z(u,v)\mathbf{k}, $ όπου $ (u,v) \in D \subseteq
\mathbb{R}^{2} $. Υποθέτουμε ακόμα, ότι $ \mathbf{F}(x,y,z) =
P(x,y,z)\mathbf{i}+Q(x,y,z)\mathbf{j}+R(x,y,z)\mathbf{k} $ είναι μία διανυσματική 
συνάρτηση, συνεχής ορισμένη πάνω στην επιφάνεια $S$. Ορίζουμε το 
\textcolor{Col1}{επιφανειακό ολοκλήρωμα ΙΙου είδους} της διανυσματικής συνάρτησης 
$ \mathbf{F}(x,y,z) $ πάνω στην επιφάνεια $S$,
\[
  \iint_{S} \mathbf{F}(x,y,z) \, d \mathbf{S}  = \pm \iint_{D} 
  \mathbf{F}(\mathbf{r}(u,v)) \cdot (\mathbf{r}_{u} \times \mathbf{r}_{v}) \, dudv   
\]

\begin{rems}
\item {}
  \begin{enumerate}
    \item Το πρόσημο $ + $ αντιστοιχεί στην περίπτωση που τα διανύσματα 
      $ \mathbf{\widehat{n}} $ και $ \mathbf{r}_{u} \times \mathbf{r}_{v} $ 
      είναι ομόρροπα και το πρόσημο $ - $ αντιστοιχεί στην περίπτωση που είναι 
      αντίρροπα. Με $ \mathbf{\widehat{n}} $ συμβολίζουμε το μοναδιαίο κάθετο διάνυσμα 
      στο σημείο $ \mathbf{r}(u,v) $ της επιφάνειας $S$.
    \item Το διάνυσμα $ d \mathbf{S} $ παριστάνει το στοιχειώδες διανυσματικό εμβαδό 
      της επιφάνειας $S$, που είναι ίσο με $ d \mathbf{S} = \mathbf{r}_{u} \times
      \mathbf{r}_{v} du dv $. Ισχύει επίσης, ότι $ d \mathbf{S} = \mathbf{\widehat{n}}
      \cdot d \mathbf{S} $ και τότε ισχύει ο τύπος 
      \[
        \iint_{S} \mathbf{F}(x,y,z) \, d \mathbf{S} = \iint_{S} \mathbf{F} \cdot
        \mathbf{\widehat{n}} \,{dS}
      \] 
      ο οποίος μετασχηματίζει το επιφανειακό ολοκλήρωμα ΙΙου είδους σε ένα επιφανειακό 
      ολοκλήρωμα Ιου είδους. Το ολοκλήρωμα $ \iint_{S} \mathbf{F}\cdot
      \mathbf{\widehat{n}} \,{dS} $ ονομάζεται \textcolor{Col1}{ροή} του 
      διανυσματικού πεδίου $ \mathbf{F} $, δια μέσου της επιφάνειας $S$.
    \item Το επιφανειακό ολοκλήρωμα ΙΙου είδους, συμβολίζεται και ως 
      \[
        \iint_{S} Pdydz + Qdxdz + Rdxdy 
      \] 
  \end{enumerate}
\end{rems} 

\begin{rem}
  Το επιφανειακό ολοκλήρωμα ΙΙου είδους, εξαρτάται από τον προσανατολισμό της επιφάνειας.
  Επομένως, αν $S$ και $ S' $ είναι οι δύο όψεις της επιφάνειας, τότε ισχύει ότι 
  \[
    \iint_{S} \mathbf{F} \,{d \mathbf{S}} = - \iint_{S'} \mathbf{F} \,{d \mathbf{S}} 
   \] 
\end{rem}
%todo να εξηγήσω καλύτερα τον προσανατολισμό επιφανειών.
  
\end{document}
