\documentclass[a4paper,12pt]{article}
\usepackage{etex}
%%%%%%%%%%%%%%%%%%%%%%%%%%%%%%%%%%%%%%
% Babel language package
\usepackage[english,greek]{babel}
% Inputenc font encoding
\usepackage[utf8]{inputenc}
%%%%%%%%%%%%%%%%%%%%%%%%%%%%%%%%%%%%%%

%%%%% math packages %%%%%%%%%%%%%%%%%%
\usepackage{amsmath}
\usepackage{amssymb}
\usepackage{amsfonts}
\usepackage{amsthm}
\usepackage{proof}

\usepackage{physics}

%%%%%%% symbols packages %%%%%%%%%%%%%%
\usepackage{dsfont}
\usepackage{stmaryrd}
%%%%%%%%%%%%%%%%%%%%%%%%%%%%%%%%%%%%%%%


%%%%%% graphicx %%%%%%%%%%%%%%%%%%%%%%%
\usepackage{graphicx}
\usepackage{color}
%\usepackage{xypic}
\usepackage[all]{xy}
\usepackage{calc}
%%%%%%%%%%%%%%%%%%%%%%%%%%%%%%%%%%%%%%%

\usepackage{enumerate}

\usepackage{fancyhdr}
%%%%% header and footer rule %%%%%%%%%
\setlength{\headheight}{14pt}
\renewcommand{\headrulewidth}{0pt}
\renewcommand{\footrulewidth}{0pt}
\fancypagestyle{plain}{\fancyhf{}
\fancyhead{}
\lfoot{}
\rfoot{\small \thepage}}
\fancypagestyle{vangelis}{\fancyhf{}
\rhead{\small \leftmark}
\lhead{\small }
\lfoot{}
\rfoot{\small \thepage}}
%%%%%%%%%%%%%%%%%%%%%%%%%%%%%%%%%%%%%%%

\usepackage{hyperref}
\usepackage{url}
%%%%%%% hyperref settings %%%%%%%%%%%%
\hypersetup{pdfpagemode=UseOutlines,hidelinks,
bookmarksopen=true,
pdfdisplaydoctitle=true,
pdfstartview=Fit,
unicode=true,
pdfpagelayout=OneColumn,
}
%%%%%%%%%%%%%%%%%%%%%%%%%%%%%%%%%%%%%%



\usepackage{geometry}
\geometry{left=25.63mm,right=25.63mm,top=36.25mm,bottom=36.25mm,footskip=24.16mm,headsep=24.16mm}

%\usepackage[explicit]{titlesec}
%%%%%% titlesec settings %%%%%%%%%%%%%
%\titleformat{\chapter}[block]{\LARGE\sc\bfseries}{\thechapter.}{1ex}{#1}
%\titlespacing*{\chapter}{0cm}{0cm}{36pt}[0ex]
%\titleformat{\section}[block]{\Large\bfseries}{\thesection.}{1ex}{#1}
%\titlespacing*{\section}{0cm}{34.56pt}{17.28pt}[0ex]
%\titleformat{\subsection}[block]{\large\bfseries{\thesubsection.}{1ex}{#1}
%\titlespacing*{\subsection}{0pt}{28.80pt}{14.40pt}[0ex]
%%%%%%%%%%%%%%%%%%%%%%%%%%%%%%%%%%%%%%

%%%%%%%%% My Theorems %%%%%%%%%%%%%%%%%%
\newtheorem{thm}{Θεώρημα}[section]
\newtheorem{cor}[thm]{Πόρισμα}
\newtheorem{lem}[thm]{λήμμα}
\theoremstyle{definition}
\newtheorem{dfn}{Ορισμός}[section]
\newtheorem{dfns}[dfn]{Ορισμοί}
\theoremstyle{remark}
\newtheorem{remark}{Παρατήρηση}[section]
\newtheorem{remarks}[remark]{Παρατηρήσεις}
%%%%%%%%%%%%%%%%%%%%%%%%%%%%%%%%%%%%%%%




\newcommand{\vect}[2]{(#1_1,\ldots, #1_#2)}
%%%%%%% nesting newcommands $$$$$$$$$$$$$$$$$$$
\newcommand{\function}[1]{\newcommand{\nvec}[2]{#1(##1_1,\ldots, ##1_##2)}}

\newcommand{\linode}[2]{#1_n(x)#2^{(n)}+#1_{n-1}(x)#2^{(n-1)}+\cdots +#1_0(x)#2=g(x)}

\newcommand{\vecoffun}[3]{#1_0(#2),\ldots ,#1_#3(#2)}


\input{tikz.tex}
\input{myboxes.tex}

\pagestyle{askhseis}

\begin{document}

\chapter*{Επιφανειακό Ολοκλήρωμα ΙΙου Είδους}

\setcounter{chapter}{1}

\section{Ορισμός}

Έστω $ S $ μια απλή, λεία και \textbf{προσανατολισμένη} επιφάνεια του 
$ \mathbb{R}^{3} $, με διανυσματική παραμετρική εξίσωση $ \mathbf{r}(u,v) =
x(u,v)\mathbf{i}+y(u,v)\mathbf{j}+z(u,v)\mathbf{k}, $ όπου $ (u,v) \in D \subseteq
\mathbb{R}^{2} $. Υποθέτουμε ακόμα, ότι $ \mathbf{F}(x,y,z) =
P(x,y,z)\mathbf{i}+Q(x,y,z)\mathbf{j}+R(x,y,z)\mathbf{k} $ είναι μία διανυσματική 
συνάρτηση, συνεχής, ορισμένη πάνω στην επιφάνεια $S$. Ορίζουμε το 
\textcolor{Col1}{επιφανειακό ολοκλήρωμα ΙΙου είδους} της διανυσματικής συνάρτησης 
$ \mathbf{F}(x,y,z) $ πάνω στην επιφάνεια $S$,
\[
  \iint_{S} \mathbf{F}(x,y,z) \cdot \mathbf{dS}  = \pm \iint_{D} 
  \mathbf{F}(\mathbf{r}(u,v)) \cdot (\mathbf{r}_{u} \times \mathbf{r}_{v}) \, dudv   
\]

\begin{rems}
\item {}
  \begin{enumerate}
    \item Το πρόσημο $ + $ αντιστοιχεί στην περίπτωση που τα διανύσματα 
      $ \mathbf{\widehat{n}} $ και $ \mathbf{r}_{u} \times \mathbf{r}_{v} $ 
      είναι ομόρροπα και το πρόσημο $ - $ αντιστοιχεί στην περίπτωση που είναι 
      αντίρροπα. Με $ \mathbf{\widehat{n}} $ συμβολίζουμε το \textbf{μοναδιαίο} κάθετο 
      διάνυσμα στο σημείο $ \mathbf{r}(u,v) $ της επιφάνειας $S$.
    \item Το διάνυσμα $ \mathbf{dS} $ παριστάνει το στοιχειώδες διανυσματικό εμβαδό 
      της επιφάνειας $S$, που είναι ίσο με $ \mathbf{dS} = (\mathbf{r}_{u} \times
      \mathbf{r}_{v})\, du dv $. Ισχύει επίσης, ότι $ \mathbf{dS} = \mathbf{\widehat{n}}
      \,dS $ και τότε ο τύπος 
      \begin{equation}\label{eq:flux1}
        \boxed{\iint_{S} \mathbf{F}(x,y,z) \cdot \mathbf{dS} = \iint_{S} \mathbf{F} \cdot
        \mathbf{\widehat{n}} \,{dS}}
      \end{equation} 
      μετασχηματίζει το επιφανειακό ολοκλήρωμα ΙΙου είδους σε ένα επιφανειακό 
      ολοκλήρωμα \textbf{Ιου είδους}. Το ολοκλήρωμα $ \iint_{S} \mathbf{F}\cdot
      \mathbf{\widehat{n}} \,{dS} $ ονομάζεται \textcolor{Col1}{ροή} του 
      διανυσματικού πεδίου $ \mathbf{F} $, \textcolor{Col1}{δια μέσου} της επιφάνειας 
      $S$.
    \item Το επιφανειακό ολοκλήρωμα ΙΙου είδους, συμβολίζεται και ως 
      \[
        \iint_{S} Pdydz + Qdxdz + Rdxdy 
      \] 
  \end{enumerate}
\end{rems} 

\begin{rem}
  Το επιφανειακό ολοκλήρωμα ΙΙου είδους, εξαρτάται από τον προσανατολισμό της επιφάνειας.
  Επομένως, αν $S$ και $ S' $ είναι οι δύο όψεις της επιφάνειας, τότε ισχύει ότι 
  \[
    \iint_{S} \mathbf{F} \cdot {\mathbf{dS}} = - \iint_{S'} \mathbf{F} \cdot
    {\mathbf{dS}} 
  \] 
\end{rem}

%todo να εξηγήσω καλύτερα τον προσανατολισμό επιφανειών.


\section{Μέθοδος Υπολογισμού του Επιφανειακού Ολοκληρώματος IIου είδους}

\subsection*{Η επιφάνεια $S$ δίνεται σε διανυσματική παραμετρική μορφή}

Αν η επιφάνεια $S$ ορίζεται από τη διανυσματική παραμετρική εξίσωση 
$ \mathbf{r}(u,v) = x(u,v)\mathbf{i}+y(u,v)\mathbf{j}+z(u,v)\mathbf{k} $, όπου 
$ (u,v) \in D \subseteq \mathbb{R}^{2} $, τότε:
\[
  \iint_{S} \mathbf{F} \cdot \,{\mathbf{dS}} = \iint_{S} \mathbf{F} \cdot
  \mathbf{\widehat{n}}\,{dS} = \pm\iint_{D} \mathbf{F} \cdot \frac{\mathbf{r}_{u} \times
  \mathbf{r}_{v}}{\cancel{\norm{\mathbf{r}_{u} \times \mathbf{r}_{v}}}}
  \cancel{\norm{\mathbf{r}_{u} \times \mathbf{r}_{v}}}\,{du}{dv} = \pm\iint_{D} 
  \mathbf{F} \cdot (\mathbf{r}_{u} \times \mathbf{r}_{v}) \,{du}{dv}
\]
άρα
\begin{equation*}
  \boxed{\iint_{S} \mathbf{F}(x,y,z) \cdot \,{\mathbf{dS}} = \pm\iint_{D} 
    \mathbf{F}(x(u,v),y(u,v),z(u,v)) \cdot (\mathbf{r}_{u} \times \mathbf{r}_{v}) 
  \,{du}{dv}}
\end{equation*}

\subsection*{Η επιφάνεια $S$ δίνεται ως συνάρτηση 2 μεταβλητών}

\begin{enumerate}
  \item Αν η επιφάνεια $S$ είναι συνάρτηση της μορφής $ z=z(x,y) $ και $D$ είναι η 
    προβολή της $S$ στο επίπεδο $ xy $, τότε: 
    \[
      \iint_{S} \mathbf{F} \cdot \,{\mathbf{dS}} = \iint_{S} \mathbf{F} \cdot
      \mathbf{\widehat{n}} \,{dS} = 
      \pm\iint_{D} 
      \mathbf{F} \cdot \frac{(-z_{x},-z_{y},1)}{\cancel{\sqrt{1+(z_{x})^{2}+
      (z_{y})^{2}}}} \cancel{\sqrt{1+(z_{x})^{2}+(z_{y})^{2}}} \,{dx}{dy} = 
      \pm\iint_{D} \mathbf{F} \cdot (-z_{x},-z_{y},1) \,{dx}{dy}
    \]
    άρα
    \begin{equation*}
      \boxed{\iint_{S} \mathbf{F}(x,y,z) \cdot \,{\mathbf{dS}} = \pm\iint_{D} 
      \mathbf{F}(x,y,z(x,y)) \cdot (-z_{x},-z_{y},1) \,{dx}{dy}}
    \end{equation*} 

  \item Αν η επιφάνεια $S$ είναι συνάρτηση της μορφής $ y=y(x,z) $ και $D$ είναι η 
    προβολή της $S$ στο επίπεδο $ xz $, τότε: 
    \begin{equation*}
      \boxed{\iint_{S} \mathbf{F}(x,y,z) \cdot \,{\mathbf{dS}} = \pm\iint_{D} 
      \mathbf{F}(x,y(x,z),z) \cdot (-y_{x},1,-y_{z}) \,{dx}{dz}}
    \end{equation*}

  \item Αν η επιφάνεια $S$ είναι συνάρτηση της μορφής $ x=x(y,z) $ και $D$ είναι η 
    προβολή της $S$ στο επίπεδο $ yz $, τότε: 
    \begin{equation*}
      \boxed{\iint_{S} \mathbf{F}(x,y,z) \cdot \,{\mathbf{dS}} = \pm\iint_{D} 
      \mathbf{F}(x(y,z),y,z) \cdot (1,-x_{y},-x_{z}) \,{dy}{dz}}
    \end{equation*}

\end{enumerate}

\subsection*{Η επιφάνεια $S$ δίνεται ως εξίσωση της μορφής g(x,y,z)=0}
Αν η επιφάνεια $S$ δίνεται από την εξίσωση $ g(x,y,z) = 0 $, όπου $ g $ είναι μια 
συνεχώς διαφορίσιμη συνάρτηση, και $D$ είναι η προβολή της $S$ στο επίπεδο που βρίσκεται 
από \textit{κάτω} της, τότε:
\begin{equation*}
  \iint_{S} \mathbf{F}\cdot {\mathbf{dS}} = \iint_{S} \mathbf{F} \cdot 
  \mathbf{\widehat{n}} \,{dS} = 
  \pm\iint_{D} \mathbf{F} \cdot \frac{\grad{g}}{\cancel{\norm{\grad
  g}}}\frac{\cancel{\norm{\grad g}}}{\abs{\grad(g) \cdot \mathbf{p}}} \,{dA} =
  \pm\iint_{D} \mathbf{F} \cdot 
  \frac{\grad{g}}{\abs{\grad(g) \cdot \mathbf{p}}} \,{dA} 
\end{equation*}
άρα
\begin{equation}\label{eq:surface_int3}
  \boxed{\iint_{S} \mathbf{F}(x,y,z) \cdot {\mathbf{dS}} = 
    \pm\iint_{D} \mathbf{F}(x,y,z) \cdot 
  \frac{\grad{g}}{\abs{\grad(g) \cdot \mathbf{p}}} \,{dA}}
\end{equation}
όπου $ \mathbf{p} $ είναι ένα μοναδιαίο κάθετο διάνυσμα στην περιοχή $D$ (συνήθως 
$ \mathbf{p} = \mathbf{i} $ ή $ \mathbf{j} $ ή $ \mathbf{k} $) και 
$ \grad(g)\cdot \mathbf{p} \neq 0 $ .  

Ειδικότερα, αν η επιφάνεια $S$ δίνεται από την εξίσωση $ g(x,y,z) = 0 $ και $D$ είναι η
προβολή της $S$ στο επίπεδο $ xy $, τότε η εξίσωση~\eqref{eq:surface_int3} γίνεται:
\[
  \iint_{S} \mathbf{F}(x,y,z) \,{dS} = \pm\iint_{D} \mathbf{F}(x,y,z(x,y)) 
  \cdot \frac{(g_{x},g_{y},g_{z})}{\abs{g_{z}}} \,{dx}{dy} 
\]


\section{Θεώρημα Stokes}

Έστω $S$ μια προσανατολισμένη και τμηματικά λεία επιφάνεια που φράσσεται από μια απλή, 
κλειστή και τμηματικά λεία καμπύλη $C$ με θετικό προσανατολισμό (ως προς το 
$ \mathbf{\widehat{n}} $).  Έστω επίσης, 
$ \mathbf{F}(x,y,z) = P(x,y,z)\mathbf{i}+Q(x,y,z)\mathbf{j}+R(x,y,z)\mathbf{k} $ ένα 
διανυσματικό πεδίο με συνεχώς παραγωγίσιμες συνιστώσες σε μια ανοιχτή περιοχή του 
$ \mathbb{R}^{3} $ που περιέχει την επιφάνεια $S$, τότε:
\[
  \oint_{C} \mathbf{F} \cdot \mathbf{dr} = \iint_{S} (\curl \mathbf{F}) \cdot \,{ \mathbf{dS}} 
\]

\subsection*{Φυσική Ερμηνεία του θεωρήματος Stokes}

Η \textbf{κυκλοφορία} ενός διανυσματικού πεδίου $ \mathbf{F} $ κατά μήκος μιας κλειστής 
καμπύλης $C$, ισούται με τη \textbf{ροή του στροβιλισμού} του πεδίου δια μέσου
\textit{οποιασδήποτε} επιφάνειας $S$ με σύνορο την καμπύλη $C$. 

\begin{rems}
\item {}
  \begin{myitemize}
    \item Το θεώρημα Stokes αποτελεί γενίκευση του θεωρήματος Green (Κυκλοφορία -
      Στροβιλισμός) στις 3 διαστάσεις. 
      Πράγματι, αν $S$ είναι επίπεδη, έστω κάποιο τμήμα $D$ του επιπέδου $ xy $, 
      με θετικό προσανατολισμό "προς τα πάνω" ($ \mathbf{\widehat{n}}= \mathbf{k} $), 
      τότε έχουμε 
      \[
        \oint_{C} \mathbf{F}\cdot {\mathbf{dr}} = \iint_{S} (\curl\mathbf{F}) \cdot
        \mathbf{dS} = \iint_{S} (\curl\mathbf{F}) \cdot \mathbf{\widehat{n}} \,{dS} = 
        \iint_{S} (\curl \mathbf{F}) \cdot \mathbf{k} \,{dS} = 
        \iint_{D} (Q_{x}-P_{y}) \,{dx}{dy}
      \] 
    \item Από το θεώρημα Stokes προκύπτει ότι η ροή του στροβιλισμού ενός διανυσματικού
      πεδίου $ \mathbf{F} $ δια μέσου \textbf{κλειστής} επιφάνειας $S$ είναι πάντα 0.
      \[
        \oiint_{S} (\curl\mathbf{F}) \cdot {\mathbf{dS}} = 0 
      \]
    \item Αν το διανυσματικό πεδίο $ \mathbf{F} $ είναι \textbf{αστρόβιλο}, 
      τότε $ \curl \mathbf{F} = \mathbf{0} $, οπότε από το θεώρημα Stokes προκύπτει ότι 
      η \textbf{κυκλοφορία} του πεδίου κατά μήκος κλειστής καμπύλης είναι 0.
  \end{myitemize}
\end{rems}




\section*{Θεώρημα Gauss}

Έστω $\Omega$ μια απλή στερεά περιοχή του $ \mathbb{R}^{3} $ που φράσσεται από 
την κλειστή και τμηματικά λεία επιφάνεια $S$ με θετικό προσανατολισμό προς τα έξω.  
Έστω επίσης, 
$ \mathbf{F}(x,y,z) = P(x,y,z)\mathbf{i}+Q(x,y,z)\mathbf{j}+R(x,y,z)\mathbf{k} $ ένα 
διανυσματικό πεδίο με συνεχώς παραγωγίσιμες συνιστώσες σε μια ανοιχτή περιοχή του 
$ \mathbb{R}^{3} $ που περιέχει την στερεά περιοχή $\Omega$, τότε:
\[
  \oiint_{S} \mathbf{F} \cdot \mathbf{\widehat{n}} \,{dS} = \iiint_{\Omega} (\div
  \mathbf{F}) \,{dx}{dy}{dz}
\]

\subsection*{Φυσική Ερμηνεία του θεωρήματος Gauss}

Η \textbf{εξερχόμενη ροή} ενός διανυσματικού πεδίου $ \mathbf{F} $ δια μέσου μιας
κλειστής επιϕάνειας $S$, ισούται με το τριπλό ολοκλήρωμα της \textbf{απόκλισης} του 
πεδίου στη στερεά περιοχή που περικλείεται από την επιφάνεια $S$.

\begin{rems}
\item {}
  \begin{myitemize}
    \item Το θεώρημα Gauss αποτελεί γενίκευση του θεωρήματος Green (Εξερχόμενη Ροή -
      Απόκλιση) στις 3 διαστάσεις. 
    \item Αν το διανυσματικό πεδίο $ \mathbf{F} $ είναι \textbf{σωληνοειδές}, 
      τότε $ \div \mathbf{F} = 0 $, οπότε από το θεώρημα Gauss προκύπτει ότι 
      η \textbf{εξερχόμενη ροή} του πεδίου δια μέσου κλειστής επιφάνειας είναι 0.
  \end{myitemize}
\end{rems}

\end{document}
