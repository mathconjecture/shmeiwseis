\documentclass[a4paper,12pt]{article}
\usepackage{etex}
%%%%%%%%%%%%%%%%%%%%%%%%%%%%%%%%%%%%%%
% Babel language package
\usepackage[english,greek]{babel}
% Inputenc font encoding
\usepackage[utf8]{inputenc}
%%%%%%%%%%%%%%%%%%%%%%%%%%%%%%%%%%%%%%

%%%%% math packages %%%%%%%%%%%%%%%%%%
\usepackage{amsmath}
\usepackage{amssymb}
\usepackage{amsfonts}
\usepackage{amsthm}
\usepackage{proof}

\usepackage{physics}

%%%%%%% symbols packages %%%%%%%%%%%%%%
\usepackage{dsfont}
\usepackage{stmaryrd}
%%%%%%%%%%%%%%%%%%%%%%%%%%%%%%%%%%%%%%%


%%%%%% graphicx %%%%%%%%%%%%%%%%%%%%%%%
\usepackage{graphicx}
\usepackage{color}
%\usepackage{xypic}
\usepackage[all]{xy}
\usepackage{calc}
%%%%%%%%%%%%%%%%%%%%%%%%%%%%%%%%%%%%%%%

\usepackage{enumerate}

\usepackage{fancyhdr}
%%%%% header and footer rule %%%%%%%%%
\setlength{\headheight}{14pt}
\renewcommand{\headrulewidth}{0pt}
\renewcommand{\footrulewidth}{0pt}
\fancypagestyle{plain}{\fancyhf{}
\fancyhead{}
\lfoot{}
\rfoot{\small \thepage}}
\fancypagestyle{vangelis}{\fancyhf{}
\rhead{\small \leftmark}
\lhead{\small }
\lfoot{}
\rfoot{\small \thepage}}
%%%%%%%%%%%%%%%%%%%%%%%%%%%%%%%%%%%%%%%

\usepackage{hyperref}
\usepackage{url}
%%%%%%% hyperref settings %%%%%%%%%%%%
\hypersetup{pdfpagemode=UseOutlines,hidelinks,
bookmarksopen=true,
pdfdisplaydoctitle=true,
pdfstartview=Fit,
unicode=true,
pdfpagelayout=OneColumn,
}
%%%%%%%%%%%%%%%%%%%%%%%%%%%%%%%%%%%%%%



\usepackage{geometry}
\geometry{left=25.63mm,right=25.63mm,top=36.25mm,bottom=36.25mm,footskip=24.16mm,headsep=24.16mm}

%\usepackage[explicit]{titlesec}
%%%%%% titlesec settings %%%%%%%%%%%%%
%\titleformat{\chapter}[block]{\LARGE\sc\bfseries}{\thechapter.}{1ex}{#1}
%\titlespacing*{\chapter}{0cm}{0cm}{36pt}[0ex]
%\titleformat{\section}[block]{\Large\bfseries}{\thesection.}{1ex}{#1}
%\titlespacing*{\section}{0cm}{34.56pt}{17.28pt}[0ex]
%\titleformat{\subsection}[block]{\large\bfseries{\thesubsection.}{1ex}{#1}
%\titlespacing*{\subsection}{0pt}{28.80pt}{14.40pt}[0ex]
%%%%%%%%%%%%%%%%%%%%%%%%%%%%%%%%%%%%%%

%%%%%%%%% My Theorems %%%%%%%%%%%%%%%%%%
\newtheorem{thm}{Θεώρημα}[section]
\newtheorem{cor}[thm]{Πόρισμα}
\newtheorem{lem}[thm]{λήμμα}
\theoremstyle{definition}
\newtheorem{dfn}{Ορισμός}[section]
\newtheorem{dfns}[dfn]{Ορισμοί}
\theoremstyle{remark}
\newtheorem{remark}{Παρατήρηση}[section]
\newtheorem{remarks}[remark]{Παρατηρήσεις}
%%%%%%%%%%%%%%%%%%%%%%%%%%%%%%%%%%%%%%%




\newcommand{\vect}[2]{(#1_1,\ldots, #1_#2)}
%%%%%%% nesting newcommands $$$$$$$$$$$$$$$$$$$
\newcommand{\function}[1]{\newcommand{\nvec}[2]{#1(##1_1,\ldots, ##1_##2)}}

\newcommand{\linode}[2]{#1_n(x)#2^{(n)}+#1_{n-1}(x)#2^{(n-1)}+\cdots +#1_0(x)#2=g(x)}

\newcommand{\vecoffun}[3]{#1_0(#2),\ldots ,#1_#3(#2)}


\input{tikz.tex}

% \geometry{top=3.5cm}
\pagestyle{vangelis}
% \everymath{\displaystyle}
\setcounter{chapter}{1}

\input{insbox}

\pgfplotsset{
  myaxis/.style={axis lines=center,axis line style={thick,blue!50,-stealth},tick label
  style={font=\large,blue!75,above=5pt},xtick=\empty,ytick=\empty,tick style={blue!50}},
  myplot/.style={Col1!75,ultra thick,samples=500,no marks,smooth,},
  dashed lines/.style={dashed,blue!50,ultra thin},
}

\newcommand{\twocolumnsiderp}[2]{\begin{minipage}[c]{0.50\linewidth}\raggedright
    #1
    \end{minipage}\hfill\begin{minipage}[c]{0.45\linewidth}\raggedright
    #2
  \end{minipage}
}

\newcommand{\twocolumnsiderq}[2]{\begin{minipage}[c]{0.64\linewidth}\raggedright
    #1
    \end{minipage}\hfill\begin{minipage}[c]{0.31\linewidth}\raggedright
    #2
  \end{minipage}
}




\begin{document}




\chapter*{Σειρές Fourier}

\section{Περιοδικές Συναρτήσεις}


\begin{dfn}
  Μια συνάρτηση $ f \colon \mathbb{R} \to \mathbb{R} $ λέγεται 
  \textcolor{Col1}{περιοδική} αν 
  $
  f(x+T)=f(x),\quad \forall x \in \mathbb{R} 
  $ 
  όπου $ T>0 $, ονομάζεται \textcolor{Col1}{περίοδος} της συνάρτησης.
\end{dfn}

\twocolumnsiderp{
  \begin{examples}
  \item {}
    \begin{enumerate}[i)]
      \item Η $ f(x) = \sin{x} $ είναι περιοδική με περίοδο $ T=2 \pi $.
      \item Η $ f(x) = \sin{nx} $ είναι περιοδική με περίοδο $ T= \frac{2 \pi}{n} $.
      \item Η $ f(x) = \tan{x} $ είναι περιοδική με περίοδο $ T= \pi $.
      \item Η $ f(x) = c, \; c \in \mathbb{R} $, η σταθερή συνάρτηση, είναι περιοδική με 
        περίοδο οποιονδήποτε θετικό αριθμό.
    \end{enumerate}
  \end{examples}
  }{
  \twocolumnsides{
    \centering
    \begin{tikzpicture}[scale=0.5]
      \begin{axis}[myaxis,
        ymin=-2,ymax=2,
        xmin=-7,xmax=7,
        extra x ticks={0,6.28},
        extra x tick labels={$0$,$2 \pi$},
        tick label style={blue!75,above left=3pt},
        domain=-2*pi:2*pi,
        ]
        \addplot[myplot] {sin(deg(x))} node[above right,pos=0.65]{\Large$\sin x$} ;
      \end{axis}
    \end{tikzpicture}
    }{
    \centering
    \begin{tikzpicture}[scale=0.5]
      \begin{axis}[myaxis,
        ymin=-6,ymax=6,
        xmin=-6,xmax=6,
        extra x ticks={-1.57,1.57},
        extra x tick labels={$-\frac{\pi}{2}$,$\frac{\pi}{2}$},
        tick label style={font=\Large,blue!75,below left=3pt},
        domain=-1.8*pi:1.8*pi,
        ]
        \addplot [myplot,dashed lines] {tan(deg(x))};
        \addplot [myplot,restrict y to domain=-10:10] {tan(deg(x))} node
          [right,pos=0.55,xshift=5pt] {\Large$\tan x$};
      \end{axis}
    \end{tikzpicture}
}}


\subsection*{Ιδιότητες Περιοδικών Συναρτήσεων}

\begin{myitemize}
  \item Αν $f$ περιοδική με περίοδο $T$, τότε $ f(x+nT) = f(x) $, για κάθε $x$ στο 
    πεδίο ορισμού της $f$, και για κάθε $ n \in \mathbb{N} $.
  \item Αν $f$ περιοδική με περίοδο $T$, τότε $ f(ax) $ επίσης περιοδική με περίοδο 
    $ T/ \abs{a} $, για κάθε $ a \in \mathbb{R}^{*} $.
\end{myitemize}

\begin{prop}
  Αν οι συναρτήσεις $ f $ και $g$ είναι περιοδικές με περίοδο $T$, τότε και οι 
  $ f+g $ και $ f\cdot g $ είναι περιοδικές με την ίδια περίοδο.
\end{prop}
\begin{proof}
  \begin{align*}
    (f+g)(x+T) = f(x+T)+g(x+T)=f(x)+g(x)=(f+g)(x) \\
    (f\cdot g)(x+T) = f(x+T)\cdot g(x+T)=f(x)\cdot g(x)=(f\cdot g)(x) \hfill
  \end{align*}
\end{proof}

\begin{prop}
  Έστω $ f(x) $ περιοδική συνάρτηση με περίοδο $T$, και έστω $ x_{0} \in [0,T] $. 
  Αν $f$ ολοκληρώσιμη στο $ [0,T] $, τότε, ισχύει:
  \[
    \int _{0}^{T} f(x) \,{dx} = \int _{x_{0}}^{x_{0}+T} f(x) \,{dx}  
  \] 
\end{prop}

\twocolumnsiderq{
  \begin{dfn}
    Μια συνάρτηση $ f $ λέγεται \textcolor{Col1}{τμηματικά συνεχής} στο $ [a,b] $, 
    αν η  $f$ είναι συνεχής στο $ [a,b] $ εκτός ίσως από ένα  πεπερασμένο πλήθος 
    σημείων ασυνέχειας, στα οποία όμως, υπάρχουν τα πλευρικά όρια και είναι πεπερασμένα.
  \end{dfn}
  \begin{dfn}
    Μια συνάρτηση  $ f $  λέγεται  \textcolor{Col1}{τμηματικά λεία} στο $ [a,b] $ αν η 
    $f$ και η $f'$ είναι τμηματικά συνεχείς στο $ [a,b] $, με τα ίδια σημεία 
    ασυνέχειας.     
  \end{dfn}
  }{\begin{tikzpicture}[scale=0.50]
    \coordinate (o) at (0,0) ;
    \node  (0) at (1, 1)  {} ;
    \node  (1) at (3.5, 1.5) {};
    \node  (2) at (3.5, 2.5) {};
    \fill (1) circle (2pt) ;
    \node  (3) at (6, 2) {};
    \fill (3) circle (2pt) ;
    \node  (4) at (-1.5, 3.75) {};
    \node  (5) at (1, 3) {};
    \fill (5) circle (2pt) ;
    \node  (6) at (-2, 0) {};
    \coordinate  (7) at (7, 0) {};
    \node  (8) at (0, -1) {};
    \coordinate  (9) at (0, 4) {};
    \node  at (9) [left,blue!75] {$y$} ;
    \node  at (7) [below,blue!75] {$x$} ;
    \node  at (3,3) [above,Col1!75] {$y=f(x)$} ;
    \draw [in=-120, out=30, looseness=1.75,Col1!75, thick] (0.center) to (1.center);
    \draw [in=-150, out=15, looseness=1.25,Col1!75, thick] (2.center) to (3.center);
    \draw [in=-165, out=0, looseness=1.75,Col1!75, thick] (4.center) to (5.center);
    \draw[-latex,blue!50] (6.center) to (7.center);
    \draw[-latex,blue!50] (8.center) to (9.center);
    \draw[dashed,dash expand off,blue!50] (5.center) -- (o-|5) node[below] {$x_{0}$} ;
    \draw[dashed,dash expand off,blue!50] (2.center) -- (o-|2) node[below] {$x_{1}$} ;
    \draw[dashed,dash expand off,Col1!75] (4.center) -- (o-|4) node[below] {$a$} ;
    \draw[dashed,dash expand off,Col1!75] (3.center) -- (o-|3) node[below] {$b$} ;
    \draw[black,fill=white] (0) circle (2pt) ;
    \draw[black,fill=white] (2) circle (2pt) ;
    \draw[black,fill=white] (4) circle (2pt) ;
\end{tikzpicture}}



\section{Περιοδική Επέκταση Συναρτήσεων}


\begin{enumerate}

  \item Αν $ f(x) $ ορισμένη στο $ [-L,L] $ με $ f(-L)=f(L) $ τότε η συνάρτηση
    \[
      F(x) = f(x), \; -L \leq x \leq L \quad \text{και} \quad F(x+2L)=F(x)
    \]
    με περίοδο $ 2L $ λέγεται \textcolor{Col1}{περιοδική επέκταση} της $f$. 

  \item Αν $ f(x) $ ορισμένη στο $ [0,L] $, τότε μπορεί να επεκταθεί 
    κατά άπειρους τρόπους σε μια περιοδική συνάρτηση 
    \begin{alignat*}{4}
            &F(x) = 
            \begin{cases}  
              f(x), & \phantom{-} 0 \leq x \leq L \\
              \phi(x) & -L < x < 0
            \end{cases}  & \quad & \text{και} \quad F(x+2L)=F(x)
    \end{alignat*}
    με περίοδο $ 2L $, όπου $ \phi(x) $ αυθαίρετη συνάρτηση. 

  \item\label{artia} Αν $ f(x) $ ορισμένη στο $ [0,L] $ τότε η συνάρτηση 
    \begin{alignat*}{4}
            & F(x) = 
            \begin{cases}  
              f(x), & \phantom{-} 0 \leq x \leq L \\
              f(-x), & -L \leq x \leq 0
            \end{cases}  & \quad \text{και} \quad F(x+2L)=F(x)
    \end{alignat*}
    με περίοδο $ 2L $ λέγεται \textcolor{Col1}{άρτια περιοδική επέκταση} της $f$.

  \item\label{peritth} Αν $ f(x) $ ορισμένη στο $ [0,L] $, με $ f(0)=f(L)=0 $ 
    τότε η συνάρτηση
    \begin{alignat*}{4}
            &F(x) = 
            \begin{cases}  
              \phantom{-} f(x), & \phantom{-} 0 \leq x \leq L \\
              -f(-x), & -L \leq x \leq 0
            \end{cases}  & \quad \text{και} \quad F(x+2L)=F(x)
    \end{alignat*}
    με περίοδο $ 2L $ λέγεται \textcolor{Col1}{περιττή περιοδική επέκταση} της $f$.

  \item Αν $ f(x) $ ορισμένη στο $ (0,L) $, τότε επεκτείνεται σε 
    άρτια (αντ. περιττή) περιοδική συνάρτηση $ F(x) $ θέτοντας $ F(0) $, 
    $ F(L) $ οποιαδήποτε τιμή (αντ. $F(0)=F(L)=0$) και στη συνέχεια εφαρμόσουμε 
    τις επεκτάσεις~\ref{artia} και~\ref{peritth}.
\end{enumerate}



\section{Συντελεστές Fourier}

\begin{dfn}
  Κάθε σειρά της μορφής 
  \begin{equation}\label{eq:trig}
    \frac{a_{0}}{2} + \sum_{n=1}^{\infty} [a_{n} \cos{(nx)} + b_{n} \sin{(nx)}] 
  \end{equation} 
  όπου, $ a_{0}, a_{n}, b_{n} \in \mathbb{R} $ και $ n \in \mathbb{N} $, με 
  $ x \in \mathbb{R} $, ονομάζεται \textcolor{Col1}{τριγωνομετρική σειρά}.
\end{dfn}

\begin{prop}\label{prop:conv}
  Αν η σειρά $ \sum_{n=1}^{\infty} \sqrt{a_{n}^{2}+b_{n}^{2}} $ συγκλίνει,
  τότε η τριγωνομετρική σειρά~\eqref{eq:trig}, συγκλίνει \textbf{απολύτως} και
  \textbf{ομοιόμορφα} σε μια συνάρτηση $ f(x) $, η οποία είναι \textbf{συνεχής} στο 
  $ \mathbb{R} $ και περιοδική με περίοδο $ 2 \pi $.
\end{prop}

Έστω, λοιπόν η τριγωνομετρική σειρά~\eqref{eq:trig} η οποία συγκλίνει για κάθε 
$ x \in [- \pi , \pi] $ προς τη συνάρτηση $ f(x) $, δηλαδή
\begin{equation}\label{eq:trig2}
  f(x) = \frac{a_{0}}{2} + \sum_{n=1}^{\infty} [a_{n} \cos{(nx)} + b_{n} \sin{(nx)}],
  \quad \forall x \in [- \pi , \pi]
\end{equation} 

Αν υποθέσουμε ότι η σύγκλιση της σειράς είναι \textbf{ομοιόμορφη} ώστε η συνάρτηση 
$ f(x) $ να είναι συνεχής, τότε μπορούμε να πολλαπλασιάσουμε την~\eqref{eq:trig2} με 
$ \sin{mx} $ και να ολοκληρώσουμε \textbf{όρο προς όρο}, σε διάστημα μίας περιόδου,
οπότε: 

\begin{gather*}
  % f(x) = \frac{a_{0}}{2} + \sum_{n=1}^{\infty} [a_{n} \cos{nx} + b_{n} \sin{nx}]  \\
  f(x) \sin{mx} = \frac{a_{0}}{2} \sin{mx} + \sum_{n=1}^{\infty} 
  [a_{n} \cos{nx} \sin{mx} + b_{n} \sin{nx} \sin{mx}]   \\
  \int _{- \pi} ^{\pi} f(x) \sin{mx} \,{dx} = \frac{a_{0}}{2} \int _{- \pi}^{\pi} 
  \sin{mx} \,{dx} + \sum_{n=1}^{\infty} \left[a_{n} \int _{- \pi}^{\pi} 
    \cos{nx} \sin{mx}  \,{dx} + b_{n} \int _{- \pi }^{\pi} \sin{nx} \sin{mx} 
  \,{dx}\right] \\
  \int _{- \pi }^{\pi} f(x) \sin{mx} \,{dx} = \frac{a_{0}}{2} \cdot 0 + 
  a_{m}\cdot 0 + b_{m} \cdot \int _{- \pi}^{\pi } \sin^{2}{mx} \,{dx} \\
  \int _{- \pi }^{\pi} f(x) \sin{mx} \,{dx} =  b_{m} \cdot \pi 
\end{gather*}   

Θέτοντας τώρα, όπου $ n=m $, έχουμε
\begin{equation}\label{eq:four1}
  \boxed{b_{n} = \frac{1}{\pi} \int _{- \pi }^{\pi} f(x) \sin{nx} \,{dx}, \quad n =
  1,2,3, \ldots}
\end{equation} 

Ομοίως αν πολλαπλασιάσουμε την~\eqref{eq:trig2} με $ \cos{mx} $ και ολοκληρώσουμε 
σε διάστημα μιας περιόδου, έχουμε:

\begin{gather*}
  % f(x) = \frac{a_{0}}{2} + \sum_{n=1}^{\infty} [a_{n} \cos{nx} + b_{n} \sin{nx}]  \\
  f(x) \cos{mx} = \frac{a_{0}}{2} \cos{mx} + \sum_{n=1}^{\infty} 
  [a_{n} \cos{nx} \cos{mx} + b_{n} \sin{nx} \cos{mx}]   \\
  \int _{- \pi} ^{\pi} f(x) \cos{mx} \,{dx} = \frac{a_{0}}{2} \int _{- \pi}^{\pi} 
  \cos{mx} \,{dx} + \sum_{n=1}^{\infty} \left[a_{n} \int _{- \pi}^{\pi}
    \cos{nx} \cos{mx}  \,{dx} + b_{n} \int _{- \pi }^{\pi} \sin{nx} \cos{mx} 
  \,{dx}\right] \\
\end{gather*}
και από τις συνθήκες ορθογωνιότητας για τις διάφορες περιπτώσεις του $ m $, προκύπτει
\begin{gather*}
  \begin{cases} 
    \displaystyle{\int _{- \pi }^{\pi} f(x) \cos{mx} \,{dx} = 
      \frac{a_{0}}{2} \cdot 0 + a_{m} \int _{- \pi }^{\pi } \cos^{2}{mx} 
    \,{dx} + b_{m} \cdot 0}, 
         & m = 1,2,3, \ldots \\
         \displaystyle{\int _{- \pi }^{\pi} f(x) \,{dx} = \frac{a_{0}}{2}} 
         \int _{- \pi}^{\pi} \,{dx}, 
         & m=0 \\
  \end{cases} 
  \intertext{άρα}
  \begin{cases} 
    \displaystyle{\int _{- \pi }^{\pi} f(x) \cos{nx} \,{dx} = 
    a_{n}\cdot \pi }, \quad n=1,2,3\ldots \\[0.5cm]
    \displaystyle{\int _{- \pi }^{\pi} f(x) \,{dx} = \frac{a_{0}}{2}
    2 \pi } 
  \end{cases} 
  \Leftrightarrow 
  \begin{cases} 
    \displaystyle{a_{n} = \frac{1}{\pi} \int _{- \pi }^{\pi } f(x) \cos{nx} 
    \,{dx}}, \quad n=1,2,3\ldots \\[0.5cm]
    \displaystyle{a_{0} = \frac{1}{\pi} \int _{- \pi}^{\pi} f(x) \,{dx}}
  \end{cases}
\end{gather*}
ή ισοδύναμα
\begin{equation}\label{eq:four2}
  \boxed{a_{n} = \frac{1}{\pi} \int _{- \pi }^{\pi} f(x) \cos{nx} \,{dx}, \quad
  n = \textcolor{Col1}{0},1,2,3, \ldots}
\end{equation}

\begin{rem}
  Οι τύποι~\eqref{eq:four1} και~\eqref{eq:four2} ονομάζονται 
  \textcolor{Col1}{συντελεστές Fourier} και μπορούν να υπολογιστούν σε 
  \textbf{οποιαδήποτε} πλήρη περίοδο.  Για παράδειγμα, αν η $f$ είναι ορισμένη σε 
  διάστημα της μορφής $ [x_{0}, x_{0}+2 \pi] $, τότε οι συντελεστές 
  Fourier δίνονται από τις σχέσεις:
  \begin{align*}
    a_{0} &= \frac{1}{\pi} \int _{x_{0}}^{x_{0}+ 2 \pi }f(x) \,{dx} \\
    a_{n} &= \frac{1}{\pi} \int _{x_{0}}^{x_{0}+ 2 \pi } f(x) \cos{(nx)} \,{dx}  \\
    b_{n} &= \frac{1}{\pi} \int _{x_{0}}^{x_{0}+ 2 \pi } f(x) \sin{(nx)} \,{dx}  
  \end{align*} 
\end{rem}

\begin{rem}
  Για τους συντελεστές Fourier ισχύει προφανώς, ότι:
  \begin{myitemize}
    \item $ a_{n} = \frac{1}{L} \int _{-L}^{L} f(x) \cos{(nx)} \,{dx} $ είναι
      \textbf{άρτια} συνάρτηση.
    \item $ b_{n} = \frac{1}{L} \int _{-L}^{L} f(x) \sin{(nx)} \,{dx} $ είναι
      \textbf{περιττή} συνάρτηση.
  \end{myitemize}
\end{rem}

\begin{rem}
  Για τον υπολογισμό των συντελεστών Fourier, παραπάνω θεωρήσαμε ότι η 
  σειρά~\eqref{eq:trig} συγκλίνει \textbf{ομοιόμορφα} στην $f$, κάτι που δεν το 
  γνωρίζουμε εκ των προτέρων.  Γεγονός είναι ότι για κάθε συνάρτηση $f$ που είναι 
  ολοκληρώσιμη στο $ [- \pi , \pi] $, μπορούμε να υπολογίσουμε τους συντελεστές 
  Fourier και τυπικά να σχηματίσουμε τη σειρά, γράφοντας
  \[
    f(x) \sim \frac{a_{0}}{2} + \sum_{n=1}^{\infty} [a_{n} \cos{nx} + b_{n} \sin{nx}]
  \] 
  η οποία ονομάζεται σειρά \textcolor{Col1}{Fourier} της $f$. Υπό ποιες προϋποθέσεις η 
  σειρά Fourier μιας συνάρτησης $f$ \textbf{συγκλίνει} στην $f$, απαντάται στο επόμενο 
  θεώρημα.
\end{rem}



\section{Σειρά Fourier}

\begin{thm}
  Έστω συνάρτηση $ f $, \textbf{τμηματικά λεία} στο $ [- \pi , \pi] $ και 
  \textbf{περιοδική} στο $\mathbb{R}$ με περίοδο $ 2 \pi $.
  Τότε η σειρά Fourier της $f$ είναι συγκλίνουσα, και ισχύει:
  \begin{align*}
    f(x) &= \frac{a_{0}}{2} + \sum_{n=1}^{\infty} [a_{n} \cos{(nx)} + b_{n}
    \sin{(nx)} ], \quad \text{για κάθε $x$ σημείο \textbf{συνέχειας} της $f$} \\
    \frac{f(x_{0}^{-})+f(x_{0}^{+})}{2} &= \frac{a_{0}}{2} + \sum_{n=1}^{\infty} [a_{n}
    \cos{(n x_{0})} + b_{n} \sin{(n x_{0})} ], \quad \text{για κάθε $ x_{0} $ σημείο
    \textbf{ασυνέχειας} της $f$}   
  \end{align*} 
  όπου $ f({x_{0}}^{-}) = \lim\limits_{x \to {x_{0}}^{-}} f(x) $ και 
  $ f({x_{0}}^{+}) = \lim\limits_{x \to {x_{0}}^{+}} f(x) $ 
\end{thm}

\begin{rem}
\item {}
  Αν $f$ συνεχής για κάθε $x \in \mathbb{R} $, τότε η σειρά Fourier της $f$ 
  συγκλίνει \textbf{απόλυτα} και \textbf{ομοιόμορφα} στην $f$ στο $ \mathbb{R} $.
\end{rem}

\begin{rem}
  Επειδή οι όροι $ \sin{nx} $ και $ \cos{nx} $ της σειράς Fourier, έχουν μηδενική μέση
  τιμή, σε διάστημα μιας περιόδου (λόγω συμμετρίας), 
  ο σταθερός όρος $ a_{0}/2 $ αναμένεται να εμφανίζεται πάντα, στη σειρά Fourier κάθε 
  συνάρτησης $f$ με μη μηδενική \textbf{μέση τιμή}. Επομένως
  \[
    \frac{a_{0}}{2} = \frac{1}{2 \pi} \int _{- \pi }^{\pi} f(x)\,{dx} =
    \text{\textbf{Μέση Τιμή} της $f$ σε διάστημα μίας περιόδου}
  \] 
\end{rem}



\section{Συναρτήσεις με περίοδο 2L}

\begin{rem}
  Αν $ f(x) $, τμηματικά συνεχής στο $ [-L, L] $ και περιοδική  με περίοδο 
  $ 2L>0 $, τότε, θέτοντας 
  \[ 
    \frac{x}{t} = \frac{2L}{2 \pi} \Leftrightarrow \boxed{x= \frac{L}{\pi} t}
  \]
  η συνάρτηση $ g(t) = f\Bigl(\frac{L}{\pi}t\Bigr) $, είναι τμηματικά συνεχής στο 
  $ [- \pi , \pi] $ και περιοδική με περίοδο $ 2 \pi $.
\end{rem}
Οπότε, για την $ g(t) $ η σειρά Fourier είναι 
\[
  \frac{a_{0}}{2} + \sum_{n=1}^{\infty} [a_{n} \cos{(nt)} + b_{n} \sin{(nt)}]
\] 
όπου 
\[
  a_{n} = \frac{1}{\pi} \int _{- \pi}^{\pi} g(t) \cos{nt} \,{dt} = 
  \frac{1}{\pi} \int _{- \pi}^{\pi} f\Bigl(\frac{L}{\pi} t\Bigr) \cos{nt} \,{dt}, 
  \quad n = 0,1,2,3, \ldots
\] 
και 
\[
  b_{n} = \frac{1}{\pi} \int _{- \pi}^{\pi} g(t) \sin{nt} \,{dt} = 
  \frac{1}{\pi} \int _{- \pi}^{\pi} f\Bigl(\frac{L}{\pi} t\Bigr) \sin{nt} \,{dt}, 
  \quad n = 1,2,3, \ldots
\]
Επιστρέφοντας πίσω, στην αρχική μεταβλητή $x$, θέτοντας όπου $ t = \frac{\pi x}{L} $ 
και $ dt = \frac{\pi}{L} dx $, έχουμε ότι η σειρά Fourier της συνάρτησης $ f(x) $ είναι
\[
  \boxed{\frac{a_{0}}{2} + \sum_{n=1}^{\infty} \left[a_{n} 
      \cos{\left(\frac{n \pi x}{L}\right)} + b_{n} 
  \sin{\left(\frac{n \pi x}{L}\right)} \right]}
\]
όπου 
\[
  a_{n} = \frac{1}{L} \int _{- L}^{L} f(x) 
  \cos{\left(\frac{n \pi x}{L}\right)} \,{dx}, \quad n = 0,1,2,3, \ldots
\] 
και 
\[
  b_{n} = \frac{1}{L} \int _{- L}^{L} f(x) 
  \sin{\left(\frac{n \pi x}{L}\right)} \,{dx}, \quad n = 1,2,3, \ldots
\]

\begin{rem}
  Αν η $f$ είναι ορισμένη σε διάστημα της μορφής $ [x_{0}, x_{0}+2L] $, 
  τότε η σειρά Fourier θα είναι 
  \[
    \frac{a_{0}}{2} + \sum_{n=1}^{\infty} \left[a_{n} 
      \cos{\left(\frac{n \pi x}{L}\right)} + b_{n} 
    \sin{\left(\frac{n \pi x}{L}\right)} \right]
  \]
  και οι συντελεστές Fourier δίνονται απευθείας από τις σχέσεις:
  \begin{align*}
    a_{0} &= \frac{1}{\pi} \int _{x_{0}}^{x_{0}+ 2 L}f(x) \,{dx} \\
    a_{n} &= \frac{1}{\pi} \int _{x_{0}}^{x_{0}+ 2 L} f(x) \cos{\left(\frac{n \pi x}{L}\right)} \,{dx}  \\
    b_{n} &= \frac{1}{\pi} \int _{x_{0}}^{x_{0}+ 2 L} f(x) \sin{\left(\frac{n \pi x}{L}\right)} \,{dx}  
  \end{align*} 
\end{rem}

\begin{rem}
  Αν, αντίστοιχα, για την $f$ το διάστημα περιοδικότητας, είναι τυχαίο διάστημα 
  $ [a,b] $, τότε θέτουμε:
  \[ 
    x = \frac{a+b}{2} + \frac{b-a}{2 \pi} t 
  \] 
  οπότε μετασχηματίζεται ανάλογα  σε μια περιοδική συνάρτηση, ορισμένη στο 
  $ [- \pi , \pi] $ με περίοδο $ 2 \pi $, και εργαζόμαστε όπως παραπάνω.
\end{rem}



\section{Ημιτονική και Συνημιτονική σειρά Fourier}

\subsection*{Άρτιες - Περιττές Συναρτήσεις}

\begin{dfn}
  Μια συνάρτηση $ f \colon A \to \mathbb{R} $ θα λέγεται:
  \begin{alignat*}{4}
        &\textcolor{Col1}{\text{άρτια}}   & \quad 
        &\overset{\text{ορ.}}{\Leftrightarrow} & \quad f(-x) &= f(x), & 
        &\quad \forall x \in A \\
        &\textcolor{Col1}{\text{περιττή}}  & \quad 
        &\overset{\text{ορ.}}{\Leftrightarrow} & \quad f(-x) &= -f(x), & 
        &\quad \forall x \in A \\
  \end{alignat*}
\end{dfn}

\begin{examples}
\item {}
  \begin{enumerate}
    \item Οι συναρτήσεις $ f(x)=x^{2} $ και $ g(x)= \cos (nx) $ είναι άρτιες. 
    \item συναρτήσεις $ f(x)=x^{3} $ και $ g(x)= \sin (nx) $ είναι περιττές.
  \end{enumerate}
\end{examples}

\begin{prop}
\item {}
  \begin{enumerate}\label{prop:ginart}
    \item άρτια $ \cdot $ άρτια $=$ άρτια
    \item περιττή $ \cdot $ περιττή $=$ άρτια
    \item άρτια $ \cdot $ περιττή $=$ περιττή
  \end{enumerate}
\end{prop}

\begin{rem}
  Η γραφική παράσταση μιας άρτιας συνάρτησης είναι συμμετρική ως προς τον άξονα $y$, 
  ενώ μιας περιττής συνάρτησης ως προς την αρχή των αξόνων.
\end{rem}

\begin{prop}\label{prop:artper}
\item {}
  \InsertBoxR{1}{\parbox[b][4\baselineskip][c]{0.48\textwidth}
    {\begin{tikzpicture}[scale=0.8]
        \coordinate (o) at (0,0) ;
        \node at (o) [below left,blue!50] {\small$0$} ;
        \coordinate (a) at (-1.25,1.25*1.25) ;
        \coordinate (b) at (1.25,1.25*1.25) ;
        \fill[domain=-1.30:1.30,smooth,variable=\x,blue!25,very thick] (-1.3,0)
          node[below,blue!50] {\small$-a$} -- plot ({\x},{\x*\x}) -- 
          (1.3,0) node[below,blue!50] {\small$a$} -- cycle;
        \draw[domain=-1.4:1.4,smooth,variable=\x,Col1,very thick] plot ({\x},{\x*\x})
          node[right]{\small $y=x^{2}$};
        \draw[-stealth,blue!50] (-2,0) -- (2,0) node[right] {};
        \draw[-stealth,blue!50] (0,-2) -- (0,2) node[left] {\small$y$};
      \end{tikzpicture} 
      \hspace{0.2\baselineskip}
      \begin{tikzpicture}[scale=0.8]
        \coordinate (o) at (0,0) ;
        \node at (o) [below left,blue!50] {\small$0$} ;
        \coordinate (a) at (-1.2,1.2*1.2) ;
        \coordinate (b) at (1.2,1.2*1.2) ;
        \fill[domain=-1.20:1.20,smooth,variable=\x,blue!25,very thick] (-1.2,0) 
          node[below left,blue!50] {\small$-a$} --
          plot ({\x},{\x*\x*\x}) -- (1.2,0) node[below,blue!50] {\small$a$} -- cycle ;
        \draw[domain=-1.25:1.25,smooth,variable=\x,Col1,very thick] 
          plot ({\x},{\x*\x*\x}) node[below right]{\small $y=x^{3}$};
        \draw[-stealth,blue!50] (-2,0) -- (2,0) node[right] {};
        \draw[-stealth,blue!50] (0,-2) -- (0,2) node[left] {\small$y$};
  \end{tikzpicture}}} 
  Αν $ f(x) $ ολοκληρώσιμη στο διάστημα $[-a,a]$ τότε ισχύουν:

  \begin{myitemize}
    \item $ f(x) $ άρτια $ \Rightarrow \displaystyle{\int _{-a}^{a} f(x) \,{dx} = 2 
      \int _{0}^{a} f(x) \,{dx}} $ 
    \item $ f(x) $ περιττή $ \Rightarrow \displaystyle{\int _{-a}^{a} f(x) \,{dx} = 0} $
  \end{myitemize}
\end{prop}



\subsection*{Συνημιτονική σειρά Fourier συνάρτησης με περίοδο $ 2 \pi $}

Αν η $f$ είναι \textbf{άρτια}, με περίοδο $ 2 \pi $, τότε η συνάρτηση 
$ f(x) \cos{(nx)} $ είναι επίσης \textbf{άρτια} (από πρόταση~\ref{prop:ginart}), 
ενώ η συνάρτηση $ f(x) \sin{(nx)} $ είναι \textbf{περιττή} 
(από πρόταση~\ref{prop:ginart}), για αυτό, αν λάβουμε υπόψιν 
μας τις σχέσεις της πρότασης~\ref{prop:artper}, έχουμε:
\[
  a_{n} = \frac{2}{\pi} \int _{0}^{\pi} f(x) \cos{(nx)} \,{dx}, 
  \quad n=\textcolor{Col1}{0},1,2,\ldots \quad \text{και} \quad b_{n} = 0
\] 
και τότε η σειρά Fourier της $f$ γίνεται:
\[
  \boxed{\frac{a_{0}}{2} + \sum_{n=1}^{\infty} a_{n} \cos{(nx)}} \quad 
  \text{\color{Col1} Συνημιτονική σειρά Fourier}
\]



\subsection*{Ημιτονική σειρά Fourier συνάρτησης με περίοδο $ 2 \pi $}

Αν η $f$ είναι \textbf{περιττή}, με περίοδο $ 2 \pi $,  τότε η συνάρτηση 
$ f(x) \cos{(nx)} $ είναι επίσης \textbf{περιττή} (από πρόταση~\ref{prop:ginart}), 
ενώ η συνάρτηση $ f(x) \sin{(nx)} $ είναι \textbf{άρτια} 
(από πρόταση~\ref{prop:ginart}), για αυτό, αν λάβουμε υπόψιν μας 
τις σχέσεις της πρότασης~\ref{prop:artper}, έχουμε:
\[
  a_{n} = 0, \quad \text{και} \quad
  b_{n} = \frac{2}{\pi} \int _{0}^{\pi} f(x) \sin{(nx)} \,{dx} 
\] 
και τότε η σειρά Fourier της $f$ γίνεται:
\[
  \boxed{\sum_{n=1}^{\infty} b_{n} \sin{(nx)}} \quad 
  \text{\color{Col1} Ημιτονική σειρά Fourier}
\]



\subsection*{Συνημιτονική σειρά Fourier συνάρτησης με περίοδο $ 2 L$}

Αν η $f$ είναι \textbf{άρτια}, με περίοδο $ 2 L$, τότε η συνάρτηση 
$ f(x) \cos{\left(\frac{n \pi x}{L}\right)} $ είναι επίσης \textbf{άρτια} 
(από πρόταση~\ref{prop:ginart}), 
ενώ η συνάρτηση $ f(x) \sin{\left(\frac{n \pi x}{L}\right)} $ είναι \textbf{περιττή} 
(από πρόταση~\ref{prop:ginart}), για αυτό, αν λάβουμε υπόψιν 
μας τις σχέσεις της πρότασης~\ref{prop:artper}, έχουμε:
\[
  a_{n} = \frac{2}{L} \int _{0}^{L} f(x) \cos{\left(\frac{n \pi x}{L}\right)} \,{dx}, 
  \quad n=\textcolor{Col1}{0},1,2,\ldots \quad \text{και} \quad b_{n} = 0
\] 
και τότε η σειρά Fourier της $f$ γίνεται:
\[
  \boxed{\frac{a_{0}}{2} + \sum_{n=1}^{\infty} a_{n} \cos{\left(\frac{n \pi
  x}{L}\right)}} \quad 
  \text{\color{Col1} Συνημιτονική σειρά Fourier}
\]



\subsection*{Ημιτονική σειρά Fourier συνάρτησης με περίοδο $ 2 L$}

Αν η $f$ είναι \textbf{περιττή}, με περίοδο $ 2 L$,  τότε η συνάρτηση 
$ f(x) \cos{\left(\frac{n \pi x}{L}\right)} $ είναι επίσης \textbf{περιττή} 
(από πρόταση~\ref{prop:ginart}), ενώ η συνάρτηση 
$ f(x) \sin{\left(\frac{n \pi x}{L}\right)} $ είναι \textbf{άρτια} 
(από πρόταση~\ref{prop:ginart}), για αυτό, αν λάβουμε υπόψιν μας 
τις σχέσεις της πρότασης~\ref{prop:artper}, έχουμε:
\[
  a_{n} = 0, \quad \text{και} \quad
  b_{n} = \frac{2}{L} \int _{0}^{L} f(x) \sin{\left(\frac{n \pi x}{L}\right)} \,{dx} 
\] 
και τότε η σειρά Fourier της $f$ γίνεται:
\[
  \boxed{\sum_{n=1}^{\infty} b_{n} \sin{\left(\frac{n \pi x}{L}\right)}} \quad 
  \text{\color{Col1} Ημιτονική σειρά Fourier}
\]



\section{Παραγώγιση και Ολοκλήρωση Σειρών Fourier}

Η Παραγώγιση όρο προς όρο της σειράς Fourier, γενικώς, \textbf{δεν} επιτρέπεται.
Η σειρά Fourier της συνάρτησης
\[
  f(x) = 
  \begin{cases}
    x, & - \pi < x < \pi \\
    0, & x= \pm \pi
  \end{cases} \quad \text{είναι η} \quad 
  2\left[ \sin{x} - \frac{\sin{2x}}{2} + \frac{\sin{(3x)}}{3} - \cdots\right]
\] 
η οποία συγκλίνει για κάθε $x$. Όμως η σειρά των παραγώγων των όρων της, είναι 
\[
  2[\cos{x} - \cos{2x} + \cos{(3x)} - \cdots] 
\] 
η οποία δεν συγκλίνει για κανένα $x$. Αυτό οφείλεται στο γεγονός ότι η περιοδική 
επέκταση της $f$ στο $\mathbb{R}$ δεν είναι συνεχής. Η επόμενη πρόταση, βεβαιώνει ότι 
η συνέχεια της περιοδικής συνάρτησης είναι μία από τις αναγκαίες συνθήκες για να είναι 
δυνατή παραγώγιση της σειράς Fourier όρο προς όρο.

\begin{thm}[Παραγώγισης]
  Αν η συνάρτηση $ f(x) $ είναι συνεχής στο $ [- \pi , \pi] $ με $ f(- \pi ) = f(\pi) $ 
  και η $ f'(x) $ είναι τμηματικά λεία στο $ [- \pi , \pi] $, τότε η σειρά Fourier 
  της $ f'(x) $ προκύπτει από την παραγώγιση όρο προς όρο της σειράς Fourier της 
  $ f(x) $, και συγκλίνει στην $ f'(x) $ για κάθε $x$ σημείο συνέχειας της $f'$, 
  ενώ συγκλίνει στην τιμή
  \[
    \frac{f'(x^{-})+f'(x^{+})}{2}   \quad \text{για κάθε $x$ σημείο ασυνέχειας της $f'$} 
  \] 
  Μάλιστα η σειρά Fourier της $ f(x) $ συγκλίνει ομοιόμορφα στην $ f(x) $.
\end{thm}

\begin{thm}[Ολοκλήρωσης]
  Αν η συνάρτηση $ f(x) $ είναι τμηματικά συνεχής στο $ [- \pi , \pi] $ και περιοδική 
  με περίοδο $ 2 \pi $, τότε η σειρά Fourier της $ f(x) $ 
  \[
    \frac{a_{0}}{2} + \sum_{n=1}^{\infty} \left[a_{n} \cos{(nx)} + b_{n} \sin{(nx)}\right] 
  \] 
  είτε συγκλίνει, είτε όχι, μπορεί να ολοκληρωθεί όρο προς όρο σε οποιοδήποτε διάστημα 
  $ [a,b] $.
\end{thm}



\section{Ιδιότητες των Σειρών Fourier}

\begin{enumerate}
  \item\label{idiot:one} Αν η συνάρτηση $ f $ είναι συνεχής στο $ [- \pi , \pi] $ με 
    $ f(- \pi) = f(\pi) $ και η $ f' $ είναι τμηματικά συνεχής στο $ [- \pi , \pi] $, 
    τότε οι σειρές 
    \[
      \sum_{n=1}^{\infty} \sqrt{a_{n}^{2}+b_{n}^{2}} \quad \text{καθώς και οι} \quad 
      \sum_{n=1}^{\infty} \abs{a_{n}} \quad \text{και} \quad
      \sum_{n=1}^{\infty} \abs{b_{n}}
    \] 
    όπου $ a_{n}, b_{n} $, οι συντελεστές Fourier, \textbf{συγκλίνουν}.
  \item Αν η σειρά Fourier της συνάρτησης $ f $, συγκλίνει απόλυτα και ομοιόμορφα 
    προς τη συνάρτηση $ f $ στο $ [- \pi , \pi] $, και $ a_{n}, b_{n} $ είναι οι 
    συντελεστές Fourier, τότε ισχύει:
    \[
      \boxed{\frac{1}{L} \int _{-L}^{L} [f(x)]^{2} \,{dx} = \frac{a_{0}^{2}}{2} +
      \sum_{n=1}^{\infty} (a_{n}^{2}+b_{n}^{2})} \quad \text{\textcolor{Col1}{Parseval}} 
    \] 
\end{enumerate}
\begin{rem}
\item {}
  \begin{myitemize}
    \item Επομένως, σύμφωνα με την πρόταση~\ref{prop:conv}, η σειρά Fourier της 
      συνάρτησης $ f $, που ικανοποιεί τις προϋποθέσεις της ιδιότητας~\ref{idiot:one} 
      θα συγκλίνει απόλυτα και ομοιόμορφα προς την $f$.
    \item Ουσιαστικά, η συνθήκη $ f(- \pi ) = f(\pi) $ εξασφαλίζει τη συνέχεια της 
      περιοδικής επέκτασης της $f$ στο $ \mathbb{R} $.
  \end{myitemize}
\end{rem}



\section{Μιγαδική Μορφή της Σειράς Fourier}



Έστω $f(t)$ μια τμηματικά λεία και περιοδική συνάρτηση με περίοδο $ T=2 \pi $. Τότε 
για τη σειρά Fourier, έχουμε:
\begin{align}
  \label{eq:fouri}
  f(t) &= \frac{a_{0}}{2} + \sum_{n=1}^{\infty} \left[a_{n} \cos{(nt)} + b_{n} \sin{(nt)}\right] \notag \\
       &= \frac{a_{0}}{2} + \sum_{n=1}^{\infty} \left[a_{n} \frac{e^{int}+e^{-int}}{2} + b_{n} \frac{e^{int}-e^{-int}}{2i} \right] \notag \\
       &= \frac{a_{0}}{2} + \sum_{n=1}^{\infty} \left[a_{n} \frac{e^{int}+e^{-int}}{2} - i b_{n} \frac{e^{int}-e^{-int}}{2}\right] \notag \\
       &= \underbrace{\frac{a_{0}}{2}}_{c_{0}} + \sum_{n=1}^{\infty} \left[\underbrace{\left(\frac{a_{n}-ib_{n}}{2}\right)}_{c_{n}}e^{int} +
       \left(\frac{a_{n}+ib_{n}}{2}\right) e^{-int}\right]
\end{align}
Θέτουμε $ c_{0} = \frac{a_{0}}{2} $ και $ c_{n} = \frac{a_{n}-ib_{n}}{2} $ και
παρατηρούμε ότι:
\begin{myitemize}
  \item Η τιμή $ c_{0} = \frac{a_{0}}{2} $ δίνεται από τον τύπο $ c_{n} =
    \frac{a_{n}- ib_{n}}{2} $ για $ n=0 $. Πράγματι: 
    $ c_{0} = \frac{a_{0}+ib_{0}}{2} \overset{b_{0}=0}{=} \frac{a_{0}}{2}$.
  \item 
    $ c_{-n} = \frac{a_{-n}-ib_{-n}}{2} = \frac{a_{n}+ib_{n}}{2} $, αφού 
    $ a_{n} $ άρτια και $ b_{n} $ περιττή.
  \item Ισχύει, προφανώς ότι $ c_{-n} = \frac{a_{n}+ib_{n}}{2} = \overline{c_{n}}
    $, είναι ο συζυγής του $ c_{n} = \frac{a_{n}-i b_{n}}{2} $.
\end{myitemize}
Επομένως, η σειρά~\eqref{eq:fouri}, γίνεται:
\[
  f(t) = c_{0} + \sum_{n=1}^{\infty} c_{n} e^{int} + \sum_{n=1}^{\infty} c_{-n}
  e^{-int} 
\] 
Αν θέσουμε $ k=-n $ το τελευταίο άθροισμα γίνεται $ \sum_{k=-\infty}^{-1} c_{k}
e^{ikt} $, οπότε η σειρά γράφεται στην πιο συμπαγή μορφή:
\[
  f(t) = c_{0} + \sum_{n=1}^{\infty} c_{n} e^{int} + \sum_{n=-\infty}^{-1} c_{n}
  e^{int} = \sum_{n=- \infty}^{\infty} c_{n} e^{int}
\]
Για τους συντελεστές, έχουμε:
\[
  c_{0} = \frac{a_{0}}{2} = \frac{1}{2\pi} \int _{- \pi }^{\pi} f(t) \,{dt}
\] 
\begin{align*}
  c_{n} &= \frac{a_{n}-ib_{n}}{2} = \frac{1}{2} 
  \left(\frac{1}{\pi} \int _{- \pi }^{\pi } f(t) \cos{(nt)} \,{dt} - i
  \frac{1}{\pi} \int _{- \pi }^{\pi } f(t) \sin{(nt)} \,{dt}  \right) \\ 
        &= \frac{1}{2 \pi} \int _{- \pi }^{\pi} f(t) [\cos{(nt)} - i \sin{(nt)}] 
        \,{dt} = \frac{1}{2 \pi} \int _{- \pi }^{\pi } f(t) e^{-int} \,{dt}
\end{align*} 
\begin{align*}
  c_{-n} &= \frac{a_{n}+ib_{n}}{2} = \frac{1}{2} 
  \left(\frac{1}{\pi} \int _{- \pi }^{\pi } f(t) \cos{(nt)} \,{dt} + i
  \frac{1}{\pi} \int _{- \pi }^{\pi } f(t) \sin{(nt)} \,{dt}  \right) \\ 
         &= \frac{1}{2 \pi} \int _{- \pi }^{\pi} f(t) [\cos{(nt)} + i \sin{(nt)}] 
         \,{dt} = \frac{1}{2 \pi} \int _{- \pi }^{\pi } f(t) e^{int} \,{dt} = 
         \frac{1}{2 \pi} \int _{- \pi }^{\pi } f(t) e^{-i(-n)t}) \,{dt} 
\end{align*}
και αν θέσουμε $ k=-n $ ο συντελεστής $ c_{-n} $ γίνεται $ c_{k} =
\frac{1}{2 \pi} \int _{- \pi }^{\pi} f(t) \mathrm{e}^{-ikt}\,{dt} $.

Άρα η σειρά Fourier στη μιγαδική της μορφή, είναι:
\[
  \boxed{f(t) = \sum_{n=- \infty}^{\infty} c_{n} e^{-int} \quad \text{με} \quad c_{n} =
  \frac{1}{2 \pi} \int _{- \pi}^{\pi } f(t) e^{-int} \,{dt}, \; n \in \mathbb{Z}}
\]

\begin{rem}
  Οι μιγαδικοί και οι πραγματικοί συντελεστές Fourier συνδέονται μέσω των σχέσεων:
  \begin{equation}\label{eq:complex_coef}
    a_{0}= c_{0} \quad \text{και} \quad a_{n}= c_{n}+c_{-n} \quad \text{και} \quad 
    b_{n} = i(c_{n}-c_{-n})
  \end{equation} 
\end{rem}

%todo παραδειγμα με Parseval δες Επιστημη Υλικων


\end{document}





