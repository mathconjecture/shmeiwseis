\documentclass[a4paper,12pt]{article}
\usepackage{etex}
%%%%%%%%%%%%%%%%%%%%%%%%%%%%%%%%%%%%%%
% Babel language package
\usepackage[english,greek]{babel}
% Inputenc font encoding
\usepackage[utf8]{inputenc}
%%%%%%%%%%%%%%%%%%%%%%%%%%%%%%%%%%%%%%

%%%%% math packages %%%%%%%%%%%%%%%%%%
\usepackage{amsmath}
\usepackage{amssymb}
\usepackage{amsfonts}
\usepackage{amsthm}
\usepackage{proof}

\usepackage{physics}

%%%%%%% symbols packages %%%%%%%%%%%%%%
\usepackage{dsfont}
\usepackage{stmaryrd}
%%%%%%%%%%%%%%%%%%%%%%%%%%%%%%%%%%%%%%%


%%%%%% graphicx %%%%%%%%%%%%%%%%%%%%%%%
\usepackage{graphicx}
\usepackage{color}
%\usepackage{xypic}
\usepackage[all]{xy}
\usepackage{calc}
%%%%%%%%%%%%%%%%%%%%%%%%%%%%%%%%%%%%%%%

\usepackage{enumerate}

\usepackage{fancyhdr}
%%%%% header and footer rule %%%%%%%%%
\setlength{\headheight}{14pt}
\renewcommand{\headrulewidth}{0pt}
\renewcommand{\footrulewidth}{0pt}
\fancypagestyle{plain}{\fancyhf{}
\fancyhead{}
\lfoot{}
\rfoot{\small \thepage}}
\fancypagestyle{vangelis}{\fancyhf{}
\rhead{\small \leftmark}
\lhead{\small }
\lfoot{}
\rfoot{\small \thepage}}
%%%%%%%%%%%%%%%%%%%%%%%%%%%%%%%%%%%%%%%

\usepackage{hyperref}
\usepackage{url}
%%%%%%% hyperref settings %%%%%%%%%%%%
\hypersetup{pdfpagemode=UseOutlines,hidelinks,
bookmarksopen=true,
pdfdisplaydoctitle=true,
pdfstartview=Fit,
unicode=true,
pdfpagelayout=OneColumn,
}
%%%%%%%%%%%%%%%%%%%%%%%%%%%%%%%%%%%%%%



\usepackage{geometry}
\geometry{left=25.63mm,right=25.63mm,top=36.25mm,bottom=36.25mm,footskip=24.16mm,headsep=24.16mm}

%\usepackage[explicit]{titlesec}
%%%%%% titlesec settings %%%%%%%%%%%%%
%\titleformat{\chapter}[block]{\LARGE\sc\bfseries}{\thechapter.}{1ex}{#1}
%\titlespacing*{\chapter}{0cm}{0cm}{36pt}[0ex]
%\titleformat{\section}[block]{\Large\bfseries}{\thesection.}{1ex}{#1}
%\titlespacing*{\section}{0cm}{34.56pt}{17.28pt}[0ex]
%\titleformat{\subsection}[block]{\large\bfseries{\thesubsection.}{1ex}{#1}
%\titlespacing*{\subsection}{0pt}{28.80pt}{14.40pt}[0ex]
%%%%%%%%%%%%%%%%%%%%%%%%%%%%%%%%%%%%%%

%%%%%%%%% My Theorems %%%%%%%%%%%%%%%%%%
\newtheorem{thm}{Θεώρημα}[section]
\newtheorem{cor}[thm]{Πόρισμα}
\newtheorem{lem}[thm]{λήμμα}
\theoremstyle{definition}
\newtheorem{dfn}{Ορισμός}[section]
\newtheorem{dfns}[dfn]{Ορισμοί}
\theoremstyle{remark}
\newtheorem{remark}{Παρατήρηση}[section]
\newtheorem{remarks}[remark]{Παρατηρήσεις}
%%%%%%%%%%%%%%%%%%%%%%%%%%%%%%%%%%%%%%%




\newcommand{\vect}[2]{(#1_1,\ldots, #1_#2)}
%%%%%%% nesting newcommands $$$$$$$$$$$$$$$$$$$
\newcommand{\function}[1]{\newcommand{\nvec}[2]{#1(##1_1,\ldots, ##1_##2)}}

\newcommand{\linode}[2]{#1_n(x)#2^{(n)}+#1_{n-1}(x)#2^{(n-1)}+\cdots +#1_0(x)#2=g(x)}

\newcommand{\vecoffun}[3]{#1_0(#2),\ldots ,#1_#3(#2)}


\input{tikz.tex}

\pgfplotsset{
  myaxis/.style={axis lines=center,axis line style={thick,blue!50,-stealth},tick label
  style={font=\large,blue!75,above=5pt},xtick=\empty,ytick=\empty,tick style={blue!50}},
  myplot/.style={Col1!75,ultra thick,samples=500,no marks,smooth,},
  dashed lines/.style={dashed,blue!50,ultra thin},
}

\newcommand{\twocolumnsiderp}[2]{\begin{minipage}[c]{0.50\linewidth}\raggedright
        #1
        \end{minipage}\hfill\begin{minipage}[c]{0.45\linewidth}\raggedright
        #2
    \end{minipage}
}


\pagestyle{vangelis}
% \everymath{\displaystyle}
\setcounter{chapter}{1}

\input{insbox}

\begin{document}

\chapter*{Σειρές Fourier}

\section{Περιοδικές Συναρτήσεις}

\vspace{\baselineskip}

\begin{dfn}
  Μια συνάρτηση $ f \colon \mathbb{R} \to \mathbb{R} $ λέγεται 
  \textcolor{Col1}{περιοδική} αν 
  $
  f(x+T)=f(x),\quad \forall x \in \mathbb{R} 
  $ 
  όπου $ T>0 $, ονομάζεται \textcolor{Col1}{περίοδος} της συνάρτησης.
\end{dfn}

\twocolumnsiderp{
\begin{examples}
\item {}
  \begin{enumerate}[i)]
    \item Η $ f(x) = \sin{x} $ είναι περιοδική με περίοδο $ T=2 \pi $.
    \item Η $ f(x) = \sin{nx} $ είναι περιοδική με περίοδο $ T= \frac{2 \pi}{n} $.
    \item Η $ f(x) = \tan{x} $ είναι περιοδική με περίοδο $ T= \pi $.
    \item Η $ f(x) = c, \; c \in \mathbb{R} $, η σταθερή συνάρτηση, είναι περιοδική με 
      περίοδο οποιονδήποτε θετικό αριθμό.
  \end{enumerate}
\end{examples}
}{
\twocolumnsides{
  \centering
  \begin{tikzpicture}[scale=0.5]
    \begin{axis}[myaxis,
      ymin=-2,ymax=2,
      xmin=-7,xmax=7,
      extra x ticks={6.28},
      extra x tick labels={$2 \pi$},
      domain=-2*pi:2*pi,
      ]
      \addplot[myplot] {sin(deg(x))} node[above right,pos=0.65]{\Large$\sin x$} ;
    \end{axis}
  \end{tikzpicture}
  }{
  \centering
  \begin{tikzpicture}[scale=0.5]
    \begin{axis}[myaxis,
      ymin=-6,ymax=6,
      xmin=-6,xmax=6,
      extra x ticks={-1.57,1.57},
      extra x tick labels={$-\frac{\pi}{2}$,$\frac{\pi}{2}$},
      tick label style={font=\Large,blue!75,below left=3pt},
      domain=-1.8*pi:1.8*pi,
      ]
      \addplot [myplot,dashed lines] {tan(deg(x))};
      \addplot [myplot,restrict y to domain=-10:10] {tan(deg(x))} node
        [right,pos=0.55,xshift=5pt] {\Large$\tan x$};
    \end{axis}
  \end{tikzpicture}
}}

\begin{prop}
  Αν οι συναρτήσεις $ f $ και $g$ είναι περιοδικές με περίοδο $T$, τότε και οι 
  $ f+g $ και $ f\cdot g $ είναι περιοδικές με την ίδια περίοδο.
\end{prop}
\begin{proof}
  \begin{align*}
    (f+g)(x+T) = f(x+T)+g(x+T)=f(x)+g(x)=(f+g)(x) \\
    (f\cdot g)(x+T) = f(x+T)\cdot g(x+T)=f(x)\cdot g(x)=(f\cdot g)(x) \hfill
  \end{align*}
\end{proof}

\begin{dfn}
  Μια συνάρτηση $ f \colon [a,b] \to \mathbb{R} $ λέγεται 
  \textcolor{Col1}{τμηματικά συνεχής}, αν η  $f$ έχει πεπερασμένου πλήθους 
  σημεία ασυνέχειας, στα οποία υπάρχουν τα πλευρικά όρια και είναι πεπερασμένα.
\end{dfn}

\begin{dfn}
  Μια συνάρτηση  $ f \colon [a,b] \to \mathbb{R} $  λέγεται  
  \textcolor{Col1}{τμηματικά λεία} αν η $f$ και η $f$' είναι τμηματικά συνεχείς 
  στο $ [a,b] $.     
\end{dfn}


\section{Περιοδική Επέκταση Συναρτήσεων}

\vspace{\baselineskip}

%todo fix δεν είναι ορισμοί, δες Κραββαρίτη
\begin{enumerate}

  \item Αν $ f(x) $ ορισμένη στο $ [-L,L] $ με $ f(-L)=f(L) $ τότε η συνάρτηση
    \[
      F(x) = f(x), \; -L \leq x \leq L \quad \text{και} \quad F(x+2L)=F(x)
    \]
    με περίοδο $ 2L $ λέγεται \textcolor{Col1}{περιοδική επέκταση} της $f$. 

  \item Αν $ f(x) $ ορισμένη στο $ [0,L] $, τότε μπορεί να επεκταθεί 
    κατά άπειρους τρόπους σε μια περιοδική συνάρτηση 
    \begin{alignat*}{4}
            &F(x) = 
            \begin{cases}  
              f(x), & \phantom{-} 0 \leq x \leq L \\
              \phi(x) & -L < x < 0
            \end{cases}  & \quad & \text{και} \quad F(x+2L)=F(x)
    \end{alignat*}
    με περίοδο $ 2L $, όπου $ \phi(x) $ αυθαίρετη συνάρτηση. 

  \item\label{artia} Αν $ f(x) $ ορισμένη στο $ [0,L] $ τότε η συνάρτηση 
    \begin{alignat*}{4}
            & F(x) = 
            \begin{cases}  
              f(x), & \phantom{-} 0 \leq x \leq L \\
              f(-x), & -L \leq x \leq 0
            \end{cases}  & \quad \text{και} \quad F(x+2L)=F(x)
    \end{alignat*}
    με περίοδο $ 2L $ λέγεται \textcolor{Col1}{άρτια περιοδική επέκταση} της $f$.

  \item\label{peritth} Αν $ f(x) $ ορισμένη στο $ [0,L] $, με $ f(0)=f(L)=0 $ 
    τότε η συνάρτηση
    \begin{alignat*}{4}
            &F(x) = 
            \begin{cases}  
              \phantom{-} f(x), & \phantom{-} 0 \leq x \leq L \\
              -f(-x), & -L \leq x \leq 0
            \end{cases}  & \quad \text{και} \quad F(x+2L)=F(x)
    \end{alignat*}
    με περίοδο $ 2L $ λέγεται \textcolor{Col1}{περιττή περιοδική επέκταση} της $f$.

  \item Αν $ f(x) $ ορισμένη στο $ (0,L) $, τότε επεκτείνεται σε 
    άρτια (αντ. περιττή) περιοδική συνάρτηση $ F(x) $ θέτωντας $ F(0) $, 
    $ F(L) $ οποιαδήποτε τιμή (αντ. $F(0)=F(L)=0$) και στη συνέχεια εφαρμόσουμε 
    τις επεκτάσεις~\ref{artia} και~\ref{peritth}.
\end{enumerate}


\section{Συντελεστές Fourier}

Έστω $ f \colon \mathbb{R} \to \mathbb{R} $ μια περιοδική συνάρτηση, με περίοδο 
$ 2 \pi $. Τότε υπό κατάλληλες προϋποθέσεις η $f$ μπορεί να αναπαρασταθεί ως 
\textcolor{Col1}{τριγωνομετρική} σειρά της μορφής:

\begin{equation}\label{eq:trigser}
  f(x) = \frac{a_{0}}{2} + \sum_{n=1}^{\infty} [a_{n} \cos{nx}  + b_{n} \sin{nx}]
\end{equation} 

Αν υποθέσουμε ότι υπάρχει η αναπαράσταση της $f$ ως τριγωνομετρική σειρά και ότι 
η σύγκλιση της σειράς είναι \textbf{ομοιόμορφη}, τότε αν πολλαπλασιάσουμε 
την~\eqref{eq:trigser} με $ \sin{mx} $ και ολοκληρώσουμε σε διάστημα μίας περιόδου: 

\begin{gather*}
  % f(x) = \frac{a_{0}}{2} + \sum_{n=1}^{\infty} [a_{n} \cos{nx} + b_{n} \sin{nx}]  \\
  f(x) \sin{mx} = \frac{a_{0}}{2} \sin{mx} + \sum_{n=1}^{\infty} 
  [a_{n} \cos{nx} \sin{mx} + b_{n} \sin{nx} \sin{mx}]   \\
  \int _{- \pi} ^{\pi} f(x) \sin{mx} \,{dx} = \frac{a_{0}}{2} \int _{- \pi}^{\pi} 
  \sin{mx} \,{dx} + \sum_{n=1}^{\infty} \left[a_{n} \int _{- \pi}^{\pi} 
    \cos{nx} \sin{mx}  \,{dx} + b_{n} \int _{- \pi }^{\pi} \sin{nx} \sin{mx} 
  \,{dx}\right] \\
  \int _{- \pi }^{\pi} f(x) \sin{mx} \,{dx} = \frac{a_{0}}{2} \cdot 0 + 
  a_{m}\cdot 0 + b_{m} \cdot \int _{- \pi}^{\pi } \sin^{2}{mx} \,{dx} \\
  \int _{- \pi }^{\pi} f(x) \sin{mx} \,{dx} =  b_{m} \cdot \pi 
\end{gather*}   

Οπότε θέτωντας όπου $ n=m $, έχουμε
\begin{equation}\label{eq:four1}
  \boxed{b_{n} = \frac{1}{\pi} \int _{- \pi }^{\pi} f(x) \sin{nx} \,{dx}, \quad n =
  1,2,3, \ldots}
\end{equation} 

Ομοίως αν πολλαπλασιάσουμε την~\eqref{eq:trigser} με $ \cos{mx} $ και ολοκληρώσουμε 
σε διάστημα μιας περιόδου, έχουμε:

\begin{gather*}
  % f(x) = \frac{a_{0}}{2} + \sum_{n=1}^{\infty} [a_{n} \cos{nx} + b_{n} \sin{nx}]  \\
  f(x) \cos{mx} = \frac{a_{0}}{2} \cos{mx} + \sum_{n=1}^{\infty} 
  [a_{n} \cos{nx} \cos{mx} + b_{n} \sin{nx} \cos{mx}]   \\
  \int _{- \pi} ^{\pi} f(x) \cos{mx} \,{dx} = \frac{a_{0}}{2} \int _{- \pi}^{\pi} 
  \cos{mx} \,{dx} + \sum_{n=1}^{\infty} \left[a_{n} \int _{- \pi}^{\pi}
    \cos{nx} \cos{mx}  \,{dx} + b_{n} \int _{- \pi }^{\pi} \sin{nx} \cos{mx} 
  \,{dx}\right] \\
\end{gather*}

και από τις συνθήκες ορθογωνιότητας για τις διάφορες περιπτώσεις του $ m $, προκύπτει

\begin{gather*}
  \begin{cases} 
    \displaystyle{\int _{- \pi }^{\pi} f(x) \cos{mx} \,{dx} = 
      \frac{a_{0}}{2} \cdot 0 + a_{m} \int _{- \pi }^{\pi } \cos^{2}{mx} 
    \,{dx} + b_{m} \cdot 0}, 
         & m = 1,2,3, \ldots \\
         \displaystyle{\int _{- \pi }^{\pi} f(x) \,{dx} = \frac{a_{0}}{2}} 
         \int _{- \pi}^{\pi} \,{dx}, 
         & m=0 \\
  \end{cases} 
  \intertext{άρα}
  \begin{cases} 
    \displaystyle{\int _{- \pi }^{\pi} f(x) \cos{nx} \,{dx} = 
    a_{n}\cdot \pi }, \quad n=1,2,3\ldots \\
    \displaystyle{\int _{- \pi }^{\pi} f(x) \,{dx} = \frac{a_{0}}{2}
    2 \pi } 
  \end{cases} 
  \Leftrightarrow 
  \begin{cases} 
    \displaystyle{a_{n} = \frac{1}{\pi} \int _{- \pi }^{\pi } f(x) \cos{nx} 
    \,{dx}}, \quad n=1,2,3\ldots \\
    \displaystyle{a_{0} = \frac{1}{\pi} \int _{- \pi}^{\pi} f(x) \,{dx}}
  \end{cases}
\end{gather*}
ή ισοδύναμα, μαζεύοντας
\begin{equation}\label{eq:four2}
  \boxed{a_{n} = \frac{1}{\pi} \int _{- \pi }^{\pi} f(x) \cos{nx} \,{dx}, \quad
  n = \textcolor{Col1}{0},1,2,3, \ldots}
\end{equation}

\begin{rem}
  Οι τύποι~\eqref{eq:four1} και~\eqref{eq:four2} ονομάζονται 
  \textcolor{Col1}{συντελεστές Fourier} και για τον υπολογισμό τους, 
  θεωρήσαμε ότι η σειρά~\eqref{eq:trigser} συγκλίνει ομοιόμορφα στην $f$, κάτι 
  που δεν το γνωρίζουμε εκ των προτέρων. 
  Γεγονός είναι ότι για κάθε συνάρτηση $f$ που είναι ολοκληρώσιμη στο 
  $ [- \pi , \pi] $, μπορούμε να υπολογίσουμε τους συντελεστές Fourier
  και τυπικά να σχηματίσουμε τη σειρά,γράφοντας
  \[
    f(x) \sim \frac{a_{0}}{2} + \sum_{n=1}^{\infty} [a_{n} \cos{nx} + b_{n} \sin{nx}]
  \] 
  η οποία ονομάζεται σειρά Fourier της $f$. Υπό ποιες προϋποθέσεις η σειρά Fourier 
  μιας συνάρτησης $f$ \textbf{συγκλίνει} στην $f$, απαντάται στο επόμενο θεώρημα.
\end{rem}

\section{Σειρές Fourier}

\begin{thm}
  Έστω $ f \colon \mathbb{R} \to \mathbb{R} $ μια \textbf{περιοδική} συνάρτηση με 
  περίοδο $ 2 \pi $ και \textbf{τμηματικά λεία}. Τότε η σειρά Fourier της $f$ 
  συγκλίνει για κάθε $ x \in \mathbb{R} $, στην τιμή
  \[
    \frac{f(x^{-}) + f(x^{+})}{2} 
  \] 
  όπου $ f({x_{0}}^{-}) = \lim\limits_{x \to {x_{0}}^{-}} f(x) $ και 
  $ f({x_{0}}^{+}) = \lim\limits_{x \to {x_{0}}^{+}} f(x) $ 
\end{thm}

\begin{rem}
\item {}
  \begin{myitemize}
    \item Αν $f$ συνεχής στο $x$, τότε η σειρά Fourier της $f$ συγκλίνει στο $f(x)$, 
      γιατί $ f(x^{-}) = f(x^{+}) $.
    \item Αν $f$ συνεχής για κάθε $x \in \mathbb{R} $, τότε η σειρά Fourier της $f$ 
      συγκλίνει \textbf{απόλυτα} και \textbf{ομοιόμορφα} στην $f$ στο $ \mathbb{R} $.
  \end{myitemize}
\end{rem}

\begin{rem}
  Δηλαδή συνοψίζοντας η σειρά Fourier μιας περιοδικής με περίοδο $ 2 \pi $ και 
  τμηματικά λείας συνάρτησης $ f $ δίνεται από τον τύπο 
  \[
    \frac{a_{0}}{2} + \sum_{n=1}^{\infty} [a_{n} \cos{(nx)} + b_{n} \sin{(nx)}] =  
    \frac{f(x^{-}) + f(x^{+})}{2} 
  \] 
  και οι συντελεστές $ a_{n} $ και $ b_{n} $ από τους τύπους~\eqref{eq:four1}
  και~\eqref{eq:four2}.
\end{rem}

\section{Συναρτήσεις με περίοδο 2L}

\begin{rem}
  Κάθε περιοδική συνάρτηση $ f(x) $ με περίοδο $ 2L>0 $ μπορεί να 
  \textbf{μετασχηματιστεί} σε μια συνάρτηση με περίοδο $ 2 \pi $. Πράγματι, θέτωντας 
  \[ 
    \frac{x}{t} = \frac{2L}{2 \pi} \Leftrightarrow \boxed{x= \frac{L}{\pi} t}
  \]
  η συνάρτηση $ g(t) = f\Bigl(\frac{L}{\pi}t\Bigr) $ που προκύπτει 
  έχει περίοδο $ 2 \pi $.
\end{rem}
Για την $ g(t) $ η σειρά Fourier είναι 
\[
  \frac{a_{0}}{2} + \sum_{n=1}^{\infty} [a_{n} \cos{(nt)} + b_{n} \sin{(nt)}]
\] 
όπου 
\[
  a_{n} = \frac{1}{\pi} \int _{- \pi}^{\pi} g(t) \cos{nt} \,{dt} = 
  \frac{1}{\pi} \int _{- \pi}^{\pi} f\Bigl(\frac{L}{\pi} t\Bigr) \cos{nt} \,{dt}, 
  \quad n = 0,1,2,3, \ldots
\] 
και 
\[
  b_{n} = \frac{1}{\pi} \int _{- \pi}^{\pi} g(t) \sin{nt} \,{dt} = 
  \frac{1}{\pi} \int _{- \pi}^{\pi} f\Bigl(\frac{L}{\pi} t\Bigr) \sin{nt} \,{dt}, 
  \quad n = 1,2,3, \ldots
\]
Επιστρέφοντας πίσω, στην αρχική μεταβλητή $x$, θέτοντας όπου $ t = \frac{\pi x}{L} $, 
έχουμε ότι η σειρά Fourier της συνάρτησης $ f(x) $ είναι
\[
  \boxed{\frac{a_{0}}{2} + \sum_{n=1}^{\infty} \left[a_{n} \cos{\left(\frac{n \pi x}{L}\right)} 
  + b_{n} \sin{\left(\frac{n \pi x}{L}\right)} \right]}
\]
όπου 
\[
  a_{n} = \frac{1}{L} \int _{- L}^{L} f(x) 
  \cos{\left(\frac{n \pi x}{L}\right)} \,{dx}, \quad n = 0,1,2,3, \ldots
\] 
και 
\[
  b_{n} = \frac{1}{L} \int _{- L}^{L} f(x) 
  \sin{\left(\frac{n \pi x}{L}\right)} \,{dx}, \quad n = 1,2,3, \ldots
\]

\section{Ημιτονική και Συνημιτονική σειρά Fourier}


\subsection*{Άρτιες - Περιττές Συναρτήσεις}

%todo add some graphs
\begin{dfn}
  Μια συνάρτηση $ f(x) $ θα λέγεται:
  \begin{alignat*}{4}
        &\textcolor{Col1}{\text{άρτια}}   & \quad 
        &\overset{\text{ορ.}}{\Leftrightarrow} & \quad f(-x) &= f(x), & 
        &\quad \forall x \in \mathbb{R} \\
        &\textcolor{Col1}{\text{περιττή}}  & \quad 
        &\overset{\text{ορ.}}{\Leftrightarrow} & \quad f(-x) &= -f(x), & 
        &\quad \forall x \in \mathbb{R}. \\
  \end{alignat*}
\end{dfn}

\begin{examples}
\item {}
  \begin{enumerate}
    \item Οι συναρτήσεις $ f(x)=x^{2} $ και $ g(x)= \cos (nx) $ είναι άρτιες. 
    \item συναρτήσεις $ f(x)=x^{3} $ και $ g(x)= \sin (nx) $ είναι περιττές.
  \end{enumerate}
\end{examples}

\begin{prop}
\item {}
  \begin{enumerate}\label{prop:ginart}
    \item άρτια $ \cdot $ άρτια $=$ άρτια
    \item περιττή $ \cdot $ περιττή $=$ άρτια
    \item άρτια $ \cdot $ περιττή $=$ περιττή
  \end{enumerate}
\end{prop}

%todo add some graphs showing the integrals relations
\begin{rem}
  Η γραφική παράσταση μιας άρτιας συνάρτησης είναι συμμετρική ως προς τον άξονα $y$, 
  ενώ μιας περιττής συνάρτησης ως προς την αρχή των αξόνων.
\end{rem}

\begin{prop}\label{prop:artper}
\item {}
  Αν $ f(x) $ ολοκληρώσιμη στο διάστημα $[-a,a]$ τότε ισχύουν:

  \begin{myitemize}
    \item $ f(x) $ άρτια $ \Rightarrow \int _{-a}^{a} f(x) \,{dx} = 2 \int _{0}^{a} f(x)
      \,{dx} $ 
    \item $ f(x) $ περιττή $ \Rightarrow \int _{-a}^{a} f(x) \,{dx} = 0 $
  \end{myitemize}
\end{prop}


\subsection*{Συνημιτονική σειρά Fourier συνάρτησης με περίοδο $ 2 \pi $}

Αν η $f$ είναι \textbf{άρτια}, με περίοδο $ 2 \pi $, τότε η συνάρτηση 
$ f(x) \cos{(nx)} $ είναι επίσης \textbf{άρτια} (από πρόταση~\ref{prop:ginart}), 
ενώ η συνάρτηση $ f(x) \sin{(nx)} $ είναι \textbf{περιττή} 
(από πρόταση~\ref{prop:ginart}), για αυτό, αν λάβουμε υπόψιν 
μας τις σχέσεις της πρότασης~\ref{prop:artper}, έχουμε:
\[
  a_{n} = \frac{2}{\pi} \int _{0}^{\pi} f(x) \cos{(nx)} \,{dx}, 
  \quad n=\textcolor{Col1}{0},1,2,\ldots \quad \text{και} \quad b_{n} = 0
\] 
και τότε η σειρά Fourier της $f$ γίνεται:
\[
  \boxed{\frac{a_{0}}{2} + \sum_{n=1}^{\infty} a_{n} \cos{(nx)}} \quad 
  \text{\color{Col1} Συνημιτονική σειρά Fourier}
\]


\subsection*{Ημιτονική σειρά Fourier συνάρτησης με περίοδο $ 2 \pi $}

Αν η $f$ είναι \textbf{περιττή}, με περίοδο $ 2 \pi $,  τότε η συνάρτηση 
$ f(x) \cos{(nx)} $ είναι επίσης \textbf{περιττή} (από πρόταση~\ref{prop:ginart}), 
ενώ η συνάρτηση $ f(x) \sin{(nx)} $ είναι \textbf{άρτια} 
(από πρόταση~\ref{prop:ginart}), για αυτό, αν λάβουμε υπόψιν μας 
τις σχέσεις της πρότασης~\ref{prop:artper}, έχουμε:
\[
  a_{n} = 0, \quad \text{και} \quad
  b_{n} = \frac{2}{\pi} \int _{0}^{\pi} f(x) \sin{(nx)} \,{dx} 
\] 
και τότε η σειρά Fourier της $f$ γίνεται:
\[
  \boxed{\sum_{n=1}^{\infty} b_{n} \sin{(nx)}} \quad \text{\color{Col1} Ημιτονική σειρά Fourier}
\]

\subsection*{Συνημιτονική σειρά Fourier συνάρτησης με περίοδο $ 2 L$}

Αν η $f$ είναι \textbf{άρτια}, με περίοδο $ 2 L$, τότε η συνάρτηση 
$ f(x) \cos{\left(\frac{n \pi x}{L}\right)} $ είναι επίσης \textbf{άρτια} 
(από πρόταση~\ref{prop:ginart}), 
ενώ η συνάρτηση $ f(x) \sin{\left(\frac{n \pi x}{L}\right)} $ είναι \textbf{περιττή} 
(από πρόταση~\ref{prop:ginart}), για αυτό, αν λάβουμε υπόψιν 
μας τις σχέσεις της πρότασης~\ref{prop:artper}, έχουμε:
\[
  a_{n} = \frac{2}{L} \int _{0}^{L} f(x) \cos{\left(\frac{n \pi x}{L}\right)} \,{dx}, 
  \quad n=\textcolor{Col1}{0},1,2,\ldots \quad \text{και} \quad b_{n} = 0
\] 
και τότε η σειρά Fourier της $f$ γίνεται:
\[
  \boxed{\frac{a_{0}}{2} + \sum_{n=1}^{\infty} a_{n} \cos{\left(\frac{n \pi
  x}{L}\right)}} \quad 
  \text{\color{Col1} Συνημιτονική σειρά Fourier}
\]


\subsection*{Ημιτονική σειρά Fourier συνάρτησης με περίοδο $ 2 L$}

Αν η $f$ είναι \textbf{περιττή}, με περίοδο $ 2 L$,  τότε η συνάρτηση 
$ f(x) \cos{\left(\frac{n \pi x}{L}\right)} $ είναι επίσης \textbf{περιττή} 
(από πρόταση~\ref{prop:ginart}), ενώ η συνάρτηση 
$ f(x) \sin{\left(\frac{n \pi x}{L}\right)} $ είναι \textbf{άρτια} 
(από πρόταση~\ref{prop:ginart}), για αυτό, αν λάβουμε υπόψιν μας 
τις σχέσεις της πρότασης~\ref{prop:artper}, έχουμε:
\[
  a_{n} = 0, \quad \text{και} \quad
  b_{n} = \frac{2}{L} \int _{0}^{L} f(x) \sin{\left(\frac{n \pi x}{L}\right)} \,{dx} 
\] 
και τότε η σειρά Fourier της $f$ γίνεται:
\[
  \boxed{\sum_{n=1}^{\infty} b_{n} \sin{\left(\frac{n \pi x}{L}\right)}} \quad 
  \text{\color{Col1} Ημιτονική σειρά Fourier}
\]

\end{document}
