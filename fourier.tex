\input{$HOME/Desktop/tufte/preamble.tex}
\newcommand{\vect}[2]{(#1_1,\ldots, #1_#2)}
%%%%%%% nesting newcommands $$$$$$$$$$$$$$$$$$$
\newcommand{\function}[1]{\newcommand{\nvec}[2]{#1(##1_1,\ldots, ##1_##2)}}

\newcommand{\linode}[2]{#1_n(x)#2^{(n)}+#1_{n-1}(x)#2^{(n-1)}+\cdots +#1_0(x)#2=g(x)}

\newcommand{\vecoffun}[3]{#1_0(#2),\ldots ,#1_#3(#2)}



\pagestyle{vangelis}
\everymath{\displaystyle}


\begin{document}

\chapter{Σειρές \textlatin{Fourier}}


\section{Περιοδικές Συναρτήσεις}

\vspace{\baselineskip}

\begin{dfn}
    Μια συνάρτηση $ f \colon \mathbb{R} \to \mathbb{R} $ λέγεται \textcolor{magenta}{περιοδική} αν 
    \[
        f(x+T)=f(x),\quad \forall x \in \mathbb{R} 
    \] 
    όπου $ T>0 $, ονομάζεται \textcolor{magenta}{περίοδος} της συνάρτησης.
\end{dfn}

\begin{rem}
    Κάθε περιοδική συνάρτηση $ f(x) $ με περίοδo $ 2L>0 $ μπορεί να μετασχηματιστεί σε μια συνάρτηση
    με περίοδο $ 2 \pi $. Πράγματι, θέτωντας 
    \[ 
        \frac{x}{t} = \frac{2L}{2 \pi} \Leftrightarrow x= t \frac{L}{\pi} 
    \]
    η συνάρτηση $ g(t) = f\Bigl(t \frac{L}{\pi}\Bigr) $ που προκύπτει έχει περίοδο $ 2 \pi $.
\end{rem}

\begin{dfn}
    Μια συνάρτηση $ f \colon [a,b] \to \mathbb{R} $ λέγεται \textcolor{magenta}{τμηματικά συνεχής},
    αν η  $f$ έχει πεπερασμένου πλήθους σημεία ασυνέχειας, στα οποία υπάρχουν τα πλευρικά όρια και
    είναι πεπερασμένα.
\end{dfn}

\begin{dfn}
    Μια συνάρτηση  $ f \colon [a,b] \to \mathbb{R} $  λέγεται  \textcolor{magenta}{τμηματικά λεία}
    αν η $f$ και η $f$' είναι τμηματικά συνεχείς στο $ [a,b] $.     
\end{dfn}

\begin{dfn}
    Μια συνάρτηση $ f(x) $ θα λέγεται:
    \begin{alignat*}{4}
        &\textcolor{magenta}{\text{άρτια}}   & \quad &\overset{\text{ορ.}}{\Leftrightarrow} & \quad f(-x) &= f(x), & &\quad \forall x \in \mathbb{R} \\
        &\textcolor{magenta}{\text{περιττή}}  & \quad &\overset{\text{ορ.}}{\Leftrightarrow} & \quad f(-x) &= -f(x), & &\quad \forall x \in \mathbb{R}. \\
    \end{alignat*}
\end{dfn}



\section{Περιοδική Επέκταση Συναρτήσεων}

\vspace{\baselineskip}

\begin{enumerate}

    \item Αν $ f(x) $ ορισμένη στο $ [-L,L] $ με $ f(-L)=f(L) $ τότε η συνάρτηση
        \[
            F(x) = f(x), \; -L \leq x \leq L \quad \text{και} \quad F(x+2L)=F(x)
        \]
        με περίοδο $ 2L $ λέγεται \textcolor{magenta}{περιοδική επέκταση} της $f$. 

    \item Αν $ f(x) $ ορισμένη στο $ [0,L] $, τότε μπορεί να επεκταθεί κατά άπειρους τρόπους 
        σε μια περιοδική συνάρτηση 
        \begin{alignat*}{4}
            &F(x) = 
            \begin{cases}  
            f(x), & \phantom{-} 0 \leq x \leq L \\
            \phi(x) & -L < x < 0
        \end{cases}  & \quad & \text{και} \quad F(x+2L)=F(x)
        \end{alignat*}
        με περίοδο $ 2L $, όπου $ \phi(x) $ αυθαίρετη συνάρτηση. 

    \item \label{artia} Αν $ f(x) $ ορισμένη στο $ [0,L] $ τότε η συνάρτηση 
        \begin{alignat*}{4}
            &F(x) = 
            \begin{cases}  
            f(x), & \phantom{-} 0 \leq x \leq L \\
            f(-x), & -L \leq x \leq 0
        \end{cases}  & \quad & \text{και} \quad F(x+2L)=F(x)
        \end{alignat*}
        με περίοδο $ 2L $ λέγεται \textcolor{magenta}{άρτια περιοδική επέκταση} της $f$.

    \item \label{peritth} Αν $ f(x) $ ορισμένη στο $ [0,L] $, με $ f(0)=f(L)=0 $ τότε η συνάρτηση
        \begin{alignat*}{4}
            &F(x) = 
            \begin{cases}  
            \phantom{-} f(x), & \phantom{-} 0 \leq x \leq L \\
            -f(-x), & -L \leq x \leq 0
        \end{cases}  & \quad & \text{και} \quad F(x+2L)=F(x)
        \end{alignat*}
        με περίοδο $ 2L $ λέγεται \textcolor{magenta}{περιττή περιοδική επέκταση} της $f$.

    \item Αν $ f(x) $ ορισμένη στο $ (0,L) $, τότε επεκτείνεται σε άρτια (αντ. περιττή) περιοδική
        συνάρτηση $ F(x) $ θέτωντας $ F(0) $, $ F(L) $ οποιαδήποτε τιμή (αντ. $F(0)=F(L)$) και στη
        συνέχεια εφαρμόσουμε τις επεκτάσεις \ref{artia} και \ref{peritth}.

\end{enumerate}





\end{document}
