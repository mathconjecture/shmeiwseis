\documentclass[a4paper,12pt]{article}
\usepackage{etex}
%%%%%%%%%%%%%%%%%%%%%%%%%%%%%%%%%%%%%%
% Babel language package
\usepackage[english,greek]{babel}
% Inputenc font encoding
\usepackage[utf8]{inputenc}
%%%%%%%%%%%%%%%%%%%%%%%%%%%%%%%%%%%%%%

%%%%% math packages %%%%%%%%%%%%%%%%%%
\usepackage{amsmath}
\usepackage{amssymb}
\usepackage{amsfonts}
\usepackage{amsthm}
\usepackage{proof}

\usepackage{physics}

%%%%%%% symbols packages %%%%%%%%%%%%%%
\usepackage{dsfont}
\usepackage{stmaryrd}
%%%%%%%%%%%%%%%%%%%%%%%%%%%%%%%%%%%%%%%


%%%%%% graphicx %%%%%%%%%%%%%%%%%%%%%%%
\usepackage{graphicx}
\usepackage{color}
%\usepackage{xypic}
\usepackage[all]{xy}
\usepackage{calc}
%%%%%%%%%%%%%%%%%%%%%%%%%%%%%%%%%%%%%%%

\usepackage{enumerate}

\usepackage{fancyhdr}
%%%%% header and footer rule %%%%%%%%%
\setlength{\headheight}{14pt}
\renewcommand{\headrulewidth}{0pt}
\renewcommand{\footrulewidth}{0pt}
\fancypagestyle{plain}{\fancyhf{}
\fancyhead{}
\lfoot{}
\rfoot{\small \thepage}}
\fancypagestyle{vangelis}{\fancyhf{}
\rhead{\small \leftmark}
\lhead{\small }
\lfoot{}
\rfoot{\small \thepage}}
%%%%%%%%%%%%%%%%%%%%%%%%%%%%%%%%%%%%%%%

\usepackage{hyperref}
\usepackage{url}
%%%%%%% hyperref settings %%%%%%%%%%%%
\hypersetup{pdfpagemode=UseOutlines,hidelinks,
bookmarksopen=true,
pdfdisplaydoctitle=true,
pdfstartview=Fit,
unicode=true,
pdfpagelayout=OneColumn,
}
%%%%%%%%%%%%%%%%%%%%%%%%%%%%%%%%%%%%%%



\usepackage{geometry}
\geometry{left=25.63mm,right=25.63mm,top=36.25mm,bottom=36.25mm,footskip=24.16mm,headsep=24.16mm}

%\usepackage[explicit]{titlesec}
%%%%%% titlesec settings %%%%%%%%%%%%%
%\titleformat{\chapter}[block]{\LARGE\sc\bfseries}{\thechapter.}{1ex}{#1}
%\titlespacing*{\chapter}{0cm}{0cm}{36pt}[0ex]
%\titleformat{\section}[block]{\Large\bfseries}{\thesection.}{1ex}{#1}
%\titlespacing*{\section}{0cm}{34.56pt}{17.28pt}[0ex]
%\titleformat{\subsection}[block]{\large\bfseries{\thesubsection.}{1ex}{#1}
%\titlespacing*{\subsection}{0pt}{28.80pt}{14.40pt}[0ex]
%%%%%%%%%%%%%%%%%%%%%%%%%%%%%%%%%%%%%%

%%%%%%%%% My Theorems %%%%%%%%%%%%%%%%%%
\newtheorem{thm}{Θεώρημα}[section]
\newtheorem{cor}[thm]{Πόρισμα}
\newtheorem{lem}[thm]{λήμμα}
\theoremstyle{definition}
\newtheorem{dfn}{Ορισμός}[section]
\newtheorem{dfns}[dfn]{Ορισμοί}
\theoremstyle{remark}
\newtheorem{remark}{Παρατήρηση}[section]
\newtheorem{remarks}[remark]{Παρατηρήσεις}
%%%%%%%%%%%%%%%%%%%%%%%%%%%%%%%%%%%%%%%




\newcommand{\vect}[2]{(#1_1,\ldots, #1_#2)}
%%%%%%% nesting newcommands $$$$$$$$$$$$$$$$$$$
\newcommand{\function}[1]{\newcommand{\nvec}[2]{#1(##1_1,\ldots, ##1_##2)}}

\newcommand{\linode}[2]{#1_n(x)#2^{(n)}+#1_{n-1}(x)#2^{(n-1)}+\cdots +#1_0(x)#2=g(x)}

\newcommand{\vecoffun}[3]{#1_0(#2),\ldots ,#1_#3(#2)}





\begin{document}

\chapter{Σειρές Fourier}


\section{Περιοδικές Συναρτήσεις}

\vspace{\baselineskip}

\begin{dfn}
    Μια συνάρτηση $ f \colon \mathbb{R} \to \mathbb{R} $ λέγεται 
    \textcolor{Col1}{περιοδική} αν 
    \[
        f(x+T)=f(x),\quad \forall x \in \mathbb{R} 
    \] 
    όπου $ T>0 $, ονομάζεται \textcolor{Col1}{περίοδος} της συνάρτησης.
\end{dfn}

\begin{rem}
    Κάθε περιοδική συνάρτηση $ f(x) $ με περίοδo $ 2L>0 $ μπορεί να 
    μετασχηματιστεί σε μια συνάρτηση με περίοδο $ 2 \pi $. Πράγματι, θέτωντας 
    \[ 
        \frac{x}{t} = \frac{2L}{2 \pi} \Leftrightarrow x= t \frac{L}{\pi} 
    \]
    η συνάρτηση $ g(t) = f\Bigl(t \frac{L}{\pi}\Bigr) $ που προκύπτει 
    έχει περίοδο $ 2 \pi $.
\end{rem}

\begin{dfn}
    Μια συνάρτηση $ f \colon [a,b] \to \mathbb{R} $ λέγεται 
    \textcolor{Col1}{τμηματικά συνεχής}, αν η  $f$ έχει πεπερασμένου πλήθους 
    σημεία ασυνέχειας, στα οποία υπάρχουν τα πλευρικά όρια και είναι πεπερασμένα.
\end{dfn}

\begin{dfn}
    Μια συνάρτηση  $ f \colon [a,b] \to \mathbb{R} $  λέγεται  
    \textcolor{Col1}{τμηματικά λεία} αν η $f$ και η $f$' είναι τμηματικά συνεχείς 
    στο $ [a,b] $.     
\end{dfn}

\begin{dfn}
    Μια συνάρτηση $ f(x) $ θα λέγεται:
    \begin{alignat*}{4}
        &\textcolor{Col1}{\text{άρτια}}   & \quad 
        &\overset{\text{ορ.}}{\Leftrightarrow} & \quad f(-x) &= f(x), & 
        &\quad \forall x \in \mathbb{R} \\
        &\textcolor{Col1}{\text{περιττή}}  & \quad 
        &\overset{\text{ορ.}}{\Leftrightarrow} & \quad f(-x) &= -f(x), & 
        &\quad \forall x \in \mathbb{R}. \\
    \end{alignat*}
\end{dfn}



\section{Περιοδική Επέκταση Συναρτήσεων}

\vspace{\baselineskip}

\begin{enumerate}

    \item Αν $ f(x) $ ορισμένη στο $ [-L,L] $ με $ f(-L)=f(L) $ τότε η συνάρτηση
        \[
            F(x) = f(x), \; -L \leq x \leq L \quad \text{και} \quad F(x+2L)=F(x)
        \]
        με περίοδο $ 2L $ λέγεται \textcolor{Col1}{περιοδική επέκταση} της $f$. 

    \item Αν $ f(x) $ ορισμένη στο $ [0,L] $, τότε μπορεί να επεκταθεί 
        κατά άπειρους τρόπους σε μια περιοδική συνάρτηση 
        \begin{alignat*}{4}
            &F(x) = 
            \begin{cases}  
                f(x), & \phantom{-} 0 \leq x \leq L \\
                \phi(x) & -L < x < 0
            \end{cases}  & \quad & \text{και} \quad F(x+2L)=F(x)
        \end{alignat*}
        με περίοδο $ 2L $, όπου $ \phi(x) $ αυθαίρετη συνάρτηση. 

    \item \label{artia} Αν $ f(x) $ ορισμένη στο $ [0,L] $ τότε η συνάρτηση 
        \begin{alignat*}{4}
            & F(x) = 
            \begin{cases}  
                f(x), & \phantom{-} 0 \leq x \leq L \\
                f(-x), & -L \leq x \leq 0
            \end{cases}  & \quad \text{και} \quad F(x+2L)=F(x)
        \end{alignat*}
        με περίοδο $ 2L $ λέγεται \textcolor{Col1}{άρτια περιοδική επέκταση} της $f$.

    \item \label{peritth} Αν $ f(x) $ ορισμένη στο $ [0,L] $, με $ f(0)=f(L)=0 $ 
        τότε η συνάρτηση
        \begin{alignat*}{4}
            &F(x) = 
            \begin{cases}  
                \phantom{-} f(x), & \phantom{-} 0 \leq x \leq L \\
                -f(-x), & -L \leq x \leq 0
            \end{cases}  & \quad \text{και} \quad F(x+2L)=F(x)
        \end{alignat*}
        με περίοδο $ 2L $ λέγεται \textcolor{Col1}{περιττή περιοδική επέκταση} της $f$.

    \item Αν $ f(x) $ ορισμένη στο $ (0,L) $, τότε επεκτείνεται σε 
        άρτια (αντ. περιττή) περιοδική συνάρτηση $ F(x) $ θέτωντας $ F(0) $, 
        $ F(L) $ οποιαδήποτε τιμή (αντ. $F(0)=F(L)$) και στη συνέχεια εφαρμόσουμε 
        τις επεκτάσεις~\ref{artia} και~\ref{peritth}.
\end{enumerate}


\section{Τριγωνομετρικές Σειρές}

Έστω $ f \colon \mathbb{R} \to \mathbb{R} $ μια περιοδική συνάρτηση, με περίοδο 
$ 2 \pi $. Τότε υπό κατάλληλες προυποθέσεις η $f$ μπορεί να αναπαρασταθεί ως 
\textcolor{Col1}{τριγωνομετρική} σειρά της μορφής:

\begin{equation}\label{eq:trigser}
    f(x) = \frac{a_{0}}{2} + \sum_{n=1}^{\infty} [a_{n} \cos{nx}  + b_{n} \sin{nx}]
\end{equation} 

Αν υποθέσουμε ότι υπάρχει η αναπαράσταση της $f$ ως τριγωνομετρική σειρά και ότι 
η σύγκλιση της σειράς είναι ομοιόμορφη, τότε αν πολλαπλασιάσουμε την~\eqref{eq:trigser}
με $ \sin{mx} $ και ολοκληρώσουμε σε διάστημα μίας περιόδου: 

\begin{gather*}
    f(x) = \frac{a_{0}}{2} + \sum_{n=1}^{\infty} [a_{n} \cos{nx} + b_{n} \sin{nx}]  \\
    f(x) \sin{mx} = \frac{a_{0}}{2} \sin{mx} + \sum_{n=1}^{\infty} 
    [a_{n} \cos{nx} = \sin{mx} + b_{n} \sin{nx} \sin{mx}]   \\
    \int _{- \pi} ^{\pi} f(x) \sin{mx} \,{dx} = \frac{a_{0}}{2} \int _{- \pi}^{\pi} 
    \sin{mx} \,{dx} + \sum_{n=1}^{\infty} \left[a_{n} \int _{- \pi}^{\pi} 
        \cos{nx} \sin{mx}  \,{dx} + b_{n} \int _{- \pi }^{\pi} \sin{nx} \sin{mx} 
    \,{dx}\right] \\
    \int _{- \pi }^{\pi} f(x) \sin{nx} \,{dx} = \frac{a_{0}}{2} \cdot 0 + 
    a_{n}\cdot 0 + b_{n} \cdot \int _{- \pi}^{\pi } \sin^{2}{nx} \,{dx} \\
    \int _{- \pi }^{\pi} f(x) \sin{nx} \,{dx} =  b_{n} \cdot \pi 
\end{gather*}   

Οπότε 
\[
    \boxed{b_{n} = \frac{1}{\pi} \int _{- \pi }^{\pi} f(x) \sin{nx} \,{dx}}
\] 

Ομοίως αν πολλαπλασιάσουμε την~\eqref{eq:trigser} με $ \cos{mx} $ και ολοκληρώσουμε 
σε διάστημα μιας περιόδου, έχουμε:

\begin{gather*}
    f(x) = \frac{a_{0}}{2} + \sum_{n=1}^{\infty} [a_{n} \cos{nx} + b_{n} \sin{nx}]  \\
    f(x) \cos{mx} = \frac{a_{0}}{2} \cos{mx} + \sum_{n=1}^{\infty} 
    [a_{n} \cos{nx} \cos{mx} + b_{n} \sin{nx} \cos{mx}]   \\
    \int _{- \pi} ^{\pi} f(x) \cos{mx} \,{dx} = \frac{a_{0}}{2} \int _{- \pi}^{\pi} 
    \cos{mx} \,{dx} + \sum_{n=1}^{\infty} \left[a_{n} \int _{- \pi}^{\pi}
        \cos{nx} \cos{mx}  \,{dx} + b_{n} \int _{- \pi }^{\pi} \sin{nx} \cos{mx} 
    \,{dx}\right] \\
\end{gather*}

και από τις συνθήκες ορθογωνιότητας για τις διάφορες περιπτώσεις του $ m $, προκύπτει

\begin{gather*}
    \begin{cases} 
        \displaystyle{\int _{- \pi }^{\pi} f(x) \cos{mx} \,{dx} = 
            \frac{a_{0}}{2} \cdot 0 + a_{m} \int _{- \pi }^{\pi } \cos^{2}{mx} 
        \,{dx} + b_{m} \cdot 0}, 
         & m = 1,2,3, \ldots \\
         \displaystyle{\int _{- \pi }^{\pi} f(x) \,{dx} = \frac{a_{0}}{2}} 
         \int _{- \pi}^{\pi} \,{dx}, 
         & m=0 \\
    \end{cases} 
    \intertext{άρα}
    \begin{cases} 
        \displaystyle{\int _{- \pi }^{\pi} f(x) \cos{nx} \,{dx} = 
        a_{n}\cdot \pi } \\
        \displaystyle{\int _{- \pi }^{\pi} f(x) \,{dx} = \frac{a_{0}}{2}
        2 \pi } 
    \end{cases} 
    \Leftrightarrow 
    \begin{cases} 
        \displaystyle{a_{n} = \frac{1}{\pi} \int _{- \pi }^{\pi } f(x) \cos{nx} 
        \,{dx}} \\
        \displaystyle{a_{0} = \frac{1}{\pi} \int _{- \pi}^{\pi} f(x) \,{dx}}
    \end{cases}
    \intertext{ή ισοδύναμα, μαζεύοντας}
    \boxed{a_{n} = \frac{1}{\pi} \int _{- \pi }^{\pi} f(x) \cos{nx} \,{dx}, 
    n = 0,1,2,3, \ldots}
\end{gather*}

\begin{thm}
    Έστω $ f \colon \mathbb{R} \to \mathbb{R} $ μια περιοδική συνάρτηση με περίοδο 
    $ 2 \pi $ και τμηματικά λεία. Τότε η σειρά Fourier της $f$ συγκλίνει για κάθε 
    $ x \in \mathbb{R} $, στην τιμή
    \[
        \frac{f(x^{-}) + f(x^{+})}{2} 
    \] 
\end{thm}

\begin{rem}
\item {}
    \begin{myitemize}
    \item Αν $f$ συνεχής στο $x$, τότε η σειρά Fourier συγκλίνει στο f(x), γιατί 
        $ f(x^{-}) = f(x^{+}) $.
    \item Αν $f$ συνεχής για κάθε $x \in \mathbb{R} $, τότε η σειρά Fourier 
        συγκλίνει απόλυτα και ομοιόμορφα στην $f$ στο $ \mathbb{R} $.
    \end{myitemize}
\end{rem}

\section{Ορθογωνιότητα}

\begin{dfn}
    Έστω δυο πραγματικές συναρτήσεις $ f(x), g(x) $ ορισμένες σ᾽ ένα διάστημα 
    $ [a,b] \subseteq \mathbb{R} $. Ορίζουμε ως \textcolor{Col1}{εσωτερικό γινόμενο} 
    των $ f(x) $ και $ g(x) $ και το συμβολίζουμε με $ <f,g> $ το ολοκλήρωμα
    \[
        <f,g> = \int _{a}^{b} w(x) f(x)g(x) \,{dx} 
    \] 
    όπου $ w(x) $ γνωστή θετική συνάρτηση, $ \forall x \in [a,b] $, η οποία 
    ονομάζεται \textcolor{Col1}{συνάρτηση βάρους}.
\end{dfn}

\begin{rem}
    Τα όρια ολοκλήρωσης στο παραπάνω ολοκλήρωμα μπορεί να είναι και $ \infty $.
\end{rem}

\begin{dfn}
    Δυο πραγματικές συναρτήσεις $ f(x), g(x) $ είναι \textcolor{Col1}{ορθογώνιες} 
    στο διάστημα $ [a,b] \subseteq \mathbb{R} $ ως προς τη συνάρτηση βάρους $ w(x) $, 
    αν το εσωτερικό του γινόμενο είναι μηδέν, δηλαδή:
    \[
        \int _{a}^{b} w(x)f(x)g(x) \,{dx} = 0
    \] 
\end{dfn}

\begin{dfn}
    Μια ακολουθία πραγματικών συναρτήσεων $ \{ f_{n}(x) \}_{n=1}^{\infty} $ αποτελεί 
    ένα σύστημα ορθογώνιων συναρτήσεων στο διάστημα $ [a,b] $, ως προς τη 
    συνάρτηση βάρους $ w(x) $, αν τα μέλη της είναι ανα δύο ορθογώνια, δηλαδή:
    \[
        \int _{a}^{b} w(x) f_{n}(x)f_{m}(x) \,{dx} = 0 \quad \forall n \neq m, \; n,m = 
        1,2,3, \ldots
    \] 
\end{dfn}

\begin{prop}
    Το σύστημα των συναρτήσεων $ \{ \cos{(nx)} \}_{n=1}^{\infty} $ είναι ορθογώνιο 
    στο διάστημα $ [0, \pi] $, ως προς τη συνάρτηση βάρους $ w(x)=1 $.

    \begin{proof}
    \item {}
        \begin{myitemize}
        \item Αν $ n \neq m $, τότε:
            \begin{align*}
                \int _{0}^{\pi} \cos{nx} \cdot \cos{mx} \,{dx} &= \frac{1}{2} 
                \int _{0}^{\pi} [\cos{(nx+mx)} + \cos{(nx - mx)}] \,{dx} = 
                \frac{1}{2} 
                \left[\frac{\sin{(n+m)x}}{n+m} + \frac{\sin{(n-m)x}}{n-m}
                \right]_{0}^{\pi} = 0 \\
            \end{align*} 

        \item Αν $ n = m $, τότε:
            \begin{align*}
                \int _{0}^{\pi} cos^{2}(nx) \,{dx} = 
                \int _{0}^{\pi} \frac{1+ \cos{2nx}}{2}
                \,{dx} = 
                \frac{1}{2} \left[x + \frac{\sin{2nx}}{2n}\right]_{0}^{\pi} =
                \frac{\pi}{2} 
            \end{align*}
        \end{myitemize}
    \end{proof}
\end{prop}


\begin{prop}
    Το σύστημα των συναρτήσεων $ \{ \sin{(nx)} \} _{n=1}^{\infty} $ είναι ορθογώνιο
    στο διάστημα $ [0, \pi] $, ως προς τη συνάρτηση βάρους $ w(x)=1 $.
\end{prop}

\begin{proof}
\item {}
    Ομοίως
\end{proof}

\begin{prop}
    Το σύστημα των συναρτήσεων Bessel 1ου είδους $ \{ J_{n}(x) \} _{n=1}^{\infty} $ 
    είναι ορθογώνιο στο διάστημα $ [0,a] $, ως προς τη συνάρτηση βάρους $ w(x)=x $.
\end{prop}

\begin{prop}
    Το σύστημα των πραγματικών συναρτήσεων 
    $ \{ 1, \sin{\frac{n \pi x}{L}, \cos{\frac{n \pi x}{L} } } \}_{n=1}^{\infty} $ 
    είναι ορθογώνιο στο $ [-L,L] $, με $ L>0 $, ως προς τη συνάρτηση βάρους 
    $ w(x)=1 $.
\end{prop}

\begin{proof}
\item {}
    \begin{myitemize}
    \item 
        \[
            \int _{-L}^{L} \sin{\frac{n \pi x}{L}} \,{dx} =  
            \left[\frac{- \cos{\frac{n \pi x}{L}}}{\frac{n \pi }{L}} \right]_{-L}^{L} 
            = - \frac{L}{n \pi} \left[\cos{\frac{n \pi x}{L}} \right]_{-L}^{L} = 
            - \frac{L}{n \pi} [ \cos{n \pi} - \cos{(- n \pi)}] = - \frac{L}{n \pi } 
            [ \cos{n \pi}- \cos{n \pi}] = 0 
        \] 

    \item 
        \[
            \int _{-L}^{L} \cos{\frac{n \pi x}{L}} \,{dx} =  
            \left[\frac{\sin{\frac{n \pi x}{L}}}{\frac{n \pi }{L}} \right]_{-L}^{L} 
            = \frac{L}{n \pi} \left[\sin{\frac{n \pi x}{L}} \right]_{-L}^{L} = 
            \frac{L}{n \pi} [ \sin{n \pi} - \sin{(- n \pi)}] =  \frac{2L}{n \pi } 
            \sin{n \pi} = 0 
        \]

    \item 
        \begin{align*}
            \int _{-L}^{L} \sin{\frac{n \pi x }{L}} \sin{\frac{m \pi x}{L}} \,{dx} 
            &= \frac{1}{2} \int _{-L}^{L} \left[\cos{\frac{(n-m) \pi x}{L}} - 
            \cos{\frac{(n+m) \pi x}{L}} \right] \,{dx} = 
            \frac{1}{2} \left[\frac{\sin{\frac{(n-m) \pi x}{L}}}
                {\frac{(n-m) \pi}{L}} - \frac{\sin{\frac{(n+m) \pi x}{L}}}
            {\frac{(n+m) \pi x}{L}}\right]_{-L}^{L} \\ 
            &= \frac{L}{2 \pi} \left[\frac{1}{n-m} \sin{\frac{(n-m) \pi x}{L} - 
            \frac{1}{n+m} \sin{\frac{(n+m) \pi x}{L}}}\right]_{-L}^{L} \\ 
            &= \frac{L}{2 \pi}
            [\frac{1}{n-m} \sin{(n-m) \pi} - \frac{1}{n+m} \sin{(n+m) \pi}] = 
\end{align*} 
    \end{myitemize}
\end{proof}

\end{document}
