\input{preamble.tex}
\newcommand{\vect}[2]{(#1_1,\ldots, #1_#2)}
%%%%%%% nesting newcommands $$$$$$$$$$$$$$$$$$$
\newcommand{\function}[1]{\newcommand{\nvec}[2]{#1(##1_1,\ldots, ##1_##2)}}

\newcommand{\linode}[2]{#1_n(x)#2^{(n)}+#1_{n-1}(x)#2^{(n-1)}+\cdots +#1_0(x)#2=g(x)}

\newcommand{\vecoffun}[3]{#1_0(#2),\ldots ,#1_#3(#2)}



\zexternaldocument{transformations}

\pagestyle{vangelis}






\begin{document}

\begin{cor}
  Έστω $V$ ένας $ \mathbb{K}- $χώρος πεπερασμένης διάστασης με μια διατεταγμένη 
  βάση $ \beta $, και έστω $ T \colon V \to V $ γραμμικός. Τότε $T$ αντιστρέψιμος 
  αν και μόνον αν $ [T]_{\beta} $ είναι αντιστρέψιμος. Επιπλέον, $ [T^{-1}]_{\beta} = 
  ([T]_{\beta })^{-1}$.
\end{cor}

\begin{cor}
  Έστω $ A \in \textbf{M}_{n}(\mathbb{K})  $. Τότε ο $A$ είναι αντιστρέψιμος αν και 
  μόνον αν ο (αριστερός - πολλάπλασιασμός μετασχηματισμός) $ T_{A} $ είναι 
  αντιστρέψιμος. Επιπλέον, $ (T_{Α})^{-1} = T_{A^{-1}} $.
\end{cor}

\begin{dfn}
  Έστω $V$ και $W$ δύο $ \mathbb{K}- $χώροι. Λέμε ότι ο $V$ είναι ισόμορφος με τον 
  $W$ και το συμβολίζουμε με $ V \cong W $ αν υπάρχει γραμμικός μετασχηματισμός 
  $ T \colon V \to W $ ο οποίος είναι αντιστρέψιμος (δηλαδή $ 1-1 $ και επί). Ένας 
  τέτοιος γραμμικός μετασχηματισμός καλείται ισομορφισμός απο τον $V$ στον $W$.
\end{dfn}

\begin{rem}
  Κάθε $ \mathbb{K}- $χώρος είναι ισόμορφος με τον εαυτό του. (Ο ταυτοτικός 
  μετασχηματισμός $ I_{V} \colon V \to V $ ) είναι ένας ισομορφισμός από τον $V$ επί 
  του $V$.
\end{rem}

\begin{rem}
  Αν ο $ \mathbb{K}- $χωρος $V$ είαι ισόμορφος με τον $ \mathbb{K}- $χώρο $W$, τότε 
  και ο $W$ είναι ισόμορφος με τον $V$ (Αν $ T \colon V \to W $ είναι ένας ισομορφισμός 
  από τον $V$ στον $W$, τότε ο $ T^{-1} \colon W \to V $ είναι ένας ισομορφισμός από 
  τον $W$ στον $V$). 
\end{rem}

\begin{rem}
  Αν ο $ \mathbb{K}- $χώρος $V$ είναι ισόμορφος με τον $ \mathbb{K}- $χώρο $W$ και ο 
  $W$ είναι ισόμορφος με τον $ \mathbb{K}- $χώρο $Z$, τότε ο $V$ είναι ισόμορφος με 
  τον $Z$. (Αν $ T \colon V \to W $ και $ S \colon W\to Z $ είναι ισομορφισμοί, τότε 
  η σύνθεση $ S \circ T \colon V \to Z$ είναι ισομορφισμός από τον $V$ στον $ Z $ ).
\end{rem}

Επομέως από τις παραπάνω παρατηρήσεις η σχέση ισομορφισμού είναι μια σχέση 
ισοδυναμίας επί του συνόλου των $ \mathbb{K}- $χώρων.

\begin{thm}
  Έστω $V$ και $W$ δύο $ \mathbb{K}- $χώροι πεπερασμένης διάστασης. Τότε
\[
  V \quad \text{ισόμορφος με τον $W$} \Leftrightarrow \dim(V) = \dim(W)   
 \] 
\end{thm}

\begin{cor}
  Έστω $V$ ένας $ \mathbb{K}- $χώρος. Τότε ο $V$ είναι ισόμορφος με τον 
  $ \mathbb{K}^{n} $ αν και μόνον αν $ \dim(V) = n $ 
\end{cor}

\begin{rem}
  Αν $ V $ είναι ένας $ \mathbb{K}- $χώρος διάστασης $n$ με μια διατεταγμένη βάση 
  $\beta$, τότε ένας ισομορφισμός $ T \colon V \to \mathbb{K}^{n} $ είναι ο 
  \[
    T(\mathbf{v}) = [\mathbf{v}]_{\beta}, \quad \forall \mathbf{v} \in V
   \] 
   όπου $ [\mathbf{v}]_{\beta} $ είναι το διάνυσμα των συντεταγμένων του $ \mathbf{v} $ 
   ως προς τη βάση $\beta$. 
\end{rem}

\begin{thm}
  Έστω $V$ και $W$ δύο $ \mathbb{K}- $χώροι πεπερασμένων διαστάσεων, έστω $n$ και 
  $ m $ αντίστοιχα, και έστω $\beta$ και $\gamma$ διατεταγμένες βάσεις για τον $V$ και 
  $W$ αντίστοιχα. Τότε η συνάρτηση $ \phi \colon \mathcal{L}(V,W) \to 
  \textbf{M}_{m \times n}(\mathbb{K})  $ η οποία ορίζεται ως 
  $ \phi (T) = [T]_{beta}^{\gamma}, \quad \forall 
  T \in \mathcal{L}(V,W) $ είναι ένας ισομορφισμός.
\end{thm}

\begin{cor}
  Έστω $V$ και $W$ δύο $ \mathbb{K}- $χώροι πεπερασμένων διαστάσεων, με διαστάσεις 
  $ n $ και $ m $ αντίστοιχα. Τότε ο $ \mathbb{K}- $χώρος $ \mathcal{L}(V,W) $ είναι 
  πεπερασμένης διάστασης με διαστάση $ m \cdot n $.
\end{cor}

\end{document}
