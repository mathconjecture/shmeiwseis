\documentclass[a4paper,12pt]{article}
\usepackage{etex}
%%%%%%%%%%%%%%%%%%%%%%%%%%%%%%%%%%%%%%
% Babel language package
\usepackage[english,greek]{babel}
% Inputenc font encoding
\usepackage[utf8]{inputenc}
%%%%%%%%%%%%%%%%%%%%%%%%%%%%%%%%%%%%%%

%%%%% math packages %%%%%%%%%%%%%%%%%%
\usepackage{amsmath}
\usepackage{amssymb}
\usepackage{amsfonts}
\usepackage{amsthm}
\usepackage{proof}

\usepackage{physics}

%%%%%%% symbols packages %%%%%%%%%%%%%%
\usepackage{dsfont}
\usepackage{stmaryrd}
%%%%%%%%%%%%%%%%%%%%%%%%%%%%%%%%%%%%%%%


%%%%%% graphicx %%%%%%%%%%%%%%%%%%%%%%%
\usepackage{graphicx}
\usepackage{color}
%\usepackage{xypic}
\usepackage[all]{xy}
\usepackage{calc}
%%%%%%%%%%%%%%%%%%%%%%%%%%%%%%%%%%%%%%%

\usepackage{enumerate}

\usepackage{fancyhdr}
%%%%% header and footer rule %%%%%%%%%
\setlength{\headheight}{14pt}
\renewcommand{\headrulewidth}{0pt}
\renewcommand{\footrulewidth}{0pt}
\fancypagestyle{plain}{\fancyhf{}
\fancyhead{}
\lfoot{}
\rfoot{\small \thepage}}
\fancypagestyle{vangelis}{\fancyhf{}
\rhead{\small \leftmark}
\lhead{\small }
\lfoot{}
\rfoot{\small \thepage}}
%%%%%%%%%%%%%%%%%%%%%%%%%%%%%%%%%%%%%%%

\usepackage{hyperref}
\usepackage{url}
%%%%%%% hyperref settings %%%%%%%%%%%%
\hypersetup{pdfpagemode=UseOutlines,hidelinks,
bookmarksopen=true,
pdfdisplaydoctitle=true,
pdfstartview=Fit,
unicode=true,
pdfpagelayout=OneColumn,
}
%%%%%%%%%%%%%%%%%%%%%%%%%%%%%%%%%%%%%%



\usepackage{geometry}
\geometry{left=25.63mm,right=25.63mm,top=36.25mm,bottom=36.25mm,footskip=24.16mm,headsep=24.16mm}

%\usepackage[explicit]{titlesec}
%%%%%% titlesec settings %%%%%%%%%%%%%
%\titleformat{\chapter}[block]{\LARGE\sc\bfseries}{\thechapter.}{1ex}{#1}
%\titlespacing*{\chapter}{0cm}{0cm}{36pt}[0ex]
%\titleformat{\section}[block]{\Large\bfseries}{\thesection.}{1ex}{#1}
%\titlespacing*{\section}{0cm}{34.56pt}{17.28pt}[0ex]
%\titleformat{\subsection}[block]{\large\bfseries{\thesubsection.}{1ex}{#1}
%\titlespacing*{\subsection}{0pt}{28.80pt}{14.40pt}[0ex]
%%%%%%%%%%%%%%%%%%%%%%%%%%%%%%%%%%%%%%

%%%%%%%%% My Theorems %%%%%%%%%%%%%%%%%%
\newtheorem{thm}{Θεώρημα}[section]
\newtheorem{cor}[thm]{Πόρισμα}
\newtheorem{lem}[thm]{λήμμα}
\theoremstyle{definition}
\newtheorem{dfn}{Ορισμός}[section]
\newtheorem{dfns}[dfn]{Ορισμοί}
\theoremstyle{remark}
\newtheorem{remark}{Παρατήρηση}[section]
\newtheorem{remarks}[remark]{Παρατηρήσεις}
%%%%%%%%%%%%%%%%%%%%%%%%%%%%%%%%%%%%%%%




\newcommand{\vect}[2]{(#1_1,\ldots, #1_#2)}
%%%%%%% nesting newcommands $$$$$$$$$$$$$$$$$$$
\newcommand{\function}[1]{\newcommand{\nvec}[2]{#1(##1_1,\ldots, ##1_##2)}}

\newcommand{\linode}[2]{#1_n(x)#2^{(n)}+#1_{n-1}(x)#2^{(n-1)}+\cdots +#1_0(x)#2=g(x)}

\newcommand{\vecoffun}[3]{#1_0(#2),\ldots ,#1_#3(#2)}






\begin{document}


\begin{center}
    \fbox{\large\bfseries Ιδιότητες των Οριζουσών}
\end{center}

\vspace{\baselineskip}

\begin{description}

    \item[Ιδιότητα 1.] Αν μια ορίζουσα έχει σε μια γραμμή (ή στήλη) όλα τα στοιχεία της μηδέν, τότε η τιμή της ορίζουσας είναι \textbf{μηδέν}.

    \item[Ιδιότητα 2.] Το πρόσημο της ορίζουσας \textbf{αλλάζει}, αν αλλάξουμε 2 γραμμές (ή στήλες) μεταξύ τους.

        \textit{Παράδειγμα}: 
        \[
            \begin{vmatrix}
                1 & 2 \\
                3 & 4 
            \end{vmatrix}
            = -2, \quad 
            \begin{vmatrix}
                3 & 4 \\
                1 & 2 
            \end{vmatrix}=2
        \]

    \item[Ιδιότητα 3.] Αν πολλαπλασιάσουμε τα στοιχεία μιας γραμμής (ή στήλης) με έναν παράγοντα $k$, τότε η τιμή της ορίζουσας \textbf{πολλαπλασιάζεται} επί αυτόν τον παράγοντα. Δηλαδή μπορούμε να βγάζουμε κοινό παράγοντα (εκτός ορίζουσας) από μια γραμμή (ή στήλη) έναν αριθμό και αυτός ο αριθμός πολλαπλασιάζει την τιμή ολόκληρης της ορίζουσας.

        \textit{Παράδειγμα}:
        \[
            \begin{vmatrix}
                4 & 3 \\
                -1 & 1
                \end{vmatrix}=7 \Rightarrow \begin{vmatrix}
                2\cdot 4 & 2\cdot 3 \\
                1 & 1 
                \end{vmatrix}=2 \begin{vmatrix}
                4 & 3 \\
                -1 & 1 
            \end{vmatrix}2\cdot 7 = 14
        \]


    \item[Ιδιότητα 4.] Αν δυο γραμμες (ή στήλες) μιας ορίζουσας είναι ίδιες ή ανάλογες, τότε η τιμή της ορίζουσας ειναι \textbf{μηδεν}.

        \textit{Παράδειγμα}: 
        \[
            \begin{vmatrix}
                6 & 4\\
                3 & 2 
                \end{vmatrix} = \begin{vmatrix}
                2\cdot 3 & 2\cdot 2\\
                3 & 2 
                \end{vmatrix} = 2\begin{vmatrix}
                3 & 2 \\
                3 & 2 
            \end{vmatrix}=2\cdot 0 = 0
        \]



    \item[Ιδιότητα 5.] Αν τα στοχεία μιας γραμμής (ή στήλης) πολλαπλασιασθούν επί έναν παράγοντα $k$ και κατόπιν προστεθούν στα αντίστοιχα στοιχεία μιας άλλης γραμμής (ή στήλης), τότε η τιμή της ορίζουσας \textbf{δεν αλλάζει}. Η ιδιότητα αυτή χρησιμεύει κυρίως για να ((φτιάχνω)) μηδενικά μέσα στην ορίζουσα.

        \textit{Παραδειγμα}:
        \[
            \begin{vmatrix}
                3 & 2\\
                -1 & 4
            \end{vmatrix}=14
            \xRightarrow{\Gamma_2=2\Gamma_1+\Gamma_2}\begin{vmatrix}
                3 & 2 \\
                -1 + 2\cdot 3 & 4 + 2\cdot 2 
                \end{vmatrix}=\begin{vmatrix}
                3 & 2\\
                5 & 8 
            \end{vmatrix}=14
        \]

    \item[Ιδιότητα 6.] Ισχύουν οι σχέσεις:
        \begin{enumerate}[i)]

            \item \[
                    \begin{vmatrix}
                        a\pm a' & b\pm b'\\
                        c & d 
                    \end{vmatrix}
                    =
                    \begin{vmatrix}
                        a & b \\
                        c & d 
                    \end{vmatrix}
                    \pm
                    \begin{vmatrix}
                        a' & b' \\
                        c & d 
                    \end{vmatrix}
                \]

            \item \begin{align*}
                    \begin{vmatrix}
                        a + a' & b + b' \\
                        c + c' & d + d'
                    \end{vmatrix}
&=
\begin{vmatrix}
    a & b \\
    c + c' & d + d'
\end{vmatrix}
+
\begin{vmatrix}
    a' & b' \\
    c+c' & d+d' 
\end{vmatrix} 
\\[10pt]
&= 
\begin{vmatrix}
    a & b \\
    c & d
\end{vmatrix}
+
\begin{vmatrix}
    a & b \\
    c' & d' 
\end{vmatrix}
+ 
\begin{vmatrix}
    a' &  b' \\
    c & d 
\end{vmatrix}
+
\begin{vmatrix}
    a' & b' \\
    c' & d'
\end{vmatrix}
                \end{align*}

        \end{enumerate}


    \item[Ιδιότητα 7.] Η τιμή της ορίζουσας ενός \textbf{διαγώνιου} ή \textbf{τριγωνικού} πίνακα, είναι ίση με το γινόμενο των στοιχείων της κύριας διαγωνίου.

        \textit{Παράδειγμα}:
        \[
            \begin{vmatrix}
                1 & -34 & 22\\
                0 & -2 & 17 \\
                0 & 0 & 3 
            \end{vmatrix}=1\cdot (-2)\cdot 3= -6
        \]

    \item[Ιδιότητα 8] Ισχύουν οι παρακάτω σχέσεις.

        \twocolumnsides{
            \begin{enumerate}[i)]
                \item $ \abs{\mathbb{O}}=0$
                \item $\abs{\mathbb{I}}=1$
                \item $|A|=|A^T|$
            \end{enumerate}
            }{
            \begin{enumerate}[i)]
                \setcounter{enumi}{3}
            \item $|A\cdot B|=|A|\cdot |B|$
                % \item $|A^n|=|A|^n$
            \item $ |(-A)| = (-1)^{n} |A|, \quad \text{όπου} A_{n \times n} $
            \item $|A^{-1}|=\frac{1}{|A|}$
            \end{enumerate}
        }



\end{description}








\end{document}
