\input{preamble2.tex}
\input{definitions2.tex}
\input{tikz.tex}
\input{myboxes.tex}

\everymath{\displaystyle}
\pagestyle{vangelis}

%adds a rectangle background box behind every picture
 % \tikzset{every picture/.append style={background rectangle/.style={fill=blue!05,rounded
 %   corners},show background rectangle}}


\usepackage[font={color=Col1},labelfont={bf},hypcap=false]{caption}


\geometry{top=2.0cm,left=1.5cm,right=1.5cm}

\input{insbox}


\newcommand{\twocolumnsidell}[2]{\begin{minipage}[c]{0.68\linewidth}
    #1
    \end{minipage}\hfill\begin{minipage}[c]{0.30\linewidth}
    #2
  \end{minipage}
}

\newcommand{\twocolumnsidesrr}[2]{\begin{minipage}[t]{0.35\linewidth}
    #1
    \end{minipage}\hfill\begin{minipage}[t]{0.60\linewidth}
    #2
  \end{minipage}
}

\newcommand{\twocolumnsidesss}[2]{\begin{minipage}[c]{0.48\linewidth}\raggedright
    #1
    \end{minipage}\hfill\begin{minipage}[c]{0.48\linewidth}\raggedright
    #2
  \end{minipage}
}


\tikzset{vangelis/.style = {ultra thick,magenta,dashed}, mystyle/.style={line width=3pt,blue!55,double,rounded corners}}

\pgfplotsset{
  myaxis/.style={axis lines=center,axis line style={thick,blue!50,-stealth},tick label
  style={font=\small,blue!50},xtick=\empty,ytick=\empty,tick style={blue!50}},
  myplot/.style={Col1!75,ultra thick,samples=500,no marks,smooth,},
  dashed lines/.style={dashed,blue!50,ultra thin},
  polax/.style={grid=none},
}

\tikzset{mygraph/.pic={\begin{scope}[scale=0.7]
    \draw [in=-180, out=90, looseness=0.75] (2.3,1) to (3.55,2.75);
    \draw [in=180, out=0] (3.55,2.75) to (4.475,2);
    \draw [in=-90, out=0] (4.475,2) to (4.925,2.5);
    \end{scope}
}}


\begin{document}


\chapter{Εισαγωγικές Έννοιες}


\section{Η Έννοια του Συνόλου}

Με την έννοια \textcolor{Col1}{σύνολο}, αναφερόμαστε σε μια συλλογή αντικειμένων που τα 
θεωρούμε ως μία \textit{ολότητα}. 
Τα αντικείμενα αυτά ονομάζονται \textcolor{Col1}{στοιχεία} του 
συνόλου. Δεχόμαστε την ύπαρξη συνόλου που δεν έχει κανένα στοιχείο, το \textbf{κενό} 
σύνολο και το συμβολίζουμε με $ \emptyset $. Τα σύνολα τα συμβολίζουμε με κεφαλαία
γράμματα, $ A, B, \ldots, X, Y, \ldots $ και τα στοιχεία τους με πεζά γράμματα $ a,b,
\ldots, x,y, \ldots $. Χρησιμοποιούμε συχνά τα σύμβολα $ \forall $:
«\textcolor{Col1}{για κάθε}» και $ \exists $: «\textcolor{Col1}{υπάρχει}», 
καθώς και τις παρακάτω εκφράσεις:
\begin{center}
  \begin{myitemize}
    \item $ \textcolor{Col2}{x \in A} : $ «το στοιχείο $x$ \textcolor{Col1}{ανήκει} στο 
      σύνολο $A$» \\
    \item $ \textcolor{Col2}{x \not\in A} : $ «το στοιχείο $x$ \textcolor{Col1}{δεν 
      ανήκει} στο σύνολο $A$» \\
    \item $\textcolor{Col2}{A \subseteq B} :$ «το σύνολο $A$ είναι
      \textcolor{Col1}{υποσύνολο} του $B$, Δηλαδή, \textbf{όλα} τα στοιχεία του $A$ είναι
      και στοιχεία του $B$»
  \end{myitemize}
\end{center}

\begin{example}
  Έστω το σύνολο $ A = \{ 1,2,3,5,8 \} $. Παρατηρούμε ότι $ 2 \in A $ και $ 3 \in A $, 
  αλλά $ 4 \not \in A $. Έστω επίσης το σύνολο $ B = \{ 1,2,3 \} $. Παρατηρούμε ότι 
  $ B \subseteq A $, γιατί κάθε στοιχείο του $B$ είναι και στοιχείο του $A$.
\end{example}


\section{Πραγματικοί αριθμοί} 

\enlargethispage{3\baselineskip}

Θα χρησιμοποιούμε συχνά τα παρακάτω σύνολα αριθμών.
\begin{myitemize}
  \item Το σύνολο των \textcolor{Col1}{Φυσικών} αριθμών 
    $ \mathbb{N} = \{ 0,1,2,3,\ldots \} $
  \item Το σύνολο των \textcolor{Col1}{Ακεραίων} αριθμών 
    $ \mathbb{Z} = \{\ldots,-3,-2,-1, 0,1,2,3,\ldots \} $
  \item Το σύνολο των \textcolor{Col1}{Ρητών} αριθμών $ \mathbb{Q} = 
    \{ m/n \; : \; m,n \in \mathbb{Z}, \; n \neq 0 \} $
  \item Το σύνολο των \textcolor{Col1}{Άρρητων} αριθμών όπως οι 
    $ \sqrt{2} = 1,414\ldots, \sqrt{3} = 1, 732\ldots, \pi = 3,1415\ldots, \mathrm{e} = 
    2,718\ldots $  
  \item Το σύνολο των \textbf{Ρητών} και \textbf{Άρρητων} ονομάζεται σύνολο των
    \textcolor{Col1}{Πραγματικών} αριθμών και συμβολίζεται με $ \mathbb{R} $.
\end{myitemize}
\begin{rem}
  Ισχύει ότι $ \mathbb{N} \subseteq \mathbb{Z} \subseteq\mathbb{Q}\subseteq\mathbb{R} $. 
\end{rem}
\begin{rem}
  Όταν τα παραπάνω σύνολα, έχουν έναν \textbf{αστερίσκο}, τότε στα σύνολα αυτά
  \textbf{δεν} ανήκει το 0. Για παράδειγμα, 
  $ \mathbb{Z}^{*} = \mathbb{Z}- \{ 0 \} $.
\end{rem}


\twocolumnsidess{
  \subsection{Ιδιότητες Γινομένου} 
  \begin{myitemize}
    \item $ a\cdot b = 0 \Leftrightarrow a = 0 \quad \text{ή} \quad b=0, 
      \quad \forall a,b \in \mathbb{R} $ 
    \item $ a\cdot b \neq 0 \Leftrightarrow a \neq 0 \quad \text{ή} \quad b \neq 0,
      \quad \forall a,b \in \mathbb{R} $
  \end{myitemize}
  \subsection{Πράξεις με Κλάσματα} 
  \begin{myitemize}
    \item $ \frac{a}{c} \pm \frac{b}{c} = \frac{a \pm b}{c} \quad \text{ή} \quad  
      \frac{a}{b} \pm \frac{c}{d} = \frac{a\cdot d \pm b\cdot c}{b \cdot d} $
    \item $ \frac{a}{b} \cdot \frac{c}{d} = \frac{a\cdot c}{b\cdot d} \quad \text{και}
      \quad \frac{a}{b} : \frac{c}{d} = \frac{a}{b} \cdot \frac{d}{c} = \frac{a\cdot d}{b
      \cdot c} $
    \item $ \dfrac{\dfrac{a}{b}}{\dfrac{c}{d}} = \frac{a\cdot d}{b\cdot c} \; \text{και}
      \; \frac{\dfrac{a}{b}}{c} = \frac{\dfrac{a}{b}}{\dfrac{c}{1}} = \frac{a}{b\cdot c}
      \; \text{ή} \; \frac{a}{\dfrac{b}{c}} = \frac{\dfrac{a}{1}}{\dfrac{b}{c}} =
      \frac{a\cdot c}{b} $ 
  \end{myitemize}
}{
  \subsection{Ιδιότητες διάταξης} 
  \begin{myitemize}
    \item Αν $ a \leq b $ και $ b \leq c $ τότε $ a \leq c $ 
    \item Αν $ a \leq b $ και $ c \leq d $ τότε $ a + c \leq b + d $
    \item $ a \leq b \Leftrightarrow a \cdot c \leq b \cdot c $, όταν $ c>0 $ 
    \item $ a \leq b \Leftrightarrow a \cdot c \geq b \cdot c $, όταν $ c<0 $ 
    \item Αν $ 0 < a \leq b $ και $ 0 < c \leq d $ τότε $ a \cdot c \leq b \cdot d $
    \item Αν $ a, b $ \textbf{ομόσημοι}, τότε $ a \leq b \Leftrightarrow \frac{1}{a} \geq
      \frac{1}{b} $
    \item Αν $ a,b $ \textbf{θετικοί} και $ n \in \mathbb{N}^{*} $ τότε 
      $ a<b \Leftrightarrow   a^{n} < b^{n} $ 
    \item Αν $ 0<a \leq b $ και $ n \in \mathbb{N} $ τότε $ \frac{1}{a^{n}} \geq
      \frac{1}{b^{n}} $
\end{myitemize}}


\section{Απόλυτη Τιμή}

\twocolumnsidesrr{
  \subsection{Ορισμός} 
  \begin{empheq}[box=\mathboxg]{equation*}
    \abs{a} = 
    \begin{cases}
      a, & \text{όταν } a \geq 0 \\
      -a, & \text{όταν } a<0 
    \end{cases} 
  \end{empheq}
  \begin{myitemize}
    \item Ισχύει $ \abs{a} \geq 0, \quad \forall a \in \mathbb{R} $
    \item Ισχύει $ - \abs{a} \leq a \leq \abs{a}, \quad \forall a \in \mathbb{R} $
  \end{myitemize}
}{
  \subsection{Ιδιότητες απόλυτης τιμής} 
  \twocolumnsidesrr{
    \begin{myitemize}
      \item $ \sqrt{a^{2}} = \abs{a} $
      \item $ \abs{a\cdot b} = \abs{a} \cdot \abs{b} $
      \item $ \abs{\frac{a}{b}} = \frac{\abs{a}}{\abs{b}} $
  \end{myitemize}}{
  \begin{myitemize}
    \item $ \abs{\abs{a} - \abs{b}} \leq \abs{a \pm b} \leq \abs{a} + \abs{b} $
    \item $ \abs{x} < a \Leftrightarrow -a < x < a $
    \item $ \abs{x} > a \Leftrightarrow x > a \; \text{και} \; x < -a $
  \end{myitemize}
}}
\begin{example}
  Ισχύει ότι $ \abs{3} = 3 $ και $ \abs{-3} = 3 $. Δηλαδή η απόλυτη τιμή «τρώει» το 
  πρόσημο.
\end{example}
\begin{example}
  $ \abs{x-3} = 
  \begin{cases}
    x-3, & x \geq 3 \\
    -(x-3), & x < 3
  \end{cases} = 
  \begin{cases}
    x-3, & x \geq 3 \\
    3-x, & x <3
  \end{cases}$
\end{example}
\begin{rem}
  Η απόλυτη τιμή $ \abs{a} $, παριστάνει την \textbf{απόσταση} 
  του σημείου με συντεταγμένη $ a $ από την αρχή του άξονα. 
  Η \textbf{απόσταση} δύο σημείων πάνω στον πραγματικό άξονα, 
  με συντεταγμένες $ a $ και $ b $ αντίστοιχα, είναι $d = \abs{b-a}$ 
\end{rem}
\begin{center}
  \begin{tikzpicture}
    \draw[-latex] (-0.6,0) -- (0,0) coordinate (o) to coordinate[pos=0.3] (a) 
      coordinate[pos=0.8] (b) (5,0) ;
    \draw[Col1,ultra thick] (a) -- (b) ;
    \fill (o) node[above] {$0$} circle (2pt) ;  
    \fill[Col1] (a) node[above] {$a$} circle (2pt) ;  
    \fill[Col1] (b) node[above] {$b$} circle (2pt) ;  
    \coordinate (o1) at ($(o)-(0,0.4)$) ;
    \coordinate (a1) at ($(a)-(0,0.4)$) ;
    \coordinate (o2) at ($(o)-(0,0.8)$) ;
    \coordinate (b2) at ($(b)-(0,0.8)$) ;
    \coordinate (a3) at ($(a)-(0,0.4)$) ;
    \coordinate (b3) at ($(b)-(0,0.4)$) ;
    \draw[{Stealth}-{Stealth}] (o1) node {$|$} --
      node[fill=white]{\smaller$\abs{a}$} (a1) node {$|$} ;
    \draw[{Stealth}-{Stealth}] (o2) node {$|$} --
      node[fill=white]{\smaller$\abs{b}$} (b2) node {$|$} ;
    \draw[{Stealth}-{Stealth},Col1] (a3) node {$|$} --
      node[fill=white]{\smaller$\abs{b-a}$} (b3) node {$|$} ;
  \end{tikzpicture}
  \hspace{6\baselineskip}
  \begin{tikzpicture}
    \draw[-latex] (-2,0) -- (0,0) coordinate (o) to coordinate[pos=-0.4] (a)
      coordinate[pos=0.7] (b) (3.5,0) ;
    \draw[Col1,ultra thick] (a) -- (b) ;
    \fill (o) node[above] {$0$} circle (2pt) ;  
    \fill[Col1] (a) node[above] {$a$} circle (2pt) ;  
    \fill[Col1] (b) node[above] {$b$} circle (2pt) ;  
    \coordinate (o1) at ($(o)-(0,0.4)$) ;
    \coordinate (a1) at ($(a)-(0,0.4)$) ;
    \coordinate (o2) at ($(o)-(0,0.4)$) ;
    \coordinate (b2) at ($(b)-(0,0.4)$) ;
    \coordinate (a3) at ($(a)-(0,0.8)$) ;
    \coordinate (b3) at ($(b)-(0,0.8)$) ;
    \draw[{Stealth}-{Stealth}] (o1) node {$|$} --
      node[fill=white]{\smaller$\abs{a}$} (a1) node {$|$} ;
    \draw[{Stealth}-{Stealth}] (o2) node {$|$} --
      node[fill=white]{\smaller$\abs{b}$} (b2) node {$|$} ;
    \draw[{Stealth}-{Stealth},Col1] (a3) node {$|$} --
      node[fill=white]{\smaller$\abs{b-a}$} (b3) node {$|$} ;
  \end{tikzpicture}
\end{center}


\section{Διαστήματα}

Αν $ a,b \in \mathbb{R} $ με $ a<b $, τότε ονομάζουμε \textcolor{Col1}{διαστήματα} του 
$ \mathbb{R} $ κάθε ένα από τα παρακάτω σύνολα.
\begin{myitemize}
  \item $ (a,b) = \{ x \in \mathbb{R} \; : \; a<x<b \} $ \textbf{ανοιχτό} διάστημα 
    \hfill
    \begin{tikzpicture}
      \draw[-latex] (0,0) to coordinate[pos=0.2] (a) coordinate[pos=0.8] (b) (4,0) ;
      \draw[ultra thick,Col1] (a) -- (b) ;
      \draw[Col1,fill=white] (a) node[above] {$a$} circle (2pt) ;  
      \draw[Col1,fill=white] (b) node[above] {$b$} circle (2pt) ;  
    \end{tikzpicture}
  \item $ [a,b] = \{ x \in \mathbb{R} \; : \; a \leq x \leq b \} $ \textbf{κλειστό} 
    διάστημα
    \hfill
    \begin{tikzpicture}
      \draw[-latex] (0,0) to coordinate[pos=0.2] (a) coordinate[pos=0.8] (b) (4,0) ;
      \draw[ultra thick,Col1] (a) -- (b) ;
      \fill[Col1] (a) node[above] {$a$} circle (2pt) ;  
      \fill[Col1] (b) node[above] {$b$} circle (2pt) ;  
    \end{tikzpicture}
  \item $ [a,b) = \{ x \in \mathbb{R} \; : \; a \leq x < b \} $ \textbf{κλειστό-ανοιχτό} 
    διάστημα
    \hfill
    \begin{tikzpicture}
      \draw[-latex] (0,0) to coordinate[pos=0.2] (a) coordinate[pos=0.8] (b) (4,0) ;
      \draw[ultra thick,Col1] (a) -- (b) ;
      \fill[Col1] (a) node[above] {$a$} circle (2pt) ;  
      \draw[Col1,fill=white] (b) node[above] {$b$} circle (2pt) ;  
    \end{tikzpicture}
  \item $ (a,b] = \{ x \in \mathbb{R} \; : \; a < x \leq b \} $ \textbf{ανοιχτό-κλειστό} 
    διάστημα
    \hfill
    \begin{tikzpicture}
      \draw[-latex] (0,0) to coordinate[pos=0.2] (a) coordinate[pos=0.8] (b) (4,0) ;
      \draw[ultra thick,Col1] (a) -- (b) ;
      \draw[Col1,fill=white] (a) node[above] {$a$} circle (2pt) ;  
      \fill[Col1] (b) node[above] {$b$} circle (2pt) ;  
    \end{tikzpicture}
\end{myitemize}

Αν $ a \in \mathbb{R} $ τότε ονομάζουμε \textcolor{Col1}{μη φραγμένα} διαστήματα του 
$ \mathbb{R} $ με άκρο $a$ κάθε ένα από τα παρακάτω σύνολα.
\begin{myitemize}
  \item $ (- \infty, a) = \{ x \in \mathbb{R} \; : \; x<a \} $
    \hfill
    \begin{tikzpicture}
      \draw[-latex] (0,0) coordinate (o) to coordinate[pos=0.2] (a) 
        coordinate[pos=0.8] (b) (4,0) coordinate (p) ;
      \draw[ultra thick,Col1] (o) -- (b) ;
      \draw[Col1,fill=white] (b) node[above] {$a$} circle (2pt) ;  
    \end{tikzpicture}
  \item $ (- \infty, a] = \{ x \in \mathbb{R} \; : \; x \leq a \} $
    \hfill
    \begin{tikzpicture}
      \draw[-latex] (0,0) coordinate (o) to coordinate[pos=0.2] (a) 
        coordinate[pos=0.8] (b) (4,0) coordinate (p) ;
      \draw[ultra thick,Col1] (o) -- (b) ;
      \fill[Col1] (b) node[above] {$a$} circle (2pt) ;  
    \end{tikzpicture}
  \item $ (a, +\infty) = \{ x \in \mathbb{R} \; : \; x>a \} $
    \hfill
    \begin{tikzpicture}
      \draw (0,0) coordinate (o) to coordinate[pos=0.2] (a) coordinate[pos=0.8] (b)
        (4,0) coordinate (p) ;
      \draw[ultra thick,Col1] (a) -- (p) ;
      \draw[Col1,fill=white] (a) node[above] {$a$} circle (2pt) ;  
    \end{tikzpicture}
  \item $ [a, +\infty) = \{ x \in \mathbb{R} \; : \; x \geq a \} $
    \hfill
    \begin{tikzpicture}
      \draw (0,0) coordinate (o) to coordinate[pos=0.2] (a) coordinate[pos=0.8] (b)
        (4,0) coordinate (p) ;
      \draw[ultra thick,Col1] (a) -- (p) ;
      \fill[Col1] (a) node[above] {$a$} circle (2pt) ;  
    \end{tikzpicture}
\end{myitemize}
Ορίζουμε ακόμη $ (- \infty, + \infty) = \mathbb{R} $


\twocolumnsidess{
  \section{Χρήσιμες Ταυτότητες}
  \begin{myitemize}
    \item $ (a\pm b)^{2} = a^{2} \pm 2ab + b^{2} $ 
    \item $ (a\pm b)^{3} = a^{3} \pm 3a^{2}b + 3ab^{2} \pm b^{2} $ 
    \item $ a^{2}-b^{2} = (a-b)(a+b) $
    \item $ a^{3}-b^{3} = (a-b)(a^{2}+ab+b^{2}) $ 
    \item $ a^{3}+b^{3} = (a+b)(a^{2}-ab+b^{2}) $ 
    \item $ (a+b+c)^{2}=a^{2}+b^{2}+c^{2}+2ab+2bc+2ca $
  \end{myitemize}
}{
  \section{Παραγοντοποίηση Τριωνύμου}
  Έστω \textbf{τριώνυμο} $ ax^{2}+bx+c $, με $ \Delta = b^{2}-4ac $ και 
  \textbf{ρίζες}
  \begin{empheq}[box=\mathboxg]{equation*}
    x_{1,2} = \frac{-b \pm \sqrt{\Delta}}{2a}, \; \text{αν } \Delta >0 \;\;
    \text{και} \;\; x_{1} = \frac{-b}{2a}, \; \text{αν } \Delta =0 
  \end{empheq}
  \begin{myitemize}
    \item $ \Delta > 0 \Rightarrow ax^{2}+bx+c = a(x- x_{1})(x- x_{2}) $
    \item $ \Delta = 0 \Rightarrow ax^{2}+bx+c = a(x- x_{1})^{2} $
    \item $ \Delta < 0 \Rightarrow ax^{2}+bx+c \; $ δεν παραγοντοποιείται
  \end{myitemize}
}


\section{Η Έννοια της Πραγματικής Συνάρτησης}

\twocolumnsidell{
  \begin{dfn}
    Μια \textcolor{Col1}{συνάρτηση} με \textbf{πεδίο ορισμού} το σύνολο 
    $A \subseteq \mathbb{R}$, συμβολίζεται $ f \colon A \to \mathbb{R} $ και είναι μια 
    \textit{διαδικασία}, με την οποία σε \textbf{κάθε} στοιχείο $ x \in A $ 
    αντιστοιχίζουμε ένα \textbf{μοναδικό} στοιχείο $ y \in \mathbb{R} $. Το $ y $ 
    ονομάζεται τιμή της $f$ στο $x$ και συμβολίζεται $ y= f(x) $.
  \end{dfn}
  Το σύνολο όλων των τιμών της $f$, για τις διάφορες τιμές του $x$ στο πεδίο ορισμού, 
  ονομάζεται \textcolor{Col1}{σύνολο τιμών} της συνάρτησης.
}{
  \begin{tikzpicture}[scale=0.7]
    \node (0) at (0.8, 0.8) {};
    \node (1) at (3, 2.8) {};
    \node (2) at (4, 2) {};
    \coordinate (o) at (0, 0) {};
    \coordinate (y) at (0, 3.8) ;
    \coordinate (x) at (4.7, 0) ;
    \fill[Col1] (0) circle (1.5pt) ;
    \fill[Col1] (2) circle (1.5pt) ;
    \draw[-latex,axis] (o) -- (x) node[below] {$x$} ;
    \draw[-latex,axis] (o) -- (y) node[left] {$y$} ;
    \draw [in=180, out=45, looseness=0.75,graph] (0.center) to
      node[pos=0.9,above right=2pt]%[right=5pt] 
      {\small$f(x)$} (1.center);
    \draw [bend left=45,graph] (1.center) to (2.center);
    \draw[dashed lines] (o|-0) -- (0) -- (0|-o)  ;
    \draw[dashed lines] (2) -- (2|-o)  ;
    \draw[dashed lines] (1) -- (o|-1)  ;
    \draw[very thick, Col2] (o|-0) -- (o|-1) ;
    \draw[ultra thick, Col2] (o|-0) -- (o|-1) ;
    \draw[ultra thick, Col1] (o-|0) -- (o-|2) ;
    \draw[decorate,decoration={brace,amplitude=5pt,raise=5pt}] (o|-0) -- (o|-1) 
      node[midway,align=center,xshift=-1cm,Col2] {σύνολο \\ τιμών} ;
    \draw[decorate,decoration={brace,amplitude=5pt,raise=5pt,mirror}] (o-|0) -- (o-|2) 
      node[midway,yshift=-0.6cm,Col1] {πεδίο ορισμού} ;
  \end{tikzpicture}
}


\subsection{Γραφική Παράσταση Συνάρτησης}


\twocolumnsidell{
  Έστω  μια συνάρτηση $ f(x) $ με πεδίο ορισμού $A$. Η 
  \textcolor{Col1}{γραφική παράσταση} της $f$, είναι το σύνολο των σημείων 
  $(x, f(x))$ του καρτεσιανού επιπέδου, για τις διάφορες τιμές του $ x $ στο πεδίο 
  ορισμού της. Δηλαδή
  \begin{empheq}[box=\mathboxg]{equation*}
    C_{f} = \{ (x,f(x)) \; : \; x \in A \} 
  \end{empheq}
}{
  \begin{tikzpicture}[scale=0.7]
    \node (0) at (0.8, 1) {};
    \node (1) at (3, 3) {};
    \node (2) at (4, 2) {};
    \coordinate (o) at (0, 0) {};
    \coordinate (y) at (0, 3.8) ;
    \coordinate (x) at (4.7, 0) ;
    \draw[-latex,axis] (o) -- (x) node[below] {$x$} ;
    \draw[-latex,axis] (o) -- (y) node[left] {\smash{$y$}} ;
    \draw [in=180, out=45, looseness=0.75,graph] (0.center) to coordinate[pos=0.5] (p) 
      node[above=3pt,pos=0.8,xshift=15pt]{$C_{f}$} (1.center) ;
    \draw [bend left=45,graph] (1.center) to (2.center);
    \draw[dashed lines] (o|-p) node[left] {\phantom{σύνο}$f(x)$} -- (p) -- (p|-o) 
      node[below] {$x$} ;
    \draw[decorate,decoration={brace,amplitude=5pt,raise=5pt}] (p) -- (p|-o) 
      node[midway,right=10pt] {$f(x)$} ;
    \node at (p) [pin={[pin edge={black,latex-},Col1,xshift=10pt]93:
      {\small$\bigl(x,f(x)\bigr)$}}]{} ;
    \fill[Col1] (p) circle (2.0pt) ;
  \end{tikzpicture}
}


\subsection{Άύξουσες - Φθίνουσες Συναρτήσεις}

Αν η γραφική παράσταση μιας συνάρτησης \textbf{ανηφορίζει} καθώς την παρατηρούμε από τα 
αριστερά προς τα δεξιά, τότε λέμε ότι η συνάρτηση είναι αύξουσα, ενώ αν 
\textbf{κατηφορίζει}, τότε λέμε ότι είναι φθίνουσα. Πιο συγκεκριμένα
\begin{dfn}
  Έστω $f$ συνάρτηση ορισμένη σε ένα διάστημα $\Delta$ και έστω $ x_{1}, x_{2} $ 
  τυχαία στοιχεία του $\Delta$. Η $f$ είναι
  \begin{myitemize}
    \item \textcolor{Col1}{γνησίως αύξουσα} στο 
      $\Delta \Leftrightarrow f(x_{1}) < f(x_{2}) \; \text{για κάθε } x_{1}, x_{2} 
      \in \Delta \; \text{με } x_{1} < x_{2} $ 
    \item \textcolor{Col1}{γνησίως φθίνουσα} στο 
      $\Delta \Leftrightarrow f(x_{1}) > f(x_{2}) \; \text{για κάθε } x_{1}, x_{2} 
      \in \Delta \; \text{με } x_{1} < x_{2} $ 
  \end{myitemize}
\end{dfn}

\begin{center}
  \begin{tikzpicture}[scale=0.7]
    \coordinate (o) at (0, 0) {};
    \coordinate (y) at (0, 4.0) ;
    \coordinate (x) at (4.5, 0) ;
    \draw[-latex,axis] (o) -- (x) node[below] {$x$} ;
    \draw[-latex,axis] (o) -- (y) node[left] {$y$} ;
    \draw[graph] (0.5,0.5) to[out=0, in=-90] node[pos=0.6,left=5pt] {$f(x)$} 
      coordinate[pos=0.4] (p) coordinate[pos=0.8] (q) (3.7,3.7) ;
    \draw[dashed lines] (o|-p) node[left] {$f(x_{1})$} -- (p) -- (p|-o) node[below]
      {$x_{1}$} ;
    \draw[dashed lines] (o|-q) node[left] {$f(x_{2})$} -- (q) -- (q|-o) node[below]
      {$x_{2}$} ;
    \node at (3,2) [right=15pt,align=center] {γνησίως\\αύξουσα} ;
  \end{tikzpicture}
  \hspace{6\baselineskip}
  \begin{tikzpicture}[scale=0.7]
    \coordinate (o) at (0, 0) {};
    \coordinate (y) at (0, 4.0) ;
    \coordinate (x) at (4.5, 0) ;
    \draw[-latex,axis] (o) -- (x) node[below] {$x$} ;
    \draw[-latex,axis] (o) -- (y) node[left] {$y$} ;
    \draw[graph] (0.5,3.5) to[out=-90, in=180] node[pos=0.4,right=5pt] {$f(x)$} 
      coordinate[pos=0.35] (p) coordinate[pos=0.8] (q) (3.7,0.5) ;
    \draw[dashed lines] (o|-p) node[left] {$f(x_{1})$} -- (p) -- (p|-o) node[below]
      {$x_{1}$} ;
    \draw[dashed lines] (o|-q) node[left] {$f(x_{2})$} -- (q) -- (q|-o) node[below]
      {$x_{2}$} ;
    \node at (3,2) [right=5pt,align=center] {γνησίως\\φθίνουσα} ;
  \end{tikzpicture}
\end{center}


\subsection{Άρτιες - Περιττές Συναρτήσεις}

\begin{dfn}
  Μια συνάρτηση $f$ λέγεται \textcolor{Col1}{άρτια}, αν 
  $f(-x) = f(x), \; \text{για κάθε } x \; \text{στο πεδίο ορισμού της}$ 
\end{dfn}
\begin{dfn}
  Μια συνάρτηση $f$ λέγεται \textcolor{Col1}{περιττή}, αν 
$f(-x) = -f(x), \; \text{για κάθε } x \; \text{στο πεδίο ορισμού της}$ \end{dfn}

\begin{rem}
  Η γραφική παράσταση μιας άρτιας συνάρτησης, είναι \textbf{συμμετρική ως προς τον άξονα
  $y$}, ενώ, η γραφική παράσταση μιας περιττής συνάρτησης είναι \textbf{συμμετρική ως 
  προς την αρχή των αξόνων}.
\end{rem}

\begin{examples}
Οι συναρτήσεις $ y=x^{2} $ και $ y= \cos{x} $ είναι άρτιες, ενώ
οι συναρτήσεις $ y=x^{3} $ και  $ y= \sin{x} $ είναι περιττές.
\end{examples}
\enlargethispage{2\baselineskip}

\begin{center}
  \begin{tikzpicture}[scale=0.8]
    \coordinate (o) at (0,0) ;
    \node at (o) [below left,blue!75] {$0$} ;
    \coordinate (a) at (-1.1,1.1*1.1) ;
    \coordinate (b) at (1.1,1.1*1.1) ;
    \draw[domain=-1.4:1.4,smooth,variable=\x,Col1,graph] plot ({\x},{\x*\x})
      node[right]{$y=x^{2}$};
    \draw[axis,-stealth] (-2,0) -- (2,0) node[below] {$x$};
    \draw[axis,-stealth] (0,-2) -- (0,2.5) node[left] {$y$};
    \draw[dashed] (a) -- (a|-o) node[below] {$-x$} ;
    \draw[dashed] (b) -- (b|-o) node[below] {$x$} ;
    \draw[dashed] (a) -- (b) ;
    \node [above left] at (o|-a) {$f(x)$} ;
    \node at (1.5,1.0) [right] {άρτια} ;
  \end{tikzpicture} 
  \hfill
  \begin{tikzpicture}
    \begin{axis}[scale=0.6,
      axis lines=none,
      ymin=-2,ymax=2,
      xmin=-7,xmax=7,
      domain=-2*pi:2*pi,
      xtick=\empty,
      ytick=\empty,
      ]
      \addplot[myplot] {cos(deg(x))} node[right=3pt,pos=0.55]{$\cos x$} 
        node[above=20pt,pos=0.90,black]{άρτια} ; 
      \coordinate (o) at (0,0) ;
    \end{axis}
    \node[below left,blue!75] at (o) {$0$} ;
    \draw[axis,-stealth]  (o)+(-2,0) -- +(2,0) node[below] {$x$};
    \draw[axis,-stealth]  (o)+(0,-1.8) -- +(0,1.8) node[left] {$y$};
  \end{tikzpicture}
  \hfill
  \begin{tikzpicture}[scale=0.8]
    \coordinate (o) at (0,0) ;
    \node at (o) [below left,blue!75] {$0$} ;
    \coordinate (a) at (-1.1,-1.1*1.1*1.1) ;
    \coordinate (b) at (1.1,1.1*1.1*1.1) ;
    \draw[domain=-1.25:1.25,smooth,variable=\x,Col1,graph] 
      plot ({\x},{\x*\x*\x}) node[right]{$y=x^{3}$};
    \draw[axis,-stealth] (-2,0) -- (2,0) node[below] {$x$};
    \draw[axis,-stealth] (0,-2) -- (0,2.5) node[left] {$y$};
    \draw[dashed] (a) -- (a|-o) node[above,xshift=-5pt] {$-x$} ;
    \draw[dashed] (a) -- (a-|o) node[right] {$-f(x)$} ;
    \draw[dashed] (b) -- (b|-o) node[below] {$x$} ;
    \draw[dashed] (b) -- (o|-b) node[left] {$f(x)$} ;
    \node at (1.5,1.0) [right] {περιττή} ;
  \end{tikzpicture}
  \hfill
  \begin{tikzpicture}
    \begin{axis}[scale=0.6,
      axis lines=none,
      ymin=-2,ymax=2,
      xmin=-7,xmax=7,
      domain=-2*pi:2*pi,
      xtick=\empty,
      ytick=\empty,
      ]
      \addplot[myplot] {sin(deg(x))} node[above right,pos=0.62]{$\sin x$} 
        node[above=25pt,pos=0.90,black]{περιττή} ; 
      \coordinate (o) at (0,0) ;
    \end{axis}
    \node[below left,xshift=-3pt,blue!75] at (o) {$0$} ;
    \draw[axis,-stealth]  (o)+(-2,0) -- +(2,0) node[below] {$x$};
    \draw[axis,-stealth]  (o)+(0,-1.8) -- +(0,1.8) node[left] {$y$};
  \end{tikzpicture}
\end{center}


% \subsection{Πράξεις μεταξύ Συναρτήσεων}

% Αν $ f \colon A \to \mathbb{R} $ και $ g \colon B \to \mathbb{R} $ είναι συναρτήσεις 
% και $ A \cap B $ είναι το σύνολο των \textbf{κοινών} στοιχείων των $A$ και $B$, τότε
% \begin{myitemize}
%   \item $ (f+g)(x) = f(x) + g(x) $ με πεδίο ορισμού το σύνολο $ A \cap B $
%   \item $ (f-g)(x) = f(x) - g(x) $ με πεδίο ορισμού το σύνολο $ A \cap B $
%   \item $ (f\cdot g)(x) = f(x) \cdot  g(x) $ με πεδίο ορισμού το σύνολο$ A \cap B $
%   \item $ \Bigl(\frac{f}{g}\Bigr)(x) = \frac{f(x)}{g(x)} $ με πεδίο ορισμού το σύνολο
%     $ \{ x \in  A \cap B \; : \; g(x) \neq 0 \} $ 
% \end{myitemize}

\subsection{Σύνθεση Συναρτήσεων}

\begin{dfn}
  Αν $f$ και $g$ συναρτήσεις, τότε η \textcolor{Col1}{σύνθεση της $f$ με τη $g$},
  συμβολίζεται με $ g \circ f $ και ορίζεται ως 
  \begin{empheq}[box=\mathboxr]{equation*}
    (g \circ f) (x) = g(f(x))
  \end{empheq}
  Το πεδίο ορισμού της $ g \circ f $ είναι το σύνολο όλων των $x$ που ανήκουν στο 
  πεδίο ορισμού της $f$ ώστε το $ f(x) $ να ανήκει στο πεδίο ορισμού της $g$. 
  Δηλαδή, η συνάρτηση $ g \circ f $ ορίζεται όταν ορίζονται, τόσο το $ f(x) $ όσο 
  και το $ g(f(x)) $.
\end{dfn}

\begin{example}
Αν $ f(x) = \sin{x} $ και $ g(x) = 5x^{2}-3x $ τότε ορίζεται η σύνθεση της $f$ με τη 
$g$ για κάθε $ x \in \mathbb{R} $, γιατί
\[
  -1 \leq \sin{x} \leq 1, \; \forall x \in \mathbb{R}
\]
που σημαίνει ότι όλες οι τιμές της $ f(x)$, ανήκουν στο πεδίο ορισμού της $g$ που 
είναι το $ \mathbb{R} $, επομένως 
\[
  (gof)(x) = g(f(x)) = g(\sin{x}) = 5(\sin{x} )^{2}-3(\sin{x}) = 5 \sin^{2}{x} - 3
  \sin{x}
\]
\end{example}

\vspace{\baselineskip}

\twocolumnsidesss{
\subsection{Κάθετη και Οριζόντια μετατόπιση}
  Έστω $ c>0 $. Τότε η γραφική παράσταση της 
  \begin{myitemize}
    \item $ y= f(x) + c $ είναι η γραφ.\ παράσταση της $ y=f(x) $ 
      \textbf{μετατοπισμένη κατά \boldsymbol{$c$} προς τα \textcolor{Col1}{πάνω}.}
    \item $ y= f(x) - c $ είναι η γραφ.\ παράσταση της $ y=f(x) $ 
      \textbf{μετατοπισμένη κατά \boldsymbol{$c$} προς τα \textcolor{Col1}{κάτω}.}
    \item $ y= f(x-c) $ είναι η γραφ.\ παράσταση της $ y=f(x) $ 
      \textbf{μετατοπισμένη κατά \boldsymbol{$c$} προς τα \textcolor{Col1}{δεξιά}.}
    \item $ y= f(x+c) $ είναι η γραφ.\ παράσταση της $ y=f(x) $ 
      \textbf{μετατοπισμένη κατά \boldsymbol{$c$} προς τα \textcolor{Col1}{αριστερά}.}
  \end{myitemize}
}{
  \begin{tikzpicture}
    \begin{scope}[scale=0.7]
      \coordinate  (o) at (-1, 0) node[below left] {$0$};
      \draw[-latex,axis] (-2.5,0) -- (o) -- (8.5,0) node[below] (x) {$x$} ;
      \draw[-latex,axis] (-1,-2) -- (o) -- (-1,5.5) node[left] (y) {$y$} ;
      \path [in=-180, out=90, looseness=0.75] (2.3,1) to coordinate[pos=0.2] (p) 
        (3.55,2.75) node[above,Col1] {$y=f(x)$} ;
      \draw [name path=cent,Col1, ultra thick] (0,0) pic {mygraph} ;
      \draw [name path=up,Col2, thick] (0,3) pic {mygraph} ;
      \draw [name path=down,black!50, thick] (0,-3) pic {mygraph} ;
      \draw [name path=right,black, thick] (3,0) pic {mygraph} ;
      \draw [name path=left,blue!75, thick] (-3,0) pic {mygraph} ;
      \fill (p) circle (1.5pt) ;
      \draw[dashed] (p) -- node[pos=0.5,below] {$c$} +(-3,0) coordinate (pl) ;
      \draw[dashed] (p) -- node[pos=0.5,below] {$c$} +(3,0) coordinate (pr) ;
      \draw[dashed] (p) -- node[pos=0.5,left] {$c$} +(0,3) coordinate (pu) ;
      \draw[dashed] (p) -- node[pos=0.6,left] {$c$} +(0,-3) coordinate (pd) ;
      \fill (pu) circle (1.5pt) ;
      \fill (pd) circle (1.5pt) ;
      \fill (pl) circle (1.5pt) ;
      \fill (pr) circle (1.5pt) ;
      \node at (pl) [above=1.0cm,blue!75,xshift=0.8cm] {$y=f(x+c)$} ;
      \node at (pr) [above=1.0cm,xshift=1.0cm] {$y=f(x-c)$} ;
      \node at (pu) [Col2,yshift=20pt,xshift=3cm] {$y=f(x)+c$} ;
      \node at (pd) [right=0.4,black!75] {$y=f(x)-c$} ;
    \end{scope}
  \end{tikzpicture}
}

\subsection{Χρήσιμοι Μετασχηματισμοί}

\twocolumnsidesss{
  Έστω $ c>1 $. Τότε η γραφική παράσταση της 
  \begin{myitemize}
    \item $ y= cf(x) $ είναι η γραφ.\ παράσταση της $ y=f(x) $ 
      \textbf{τεντωμένη κάθετα} κατά $c$.
    \item $ y= \frac{1}{c}f(x) $ είναι η γραφ.\ παράσταση της $ y=f(x) $
      \textbf{συμπιεσμένη κάθετα} κατά $c$.
    \item $ y= f(cx) $ είναι η γραφ.\ παράσταση της $ y=f(x) $
      \textbf{συμπιεσμένη οριζόντια} κατά $c$.
    \item $ y=  f\Bigl(\frac{1}{c}x\Bigr) $ είναι η γραφ.\ παράσταση της $ y=f(x) $
      \textbf{τεντωμένη οριζόντια} κατά $c$.
    \item $ y=  -f(x) $ είναι η γραφ.\ παράσταση της $ y=f(x) $
      \linebreak η οποία έχει \textbf{ανακλαστεί} γύρω από τον 
      \textcolor{Col1}{άξονα $x$}.
    \item $ y=  f(-x) $ είναι η γραφ. παράσταση της $ y=f(x) $
      \linebreak η οποία έχει \textbf{ανακλαστεί} γύρω από τον 
      \textcolor{Col1}{άξονα $y$}.
  \end{myitemize}
}{
\begin{tikzpicture}[scale=0.7]
    \coordinate (0) at (0.75, 1.5) {};
    \coordinate (1) at (2, 3.25) {};
    \coordinate (2) at (2.925, 2.5) {};
    \coordinate (3) at (3.375, 3) {};
    \coordinate (o) at (0, 0) {};
    \coordinate (5) at (0.875, 3.75) {};
    \coordinate (6) at (2.125, 6.25) {};
    \coordinate (7) at (3.05, 4.75) {};
    \coordinate (8) at (3.5, 5.5) {};
    \coordinate (9) at (0.75, 0.25) {};
    \coordinate (10) at (2, 1.425) {};
    \coordinate (11) at (2.925, 1.2) {};
    \coordinate (12) at (3.375, 1.5) {};
    \coordinate (13) at (-0.625, 1.5) {};
    \coordinate (14) at (-1.875, 3.25) {};
    \coordinate (15) at (-2.8, 2.5) {};
    \coordinate (16) at (-3.25, 3) {};
    \coordinate (17) at (0.75, -1.5) {};
    \coordinate (18) at (2, -3.25) {};
    \coordinate (19) at (2.925, -2.5) {};
    \coordinate (20) at (3.375, -3) {};
    \draw [ultra thick,Col1,in=-180, out=90, looseness=0.75] (0.center) to (1.center);
    \draw [ultra thick,Col1,in=180, out=0] (1.center) to (2.center);
    \draw [ultra thick,Col1,in=-90, out=0] (2.center) to (3.center) 
      node[above right] {\boldsymbol{$y=f(x)$}};
    \draw [Col2,in=-180, out=90, looseness=0.75] (5.center) to (6.center);
    \draw [Col2,in=180, out=0] (6.center) to (7.center);
    \draw [Col2,in=-90, out=0] (7.center) to (8.center) node[above right,align={center}] 
      {$y=cf(x)$};
    \draw [in=-180, out=90, looseness=0.75] (9.center) to (10.center);
    \draw [in=180, out=0] (10.center) to (11.center);
    \draw [in=-90, out=0] (11.center) to (12.center) 
      node[black,right,align={center}] {$y= \dfrac{1}{c}f(x)$};
    \draw [blue!75,in=0, out=90, looseness=0.75] (13.center) to (14.center);
    \draw [blue!75,in=0, out=180] (14.center) to (15.center);
    \draw [blue!75,in=-90, out=180] (15.center) to (16.center) node[above
      right,yshift=8pt] {$y=f(-x)$};
    \draw [black!75,in=180, out=-90, looseness=0.75] (17.center) to (18.center);
    \draw [black!75,in=-180, out=0] (18.center) to (19.center);
    \draw [black!75,in=90, out=0] (19.center) to (20.center) 
      node[above right,yshift=15pt] {$y=-f(x)$};
    \draw[-latex,axis] (-3.5,0) -- (o) -- (6.0,0) node[below] (x) {$x$} ;
    \draw[-latex,axis] (0,-3.5) -- (o) -- (0,7.0) node[left] (y) {$y$} ;
    \draw [thick,orange!75,in=-180, out=90, looseness=0.75] 
      (0.75,1.5) to (1.3,3.25) ;
    \draw [thick,orange!75,in=180, out=0] (1.3,3.25) to (1.925,2.5);
    \draw [thick,orange!75,in=-90, out=0] (1.925,2.5) to (2.175,3) 
      node[above=5pt,orange!75!black] {$y=f(cx)$};
    \draw [thick,blue!50,in=-180, out=90, looseness=0.75] 
      (0.75,1.5) to (3.3,3.25) ;
    \draw [thick,blue!50,in=180, out=0] (3.3,3.25) to (4.925,2.5);
    \draw [thick,blue!50,in=-90, out=0] (4.925,2.5) to (5.375,3) 
      node[right=5pt,blue!75] {$y=f\Bigl(\frac{1}{c}x\Bigr)$};
    \end{tikzpicture}
}


\end{document}
