\documentclass[a4paper,table]{report}
\input{preamble_ask.tex}
\newcommand{\vect}[2]{(#1_1,\ldots, #1_#2)}
%%%%%%% nesting newcommands $$$$$$$$$$$$$$$$$$$
\newcommand{\function}[1]{\newcommand{\nvec}[2]{#1(##1_1,\ldots, ##1_##2)}}

\newcommand{\linode}[2]{#1_n(x)#2^{(n)}+#1_{n-1}(x)#2^{(n-1)}+\cdots +#1_0(x)#2=g(x)}

\newcommand{\vecoffun}[3]{#1_0(#2),\ldots ,#1_#3(#2)}


\input{tikz.tex}
\input{myboxes.tex}

\geometry{left=17mm,right=17mm,top=25.00mm,bottom=30.00mm,footskip=24.16mm,headsep=24.16mm}

\everymath{\displaystyle}
\pagestyle{askhseis}

\begin{document}

% \chapter*{Επιφάνειες}

\setcounter{chapter}{1}

\begin{center}
\minibox{\large \bfseries \textcolor{Col1}{Στοιχεία Θεωρίας για το Επιφανειακό
Ολοκλήρωμα}}
\end{center}

\vspace{\baselineskip}



\subsection*{Μοναδιαίο Διάνυσμα}

\begin{dfn}
  Το \textbf{μοναδιαίο} κάθετο διάνυσμα $ \mathbf{n} $ μιας επιφάνειας $ S $ μπορεί να 
  προσδιοριστεί από τις παρακάτω σχέσεις, ανάλογα με τον τρόπο περιγραφής της επιφάνειας.
  \begin{myitemize}
    \item Αν $ S\colon \mathbf{r}(u,v) = x(u,v)\mathbf{i}+y(u,v)\mathbf{j}+
      z(u,v)\mathbf{k} $ τότε $ \mathbf{\widehat{n}} = \frac{\mathbf{r_{u}} \times
      \mathbf{r_{v}}}{\norm{\mathbf{r_{u}} \times \mathbf{r_{v}}}} $ 
    \item Αν $ S\colon g(x,y,z)=0 $ τότε $ \mathbf{\widehat{n}} = 
      \frac{\grad g}{\norm{\grad g}} $
    \item Αν $ S\colon z=z(x,y) $ τότε θέτουμε $ g(x,y,z)=z-z(x,y)=0 $ οπότε
      $ \mathbf{\widehat{n}} = \frac{\grad g}{\norm{\grad g}} = 
      \frac{(-z_{x},-z_{y},1)}{\sqrt{1+(z_{x})^{2}+(z_{y})^{2}}} $ 
    \item Αν $ S\colon y=y(x,z) $ τότε θέτουμε $ g(x,y,z)=y-y(x,z)=0 $ οπότε
      $ \mathbf{\widehat{n}} = \frac{\grad g}{\norm{\grad g}} = 
      \frac{(-y_{x},1,-y_{z})}{\sqrt{1+(y_{x})^{2}+(y_{z})^{2}}} $ 
    \item Αν $ S\colon x=x(y,z) $ τότε θέτουμε $ g(x,y,z)=x-x(y,z)=0 $ οπότε
      $ \mathbf{\widehat{n}} = \frac{\grad g}{\norm{\grad g}} = 
      \frac{(1,-x_{y},-x_{z})}{\sqrt{1+(x_{y})^{2}+(x_{z})^{2}}} $ 
  \end{myitemize}
\end{dfn}

\begin{rem}
  Σε κάθε σημείο μίας λείας επιφάνειας, υπάρχουν δύο μοναδιαία διανύσματα, το $
  \mathbf{\widehat{n}} $ και το $ -\mathbf{\widehat{n}} $.  
\end{rem}


\subsection*{Διαφορικό Στοιχείο}
 
\begin{dfn}
  Το διαφορικό $ dS $ μιας επιφάνειας $S$ μπορεί να προσδιοριστεί από τις παρακάτω 
  σχέσεις, ανάλογα με τον τρόπο περιγραφής της επιφάνειας.
  \begin{myitemize}
    \item Αν $ S\colon \mathbf{r}(u,v) = x(u,v)\mathbf{i}+y(u,v)\mathbf{j}+z(u,v)\mathbf{k} $ 
      τότε $ dS = \norm{\mathbf{r_{u}} \times \mathbf{r_{v}}} \, du dv $ 
    \item Αν $ S\colon g(x,y,z)=0 $ τότε $dS = \frac{\norm{\grad g}}{\abs{\grad g \cdot 
      \mathbf{p}}}\, dA $ όπου $ \mathbf{p} $ είναι το μοναδιαίο κάθετο διάνυσμα
      στο επίπεδο κάτω από την $S$, συνήθως $ \mathbf{p} = \mathbf{i} $ 
      ή $ \mathbf{j} $ ή $ \mathbf{k} $, $ dA $ είναι το στοιχειώδες εμβαδό σε αυτό το
      επίπεδο και $ \grad(g)\cdot \mathbf{p} \neq 0 $ .  
    \item Αν $ S\colon z=z(x,y) $ τότε θέτουμε $ g(x,y,z)=z-z(x,y)=0 $ οπότε $ dS =
      \frac{\norm{\grad g}}{\abs{\grad g \cdot \mathbf{k}}}\, dxdy = 
      \sqrt{1+(z_{x})^{2}+(z_{y})^{2}} \,dxdy $ 
    \item Αν $ S\colon y=y(x,x) $ τότε θέτουμε $ g(x,y,z)=y-y(x,z)=0 $ οπότε $ dS =
      \frac{\norm{\grad g}}{\abs{\grad g \cdot \mathbf{j}}}\, dxdz = 
      \sqrt{1+(y_{x})^{2}+(y_{z})^{2}} \,dxdz $ 
    \item Αν $ S\colon x=x(y,z) $ τότε θέτουμε $ g(x,y,z)=x-x(y,z)=0 $ οπότε $ dS =
      \frac{\norm{\grad g}}{\abs{\grad g \cdot \mathbf{i}}}\, dydz = 
      \sqrt{1+(x_{y})^{2}+(x_{z})^{2}} \,dydz $ 
  \end{myitemize}
\end{dfn}


\subsection*{Διανυσματικό Διαφορικό Στοιχείο}

\begin{dfn}
  Για το διανυσματικό διαφορικό μιας επιφάνειας $S$ ισχύει $ \mathbf{dS} =
  \mathbf{\widehat{n}}\,dS $, οπότε προσδιορίζεται από τις παρακάτω σχέσεις, ανάλογα με τον τρόπο περιγραφής της επιφάνειας.
  \begin{myitemize}
    \item Αν $ S\colon \mathbf{r}(u,v) = x(u,v)\mathbf{i}+y(u,v)\mathbf{j}+z(u,v)\mathbf{k} $ 
      τότε $ \mathbf{dS} = \mathbf{\widehat{n}}\, dS = \frac{\mathbf{r_{u}} \times 
      \mathbf{r_{v}}}{\norm{\cancel{\mathbf{r_{u}} \times \mathbf{r_{v}}}}}
      \norm{\cancel{\mathbf{r_{u}} \times \mathbf{r_{v}}}} \, du dv = (\mathbf{r_{u}} \times 
      \mathbf{r_{v}}) \,du dv $ 
    \item Αν $ S\colon g(x,y,z)=0 $ τότε $ \mathbf{dS} = \mathbf{\widehat{n}}\, dS = 
      \frac{\grad g}{\norm{\cancel{\grad g}}} \frac{\norm{\cancel{\grad g}}}{\abs{\grad g 
      \cdot \mathbf{p}}}\, dA = \frac{\grad g}{\abs{\grad g \cdot \mathbf{p}}}\, dA$
    \item Αν $ S\colon z=z(x,y) $ τότε θέτουμε $ g(x,y,z)=z-z(x,y)=0 \Rightarrow \mathbf{dS} 
      = \frac{\grad g}{\abs{\grad g \cdot \mathbf{k}}}\, dxdy = (-z_{x},-z_{y},1) 
      \, dx dy $
    \item Αν $ S\colon y=y(x,z) $ τότε θέτουμε $ g(x,y,z)=z-z(x,y)=0 \Rightarrow \mathbf{dS} 
      = \frac{\grad g}{\abs{\grad g \cdot \mathbf{j}}}\, dxdz = (-y_{x},1,-y_{z}) 
      \, dx dz $
    \item Αν $ S\colon x=x(y,z) $ τότε θέτουμε $ g(x,y,z)=z-z(x,y)=0 \Rightarrow \mathbf{dS} 
      = \frac{\grad g}{\abs{\grad g \cdot \mathbf{i}}}\, dydz = (1,-x_{y},-x_{y}) 
      \, dy dz $
  \end{myitemize}
\end{dfn}


\subsection*{Διάφοροι Ορισμοί}
 
\begin{dfn}
  Μια επιφάνεια $S$ με διανυσματική παραμετρική μορϕή 
  $ \mathbf{r} =  \mathbf{r}(u,v) $, όπου $ (u,v) \in D \subseteq 
  \mathbb{R}^{2} $, λέγεται:
  \begin{myitemize}
    \item \textbf{απλή} αν η συνάρτηση $ \mathbf{r}(u,v) $ είναι 1-1 στο $D$ και άρα
      η επιφάνεια $S$ δεν τέμνει τον εαυτό της.
    \item \textbf{λεία} αν υπάρχουν και είναι συνεχείς οι μερικές παράγωγοι 
      $ \mathbf{r_{u}}$ και $ \mathbf{r_{v}} $ και ισχύει 
      $ \mathbf{r_{u}} \times \mathbf{r_{v}} \neq \mathbf{0}, \; \forall (u,v) 
      \in D \subseteq \mathbb{R}^{2} $.
    \item \textbf{κανονική} αν είναι απλή και λεία. 
    \item \textbf{προσανατολίσιμη} (ή δίπλευρη) όταν κινούμενοι συνέχεια σε αυτή δεν 
      είναι δυνατόν να μεταβούμε από τη μία πλευρά της στην άλλη χωρίς να συναντήσουμε 
      το σύνορό της.
    \item \textbf{προσανατολισμένη} όταν έχει καθοριστεί η θετική ή αρνητική πλευρά της.
  \end{myitemize}
\end{dfn}



\end{document}
