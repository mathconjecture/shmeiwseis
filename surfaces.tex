\input{preamble.tex}
\newcommand{\vect}[2]{(#1_1,\ldots, #1_#2)}
%%%%%%% nesting newcommands $$$$$$$$$$$$$$$$$$$
\newcommand{\function}[1]{\newcommand{\nvec}[2]{#1(##1_1,\ldots, ##1_##2)}}

\newcommand{\linode}[2]{#1_n(x)#2^{(n)}+#1_{n-1}(x)#2^{(n-1)}+\cdots +#1_0(x)#2=g(x)}

\newcommand{\vecoffun}[3]{#1_0(#2),\ldots ,#1_#3(#2)}


\input{tikz.tex}
\input{myboxes.tex}

\everymath{\displaystyle}
\pagestyle{askhseis}

\begin{document}

% \chapter*{Επιφάνειες}

\setcounter{chapter}{1}


\twocolumnsides{
  \subsection*{Καμπύλες}
  \begin{myitemize}
    \item $ y=f(x) $ ή $ x=g(y) $
    \item $ F(x,y)=0 $ 
    \item $ r(t) = x(t) \mathbf{i} + y(t) \mathbf{j} + z(t) \mathbf{k} $ 
  \end{myitemize}
  }{
  \subsection*{Επιφάνειες}
  \begin{myitemize}
    \item $ z=f(x,y) $ ή $ y=g(x,z) $ ή $ x=h(y,z) $ 
    \item $ F(x,y,z)=0 $ 
    \item $ r(u,v) = x(u,v) \mathbf{i} + y(u,v) \mathbf{j} + z(u,v) \mathbf{k} $ 
  \end{myitemize}
}


\begin{dfn}
  Μια επιφάνεια $S$ με διανυσματική παραμετρική μορϕή $ \mathbf{r}(u,v) $, όπου 
  $ (u,v) \in D \subseteq \mathbb{R}^{2} $, λέγεται:
  \begin{myitemize}
    \item \textbf{απλή} αν η συνάρτηση $ \mathbf{r}(u,v) $ είναι 1-1 στο $D$ και άρα
      η επιφάνεια $S$ δεν τέμνει τον εαυτό της.
    \item \textbf{λεία} αν υπάρχουν και είναι συνεχείς οι μερικές παράγωγοι 
      $ \mathbf{r_{u}}$ και $ \mathbf{r_{v}} $ και ισχύει 
      $ \mathbf{r_{u}} \times \mathbf{r_{v}} \neq \mathbf{0} $.
    \item \textbf{κανονική} αν είναι απλή και λεία. 
    \item \textbf{προσανατολίσιμη} (ή δίπλευρη) όταν κινούμενοι συνέχεια σε αυτή δεν 
      είναι δυνατόν να μεταβούμε από τη μία πλευρά της στην άλλη χωρίς να συναντήσουμε 
      το σύνορό της.
    \item \textbf{προσανατολισμένη} όταν έχει καθοριστεί η θετική ή αρνητική πλευρά της.
  \end{myitemize}
\end{dfn}

\begin{dfn}
  Το μοναδιαίο κάθετο διάνυσμα $ \mathbf{n} $ μιας επιφάνειας $ S $ μπορεί να 
  προσδιοριστεί από τις παρακάτω σχέσεις, ανάλογα με τον τρόπο περιγραφής της επιφάνειας.
  \begin{myitemize}
    \item Αν $ \mathbf{r}(u,v) = x(u,v)\mathbf{i}+y(u,v)\mathbf{j}+z(u,v)\mathbf{k} $ 
      τότε $ \mathbf{\widehat{n}} = \frac{\mathbf{r_{u}} \times
      \mathbf{r_{v}}}{\norm{\mathbf{r_{u}} \times \mathbf{r_{v}}}} $ 
    \item Αν $ g(x,y,z)=0 $ τότε $ \mathbf{\widehat{n}} = 
      \frac{\grad g}{\norm{\grad g}} $
    \item Αν $ z=z(x,y) $ τότε θέτουμε $ g(x,y,z)=z-z(x,y)=0 \Rightarrow 
      \mathbf{\widehat{n}} = \frac{\grad g}{\norm{\grad g}} = 
      \frac{(-z_{x},-z_{y},1)}{\sqrt{1+(z_{x})^{2}+(z_{y})^{2}}} $ 
  \end{myitemize}
\end{dfn}

\begin{rem}
  Σε κάθε ομαλό σημείο μίας επιφάνειας, υπάρχουν δύο μοναδιαία διανύσματα, το $
  \mathbf{\widehat{n}} $ και το $ -\mathbf{\widehat{n}} $.  
\end{rem}

\begin{dfn}
  Το διαφορικό $ dS $ μιας επιφάνειας $S$ μπορεί να προσδιοριστεί από τις παρακάτω 
  σχέσεις, ανάλογα με τον τρόπο περιγραφής της επιφάνειας.
  \begin{myitemize}
    \item Αν $ \mathbf{r}(u,v) = x(u,v)\mathbf{i}+y(u,v)\mathbf{j}+z(u,v)\mathbf{k} $ 
      τότε $ dS = \norm{\mathbf{r_{u}} \times \mathbf{r_{v}}} \, du dv $ 
    \item Αν $ g(x,y,z)=0 $ τότε $dS = \frac{\norm{\grad g}}{\abs{\grad g \cdot 
      \mathbf{p}}}\, dA $ όπου $ \mathbf{p} $ είναι συνήθως $ \mathbf{p} = \mathbf{i} $ 
      ή $ \mathbf{j} $ ή $ \mathbf{k} $ και $ \grad(g)\cdot \mathbf{p} \neq 0 $ .  
    \item Αν $ z=z(x,y) $ τότε θέτουμε $ g(x,y,z)=z-z(x,y)=0 \Rightarrow dS =
      \frac{\norm{\grad g}}{\abs{\grad g \cdot \mathbf{p}}}\, dA = 
      \sqrt{1+(z_{x})^{2}+(z_{y})^{2}} \,dxdy $ 
  \end{myitemize}
\end{dfn}

\begin{dfn}
  Για το διανυσματικό διαφορικό μιας επιφάνειας $S$ ισχύει $ \mathbf{dS} =
  \mathbf{\widehat{n}}dS $, οπότε προσδιορίζεται από τις παρακάτω σχέσεις, ανάλογα με τον τρόπο περιγραφής της επιφάνειας.
  \begin{myitemize}
    \item Αν $ \mathbf{r}(u,v) = x(u,v)\mathbf{i}+y(u,v)\mathbf{j}+z(u,v)\mathbf{k} $ 
      τότε $ \mathbf{dS} = \mathbf{\widehat{n}} dS = (\mathbf{r_{u}} \times 
      \mathbf{r_{v}}) \, du dv $ 
    \item Αν $ g(x,y,z)=0 $ τότε $ \mathbf{dS} = \frac{\grad g}{\abs{\grad g 
      \cdot \mathbf{p}}}\, dA $
    \item Αν $ z=z(x,y) $ τότε θέτουμε $ g(x,y,z)=z-z(x,y)=0 \Rightarrow \mathbf{dS} 
      = \frac{\grad g}{\abs{\grad g \cdot \mathbf{p}}}\, dA = (-z_{x},-z_{y},1) 
      \, dx dy $
  \end{myitemize}
\end{dfn}

\end{document}
