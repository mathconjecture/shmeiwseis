\input{preamble_test.tex}
\input{definitions_ask.tex}

\pagestyle{vangelis}
\everymath{\displaystyle}

\begin{document}

\section{Διαφορικοί Τελεστές}

\begin{dfn}
  Τελεστής  \textcolor{Col1}{Hamilton} ή \textcolor{Col1}{ανάδελτα} 
  \[ \grad = \pdv{}{x} \mathbf{i} + \pdv{}{y} \mathbf{j} + \pdv{}{z} \mathbf{k} \]
\end{dfn}
\begin{rem}
Ο τελεστής ανάδελτα συμπεριφέρεται ακριβώς όπως ένα διάνυσμα.
\end{rem}

\section{Κλίση}

Έστω $ f(x,y,z) $ πραγματική συνάρτηση (βαθμωτό πεδίο), συνεχής, με συνεχείς μερικές 
παραγώγους 1ης τάξης.

\begin{dfn}
  \textcolor{Col1}{Κλίση} της $f$ ονομάζεται η \textcolor{Col1}{διανυσματική} συνάρτηση
  \[
    \grad f = \pdv{f}{x} \mathbf{i} + \pdv{f}{y} \mathbf{j} + \pdv{f}{z} \mathbf{k} 
  \] 
\end{dfn}

\section{Ιδιότητες}

\begin{enumerate}[i)]
  \item $ \grad (f \pm g) = \grad f \pm \grad g $
  \item $ \grad (a f) = a \grad f, \quad \forall a \in \mathbb{R}  $
  \item $ \grad (fg) =  g \grad f +  f\grad g $
  \item $ \grad (\frac{f}{g}) = \frac{g \grad f - f \grad g}{g^{2}}  $
\end{enumerate}

\end{document}
