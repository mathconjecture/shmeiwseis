\input{preamble_ask.tex}
\input{definitions_ask.tex}
\input{tikz.tex}

\geometry{top=2.7cm}

\pgfdeclarelayer{bg}
\pgfdeclarelayer{fg}
\pgfsetlayers{bg,main,fg}

\renewcommand{\qedsymbol}{}

\pagestyle{vangelis}

\newcommand{\twocolumnsidesn}[2]{\begin{minipage}[t]{0.43\linewidth}\raggedright
        #1
        \end{minipage}\hfill\begin{minipage}[t]{0.48\linewidth}\raggedright
        #2
    \end{minipage}
}

\newcommand{\threecolumnsides}[3]{\begin{minipage}[c]{0.39\linewidth}\raggedright
        #1
        \end{minipage}\hfill\begin{minipage}[c]{0.30\linewidth}\raggedright
        #2
        \end{minipage}\hfill\begin{minipage}[c]{0.30\linewidth}\raggedright
        #3
    \end{minipage}
}

\newcommand{\threecolumnsidess}[3]{\begin{minipage}[c]{0.30\linewidth}\raggedright
        #1
        \end{minipage}\hfill\begin{minipage}[c]{0.30\linewidth}\raggedright
        #2
        \end{minipage}\hfill\begin{minipage}[c]{0.30\linewidth}\raggedright
        #3
    \end{minipage}
}

\begin{document}

\begin{center}
  \minibox{\large\bfseries \textcolor{Col1}{Αλλαγή Μεταβλητών στο Τριπλό Ολοκλήρωμα}}
\end{center}



\section*{Γενική Μεθοδολογία}

\begin{enumerate}
  \item Επιλέγουμε \textbf{κατάλληλες} εξισώσεις μετασχηματισμού ανάλογα με το σχήμα του 
    χωρίου 
    $\Omega$. 
    \[
      \left.
        \begin{matrix}
          x=x(u,v,w) \\
          y=y(u,v,w) \\
          z=z(u,v,w)
        \end{matrix} 
      \right\} \quad \text{και} \quad J = \pdv{(x,y,z)}{(u,v,w)} = 
      \begin{vmatrix*}[r]
        x_{u} & x_{v} & x_{w} \\
        y_{u} & y_{v} & y_{w} \\
        z_{u} & z_{v} & z_{w}
      \end{vmatrix*}
    \] 
  \item Μετασχηματίζουμε τη συνάρτηση $ f(x,y,z) $, θέτοντας όπου $ x=x(u,v,w) $, 
    όπου $ y=y(u,v,w) $ και όπου $ z=z(u,v,w) $
    \[
      g(u,v,w) = f(x(u,v,w),y(u,v,w),z(u,v,w)) 
    \] 
  \item Μετασχηματίζουμε το χωρίο $\Omega$, μετασχηματίζοντας συνήθως τις 
    \textbf{επιφάνειες} που αποτελούν το \textcolor{Col1}{σύνορο} του $\Omega$, 
    θέτοντας όπου $ x=x(u,v,w) $, όπου $ y=y(u,v,w) $ και όπου $ z=z(u,v,w) $ και 
    στη συνέχεια σχεδιάζουμε το νέο χωρίο $ \Omega^{*} $ στο χώρο $ uvw $.  
  \item Χρησιμοποιούμε τον τύπο αλλαγής μεταβλητών για το τριπλό ολοκλήρωμα:
    \[
      \boxed{\iiint_{\Omega} f(x,y,z) \,{dx}{dy}{dz} = \iiint_{\Omega^{*}} g(u,v,w) 
        \abs{\pdv{(x,y,z)}{(u,v,w)}}
      \,{du}{dv}{dw}}
    \] 
    όπου $ \abs{\pdv{(x,y,z)}{(u,v,w)}} $ είναι η \textbf{απόλυτη τιμή} της
    \textcolor{Col1}{Ιακωβιανής ορίζουσας} του μετασχηματισμού.
\end{enumerate}

\section*{Επιλογή Κατάλληλου Μετασχηματισμού}

\begin{myitemize}[leftmargin=*]
  \item Όταν το χωρίο $\Omega$ είναι \textbf{κύλινδρος}, ή έχει κυλινδρική συμμετρία, 
    τότε μετασχηματίζουμε σε \textcolor{Col1}{κυλινδρικές} συντεταγμένες:
    \[
      \left.
        \begin{matrix}
          x=r \cos{\theta} \\
          y=r \sin{\theta} \\
          z=z
        \end{matrix} 
      \right\} \quad \text{και} \quad J = \pdv{(x,y,z)}{(r, \theta,z)} =  
      \begin{vmatrix*}[r]
        x_{r} & x_{\theta} & x_{z} \\
        y_{r} & y_{\theta} & y_{z} \\
        x_{z} & y_{z} & z_{z}
      \end{vmatrix*} = 
      \begin{vmatrix*}[r]
        \cos{\theta} & -r \sin{\theta} & 0 \\
        \sin{\theta} & r \cos{\theta} & 0 \\
        0 & 0 & 1
      \end{vmatrix*} = r 
    \] 

  \item Όταν το χωρίο $\Omega$ είναι \textbf{σφαίρα}, ή τμήμα σφαίρας, τότε
    μετασχηματίζουμε σε \textcolor{Col1}{σφαιρικές} συντεταγμένες:
    \[
      \left.
        \begin{matrix}
          x=\rho \sin{\phi} \cos{\theta} \\
          y=\rho \sin{\phi} \sin{\theta} \\
          z=\rho \cos{\phi}
        \end{matrix} 
      \right\} \;\; \text{και} \;\; J = \pdv{(x,y,z)}{(\rho, \phi, \theta)} = 
      \begin{vmatrix*}[r]
        x_{\rho} & x_{\phi} & x_{\theta} \\
        y_{\rho} & y_{\phi} & y_{\theta} \\
        z_{\rho} & z_{\phi} & z_{\theta}
      \end{vmatrix*} = 
      \begin{vmatrix*}[r]
        \sin{\phi} \cos{\theta} & \rho \cos{\phi} \cos{\theta} & -\rho \sin{\phi}
        \sin{\theta} \\
        \sin{\phi} \sin{\theta} & \rho \cos{\phi} \sin{\theta} & \rho \sin{\phi}
        \cos{\theta} \\
        \cos{\phi} & - \rho \sin{\phi} & 0
      \end{vmatrix*} = \rho ^{2} \sin{\phi} 
    \] 
\end{myitemize}


\vspace{\baselineskip}


\twocolumnsidesn{
  \section*{Κυλινδρικές Συντεταγμένες}

  \begin{tikzpicture}[scale=0.8]
    \coordinate (0) at (0, 20) ;
    \node at (0) [left,xshift=-5pt,yshift=4pt] {$0$} ;
    \coordinate (y) at (4, 20) ;
    \node at (y) [right] {$y$} ;
    \coordinate (z) at (0, 24) ;
    \node at (z) [left] {$z$} ;
    \coordinate (p) at (1.5, 22.5) ;
    \fill (p) circle (2pt) ;
    \node at (p) [right] {$P$} ;
    \coordinate (p1) at (1.5, 19) ;
    \fill (p1) circle (2pt) ;
    \node at (p1) [below right] {$P'$} ;
    \coordinate (x) at ($(0)+(-150:3.5)$) ;
    \draw[name path=xaxis,-latex] (0)-- (x) node[left] {$x$} ;
    \draw[name path=yaxis,-latex] (0.center) to (y.center);
    \draw[dashed] (p.center) to node[pos=0.4,right] {$z$} (p1.center);
    \coordinate (z1) at ($(p)+(150:3)$) ;
    \coordinate (x1) at ($(p1)+(180:5)$) ;
    \coordinate (y1) at ($(p1)+(30:3)$) ;
    \coordinate (xp) at (intersection of p1--x1 and 0--x) ;
    \coordinate (yp) at (intersection of p1--y1 and 0--y) ;
    \draw[dashed] (p1) -- (xp) node [left=3pt,yshift=2pt] {$x$} ;
    \draw[dashed] (p1) -- (yp) node [above right] {$y$};
    \path pic[fill=Col1!25,draw,-latex,angle radius=12pt,"$\theta$"] {angle=xp--0--p1} ;
    \draw[-latex] (0.center) to (z.center);
    \draw (0.center) to node[pos=0.6,above] {$r$} (p1.center);
  \end{tikzpicture}
  \threecolumnsidess{
    \begin{empheq}[box=\mathboxr]{gather*}
      x= r\cos{\theta} \\
      y= r\sin{\theta} \\
      z= z
    \end{empheq}
  }
  {\begin{empheq}[box=\mathboxg]{gather*}
      J=r
  \end{empheq}}
  {\begin{empheq}[box=\mathboxg]{gather*}
      r \geq 0 \\
      0 \leq \theta \leq 2 \pi \\
      J = r
  \end{empheq}}
  }{
  \section*{Σφαιρικές Συντεταγμένες}
  \begin{tikzpicture}[scale=0.8]
    \coordinate (0) at (0, 20) ;
    \node at (0) [left,xshift=-5pt,yshift=4pt] {$0$} ;
    \coordinate (y) at (4, 20) ;
    \node at (y) [right] {$y$} ;
    \coordinate (z) at (0, 24) ;
    \node at (z) [left] {$z$} ;
    \coordinate (p) at (1.5, 22.5) ;
    \fill (p) circle (2pt) ;
    \node at (p) [right] {$P$} ;
    \coordinate (p1) at (1.5, 19) ;
    \fill (p1) circle (2pt) ;
    \node at (p1) [below right] {$P'$} ;
    \coordinate (x) at ($(0)+(-150:3.5)$) ;
    \draw[name path=xaxis,-latex] (0)-- (x) node[left] {$x$} ;
    \draw[name path=yaxis,-latex] (0.center) to (y.center);
    \draw[dashed] (p.center) to node[pos=0.4,right] {$z$} (p1.center);
    \coordinate (z1) at ($(p)+(150:3)$) ;
    \coordinate (zp) at (intersection of p--z1 and 0--z) ;
    \coordinate (x1) at ($(p1)+(180:5)$) ;
    \coordinate (y1) at ($(p1)+(30:3)$) ;
    \coordinate (xp) at (intersection of p1--x1 and 0--x) ;
    \coordinate (yp) at (intersection of p1--y1 and 0--y) ;
    \draw[dashed] (p1) -- (xp) node [left=3pt,yshift=2pt] {$x$} ;
    \draw[dashed] (p1) -- (yp) node [above right] {$y$};
    \draw[dashed] (p) -- (zp) node [left] {$z$};
    \coordinate (r1) at ($(p)!0.85!(zp)$) ;
    \draw (r1) -- ++(0,-8pt) -- ++(150:7.5pt) ;
    \path pic[fill=Col1!25,draw,-latex,angle radius=12pt,"$\theta$"] {angle=xp--0--p1} ;
    \path pic[fill=Col2!25,draw,latex-,angle radius=25pt,"$\phi$"] {angle=p--0--z} ;
    \draw[Col1] (0.center) to node[pos=0.6,left] {$\rho$} (p.center);
    \draw[-latex] (0.center) to (z.center);
    \draw (0.center) to node[pos=0.6,above] {$r$} (p1.center);
  \end{tikzpicture}
  \threecolumnsides{
    \begin{empheq}[box=\mathboxr]{gather*}
      x= \rho \sin{\phi} \cos{\theta} \\
      y= \rho \sin{\phi} \sin{\theta} \\
      z= \rho \cos{\phi} 
    \end{empheq}
  }
  {\begin{empheq}[box=\mathboxg]{gather*}
      J=\rho^{2}\sin{\phi}
  \end{empheq}}
  {\begin{empheq}[box=\mathboxg]{gather*}
      \rho \geq 0 \\
      0 \leq \theta \leq 2 \pi \\
      0 \leq \phi \leq \pi 
  \end{empheq}}
}




\end{document}


