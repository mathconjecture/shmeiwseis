\documentclass[a4paper,table]{report}
\input{preamble_ask.tex}
\newcommand{\vect}[2]{(#1_1,\ldots, #1_#2)}
%%%%%%% nesting newcommands $$$$$$$$$$$$$$$$$$$
\newcommand{\function}[1]{\newcommand{\nvec}[2]{#1(##1_1,\ldots, ##1_##2)}}

\newcommand{\linode}[2]{#1_n(x)#2^{(n)}+#1_{n-1}(x)#2^{(n-1)}+\cdots +#1_0(x)#2=g(x)}

\newcommand{\vecoffun}[3]{#1_0(#2),\ldots ,#1_#3(#2)}


\input{tikz.tex}
\input{myboxes}

% \geometry{top=3.5cm}
\pagestyle{vangelis}
\everymath{\displaystyle}
\setcounter{chapter}{1}

\input{insbox}


\newcommand{\twocolumnsidelcc}[2]{\begin{minipage}[c]{0.28\linewidth}
        #1
        \end{minipage}\hfill\begin{minipage}[c]{0.70\linewidth}
        #2
    \end{minipage}
}
\newcommand{\twocolumnsidelccc}[2]{\begin{minipage}[c]{0.55\linewidth}
        #1
        \end{minipage}\hfill\begin{minipage}[c]{0.45\linewidth}
        #2
    \end{minipage}
}






\begin{document}

\begin{center}
  \minibox{\large \bfseries \textcolor{Col1}{Διπλό Ολοκλήρωμα}}
\end{center}

\subsection*{$D$ Ορθογώνιο}

Έστω η \textbf{ορθογώνια} επίπεδη περιοχή $ D = \{ (x,y) \in \mathbb{R}^{2} \; : \; a \leq x \leq
b \quad \text{και} \quad c \leq y \leq d \} $.

\twocolumnsidelcc{
  \vspace{\baselineskip}
  \begin{tikzpicture}[scale=0.8]
    \draw[thick,stealth-stealth,blue!50] (0,2.8) node[left] {$y$} -- (0,0) node (o) {} --
      (3.5,0) node[right] {$x$} ;
    % \node (0) at (0, 0) {};
    \node (3) at (0.65, 0.65) {};
    \node (4) at (2.75, 0.65) {};
    \node (5) at (0.65, 2.05) {};
    \node (6) at (2.75, 2.05) {};
    \draw[very thick,Col1,fill=blue!10] (3) rectangle (6) ;
    \draw[dashed] (3) -- (o-|3) node[below] {$a$} ;
    \draw[dashed] (3) -- (3-|o) node[left] {$c$} ;
    \draw[dashed] (4) -- (o-|4) node[below] {$b$} ;
    \draw[dashed] (5) -- (5-|o) node[left] {$d$};
    \draw (1.65,1.45) node (D) {$D$};
  \end{tikzpicture}
}
{
  \begin{mybox2}
    \vspace{0.5\baselineskip}
    \begin{thm}
      Αν η $ f(x,y) $ είναι συνεχής στην \textbf{ορθογώνια} περιοχή $ D $, τότε:
      \begin{empheq}[box=\mathboxg]{equation*} 
        \iint_{D} f(x,y) \,dx\,dy = \int_{c}^{d}\int_{a}^{b} f(x,y) \,dx\,dy   
        =\int_{a}^{b}\int_{c}^{d} f(x,y) \,dy\,dx   
      \end{empheq} 
    \end{thm}
  \end{mybox2}
}


\subsection*{$D$ χωρίο τύπου Ι}

Έστω η \textbf{τύπου} $I$ επίπεδη περιοχή $ D = \{ (x,y) \in \mathbb{R}^{2} \; : \; a \leq x \leq
b \quad \text{και} \quad y_{1}(x) \leq y \leq y_{2}(x) \} $, όπου $y_{1}(x)$, $ y_{2}(x)$
συνεχείς στο $ [a,b] $.

\twocolumnsidelcc{
  \vspace{\baselineskip}
  \begin{tikzpicture}[scale=0.8]
    \draw[thick,stealth-stealth,blue!50] (0,2.8) node[left] {$y$} -- (0,0) coordinate (O)
      -- (4.0,0) node[right] {$x$} ;
    \begin{scope}[xshift=-08pt]
      \draw[very thick,Col1,name path=curve1] (0.7,1.5) coordinate (A) .. controls (2.0,3.5) 
        and (2.4,0.5) ..  (3.3,1.7) coordinate (B) 
        node[pin={[xshift=5pt,pin edge={black,latex-},right,Col1]75: $y_{2}(x)$}] {} ;
      \draw[very thick,Col2,name path=curve2] (0.7,0.5) coordinate (C){} .. controls (2.0,1.5) 
        and (2.4,-0.5) ..  (3.3,0.7) coordinate (D) 
        node[pin={[xshift=5pt,pin edge={black,latex-},right,Col2]75: $y_{1}(x)$}] {} ;
      \draw[dashed] (A) -- (A|-O) node[below]{$a$}  (B) -- (B|-O) node[below]{$b$};
      \path[name path=xaxis] (A|-O) -- (B|-O);
      \tikzfillbetween[of = curve1 and curve2,on layer=background]{blue!10};
    \draw (1.7,1.5) node (D) {$D$};
  \end{scope}
\end{tikzpicture}
}
{
  \begin{mybox2}
    \vspace{0.5\baselineskip}
    \begin{thm}
      Αν η $ f(x,y) $ είναι συνεχής στην \textbf{τύπου I} περιοχή $ D $, τότε:
      \begin{empheq}[box=\mathboxg]{equation*} 
        \iint_{D} f(x,y) \,dx\,dy = \int_{a}^{b}\int_{y_{1}(x)}^{y_{2}(x)} f(x,y) \,dy\,dx   
      \end{empheq} 
    \end{thm}
  \end{mybox2}
}

\begin{mybox1}
\begin{rem}
  Αν μια επίπεδη περιοχή, είναι τύπου $I$, τότε:
  \begin{myitemize}
    \item Ολοκληρώνουμε πρώτα ως προς $y$.
    \item Λύνουμε,ως προς $y$ την πάνω (προς τα θετικά του άξονα $y$) και κάτω 
      (προς τα αρνητικά του άξονα $y$) καμπύλη του συνόρου του D.
  \end{myitemize}
\end{rem}
\end{mybox1}

\subsection*{$D$ χωρίο τύπου ΙΙ}

Έστω η \textbf{τύπου} $II$ επίπεδη περιοχή $ D = \{ (x,y) \in \mathbb{R}^{2} \; : \; c \leq y \leq
d \quad \text{και} \quad x_{1}(y) \leq x \leq x_{2}(y) \} $, όπου $x_{1}(y)$, $ x_{2}(y)$
συνεχείς στο $ [c,d] $.

\twocolumnsidelcc{
  \vspace{\baselineskip}
  \begin{tikzpicture}[scale=0.8]
  \path (1.4,1.5) coordinate (D) ;
  \path[thick,stealth-stealth,blue!50] (0,3.2) -- (0,0) coordinate (O)   -- (3.5,0) ;
    \begin{scope}[rotate around={-90:(D)}]
      \draw[very thick,Col1,name path=curve1] (0.5,2.0) coordinate (A) 
        node[pin={[yshift=10pt,pin edge={black,latex-},left,Col1]40: $x_{2}(y)$}] {} .. controls 
        (1.5,3.5) and (2.0,0.5) ..  (2.5,2.4) coordinate (B) ;
      \draw[very thick,Col2,name path=curve2] (0.5,0.9) coordinate (C) 
        node[pin={[yshift=10pt,pin edge={black,latex-},left,Col2]40: $x_{1}(y)$}] {} .. controls 
        (1.5,1.5) and (2.0,-0.5) ..  (2.5,1.2) coordinate (D);
      \draw[dashed] (A) -- (A|-O) node[left]{$c$}  (B) -- (B|-O) node[left]{$d$};
      \path[name path=xaxis] (A|-O) -- (B|-O);
      \tikzfillbetween[of = curve1 and curve2,on layer=background]{blue!10};
  \end{scope}
  \draw[thick,stealth-stealth,blue!50] (0,3.2) node[left] {$y$} -- (0,0) coordinate (O)  --
    (3.5,0) node[right] {$x$} ;
  \draw (1.4,1.5) node (D) {$D$};
\end{tikzpicture}
}
{
  \begin{mybox2}
    \vspace{0.5\baselineskip}
    \begin{thm}
      Αν η $ f(x,y) $ είναι συνεχής στην \textbf{τύπου ΙΙ} περιοχή $ D $, τότε:
      \begin{empheq}[box=\mathboxg]{equation*} 
        \iint_{D} f(x,y) \,dx\,dy = \int_{c}^{d}\int_{x_{1}(y)}^{x_{2}(y)} f(x,y) \,dx\,dy   
      \end{empheq} 
    \end{thm}
  \end{mybox2}
}

\begin{mybox1}
\begin{rem}
  Αν μια επίπεδη περιοχή, είναι τύπου $II$, τότε:
  \begin{myitemize}
    \item Ολοκληρώνουμε πρώτα ως προς $x$.
    \item Λύνουμε ως προς $x$ την πάνω (προς τα θετικά του άξονα $x$) και κάτω 
      (προς τα αρνητικά του άξονα $x$) καμπύλη του συνόρου του D.
  \end{myitemize}
\end{rem}
\end{mybox1}

\twocolumnsidelcc{
  \vspace{\baselineskip}
  \begin{tikzpicture}[scale=0.8]
    \draw[thick,stealth-stealth,blue!50] (0,2.8) node[left] {$y$} -- (0,0) coordinate (O)
      -- (4.0,0) node[right] {$x$} ;
    \begin{scope}[xshift=-08pt]
      \draw[very thick,Col1,name path=curve1] (0.7,1.5) coordinate (A) .. controls (2.0,3.5) 
        and (2.4,0.5) ..  (3.3,1.7) coordinate (B) 
        node[pin={[xshift=5pt,pin edge={black,latex-},right,Col1]75: $y_{2}(x)$}] {} ;
      \draw[very thick,Col2,name path=curve2] (0.7,0.5) coordinate (C){} .. controls (2.0,1.5) 
        and (2.4,-0.5) ..  (3.3,0.7) coordinate (D) 
        node[pin={[xshift=5pt,pin edge={black,latex-},right,Col2]75: $y_{1}(x)$}] {} ;
      \draw[dashed] (A) -- (A|-O) node[below]{$a$}  (B) -- (B|-O) node[below]{$b$};
      \path[name path=xaxis] (A|-O) -- (B|-O);
      \tikzfillbetween[of = curve1 and curve2,on layer=background]{blue!10};
    \draw (1.7,1.5) node (D) {$D$};
  \begin{scope}[opacity=0.4]
    \draw[-latex,thick,Col1] (1.0,0.1) -- (1.0,2.6);
    \draw[-latex,thick,Col1] (1.4,0.1) -- (1.4,2.6);
    \draw[-latex,thick,Col1] (1.8,0.1) -- (1.8,2.6);
    \draw[-latex,thick,Col1] (2.2,0.1) -- (2.2,2.6);
    \draw[-latex,thick,Col1] (2.6,0.1) -- (2.6,2.6);
    \draw[-latex,thick,Col1] (3.0,0.1) -- (3.0,2.6);
  \end{scope}
  \end{scope}
\end{tikzpicture}
}
{
  \begin{rem}
    Μια επίπεδη περιοχή είναι τύπου $I$, αν κάθε ευθεία παράλληλη προς τον άξονα $y$ , 
    που διέρχεται από το εσωτερικό της περιοχής, τέμνει την κάτω καμπύλη του συνόρου 
    του $D$, σε ένα ακριβώς σημείο και αυτή η καμπύλη, είναι η ίδια καμπύλη, με την 
    έννοια ότι περιγράφεται από την ίδια εξίσωση. Ομοίως και για την πάνω καμπύλη.
  \end{rem}
}

\twocolumnsidelcc{
  \vspace{\baselineskip}
  \begin{tikzpicture}[scale=0.8]
  \path (1.4,1.5) coordinate (D) ;
  \path[thick,stealth-stealth,blue!50] (0,3.2) -- (0,0) coordinate (O)   -- (3.5,0) ;
    \begin{scope}[rotate around={-90:(D)}]
      \draw[very thick,Col1,name path=curve1] (0.5,2.0) coordinate (A) 
        node[pin={[yshift=10pt,pin edge={black,latex-},left,Col1]40: $x_{2}(y)$}] {} .. controls 
        (1.5,3.5) and (2.0,0.5) ..  (2.5,2.4) coordinate (B) ;
      \draw[very thick,Col2,name path=curve2] (0.5,0.9) coordinate (C) 
        node[pin={[yshift=10pt,pin edge={black,latex-},left,Col2]40: $x_{1}(y)$}] {} .. controls 
        (1.5,1.5) and (2.0,-0.5) ..  (2.5,1.2) coordinate (D);
      \draw[dashed] (A) -- (A|-O) node[left]{$c$}  (B) -- (B|-O) node[left]{$d$};
      \path[name path=xaxis] (A|-O) -- (B|-O);
      \tikzfillbetween[of = curve1 and curve2,on layer=background]{blue!10};
    \begin{scope}[opacity=0.4,xshift=-8pt,yshift=6pt]
    \draw[-latex,thick,Col1] (1.0,0.1) -- (1.0,2.6);
    \draw[-latex,thick,Col1] (1.4,0.1) -- (1.4,2.6);
    \draw[-latex,thick,Col1] (1.8,0.1) -- (1.8,2.6);
    \draw[-latex,thick,Col1] (2.2,0.1) -- (2.2,2.6);
    \draw[-latex,thick,Col1] (2.6,0.1) -- (2.6,2.6);
  \end{scope}
  \end{scope}
  \draw[thick,stealth-stealth,blue!50] (0,3.2) node[left] {$y$} -- (0,0) coordinate (O)  --
    (3.5,0) node[right] {$x$} ;
  \draw (1.4,1.5) node (D) {$D$};
\end{tikzpicture}
}
{
\begin{rem}
    Μια επίπεδη περιοχή είναι τύπου $II$, αν κάθε ευθεία παράλληλη προς τον άξονα $x$ , 
    που διέρχεται από το εσωτερικό της περιοχής, τέμνει την κάτω καμπύλη του συνόρου 
    του $D$, σε ένα ακριβώς σημείο και αυτή η καμπύλη, είναι η ίδια καμπύλη, με την 
    έννοια ότι περιγράφεται από την ίδια εξίσωση. Ομοίως και για την πάνω καμπύλη.
  \end{rem}
}

 

%todo να φτιαξω και να ολοκληρωσω σημειωσεις 2πλο ολοκληρωμα


\end{document}
