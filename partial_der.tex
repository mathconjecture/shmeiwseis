\documentclass[a4paper,12pt]{article}
\usepackage{etex}
%%%%%%%%%%%%%%%%%%%%%%%%%%%%%%%%%%%%%%
% Babel language package
\usepackage[english,greek]{babel}
% Inputenc font encoding
\usepackage[utf8]{inputenc}
%%%%%%%%%%%%%%%%%%%%%%%%%%%%%%%%%%%%%%

%%%%% math packages %%%%%%%%%%%%%%%%%%
\usepackage{amsmath}
\usepackage{amssymb}
\usepackage{amsfonts}
\usepackage{amsthm}
\usepackage{proof}

\usepackage{physics}

%%%%%%% symbols packages %%%%%%%%%%%%%%
\usepackage{dsfont}
\usepackage{stmaryrd}
%%%%%%%%%%%%%%%%%%%%%%%%%%%%%%%%%%%%%%%


%%%%%% graphicx %%%%%%%%%%%%%%%%%%%%%%%
\usepackage{graphicx}
\usepackage{color}
%\usepackage{xypic}
\usepackage[all]{xy}
\usepackage{calc}
%%%%%%%%%%%%%%%%%%%%%%%%%%%%%%%%%%%%%%%

\usepackage{enumerate}

\usepackage{fancyhdr}
%%%%% header and footer rule %%%%%%%%%
\setlength{\headheight}{14pt}
\renewcommand{\headrulewidth}{0pt}
\renewcommand{\footrulewidth}{0pt}
\fancypagestyle{plain}{\fancyhf{}
\fancyhead{}
\lfoot{}
\rfoot{\small \thepage}}
\fancypagestyle{vangelis}{\fancyhf{}
\rhead{\small \leftmark}
\lhead{\small }
\lfoot{}
\rfoot{\small \thepage}}
%%%%%%%%%%%%%%%%%%%%%%%%%%%%%%%%%%%%%%%

\usepackage{hyperref}
\usepackage{url}
%%%%%%% hyperref settings %%%%%%%%%%%%
\hypersetup{pdfpagemode=UseOutlines,hidelinks,
bookmarksopen=true,
pdfdisplaydoctitle=true,
pdfstartview=Fit,
unicode=true,
pdfpagelayout=OneColumn,
}
%%%%%%%%%%%%%%%%%%%%%%%%%%%%%%%%%%%%%%



\usepackage{geometry}
\geometry{left=25.63mm,right=25.63mm,top=36.25mm,bottom=36.25mm,footskip=24.16mm,headsep=24.16mm}

%\usepackage[explicit]{titlesec}
%%%%%% titlesec settings %%%%%%%%%%%%%
%\titleformat{\chapter}[block]{\LARGE\sc\bfseries}{\thechapter.}{1ex}{#1}
%\titlespacing*{\chapter}{0cm}{0cm}{36pt}[0ex]
%\titleformat{\section}[block]{\Large\bfseries}{\thesection.}{1ex}{#1}
%\titlespacing*{\section}{0cm}{34.56pt}{17.28pt}[0ex]
%\titleformat{\subsection}[block]{\large\bfseries{\thesubsection.}{1ex}{#1}
%\titlespacing*{\subsection}{0pt}{28.80pt}{14.40pt}[0ex]
%%%%%%%%%%%%%%%%%%%%%%%%%%%%%%%%%%%%%%

%%%%%%%%% My Theorems %%%%%%%%%%%%%%%%%%
\newtheorem{thm}{Θεώρημα}[section]
\newtheorem{cor}[thm]{Πόρισμα}
\newtheorem{lem}[thm]{λήμμα}
\theoremstyle{definition}
\newtheorem{dfn}{Ορισμός}[section]
\newtheorem{dfns}[dfn]{Ορισμοί}
\theoremstyle{remark}
\newtheorem{remark}{Παρατήρηση}[section]
\newtheorem{remarks}[remark]{Παρατηρήσεις}
%%%%%%%%%%%%%%%%%%%%%%%%%%%%%%%%%%%%%%%




\newcommand{\vect}[2]{(#1_1,\ldots, #1_#2)}
%%%%%%% nesting newcommands $$$$$$$$$$$$$$$$$$$
\newcommand{\function}[1]{\newcommand{\nvec}[2]{#1(##1_1,\ldots, ##1_##2)}}

\newcommand{\linode}[2]{#1_n(x)#2^{(n)}+#1_{n-1}(x)#2^{(n-1)}+\cdots +#1_0(x)#2=g(x)}

\newcommand{\vecoffun}[3]{#1_0(#2),\ldots ,#1_#3(#2)}





\everymath{\displaystyle}

\begin{document}

\chapter{Μερικές Παράγωγοι}

\section{Ορισμός}

\subsection{Δύο Μεταβλητών}

Έστω $ f \colon A \subseteq \mathbb{R}^{2} \to \mathbb{R} $
και $ (x_{0}, y_{0}) \in A $. Τότε η μερική παράγωγος της 
$f$ ως προς $x$ είναι:
\begin{align*}
    \eval{\pdv{f}{x}}_{(x_{0}, y_{0})} = \lim_{x \to x_{0}} 
    \frac{f(x, y_{0}) - f(x_{0}, y_{0})}{x - x_{0}} \overset{h=x- x_{0}}{=} \lim_{h \to 0}
    \frac{f(x_{0}+h, y_{0}) - f(x_{0}, y_{0})}{h}  
 \intertext{και η μερική παράγωγος της $f$ ως προς $y$ είναι:}
    \eval{\pdv{f}{y}}_{(x_{0}, y_{0})} = \lim_{y \to y_{0}} 
    \frac{f(x_{0}, y) - f(x_{0}, y_{0})}{y - y_{0}} \overset{k=y- y_{0}}{=} \lim_{k \to 0}
    \frac{f(x_{0}, y_{0}+k) - f(x_{0}, y_{0})}{k}  
 \end{align*}

 \subsection{Τριών Μεταβλητών}
 Έστω $ f \colon A \subseteq \mathbb{R}^{3} \to \mathbb{R} $ και $ (x_{0}, y_{0}, z_{0}) \in A $.
 Τότε η μερική παράγωγος της $f$ ως προς $x$ είναι :
\begin{align*}
    \eval{\pdv{f}{x}}_{(x_{0}, y_{0}, z_{0})} = \lim_{x \to x_{0}} 
    \frac{f(x, y_{0}, z_{0}) - f(x_{0}, y_{0}, z_{0})}{x - x_{0}} \overset{h=x- x_{0}}{=} \lim_{h \to 0}
    \frac{f(x_{0}+h, y_{0}, z_{0}) - f(x_{0}, y_{0}, z_{0})}{h}  
 \intertext{και η μερική παράγωγος της $f$ ως προς $y$ είναι:}
    \eval{\pdv{f}{y}}_{(x_{0}, y_{0}, z_{0})} = \lim_{y \to y_{0}} 
    \frac{f(x_{0}, y, z_{0}) - f(x_{0}, y_{0}, z_{0})}{y - y_{0}} \overset{k=y- y_{0}}{=} \lim_{k \to 0}
    \frac{f(x_{0}, y_{0}+k, z_{0}) - f(x_{0}, y_{0}, z_{0})}{k}  
 \intertext{και η μερική παράγωγος της $f$ ως προς $y$ είναι:}
    \eval{\pdv{f}{z}}_{(x_{0}, y_{0}, z_{0})} = \lim_{z \to z_{0}} 
    \frac{f(x_{0}, y_{0}, z) - f(x_{0}, y_{0}, z_{0})}{z - z_{0}} \overset{s=z- z_{0}}{=} \lim_{s \to 0}
    \frac{f(x_{0}, y_{0}, z_{0}+s) - f(x_{0}, y_{0}, z_{0})}{s}  
 \end{align*}

\section{Συμβολισμός}

Συμβολισμοί για τις μερικές παραγώγους της $f(x,y)$ ως 
προς $x$ και ως προς $y$ είναι:
\begin{align*}
    \eval{\pdv{f}{x} }_{(x_{0}, y_{0})} = \pdv{f(x_{0}, y_{0})}{x} = f_{x}(x_{0}, y_{0}) =
    {f'}_{x}(x_{0}, y_{0} ) \quad \text{και} \quad
    \eval{\pdv{f}{y} }_{(x_{0}, y_{0})} = \pdv{f(x_{0}, y_{0})}{y} = f_{y}(x_{0}, y_{0}) =
    {f'}_{y}(x_{0}, y_{0} ) 
\end{align*} 

\begin{examples}
\item {}
    \begin{enumerate}
        \item Δίνεται η $ f(x,y)=3xy^{2}-2x^{3} $. Να 
            βρεθούν οι $ f_{x}(0,1) $ και $ f_{y}(2,1) $.
            \begin{solution}
      \begin{align*}
          f_{x}(0,1) &= \lim_{x \to 0} \frac{f(x,1)-f(0,1)}{x-0} = \lim_{x \to 0}
          \frac{3x1^{2}-2x^{3}-0}{x} = \lim_{x \to 0} (3-2x^{2}) = 3
          \intertext{και}
          f_{y}(2,1) &= \lim_{y \to 1} \frac{f(2,y)-f(2,1)}{y-1} = \lim_{y \to 0} \frac{3\cdot
          2y^{2}-2\cdot 2^{3}-3\cdot 2\cdot 1^{2}+2\cdot 2^{3}}{y-1} = \lim_{y \to 1}
          \frac{6y^{2}-16-6+16}{y-1} \\ 
                     &= \lim_{y \to 1} \frac{6(y^{2}-1)}{y-1} = \lim_{y \to 1}
                     \frac{6(y-1)(y+1)}{y-1} = \lim_{y \to 1}[6(y+1)] = 12
      \end{align*}          
            \end{solution}
    \end{enumerate}
\end{examples}

\section{Συναρτήσεις Μερικών Παραγώγων}

Έστω $ f(x,y) $ συνάρτηση δύο μεταβλητών. 
\begin{myitemize}
\item Η μερική παράγωγος της $f$ ως προς $x$ υπολογίζεται 
    παραγωγίζοντας την συνάρτηση $ f(x,y) $ ως προς $x$, 
    θεωρώντας το $y$ σταθερό. 
\item Η μερική παράγωγος της $f$ ως προς $y$ υπολογίζεται 
    παραγωγίζοντας την συνάρτηση $ f(x,y) $ ως προς $y$, 
    θεωρώντας το $x$ σταθερό. 
\end{myitemize}

\begin{examples}
\item {}
    \begin{enumerate}
        \item Έστω $ f(x,y)=x^{2}y^{3}+4xy^{2}+4y+5 $. Να 
            υπολογιστούν οι μερικές παράγωγοι $ f_{x} $ και 
            $ f_{y} $.
            \begin{solution}
                \begin{align*}
                    f_{x} &= (x^{2}y^{3}+4xy^{2}+4y+5)_{x} =
                    (x^{2}y^{3})_{x}+(4xy^{2})_{x}+(4y)_{x}+(5)_{x} = 2xy^{3} + 4y^{2}
                    \intertext{και}
                    f_{y}&=(x^{2}y^{3}+4xy^{2}+4y+5)_{y} = 
                    (x^{2}y^{3})_{y}+(4xy^{2})_{y}+(4y)_{y}+(5)_{y} = 3x^{2}y^{2} + 8xy + 4
                 \end{align*} 
            \end{solution}
        \item Έστω $ f(x,y)=2x^{2}y+3 \cos{3y} +1 $. Να υπολογιστούν οι μερικές παράγωγοι $ f_{x}$
            και $ f_{y} $.
            \begin{solution}
                \[
                    f_{x}=4xy \quad \text{και} \quad f_{y}=2x^{2}-3 \sin{3y} (3y)_{y} = 2x^{2}-9 \sin{3y}
                \] 
            \end{solution}
            \item Έστω $ f(x,y,z)=x^{2}yz - y \cos{(xy)} $. Να
                υπολογιστούν οι μερικές παράγωγοι $ f_{x}, f_{y}, f_{z} $. 
                \begin{solution}
                \item {}
                    \begin{align*}
                        f_{x}&=2xyz- \cos{(xy)}(xy)_{x} = 2xyz-y \cos{xy} \\
                        f_{y}&=x^{2}z- \cos{xy}(xy)_{y}=x^{2}z - x \cos{xy} \\
                        f_{z}&=x^{2}z
                    \end{align*}
                \end{solution}
    \end{enumerate}
\end{examples}

\section{Μερικές Παράγωγοι Ανώτερης Τάξης}

\begin{example}
    Έστω $ f(x,y,z) = 3x^{2}y^{2} + xy^{3} + 3x +1 $. 
    Να υπολογιστούν οι μερικές παράγωγοι 1ης και 2ης τάξης.
    \begin{solution}
    \item {} 
        \begin{align*}
            \text{\bfseries 1ης τάξης:} \quad f_{x} &= 6xy^{2}+y^{3}+3 \quad \text{και} \quad 
            f_{y} = 6x^{2}y+3xy^{2} \\
            \text{\bfseries 2ης τάξης:} \quad f_{xx} &= (f_{x})_{x} = (6xy^{2}+y^{3}+3)_{x} =
            6y^{2} \\
            f_{yy} &= (f_{y})_{y} = (6x^{2}y+3xy^{2})_{y} = 
            6x^{2}+6xy \\
            f_{xy} &= (f_{x})_{y} = (6xy^{2}+y^{3}+3)_{y} = 
            12xy = 3y^{2} \tikzmark{a} \\
            f_{yx} &= (f_{y})_{x} = (6x^{2}y+3xy^{2})_{x} = 
            12xy+3y^{2} \; \; \, \tikzmark{b}
            \mybrace{a}{b}[\text{Μικτές Παράγωγοι}]
        \end{align*}
    \end{solution}
\end{example}

\begin{rem}
    Για τις μικτές παραγώγους $ f_{xy} $ και $ f_{yx} $ 
    ισχύει:
    \begin{align*}
        \pdv[2]{f}{x}{y} = \pdv{}{x} \left(\pdv{f}{y}\right) = \pdv{}{x} (f_{y}) = (f_{y})_{x} =
        f_{yx}
        \intertext{και}
        \pdv[2]{f}{y}{x} = \pdv{}{y} \left(\pdv{f}{x}\right) = \pdv{}{y} (f_{x}) = (f_{x})_{y} =
        f_{xy}
     \end{align*} 
\end{rem}

\begin{thm}[Schwarz]
\item {}
    Αν για τη συνάρτηση $ f(x,y) $ υπάρχουν οι μερικές παράγωγοι $ f_{xy} $ και $ f_{yx} $ και είναι συνεχείς σε μια περιοχή του σημείου $ (x_{0}, y_{0}) $, τότε $ f_{xy}=f_{yx} $ στην περιοχή αυτή.
\end{thm}

\begin{thm}
\item {}
    Αν για τη συνάρτηση $ f(x,y) $ υπάρχουν οι μερικές παράγωγοι $ f_{x}, f_{y} $ και $ f_{xy} $ και
    είναι συνεχείς σε μια περιοχή του σημείου $ (x_{0}, y_{0}) $, τότε υπάρχει και η $
    f_{yx} $ και
    ισχύει $ f_{xy}=f_{yx} $ στην περιοχή αυτή.
\end{thm}



\begin{rem}
    Οι πολυωνυμικές συναρτήσεις δύο (ή περισσότερων) μεταβλητών, 
    έχουν μερικές παραγώγους σε κάθε σημείο του $ \mathbb{R}^{2} $ (ή $\mathbb{R}^{n}$). Οι λοιπές
    στοιχειώδεις συναρτήσεις $ \sin{f(x,y)}, \cos{f(x,y)}, a^{f(x,y)}, \ln{f(x,y)} $ κ.λ.π. όπου $
    f(x,y) $ πολυωνυμική συνάρτηση, έχουν μερικές παραγώγους σε κάθε σημείο του πεδίου ορισμού τους.
    Για αυτές τις συναρτήσεις ισχύει $ f_{xy}=f_{yx} $.
\end{rem}

\begin{prop}
    Αν για μια συνάρτηση $ f(x,y)$ υπάρχουν οι μερικές παράγωγοι 1ης τάξης στο σημείο $ (x_{0},
    y_{0}) $ και είναι συνεχείς, τότε η συνάρτηση $f$ είναι συνεχής στο σημείο $ (x_{0}, y_{0}) $.
\end{prop}



\section{Μερική Ολοκλήρωση}

\begin{rem}
    Αν $ f_{x}(x,y) = g(x,y)$ και $ f_{y}(x,y)=h(x,y) $ τότε ισχύει:
    \begin{align*}
        f(x,y) = \int g(x,y) \,{dx} + c(y) \quad \text{και} \quad f(x,y) = \int h(x,y) \,{dy} + c(x) 
     \end{align*} 
\end{rem}

\begin{example}
    Έστω $ f(x,y)$ με $ f_{x}=e^{x+y} $ και $ f(0,y)=e^{y} $. Να βρεθεί ο τύπος της $f$.
    \begin{solution}
        \begin{align*}
            f(x,y) = \int e^{x+y} \,{dx} = e^{x+y} + c(y) \; \tikzmark{a} \\ 
            f(0,y) = e^{y} \Leftrightarrow e^{y}+ c(y) = e^{y} \Rightarrow c(y) = 0 \; \tikzmark{b}
         \end{align*}
         \mybrace{a}{b}[ $f(x,y) = e^{x+y}$ ]
    \end{solution}
\end{example}

\section{Ασκήσεις}

\begin{example}
    Να εξετάσετε την ύπαρξη των μερικών παραγώγων της συνάρτησης 
    \[
        f(x,y) = \begin{cases} x \frac{x^{2}-y^{2}}{x^{2}+y^{2}}, & (x,y) \neq (0,0) \\ 0, & (x,y) =
        (0,0) \end{cases}  
    \] 
    \begin{solution}
    \item {}
        \begin{myitemize}
        \item Αν $ (x,y) \neq (0,0) $ τότε 
            \[ f_{x} = \left(x
                \frac{x^{2}-y^{2}}{x^{2}+y^{2}}\right)_{x}
                = \cdots = \frac{x^{4}+4x^{2}y^{2}-y^{4}}{(x^{2}+y^{2})^{2}} \quad \text{και} \quad f_{y} = \left(x
                \frac{x^{2}-y^{2}}{x^{2}+y^{2}}\right)_{y} = \cdots - \frac{4x^{3}y}{(x^{2}+y^{2})^{2}}
            \]
        \item Αν $ (x,y) = (0,0) $ τότε εξετάζουμε τα όρια:
            \begin{align*}
                f_{x}(0,0) = \lim_{x \to 0} \frac{f(x,0)-f(0,0)}{x-0} = \lim_{x \to 0}
                \frac{x}{x} = \lim_{x \to 0} 1 = 1
                \intertext{και}
                f_{y}(0,0) = \lim_{y \to 0} \frac{f(0,y)-f(0,0)}{y-0} = \lim_{y \to 0} 0 = 0 
             \end{align*} 
             Άρα $ f_{x}= \begin{cases} \frac{x^{4}+4x^{2}y^{2}-y^{4}}{(x^{2}+y^{2})^{2}}, &(x,y)
                 \neq (0,0) \\ , &(x,y)=(0,0) \end{cases} $ και $ f_{y} = \begin{cases} -
                 \frac{4x^{3}y}{(x^{2}+y^{2})^{2}}, &(x,y) \neq (0,0) \\ 0, &(x,y)=(0,0)
             \end{cases}  $
        \end{myitemize}
    \end{solution}
\end{example}

\begin{rem}
    Η ύπαρξη των μερικών παραγώγων 1ης τάξης μιας συνάρτησης σε ένα σημείο, δεν σημαίνει ότι η
    συνάρτηση είναι συνεχής στο σημείο αυτό.
\end{rem}

\begin{example}
    Να εξετάσετε ως προς την ύπαρξη των μερικών παραγώγων και τη συνέχεια την συνάρτηση
    \[
        f(x,y) = \begin{cases} \frac{xy}{x^{2}+y^{2}}, &(x,y) \neq (0,0) \\ 0, &(x,y) = (0,0) \end{cases}  
    \]
    \begin{solution}
            \begin{align*}
                f_{x}(0,0) &= \lim_{x \to 0} \frac{f(x,0)-f(0,0)}{x-0} = \lim_{x \to 0}
                \frac{0}{x} = \lim_{x \to 0} 0 = 0 \quad \text{(υπάρχει)}
                \intertext{και}
                f_{y}(0,0) &= \lim_{y \to 0} \frac{f(0,y)-f(0,0)}{y-0} = \lim_{y \to 0}
                \frac{0}{y} = \lim_{y \to 0} = 0 \quad \text{(υπάρχει)}
                \intertext{όμως}
                \lim\limits_{\substack{x\to 0 \\y \to 0}} f(x,y) &= \lim\limits_{\substack{x\to 0
                \\y \to 0}} \frac{xy}{x^{2}+y^{2}} \overset{y=\lambda x}{=} \lim_{x \to 0}
                \frac{\lambda x^{2}}{x^{2}+ \lambda ^{2}x^{2}} = \lim_{x \to 0} \frac{\lambda
                x^{2}}{x^{2}(1+ \lambda ^{2})} = \lim_{x \to 0} \frac{\lambda}{1 + \lambda ^{2}} =
                \frac{\lambda}{1 + \lambda ^{2}} \quad \text{(δεν υπάρχει)}
             \end{align*} 
             Επομένως παρόλο που υπάρχουν οι μερικές παράγωγοι της συνάρτησης στο $ (0,0) $, η
             συνάρτηση δεν είναι συνεχής στο $ (0,9) $, 
             γιατί δεν υπάρχει το όριο της εκεί.
    \end{solution}
 
    \end{example}


\end{document}
