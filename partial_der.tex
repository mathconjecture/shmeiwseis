\documentclass[a4paper,12pt]{article}
\usepackage{etex}
%%%%%%%%%%%%%%%%%%%%%%%%%%%%%%%%%%%%%%
% Babel language package
\usepackage[english,greek]{babel}
% Inputenc font encoding
\usepackage[utf8]{inputenc}
%%%%%%%%%%%%%%%%%%%%%%%%%%%%%%%%%%%%%%

%%%%% math packages %%%%%%%%%%%%%%%%%%
\usepackage{amsmath}
\usepackage{amssymb}
\usepackage{amsfonts}
\usepackage{amsthm}
\usepackage{proof}

\usepackage{physics}

%%%%%%% symbols packages %%%%%%%%%%%%%%
\usepackage{dsfont}
\usepackage{stmaryrd}
%%%%%%%%%%%%%%%%%%%%%%%%%%%%%%%%%%%%%%%


%%%%%% graphicx %%%%%%%%%%%%%%%%%%%%%%%
\usepackage{graphicx}
\usepackage{color}
%\usepackage{xypic}
\usepackage[all]{xy}
\usepackage{calc}
%%%%%%%%%%%%%%%%%%%%%%%%%%%%%%%%%%%%%%%

\usepackage{enumerate}

\usepackage{fancyhdr}
%%%%% header and footer rule %%%%%%%%%
\setlength{\headheight}{14pt}
\renewcommand{\headrulewidth}{0pt}
\renewcommand{\footrulewidth}{0pt}
\fancypagestyle{plain}{\fancyhf{}
\fancyhead{}
\lfoot{}
\rfoot{\small \thepage}}
\fancypagestyle{vangelis}{\fancyhf{}
\rhead{\small \leftmark}
\lhead{\small }
\lfoot{}
\rfoot{\small \thepage}}
%%%%%%%%%%%%%%%%%%%%%%%%%%%%%%%%%%%%%%%

\usepackage{hyperref}
\usepackage{url}
%%%%%%% hyperref settings %%%%%%%%%%%%
\hypersetup{pdfpagemode=UseOutlines,hidelinks,
bookmarksopen=true,
pdfdisplaydoctitle=true,
pdfstartview=Fit,
unicode=true,
pdfpagelayout=OneColumn,
}
%%%%%%%%%%%%%%%%%%%%%%%%%%%%%%%%%%%%%%



\usepackage{geometry}
\geometry{left=25.63mm,right=25.63mm,top=36.25mm,bottom=36.25mm,footskip=24.16mm,headsep=24.16mm}

%\usepackage[explicit]{titlesec}
%%%%%% titlesec settings %%%%%%%%%%%%%
%\titleformat{\chapter}[block]{\LARGE\sc\bfseries}{\thechapter.}{1ex}{#1}
%\titlespacing*{\chapter}{0cm}{0cm}{36pt}[0ex]
%\titleformat{\section}[block]{\Large\bfseries}{\thesection.}{1ex}{#1}
%\titlespacing*{\section}{0cm}{34.56pt}{17.28pt}[0ex]
%\titleformat{\subsection}[block]{\large\bfseries{\thesubsection.}{1ex}{#1}
%\titlespacing*{\subsection}{0pt}{28.80pt}{14.40pt}[0ex]
%%%%%%%%%%%%%%%%%%%%%%%%%%%%%%%%%%%%%%

%%%%%%%%% My Theorems %%%%%%%%%%%%%%%%%%
\newtheorem{thm}{Θεώρημα}[section]
\newtheorem{cor}[thm]{Πόρισμα}
\newtheorem{lem}[thm]{λήμμα}
\theoremstyle{definition}
\newtheorem{dfn}{Ορισμός}[section]
\newtheorem{dfns}[dfn]{Ορισμοί}
\theoremstyle{remark}
\newtheorem{remark}{Παρατήρηση}[section]
\newtheorem{remarks}[remark]{Παρατηρήσεις}
%%%%%%%%%%%%%%%%%%%%%%%%%%%%%%%%%%%%%%%




\newcommand{\vect}[2]{(#1_1,\ldots, #1_#2)}
%%%%%%% nesting newcommands $$$$$$$$$$$$$$$$$$$
\newcommand{\function}[1]{\newcommand{\nvec}[2]{#1(##1_1,\ldots, ##1_##2)}}

\newcommand{\linode}[2]{#1_n(x)#2^{(n)}+#1_{n-1}(x)#2^{(n-1)}+\cdots +#1_0(x)#2=g(x)}

\newcommand{\vecoffun}[3]{#1_0(#2),\ldots ,#1_#3(#2)}



\everymath{\displaystyle}

\begin{document}

\chapter{Μερική Παράγωγος}

\section{Ορισμός}

\subsection{Δύο Μεταβλητών}

Έστω $ f \colon A \subseteq \mathbb{R}^{2} \to \mathbb{R} $
και $ (x_{0}, y_{0}) \in A $. Τότε η μερική παράγωγος της 
$f$ ως προς $x$ είναι:
\begin{align*}
  \eval{\pdv{f}{x}}_{(x_{0}, y_{0})} = \lim_{x \to x_{0}} 
  \frac{f(x, y_{0}) - f(x_{0}, y_{0})}{x - x_{0}} \overset{h=x- x_{0}}{=} 
  \lim_{h \to 0} \frac{f(x_{0}+h, y_{0}) - f(x_{0}, y_{0})}{h}  
  \intertext{και η μερική παράγωγος της $f$ ως προς $y$ είναι:}
  \eval{\pdv{f}{y}}_{(x_{0}, y_{0})} = \lim_{y \to y_{0}} 
  \frac{f(x_{0}, y) - f(x_{0}, y_{0})}{y - y_{0}} \overset{k=y- y_{0}}{=} 
  \lim_{k \to 0} \frac{f(x_{0}, y_{0}+k) - f(x_{0}, y_{0})}{k}  
\end{align*}

\subsection{Τριών Μεταβλητών}
Έστω $ f \colon A \subseteq \mathbb{R}^{3} \to \mathbb{R} $ και 
$ (x_{0}, y_{0}, z_{0}) \in A $.
Τότε η μερική παράγωγος της $f$ ως προς $x$ είναι :
\begin{align*}
  \eval{\pdv{f}{x}}_{(x_{0}, y_{0}, z_{0})} = \lim_{x \to x_{0}} 
  \frac{f(x, y_{0}, z_{0}) - f(x_{0}, y_{0}, z_{0})}{x - x_{0}} 
  \overset{h=x- x_{0}}{=} \lim_{h \to 0}
  \frac{f(x_{0}+h, y_{0}, z_{0}) - f(x_{0}, y_{0}, z_{0})}{h}  
  \intertext{και η μερική παράγωγος της $f$ ως προς $y$ είναι:}
  \eval{\pdv{f}{y}}_{(x_{0}, y_{0}, z_{0})} = \lim_{y \to y_{0}} 
  \frac{f(x_{0}, y, z_{0}) - f(x_{0}, y_{0}, z_{0})}{y - y_{0}} 
  \overset{k=y- y_{0}}{=} \lim_{k \to 0}
  \frac{f(x_{0}, y_{0}+k, z_{0}) - f(x_{0}, y_{0}, z_{0})}{k}  
  \intertext{και η μερική παράγωγος της $f$ ως προς $z$ είναι:}
  \eval{\pdv{f}{z}}_{(x_{0}, y_{0}, z_{0})} = \lim_{z \to z_{0}} 
  \frac{f(x_{0}, y_{0}, z) - f(x_{0}, y_{0}, z_{0})}{z - z_{0}} 
  \overset{s=z- z_{0}}{=} \lim_{s \to 0}
  \frac{f(x_{0}, y_{0}, z_{0}+s) - f(x_{0}, y_{0}, z_{0})}{s}  
\end{align*}

\section{Συμβολισμός}

Συμβολισμοί για τις μερικές παραγώγους της $f(x,y)$ ως 
προς $x$ και ως προς $y$ είναι:
\begin{align*}
  \eval{\pdv{f}{x} }_{(x_{0}, y_{0})} = \pdv{f(x_{0}, y_{0})}{x} = 
  f_{x}(x_{0}, y_{0}) = f'_{x}(x_{0}, y_{0} ) \quad \text{και} \quad
  \eval{\pdv{f}{y} }_{(x_{0}, y_{0})} = \pdv{f(x_{0}, y_{0})}{y} = 
  f_{y}(x_{0}, y_{0}) = f'_{y}(x_{0}, y_{0} ) 
\end{align*} 

\begin{example}
\item {}
  Δίνεται η $ f(x,y)=3xy^{2}-2x^{3} $. Να 
  υπολογιστούν με τον ορισμό οι $ f_{x}(0,1) $ και $ f_{y}(2,1) $.
  \begin{solution}
    \begin{align*}
      f_{x}(0,1) &= \lim_{x \to 0} \frac{f(x,1)-f(0,1)}{x-0} = 
      \lim_{x \to 0} \frac{3x1^{2}-2x^{3}-0}{x} = 
      \lim_{x \to 0} (3-2x^{2}) = 3
      \intertext{και} f_{y}(2,1) 
                 &= \lim_{y \to 1} \frac{f(2,y)-f(2,1)}{y-1} = 
                 \lim_{y \to 0} \frac{3\cdot 2y^{2}-2\cdot 2^{3}-3\cdot 
                 2\cdot 1^{2}+2\cdot 2^{3}}{y-1} = 
                 \lim_{y \to 1} \frac{6y^{2}-16-6+16}{y-1} \\ 
                 &= \lim_{y \to 1} \frac{6(y^{2}-1)}{y-1} = \lim_{y \to 1}
                 \frac{6(y-1)(y+1)}{y-1} = \lim_{y \to 1}[6(y+1)] = 12
    \end{align*}          
    Εναλλακτικά μπορούμε να χρησιμοποιήσουμε τα όρια 
    \begin{align*}
      f_{x}(0,1) &= \lim_{h \to 0} \frac{f(0+h,1)-f(0,1)}{h} = 
      \lim_{h \to 0} \frac{3(0+h)1^{2}-2(0+h)^{3}-0}{h} = 
      \lim_{h \to 0} \frac{3h-2h^{3}}{h} = \lim_{h \to 0} (3-2h^{2}) = 3 
      \intertext{και}
      f_{y}(2,1) &= \lim_{k \to 0} \frac{f(2,1+k)-f(2,1)}{k} = 
      \lim_{k \to 0} \frac{3\cdot 2(1+k)^{2}-2\cdot 2^{3}+10}{k} = 
      \lim_{k \to 0} \frac{6(1+k)^{2}-6}{k} 
      \overset{(\frac{0}{0})}{\underset{\text{L H}}{=}} 
      \lim_{k \to 0} \frac{12(1+k)}{1} = 12\!\!\!
    \end{align*}
  \end{solution}
\end{example}


\section{Συναρτήσεις Μερικών Παραγώγων}

Έστω $ f(x,y) $ συνάρτηση δύο μεταβλητών. 
\begin{myitemize}
  \item Η μερική παράγωγος της $f$ ως προς $x$ υπολογίζεται παραγωγίζοντας 
    την συνάρτηση $ f(x,y) $ ως προς $x$, θεωρώντας το $y$ σταθερό. 
  \item Η μερική παράγωγος της $f$ ως προς $y$ υπολογίζεται παραγωγίζοντας 
    την συνάρτηση $ f(x,y) $ ως προς $y$, θεωρώντας το $x$ σταθερό. 
\end{myitemize}

\section{Κανόνες Παραγώγισης}

Έστω $ f(x,y) $ και $ g(x,y) $ συναρτήσεις δύο μεταβλητών, που ορίζονται στο 
ανοικτό υποσύνολο $ A $ του $ \mathbb{R}^{2} $, και έστω ότι οι μερικές παράγωγοι 
$ \pdv{f}{x} $ και $ \pdv{g}{x} $ υπάρχουν για κάθε $ (x,y) \in A $, τότε:
\begin{myitemize}
  \item $ \pdv{x}(f+g) = \pdv{f}{x} + \pdv{g}{x} $
  \item $ \pdv{x}(af) = a \pdv{f}{x} $ 
  \item $ \pdv{x}(f\cdot g) = \pdv{f}{x} \cdot g + f \cdot \pdv{g}{x} $
  \item $ \pdv{x}(\frac{f}{g}) = \frac{\pdv{f}{x} \cdot g - f \cdot 
    \pdv{g}{x}}{g^{2}} $
\end{myitemize}

\begin{examples}
\item {}
  \begin{enumerate}
    \item Έστω $ f(x,y)=x^{2}y^{3}+4xy^{2}+4y+5 $. Να 
      υπολογιστούν οι μερικές παράγωγοι $ f_{x} $ και 
      $ f_{y} $.
      \begin{solution}
        \begin{align*}
          f_{x} &= (x^{2}y^{3}+4xy^{2}+4y+5)_{x} =
          (x^{2}y^{3})_{x}+(4xy^{2})_{x}+(4y)_{x}+(5)_{x} = 2xy^{3} + 4y^{2}
          \intertext{και}
          f_{y}&=(x^{2}y^{3}+4xy^{2}+4y+5)_{y} = 
          (x^{2}y^{3})_{y}+(4xy^{2})_{y}+(4y)_{y}+(5)_{y} = 3x^{2}y^{2} + 
          8xy + 4
        \end{align*} 
      \end{solution}
    \item Έστω $ f(x,y)=2x^{2}y+3 \cos{3y} +1 $. Να υπολογιστούν οι 
      μερικές παράγωγοι $ f_{x}$ και $ f_{y} $.
      \begin{solution}
        \[
          f_{x}=4xy \quad \text{και} \quad f_{y}=2x^{2}-3 \sin{3y} (3y)_{y} 
          = 2x^{2}-9 \sin{3y}
        \] 
      \end{solution}
    \item Έστω $ f(x,y,z)=x^{2}yz - y \cos{(xy)} $. Να υπολογιστούν οι 
      μερικές παράγωγοι $ f_{x}, f_{y}, f_{z} $. 
      \begin{solution}
      \item {}
        \begin{align*}
          f_{x}&=2xyz- \cos{(xy)}(xy)_{x} = 2xyz-y \cos{xy} \\
          f_{y}&=x^{2}z- \cos{xy}(xy)_{y}=x^{2}z - x \cos{xy} \\
          f_{z}&=x^{2}z
        \end{align*}
      \end{solution}
  \end{enumerate}
\end{examples}

Αν η συνάρτηση της οποίας θέλουμε να υπολογίσουμε τις μερικές παραγώγους είναι 
δίκλαδη, τότε εργαζόμαστε όπως στα παρακάτω παραδείγματα.

\begin{examples}
\item {}
  \begin{enumerate}

    \item Να υπολογίσετε τις μερικές παραγώγους 1ης τάξης της συνάρτησης.
      \[
        f(x,y) = 
        \begin{cases}
          x \frac{x^{2}-y^{2}}{x^{2}+y^{2}}, & (x,y) \neq (0,0) \\ 
          0, & (x,y) = (0,0) 
        \end{cases}
      \] 
      \begin{solution}
      \item {}
        \begin{myitemize}
          \item Αν $ (x,y) \neq (0,0) $ τότε 
            \[
              f_{x} = 
              \left( 
                x \frac{x^{2}-y^{2}}{x^{2}+y^{2}} 
              \right)_{x} = \cdots = 
              \frac{x^{4}+4x^{2}y^{2}-y^{4}}{(x^{2}+y^{2})^{2}} 
              \quad \text{και} \quad
              f_{y} = 
              \left(
                x \frac{x^{2}-y^{2}}{x^{2}+y^{2}} 
              \right)_{y} = \cdots - \frac{4x^{3}y}{(x^{2}+y^{2})^{2}}
            \]
          \item Αν $ (x,y) = (0,0) $ τότε εξετάζουμε τα όρια:
            \begin{align*}
              f_{x}(0,0) = \lim_{x \to 0} \frac{f(x,0)-f(0,0)}{x-0} = 
              \lim_{x \to 0} \frac{x}{x} = \lim_{x \to 0} 1 = 1
              \intertext{και}
              f_{y}(0,0) = \lim_{y \to 0} \frac{f(0,y)-f(0,0)}{y-0} = 
              \lim_{y \to 0} 0 = 0 
            \end{align*} 
            Άρα $ f_{x}= 
            \begin{cases}
              \frac{x^{4}+4x^{2}y^{2}-y^{4}}{(x^{2}+y^{2})^{2}}, &(x,y) 
              \neq (0,0) \\ 1 , &(x,y)=(0,0) 
            \end{cases}
            \quad \text{και} \quad f_{y} = 
            \begin{cases}
              - \frac{4x^{3}y}{(x^{2}+y^{2})^{2}}, &(x,y) \neq (0,0) \\ 
              0, &(x,y)=(0,0) 
            \end{cases}
            $  
        \end{myitemize}
      \end{solution}


    \item     Να υπολογίσετε τις μερικές παραγώγους 1ης τάξης της συνάρτησης 
      \[
        f(x,y) = 
        \begin{cases}
          x^{2} \sin{\frac{y}{x}}, & x \neq 0 \\
          0, & x = 0 
        \end{cases}
      \] 
      \begin{solution}
      \item {}
        \begin{myitemize}
          \item Αν $ x \neq 0 $ τότε: 

            Υπολογίζουμε τις μερικές παραγώγους της συνάρτησης στα σημεία 
            $ (x,y) $ με $ x \neq 0 $.
            \begin{align*}
              f_{x}(x,y) &= \left(x^{2} \sin{\frac{y}{x}}\right)_{x} = 2x 
              \sin{\frac{y}{x}} + x^{2} \cos{\frac{y}{x}} 
              \left(\frac{y}{x}\right)_{x} = 
              2x \sin{\frac{y}{x}} - y \cos{\frac{y}{x}} 
              \intertext{και}
              f_{y}(x,y) &= \left(x^{2} \sin{\frac{y}{x}}\right)_{y} = 
              x^{2} \cos{\frac{y}{x}} \left(\frac{y}{x}\right)_{y} = x 
              \cos{\frac{y}{x}} 
            \end{align*} 

          \item Αν $ x = 0 $ τότε: 

            Υπολογίζουμε την μερική παράγωγο ως προς $x$ στο σημείο 
            $ (0, y_{0}) $.
            \[
              f_{x}(0, y_{0}) = \lim_{x \to 0} \frac{f(x, y_{0}) - 
                f(0, y_{0})}{x-0} = \lim_{x \to 0} \frac{x^{2} 
              \sin{\frac{y_{0}}{x} - 0}}{x} = \lim_{x \to 0} x 
              \sin{\frac{y_{0}}{x}} = 0 
            \] 
            Υπολογίζουμε την μερική παράγωγο ως προς $y$ στο σημείο 
            $ (0, y_{0}) $.
            \[
              f_{y}(0, y_{0}) = \lim_{y \to 0} \frac{f(0,y)-
              f(0, y_{0})}{y-0} = \lim_{y \to 0} \frac{0-0}{y} = 
              \lim_{y \to 0} = 0 
            \] 
            Άρα $ f_{x}= 
            \begin{cases}
              2x \sin{\frac{y}{x}} - y \cos{\frac{y}{x}}, & x \neq 0 \\
              0, & x = 0
            \end{cases}
            \quad \text{και} \quad f_{y} = 
            \begin{cases}
              x \cos{\frac{y}{x}}, & x \neq 0 \\
              0, & x = 0
            \end{cases} $  
        \end{myitemize}
      \end{solution}
  \end{enumerate}
\end{examples}

\section{Μερικές Παράγωγοι Ανώτερης Τάξης}

\begin{example}
\item {}
  Έστω $ f(x,y,z) = 3x^{2}y^{2} + xy^{3} + 3x +1 $. 
  Να υπολογιστούν οι μερικές παράγωγοι 1ης και 2ης τάξης.
  \begin{solution}
  \item {} 
    \begin{align*}
      f_{x} &= 6xy^{2}+y^{3}+3 \quad \text{και} \quad 
      f_{y} = 6x^{2}y+3xy^{2} \\
      f_{xx} &= (f_{x})_{x} = (6xy^{2}+y^{3}+3)_{x} =
      6y^{2} \\
      f_{yy} &= (f_{y})_{y} = (6x^{2}y+3xy^{2})_{y} = 
      6x^{2}+6xy \\
      f_{xy} &= (f_{x})_{y} = (6xy^{2}+y^{3}+3)_{y} = 
      12xy = 3y^{2} \tikzmark{a} \\
      f_{yx} &= (f_{y})_{x} = (6x^{2}y+3xy^{2})_{x} = 
      12xy+3y^{2} \; \; \, \tikzmark{b}
      \mybrace{a}{b}[\text{Μικτές Παράγωγοι}]
    \end{align*}
  \end{solution}
\end{example}

\begin{rem}
\item {}
  Για τις μικτές παραγώγους $ f_{xy} $ και $ f_{yx} $ 
  ισχύει:
  \begin{align*}
    \pdv[2]{f}{x}{y} = \pdv{}{x} \left(\pdv{f}{y}\right) = \pdv{}{x} \left(f_{y}\right) 
    = (f_{y})_{x} = f_{yx}
    \intertext{και}
    \pdv[2]{f}{y}{x} = \pdv{}{y} \left(\pdv{f}{x}\right) = \pdv{}{y} \left(f_{x}\right) = 
    (f_{x})_{y} = f_{xy}
  \end{align*} 
\end{rem}

\begin{rem}
\item {}
  Οι πολυωνυμικές συναρτήσεις δύο (ή περισσότερων) μεταβλητών, 
  έχουν συνεχείς μερικές παραγώγους σε κάθε σημείο του $ \mathbb{R}^{2} $ 
  (ή $\mathbb{R}^{n}$).
  Οι λοιπές στοιχειώδεις συναρτήσεις $ \sin{f(x,y)}, \cos{f(x,y)}, a^{f(x,y)}, 
  \ln{f(x,y)} $ κ.λ.π. όπου $ f(x,y) $ πολυωνυμική συνάρτηση, έχουν 
  συνεχείς μερικές παραγώγους σε κάθε σημείο του πεδίου ορισμού τους.
  Για αυτές τις συναρτήσεις ισχύει $ f_{xy}=f_{yx} $.
\end{rem}

\begin{prop}
  Αν για μια συνάρτηση $ f(x,y)$ υπάρχουν οι μερικές παράγωγοι 1ης τάξης στο σημείο 
  $ (x_{0}, y_{0}) $ και είναι συνεχείς, τότε η συνάρτηση $f$ είναι συνεχής στο 
  σημείο $ (x_{0}, y_{0}) $.
\end{prop}

\begin{rem}
\item {}
  Η ύπαρξη των μερικών παραγώγων 1ης τάξης μιας συνάρτησης σε ένα σημείο, 
  δεν σημαίνει ότι η συνάρτηση είναι συνεχής στο σημείο αυτό.
\end{rem}

\begin{example}
\item {}
  Να εξετάσετε ως προς την ύπαρξη των μερικών παραγώγων και τη συνέχεια την συνάρτηση
  \[
    f(x,y) = 
    \begin{cases}
      \frac{xy}{x^{2}+y^{2}}, &(x,y) \neq (0,0) \\ 0, &(x,y) = (0,0) 
    \end{cases}
  \]
  \begin{solution}
    \begin{align*}
      f_{x}(0,0) &= \lim_{x \to 0} \frac{f(x,0)-f(0,0)}{x-0} = \lim_{x \to 0}
      \frac{0}{x} = \lim_{x \to 0} 0 = 0 \quad \text{(υπάρχει)}
      \intertext{και}
      f_{y}(0,0) &= \lim_{y \to 0} \frac{f(0,y)-f(0,0)}{y-0} = \lim_{y \to 0}
      \frac{0}{y} = \lim_{y \to 0} = 0 \quad \text{(υπάρχει)}
      \intertext{όμως}
      \lim\limits_{\substack{x\to 0 \\y \to 0}} f(x,y) &= 
      \lim\limits_{\substack{x\to 0 \\y \to 0}} \frac{xy}{x^{2}+y^{2}} 
      \overset{y=\lambda x}{=} \lim_{x \to 0} 
      \frac{\lambda x^{2}}{x^{2}+ \lambda ^{2}x^{2}} = 
      \lim_{x \to 0} \frac{\lambda x^{2}}{x^{2}(1+ \lambda ^{2})} = 
      \lim_{x \to 0} \frac{\lambda}{1 + \lambda ^{2}} =
      \frac{\lambda}{1 + \lambda ^{2}} \quad \text{(δεν υπάρχει)}
    \end{align*} 
    Επομένως παρόλο που υπάρχουν οι μερικές παράγωγοι της συνάρτησης στο 
    $ (0,0) $, η συνάρτηση δεν είναι συνεχής στο $ (0,9) $, 
    γιατί δεν υπάρχει το όριο της εκεί.
  \end{solution}
\end{example}



\begin{thm}[Schwarz]
\item {}
  Αν για τη συνάρτηση $ f(x,y) $ υπάρχουν οι μερικές παράγωγοι $ f_{xy} $ και 
  $ f_{yx} $ και είναι συνεχείς σε μια περιοχή του σημείου $ (x_{0}, y_{0}) $, τότε 
  $ f_{xy}=f_{yx} $ στην περιοχή αυτή.
\end{thm}

\begin{example}
  Να εξετάσετε αν ισχύει το θεώρημα Schwarz για την συνάρτηση 
  \[
    f(x,y) = 
    \begin{cases}
      xy \frac{x^{2}-y^{2}}{x^{2}+y^{2}}, &(x,y) \neq (0,0) \\
      0, & (x,y) = (0,0)
    \end{cases}
  \] 
\end{example}
\begin{solution}
\item {}
  Αν $ (x,y) \neq (0,0) $ τότε έχουμε
  \begin{align*}
    f_{x}(x,y) &= \left(xy\frac{x^{2}-y^{2}}{x^{2}+y^{2}}\right)_{x} = \cdots = 
    y\frac{x^{4}+4x^{2}y^{2}-y^{4}}{(x^{2}+y^{2})^{2}} 
    \quad \text{και} \quad
    f_{y}(x,y) &= \left(xy\frac{x^{2}-y^{2}}{x^{2}+y^{2}}\right)_{y} = \cdots = 
    x\frac{x^{4}-4x^{2}y^{2}-y^{4}}{(x^{2}+y^{2})^{2}} 
  \end{align*} 
  Αν $ (x,y) = (0,0) $ τότε έχουμε
  \[
    f_{x}(0,0) = \lim_{x \to 0} \frac{f(x,0)-f(0,0)}{x-0} = \lim_{x \to 0}
    \frac{0-0}{x} = 0 
    \quad \text{και} \quad
    f_{y}(0,0) = \lim_{y \to 0} \frac{f(0,y)-f(0,0)}{y-0} = \lim_{y \to 0} 
    \frac{0-0}{y} = 0 
  \]
  Επομένως οι μερικές παράγωγοι 1ης τάξης είναι 
  \[
    f_{x}(x,y) = 
    \begin{cases}
      \frac{x^{4}y+4x^{2}y^{3}-y^{5}}{(x^{2}+y^{2})^{2}}, & (x,y) \neq (0,0) \\
      0, & (x,y) = (0,0)
    \end{cases} \quad \text{και} \quad 
    f_{y}(x,y) = 
    \begin{cases}
      \frac{x^{5}-4x^{3}y^{2}-xy^{4}}{(x^{2}+y^{2})^{2}}, & (x,y) \neq (0,0) \\
      0, & (x,y) = (0,0)
    \end{cases}
  \] 
  Για τις παραγώγους 2ης τάξης, έχουμε:
  Αν $ (x,y) = (0,0) $ τότε έχουμε
  \begin{align*}
    f_{xy}(0,0) = \lim_{y \to 0} \frac{f_{x}(0,y)-f_{x}(0,0)}{y-0} = 
    \lim_{y \to 0} \frac{\frac{-y^{5}}{y^{4}} - 0}{y} = \lim_{y \to 0}
    \frac{-y}{y} = \lim_{y \to 0} -1 = -1
    \intertext{και}
    f_{yx}(0,0) = \lim_{x \to 0} \frac{f_{y}(x,0)-f_{x}(0,0)}{x-0} = 
    \lim_{x \to 0} \frac{\frac{x^{5}}{x^{4}} - 0}{x} = \lim_{x \to 0}
    \frac{x}{x} = \lim_{x \to 0} 1 = 1 
  \end{align*} 

  Επομένως, $ f_{xy}(0,0) \neq f_{yx}(0,0) $, άρα για τη συνάρτηση 
  $ f(x,y) $ δεν ισχύει το θεώρημα Schwarz.
\end{solution}


\section{Μερική Ολοκλήρωση}

\begin{rem}
\item {}
  Αν $ f_{x}(x,y) = g(x,y)$ και $ f_{y}(x,y)=h(x,y) $ τότε ισχύει:
  \begin{align*}
    f(x,y) = \int g(x,y) \,{dx} + c(y) \quad \text{και} \quad f(x,y) = 
    \int h(x,y) \,{dy} + c(x) 
  \end{align*} 
\end{rem}

\begin{example}
\item {}
  Έστω $ f(x,y)$ με $ f_{x}=e^{x+y} $ και $ f(0,y)=e^{y} $. Να βρεθεί ο τύπος της $f$.
  \begin{solution}
    \begin{align*}
      f(x,y) = \int e^{x+y} \,{dx} = e^{x+y} + c(y) \; \tikzmark{a} \\ 
      f(0,y) = e^{y} \Leftrightarrow e^{y}+ c(y) = e^{y} \Rightarrow c(y) = 0 
      \; \tikzmark{b}
    \end{align*}
    \mybrace{a}{b}[ $f(x,y) = e^{x+y}$ ]
  \end{solution}
\end{example}

\begin{rem}
  Όταν ολοκληρώνουμε μια συνάρτηση πολλών μεταβλητών, ως προς κάποια από τις 
  μεταβλητές, τότε η σταθερά ολοκλήρωσης είναι συνάρτηση των υπολοίπων μεταβλητών, 
  οι οποίες θεωρούνται σταθερές κατά την ολοκλήρωση.
\end{rem}


\chapter{Διαφορισιμότητα}

\section{Ορισμός}
\begin{dfn}
\item {}
  Μια συνάρτηση $ f(x,y) $ λέγεται διαφορίσιμη στο σημείο 
  $ P_{0}(x_{0}, y_{0}) $ αν ισχύει η παρακάτω συνθήκη
  διαφορισιμότητας: 
  \[
    \eval{\Delta f(x,y)}_{P_{0}} \!\!= f(x,y)-f(x_{0}, y_{0}) 
    \xlongequal[k=\underbrace{y- y_{0}}_{\Delta y=dy}]{h=\overbrace{x-
    x_{0}}^{\Delta x=dx}}  f(x_{0}+h, y_{0}+k)-f(x_{0}, y_{0}) =
    \underbrace{\eval{\pdv{f}{x}} _{P_{0}}\cdot h + 
    \eval{\pdv{f}{y} }_{P_{0}}\cdot k}_
    {\minibox[c]{$\eval{df(x,y)}_{P_{0}}$ \\ Πρωτεύον
    μέρος: \\ Γραμμικό ως προς $h,k$}} + \!\!\!\!  
    \underbrace{\varepsilon _{1}(h,k)h+ 
      \varepsilon _{2}(h,k)k}_{\minibox[c]{$G(h,k)$ : (πολύ μικρό) 
    \\ Δευτερεύον μέρος: \\ Μη γραμμικό ως προς $h,k$}}                 
  \]
\end{dfn}

\begin{rem}
\item {}
  Η συνάρτηση $ G(h,k) = \eval{\Delta f(x,y)}_{P_{0}} - \eval{df(x,y)}_{P_{0}} $
\end{rem}

\begin{dfn}[Επαναδιατύπωση]
\item {}
  Μια συνάρτηση $ f(x,y) $ είναι διαφορίσιμη σε ένα σημείο $ P_{0}(x_{0}, y_{0}) $ αν
  και μόνον αν
  \begin{enumerate}[i)]
    \item Υπάρχουν οι μερικές παράγωγοι $ \eval{\pdv{f}{x}}_{P_{0}},
      \eval{\pdv{f}{y}}_{P_{0}} $
    \item $ \lim\limits_{\substack{h\to 0 \\k \to 0}} \varepsilon _{1}(h,k) = 
      0 \quad \text{και} \quad \lim\limits_{\substack{h\to 0 \\k \to 0}} 
      \varepsilon _{2}(h,k)=0 \overset{\text{ή}}{\Leftrightarrow} 
      \lim\limits_{\substack{h\to 0 \\k \to 0}} 
      \frac{G(h,k)}{\sqrt{h^{2}+k^{2}}} = 0 $.
  \end{enumerate}
\end{dfn}

\begin{prop}
  Κάθε διαφορίσιμη συνάρτηση είναι και συνεχής.
\end{prop}

\begin{rem}
\item {}
  \begin{enumerate}[i)]
    \item Το αντίστροφο της παραπάνω πρότασης δεν ισχύει.
    \item Το αντιθετοαντίστροφο της πρότασης μας λέει, ότι αν μια συνάρτηση, 
      δεν είναι συνεχής, τότε δεν είναι διαφορίσιμη.
  \end{enumerate}
\end{rem}
\begin{thm}
\item {}
  Μια συνάρτηση $ f(x,y) $ είναι διαφορίσιμη στο $ P_{0}(x_{0}, y_{0}) $ 
  αν υπάρχουν οι μερικές παράγωγοι και είναι συνεχείς στο $ P_{0} $.
\end{thm}

\newpage

\begin{example}
  Να εξεταστεί ως προς τη διαφορισιμότητα στο σημείο $(0,0)$ 
  η συνάρτηση 
  \[ f(x,y) = \begin{cases} \frac{xy}{x^{2}+y^{2}}, &(x,y) \neq (0,0) \\ 0, & (x,y)
  = (0,0) \end{cases} \]
  \begin{solution}
  \item {}
    Έχουμε
    $ \lim\limits_{\substack{x\to 0 \\y \to 0}} f(x,y) =
    \lim\limits_{\substack{x\to 0 \\y \to 0}} \frac{xy}{x^{2}+y^{2}} =
    \lim_{r \to 0} \frac{r^{2} \cos{\theta} \sin{\theta}}{r^{2}} =
    \cos{\theta} \sin{\theta} $, άρα δεν υπάρχει το όριο, επομένως η συνάρτηση
    δεν είναι συνεχής, άρα ούτε κ παραγωγίσιμη στο $ (0,0) $.
  \end{solution}
\end{example}

\begin{example}
\item Να εξεταστεί ως προς τη διαφορισιμότητα στο σημείο $(0,0)$ 
  η συνάρτηση 
  \[
    f(x,y) = \begin{cases} x \frac{x^{2}-y^{2}}{x^{2}+y^{2}}, &(x,y) \neq (0,0)
    \\ 0, & (x,y) = (0,0)  \end{cases} 
  \] 
  \begin{solution}
  \item {}
    \begin{description}
      \item [Α΄ Τρόπος: (Με το θεώρημα)]
      \item {}
        Εξετάζουμε την ύπαρξη των μερικών παραγώγων της $f$ στο $ (0,0) $.
        \begin{myitemize}
          \item Αν $ (x,y) \neq (0,0) $ έχουμε:
            \[
              f_{x} = \left(x \frac{x^{2}-y^{2}}{x^{2}+y^{2}}\right)_
              {x} = \cdots = \frac{x^{4}+4x^{2}y^{2}-y^{4}}{(x^{2}+
              y^{2})^{2}} \quad \text{και} \quad f_{y} = 
              \left(x\frac{x^{2}-y^{2}}{x^{2}+y^{2}}\right)_{y} = 
              \cdots = - \frac{4x^{3}y}{(x^{2}+y^{2})^{2}} 
            \] 

          \item Αν $ (x,y) = (0,0) $ έχουμε:
            \begin{align*}
              f_{x}(0,0) = \lim_{x \to 0} \frac{f(x,0)-f(0,0)}{x-0} = 
              \lim_{x \to 0} = \lim_{x \to 0} \frac{x}{x} = 1
              \quad \text{και} \quad
              f_{y}(0,0) = \lim_{y \to 0} \frac{f(0,y)-f(0,0)}{y-0} = 
              \lim_{y \to 0} 0 = 0
            \end{align*} 
            Επομένως υπάρχουν οι μερικές παράγωγοι της $f$ στο $ (0,0) $
            και έχουμε:
            \[
              f_{x}(x,y) = 
              \begin{cases}
                \frac{x^{4}+4x^{2}y^{2}-y^{4}}{(x^{2}+y^{2})^{2}}, 
                                    & (x,y) \neq (0,0) \\ 1, & (x,y) = (0,0) 
              \end{cases} \quad \text{και} \quad
              f_{y}(x,y) = \begin{cases} - 
                \frac{4x^{3}y}{(x^{2}+y^{2})^{2}}, &(x,y)
              \neq (0,0) \\ 0, & (x,y) = (0,0) \end{cases} 
            \]
        \end{myitemize}
        Εξετάζουμε ως προς τη συνέχεια τις μερικές παραγώγους.
        \begin{myitemize}
          \item Για την $ f_{x} $
            \begin{align*}
              L_{1} = \lim_{x \to 0} 
              \left(\lim_{y \to 0}
                \frac{x^{4}+4x^{2}y^{2}-y^{4}}{(x^{2}+y^{2})^{2}}
              \right) = 
              \lim_{x \to 0} 1 = 1
              \intertext{και}
              L_{2}= \lim_{y \to 0} \left(\lim_{x \to 0}
              \frac{x^{4}+4x^{2}y^{2}-y^{4}}{(x^{2}+y^{2})^{2}}\right) = 
              \lim_{y \to 0} (-1) = -1
            \end{align*}
          \item Για την $ f_{y} $
            \begin{align*}
              \lim\limits_{\substack{x\to 0 \\y \to 0}} -
              \frac{4x^{3}y}{(x^{2}+y^{2})^{2}} = \lim_{r \to 0} -
              \frac{4r^{4} \cos^{3}{\theta} \sin{\theta}}{(r^{2})^{2}} = 
              \lim_{r \to 0} (-4 \cos^{3}{\theta} \sin{\theta}) = 
              -4 \cos^{3}{\theta} \sin{\theta} \quad \text{(δεν υπάρχει)}
            \end{align*}
        \end{myitemize}

        Επομένως καμία από τις μερικές παραγώγους της $f$ δεν είναι συνεχής στο
        $(0,0)$, άρα η $f$ δεν είναι διαφορίσιμη σύμφωνα με το θεώρημα.
      \item [Β᾽ Τρόπος: (Με ορισμό)]
      \item {}
        Εξετάζουμε με τον ορισμός την ύπαρξη των μερικών παραγώγων της $f$
        και αν η $ G(h,k) $ είναι πολύ μικρή.
        \begin{enumerate}[i)]
          \item Υπάρχουν οι μερικές παράγωγοι της $f$ στο $ (0,0) $.
          \item Εύρεση της $ G(h,k) $.
            \[
              G(h,k) = \Delta f(0,0) - df(0,0) 
            \] 
        \end{enumerate}
        \begin{myitemize}
          \item $ \Delta f(0,0) = f(0+h,0+k) - f(0,0) = 
            h \frac{h^{2}-k^{2}}{h^{2}+k^{2}} - 0 = h
            \frac{h^{2}-k^{2}}{h^{2}+k^{2}} $
          \item $ df(0,0) = f_{x}(0,0)h+f_{y}(0,0)k = 1h+0k= h $
        \end{myitemize}
        Επομένως $ G(h,k) = h \frac{h^{2}-k^{2}}{h^{2}+k^{2}} - h =
        - \frac{2hk^{2}}{h^{2}+k^{2}}$
        \[
          \lim\limits_{\substack{h\to 0 \\k \to 0}} 
          \frac{G(h,k)}{\sqrt{h^{2}+k^{2}}} =
          \lim\limits_{\substack{h\to 0 \\k \to 0}} 
          \frac{-\frac{2hk^{2}}{h^{2}+k^{2}}}{\sqrt{h^{2}+k^{2}}}
          = \lim\limits_{\substack{h\to 0 \\k \to 0}}-
          \frac{2hk^{2}}{(h^{2}+k^{2})^{\frac{3}{2}}} =
          -2 \lim_{r \to 0} \frac{4r^{3} \cos{\theta}
          \sin^{2}{\theta}}{(r^{2})^{\frac{3}{2}}} =
          -2 \cos{\theta} \sin^{2}{\theta}  
        \]
        Επομένως η $f$ δεν είναι διαφορίσιμη στο $ (0,0) $.  
    \end{description}
  \end{solution}
\end{example}

\begin{example}
  Να εξετάσετε ως προς τη διαφορισιμότητα τη συνάρτηση:
  \[
    f(x,y) = 
    \begin{cases}
      y^{2} \sin{\frac{x}{y}}, &y \neq 0 \\0, &y=0 
    \end{cases}
  \]
  \begin{solution}
    Ελέγχουμε την ύπαρξη των μερικών παραγώγων της $f$.
    \begin{myitemize}
      \item Για $ y \neq 0 $ η $f$ είναι διαφορίσιμη ως σύνθεση διαφορίσιμων
        συναρτήσεων και έχουμε: 
        \[
          f_{x} = y \cos{\frac{x}{y}}  \quad \text{και} \quad  f_{y} = 2y
          \sin{\frac{x}{y}} - x \cos{\frac{x}{y}} 
        \]
      \item Για $ y = 0 $, έχουμε
        \begin{align*}
          f_{x}(x,0) = \lim_{h \to 0} \frac{f(x+h,0)-f(x,0)}{h} = 
          \lim_{h \to 0} \frac{0 - 0}{h} = \lim_{h \to 0} 0 = 0
          \intertext{και}
          f_{y}(x,0) = \lim_{k \to 0} \frac{f(x,0+k)-f(x,0)}{k} = \lim_{k \to
          0} \frac{k^{2} \sin{\frac{x}{k}} - 0}{k} = \lim_{k \to 0}
          k \sin{\frac{x}{k}} = 0
        \end{align*}
    \end{myitemize}
    Επομένως υπάρχουν οι μερικές παράγωγοι της $ f $.  

    Ελέγχουμε τη συνέχεια της $ f_{x} $ στο $ y=0 $. Παρατηρούμε:
    \[
      f_{x}(x,y) = 
      \begin{cases}
        y \cos{\frac{x}{y}}, & y \neq 0 \\ 0, & y=0 
      \end{cases}
    \] 
    είναι συνεχής στο $ y=0 $, γιατί 
    $ \lim_{y \to 0} y \cos{\frac{x}{y}} = 0 = f_{x}(0,0) $. 
    Άρα η $f$ είναι διαφορίσιμη.
  \end{solution}
\end{example}


\chapter{Διαφορικά}

\section{Ολικό Διαφορικό}

\begin{dfn}
  Έστω η συνάρτηση $ f(x,y) $. Τα \textcolor{Col1}{ολικά διαφορικά 1ης και 
  2ης τάξης} της συνάρτηση $f$ συμβολίζονται με $ df $ και $ d^{2}f $, αντίστοιχα 
  και ισχύει:
  \[
    \boxed{df = f_{x}dx + f_{y}dy} \quad \text{και} \quad 
    \boxed{d^{2}f = f_{xx}dx^{2}+2f_{xy}dxdy+f_{yy}dy^{2}}
  \] 
  Στην περίπτωση όπου $ f= f(x_{1}, x_{2}, \ldots, x_{n}) $, το ολικό 
  διαφορικό 1ης τάξης γίνεται: 
  \[
    df = f_{x_{1}}d{x_{1}} + f_{x_{2}}d{x_{2}} + \cdots f_{x_{n}} dx_{n}
  \]
\end{dfn}


% \begin{dfn}
% \item {}
%   Η συνάρτηση $ \pdv{f}{x}\cdot h + \pdv{f}{y} \cdot k = f_{x} \Delta x 
%   + f_{y} \Delta y $ ονομάζεται ολικό διαφορικό της συνάρτησης $f$ 
%   και συμβολίζεται με $ df $. Οπότε 
%   \[
%     df = f_{x} \Delta x + f_{y} \Delta y \quad \text{ή} 
%     \quad df = f_{x}dx + f_{y}dy 
%   \] 
%   Στην περίπτωση όπου $ f= f(x_{1}, x_{2}, \ldots, x_{n}) $, έχουμε
%   \[
%     df = f_{x_{1}}d{x_{1}} + f_{x_{2}}d{x_{2}} + \cdots f_{x_{n}} dx_{n}
%   \]
% \end{dfn}

\begin{example}
  Να βρείτε το ολικό διαφορικό της συνάρτησης $ f(x,y) = xye^{x+2y} $ 
  \begin{solution}
  \item {}
    Το ολικό διαφορικό δίνεται από τη σχέση $ df = f_{x} dx + f_{y} dy $.  
    Για τις μερικές παραγώγους έχουμε ότι: 
    \[
      f_{x} = ye^{x+2y}+xye^{x+2y} \quad \text{και} \quad f_{y} = xe^{x+2y} +
      2xye^{x+2y}
    \] 
    Επομένως
    \[
      df = y(1+x)e^{x+2y} dx + x(1+2y)e^{x+2y}dy
    \]
  \end{solution}
\end{example}

\section{Εφαρμογές του Διαφορικού}

\begin{example}
  Να υπολογίσετε το $ \Delta f $ και το $ df $ στο σημείο $ (x,y) = (1,1) $, 
  για τη συνάρτηση $ f(x,y) = x^{3}+y^{2} $ και για 
  \begin{enumerate}[i)]
    \item $ \Delta x = 0,1, \; \Delta y = 0,1 $
    \item $ \Delta x = 0,01, \; \Delta y = 0,01 $
  \end{enumerate}
  Τι παρατηρείτε?
\end{example}
\begin{solution}
\item {}
  \begin{enumerate}[i)]
    \item 
      \begin{align*} 
        \Delta f &= f(x+ \Delta x, y + \Delta y) - f(x,y) = f(1+0.1,1+0.1) - 
        f(1,1) = f(1.1,1.1) - f(1,1) \\ 
                 &= [(1.1)^{3}+(1.1)^{2}] - (1^{3}+1^{2}) = 0.541 \\
        df &= f_{x}dx + f_{y}dy  = 3x^{2} dx + 2y dy \Rightarrow df (1,1) = 
        3\cdot 1^{2} \cdot 0.1 + 2 \cdot 1 \cdot 0.1 = 0.5 
      \end{align*}
    \item 
      \begin{align*} 
        \Delta f &= f(x+ \Delta x, y + \Delta y) - f(x,y) = f(1+0.01,1+0.01) - 
        f(1,1) = f(1.01,1.01) - f(1,1) \\ 
                 &= [(1.01)^{3}+(1.01)^{2}] - (1^{3}+1^{2}) = 0.0504 \\
        df &= f_{x}dx + f_{y}dy  = 3x^{2} dx + 2y dy \Rightarrow df (1,1) = 
        3\cdot 1^{2} \cdot 0.01 + 2 \cdot 1 \cdot 0.01 = 0.05 
      \end{align*}
  \end{enumerate}
  Παρατηρούμε ότι στην 1η περίπτωση, έχουμε
  $ \Delta f - df \approx 0.04 $, ενώ στη 2η περίπτωση έχουμε $ \Delta f - df \approx 
  0,0004 $, δηλαδή στη 2η περίπτωση, όπου τα $ \Delta x $ και $ \Delta y $ επιλέχθηκαν 
  να είναι μικρότερα, το διαφορικό προσεγγίζει με μεγαλύτερη ακρίβεια τη διαφορά 
  $ \Delta f $.
\end{solution}

\begin{example}%todo σχήμα (τρίγωνο) για την άσκηση. Δες λυμένα θέματα ηλεκτρ.
  Κατά τη μέτρηση των στοιχείων μιας τριγωνικής επιφάνειας, βρέθηκαν οι πλευρές να 
  έχουν μήκος \SI{100}{cm} και \SI{125}{cm} αντίστοιχα, ενώ η περιεχόμενη γωνία βρέθηκε 
  να είναι \SI{60}{\degree}. Αν το πιθανό σφάλμα κατά την μέτρηση των πλευρών είναι 
  $ \SI{0.2}{cm} $ και για τις γωνίες είναι $ \SI{1}{\degree} $, να υπολογίσετε κατά 
  προσέγγιση το σφάλμα στο εμβαδό της επιφάνειας.
\end{example}
\begin{solution}
  Έχουμε ότι $ E = \frac{y \cdot z}{2} $. Όμως από το ορθογώνιο τρίγωνο έχουμε ότι 
  $ z = x \sin{\phi} $, άρα 
  \[
    E(x,y,\phi) = \frac{y \cdot x \cdot \sin{\phi}}{2} 
  \] 
  Παρατηρούμε ότι $ \Delta x = dx = 0.2, \; \Delta y = dy = 0.2 $ και 
  $ \Delta \phi = d\phi = \SI{1}{\degree} \frac{\pi}{180} $. Οπότε,ισχύει ότι
  \begin{equation*}
    \Delta E \approx dE = E_{x} dx + E_{y} dy + E_{\phi} d\phi = \frac{y \cdot
    \sin{\phi}}{2} dx + \frac{x \cdot \sin{\phi}}{2} dy + 
    \frac{y \cdot x \cdot \cos{\phi}}{2} d\phi
  \end{equation*} 
  και πιο συγκεκριμένα ισχύει ότι 
  \begin{align*}
    \Delta E(100,125,60) \approx dE(100,125,60) 
  &= E_{x} dx + E_{y} dy + E_{\phi} d\phi \\
  &= E_{x}(100,125,60) dx + E_{y}(100,125,60) dy + E_{\phi}(100,125,60) d\phi \\
  &= \frac{1}{2} \left(125 \cdot \sin{\frac{\pi}{3}} \cdot 0.2 + 100 \cdot
    \sin{\frac{\pi}{3}} \cdot 0.2 + 100 \cdot 125 \cdot \cos{\frac{\pi}{3}} \cdot
  \frac{\pi}{180}\right) \\
  &= \SI{74.03}{cm^{2}}
  \end{align*}
  Άρα το σφάλμα μέτρησης του εμβαδού της τριγωνικής επιφάνειας είναι 
  $ \SI{74.03}{cm^{2}} $
\end{solution}

\begin{example}%todo σχήμα
  Κατά την εκτίμηση των συντεταγμένων ενός σημείου $ P(x,y) $ γίνονται σφάλματα 
  $ dx $ και $ dy $ αντίστοιχα. Να υπολογίσετε την επίδραση αυτών των σφαλμάτων 
  στη γωνία $ \theta $ και να γίνει αριθμητική εφαρμογή για $ x=1 $, $ y=2 $ και για 
  $ dx = 0.01 $ και $ dy =-0.02 $.
\end{example}
\begin{solution}
  Έστω οι συντεταγμένες του σημείου $ P'(x+ dx, y + dy) $, όπως φαίνονται στο σχήμα.
  Έχουμε ότι 
  \[
    \tan{\theta} = \frac{y}{x} \Rightarrow \theta = \arctan{\frac{y}{x}}
  \] 
  Αν η γωνία μετά το σφάλμα της εκτίμησης είναι $ \theta'$, έχουμε ότι 
  \[
    \theta ' = \arctan{\frac{y+ dy}{x + dx}} 
  \] 
  Παρατηρούμε ότι $ \Delta x = dx = 0.01 $, $ \Delta y = dy = -0.02 $. Οπότε ισχύει ότι 
  \[
    \Delta \theta \approx d \theta = \theta _{x} dx + \theta _{y} dy =
    \frac{-y}{x^{2}+y^{2}} dx + \frac{x}{x^{2}+y^{2}} dy = \frac{xdy-ydx}{x^{2}+y^{2}}
  \] 
  και μάλιστα υπολογίζουμε την απόλυτη τιμή και έχουμε
  \[
    \abs{\Delta \theta} \approx \frac{\abs{xdy-ydx}}{x^{2}+y^{2}} \leq \frac{\abs{x dy}
    + \abs{y dx}}{x^{2}+y^{2}} 
  \]
  και πιο συγκεκριμένα, για το σημείο $ P(1,2) $ έχουμε
  \[
    \abs{\Delta \theta(1,2)} \leq \frac{\abs{1(-0.02)}+\abs{2\cdot 0.01}}{1^{2}+2^{2}} =
    \frac{0.04}{5} = 0.008
  \]
\end{solution}

\section{Τέλειο Διαφορικό}

\mydfn{Θεωρούμε τις συναρτήσεις $ P(x,y) $ και $ Q(x,y) $ με πεδίο ορισμού $ A \subseteq
  \mathbb{R}^{2} $.
  Η παράσταση $ P(x,y) dx + Q(x,y) dy $ λέγεται \textcolor{Col2}{τέλειο 
  διαφορικό} αν υπάρχει συνάρτηση $ f(x,y) $ με πεδίο ορισμού το $A$, ώστε 
  \begin{gather*}
    df = P(x,y)dx + Q(x,y)dy \Leftrightarrow \pdv{f}{x} dx + \pdv{f}{y} dy = 
    P(x,y)dx + Q(x,y)dy \Leftrightarrow \\
    \boxed{\pdv{f}{x} = P(x,y) \quad \text{και} \quad \pdv{f}{y} = Q(x,y)}
\end{gather*}}


\begin{prop}
  Αν οι  $ P(x,y) $  και  $ Q(x,y) $  είναι συνεχείς συναρτήσεις και έχουν συνεχείς 
  παραγώγους πρώτης τάξης, σε μια ορθογώνια περιοχή $A$ του $ \mathbb{R}^{2} $,  
  τότε η  παράσταση  $ P(x,y)dx + Q(x,y)dy $ είναι τέλειο διαφορικό αν 
  \[
    \boxed{\pdv{p}{y} = \pdv{q}{x}} \quad \forall (x,y) \in A
  \]
\end{prop}

\begin{dfn}
  Η παράσταση  $ P(x,y,z)dx + Q(x,y,z)dy + R(x,y,z)dz $ είναι τέλειο διαφορικό 
  αν υπάρχει συνάρτηση  $ f(x,y,z) $  τέτοια ώστε  $ df = P(x,y,z)dx + Q(x,y,z)dy 
  + R(x,y,z)dz $.  Τότε ισχύουν οι παρακάτω σχέσεις:
  \[
    \boxed{\pdv{f}{x} = P(x,y,z) \quad \text{και} \quad \pdv{f}{y} = Q(x,y,z) 
    \quad \text{και} \quad \pdv{f}{z} = R(x,y,z)} 
  \] 
\end{dfn}

\begin{prop}
  Αν οι  $ P(x,y,z) $, $ Q(x,y,z) $  και  $ R(x,y,z) $ είναι συνεχείς συναρτήσεις 
  και έχουν συνεχείς παραγώγους πρώτης τάξης, σε μια ορθογώνια περιοχή Α του 
  $ \mathbb{R}^{3} $ τότε η  παράσταση 
  $ P(x,y,z)dx + Q(x,y,z)dy + R(x,y,z)dz $   είναι τέλειο διαφορικό αν 
  \[
    \boxed{\pdv{P}{y} = \pdv{Q}{x}} \quad \text{και} \quad \boxed{\pdv{Q}{z} = 
    \pdv{R}{y}} \quad \text{και} \quad  \boxed{\pdv{P}{z} = \pdv{R}{x}} 
    \quad \forall (x,y,z) \in A 
  \] 
\end{prop}

\begin{rem}\label{olokl}
  Οι συναρτήσεις  $ f(x,y) $  και  $ f(x,y,z) $ υπολογίζονται επίσης από τις 
  παρακάτω σχέσεις:
  \begin{align*}
    f(x,y) &= \int_{x_{0}}^{x} P(t,y) \,{dt} + \int_{y_{0}}^{y} Q(x_{0},t) \,{dt} \\
    f(x,y,z) &= \int_{x_{0}}^{x} P(t,y,z) \,{dt} + \int_{y_{0}}^{y} Q(x_{0},t,z) 
    \,{dt} + \int _{z_{0}}^{z} R(x_{0},y_{0},t) \,{dt}  
  \end{align*}
  όπου τα $ x_{0} $, $ y_{0} $  και  $ z_{0} $ επιλέγονται \textbf{αυθαίρετα} στο πεδίο 
  ορισμού των  $ P $, $ Q $  και  $ R $.
\end{rem}

\begin{rem}
  Η συνάρτηση $ f(x,y) $ ή η συνάρτηση $ f(x,y,z) $ του ορισμού του τέλειου διαφορικού
  λέγεται \textcolor{Col2}{συνάρτηση δυναμικού}.
\end{rem}

\begin{example}
  Να εξετάσετε αν η παράσταση $ \left(1+x- \frac{2}{y}\right)dx + 
  \left(1+ \frac{2x}{y^{2}} \right)dy $ είναι τέλειο διαφορικό και αν ναι, να 
  υπολογίσετε τη συνάρτηση δυναμικού.
\end{example}
\begin{solution}
  Ελέγχουμε με το κριτήριο:
  \[ 
    \pdv{P}{y} = \frac{2}{y^{2}} = \pdv{Q}{x} 
  \]
  Άρα η παράσταση είναι τέλειο διαφορικό. Επομένως υπάρχει 
  συνάρτηση, $ f(x,y) $ τέτοια ώστε: 
  \begin{align}
    \pdv{f}{x} &= 1 + x - \frac{2}{y} \label{fx1} \\
    \pdv{f}{y} &= 1+ \frac{2x}{y^{2}} \label{fy1}
  \end{align}
  Ολοκληρώνουμε μερικώς ως προς $x$ την~\eqref{fx1} και έχουμε
  \[
    f(x,y) = \int \left(1+x- \frac{2}{y}\right) \,{dx} = x + 
    \frac{x^{2}}{2} - \frac{2x}{y} + c(y) 
  \] 
  Άρα  
  \begin{equation}
    f(x,y) = x + \frac{x^{2}}{2} - \frac{2x}{y} + c(y) \label{fxy}
  \end{equation}
  Στη συνέχεια παραγωγίζουμε μερικώς ως προς $y$ τη συνάρτηση $ f(x,y) $ που μόλις 
  βρήκαμε και έχουμε:
  \begin{equation}
    \pdv{f}{y} = \frac{2x}{y^{2}} + c'(y) \label{ffy}
  \end{equation} 
  Στη συνέχεια εξισώνουμε την~\eqref{ffy} με την~\eqref{fy1} και προκύπτει
  \[
    c'(y) = 1 \Rightarrow c(y) = y + k 
  \] 
  Άρα, τελικά η συνάρτηση δυναμικού είναι 
  \[
    f(x,y) = x + \frac{x^{2}}{2} - \frac{2x}{y} + y + k 
  \] 
\end{solution}

\begin{example}
  Να εξετάσετε αν η παράσταση $ (3x^{2}+3y-1)dx + (z^{2}+3x)dy + (2yz+1)dz $ είναι 
  τέλειο διαφορικό και αν ναι, να υπολογίσετε τη συνάρτηση δυναμικού.
\end{example}
\begin{solution}
  Ελέγχουμε με το κριτήριο:
  \[
    \pdv{P}{y} = 3 = \pdv{Q}{x} \quad \text{και} \quad \pdv{Q}{z} = 2z = \pdv{R}{y}
    \quad \text{και} \quad \pdv{P}{z} = 0 = \pdv{R}{x}
  \] 
  Άρα η παράσταση είναι τέλειο διαφορικό. Για να βρούμε τη συνάρτηση δυναμικού, έχουμε
  \begin{description}
    \item [A᾽ Τρόπος: (Με Τύπο)]
      Θα χρησιμοποιήσουμε τον τύπο της παρατήρησης~\ref{olokl} για να υπολογίσουμε 
      τη συνάρτηση δυναμικού. 
      \begin{align*}
        f(x,y,z) &= \int _{x_{0}}^{x} P(t,y,z) \,{dt} + \int _{y_{0}}^{y} Q(x_{0},t,z) 
        \,{dt} + \int _{z_{0}}^{z} R(x_{0}, y_{0}, t) \,{dt} \\
                 &= \int _{0}^{x} (3t^{2}+3y-1) \,{dt} + \int _{0}^{y} (z^{2}+3\cdot 0) 
                 \,{dt} + \int _{0}^{z} (2\cdot 0\cdot t + 1) \,{dt} \\ 
                 &= \int _{0}^{x} (3t^{2}+3y-1) \,{dt} + \int _{0}^{y} z^{2} \,{dt} + 
                 \int _{0}^{z} 1 \,{dt} \\
                 &= \left[t^{3}+3yt-t\right]_{0}^{x} + \left[z^{2}t\right]_{0}^{y} + 
                 \bigl[t\bigr]_{0}^{z} \\
                 &= x^{3}+3xy-x + z^{2}y+z
      \end{align*}
    \item [B᾽ Τρόπος: (Με Ολοκλήρωση)] Αφού η παράσταση είναι τέλειο διαφορικό,  
      υπάρχει συνάρτηση, $ f(x,y,z) $ τέτοια ώστε: 
      \begin{align}
        \pdv{f}{x} &= 3x^{2}+3y-1 \label{fx11} \\
        \pdv{f}{y} &= z^{2}+3x \label{fy11} \\
        \pdv{f}{z} &= 2yz+1 \label{fz11}
      \end{align} 
      Ολοκληρώνουμε μερικώς την~\eqref{fx11} και έχουμε
      \begin{equation*}
        f(x,y,z) = \int (3x^{2}+3y-1) \,{dx} = x^{3} + 3xy -x + c(y,z) 
      \end{equation*} 
      Άρα 
      \begin{equation}
        f(x,y,z) =  x^{3} + 3xy -x + c(y,z) \label{fxyz}
      \end{equation}
      Στη συνέχεια παραγωγίζουμε μερικώς ως προς $y$ και ως προς $z$  
      τη συνάρτηση $f(x,y,z)$ που μόλις βρήκαμε και έχουμε:
      \begin{align}
        \pdv{f}{y} &= 3x + \pdv{c}{y} \label{cy} \\
        \pdv{f}{z} &= \pdv{c}{z} \label{cz}
      \end{align}
      Στη συνέχεια εξισώνοντας την~\eqref{cy} με την~\eqref{fy11} προκύπτει 
      \begin{equation}
        \pdv{c}{y} = z^{2} \label{last1}
      \end{equation}
      και εξισώνοντας την~\eqref{cz} με την~\eqref{fz11} προκύπτει
      \begin{equation}
        \pdv{c}{z} = 2yz+1 \label{last2}
      \end{equation}
      Από τις σχέσεις~\eqref{last1} και~\eqref{last2} μπορούμε να βρούμε τη συνάρτηση 
      $ c(y,z) $ ακολουθώντας τη διαδικασία του προηγούμενου παραδείγματος, αφού 
      ουσιαστικά έχουμε τις μερικές παραγώγους της συνάρτησης και ζητάμε 
      να βρούμε την ίδια τη συνάρτηση. Οπότε ολοκληρώνουμε μερικώς ως προς 
      $y$ την~\eqref{last1} και έχουμε
      \begin{equation}
        c(y,z) = \int z^{2} \,{dy} = z^{2}y + k(z) \label{kz} 
      \end{equation} 
      Στη συνέχεια παραγωγίζουμε ως προς $z$ τη συνάρτηση $ c(y,z) $ που μόλις βρήκαμε 
      και έχουμε
      \begin{equation}
        \pdv{c}{z} = 2yz + k'(z) \label{kz1}
      \end{equation} 
      Έπειτα εξισώνουμε τις σχέσεις~\eqref{last2} και~\eqref{kz1} και προκύπτει 
      \[
        k'(z) = 1 \Rightarrow k(z) = z + c_{1} \label{finally}
      \] 
      Οπότε με αντικατάσταση της~\eqref{finally} στην~\eqref{kz} βρίσκουμε 
      \begin{equation}
        c(y,z) = z^{2}y+z + c_{1} \label{cyz}
      \end{equation}
      Τέλος με αντικατάσταση της~\eqref{cyz} στη συνάρτηση $ f(x,y,z) $ από τη 
      σχέση~\eqref{fxyz} βρίσκουμε
      \[
        f(x,y,z) = x^{3}+3xy-x+z^{2}y+z+ c_{1} 
      \] 
  \end{description}
\end{solution}


\section{Αρμονικές Συναρτήσεις}

\begin{dfn}
  Η συνάρτηση $ f(x,y) $ λέγεται αρμονική αν ικανοποιεί την εξίσωση 
  $ \pdv[2]{f}{x} + \pdv[2]{f}{y} = 0 $ 
\end{dfn}

\begin{rem}
\item {}
  \begin{enumerate}
    \item Η παραπάνω εξίσωση, ονομάζεται εξίσωση Laplace.
    \item Ο τελεστής $ \pdv[2]{}{x} + \pdv[2]{}{y} $ λέγεται τελεστής 
      Laplace.
    \item Η εξίσωση Laplace για συναρτήσεις πολλών μεταβλητών παίρνει τη 
      μορφή
      \[
        \pdv[2]{f}{x_{1}} + \pdv[2]{f}{x_{2}} + \cdots + 
        \pdv[2]{f}{x_{n}}=0 
      \] 
  \end{enumerate}
\end{rem}

\begin{prop}
\item {}
  Αν $ f(x,y) $ συνεχής, έχει συνεχείς μερικές παραγώγους 1ης τάξης 
  και είναι αρμονική, τότε και οι συναρτήσεις $ f_{x} $ και $ f_{y} $
  είναι αρμονικές. 
\end{prop}
\begin{proof}
\item {}
  $f$ αρμονική $ \Rightarrow f_{xx}+f_{yy}=0 $.

  Θέτουμε $ g(x,y)=f_{x}(x,y) $. Για να δείξουμε ότι $ f_{x} $ αρμονική, 
  αρκεί να δείξουμε ότι $ g(x,y) $ είναι αρμονική. Πράγματι:
  \begin{align*}
    g_{x} &= f_{xx} \quad \text{και} \quad g_{xx} = (f_{xx})_{x} \\ 
    g_{y} &= f_{xy} \quad \text{και} \quad g_{yy} = (f_{xy})_{y} =
    (f_{yx})_{y} = f_{yxy} = f_{yyx} = (f_{yy})_{x}
  \end{align*}
  Άρα 
  \[
    g_{xx}+g_{yy} = (f_{xx})_{x} + (f_{yy})_{x} = 
    (f_{xx}+f_{yy})_{x}= (0)_{x} =0
  \] 
\end{proof}

\begin{examples}
\item {}
  \begin{enumerate}
    \item 
      Να εξεταστεί αν η συνάρτηση $ f(x,y)= \ln{\sqrt{x^{2}+y^{2}}} $ είναι 
      αρμονική.
      \begin{solution}
        \begin{align*}
          \pdv{f}{x} &= f_{x} = \left(\ln{\sqrt{x^{2}+y^{2}}} \right)_{x} = 
          \frac{1}{\sqrt{x^{2}+y^{2}}} \cdot 
          \left(\sqrt{x^{2}+y^{2}}\right)_{x} = 
          \frac{1}{\sqrt{x^{2}+y^{2}}} \cdot \frac{\cancel{2}x}{\cancel{2} 
          \sqrt{x^{2}+y^{2}}} = \frac{x}{x^{2}+y^{2}} \\
          \pdv[2]{f}{x} &= f_{xx} = \left(\frac{x}{x^{2}+y^{2}}\right)_{x} = 
          \frac{x^{2}+y^{2}-2x^{2}}{(x^{2}+y^{2})^{2}} = 
          \frac{y^{2}-x^{2}}{(x^{2}+y^{2})^{2}} 
          \intertext{και}
          \pdv{f}{y} &= f_{y} = \left(\ln{\sqrt{x^{2}+y^{2}}} \right)_{y} = 
          \frac{1}{\sqrt{x^{2}+y^{2}}} \cdot 
          \left(\sqrt{x^{2}+y^{2}}\right)_{y} = 
          \frac{1}{\sqrt{x^{2}+y^{2}}} \cdot \frac{\cancel{2}y}{\cancel{2} 
          \sqrt{x^{2}+y^{2}}} = \frac{y}{x^{2}+y^{2}} \\
          \pdv[2]{f}{y} &= f_{yy} = \left(\frac{y}{x^{2}+y^{2}}\right)_{y} = 
          \frac{x^{2}+y^{2}-2y^{2}}{(x^{2}+y^{2})^{2}} = 
          \frac{x^{2}-y^{2}}{(x^{2}+y^{2})^{2}} 
        \end{align*}  
        Προφανώς έχουμε ότι
        \[
          \pdv[2]{f}{x} + \pdv[2]{f}{y} = f_{xx}+f_{yy} =
          \frac{y^{2}-x^{2}}{(x^{2}+y^{2})^{2}} + 
          \frac{x^{2}-y^{2}}{(x^{2}+y^{2})^{2}} = 0  
        \] 
      \end{solution}
  \end{enumerate}
\end{examples}

\section{Τύπος Taylor και Maclaurin}

\begin{example}
  Να υπολογιστεί το ανάπτυγμα της συνάρτησης $f(x,y)=x^3+y^3+xy^2$ γύρω από το 
  σημείο $ (1,2) $ (ή ισοδύναμα σε δυνάμεις του $(x-1)$ και $(y-2)$).
\end{example}

\begin{solution}
  Για να υπολογίσουμε το ανάπτυγμα της συνάρτησης σε δυνάμεις του $(x-1)$ και $(y-2)$ 
  αρκεί ισοδύναμα να υπολογίσουμε το ανάπτυγμα της συνάρτησης γύρω από το 
  σημείο $(x_0,y_0)=(1,2)$

  Παρατηρώ ότι δεν αναφέρεται στην εκφώνηση μέχρι τους όρους ποιας τάξης 
  xρειάζεται να βρω το ανάπτυγμα.  Γι' αυτό, μιας και η συνάρτηση είναι πολυωνυμική,
  βρίσκω μέχρι την τάξη όπου μηδενίζονται οι μερικές παράγωγοι: 

  (Δηλαδή στο συγκεκριμένο παράδειγμα μέχρι $3$ης τάξης, αφού όλες οι παράγωγοι 
  $4$ης και ανώτερης τάξης, θα είναι όλες μηδέν)

  \vspace{\baselineskip}

  \twocolumnsides{\begin{itemize}
      \item $f_x=3x^2+y^2\Rightarrow f_x(1,2)=7$
      \item $f_y=3y^2+2xy\Rightarrow f_y(1,2)=16$
      \item $f_{xx}=6x\Rightarrow f_{xx}(1,2)=6$
      \item $f_{xy}=2y\Rightarrow f_{xy}(1,2)=4$
      \item $f_{yy}=6y+2x\Rightarrow f_{yy}(1,2)=14$
      \end{itemize}}{\begin{itemize}
      \item $f_{xxx}=6$
      \item $f_{xxy}=f_{xyx}=0$
      \item $f_{xyy}=f_{yxy}=2$
      \item $f_{yyy}=6$ 
  \end{itemize}}

  \vspace{\baselineskip}

  Με αντικατάσταση των μερικών παραγώγων στον τύπο Taylor, έχουμε:
  \begin{align*}
    f(x,y)&=13+\Bigl(7(x-1)+16(y-2)\Bigr)+ \\ 
          &\quad +\frac{1}{2!}\Bigl(6(x-1)^2 +2\cdot 4(x-1)(y-2)+14(y-2)^2\Bigr)+ \\
          &\quad +\frac{1}{3!}\Bigl(6(x-1)^3+3\cdot 0(x-1)^2(y-2)+3
          \cdot 2(x-1)(y-2)^2+6(y-2)^3\Bigr).
  \end{align*}
  Και μετά τις πράξεις, έχουμε: 
  \begin{align*}
    f(x,y)&=13+7(x-1)(y-2)+16(y-2)+3(x-1)^2+4(x-1)(y-2) \\
          &\quad +7(y-2)^2+(x-1)^3+(x-1)(y-2)^2+(y-2)^3.
  \end{align*}
  Δεν κάνουμε άλλες πράξεις.

  Έχουμε το ανάπτυγμα της $f(x,y)$ σε δυνάμεις του $(x-1)$ και $(y-2)$ όπως ζητήθηκε.
\end{solution}


      \end{document}
