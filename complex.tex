\input{preamble_ask.tex}
\input{definitions_ask.tex}
\input{tikz.tex}
\input{myboxes.tex}
\input{insbox}

\pagestyle{vangelis}
\geometry{top=2.5cm}

\DeclareMathOperator{\Arg}{Arg}
\setcounter{chapter}{1}

\pagestyle{vangelis}
\everymath{\displaystyle}

\newcommand{\twocolumnsiderr}[2]{\begin{minipage}[t]{0.38\linewidth}\raggedright
    #1
    \end{minipage}\hfill\begin{minipage}[t]{0.52\linewidth}\raggedright
    #2
  \end{minipage}
}

\newcommand{\twocolumnsidell}[2]{\begin{minipage}[t]{0.55\linewidth}\raggedright
    #1
    \end{minipage}\hfill\begin{minipage}[t]{0.40\linewidth}\raggedright
    #2
  \end{minipage}
}

\newcommand{\threecolumnsidess}[3]{\begin{minipage}[c]{0.30\linewidth}\raggedright
    #1
    \end{minipage}\hfill\begin{minipage}[c]{0.35\linewidth}\raggedright
    #2
    \end{minipage}\hfill\begin{minipage}[c]{0.30\linewidth}\raggedright
    #3
  \end{minipage}
}

\newcommand{\fourcolumnsides}[4]{\begin{minipage}[c]{0.22\linewidth}\raggedright
    #1
    \end{minipage}\hfill\begin{minipage}[c]{0.28\linewidth}\raggedright
    #2
    \end{minipage}\hfill\begin{minipage}[c]{0.28\linewidth}\raggedright
    #3
    \end{minipage}\hfill\begin{minipage}[c]{0.22\linewidth}\raggedright
    #4
  \end{minipage}
}
\begin{document}

\chapter*{Μιγαδικοί Αριθμοί}


Οι \textbf{μιγαδικοί} αριθμοί είναι εκφράσεις της μορφής $ a+bi $, 
όπου $ a,b \in \mathbb{R} $ και $ i= \sqrt{-1} $ είναι η \textit{φανταστική} μονάδα. 
Αν $ z=a+bi $ μιγαδικός αριθμός, τότε ο $a$ λέγεται 
\textcolor{Col1}{πραγματικό μέρος} και γράφουμε $ a = \Re(z) $, 
ενώ ο $b$ λέγεται \textcolor{Col1}{φανταστικό μέρος} και γράφουμε $b=\Im($z$)$. 
Το σύνολο των μιγαδικών αριθμών συμβολίζεται με $ \mathbb{C} $, ενώ με $ \mathbb{I} $ 
συμβολίζουμε το σύνολο των φανταστικών αριθμών.
\begin{align*}
  \mathbb{C} = \{ a+bi \; : \; a,b \in \mathbb{R} \; 
  \text{και} \; i = \sqrt{-1} \} \quad 
  \text{και} \quad \mathbb{I} = \{ bi \; : \; b \in \mathbb{R} 
  \; \text{και} \; i = \sqrt{-1} \}  
\end{align*}

\section*{Ισότητα Μιγαδικών Αριθμών}

\begin{mybox1}
  \begin{dfn}
    Έστω οι μιγαδικοί αριθμοί $ z_{1}=a+bi $ και $ z_{2}=c+di $. Τότε 
    $
    z_{1} = z_{2} \Leftrightarrow a=c \; \text{και} \; b=d 
    $
  \end{dfn}
\end{mybox1}

\twocolumnsides{
  \subsection*{Γεωμετρική Αναπαράσταση Μιγαδικών}
  Έστω $xOy$ καρτεσιανό σύστημα αξόνων και $z=a+bi$ μιγαδικός αριθμός. Στον μιγαδικό 
  αριθμό $z$  αντιστοιχίζουμε το ζεύγος $(a,b)$ και άρα το σημείο $M(a,b)$ με 
  συντεταγμένες $a,b$ το οποίο ονομάζουμε \textcolor{Col1}{γεωμετρική εικόνα} του 
  μιγαδικού $z$.  
  % Επίσης στο σημείο $M(a,b)$ και άρα και στον μιγαδικό αριθμό $z$ 
  % αντιστοιχίζουμε το διάνυσμα θέσης $\vec{OM}$.

  \begin{center}
    \begin{tikzpicture}
      \draw[-stealth] (0,0) coordinate (O) -- (2.2,0) node[right] {$x$} ;
      \node at (O) [left] {$O$} ;
      \draw[-stealth] (0,0) -- (0,2.2) node[left] {$y$} ;
      \coordinate (p) at (1.4,1.4) ;
      \fill[Col1] (p) node[above right,font=\small,xshift=-5pt] {$z=a+bi$} 
        node [right,Col2,yshift=-5pt,font=\small] {$M(a,b)$} circle (1.5pt) ;
      \draw[dashed,Col2] (O|-p) node[left]{$b$} -- (p) -- (p|-O) 
        node[below]{$a$} ;
    \end{tikzpicture}
    \hfill
    \begin{tikzpicture}
      \draw[-stealth] (0,0) coordinate (O) -- (2.2,0) node[right] {$x$} ;
      \node at (O) [left] {$O$} ;
      \draw[-stealth] (0,0) -- (0,2.2) node[left] {$y$} ;
      \coordinate (p) at (1.8,1.4) ;
      \coordinate (q) at (0.8,0.6) ;
      \fill[Col1] (p) node[above,font=\small,xshift=-5pt] {$z_{1}=a_{1}+b_{1}i$} 
        circle (1.5pt) ;
      \fill[Col1] (q) node[below,font=\small,xshift=10pt] {$z_{2}=a_{2}+b_{2}i$} 
        circle (1.5pt) ;
      \draw (p) -- (q) ;
      \node (r) at (2,0) {$\phantom{a}$} ;
    \end{tikzpicture}
  \end{center}
}{
  \subsection*{Μέτρο Μιγαδικών}
  Ονομάζουμε \textcolor{Col1}{μέτρο} ενός μιγαδικού αριθμού $ z=a+bi $, και το 
  συμβολίζουμε με $ \abs{z} $, την τιμή 
  \[
    \abs{z} = \sqrt{a^{2}+b^{2}} \geq 0
  \]
  Το μέτρο, παριστάνει την \textbf{απόσταση} της εικόνας 
  του μιγαδικού αριθμού από την αρχή των αξόνων.
  \[
    \begin{tikzpicture}
      \draw[-stealth] (0,0) coordinate (O) -- (2.2,0) node[right] {$x$} ;
      \node at (O) [left] {$O$} ;
      \draw[-stealth] (0,0) -- (0,2.2) node[left] {$y$} ;
      \coordinate (p) at (1.4,1.4) ;
      \fill (p) node[above right,font=\small,xshift=-5pt] {$z=a+bi$} circle (1.5pt) ;
      \draw[dashed,Col2] (O|-p) node[left]{$b$} -- (p) -- (p|-O) node[below]{$a$} ;
      \draw (O) -- node[above left,xshift=3pt,Col1] {$\abs{z}$} (p) ;
      \draw[Col2] (O-|p)++(0,0.25) -- ++(-0.25,0) -- ++ (0,-0.25) ; 
    \end{tikzpicture}
  \]
}


\twocolumnsides{
  \subsection*{Απόσταση Μιγαδικών αριθμών}

  Αν $ z_{1}= a_{1}+ b_{1}i $ και $ z_{2}= a_{2}+ b_{2}i $, τότε το μέτρο
  \[
    \abs{z_{1}- z_{2}} = \sqrt{(a_{1}- a_{2})^{2} + (b_{1}- b_{2})^{2}} 
  \] 
  παριστάνει την απόσταση των μιγαδικών αριθμών, $ z_{1} $ και $ z_{2} $.
}{
  \subsection*{Ιδιότητες του Μέτρου}
  \twocolumnsidess{
    \begin{enumerate}
      \item $ \abs{z_{1} \cdot z_{2}} = \abs{z_{1}} \cdot \abs{z_{2}} $
      \item $ \abs{\frac{z_{1}}{z_{2}}} = \frac{\abs{z_{1}}}{\abs{z_{2}}}, 
        \; z_{2} \neq 0 $
    \end{enumerate}
  }{
    \begin{enumerate}[start=5]
      \item $ \abs{z^{k}} = \abs{z} ^{k} $
    \end{enumerate}
  }
  \begin{enumerate}[start=3]
    \item $ \abs{z} = \abs{-z} = \abs{\overline{z}} = \abs{- \overline{z}} $
    \item $ \abs{\abs{z_{1}}- \abs{z_{2}}} \leq \abs{z_{1} \pm z_{2}} \leq 
      \abs{z_{1}} + \abs{z_{2}} $
  \end{enumerate}
}

\section*{Πράξεις Μιγαδικών Αριθμών}

Έστω $z_{1}=a+bi$ και $z_{2}=c+di$ μιγαδικοί αριθμοί.

\twocolumnsides{
  \subsubsection*{Πρόσθεση}
  $z_{1}+z_{2}=(a+bi)+(c+di)=(a+c)+(b+d)i$
}{
  \subsubsection*{Αφαίρεση}
  $z_{1}-z_{2}=(a+bi)-(c+di)=(a-c)+(b-d)i$
}


\twocolumnsides{
  \subsubsection*{Πολλαπλασιασμός}
  $
  \begin{aligned}
    z_{1}\cdot z_{2}&=(a+bi)\cdot (c+di) \\
                    &=ac+adi+bci+bdi^{2} \\
                    &=ac-bd+(ad+bc)i \\
  \end{aligned}
  $
}{
  \subsubsection*{Διαίρεση}
  $
  \begin{aligned}
    \frac{z_{1}}{z_{2}}&=\frac{(a+bi)}{(c+di)}     =\frac{(a+bi)(c-di)}{(c+di)(c-di)} \\
                       &=\frac{ac+bd+(-ad+bc)}{c^{2}+d^{2}} \\
                       &=\frac{ac+bd}{c^2+d^2}+\frac{bc-ad}{c^2+d^2}
  \end{aligned}
  $
}

% \subsection*{Συνοπτικά οι Πράξεις}

% \twocolumnsides{
%   \begin{myitemize}
%     \item $z_{1}+z_{2}=(a+c)+(b+d)i$
%     \item $z_{1}-z_{2}=(a-c)+(b-d)i$
%   \end{myitemize}
%   }{
%   \begin{myitemize}
%     \item $z_{1}\cdot z_{2}=(ac-bd)+(ad+bc)i$
%     \item $\frac{z_{1}}{z_{2}}=\frac{ac+bd}{c^2+d^2}+\frac{bc-ad}{c^2+d^2}i$
%   \end{myitemize}
% }

\vspace{\baselineskip}

\twocolumnsides{
  \section*{Αντίθετος}
  Έστω ο μιγαδικός $ z=a+bi $. Ονομάζουμε \textcolor{Col1}{αντίθετο} του 
  $z$ τον μιγαδικό αριθμό $ -z=-a-bi $.
  \[
    \begin{tikzpicture}
      \coordinate (0) at (0,0)  ;
      \draw[-stealth] (-2,0) -- (2,0) node[right] {$x$} ;
      \node at (O) [above left] {$O$} ;
      \draw[-stealth] (0,-2) -- (0,2) node[left] {$y$} ;
      \coordinate (p) at (1.2,1.2) ;
      \coordinate (q) at (-1.2,-1.2) ;
      \fill (p) node[above right,font=\small,xshift=-5pt] {$z=a+bi$} circle (1.5pt) ;
      \fill[Col1] (q) node[below,font=\small,Col1] {$z=-a-bi$} circle (1.5pt) ;
      \draw[dashed,Col2] (O|-p) node[left]{$b$} -- (p) -- (p|-O) node[below]{$a$} ;
      \draw[Col2] (O-|p)++(0,0.25) -- ++(-0.25,0) -- ++ (0,-0.25) ; 
      \draw[dashed,Col2] (O|-q) node[right]{$-b$} -- (q) -- (q|-O) node[above]{$-a$} ;
    \end{tikzpicture}
  \]
}{
  \section*{Συζυγής}
  Έστω ο μιγαδικός $ z=a+bi $. Ονομάζουμε \textcolor{Col1}{συζυγή} του 
  $z$ τον μιγαδικό αριθμό $ \overline{z}=a-bi $.
  \[
    \begin{tikzpicture}
      \coordinate (0) at (0,0)  ;
      \draw[-stealth] (-0.5,0) -- (2,0) node[right] {$x$} ;
      \node at (O) [above left] {$O$} ;
      \draw[-stealth] (0,-2) -- (0,2) node[left] {$y$} ;
      \coordinate (p) at (1.2,1.2) ;
      \coordinate (q) at (1.2,-1.2) ;
      \fill (p) node[above right,font=\small,xshift=-5pt] {$z=a+bi$} circle (1.5pt) ;
      \fill[Col1] (q) node[below,font=\small,Col1] {$z=a-bi$} circle (1.5pt) ;
      \draw[dashed,Col2] (O|-p) node[left]{$b$} -- (p) -- (p|-O) node[below right]{$a$} ;
      \draw[dashed,Col2] (O|-q) node[left]{$-b$} -- (q) -- (q|-O) 
        node[below right]{$a$} ;
    \end{tikzpicture}
  \]
}

\section*{Ιδιότητες των Συζυγών}

Έστω $ z_{1}, z_{2} $ και $ z $ μιγαδικοί αριθμοί. Τότε 

\twocolumnsidell{
  \begin{enumerate}
    \item $ \overline{\overline{z}} = z $
    \item $ \overline{z_{1}+ z_{2}} = \overline{z_{1}} + \overline{z_{2}} \quad
      \text{και} \quad 
      \overline{z_{1}+ z_{2}+ \cdots + z_{n}} = \overline{z_{1}} + \overline{z_{2}} +
      \cdots \overline{z_{n}} $
    \item $ \overline{z_{1}\cdot z_{2}} = \overline{z_{1}} \cdot \overline{z_{2}} \quad
      \text{και} \quad 
      \overline{z_{1}\cdot z_{2} \cdots z_{n}} = \overline{z_{1}} \cdot
      \overline{z_{2}} \cdots \overline{z_{n}} $
    \item $ \overline{\left(\frac{z_{1}}{z_{2}}\right)} =
      \frac{\overline{z_{1}}}{\overline{z_{2}}} $, αν $ z_{2} \neq 0 $
  \end{enumerate}
}{
  \begin{enumerate}[start=5]
    \item $ z \cdot \overline{z} = a^{2}+b^{2} = \abs{z} ^{2} $
    \item $ \frac{z+ \overline{z}}{2} = \Re(z) \quad \text{και} \quad 
      \frac{z- \overline{z}}{2i} = \Im(z) $ 
    \item Αν $ z = \overline{z} $ τότε $ z \in \mathbb{R} $
    \item Αν $ z = -\overline{z} $ τότε $ z \in \mathbb{I} $
  \end{enumerate}
}

\begin{rem}
  Για να υπολογίσουμε το πηλίκο δύο μιγαδικών αριθμών, θα πολλαπλασιάζουμε τον αριθμητή
  και τον παρονομαστή με τον συζυγή αριθμό του \textbf{παρονομαστή}.
\end{rem}
\begin{example} 
  Να υπολογιστεί το πηλίκο των μιγαδικών αριθμών $ z_{1}=1+i $ και $ z_{2}=2+3i $.
\end{example}
\begin{solution}
  \begin{align*}
    \frac{1+i}{2+3i} = \frac{(1+i)(2-3i)}{(2+3i)(2-3i)} =
    \frac{2+2i-3i-3i^{2}}{2^{2}+3^{2}} = \frac{5}{2^{2}+3^{2}} - 
    \frac{1}{2^{2}+3^{2}}i = \frac{5}{13} - \frac{1}{13} i
  \end{align*}
\end{solution}


\vspace{\baselineskip}

\twocolumnsider{
  \section*{Δυνάμεις}

  Έστω $z\in \mathbb{C}$ και $n\in\mathbb{Z}$. Τότε
  \begin{myitemize}
    \item $z^{n}=\smash{\underbrace{z\cdot z \cdots z}_{n\; {\text{φορές}}}}, \; n>0$
    \item $z^{0}=1$
    \item $z^{n}=(z^{-1})^{-n}, \; n<0$
  \end{myitemize}
}{
  \section*{Ιδιότητες των Δυνάμεων}

  \twocolumnsiderr{
    \begin{myitemize}
      \item $z^{m}\cdot  z^{n}=z^{m+n}$
      \item $\frac{z^{m}}{z^{n}}=z^{m-n}$
      \item $(z^{m})^{n}=z^{m\cdot n}$
    \end{myitemize}
  }{
    \begin{myitemize}
      \item $(z_{1}\cdot z_{2})^{n}=z_{1}^{n}\cdot z_{2}^{n}$
      \item $\left(\frac{z_{1}}{z_{2}}\right)^{n}=\frac{z_{1}^{n}}{z_{2}^{n}}$
      \item $z=0\Rightarrow 0^{n}=0, \; n>0$
    \end{myitemize}
  }
}

\section*{Δυνάμεις της Φανταστικής Μονάδας}

\twocolumnsides{
  \begin{myitemize}
    \item $i^{0}=1$
    \item $i^{1}=i$
    \item $i^{2}=-1$
    \item $i^{3}=i^{2}\cdot i=-i$
  \end{myitemize}
}{
  \begin{myitemize}
    \item $i^{4}=i^{3}\cdot i=1$
    \item $i^{5}=i^{4}\cdot i=i$
    \item $i^{6}=i^{5}\cdot i=-1$
    \item $i^{7}=i^{6}\cdot i=-i$
  \end{myitemize}
}

\vspace{\baselineskip}

Αποδεικνύεται με επαγωγή ότι για κάθε $n>0$

\twocolumnsides{
  \begin{myitemize}
    \item $i^{4n}=1$
    \item $i^{4n+1}=i$
    \item $i^{4n+2}=-1$
    \item $i^{4n+3}=-i$
  \end{myitemize}
}{
  \begin{myitemize}
    \item $i^{n}=(i^{-1})^{-n}=\left(\frac{1}{i}\right)^{-n}=(-i)^{-n}, \; n<0$
  \end{myitemize}
}

\begin{example}
  Να υπολογιστεί η δύναμη του μιγαδικού $ i^{27} $.
\end{example}
\begin{solution} $i^{27} = i^{4\cdot 6 +3} = (i^{4})^{6} \cdot i ^{3} = 1^{6} 
  \cdot -i = -i \quad \text{ή πιο απλά} \quad i^{27} = i^{4\cdot 6+3} = i^{3} = -i$ 
\end{solution}

\section*{Πολική μορφή Μιγαδικού αριθμού}

\InsertBoxL{4}{\parbox[b][5.5\baselineskip][c]{0.21\textwidth}{
    \begin{tikzpicture}
      \path[-stealth] (0,0) coordinate (O) -- (2.2,0) node[right] (x) {$x$} ;
      \node at (O) [left] {$O$} ;
      \draw[-stealth] (0,0) -- (0,2.2) node[left] {$y$} ;
      \coordinate (p) at (1.4,1.4) ;
      \fill[Col1] (p) node[above right,font=\small,xshift=-5pt] {$z=a+bi$} 
        node [right,Col2,yshift=-5pt,font=\small] {$M(a,b)$} circle (1.5pt) ;
      \draw[dashed,Col2] (O|-p) node[left]{$b$} -- (p) -- (p|-O) 
        node[below]{$a$} ;
      \draw (0) -- node[left] {$r$} (p) pic[draw,fill=Col1!55,->,>=stealth,"$\theta$",
        angle eccentricity=1.5]{angle=x--0--p};
      \draw[-stealth] (0,0) coordinate (O) -- (2.2,0) node[right] (x) {$x$} ;
      \draw[Col2] (O-|p)++(0,0.25) -- ++(-0.25,0) -- ++ (0,-0.25) ; 
  \end{tikzpicture}}
}

Έστω $z=a+bi$ μιγαδικός αριθμός και $M$ η εικόνα του $z$ στο μιγαδικό επίπεδο. 
Έστω $r=\abs{z}$ και $\theta$ η γωνία που σχηματίζει ο \textbf{θετικός} ημιάξονας $x$ 
με το διάνυσμα θέσης του $z$ και η οποία μετριέται σε ακτίνια. 
Τότε $a=r\cos \theta$ και $b=r\sin \theta$ και επομένως ο μιγαδικός $z$ μπορεί να 
γραφεί σε \textcolor{Col1}{τριγωνομετρική} ή \textcolor{Col1}{πολική} μορφή: ως
\[
  \boxed{z=r(\cos\theta+i\sin\theta)}
\]
και χρησιμοποιώντας την ταυτότητα Euler $e^{ix}=\cos x+i\sin x, \; x\in\mathbb{R}$ σε
\textcolor{Col1}{εκθετική} μορφή:
\[
  z= re^{i\theta}
\]

Οι τιμές των $r, \theta$ λέγονται \textcolor{Col1}{πολικές συντεταγμένες} του $z$. Κάθε
τιμή του $\theta$ λέγεται \textcolor{Col1}{όρισμα} του $z$, και με $\arg z$ 
συμβολίζουμε το σύνολο των ορισμάτων του $z$, που προφανώς είναι άπειρα και διαφέρουν 
μεταξύ τους κατά $2\pi$. Αν $z=0$ τότε δεν ορίζεται το όρισμα του $z$. Ισχύει:
\[
  \tan\theta=\frac{b}{a}
\]


\section*{Πρωτεύον Όρισμα}

% \subsection*{$0 \leq \theta \leq 2 \pi$}
% Το \textcolor{Col1}{Πρωτεύον όρισμα} του $\arg z$ συμβολίζεται με $\Arg z$ και 
% είναι η μοναδική τιμή του $\theta\in \arg z$ ώστε $0\leq\theta<2\pi$. Ισχύει:
% \[
%   \arg z=\Arg z+2k\pi, \quad k\in\mathbb{Z}
% \]
% \begin{enumerate}
%   \item Αν $a\neq 0$ τότε από τη σχέση $\tan\theta=\frac{b}{a}$ έχουμε:
%     \[
%       \Arg z=
%       \arctan\frac{b}{a}+k\pi, \; \text{με}\; k=
%       \begin{cases}
%         0, & a>0, \; b\geq 0 \\
%         1, & a<0 \\
%         2, & a>0, \; b<0
%       \end{cases}
%     \]
%   \item Αν $a=0$ και $b\neq 0$ τότε
%     \[
%       \Arg=
%       \begin{cases}
%         \frac{\pi}{2},  &\text{για}\quad   b>0 \\[10pt]
%         \frac{3\pi}{2}, &\text{για}\quad   b<0
%       \end{cases}
%       \]
%   \end{enumerate}

\begin{mybox1}
  \begin{dfn}
    Έστω $ z=a+bi $, μιγαδικός αριθμός, με $ z \neq 0 $. 
    Ορίζουμε ως \textcolor{Col1}{πρωτεύον όρισμα} του $ z 
    $, και το συμβολίζουμε με $ \Arg z $, την \textbf{αλγεβρική τιμή} $\theta$, 
    της \textbf{κυρτής} (με μέτρο μικρότερο ή ίσο των $ \SI{180}{\degree} $), 
    \textbf{προσανατολισμένης} (οι πλευρές της θεωρούνται διατεταγμένες) γωνίας, 
    $ \widehat{(Ox,OM)} $, όπου $M$ είναι η εικόνα του 
    μιγαδικού αριθμού $z$ στο μιγαδικό επίπεδο.
  \end{dfn}
\end{mybox1}
\begin{rem}
  Το πρωτεύον όρισμα του $z$ είναι η μοναδική τιμή του 
  $\theta\in \arg z$ ώστε $ - \pi < \theta \leq \pi$. 
  Ισχύει:
  \[
    \arg z=\Arg z+2k\pi, \quad k\in\mathbb{Z}
  \]
\end{rem}

\enlargethispage*{\baselineskip}

\begin{examples}
  \item{}
    \fourcolumnsides{
      \begin{tikzpicture}[scale=0.8]
        \draw[-stealth] (0,0) coordinate (O) -- (2.2,0) node[below] {$x$} ;
        \node at (O) [left] {$O$} ;
        \draw[-stealth] (0,0) -- (0,2.2) node[left] {$y$} ;
        \coordinate (p) at (1.4,1.4) ;
        \fill[Col1] (p) node[above=5pt,font=\small,xshift=-5pt,Col2] {$z=1+i$} 
          circle (1.5pt) ;
        \draw (O) --  (p) node[right,Col1] {$M$} ;
        \fill pic[draw,fill=Col1!55,-latex,angle radius=18pt] {angle=x--0--p} ;
        \node[fill=Col2!25] at (1,0.5) [right,draw,inner sep=2pt] {$\Arg= \pi /4$} ;
      \end{tikzpicture}
    }{
      \begin{tikzpicture}[scale=0.8]
        \draw[-stealth] (0,0) coordinate (O) -- (2.2,0) node[right] {$x$} ;
        \draw[-stealth] (0,0) coordinate (O) -- (-1.7,0) node[left] {$x'$} ;
        \node at (O) [below left] {$O$} ;
        \draw[-stealth] (0,0) -- (0,2.2) node[left] {$y$} ;
        \coordinate (p) at (-1.4,1.4) ;
        \fill (p) node[above=5pt,font=\small,xshift=-5pt,Col2] {$z=-1+i$} 
          circle (1.5pt) ;
        \draw (O) --  (p) node[left,Col1] {$M$} ;
        \fill pic[draw,fill=Col1!55,-latex,opacity=0.7,angle radius=11.5pt] 
          {angle=x--0--p} ;
        \node[fill=Col2!25] at (0.5,0.7) [right,draw,inner sep=2pt] {$\Arg= 3\pi /4$} ;
      \end{tikzpicture}
    }{
      \begin{tikzpicture}[scale=0.8]
        \draw[-stealth] (0,0) coordinate (O) -- (1.7,0) node[right] {$x$} ;
        \draw[-stealth] (0,0) coordinate (O) -- (-2.2,0) node[left] {$x'$} ;
        \node at (O) [above right] {$O$} ;
        \draw[-stealth] (0,0) -- (0,1.7) node[left] {$y$} ;
        \draw[-stealth] (0,0) -- (0,-2.2) node[left] {$y'$} ;
        \coordinate (p) at (-1.4,-1.4) ;
        \fill (p) node[below=5pt,font=\small,xshift=-5pt,Col2] {$z=-1-i$} 
          circle (1.5pt) ;
        \draw (O) --  (p) node[left,Col1] {$M$} ;
        \fill pic[draw,fill=Col1!55,-latex,opacity=0.7,angle radius=11.5pt,latex-] 
          {angle=p--0--x} ;
        \node[fill=Col2!25] at (-0.5,0.7) [left,draw,inner sep=2pt] 
          {$\Arg= -3\pi /4$} ;
      \end{tikzpicture}
    }
    {
      \begin{tikzpicture}[scale=0.8]
        \draw[-stealth] (0,0) coordinate (O) -- (2.2,0) node[right] {$x$} ;
        \node at (O) [left] {$O$} ;
        \draw[-stealth] (0,0) -- (0,1.5) node[left] {$y$} ;
        \draw[-stealth] (0,0) -- (0,-2.2) node[left] {$y'$} ;
        \coordinate (p) at (1.4,-1.4) ;
        \fill (p) node[below=5pt,font=\small,xshift=-5pt,Col2] {$z=1-i$} 
          circle (1.5pt) ;
        \draw (O) --  (p) node[right,Col1] {$M$} ;
        \fill pic[draw,fill=Col1!55,latex-,angle radius=18pt] {angle=p--0--x} ;
        \node[fill=Col2!25] at (1.0,-0.5) [right,draw,inner sep=2pt] 
          {$\Arg= -\pi /4$} ;
      \end{tikzpicture}
    }
  \end{examples}


  \subsection*{Υπολογισμός Πρωτεύοντος Ορίσματος}

  \begin{mybox2}
  \item {}
    \begin{enumerate}
      \item Αν $ z \in \mathbb{R}_{+}^{*} $ τότε $ \Arg z = 0 $.
      \item Αν $ z \in \mathbb{R}_{-}^{*} $ τότε $ \Arg z = \pi $.
      \item Αν $ z \in \mathbb{C}- \mathbb{R} $ τότε $ \Arg z = - \Arg \overline{z} $
      \item Αν $ z \in \mathbb{R}^{*} $ τότε $ \Arg z = \Arg \overline{z} $
      \item Αν $a\neq 0$ τότε έχουμε:
        $
        \Arg z=\arctan\frac{b}{a}+k\pi, \; \text{με}\; k=
        \begin{cases}
          0, & a>0 \\
          1, & a<0, \; b \geq 0 \\
          -1, & a<0, \; b<0
        \end{cases}
        $
      \item Αν $a=0$ και $b\neq 0$ τότε
        $ 
        \Arg= 
        \begin{cases}
          \frac{\pi}{2},  &\text{για}\quad   b>0 \\[10pt]
          -\frac{\pi}{2}, &\text{για}\quad   b<0
        \end{cases}
        $
    \end{enumerate}
  \end{mybox2}

  \begin{example}
    Να βρεθεί η τριγωνομετρική μορφή του μιγαδικού αριθμού $ z=1+i $ . 
  \end{example}
  \begin{solution}
    Ξέρουμε ότι η τριγωνομετρική μορφή δίνεται από τον τύπο 
    $ z = \abs{z} [\cos{(\theta)} + i \sin{(\theta)}] $. 
    Επομένως, αρκεί να βρούμε το $ \abs{z} $ καθώς κ ένα όρισμα 
    του $z$. Έχουμε ότι $ \abs{z} = \sqrt{1^{2}+1^{2}} = \sqrt{2} $. Επίσης,
    \[
      \left.
        \begin{matrix}
          \cos{\theta} = \frac{1}{\sqrt{2}} \\[10pt]
          \sin{\theta} = \frac{1}{\sqrt{2}} 
        \end{matrix} 
      \right\} 
      \Rightarrow 
      \tan{\theta} = 1 \Leftrightarrow \theta = \frac{\pi}{4} \quad \text{ή} \quad 
      \theta = \pi + \frac{\pi}{4} = \frac{5 \pi}{4}
    \] 
    Επειδή όμως ο μιγαδικός $ z=1+i $ βρίσκεται στο 1ο τεταρτημόριο, έχουμε 
    τελικά ότι $ \theta = \frac{\pi}{4} $. Άρα 
    \[
      \boxed{z= \sqrt{2} \left[ \cos{\frac{\pi}{4}} + i \sin{\frac{\pi}{4}}\right]} 
    \]
  \end{solution}

  \begin{example}
    Να γραφεί σε τριγωνομετρική μορφή ο μιγαδικός $ z=- \cos{\theta} + i \sin{\theta}
    $, αν $ \theta \in [0, \pi /2] $.
  \end{example}
  \begin{solution}
    Προφανώς ο δοσμένος μιγαδικός αριθμός, δεν είναι σε τριγωνομετρική μορφή, γιατί 
    μπροστά από το $ \cos{\theta} $ υπάρχει αρνητικό πρόσημο. Άρα, εργαζόμαστε όπως 
    κ στο προηγούμενο παράδειγμα. Έχουμε ότι 
    \[ \abs{z} = \sqrt{(- \cos{\theta} )^{2}+
    (\sin{\theta} )^{2}} = \sqrt{1} = 1 \]
    Επίσης
    \[
      \left.
        \begin{matrix*}[l]
          \cos{\phi} = \frac{- \cos{\theta}}{1} = - \cos{\theta} = \cos{(\pi - \theta)}
          \\[10pt]
          \sin{\phi} = \frac{\sin{\theta}}{1} = \sin{\theta} = \sin{(\pi - \theta)}
        \end{matrix*} 
      \right\} 
      \Rightarrow 
      \phi = \pi - \theta
    \] 
    Άρα 
    \[
      \boxed{z = \left[\cos{(\pi - \theta)} + i \sin{(\pi - \theta)} \right]}
    \] 
  \end{solution}

  \section*{Ισότητα σε Πολική μορφή}

  Έστω $z_{1}=r_{1}(\cos\theta_{1}+i\sin\theta_{1}) $ και $ z_{2}
  =r_{2}(\cos\theta_{2}+i\sin\theta_{2})$.
  \[
    z_{1}=z_{2}\Leftrightarrow r_{1}=r_{2} \quad\text{και}\quad 
    \Arg z_{1}=\Arg z_{2}\quad \text{ή}\quad \theta_{1}=\theta_{2}+2k\pi,
    \quad k\in\mathbb{Z} 
  \]

  \section*{Πράξεις σε Πολική μορφή}
  Έστω $z_{1}=r_{1}(\cos\theta_{1}+i\sin\theta_{1}) $, $ z_{2}
  =r_{2}(\cos\theta_{2}+i\sin\theta_{2})$ και $ z= r (\cos{\theta} + i \sin{\theta}) $.

  \begin{center}
    \begin{Mytable}    
      \renewcommand{\arraystretch}{2.0}
      \begin{tabular}{|c|c|}
        \TabCellHead Τριγωνομετρική Μορφή  &  \TabCellHead Εκθετική Μορφή \\ \hline
        $z_{1}\cdot
        z_{2}=r_{1}r_{2}
        \left[\cos(\theta_{1}+\theta_{2})+i\sin(\theta_{1}+\theta_{2})\right]$ &
        $z_{1}\cdot z_{2}=r_{1}r_{2}e^{i(\theta_{1}+\theta_{2})}$ \\ \hline
        $\frac{z_{1}}{z_{2}}=\frac{r_{1}}{r_{2}}\left[\cos(\theta_{1}-\theta_{2})+
      i\sin(\theta_{1}-\theta_{2})\right]$ & $\frac{z_{1}}{z_{2}}=
      \frac{r_{1}}{r_{2}}e^{i(\theta_{1}-\theta_{2})}$ \\[5pt] \hline
      $z^{-1}=\frac{1}{z}=\frac{1}{r}(\cos\theta-i\sin\theta)$ &
      $z^{-1}=\frac{1}{r}e^{-i\theta}$ \\[5pt] \hline
      $\overline{z}=r(\cos\theta-i\sin\theta)$ & $\overline{z}=re^{-i\theta}$ \\ 
      \hline
      \bottomrule
    \end{tabular}
  \end{Mytable}
\end{center}


\section*{Θεώρημα De Moivre}

\begin{mybox2}
  \begin{thm}
    Αν $z=r(\cos\theta+i\sin\theta)$ και $n\in\mathbb{Z}$ τότε
    $
    \boxed{z^{n}=r^{n}[\cos (n\theta)+i\sin (n\theta)]}
    $
  \end{thm}
\end{mybox2}


\section*{Νιοστές Ρίζες μιγαδικών αριθμών}
\begin{mybox1}
  \begin{dfn}
    Έστω $ a \in \mathbb{C} $ με $ a \neq 0 $. \textcolor{Col1}{Νιοστή ρίζα} 
    του $a$, ονομάζεται κάθε μιγαδικός αριθμός $ z $ που επαληθεύει την εξίσωση 
    $ z^{n}=a $ με $ n \in \mathbb{N}- \{ 0,1 \} $.
  \end{dfn}
\end{mybox1}

\begin{mybox2}
  \begin{thm}\label{thm:niost}
    Κάθε μιγαδικός αριθμός της μορφής $a=r(\cos\theta+i\sin\theta), \; a\neq 0$,
    όπου $ a \in \mathbb{C} $, έχει $n$ ακριβώς $n$-οστές ρίζες, τις: 
    \[
      \boxed{z_{k}=\sqrt[n]{r}\left[ \cos\left(\frac{\theta+2k\pi}{n}\right)+i
      \sin\left(\frac{\theta+2k\pi}{n}\right) \right], \quad k=0,1,\ldots, n-1}
    \]
  \end{thm}
\end{mybox2}

\begin{example}
  Να λυθεί η εξίσωση $ z^{4} = -16 $. 
\end{example}
\begin{solution}
  Αρχικά, γράφουμε τον αριθμό $ a = -16 $ σε τριγωνομετρική μορφή. Έχουμε 
  $ a=-16=-16+0i $, άρα 
  \[ 
    \abs{a} = \sqrt{(-16)^{2}+0^2} = \abs{-16} = 16 
  \] 
  Επίσης
  \[
    \tan{\theta} = \frac{0}{-16} = 0 \Leftrightarrow \theta =0 \quad
    \text{ή} \quad \theta = \pi
  \] 
  Προφανώς $ \theta = \pi $, γιατί ο $a=-16$ βρίσκεται πάνω στον αρνητικό 
  ημιάξονα των πραγματικών αριθμών. Επομένως, $ a = 16[\cos{\pi} + i
  \sin{\pi}] $. Άρα η εξίσωση που ζητάμε να λύσουμε, γίνεται:
  \[
    z^{4} = 16 [\cos{\pi} + i \sin{\pi}] 
  \] 
  Από το θεώρημα~\ref{thm:niost} έχουμε ότι
  \[
    z_{k} = \sqrt[4]{16} \left[\cos{\left(\frac{2k \pi + \pi}{4}\right)} + i
    \sin{\left(\frac{2k \pi + \pi}{ 4}\right)}\right], 
    \quad \text{με} \quad k=0,1,2,3
  \] 
\end{solution}



\end{document}

