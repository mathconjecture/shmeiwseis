\documentclass[a4paper,12pt]{article}
\usepackage{etex}
%%%%%%%%%%%%%%%%%%%%%%%%%%%%%%%%%%%%%%
% Babel language package
\usepackage[english,greek]{babel}
% Inputenc font encoding
\usepackage[utf8]{inputenc}
%%%%%%%%%%%%%%%%%%%%%%%%%%%%%%%%%%%%%%

%%%%% math packages %%%%%%%%%%%%%%%%%%
\usepackage{amsmath}
\usepackage{amssymb}
\usepackage{amsfonts}
\usepackage{amsthm}
\usepackage{proof}

\usepackage{physics}

%%%%%%% symbols packages %%%%%%%%%%%%%%
\usepackage{dsfont}
\usepackage{stmaryrd}
%%%%%%%%%%%%%%%%%%%%%%%%%%%%%%%%%%%%%%%


%%%%%% graphicx %%%%%%%%%%%%%%%%%%%%%%%
\usepackage{graphicx}
\usepackage{color}
%\usepackage{xypic}
\usepackage[all]{xy}
\usepackage{calc}
%%%%%%%%%%%%%%%%%%%%%%%%%%%%%%%%%%%%%%%

\usepackage{enumerate}

\usepackage{fancyhdr}
%%%%% header and footer rule %%%%%%%%%
\setlength{\headheight}{14pt}
\renewcommand{\headrulewidth}{0pt}
\renewcommand{\footrulewidth}{0pt}
\fancypagestyle{plain}{\fancyhf{}
\fancyhead{}
\lfoot{}
\rfoot{\small \thepage}}
\fancypagestyle{vangelis}{\fancyhf{}
\rhead{\small \leftmark}
\lhead{\small }
\lfoot{}
\rfoot{\small \thepage}}
%%%%%%%%%%%%%%%%%%%%%%%%%%%%%%%%%%%%%%%

\usepackage{hyperref}
\usepackage{url}
%%%%%%% hyperref settings %%%%%%%%%%%%
\hypersetup{pdfpagemode=UseOutlines,hidelinks,
bookmarksopen=true,
pdfdisplaydoctitle=true,
pdfstartview=Fit,
unicode=true,
pdfpagelayout=OneColumn,
}
%%%%%%%%%%%%%%%%%%%%%%%%%%%%%%%%%%%%%%



\usepackage{geometry}
\geometry{left=25.63mm,right=25.63mm,top=36.25mm,bottom=36.25mm,footskip=24.16mm,headsep=24.16mm}

%\usepackage[explicit]{titlesec}
%%%%%% titlesec settings %%%%%%%%%%%%%
%\titleformat{\chapter}[block]{\LARGE\sc\bfseries}{\thechapter.}{1ex}{#1}
%\titlespacing*{\chapter}{0cm}{0cm}{36pt}[0ex]
%\titleformat{\section}[block]{\Large\bfseries}{\thesection.}{1ex}{#1}
%\titlespacing*{\section}{0cm}{34.56pt}{17.28pt}[0ex]
%\titleformat{\subsection}[block]{\large\bfseries{\thesubsection.}{1ex}{#1}
%\titlespacing*{\subsection}{0pt}{28.80pt}{14.40pt}[0ex]
%%%%%%%%%%%%%%%%%%%%%%%%%%%%%%%%%%%%%%

%%%%%%%%% My Theorems %%%%%%%%%%%%%%%%%%
\newtheorem{thm}{Θεώρημα}[section]
\newtheorem{cor}[thm]{Πόρισμα}
\newtheorem{lem}[thm]{λήμμα}
\theoremstyle{definition}
\newtheorem{dfn}{Ορισμός}[section]
\newtheorem{dfns}[dfn]{Ορισμοί}
\theoremstyle{remark}
\newtheorem{remark}{Παρατήρηση}[section]
\newtheorem{remarks}[remark]{Παρατηρήσεις}
%%%%%%%%%%%%%%%%%%%%%%%%%%%%%%%%%%%%%%%




\newcommand{\vect}[2]{(#1_1,\ldots, #1_#2)}
%%%%%%% nesting newcommands $$$$$$$$$$$$$$$$$$$
\newcommand{\function}[1]{\newcommand{\nvec}[2]{#1(##1_1,\ldots, ##1_##2)}}

\newcommand{\linode}[2]{#1_n(x)#2^{(n)}+#1_{n-1}(x)#2^{(n-1)}+\cdots +#1_0(x)#2=g(x)}

\newcommand{\vecoffun}[3]{#1_0(#2),\ldots ,#1_#3(#2)}



\begin{document}
\pagestyle{vangelis} 


\section{Εισαγωγικές Έννοιες}

\begin{dfn}
Μια \textbf{συνήθης διαφορική εξίσωση} (για συντομία σ.δ.ε.) είναι μια εξίσωση της μορφής:
\begin{equation}\label{eq:ode}
F(t,y,y',y'',\ldots,y^{(n)})=0
\end{equation}
όπου 
\begin{itemize}
\item $t$ είναι μια ανεξάρτητη μεταβλητή, την οποία συνήθως αντιλαμβανόμαστε ως \textit{χρόνο}. 
\item $y$ είναι μια εξαρτημένη μεταβλητή, άγνωστη συνάρτηση της ανεξάρτητης μεταβλητής $t$, δηλαδή: $y=y(t)$
\item $F$ είναι μια γνωστή συνάρτηση, πολλών μεταβλητών, η οποία μοντελοποιεί (μαθηματικοποιεί) τη σχέση μεταξύ των παραγώγων της εξαρτημένης μεταβλητής, όπως υποδεικνύεται συνήθως από τη μελέτη κάποιου φυσικού φαινομένου.
\end{itemize}
\end{dfn}



\begin{dfns}
\begin{enumerate}
\item Ο αριθμός $n$ καλείται \textbf{τάξη} της σ.δ.ε. και αντιστοιχεί στην τάξη της μεγαλύτερης παραγώγου που εμφανίζεται στην διαφορικη εξίσωση.
\item \textbf{Βαθμός} της σ.δ.ε. όταν η $F$ είναι πολυωνυμική συνάρτηση ως προς τις μεταβλητές $y,y',y'',\ldots, y^{(n)}$, λέγεται ο εκθέτης στον οποίο είναι υψωμένη η μεγαλύτερης τάξης παράγωγος.
\item Αν από τη σ.δ.ε. λείπει η ανεξάρτητη μεταβλητή $t$, τότε η σ.δ.ε. λέγεται \textbf{αυτόνομη}.
\item Αν η \eqref{eq:ode} μπορεί να επιλυθεί ως προς $y^{(n)}$, δηλαδή 
\begin{equation}\label{eq:solvedode}
y^{(n)}(x)=f(x,y,y',\ldots ,y^{(n-1)})
\end{equation}
όπου $f$ γνωστή συνάρτηση, τότε η \eqref{eq:solvedode} λέγεται \textbf{λυμένη} ή \textbf{κανονική} μορφή της \eqref{eq:ode}.

\end{enumerate}
\end{dfns}

\begin{dfn}
Η \eqref{eq:ode} λέγεται \textbf{γραμμική} αν η $F$ είναι γραμμική συνάρτηση ως προς τις μεταβλητές $y,y',y'',\ldots, y^{(n)}$. Δηλαδή, η γενική μορφή που έχει μια γραμμική διαφορική εξίσωση, $n$-οστής τάξης είναι:

\begin{equation}\label{eq:linode}
\linode{a}{y}
\end{equation}
\end{dfn}

\begin{rem}
Κάθε γραμμική σ.δ.ε. είναι 1ου βαθμού. Το αντίστροφο δεν ισχύει εν γενει. 
\end{rem}

\begin{dfns}
\begin{enumerate}
\item Αν $\vecoffun{a}{x}{n}$ είναι όλες σταθερές συναρτήσεις, η \eqref{eq:linode} λέγεται σ.δ.ε. με \textbf{σταθερούς συντελεστές}.

\item Αν $g(x)=0$, η \eqref{eq:linode} λέγεται \textbf{ομογενή}ς.
\end{enumerate}
\end{dfns}

\begin{dfns}
\begin{enumerate}
\item Οποιαδήποτε συνάρτηση $y(x)$, ικανοποιεί ταυτοτικά την \eqref{eq:ode}, λέγεται \textbf{λύση} της. 

\item Ειδικότερα, αν η λύση, δίνεται στη μορφή $\phi(t,y,c_1,\ldots,c_n)=0$ λέγεται \textbf{γενικό ολοκλήρωμα} ή γενική λύση σε \textbf{πλεγμένη μορφή}. Ενώ, αν δίνεται στη μορφή $y=y(t,c_1,\dots,c_n)$ λέγεται \textbf{κανονική} λύση ή γενική λύση σε \textbf{λυμένη μορφή}.

\item \textbf{Ιδιάζουσες λύσεις}, είναι αυτές που δεν προκύπτουν από τη γενική λύση, για καμία τιμή των παραμέτρων $c_1,\cdots c_n$.

\item Η γενική λύση και οι ιδιάζουσες λύσεις, λεγεται \textbf{πλήρης} λύση.
\end{enumerate}
\end{dfns}


\begin{rems}
\begin{enumerate}
\item Η γενική λύση της \eqref{eq:ode} είναι μια $n$-παραμετρική οικογένεια συναρτήσεων $y=y(t,c_1,\ldots,c_n)$, όπου $c_1,\ldots,c_n$ είναι αυθαίρετες σταθερές. 

\item Όση είναι η τάξη της σ.δ.ε. τόσες πρέπει να είναι και οι αυθαίρετες σταθερές που περιέχονται στη γενική λύση της.

\item Οι γραμμικές διαφορικές εξισώσεις, δεν έχουν ιδιάζουσες λύσεις.
\end{enumerate}
\end{rems}

\begin{dfns}
\begin{enumerate}
\item Ένα \textbf{πρόβλημα αρχικών τιμών (ΠΑΤ)} είναι μια σ.δ.ε. μαζί με κάποιες συνθήκες, που η γενική λύση της και οι παράγωγοί της θα πρέπει να ικανοποιούν, για κάποια \textit{αρχική} τιμή της ανεξάρτητης μεταβλητής.


\item Ένα \textbf{πρόβλημα συνοριακών τιμών (ΠΣΤ)} είναι μια σ.δ.ε. μαζί με κάποιες συνθήκες, που η γενική λύση της θα πρέπει να ικανοποιεί, για κάποιες τιμές της ανεξάρτητης μεταβλητής, συνήθως συνοριακές του διαστηματος στο οποίο ορίζεται η γενική λύση. 
\end{enumerate}
\end{dfns}
















\end{document}
