\documentclass[a4paper,12pt]{article}
\usepackage{etex}
%%%%%%%%%%%%%%%%%%%%%%%%%%%%%%%%%%%%%%
% Babel language package
\usepackage[english,greek]{babel}
% Inputenc font encoding
\usepackage[utf8]{inputenc}
%%%%%%%%%%%%%%%%%%%%%%%%%%%%%%%%%%%%%%

%%%%% math packages %%%%%%%%%%%%%%%%%%
\usepackage{amsmath}
\usepackage{amssymb}
\usepackage{amsfonts}
\usepackage{amsthm}
\usepackage{proof}

\usepackage{physics}

%%%%%%% symbols packages %%%%%%%%%%%%%%
\usepackage{dsfont}
\usepackage{stmaryrd}
%%%%%%%%%%%%%%%%%%%%%%%%%%%%%%%%%%%%%%%


%%%%%% graphicx %%%%%%%%%%%%%%%%%%%%%%%
\usepackage{graphicx}
\usepackage{color}
%\usepackage{xypic}
\usepackage[all]{xy}
\usepackage{calc}
%%%%%%%%%%%%%%%%%%%%%%%%%%%%%%%%%%%%%%%

\usepackage{enumerate}

\usepackage{fancyhdr}
%%%%% header and footer rule %%%%%%%%%
\setlength{\headheight}{14pt}
\renewcommand{\headrulewidth}{0pt}
\renewcommand{\footrulewidth}{0pt}
\fancypagestyle{plain}{\fancyhf{}
\fancyhead{}
\lfoot{}
\rfoot{\small \thepage}}
\fancypagestyle{vangelis}{\fancyhf{}
\rhead{\small \leftmark}
\lhead{\small }
\lfoot{}
\rfoot{\small \thepage}}
%%%%%%%%%%%%%%%%%%%%%%%%%%%%%%%%%%%%%%%

\usepackage{hyperref}
\usepackage{url}
%%%%%%% hyperref settings %%%%%%%%%%%%
\hypersetup{pdfpagemode=UseOutlines,hidelinks,
bookmarksopen=true,
pdfdisplaydoctitle=true,
pdfstartview=Fit,
unicode=true,
pdfpagelayout=OneColumn,
}
%%%%%%%%%%%%%%%%%%%%%%%%%%%%%%%%%%%%%%



\usepackage{geometry}
\geometry{left=25.63mm,right=25.63mm,top=36.25mm,bottom=36.25mm,footskip=24.16mm,headsep=24.16mm}

%\usepackage[explicit]{titlesec}
%%%%%% titlesec settings %%%%%%%%%%%%%
%\titleformat{\chapter}[block]{\LARGE\sc\bfseries}{\thechapter.}{1ex}{#1}
%\titlespacing*{\chapter}{0cm}{0cm}{36pt}[0ex]
%\titleformat{\section}[block]{\Large\bfseries}{\thesection.}{1ex}{#1}
%\titlespacing*{\section}{0cm}{34.56pt}{17.28pt}[0ex]
%\titleformat{\subsection}[block]{\large\bfseries{\thesubsection.}{1ex}{#1}
%\titlespacing*{\subsection}{0pt}{28.80pt}{14.40pt}[0ex]
%%%%%%%%%%%%%%%%%%%%%%%%%%%%%%%%%%%%%%

%%%%%%%%% My Theorems %%%%%%%%%%%%%%%%%%
\newtheorem{thm}{Θεώρημα}[section]
\newtheorem{cor}[thm]{Πόρισμα}
\newtheorem{lem}[thm]{λήμμα}
\theoremstyle{definition}
\newtheorem{dfn}{Ορισμός}[section]
\newtheorem{dfns}[dfn]{Ορισμοί}
\theoremstyle{remark}
\newtheorem{remark}{Παρατήρηση}[section]
\newtheorem{remarks}[remark]{Παρατηρήσεις}
%%%%%%%%%%%%%%%%%%%%%%%%%%%%%%%%%%%%%%%




\newcommand{\vect}[2]{(#1_1,\ldots, #1_#2)}
%%%%%%% nesting newcommands $$$$$$$$$$$$$$$$$$$
\newcommand{\function}[1]{\newcommand{\nvec}[2]{#1(##1_1,\ldots, ##1_##2)}}

\newcommand{\linode}[2]{#1_n(x)#2^{(n)}+#1_{n-1}(x)#2^{(n-1)}+\cdots +#1_0(x)#2=g(x)}

\newcommand{\vecoffun}[3]{#1_0(#2),\ldots ,#1_#3(#2)}




\pagestyle{vangelis}


\begin{document}

\chapter{Αναπαράσταση Γραμ. Μετασχ.}

\section{Ορισμός - Παραδείγματα}

\begin{dfn}
  Έστω  $V$  ένας  $ \mathbb{K}- $χώρος πεπερασμένης διάστασης. Μια
  \textcolor{Col1}{διατεταγμένη βάση} για τον $V$ είναι μια βάση εφοδιασμένη με μια 
  συγκεκριμένη διάταξη.
\end{dfn}

\begin{rem}
  Πολλές φορές αντί να γράφουμε $ \beta = (\mathbf{v_{1}}, \ldots, \mathbf{v_{n}}) $ για 
  μια διατεταγμένη βάση ενός $ \mathbb{K}- $χώρου $V$ θα γράφουμε 
  $ \beta = \{ \mathbf{v_{1}}, \ldots, \mathbf{v_{n}} \} $ και θα δηλώνουμε ότι η 
  $ \beta $ είναι διατεταγμένη βάση (με διάταξη αυτή που φαίνεται από την καταγραφή 
  των στοιχείων, δηλαδή ότι $ \mathbf{v_{1}} $ είναι το 1ο στοιχείο, 
  $ \mathbf{v_{2}} $ το 2ο κ.ο.κ.) 
\end{rem}

Για παράδειγμα, στον $ \mathbb{R}^{3} $, η (συνήθης) βάση $ \beta = \{ \mathbf{e_{1}}, 
\ldots, \mathbf{e_{n}}\} $ μπορεί να θεωρηθεί ως μια διατεταγμένη βάση. 
Επίσης, το σύνολο $ \gamma = \{ \mathbf{e_{2}}, \mathbf{e_{1}}, \mathbf{e_{3}} \} $ 
μπορεί να θεωθηθεί ως διατεταγμένη βάση. 

Τότε ισχύει ότι $ \beta \neq \gamma $ ως διατεταγμένες βάσεις. 
Ταυτίζονται όμως ως σύνολα.  
\begin{dfn}
  Έστω $V$ και $W$ δυο $ \mathbb{K}- $χώροι πεπερασμένης διάστασης. Έστω $ \beta = \{
  \mathbf{v_{1}}, \mathbf{v_{2}}, \ldots, \mathbf{v_{n}} \} $ μια διατεταγμένη βάση 
  του $V$ και έστω $ \Gamma = \{ \mathbf{w_{1}}, \mathbf{w_{2}}, \ldots, 
  \mathbf{w_{n}} \} $ μια διατεταγμένη βάση του $W$. Έστω $ t \colon V \to W $ ένας 
  γραμμικός μετασχηματισμός. Για $ j = 1,2, \ldots, n $ υπάρχουν μοναδικά $ a_{1j}, 
  a_{2j}, \ldots, a_{mj} \in \mathbb{K} $ τέτοια ώστε 
  \[
    T(\mathbf{v}_{j}) = \sum_{i=1}^{m} a_{ij} \mathbf{w}_{j}  
  \] 
  Ο $ m \times n $ πίνακας $A = (a_{ij}) $ λέγεται πίνακας (αναπαράσταση) του 
  γραμμικού μετασχηματισμού $T$ ως προς τις διατεταγμένες βάσεις $ \beta $ και 
  $ \gamma $ και συμβολίζεται με $ \prescript{}{\beta}{(T)}_{\gamma } $.
\end{dfn}

\begin{rem}
  Αν $ V=W $ και $ \beta = \gamma $ τότε γράφουμε $ A = (T)_{\beta} $ αντί για 
  $ \prescript{}{\beta }{(T)}_{\beta} $.
\end{rem}









\end{document}
