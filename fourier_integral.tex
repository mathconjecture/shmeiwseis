\documentclass[a4paper,12pt]{article}
\usepackage{etex}
%%%%%%%%%%%%%%%%%%%%%%%%%%%%%%%%%%%%%%
% Babel language package
\usepackage[english,greek]{babel}
% Inputenc font encoding
\usepackage[utf8]{inputenc}
%%%%%%%%%%%%%%%%%%%%%%%%%%%%%%%%%%%%%%

%%%%% math packages %%%%%%%%%%%%%%%%%%
\usepackage{amsmath}
\usepackage{amssymb}
\usepackage{amsfonts}
\usepackage{amsthm}
\usepackage{proof}

\usepackage{physics}

%%%%%%% symbols packages %%%%%%%%%%%%%%
\usepackage{dsfont}
\usepackage{stmaryrd}
%%%%%%%%%%%%%%%%%%%%%%%%%%%%%%%%%%%%%%%


%%%%%% graphicx %%%%%%%%%%%%%%%%%%%%%%%
\usepackage{graphicx}
\usepackage{color}
%\usepackage{xypic}
\usepackage[all]{xy}
\usepackage{calc}
%%%%%%%%%%%%%%%%%%%%%%%%%%%%%%%%%%%%%%%

\usepackage{enumerate}

\usepackage{fancyhdr}
%%%%% header and footer rule %%%%%%%%%
\setlength{\headheight}{14pt}
\renewcommand{\headrulewidth}{0pt}
\renewcommand{\footrulewidth}{0pt}
\fancypagestyle{plain}{\fancyhf{}
\fancyhead{}
\lfoot{}
\rfoot{\small \thepage}}
\fancypagestyle{vangelis}{\fancyhf{}
\rhead{\small \leftmark}
\lhead{\small }
\lfoot{}
\rfoot{\small \thepage}}
%%%%%%%%%%%%%%%%%%%%%%%%%%%%%%%%%%%%%%%

\usepackage{hyperref}
\usepackage{url}
%%%%%%% hyperref settings %%%%%%%%%%%%
\hypersetup{pdfpagemode=UseOutlines,hidelinks,
bookmarksopen=true,
pdfdisplaydoctitle=true,
pdfstartview=Fit,
unicode=true,
pdfpagelayout=OneColumn,
}
%%%%%%%%%%%%%%%%%%%%%%%%%%%%%%%%%%%%%%



\usepackage{geometry}
\geometry{left=25.63mm,right=25.63mm,top=36.25mm,bottom=36.25mm,footskip=24.16mm,headsep=24.16mm}

%\usepackage[explicit]{titlesec}
%%%%%% titlesec settings %%%%%%%%%%%%%
%\titleformat{\chapter}[block]{\LARGE\sc\bfseries}{\thechapter.}{1ex}{#1}
%\titlespacing*{\chapter}{0cm}{0cm}{36pt}[0ex]
%\titleformat{\section}[block]{\Large\bfseries}{\thesection.}{1ex}{#1}
%\titlespacing*{\section}{0cm}{34.56pt}{17.28pt}[0ex]
%\titleformat{\subsection}[block]{\large\bfseries{\thesubsection.}{1ex}{#1}
%\titlespacing*{\subsection}{0pt}{28.80pt}{14.40pt}[0ex]
%%%%%%%%%%%%%%%%%%%%%%%%%%%%%%%%%%%%%%

%%%%%%%%% My Theorems %%%%%%%%%%%%%%%%%%
\newtheorem{thm}{Θεώρημα}[section]
\newtheorem{cor}[thm]{Πόρισμα}
\newtheorem{lem}[thm]{λήμμα}
\theoremstyle{definition}
\newtheorem{dfn}{Ορισμός}[section]
\newtheorem{dfns}[dfn]{Ορισμοί}
\theoremstyle{remark}
\newtheorem{remark}{Παρατήρηση}[section]
\newtheorem{remarks}[remark]{Παρατηρήσεις}
%%%%%%%%%%%%%%%%%%%%%%%%%%%%%%%%%%%%%%%




\newcommand{\vect}[2]{(#1_1,\ldots, #1_#2)}
%%%%%%% nesting newcommands $$$$$$$$$$$$$$$$$$$
\newcommand{\function}[1]{\newcommand{\nvec}[2]{#1(##1_1,\ldots, ##1_##2)}}

\newcommand{\linode}[2]{#1_n(x)#2^{(n)}+#1_{n-1}(x)#2^{(n-1)}+\cdots +#1_0(x)#2=g(x)}

\newcommand{\vecoffun}[3]{#1_0(#2),\ldots ,#1_#3(#2)}


\input{tikz.tex}

\DeclareMathOperator{\Si}{Si}

\everymath{\displaystyle}

\begin{document}

\setcounter{chapter}{1}

\chapter*{Ολοκλήρωμα Fourier}

\section*{Ολοκληρωτική Αναπαράσταση Fourier}

Γνωρίζουμε ότι αν μια συνάρτηση είναι περιοδική με περίοδο $ 2L $, τότε αναπτύσσεται 
σε σειρά Fourier. Θα δούμε τώρα, ότι αν η συνάρτηση δεν είναι περιοδική, ή ισοδύναμα, 
αν $ L \to \infty $, ότι η σειρά Fourier γίνεται ένα ολοκλήρωμα Fourier.

\begin{thm}
  Έστω $ f \colon (- \infty, \infty) \to \mathbb{R} $, συνάρτηση για την οποία ισχύουν:
  \begin{myitemize}
    \item $ \int _{- \infty}^{\infty} \abs{f(t)} \,{dt} < \infty $, \textbf{απολύτως
      ολοκληρώσιμη} στο $ (- \infty, \infty) $
    \item $f$ τμηματικά λεία σε κάθε πεπερασμένο διάστημα
  \end{myitemize}
  Τότε, σύμφωνα με το ολοκληρωτικό θεώρημα του Fourier, το ολοκλήρωμα
  \[
    f(t) = \int _{0}^{\infty} [A(w) \cos{(wt)} + B(w) \sin{(wt)}] \,{dw} 
  \] 
  όπου
  \[
    A(w) = \frac{1}{\pi} \int _{0}^{\infty} f(t) \cos{(wt)} \,{dt} \quad \text{και}
    \quad B(w) = \frac{1}{\pi} \int _{0}^{\infty} f(t) \sin{(wt)} \,{dt} 
  \] 
  \textbf{υπάρχει} και λέγεται \textcolor{Col1}{ολοκλήρωμα Fourier} και η ισότητα, 
  η οποία ισχύει για κάθε $ t $ που είναι σημείο \textbf{συνέχειας} της $f$, 
  καλείται \textcolor{Col1}{αναπαράσταση} της $f$ με ολοκλήρωμα Fourier. Επίσης
  \[
    \frac{f(x_{0}^{-}+f(x_{0}^{+}))}{2} = \int_{0}^{\infty} [A(w) \cos{(wt)} + B(w) 
    \sin{(wt)}] \,{dw} \quad \text{για κάθε $t$ σημείο \textbf{ασυνέχειας} της $f$} 
  \]
\end{thm}



\section*{Άρτιες και Περιττές Συναρτήσεις}

\begin{myitemize}
  \item Αν $f$ \textbf{άρτια} συνάρτηση, τότε:
    \[
      B(w) = 0 \quad \text{και} \quad A(w) = \frac{2}{\pi} \int _{0}^{\infty} f(t)
      \cos{(wt)} \,{dt } 
    \] 
    και το ολοκλήρωμα Fourier γίνεται:
    \[
      f(t) = \int _{0}^{\infty} A(w) \cos{(wt)}  \,{dw}  \quad
      \text{\textcolor{Col1}{Συνημιτονικό Ολοκλήϱωμα Fourier}}
    \] 
  \item Αν $f$ \textbf{περιττή} συνάρτηση, τότε:
    \[
      A(w) = 0 \quad \text{και} \quad B(w) = \frac{2}{\pi} \int _{0}^{\infty} f(t)
      \sin{(wt)} \,{dt } 
    \] 
    και το ολοκλήρωμα Fourier γίνεται:
    \[
      f(t) = \int _{0}^{\infty} B(w) \sin{(wt)}  \,{dw}  \quad
      \text{\textcolor{Col1}{Ημιτονικό Ολοκλήρωμα Fourier}}
    \] 
\end{myitemize}



\subsection*{Ιδιότητες του Συνημιτονικού Ολοκληρώματος Fourier}

Αν $ f(t) = \int _{0}^{\infty} A(w) \cos{(wt)} \,{dw} $ με $ A(w) = \frac{2}{\pi}
\int _{0}^{\infty} f(t) \cos{(wt)} \,{dt} $, τότε:
\begin{myitemize}
  \item $ f(at) = \frac{1}{a} \int _{0}^{\infty} A(\frac{w}{a} ) \cos{(wt)} \,{dw} $
  \item $ t f(t) = - \int _{0}^{\infty} \dv{A(w)}{w} \sin{(wt)} \,{dw} $
  \item $ t^{2}f(t) = - \int _{0}^{\infty} \dv[2]{A(w)}{w} \cos{(wt)} \,{dw} $
\end{myitemize}

\begin{example}
  Έστω η συνάρτηση $ f(t) = \mathrm{e}^{-kt} $, όπου $ k>0 $ και $ t \geq 0 $. 
  Να βρεθούν η συνημιτονική και η ημιτονική αναπαράσταση της $f$.
\end{example}
\begin{solution}
\item {}
  \subsubsection*{Συνημιτονική Αναπαράσταση}
  Επεκτείνουμε την $f$ κατάλληλα στο διάστημα $ (- \infty, 0) $, ώστε η συνάρτηση να
  είναι άρτια. 
  \begin{align*}
    A(w) &= \frac{2}{\pi} \int _{0}^{\infty} \mathrm{e}^{-kt} \cos{(wt)} \,{dw} = 
    \frac{2}{\pi} \left[\frac{w \mathrm{e}^{-kt} \sin{(kt)}}{k^{2}+w^{2}} - \frac{k
    \mathrm{e}^{-kt} \cos{(wt)}}{k^{2}+w^{2}}\right]_{0}^{\infty} \\
         &= \frac{2}{\pi} \left[\lim_{t \to \infty} \left(\frac{w
               \cancelto{0}{\mathrm{e}^{-kt}} \sin{(kt)}}{k^{2}+w^{2}} - \frac{k 
             \cancelto{0}{\mathrm{e}^{-kt}} \cos{(wt)}}{k^{2}+w^{2}} 
         \right)+ \frac{k}{k^{2}+w^{2}}\right] = \frac{2}{\pi} \frac{k}{k^{2}+w^{2}}
  \end{align*} 
  Επομένως 
  \[
    \boxed{f(t) = \frac{2k}{\pi} \int _{0}^{\infty} \frac{\cos{(wt)}}{k^{2}+w^{2}} 
    \,{dw}}
  \] 



  \subsubsection*{Ημιτονική Αναπαράσταση}
  Επεκτείνουμε την $f$ κατάλληλα στο διάστημα $ (- \infty, 0) $, ώστε η συνάρτηση να
  είναι περιττή. 
  \begin{align*}
    B(w) &= \frac{2}{\pi} \int _{0}^{\infty} \mathrm{e}^{-kt} \sin{(wt)} \,{dw} = 
    \frac{2}{\pi} \left[-\frac{w \mathrm{e}^{-kt} \cos{(kt)}}{k^{2}+w^{2}} - \frac{k
    \mathrm{e}^{-kt} \sin{(wt)}}{k^{2}+w^{2}}\right]_{0}^{\infty} \\
         &= \frac{2}{\pi} \left[\lim_{t \to \infty} \left(-\frac{w
               \cancelto{0}{\mathrm{e}^{-kt}} \cos{(kt)}}{k^{2}+w^{2}} - \frac{k 
             \cancelto{0}{\mathrm{e}^{-kt}} \sin{(wt)}}{k^{2}+w^{2}} 
         \right)+ \frac{w}{k^{2}+w^{2}}\right] = \frac{2}{\pi} \frac{w}{k^{2}+w^{2}}
  \end{align*} 
  Επομένως 
  \[
    \boxed{f(t) = \frac{2}{\pi} \int _{0}^{\infty} \frac{w\sin{(wt)}}{k^{2}+w^{2}} 
    \,{dw}}
  \] 
\end{solution}

\begin{rem}
  Από τις ολοκληρωτικές αναπαραστάσεις της συνάρτησης 
  $ f(t) = \mathrm{e}^{-kt} $, με $ k>0 $, προκύπτουν οι παρακάτω τύποι, για τα 
  \textbf{ολοκληρώματα Laplace}.
  \[
    \int _{0}^{\infty} \frac{\cos{(wt)}}{k^{2}+w^{2}} \, dw = 
    \left\{
      \renewcommand{\arraystretch}{2}
      \begin{matrix*}
        \frac{\pi}{2k} \mathrm{e}^{-kt} , & t \geq 0 \\
        \frac{\pi}{2k} \mathrm{e}^{kt} , & t<0
      \end{matrix*}
    \right. \quad \text{και} \quad
    \int _{0}^{\infty} \frac{w\sin{(wt)}}{k^{2}+w^{2}} \, dw = 
    \left\{
      \renewcommand{\arraystretch}{2}
      \begin{matrix}
        \frac{\pi}{2} \mathrm{e}^{-kt} , & t > 0 \\
        -\frac{\pi}{2} \mathrm{e}^{kt} , & t< 0, & t=0
      \end{matrix} 
    \right.
  \] 
\end{rem}

\begin{example}
  Έστω $ f(t) = 
  \begin{cases}
    1, & \abs{t} \leq 1 \\
    0, & \abs{t} > 1
  \end{cases}$.  Να βρεθεί η ολοκληρωτική αναπαράσταση.
\end{example}
\begin{solution}
  Παρατηρούμε ότι η συνάρτηση είναι άρτια, επομένως θα βρούμε τη 
  συνημιτονική αναπαράσταση.
  \[
    A(w) = \frac{2}{\pi} \int _{0}^{\infty} f(t) \cos{(wt)} \,{dt} = \frac{2}{\pi}
    \int _{0}^{1} \cos{(wt)} \,{dt} = \frac{2}{\pi} \left[\frac{\sin{(wt)}}{w}\right]
    _{0}^{1} = \frac{2 \sin{w}}{\pi w} 
  \] 
  Άρα η ολοκληρωτική αναπαράσταση της $f$ είναι
  \begin{align*}
    f(t) &= \frac{2}{\pi} \int _{0}^{\infty} \frac{\sin{w} \cos{(wt)}}{w} \,{dw}, 
    \quad t \neq \pm 1 
    \intertext{και}
    \frac{(1^{-})+f(1^{+})}{2} &= \frac{1}{2} = \frac{2}{\pi} 
    \int _{0}^{\infty} \frac{\sin{w} \cos{(wt)}}{w} \,{dw}, 
    \quad t= \pm  1
  \end{align*} 
  Επομένως, προκύπτει το παρακάτω ολοκλήρωμα 
  \[
    \int _{)}^{\infty} \frac{\sin{w} \cos{(wt)}}{w} \,{dw} = 
    \begin{cases}
      \pi /2, & \abs{t} < 1 \\
      \pi /4, & \abs{t} = 1 \\
      0, & \abs{t} > 1
    \end{cases}
  \] 
  στο οποίο, αν θέσουμε $ t=0 $, παίρνουμε την τιμή του ολοκληρώματος
  \[
    \int _{0}^{\infty} \frac{\sin{w}}{w} \,{dw} = \lim_{u \to \infty} \int _{0}^{u}
    \frac{\sin{w}}{w} \,{dw} = \lim_{u \to \infty} \Si (u) = \pi /2
  \] 
\end{solution}



\section*{Μιγαδική Μορφή Ολοκληρώματος Fourier}

Είδαμε ότι για μια συνάρτηση $ f \colon (- \infty, \infty) \to \mathbb{R} $, η οποία 
είναι τμηματικά λεία σε κάθε πεπερασμένο διάστημα και απολύτως ολοκληρώσιμη στο 
$ (- \infty, \infty) $, υπάρχει το ολοκλήρωμα Fourier, και ισχύει;
\[
  f(t) = \int _{0}^{\infty} A(w) \cos{(wt)} + B(w) \sin{(wt)} \,{dw} 
\] 
όπου
\[
  A(w) = \frac{1}{\pi} \int _{- \infty}^{\infty}f(t) \cos{(wt)} \,{dt} \quad \text{και}
  \quad
  B(w) = \frac{1}{\pi} \int _{- \infty}^{\infty}f(t) \sin{(wt)} \,{dt} 
\] 
Η παραπάνω σχέση, γράφεται:
\[
  f(t) = \int _{0}^{\infty} \left[A(w) \frac{\mathrm{e}^{iwt} + 
  \mathrm{e}^{-iwt} }{2} - iB(w) \frac{\mathrm{e}^{iwt} - \mathrm{e}^{-iwt}}{2}\right] 
  \,{dw} 
  = \int _{0}^{\infty} \left(\frac{A-iB}{2} \mathrm{e}^{iwt} + \frac{A+iB}{2}
  \mathrm{e}^{-iwt}\right) \,{dw} 
\]
όπου 
\[
  \frac{A-iB}{2} = \frac{1}{2 \pi} \int _{- \infty}^{\infty} f(t) \mathrm{e}^{-iwt}
  \,{dt} \quad \text{και} \quad \frac{A+iB}{2} = \frac{1}{2 \pi} \int _{-
  \infty}^{\infty} f(t) \mathrm{e}^{iwt} \,{dt}
\] 
και με αντικατάσταση, έχουμε
\begin{align*}
  f(t) &= \frac{1}{2 \pi} \left[\int _{0}^{\infty}\left(\int _{- \infty}^{\infty} f(t) \mathrm{e}^{-iwt} \,{dt} \right) \mathrm{e}^{iwt} \,{dw} + \int _{0}^{\infty} \left(\int _{- \infty}^{\infty} f(t)\mathrm{e}^{iwt}\,{dt} \right)\mathrm{e}^{-iwt} \,{dw}\right] \\ 
       &= \frac{1}{2 \pi} \left[\int _{0}^{\infty}\left(\int _{- \infty}^{\infty} f(t) \mathrm{e}^{-iwt} \,{dt} \right) \mathrm{e}^{iwt} \,{dw} + \int _{- \infty}^{0} \left(\int _{- \infty}^{\infty} f(t) \mathrm{e}^{-iwt}\,{dt} \right) \mathrm{e}^{iwt} \,{dw}\right] \\
       &= \frac{1}{2 \pi} \int _{- \infty}^{\infty}\left(\int _{- \infty}^{\infty} f(t) \mathrm{e}^{-iwt} \,{dt} \right) \mathrm{e}^{iwt} \,{dw} \intertext{ή ισοδύναμα}
  \boxed{f(t) &= \frac{1}{\sqrt{2 \pi}} \int _{- \infty}^{\infty} \left( \frac{1}{\sqrt{2 \pi}} \int _{- \infty}^{\infty} f(t) \mathrm{e}^{-iwt} \,{dt} \right) \mathrm{e}^{iwt} \,{dw}}
\end{align*} 
που είναι και η μιγαδική μορφή του ολοκληρώματος Fourier.

\begin{rem}.
  Το Ολοκλήρωμα Fourier, αποτελεί γενίκευση της σειράς Fourier για συνεχή μεταβλητή.

  \twocolumnsidess{
    \subsection*{$n$ Διακριτή Μεταβλητή}
    \renewcommand{\arraystretch}{2.5}
    \begin{tabular}{l}
      $
      f(t) = \textcolor{Col1}{\sum_{n=1}^{\infty}} [a_{\textcolor{Col2}{n}} 
      \cos{(nt)} + b_{n} \sin{(\textcolor{Col2}{nt})}] $ \\
      $ f(t) = \textcolor{Col1}{\sum_{n=- \infty}^{\infty}} 
      \underbrace{ \left(\frac{1}{2 \pi} \int _{- \pi }^{\pi} f(t) \mathrm{e}^{-int} 
      \,{dt}\right)}_{c_{n}} \mathrm{e}^{\-int} $
    \end{tabular}
    }{
    \subsection*{$w$ Συνεχής Μεταβλητή}
    \renewcommand{\arraystretch}{2.7}
    \begin{tabular}{l}
      $
      f(t) = \textcolor{Col1}{\int _{0}^{\infty}} [A(w) \cos{(\textcolor{Col1}{w}t)} + 
      B(w) \sin{(\textcolor{Col1}{w}t)}] \, \textcolor{Col1}{dw} $ \\
      $ f(t) = \frac{1}{\sqrt{2 \pi}} 
      \textcolor{Col1}{\int _{- \infty}^{\infty}}\underbrace{\left( \frac{1}{\sqrt{2 
      \pi}} \int _{- \infty}^{\infty} f(t) \mathrm{e}^{-iwt} \,{dt} \right)} 
      \mathrm{e}^{iwt} \, \textcolor{Col1}{dw} $
    \end{tabular}
  }
\end{rem}


%todo να ελεγξω για λαθη κ να φτιάξω σχήματα για τα παραδείγματα (δες σημ Μαρκακη)






\end{document}

