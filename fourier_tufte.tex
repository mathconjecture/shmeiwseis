\documentclass[a4paper,12pt]{article}
\usepackage{etex}
%%%%%%%%%%%%%%%%%%%%%%%%%%%%%%%%%%%%%%
% Babel language package
\usepackage[english,greek]{babel}
% Inputenc font encoding
\usepackage[utf8]{inputenc}
%%%%%%%%%%%%%%%%%%%%%%%%%%%%%%%%%%%%%%

%%%%% math packages %%%%%%%%%%%%%%%%%%
\usepackage{amsmath}
\usepackage{amssymb}
\usepackage{amsfonts}
\usepackage{amsthm}
\usepackage{proof}

\usepackage{physics}

%%%%%%% symbols packages %%%%%%%%%%%%%%
\usepackage{dsfont}
\usepackage{stmaryrd}
%%%%%%%%%%%%%%%%%%%%%%%%%%%%%%%%%%%%%%%


%%%%%% graphicx %%%%%%%%%%%%%%%%%%%%%%%
\usepackage{graphicx}
\usepackage{color}
%\usepackage{xypic}
\usepackage[all]{xy}
\usepackage{calc}
%%%%%%%%%%%%%%%%%%%%%%%%%%%%%%%%%%%%%%%

\usepackage{enumerate}

\usepackage{fancyhdr}
%%%%% header and footer rule %%%%%%%%%
\setlength{\headheight}{14pt}
\renewcommand{\headrulewidth}{0pt}
\renewcommand{\footrulewidth}{0pt}
\fancypagestyle{plain}{\fancyhf{}
\fancyhead{}
\lfoot{}
\rfoot{\small \thepage}}
\fancypagestyle{vangelis}{\fancyhf{}
\rhead{\small \leftmark}
\lhead{\small }
\lfoot{}
\rfoot{\small \thepage}}
%%%%%%%%%%%%%%%%%%%%%%%%%%%%%%%%%%%%%%%

\usepackage{hyperref}
\usepackage{url}
%%%%%%% hyperref settings %%%%%%%%%%%%
\hypersetup{pdfpagemode=UseOutlines,hidelinks,
bookmarksopen=true,
pdfdisplaydoctitle=true,
pdfstartview=Fit,
unicode=true,
pdfpagelayout=OneColumn,
}
%%%%%%%%%%%%%%%%%%%%%%%%%%%%%%%%%%%%%%



\usepackage{geometry}
\geometry{left=25.63mm,right=25.63mm,top=36.25mm,bottom=36.25mm,footskip=24.16mm,headsep=24.16mm}

%\usepackage[explicit]{titlesec}
%%%%%% titlesec settings %%%%%%%%%%%%%
%\titleformat{\chapter}[block]{\LARGE\sc\bfseries}{\thechapter.}{1ex}{#1}
%\titlespacing*{\chapter}{0cm}{0cm}{36pt}[0ex]
%\titleformat{\section}[block]{\Large\bfseries}{\thesection.}{1ex}{#1}
%\titlespacing*{\section}{0cm}{34.56pt}{17.28pt}[0ex]
%\titleformat{\subsection}[block]{\large\bfseries{\thesubsection.}{1ex}{#1}
%\titlespacing*{\subsection}{0pt}{28.80pt}{14.40pt}[0ex]
%%%%%%%%%%%%%%%%%%%%%%%%%%%%%%%%%%%%%%

%%%%%%%%% My Theorems %%%%%%%%%%%%%%%%%%
\newtheorem{thm}{Θεώρημα}[section]
\newtheorem{cor}[thm]{Πόρισμα}
\newtheorem{lem}[thm]{λήμμα}
\theoremstyle{definition}
\newtheorem{dfn}{Ορισμός}[section]
\newtheorem{dfns}[dfn]{Ορισμοί}
\theoremstyle{remark}
\newtheorem{remark}{Παρατήρηση}[section]
\newtheorem{remarks}[remark]{Παρατηρήσεις}
%%%%%%%%%%%%%%%%%%%%%%%%%%%%%%%%%%%%%%%




\newcommand{\vect}[2]{(#1_1,\ldots, #1_#2)}
%%%%%%% nesting newcommands $$$$$$$$$$$$$$$$$$$
\newcommand{\function}[1]{\newcommand{\nvec}[2]{#1(##1_1,\ldots, ##1_##2)}}

\newcommand{\linode}[2]{#1_n(x)#2^{(n)}+#1_{n-1}(x)#2^{(n-1)}+\cdots +#1_0(x)#2=g(x)}

\newcommand{\vecoffun}[3]{#1_0(#2),\ldots ,#1_#3(#2)}



\pagestyle{vangelis}
\everymath{\displaystyle}


\begin{document}

\chapter{Σειρές \textlatin{Fourier}}


\section{Περιοδικές Συναρτήσεις}

\vspace{\baselineskip}

\begin{dfn}
  Μια συνάρτηση $ f \colon \mathbb{R} \to \mathbb{R} $ λέγεται 
  \textcolor{magenta}{περιοδική} αν 
  \[
    f(x+T)=f(x),\quad \forall x \in \mathbb{R} 
  \] 
  όπου $ T>0 $, ονομάζεται \textcolor{magenta}{περίοδος} της συνάρτησης.
\end{dfn}

\begin{rem}
  Κάθε περιοδική συνάρτηση $ f(x) $ με περίοδo $ 2L>0 $ μπορεί να 
  μετασχηματιστεί σε μια συνάρτηση με περίοδο $ 2 \pi $. Πράγματι, θέτωντας 
  \[ 
    \frac{x}{t} = \frac{2L}{2 \pi} \Leftrightarrow x= t \frac{L}{\pi} 
  \]
  η συνάρτηση $ g(t) = f\Bigl(t \frac{L}{\pi}\Bigr) $ που προκύπτει έχει περίοδο 
  $ 2 \pi $.
\end{rem}

\begin{dfn}
  Μια συνάρτηση $ f \colon [a,b] \to \mathbb{R} $ λέγεται 
  \textcolor{magenta}{τμηματικά συνεχής}, αν η  $f$ έχει πεπερασμένου πλήθους 
  σημεία ασυνέχειας, στα οποία υπάρχουν τα πλευρικά όρια και είναι πεπερασμένα.
\end{dfn}

\begin{dfn}
  Μια συνάρτηση  $ f \colon [a,b] \to \mathbb{R} $  λέγεται  
  \textcolor{magenta}{τμηματικά λεία} αν η $f$ και η $f$' είναι τμηματικά συνεχείς στο 
  $ [a,b] $.     
\end{dfn}

\begin{dfn}
  Μια συνάρτηση $ f(x) $ θα λέγεται:
  \begin{alignat*}{4}
        &\textcolor{magenta}{\text{άρτια}}   & \quad 
        &\overset{\text{ορ.}}{\Leftrightarrow} & \quad f(-x) &= f(x), & 
        &\quad \forall x \in \mathbb{R} \\ 
        &\textcolor{magenta}{\text{περιττή}}  
        & \quad &\overset{\text{ορ.}}{\Leftrightarrow} & \quad f(-x) &= -f(x), & 
        &\quad \forall x \in \mathbb{R}. \\
  \end{alignat*}
\end{dfn}



\section{Περιοδική Επέκταση Συναρτήσεων}

\vspace{\baselineskip}

\begin{enumerate}

  \item Αν $ f(x) $ ορισμένη στο $ [-L,L] $ με $ f(-L)=f(L) $ τότε η συνάρτηση
    \[
      F(x) = f(x), \; -L \leq x \leq L \quad \text{και} \quad F(x+2L)=F(x)
    \]
    με περίοδο $ 2L $ λέγεται \textcolor{magenta}{περιοδική επέκταση} της $f$. 

  \item Αν $ f(x) $ ορισμένη στο $ [0,L] $, τότε μπορεί να επεκταθεί κατά 
    άπειρους τρόπους σε μια περιοδική συνάρτηση 
    \begin{alignat*}{4}
            &F(x) = 
            \begin{cases}  
              f(x), & \phantom{-} 0 \leq x \leq L \\
              \phi(x) & -L < x < 0
            \end{cases}  & \quad & \text{και} \quad F(x+2L)=F(x)
    \end{alignat*}
    με περίοδο $ 2L $, όπου $ \phi(x) $ αυθαίρετη συνάρτηση. 

  \item \label{artia} Αν $ f(x) $ ορισμένη στο $ [0,L] $ τότε η συνάρτηση 
    \begin{alignat*}{4}
            &F(x) = 
            \begin{cases}  
              f(x), & \phantom{-} 0 \leq x \leq L \\
              f(-x), & -L \leq x \leq 0
            \end{cases}  & \quad & \text{και} \quad F(x+2L)=F(x)
    \end{alignat*}
    με περίοδο $ 2L $ λέγεται \textcolor{magenta}{άρτια περιοδική επέκταση} της $f$.

  \item \label{peritth} Αν $ f(x) $ ορισμένη στο $ [0,L] $, με $ f(0)=f(L)=0 $ 
    τότε η συνάρτηση
    \begin{alignat*}{4}
            &F(x) = 
            \begin{cases}  
              \phantom{-} f(x), & \phantom{-} 0 \leq x \leq L \\
              -f(-x), & -L \leq x \leq 0
            \end{cases}  & \quad & \text{και} \quad F(x+2L)=F(x)
    \end{alignat*}
    με περίοδο $ 2L $ λέγεται \textcolor{magenta}{περιττή περιοδική επέκταση} της $f$.

  \item Αν $ f(x) $ ορισμένη στο $ (0,L) $, τότε επεκτείνεται σε άρτια (αντ. περιττή) 
    περιοδική συνάρτηση $ F(x) $ θέτωντας $ F(0) $, $ F(L) $ οποιαδήποτε τιμή 
    (αντ. $F(0)=F(L)$) και στη συνέχεια εφαρμόσουμε τις επεκτάσεις 
    \ref{artia} και \ref{peritth}.

\end{enumerate}





\end{document}

