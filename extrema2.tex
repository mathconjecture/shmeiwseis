\input{preamble/preamble.tex}
\newcommand{\vect}[2]{(#1_1,\ldots, #1_#2)}
%%%%%%% nesting newcommands $$$$$$$$$$$$$$$$$$$
\newcommand{\function}[1]{\newcommand{\nvec}[2]{#1(##1_1,\ldots, ##1_##2)}}

\newcommand{\linode}[2]{#1_n(x)#2^{(n)}+#1_{n-1}(x)#2^{(n-1)}+\cdots +#1_0(x)#2=g(x)}

\newcommand{\vecoffun}[3]{#1_0(#2),\ldots ,#1_#3(#2)}



\everymath{\displaystyle}

\begin{document}

\chapter{Ακρότατα Συναρτήσεων Πολλών Μεταβλητών}

\section{Ορισμός}

\begin{dfn}
\item {}
    Έστω $ f \colon A \subseteq \mathbb{R}^{2} \to \mathbb{R} $, έστω 
    $ (x_{0}, y_{0}) \in A $ και έστω $R(x_{0}, y_{0}) $ περιοχή στοιχείων του $A$, 
    γύρω από το $ (x_{0}, y_{0}) $.
    \begin{enumerate}[i)]
        \item 
            Η $ f(x,y) $, έχει τοπικό ελάχιστο στο σημείο $ (x_{0}, y_{0}) $, αν 
            $ f(x_{0}, y_{0}) \leq f(x,y), \; \forall (x,y) \in R(x_{0}, y_{0}) $ 
        \item 
            Η $ f(x,y) $, έχει τοπικό μέγιστο στο σημείο $ (x_{0}, y_{0}) $, αν 
            $ f(x_{0}, y_{0}) \geq f(x,y), \; \forall (x,y) \in R(x_{0}, y_{0}) $ 
    \end{enumerate}
Το τοπικά μέγιστο και το τοπικά ελάχιστο, ονομάζονται τοπικά ακρότατα. Αν τα τοπικά 
ακρότατα αναφέρονται σ᾽ όλο το σύνολο $A$, τότε ονομάζονται ολικά ακρότατα ή απόλυτα 
ακρότατα.
\end{dfn}

\begin{prop}
    Αν η συνάρτηση $ f(x,y) $ έχει τοπικό ακρότατο στο σημείο $ (x_{0}, y_{0}) $, 
    τότε:
    \begin{enumerate}[i)]
        \item ή υπάρχουν οι $ f_{x}(x_{0}, y_{0}) $ και $ f_{y}(x_{0}, y_{0}) $ 
            και ισχύει $ f_{x}(x_{0}, y_{0}) = f_{y}(x_{0}, y_{0} )=0 $
        \item ή μία τουλάχιστον από τις $ f_{x}(x_{0}, y_{0}) $ και 
            $ f_{y}(x_{0}, y_{0}) $ δεν υπάρχει.
    \end{enumerate}
\end{prop}

\begin{rem}
   Το αντίστροφο της παραπάνω πρότασης δεν ισχύει. 
\end{rem}

\begin{rems}
\item {}
    \begin{enumerate}
        \item Πιθανές θέσεις ακροτάτων της $f$ είναι τα σημεία $ (x_{0}, y_{0}) $ 
            του $A$ για τα οποία οι μερικές παράγωγοι υπάρχουν και είναι ίσες με 0 
            ή τουλάχιστον μία εξ αυτών δεν υπάρχει.
    \end{enumerate}
\end{rems}



\end{document}
