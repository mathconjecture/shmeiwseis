\input{preamble/preamble.tex}
\newcommand{\vect}[2]{(#1_1,\ldots, #1_#2)}
%%%%%%% nesting newcommands $$$$$$$$$$$$$$$$$$$
\newcommand{\function}[1]{\newcommand{\nvec}[2]{#1(##1_1,\ldots, ##1_##2)}}

\newcommand{\linode}[2]{#1_n(x)#2^{(n)}+#1_{n-1}(x)#2^{(n-1)}+\cdots +#1_0(x)#2=g(x)}

\newcommand{\vecoffun}[3]{#1_0(#2),\ldots ,#1_#3(#2)}



\everymath{\displaystyle}

\begin{document}

\chapter{Ακρότατα Συναρτήσεων Πολλών Μεταβλητών}

\section{Τοπικά Ακρότατα}

\begin{dfn}
\item {}
    Έστω $ f \colon A \subseteq \mathbb{R}^{2} \to \mathbb{R} $, έστω 
    $ (x_{0}, y_{0}) \in A $ και έστω $R(x_{0}, y_{0}) $ περιοχή στοιχείων του $A$, 
    γύρω από το $ (x_{0}, y_{0}) $.
    \begin{enumerate}[i)]
        \item 
            Η $ f(x,y) $, έχει τοπικό ελάχιστο στο σημείο $ (x_{0}, y_{0}) $, αν 
            $ f(x_{0}, y_{0}) \leq f(x,y), \; \forall (x,y) \in R(x_{0}, y_{0}) $ 
        \item 
            Η $ f(x,y) $, έχει τοπικό μέγιστο στο σημείο $ (x_{0}, y_{0}) $, αν 
            $ f(x_{0}, y_{0}) \geq f(x,y), \; \forall (x,y) \in R(x_{0}, y_{0}) $ 
    \end{enumerate}
    Το τοπικά μέγιστο και το τοπικά ελάχιστο, ονομάζονται τοπικά ακρότατα. Αν οι 
    ανισότητες στον παραπάνω ορισμό, ισχύουν για κάθε σημείο στο πεδίο ορισμού
    της συνάρτησης, τότε λέμε ότι η $f$ έχει ολικό ακρότατο.
\end{dfn}

\begin{prop}
\item {}
    Αν η συνάρτηση $ f(x,y) $ έχει τοπικό ακρότατο στο σημείο $ (x_{0}, y_{0}) $, 
    τότε:
    \begin{enumerate}[i)]
        \item ή υπάρχουν οι $ f_{x}(x_{0}, y_{0}) $ και $ f_{y}(x_{0}, y_{0}) $ 
            και ισχύει $ f_{x}(x_{0}, y_{0}) = f_{y}(x_{0}, y_{0} )=0 $
        \item ή μία τουλάχιστον από τις $ f_{x}(x_{0}, y_{0}) $ και 
            $ f_{y}(x_{0}, y_{0}) $ δεν υπάρχει.
    \end{enumerate}
\end{prop}

\begin{rem}
\item {}
    Το αντίστροφο της παραπάνω πρότασης δεν ισχύει. 
\end{rem}

\begin{rems}
\item {}
    \begin{enumerate}
        \item Πιθανές θέσεις ακροτάτων της $f$ είναι τα σημεία $ (x_{0}, y_{0}) $ 
            του $A$ για τα οποία οι μερικές παράγωγοι υπάρχουν και είναι ίσες με 0 
            ή τουλάχιστον μία εξ αυτών δεν υπάρχει.
        \item Οι πιθανές θέσεις ακροτάτων λέγονται κρίσιμα ή στάσιμα σημεία.
    \end{enumerate}
\end{rems}

\begin{dfn}
    Έστω $ (x_{0}, y_{0}) $ κρίσιμο σημείο της $f$. Ορίζουμε τις παρακάτω ορίζουσες:
    \[
        \abs{H_{1}} = f_{xx}(x_{0}, y_{0}) \quad \text{και} \quad 
        \abs{H_{2}} = \begin{vmatrix}
            f_{xx} & f_{xy} \\
            f_{yx} & f_{yy}
        \end{vmatrix}_{(x_{0}, y_{0})}
    \] 
\end{dfn}

\begin{thm}[Για συνάρτηση $ f(x,y) $ δύο μεταβλητών]
    \label{thm:2var}
\item {}
    Έστω $ f(x,y) $ συνάρτηση δύο μεταβλητών, ορισμένη σε ένα ανοιχτό 
    υποσύνολο $A$ του $ \mathbb{R}^{2} $, με μερικές παραγώγους 1ης και 2ης τάξης 
    ορισμένες σε μια  περιοχή του κρίσιμου σημείου $ (x_{0}, y_{0}) \in A $ και 
    έστω ότι οι παράγωγοι 2ης τάξης είναι συνεχείς στο $ (x_{0}, y_{0}) $. Τότε
\end{thm}

\begin{myitemize}
    \item Αν $ \abs{H_{1}} > 0 $ και $ \abs{H_{2}} > 0 $ τότε η $f$ παρουσιάζει στο 
        σημείο $ (x_{0}, y_{0}) $ τοπικό ελάχιστο.
    \item Αν $ \abs{H_{1}} < 0 $ και $ \abs{H_{2}} > 0 $ τότε η $f$ παρουσιάζει στο 
        σημείο $ (x_{0}, y_{0}) $ τοπικό μέγιστο.
    \item Αν $ \abs{H_{2}} < 0 $ τότε η $f$ δεν παρουσιάζει ακρότατο και σε αυτήν 
        την περίπτωση το σημείο λέγεται σαγματικό.
    \item Αν $ \abs{H_{2}} = 0 $ τότε δεν βγαίνει κάποιο συμπέρασμα σχετικά με το 
        σημείο $ (x_{0}, y_{0}) $.
\end{myitemize}

Το προηγούμενο θεώρημα επεκτείνεται και για συναρτήσεις με τρεις μεταβλητές. Σε αυτή 
την περίπτωση,εκτός από τις ορίζουσες $ \abs{H_{1}} $ και $ \abs{H_{2}} $, ορίζεται 
ακόμη η
\[
    \abs{H_{3}} = 
    \begin{vmatrix}
        f_{xx} & f_{xy} & f_{xz} \\
        f_{yx} & f_{yy} & f_{yz} \\
        f_{zx} & f_{zy} & f_{zz}
    \end{vmatrix} 
    \hfill 
\] 

\begin{rem}
    Όλες οι παραπάνω ορίζουσες ονομάζονται Εσσιανές και περιέχουν τις παραγώγους 
    2ης τάξης της συνάρτησης. Οι αντίστοιχοι πίνακες, ονομάζονται Εσσιανοί και είναι 
    συμμετρικοί, αφού από τις προϋποθέσεις του θεωρήματος ισχύει το θεώρημα Schwartz και 
    επομένως $ f_{xy} = f_{yx} $, $ f_{xz} = f_{zx} $, κλπ.
\end{rem}


\begin{thm}[Για συνάρτηση $ f(x,y,z) $ τριών μεταβλητών]
    \label{thm:3var}
\item {}
    Έστω $ f(x,y,z) $ συνάρτηση τριών μεταβλητών, ορισμένη σε ένα ανοιχτό 
    υποσύνολο $A$ του $ \mathbb{R}^{3} $, με μερικές παραγώγους 1ης και 2ης τάξης 
    ορισμένες σε μια  περιοχή του κρίσιμου σημείου $ (x_{0}, y_{0}, z_{0}) \in A $ και 
    έστω ότι οι παράγωγοι 2ης τάξης είναι συνεχείς στο $ (x_{0}, y_{0}, z_{0}) $. Τότε
\end{thm}

\begin{myitemize}
    \item Αν $ \abs{H_{1}} > 0 $, $ \abs{H_{2}} > 0 $ και $ \abs{H_{3}} > 0 $ 
        τότε η $f$ παρουσιάζει στο σημείο $ (x_{0}, y_{0}, z_{0}) $ τοπικό ελάχιστο.
    \item Αν $ \abs{H_{1}} < 0 $, $ \abs{H_{2}} > 0 $ και $ \abs{H_{3}} < 0 $ 
        τότε η $f$ παρουσιάζει στο σημείο $ (x_{0}, y_{0}, z_{0}) $ τοπικό μέγιστο.
    \item Αν $ \abs{H_{i}} \neq 0, \forall i \in \{1,2,3\} $ και δεν ισχύει κάποια 
        από τις προηγούμενες περιπτώσεις τότε η $f$ δεν παρουσιάζει ακρότατο και το 
        σημείο λέγεται σαγματικό.
    \item Αν μία τουλάχιστον από τις ορίζουσες $ \abs{H_{1}} $, $ \abs{H_{2}} $ και 
        $ \abs{H_{3}} $ είναι μηδέν, τότε δεν βγαίνει κάποιο συμπέρασμα σχετικά με το 
        σημείο $ (x_{0}, y_{0}, z_{0}) $.
\end{myitemize}

\section{Μεθοδολογία Εύρεσης τοπικών ακροτάτων για συνάρτηση δύο μεταβλητών}

\begin{enumerate}
    \item Βρίσκουμε όλες τις μερικές παραγώγους 1ης και 2ης τάξης.
    \item Βρίσκουμε τα κρίσιμα σημεία της συνάρτησης λύνοντας το σύστημα των εξισώσεων: 
        $ \begin{rcases}
            f_{x} = 0 \\
            f_{y} = 0  
        \end{rcases} $
    \item Για κάθε κρίσιμο σημείο, έστω $ (x_{0}, y_{0}), $ υπολογίζουμε τις Εσσιανές 
        ορίζουσες σε αυτό το σημείο και εφαρμόζουμε το θεώρημα~\ref{thm:2var}.
    \item Στην περίπτωση όπου $ \abs{H_{2}} = 0 $, τότε ακολουθούμε τον ορισμό των 
        τοπικών ακροτάτων.
        \begin{myitemize}
            \item Σχηματίζουμε τη διαφορά $ f(x,y) - f(x_{0}, y_{0}) $.
            \item Προσπαθούμε να προσδιορίσουμε το πρόσημό αυτής της διαφοράς σε 
                μια περιοχή του σημείου $ (x_{0}, y_{0}) $.
                \begin{myitemize}
                    \item Αν $ f(x,y) - f(x_{0}, y_{0}) < 0, \; \forall (x,y) $ σε μια
                        περιοχή του $ (x_{0}, y_{0}) $ τότε έχουμε ελάχιστο. 
                    \item Αν $ f(x,y) - f(x_{0}, y_{0}) > 0, \; \forall (x,y) $ σε μια
                        περιοχή του $ (x_{0}, y_{0}) $ τότε έχουμε μέγιστο. 
                \end{myitemize}
            \item Ειδικότερα, για να εξετάσουμε το σημείο $ (0,0) $ επιλέγουμε μια 
                καμπύλη που περνά από αυτό το σημείο, όπως για παράδειγμα 
                $ y= \lambda x $ ή $ y= \lambda x^{2} $, ώστε η διαφορά 
                $ f(x,y) - f(x_{0}, y_{0}) $ να εξαρτάται μόνο από το $x$ και να 
                είναι πιο εύκολο να προσδιορίσουμε το πρόσημό της.
        \end{myitemize}
\end{enumerate}

\section{Μεθοδολογία Εύρεσης τοπικών ακροτάτων για συνάρτηση τριών μεταβλητών}

\begin{enumerate}
    \item Βρίσκουμε όλες τις μερικές παραγώγους 1ης και 2ης τάξης.
    \item Βρίσκουμε τα κρίσιμα σημεία της συνάρτησης λύνοντας το σύστημα των εξισώσεων: 
        $ \begin{rcases}
            f_{x} = 0 \\
            f_{y} = 0 \\
            f_{z} = 0
        \end{rcases} $
    \item Για κάθε κρίσιμο σημείο, έστω $ (x_{0}, y_{0}, z_{0}), $ υπολογίζουμε 
        τις Εσσιανές ορίζουσες σε αυτό το σημείο και εφαρμόζουμε το 
        θεώρημα~\ref{thm:3var}.
    \item Στην περίπτωση όπου κάποια από τις ορίζουσες είναι μηδέν, τότε 
        ακολουθούμε τον ορισμό των τοπικών ακροτάτων.
        \begin{myitemize}
            \item Σχηματίζουμε τη διαφορά $ f(x,y,z) - f(x_{0}, y_{0}, z_{0}) $.
            \item Προσπαθούμε να προσδιορίσουμε το πρόσημό αυτής της διαφοράς σε 
                μια περιοχή του σημείου $ (x_{0}, y_{0}, z_{0}) $.
                \begin{myitemize}
                    \item Αν $ f(x,y,z) - f(x_{0}, y_{0}, z_{0}) < 0, \; 
                        \forall (x,y,z) $ σε μια
                        περιοχή του $ (x_{0}, y_{0}, z_{0}) $ τότε έχουμε ελάχιστο. 
                    \item Αν $ f(x,y,z) - f(x_{0}, y_{0}, z_{0}) > 0, \; 
                        \forall (x,y,z) $ σε μια
                        περιοχή του $ (x_{0}, y_{0}, z_{0}) $ τότε έχουμε μέγιστο. 
                \end{myitemize}
        \end{myitemize}
\end{enumerate}

Η γενίκευση του θεωρήματος τοπικών ακροτάτων για συναρτήσεις με περισσότερες 
μεταβλητές, γίνεται ως εξής: 

\begin{thm}
    Έστω $ f(x_{1},\ldots, x_{n}) $ συνάρτηση $n$ μεταβλητών, ορισμένη σε ένα ανοιχτό 
    υποσύνολο $A$ του $ \mathbb{R}^{n} $, με μερικές παραγώγους 1ης και 2ης τάξης 
    ορισμένες σε μια  περιοχή του κρίσιμου σημείου $P_{0}$ και έστω ότι οι παράγωγοι 
    2ης τάξης είναι συνεχείς στο $P_{0}$. Τότε, αν $ a_{ij} =
    \eval{\pdv[2]{f}{i}{j}}_{P_{0}}, 
    \forall i,j = 1,\ldots, n$, ορίζουμε τις 
    \[
        \abs{H_{1}} = a_{11}, \quad 
        \abs{H_{2}} = 
        \begin{vmatrix}
            a_{11} & a_{12} \\
            a_{21} & a_{22}
        \end{vmatrix}, \quad \ldots \quad 
        \abs{H_{n}} = 
        \begin{vmatrix}
            a_{11} & a_{12} & \cdots & a_{1n} \\
            a_{21} & a_{22} & \cdots & a_{2n} \\
            \vdots & \vdots & \cdots & \vdots \\
            a_{n1} & a_{n2} & \cdots & a_{nn} \\
        \end{vmatrix} 
    \]
    Τότε:

    \begin{myitemize}
        \item Αν $ \abs{H_{1}} >0 $, $ \abs{H_{2}} >0, \ldots, \abs{H_{n}} > 0 $, 
            τότε η $f$ παρουσιάζει στο $ P_{0} $ τοπικό ελάχιστο.
        \item Αν $ \abs{H_{1}} <0 $, $ \abs{H_{2}} >0, \ldots, 
            (-1)^{n}\abs{H_{n}} > 0 $, τότε η $f$ παρουσιάζει στο $ P_{0} $ τοπικό 
            μέγιστο.
        \item Αν $ \abs{H_{i}} \neq 0, \forall i \in \{1,\ldots,n\} $ και δεν ισχύει 
            κάποια από τις προηγούμενες περιπτώσεις τότε η $f$ δεν παρουσιάζει 
            ακρότατο και το σημείο λέγεται σαγματικό.
        \item Αν μία τουλάχιστον από τις ορίζουσες $ \abs{H_{1}}, \ldots, \abs{H_{3}} $ 
            είναι μηδέν, τότε δεν βγαίνει κάποιο συμπέρασμα σχετικά με το 
            σημείο $P_{0} $.
    \end{myitemize}
\end{thm}

\begin{rem}
    Ένας σύντομος μνημονικός κανόνας για τα ακρότατα των συναρτήσεων πολλών μεταβλητών, 
    είναι ο εξής:
    \begin{myitemize}
        \item Αν όλες οι Εσσιανές ορίζουσες είναι θετικές στο $ P_{0} $, τότε η 
            $f$ παρουσιάζει τοπικό ελάχιστο.
        \item Αν οι Εσσιανές ορίζουσες, έχουν πρόσημα εναλλάξ στο $ P_{0} $, με 
            την πρώτη να είναι αρνητική, τότε η $f$ παρουσιάζει τοπικό μέγιστο.
    \end{myitemize}
\end{rem}

\section{Ολικά Ακρότατα}

Ένα υποσύνολο $D$ του $ \mathbb{R}^{2} $ ονομάζεται κλειστό, αν περιέχει και όλα 
τα συνοριακά του σημεία, ενώ ονομάζεται φραγμένο, αν υπάρχει δίσκος του 
$ \mathbb{R}^{2} $ που το περιέχει, δηλαδή, ουσιαστικά ότι είναι περιορισμένο σε έκταση.

\begin{thm}
    Αν $f(x,y)$ είναι συνεχής σε κάποιο κλειστό και φραγμένο υποσύνολο $A$ του $
    \mathbb{R}^{2} $, τότε η $f$ παίρνει μέγιστη και ελάχιστη τιμή στο $A$. Δηλαδή, 
    υπάρχουν σημεία $ (x_{1}, y_{1}) \in A $ και $ (x_{2}, y_{2}) \in A $ τέτοια ώστε 
    \[
        f(x_{1}, y_{1}) \leq f(x,y) \leq f(x_{2}, y_{2})
    \]
\end{thm}

Η μέγιστη και η ελάχιστη τιμή του προηγούμενου θεωρήματος, είναι το ολικά ακρότατα 
της συνάρτησης, και για να τα προσδιορίσουμε ακολουθούμε τα παρακάτω βήματα.

\begin{myitemize}
    \item Βρίσκουμε τις τιμές των κρίσιμων σημείων της $f$ στο $A$.
    \item Βρίσκουμε τα τοπικά ακρότατα της $f$ στο σύνορο του $A$. 
    \item Η μεγαλύτερη από τις τιμές που βρήκαμε είναι το ολικό μέγιστο της 
        συνάρτησης, ενώ η μικρότερη είναι το ολικό ελάχιστο.
\end{myitemize}







   \end{document}
