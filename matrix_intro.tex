\input{$HOME/Desktop/preamble/preamble.tex}
\newcommand{\vect}[2]{(#1_1,\ldots, #1_#2)}
%%%%%%% nesting newcommands $$$$$$$$$$$$$$$$$$$
\newcommand{\function}[1]{\newcommand{\nvec}[2]{#1(##1_1,\ldots, ##1_##2)}}

\newcommand{\linode}[2]{#1_n(x)#2^{(n)}+#1_{n-1}(x)#2^{(n-1)}+\cdots +#1_0(x)#2=g(x)}

\newcommand{\vecoffun}[3]{#1_0(#2),\ldots ,#1_#3(#2)}





\begin{document}

\chapter{Πίνακες}

\section{Εισαγωγή}

\subsection{Ορισμός}

Ένας πίνακας είναι μια ορθογώνια διάταξη στοιχείων σε γραμμές και στήλες,
δηλαδή είναι ένα σύνολο στοιχείων όπου όμως έχει σημασία η θέση του στοιχείου
μέσα στο σύνολο.

\[
	A = \begin{pmatrix*}
		a_{11} & a_{12} & \cdots & a_{1n} \\
		a_{21} & a_{22} & \cdots & a_{2n} \\
		\vdots & \vdots & \ddots & \vdots \\
		a_{m1} & a_{m2} & \cdots & a_{mm}
	\end{pmatrix*}_{m\times n} 
\] 
Ο δείκτης $ m\times n $ ονομάζεται διάσταση του πίνακα $ A $ και μας δείχνει ότι
ο πίνακας έχει $ m $ το πλήθος γραμμές και $n$ στήλες.
Πιο συνοπτικά, ο παραπάνω πίνακας συμβολίζεται ως $ A = (a_{ij})_{m\times
n} $.


\subsection{Ειδικές Περιπτώσεις Πινάκων}

	
\begin{enumerate}
	\item  Στήλη $ \Sigma = \begin{pmatrix}
			a_{11} \\
			a_{21} \\
			\vdots \\
			a_{n1}
		\end{pmatrix}_{n\times 1}$

	\item  Γραμμή $ \Gamma  = \begin{pmatrix}
			b_{11} & b_{12} & \cdots & b_{1n}
		\end{pmatrix}_{1\times n} $

	\item Μοναδιάιος $ 2\times 2 $ 
\end{enumerate}


\end{document}
