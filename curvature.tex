\input{preamble_ask.tex}
\input{definitions_ask.tex}


\everymath{\displaystyle}



\begin{document}

\chapter*{Καμπυλότητα}

\section*{Ορισμοί}


\begin{dfn}
  Η \textcolor{Col2}{καμπυλότητα $\kappa$} της καμπύλης $ y = f(x) $ περιγράφει
  το ρυθμό μεταβολής της κατεύθυνσης της καμπύλης όταν μεταβάλλεται το μήκος
  τόξου $s$ και ορίζεται ως 
  \[
    \kappa = \lim_{\Delta s\to 0} \frac{\Delta \phi}{\Delta s} 
  \] 
  όπου $\phi$ είναι η γωνία που σχηματίζει η εφαπτομένη ευθεία σ᾽ ένα σημείο
  του τόξου της καμπύλης με τον άξονα $x$. 
\end{dfn}

\begin{dfn}
  Η \textcolor{Col2}{ακτίνα καμπυλότητας $\rho$} μιας καμπύλης σ᾽ ένα σημείο
  της ορίζεται από τη σχέση 
  \[
    \rho = \frac{1}{\abs{\kappa}} 
  \] 
  και είναι η ακτίνα της περιφέρειας του κύκλου που έχει την ίδια καμπυλότητα
  με την καμπύλη στο σημείο εκείνο.
\end{dfn}

\begin{dfn}
  Αν σε ένα τυχαίο σημείο $P$ μιας καμπύλης φέρουμε την κάθετη ευθεία  προς την
  εφαπτομένη στο $P$ και πάνω στην κάθετη πάρουμε μήκος $\rho$ το οποίο
  ξεκινάει από το σημείο $P$ και καταλήγει στο σημείο $C$ τότε το σημείο $C$
  ονομάζεται \textcolor{Col2}{κέντρο καμπυλότητας} της καμπύλης στο $P$.
  Οι συντεταγμένες του κέντρου καμπυλότητας είναι
  \[
    x_{C} = x_{P} - \frac{\sin{\phi_{P}}}{\kappa_{P}}, \quad y_{C} = y_{P} +
    \frac{\cos{\phi_{P}}}{\kappa_{P}} 
  \] 
\end{dfn}

\begin{dfn}
  Αν $C$ είναι το κέντρο καμπυλότητας στο σημείο $P$ της καμπύλης τότε ο
  κύκλος με κέντρο $C$ και ακτίνα ίση με την ακτίνα καμπυλότητας $\rho$ της
  καμπύλης ονομάζεται \textcolor{Col2}{κύκλος καμπυλότητας}
\end{dfn}

\section*{Διάφοροι τύποι για την καμπυλότητα}

\begin{myitemize}
  \item Αν η καμπύλη είναι $c: y = f(x) $ τότε $ \kappa = \frac{y''}{[1 + (y')^{2}]
    ^{\frac{3}{2}}} $
  \item Αν η καμπύλη είναι $ c: x = x(t), \; y = y(t) $ τότε $ \kappa =
    \frac{\dot{x}\ddot{y} - \ddot{x}\dot{y}}{(\dot{x}^{2} +
    \dot{y}^{2})^{\frac{3}{2}}}$
\end{myitemize}



\end{document}
