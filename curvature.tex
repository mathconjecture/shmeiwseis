\documentclass[a4paper,12pt]{article}
\usepackage{etex}
%%%%%%%%%%%%%%%%%%%%%%%%%%%%%%%%%%%%%%
% Babel language package
\usepackage[english,greek]{babel}
% Inputenc font encoding
\usepackage[utf8]{inputenc}
%%%%%%%%%%%%%%%%%%%%%%%%%%%%%%%%%%%%%%

%%%%% math packages %%%%%%%%%%%%%%%%%%
\usepackage{amsmath}
\usepackage{amssymb}
\usepackage{amsfonts}
\usepackage{amsthm}
\usepackage{proof}

\usepackage{physics}

%%%%%%% symbols packages %%%%%%%%%%%%%%
\usepackage{dsfont}
\usepackage{stmaryrd}
%%%%%%%%%%%%%%%%%%%%%%%%%%%%%%%%%%%%%%%


%%%%%% graphicx %%%%%%%%%%%%%%%%%%%%%%%
\usepackage{graphicx}
\usepackage{color}
%\usepackage{xypic}
\usepackage[all]{xy}
\usepackage{calc}
%%%%%%%%%%%%%%%%%%%%%%%%%%%%%%%%%%%%%%%

\usepackage{enumerate}

\usepackage{fancyhdr}
%%%%% header and footer rule %%%%%%%%%
\setlength{\headheight}{14pt}
\renewcommand{\headrulewidth}{0pt}
\renewcommand{\footrulewidth}{0pt}
\fancypagestyle{plain}{\fancyhf{}
\fancyhead{}
\lfoot{}
\rfoot{\small \thepage}}
\fancypagestyle{vangelis}{\fancyhf{}
\rhead{\small \leftmark}
\lhead{\small }
\lfoot{}
\rfoot{\small \thepage}}
%%%%%%%%%%%%%%%%%%%%%%%%%%%%%%%%%%%%%%%

\usepackage{hyperref}
\usepackage{url}
%%%%%%% hyperref settings %%%%%%%%%%%%
\hypersetup{pdfpagemode=UseOutlines,hidelinks,
bookmarksopen=true,
pdfdisplaydoctitle=true,
pdfstartview=Fit,
unicode=true,
pdfpagelayout=OneColumn,
}
%%%%%%%%%%%%%%%%%%%%%%%%%%%%%%%%%%%%%%



\usepackage{geometry}
\geometry{left=25.63mm,right=25.63mm,top=36.25mm,bottom=36.25mm,footskip=24.16mm,headsep=24.16mm}

%\usepackage[explicit]{titlesec}
%%%%%% titlesec settings %%%%%%%%%%%%%
%\titleformat{\chapter}[block]{\LARGE\sc\bfseries}{\thechapter.}{1ex}{#1}
%\titlespacing*{\chapter}{0cm}{0cm}{36pt}[0ex]
%\titleformat{\section}[block]{\Large\bfseries}{\thesection.}{1ex}{#1}
%\titlespacing*{\section}{0cm}{34.56pt}{17.28pt}[0ex]
%\titleformat{\subsection}[block]{\large\bfseries{\thesubsection.}{1ex}{#1}
%\titlespacing*{\subsection}{0pt}{28.80pt}{14.40pt}[0ex]
%%%%%%%%%%%%%%%%%%%%%%%%%%%%%%%%%%%%%%

%%%%%%%%% My Theorems %%%%%%%%%%%%%%%%%%
\newtheorem{thm}{Θεώρημα}[section]
\newtheorem{cor}[thm]{Πόρισμα}
\newtheorem{lem}[thm]{λήμμα}
\theoremstyle{definition}
\newtheorem{dfn}{Ορισμός}[section]
\newtheorem{dfns}[dfn]{Ορισμοί}
\theoremstyle{remark}
\newtheorem{remark}{Παρατήρηση}[section]
\newtheorem{remarks}[remark]{Παρατηρήσεις}
%%%%%%%%%%%%%%%%%%%%%%%%%%%%%%%%%%%%%%%




\newcommand{\vect}[2]{(#1_1,\ldots, #1_#2)}
%%%%%%% nesting newcommands $$$$$$$$$$$$$$$$$$$
\newcommand{\function}[1]{\newcommand{\nvec}[2]{#1(##1_1,\ldots, ##1_##2)}}

\newcommand{\linode}[2]{#1_n(x)#2^{(n)}+#1_{n-1}(x)#2^{(n-1)}+\cdots +#1_0(x)#2=g(x)}

\newcommand{\vecoffun}[3]{#1_0(#2),\ldots ,#1_#3(#2)}




\everymath{\displaystyle}



\begin{document}

\chapter{Καμπυλότητα}

\section{Ορισμοί}


\begin{dfn}
	Η \textcolor{blue}{καμπυλότητα $\kappa$} της καμπύλης $ y = f(x) $ περιγράφει
	το ρυθμό μεταβολής της κατεύθυνσης της καμπύλης όταν μεταβάλλεται το μήκος
	τόξου $s$ και ορίζεται ως 
	\[
		\kappa = \lim_{\Delta s\to 0} \frac{\Delta \phi}{\Delta s} 
	\] 
	όπου $\phi$ είναι η γωνία που σχηματίζει η εφαπτομένη ευθεία σ᾽ ένα σημείο
	του τόξου της καμπύλης με τον άξονα $x$. 
\end{dfn}

\begin{dfn}
	Η \textcolor{blue}{ακτίνα καμπυλότητας $\rho$} μιας καμπύλης σ᾽ ένα σημείο
	της ορίζεται από τη σχέση 
	\[
		\rho = \frac{1}{\abs{\kappa}} 
	\] 
	και είναι η ακτίνα της περιφέρειας του κύκλου που έχει την ίδια καμπυλότητα
	με την καμπύλη στο σημείο εκείνο.
\end{dfn}

\begin{dfn}
	Αν σε ένα τυχαίο σημείο $P$ μιας καμπύλης φέρουμε την κάθετη ευθεία  προς την
	εφαπτομένη στο $P$ και πάνω στην κάθετη πάρουμε μήκος $\rho$ το οποίο
	ξεκινάει από το σημείο $P$ και καταλήγει στο σημείο $C$ τότε το σημείο $C$
	ονομάζεται \textcolor{blue}{κέντρο καμπυλότητας} της καμπύλης στο $P$.
	Οι συντεταγμένες του κέντρου καμπυλότητας είναι
	\[
		x_{C} = x_{P} - \frac{\sin{\phi_{P}}}{\kappa_{P}}, \quad y_{C} = y_{P} +
		\frac{\cos{\phi_{P}}}{\kappa_{P}} 
	\] 
\end{dfn}

\begin{dfn}
	Αν $C$ είναι το κέντρο καμπυλότητας στο σημείο $P$ της καμπύλης τότε ο
	κύκλος με κέντρο $C$ και ακτίνα ίση με την ακτίνα καμπυλότητας $\rho$ της
	καμπύλης ονομάζεται \textcolor{blue}{κύκλος καμπυλότητας}
\end{dfn}

\section{Διάφοροι τύποι για την καμπυλότητα}

\begin{itemize}
	\item Αν η καμπύλη είναι $c: y = f(x) $ τότε $ \kappa = \frac{y''}{[1 +
			(y')^{2}]^{\frac{3}{2}}} $
		\item Αν η καμπύλη είναι $ c: x = x(t), y = y(t) $ τότε $ \kappa =
			\frac{\dot{x}\ddot{y} - \ddot{x}\dot{y}}{(\dot{x}^{2} +
			\dot{y}^{2})^{\frac{3}{2}}}$
\end{itemize}



\end{document}
